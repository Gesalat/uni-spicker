\documentclass{cheat-sheet}

\pdfinfo{
  /Title (Zusammenfassung Stochastik 3)
  /Author (Tim Baumann)
}

\usepackage{bbm} % Für 1 mit Doppelstrich (Indikatorfunktion)
\usepackage{mathtools} % psmallmatrix environment

% Kleinere Klammern
\delimiterfactor=701

\newcommand{\Alg}{\mathfrak{A}} % (Mengen-)Algebra
%\newcommand{\Ring}{\mathfrak{R}} % (Mengen-)Ring
%\newcommand{\LebAlg}{\mathfrak{L}} % Lebesgue-Borel-Mengen
\renewcommand{\P}{\mathbb{P}} % Wahrscheinlichkeitsmaß
\newcommand{\E}{\mathbb{E}} % Erwartungswert
\newcommand{\Bor}{\mathfrak{B}} % Borel
%\newcommand{\Leb}{\mathcal{L}} % Lebesgue
\newcommand{\ind}{\mathbbm{1}} % Indikatorfunktion

\DeclareMathOperator{\var}{Var} % Varianz
\DeclareMathOperator{\cov}{Cov} % Kovarianz
\DeclareMathOperator{\cor}{Cor} % Korrelation

\newcommand{\Normal}{\mathcal{N}} % Gaußsche Normalverteilung

\begin{document}

\maketitle{Zusammenfassung Stochastik 3}

% Vorlesung vom 12.10.2015

% 1. Wiederholung

% 1.1 Grundbegriffe der Testtheorie

\begin{modell}
  Gegeben sei ein Parametrisches Modell, \dh eine Zufallsgröße $X$, deren Verteilungsfunktion $P_X \in \Set{P_\vartheta}{\vartheta \in \Theta \subset \R^n}$ von einem Parameter $\vartheta$ abhängt.
\end{modell}

\begin{prob}
  Anhand einer \emph{Stichprobe} $x_1, \ldots, x_n \in \R^1$ von $X$ (\dh{} $x_1, \ldots, x_n$ sind Realisierung von iid ZGen $X_1, \ldots, X_n \sim P_X$) ist zu entscheiden, ob die sogenannte \emph{Nullhypothese} $H_0 : \vartheta \in \Theta_0 \subset \Theta$ oder eine \emph{Gegenhypothese} $H_1 : \vartheta \in \Theta_1 = \Theta \setminus \Theta_0$ angenommen oder abgelehnt werden soll.
\end{prob}

\begin{defn}
  Der \emph{Stichprobenraum} ist $(\R^n, \Bor(\R^n), P_\vartheta \times \ldots \times P_\vartheta)$
\end{defn}

\begin{terminologie}
  Die Hypothese $H_i$ heißt \emph{einfach}, falls $\abs{H_i} = 1$, andernfalls \emph{zusammengesetzt}.
\end{terminologie}

\begin{defn}
  Ein (nichtrandomisierter) \emph{Test} für $H_0$ gegen $H_1$ ist eine Entscheidungsregel über die Annahme von $H_0$ basierend auf einer Stichprobe, die durch eine messbare Abbildung $\varphi : \R^n \to \{ 0, 1 \}$ augedrückt wird und zwar durch
  \[ \varphi(x_1, \ldots, x_n) = \begin{cases}
    0 & \text{bei Annahme von $H_0$,} \\
    1 & \text{bei Ablehung von $H_0$.}
  \end{cases} \]
\end{defn}

\begin{defn}
  Der \emph{Ablehungsbereich} oder \emph{kritische Bereich} von $\varphi$ ist
  \[ K_n \coloneqq \Set{(x_1, \ldots, x_n) \in \R^n}{\varphi(x_1, \ldots, x_n) = 1}. \]
\end{defn}

\begin{bem}
  Es gilt $\varphi = \ind_{K_n}$.
\end{bem}

\begin{defn}
  Ein \emph{Fehler 1. Art} ist eine Ablehnung der Nullhypothese $H_0$, obwohl $H_0$ richtig ist;
  ein \emph{Fehler 2. Art} ist eine Annahme von $H_0$, obwohl $H_0$ falsch ist.
\end{defn}

% Ausgelassen: Stichprobenraum ist [R^n, B(R^n), P_\vartheta \times \ldots \times P_\vartheta]

\begin{defn}
  Die \emph{Güte- oder Machtfunktion} des Tests $\varphi$ ist
  \begin{align*}
    m_\varphi : \Theta \to \cinterval{0}{1},
    m_\varphi(\vartheta) & \coloneqq
    \E_\vartheta \varphi(X_1, \ldots, X_n) \\
    & = \P_\vartheta ((X_1, \ldots, X_n) \in K_n) \\
    & = (P_\vartheta \times \ldots \times P_\vartheta)(K_n)
  \end{align*}
  Die Gegenwsk. $(1 {-} m_\varphi(\vartheta))$ heißt \emph{Operationscharakteristik} von $\varphi$.
\end{defn}

\begin{bem}
  Es gilt
  \begin{alignat*}{2}
    \P_\vartheta(\text{Fehler 1. Art}) &= m_\varphi(\vartheta) & \enspace\text{für $\vartheta \in \Theta_0$,} \\
    \P_\vartheta(\text{Fehler 2. Art}) &= 1 - m_\varphi(\vartheta) & \enspace\text{für $\vartheta \in \Theta_1$.}
  \end{alignat*}
\end{bem}

% Ausgelassen: Graph einer "fast idealen Kurve"

\begin{defn}
  Ein Test $\varphi : \R^n \to \{ 0, 1 \}$ mit
  \[ \sup_{\vartheta \in \Theta_0} m_\varphi(\vartheta) \leq \alpha \]
  heißt \emph{$\alpha$-Test} o. \emph{Signifikanztest} zum \emph{Signifikanzniveau} $\alpha \in \ointerval{0}{1}$. \\[2pt]
  Ein $\alpha$-Test $\varphi$ heißt \emph{unverfälscht} (erwartungstreu, unbiased), falls
  \[ \inf_{\vartheta \in \Theta_1} m_\varphi(\vartheta) \geq \alpha. \]
\end{defn}

% Konstruktion nichtrandomisierter Tests mittels Stichprobenfunktionen (= Statistiken) im Falle einfacher Nullhypothesen

\begin{situation}
  Sei nun eine Stichprobenfunktion oder \emph{Teststatistik} $T : \R^n \to \R^1$ gegeben.
  Wir wollen einen Test der einfachen Nullhypothese $H_0 : \vartheta \in \Theta_0 = \{ \vartheta_0 \}$ entwickeln.
\end{situation}

\begin{defn}
  $K_n^T \subset \R^1$ heißt \emph{kritischer Bereich der Teststatistik}, falls
  \[ K_n = T^{-1}(K_n^T). \]
\end{defn}

\begin{bem}
  Es gilt
  \begin{alignat*}{4}
    m_\varphi(\vartheta_0) &= \P_{\vartheta_0} \left( (X_1, \ldots, X_n) \in K_n \right)
    &&= \\
    &= \P_{\vartheta_0} \left( (T(X_1), \ldots, T(X_n)) \in K_n^T \right)
    &&= \Int{K_n^T}{}{f_T(x)}{x} \leq \alpha,
  \end{alignat*}
  wobei $f_T$ die Dichte von $T(X_1, \ldots, X_n)$ unter $H_0$ ist.
\end{bem}

\begin{bsp}
  Sei $X \sim \Normal(\mu, \sigma^2)$, $\sigma$ bekannt und $\alpha \in \ointerval{0}{1}$ vorgegeben. \\
  Zum Test von $H_0 : \mu = \mu_0$ vs. $H_1 : \mu \neq \mu_0$ wählen wir als Statistik
  \[
    T(X_1, \ldots, X_n) \coloneqq \tfrac{\sqrt{n}}{\sigma} \left( \overline{X}_n - \mu_0 \right) \enspace
    \text{mit }
    \overline{X}_n \coloneqq \tfrac{1}{n} \left( X_1 + \ldots + X_n \right).
  \]
  Unter Annahme von $H_0$ gilt $T(X_1, \ldots, X_n) \sim \Normal(0,1)$. \\
  Der Ablehnungsbereich der Statistik ist
  \[
    K_n^T = \Set{t \in \R^1}{\abs{t} > z_{1 - \alpha/2}}
    \quad \text{mit} \quad
    z_{1 - \alpha/2} \coloneqq \Phi^{-1}(1 - \alpha/2).
  \]
  Für $\alpha = 0,5$ gilt beispielsweise $z_{1 - \alpha/2} \approx 1,96$.
\end{bsp}

% Vorlesung vom 15.10.2015

\begin{bem}
  Es gilt
  \begin{align*}
    t \in (K_n^T)^c
    &\iff \abs{t} \leq z_{1-\alpha/2}
    \iff \abs{\overline{X}_n - \mu_0} \leq \tfrac{\sigma}{\sqrt{n}} z_{1-\alpha/2} \\
    &\iff \mu_0 \in \cinterval{\overline{X}_n - \tfrac{\sigma}{\sqrt{n}} z_{1-\alpha/2}}{\overline{X}_n + \tfrac{\sigma}{\sqrt{n}} z_{1-\alpha/2}}.
  \end{align*}
  Letzteres Intervall wird \emph{Konfidenzintervall} für $\mu_0$ zum Konfidenz- niveau $1-\alpha$ genannt.
\end{bem}

\begin{bsp}
  Sei wieder $X \sim \Normal(\mu, \sigma^2)$, $\sigma^2$ aber diesmal unbekannt. \\
  Zum Testen von $H_0 : \mu = \mu_0$ vs. $H_1 : \mu \neq \mu_0$ verwenden wir
  \[
    \hat{T}(X_1, \ldots, X_n) = \tfrac{\sqrt{n}}{S_n} \left( \overline{X}_n - \mu_0 \right), \quad
    S_n^2 \coloneqq \tfrac{1}{n-1} \sum_{i=1}^n \left( X_i - \overline{X}_n \right)^2.
  \]
  Dabei ist $S_n$ die \emph{(korrigierte) Stichprobenvarianz}.
  Man kann zeigen, dass $\hat{T}(X_1, \ldots, X_n) \sim t_{n-1}$ unter $H_0$.
  Dabei ist $t_m$ die \emph{Student'sche $t$-Verteilung} mit $m$ Freiheitsgraden. \\
  Der Ablehnungsbereich ist
  \[ K_n^T = \Set{t \in \R^1}{\abs{t} > t_{n-1,1-\alpha/2}}. \]
\end{bsp}

\begin{bem}
  $S_n^2$ und $\overline{X}_n$ sind unabhängig für $n \geq 2$. % und umgekehrt?
\end{bem}

\begin{diskussion}
  \begin{itemize}
    \item Je kleiner $\alpha$ ist, desto "`nullhypothesenfreundlicher"' ist der Test.
    Häufig verwendet wird $\alpha \in \{ 10\%, 5\%, 1\%, 0,5\% \}$.
    \item Einseitige Tests: Die Gegenhypothese zu $H_0 \!:\! \mu \!=\! \mu_0$ ist $H_1 \!:\! \mu \!>\! \mu_0$.
    Die Nullhypothese wird nur abgelehnt, falls zu große Stichproben- mittelwerte $\overline{x}_n$ vorliegen. Es ist dann $K_n^T = \ointerval{z_{1-\alpha}}{\infty}$.
  \end{itemize}
\end{diskussion}

% 1.2. Prüfverteilung bei normalverteilten Grundgesamtheit

\begin{defn}
  Es seien $X_1, \ldots, X_n \sim \Normal(0, 1)$.
  Dann heißt die Summe $X_1^2 + \ldots + X_n^2 \sim \chi_n^2$ \emph{Chi-Quadrat-verteilt} mit $n$ Freiheitsgraden.
\end{defn}

\begin{defn}
  Falls $X \sim \Normal(0,1)$ und $Y_n \sim \chi_n^2$ unabhängig sind, so heißt
  \[ \tfrac{X}{\sqrt{\tfrac{Y_n}{n}}} \sim t_n \]
  \emph{$t$-verteilt} mit $n$-Freiheitsgraden.
\end{defn}

\begin{lem}
  $\tfrac{n-1}{\sigma^2} S_n^2 \sim \chi_{n-1}^2$
\end{lem}

\begin{kor}
  $\hat{T}$ aus dem zweiten obigen Bsp ist tatsächlich $t$-verteilt.
\end{kor}

\begin{defn}
  Seien $Y_{n_i} \sim \chi_{n_i}^2$, $i = 1, 2$ zwei unabhängige ZGen.
  Dann heißt %der Quotient
  \[ \tfrac{Y_{n_1} / n_1}{Y_{n_2} / n_2} \sim F_{n_1, n_2} \]
  \emph{F-verteilt} (wie Fisher) mit $(n_1, n_2)$ Freiheitsgraden.
  % auch: Suedecor-verteilt
\end{defn}

% Vorlesung vom 19.10.2015

\begin{bsp}
  Sei $X \sim \Normal(\mu, \sigma^2)$ mit $\mu$ unbekannt.
  Wir testen $H_0 : \sigma = \sigma_0$ vs. $H_1 : \sigma \neq \sigma_0$ mit
  \[ T \coloneqq \tfrac{n-1}{\sigma_0^2} S_n^2 \]
  Unter Annahme von $H_0$ gilt $T ~ \chi_{n-1}^2$.
  Falls $\mu$ bekannt ist, muss man
  \[
    \widetilde{T} \coloneqq \tfrac{n}{\sigma_0^2} \widetilde{S}_n^2, \quad
    \widetilde{S}_n^2 \coloneqq \tfrac{1}{n} \sum_{i=1}^n (X_i - \mu)^2.
  \]
  als Statistik wählen. Unter Annahme von $H_0$ ist $\widetilde{T} \sim \chi_n^2$.
\end{bsp}

\begin{bsp}
  Seien Stichproben $X_1^{(1)}, \ldots, X_{n_1}^{(1)} \sim \Normal(\mu_1, \sigma_1^2)$ und $X_1^{(2)}, \ldots, X_{n_2}^{(2)} \sim \Normal(\mu_2, \sigma_2^2)$ gegeben.
  Wir wollen $H_0 : \sigma_1 = \sigma_2$ gegen $H_1 : \sigma_1 \neq \sigma_2$ testen.
  Dazu verwenden wir
  \[
    T = \frac{S_{X^{(1)}}^2}{S_{X^{(2)}}^2}, \quad
    S_{X^{(j)}}^2 \coloneqq \tfrac{1}{n-1} \sum_{i=1}^{n_j} \left( X_i^{(j)} - \overline{X}^{(j)}_n \right)^n.
  \]
  Falls $H_0$ gilt, so ist $T \sim F_{n_1-1,n_2-1}$.
\end{bsp}

\begin{bsp}
  Situation wie im letzten Beispiel mit $\sigma_1 = \sigma_2$.
  Wir testen $H_0 : \mu_1 = \mu_2$ vs. $H_1 : \mu_1 \neq \mu_2$ mit
  \[
    T = \sqrt{\frac{n_1 \cdot n_2}{n_1 + n_2}} \cdot \frac{\overline{X}_{n_1}^{(1)} - \overline{X}_{n_2}^{(2)}}{S_{n_1,n_2}}, \quad
    S_{n_1,n_2}^2 = \frac{(n_1{-}1) S_{X^{(1)}}^2 + (n_2{-}1) S_{X^{(2)}}^2}{n_1 + n_2 - 2}
  \]
  Unter $H_0$ gilt $T \sim t_{n_1 + n_2 - 2}$.
\end{bsp}

\begin{bsp}
  Seien $\begin{psmallmatrix} X_1 \\ Y_1 \end{psmallmatrix}, \ldots, \begin{psmallmatrix} X_n \\ Y_n \end{psmallmatrix} \sim \Normal\left(
    \begin{psmallmatrix}
      \mu_1\vphantom{\sigma_1^2} \\
      \mu_2\vphantom{\sigma_2^2}
    \end{psmallmatrix},
    \begin{psmallmatrix}
      \sigma_1^2 & \sigma_1 \sigma_2 \rho \\
      \sigma_1 \sigma_2 \rho & \sigma_2^2
    \end{psmallmatrix} \right)$. \\[2pt]
  Wir testen $H_0 : \rho = 0$ vs. $H_1 : \rho \neq 0$ mit
  \[
    T \coloneqq \frac{\sqrt{n-2} \cdot \hat{\rho}_n}{\sqrt{1 - \hat{\rho}_n^2}}, \quad
    \hat{\rho}_n \coloneqq \frac{\tfrac{1}{n-1} \sum_{i=1}^n (X_i - \overline{X}_n) (Y_i - \overline{Y}_n)}{S_{X,n} \cdot S_{Y,n}}.
    % Nenner ausgeschrieben: \sqrt{\tfrac{1}{n-1} \sum_{i=1}^n (X_i - \overline{X}_n)^2 \tfrac{1}{n-1} \sum_{i=1}^n (Y_i - \overline{Y}_n)^2}
  \]
  Falls $H_0$ richtig ist, so gilt $T \sim t_{n-2}$. \\[2pt]
  Um $H_0 : \rho = \rho_0 \in \ointerval{0}{1}$ vs. $H_1 : \rho \neq \rho_0$ zu testen, kann man
  \[ T = \tfrac{\sqrt{n-3}}{2} \left( \log \tfrac{1 + \hat{\rho}_n}{1 - \hat{\rho}_n} - \log \tfrac{1+\rho_0}{1-\rho_0} \right) \]
  verwenden. Für $n$ groß gilt $T \sim \Normal(0, 1)$ unter $H_0$.
\end{bsp}

% 1.4. Lemma von Slutzky und varianzstabilisiernde Transformationen

\begin{lem}
  Seien $(X_n)$, $(Y_n)$ zwei Folgen von ZGn über $(\Omega, \Alg, \P)$ mit $X_n \xrightarrow[n \to \infty]{\P} c = \text{const}$ (\dh{} $\fa{\epsilon > 0} \P(\abs{X_n - c} > \epsilon) \to 0$) und $Y_n \xrightarrow[n \to \infty]{d} Y$ (\dh{} $\P(Y_n \leq y) \to \P(Y \leq y)$ für alle Stetigkeitspunkte $y$ der VF $y \mapsto \P(Y \leq y)$). Dann:
  \[
    X_n + Y_n \xrightarrow{d} c + Y, \quad
    X_n \cdot Y_n \xrightarrow{d} c \cdot Y, \quad
    Y_n / X_n \xrightarrow{d} Y / c \enspace \text{(falls $c \neq 0$)}.
  \]
\end{lem}

\begin{bem}
  Unabhängigkeit von $(X_n)$ und $(Y_n)$ wird nicht vorausgesetzt!
\end{bem}

\end{document}