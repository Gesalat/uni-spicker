\documentclass{cheat-sheet}

\pdfinfo{
  /Title (Zusammenfassung Spektralsequenzen)
  /Author (Tim Baumann)
}

\usepackage{tikz}
\usetikzlibrary{cd}
%\usetikzlibrary{calc}
%\usetikzlibrary{matrix,shapes,arrows,positioning}

% http://tex.stackexchange.com/questions/117732/tikz-and-babel-error
% Es ist schierer Wahnsinn, welche Hacks LaTeX benötigt!
\tikzset{
  every picture/.prefix style={
    execute at begin picture=\shorthandoff{"}
  }
}

% Kategorientheorie-Makros

% Konzepte
\DeclareMathOperator{\Ob}{Ob} % Objekte (einer Kategorie)
\DeclareMathOperator{\Mor}{Mor} % Morphismenmenge / -klasse
\DeclareMathOperator{\Hom}{Hom} % Homomorphisms
\DeclareMathOperator{\Nat}{Nat} % Natürliche Transformationen
\DeclareMathOperator{\dom}{dom} % Domain
\DeclareMathOperator{\codom}{codom} % Codomain
\newcommand{\op}{\mathrm{op}} % opposite category
\DeclareMathOperator{\Aut}{Aut} % Automorphismengruppe
\newcommand{\ladj}{\dashv} % Links-adjungiert (left-adjoint)
\newcommand{\Lim}{\lim} % Limes
\DeclareMathOperator{\colim}{colim} % Kolimes
\newcommand{\Colim}{\colim} % Kolimes
%\newcommand{\myint}[2]{{\textstyle \int\limits_{#1}^{#2}}}
\newcommand{\EndC}[2]{\myint{#1}{} #2} % Ende
\newcommand{\CoEndC}[2]{\myint{}{#1} #2} % Koende
\DeclareMathOperator{\Ran}{Ran} % Rechts-Kan-Erweiterung
\DeclareMathOperator{\Lan}{Lan} % Links-Kan-Erweiterung

% Platzhalter
\newcommand{\DiaTodo}{\fcolorbox{red}{white}{TODO: Diagramm einfügen!}}

% Konkrete Kategorien
\newcommand{\SetC}{\mathbf{Set}} % Kategorie der Mengen
\newcommand{\sSet}{\mathbf{sSet}} % Kategorie der simplizialen Mengen
\newcommand{\Top}{\mathbf{Top}} % Kategorie der topologischen Räume
\newcommand{\AbGrp}{\mathbf{Ab}} % Kategorie der abelschen Gruppen
\newcommand{\Grp}{\mathbf{Grp}} % Kategorie der Gruppen
\newcommand{\RMod}{\mathbf{R\text{-}Mod}} % Kategorie der R-Moduln
\newcommand{\Ouv}{\mathbf{Ouv}} % Kategorie der offenene Mengen eines topol. Raumes
\newcommand{\KHaus}{\mathbf{KHaus}} % Kategorie der kompakten Hausdorffräume
\newcommand{\CatC}{\mathbf{Cat}} % Kategorie der kleinen Kategorien
\newcommand{\Vect}{\mathbf{Vect}} % Kategorie der Vektorräume über einem Körper
\newcommand{\Alg}{\mathbf{Alg}} % Kategorie der Algebren über einem Körper/Ring
\newcommand{\VectFin}{\mathbf{Vect}_{\mathrm{fin}}} % Kategorie der endlichen Vektorräume über einem Körper
\newcommand{\kVect}{\text{$k$-$\Vect$}} % Kategorie der k-Vektorräume über einem Körper k
\newcommand{\kVectFin}{\text{$k$-$\VectFin$}} % Kategorie der endlichen k-Vektorräume über einem Körper k
\newcommand{\Mod}{\mathbf{Mod}} % Kategorie der Moduln über einem Ring
\newcommand{\Kom}{\mathbf{Kom}} % Kategorie der Komplexe in einer abelschen Kategorie
\newcommand{\Der}{\mathcal{D}} % abgeleitete Kategorie einer abelschen Kategorie
\newcommand{\kAlg}{k\text{-}\Alg} % Kategorie der k-Algebren

% Bezeichnungen für Variablen, die für Kategorien stehen
\newcommand{\Aat}{\mathcal{A}} % Category-A
\newcommand{\Bat}{\mathcal{B}} % Category-B
\newcommand{\Cat}{\mathcal{C}} % Category-C
\newcommand{\Dat}{\mathcal{D}} % Category-D
\newcommand{\Eat}{\mathcal{E}} % Category-E
\newcommand{\Fat}{\mathcal{F}} % Category-F
\newcommand{\Gat}{\mathcal{G}} % Category-G
\newcommand{\Iat}{\mathcal{I}} % Category-I (Indexkategorie)
\newcommand{\Jat}{\mathcal{J}} % Category-J (Indexkategorie)
\newcommand{\MatC}{\mathcal{M}} % Category-M
\newcommand{\Sit}{\mathcal{S}} % Situs-S
 % Kategorientheorie-Makros

%\newcommand{\keS}{k.\,e.\,S.} % kurze exakte Sequenz
\newcommand{\leS}{l.\,e.\,S.} % lange exakte Sequenz
\newcommand{\nacong}{\!\cong\!} % narrow cong
%\newcommand{\ZM}[1]{\Z_{#1}} % \Z modulo n

\newenvironment{centertikzcd}
  {\begin{center}\begin{tikzcd}}
  {\end{tikzcd}\end{center}}

% Kleinere Klammern
\delimiterfactor=701


\begin{document}

\maketitle{Spektralsequenzen}

Sei $\Aat$ im Folgenden eine abelsche Kategorie.

\begin{defn}
  Eine (homologische) \emph{Spektralsequenz} (SS) besteht aus
  \begin{itemize}
    \item Objekten $E^r_{p,q} \in \Ob(\Aat)$ für alle $p, q \in \Z$ und $r \geq 1$,
    \item Morphismen $d^r_{p,q} : E^r_{p,q} \to E^r_{p-r,q+r-1}$ mit $d^r_{p-r,q+r-1} \circ d^r_{p,q} = 0$
    \item und Isos $\alpha : H_{p,q}(E^r) \!\coloneqq\! \ker(d^r_{p,q}) / \im(d^r_{p+r,q-r+1}) \xrightarrow{\cong} E^{r+1}_{p,q}$.
  \end{itemize}
\end{defn}

% XXX: Morphismus von Spektralsequenzen

\begin{sprech}
  \begin{itemize}
    \item Die Morphismen $d^r_{p,q}$ heißen \emph{Differentiale}.
    \item Die Gesamtheit $E^r \coloneqq \{ E^r_{p,q} \}_{p,q}$ mit $r \in \N$ fest heißt $r$-te \emph{Seite}.
  \end{itemize}
\end{sprech}

\begin{bem}
  Man stellt Seiten in einem 2-dim Raster dar:
  \begin{tikzpicture}[x=10pt,y=10pt]
    \begin{scope}[shift={(0,0)}]
      \foreach \x in {-1,...,4}{
        \foreach \y in {-2,-1,...,3}{
          \node[draw,circle,inner sep=0.5pt,fill] at (\x,\y) {};
        }
      }
      \foreach \x in {-1,...,5}{
        \foreach \y in {-2,-1,...,3}{
          \draw[->,gray] (\x-0.2,\y) -- (\x-0.8,\y);
        }
      }
      \draw[->] (-0.35,-0.35) -- (5.35,-0.35) node[below] {p};
      \draw[->] (-0.35,-0.35) -- (-0.35,4) node[left] {q};
      \node at (5,4.5) {$E^1$};
    \end{scope}
    \begin{scope}[shift={(8.5,0)}]
      \foreach \x in {-1,...,4}{
        \foreach \y in {-2,-1,...,3}{
          \node[draw,circle,inner sep=0.5pt,fill] at (\x,\y) {};
        }
      }
      \foreach \x in {0,...,5}{
        \foreach \y in {-2,-1,...,3}{
          \draw[->,gray] (\x-0.2,\y+0.1) -- (\x-1.8,\y+0.9);
        }
      }
      \draw[->] (-0.35,-0.35) -- (5.35,-0.35) node[below] {p};
      \draw[->] (-0.35,-0.35) -- (-0.35,4) node[left] {q};
      \node at (5,4.5) {$E^2$};
    \end{scope}
    \begin{scope}[shift={(17,0)}]
      \foreach \x in {-1,...,4}{
        \foreach \y in {-2,-1,...,3}{
          \node[draw,circle,inner sep=0.5pt,fill] at (\x,\y) {};
        }
      }
      \foreach \x in {1,...,5}{
        \foreach \y in {-2,-1,...,2}{
          \draw[->,gray] (\x-0.2,\y+0.1) -- (\x-2.8,\y+1.9);
        }
      }
      \draw[->] (-0.35,-0.35) -- (5.35,-0.35) node[below] {p};
      \draw[->] (-0.35,-0.35) -- (-0.35,4) node[left] {q};
      \node at (5,4.5) {$E^3$};
    \end{scope}
  \end{tikzpicture}
  \vspace{4pt}

  Die Differentiale in $E^2$ laufen wie Springer-Züge beim Schach.
\end{bem}

\begin{bem}
  Bei einer kohomologischen Spektralsequenz sind die Indizes vertauscht und die Differentiale laufen $d_r^{p,q} : E_r^{p,q} \to E_r^{p+r,q-r+1}$.
\end{bem}

\begin{nota}
  Man verwendet auch eine zweite, alternative Indizierung:
  $E^*_{n,p} :=: E^*_{p,q}$ mit $n = p + q$.
\end{nota}

\begin{defn}
  Ein Morphismus $E \to E'$ von (homol.) Spektralsequenzen ist gegeben durch Abbildungen $f^r_{p,q} : E^r_{p,q} \to E'^r_{p,q}$ mit
  \begin{itemize}
    \miniitem{0.6 \linewidth}{$d'^r_{p,q} \circ f^r_{p,q} = f^r_{p-r,q+r-1} \circ d^r_{p,q}$,}
    \miniitem{0.37 \linewidth}{$f^{r+1}_{p,q} = H_{p,q}(f^r_{p,q})$.}
  \end{itemize}
\end{defn}

\begin{defn}
  Eine Spektralsequenz \emph{konvergiert}, falls für alle $p, q \in \Z$ ein $R \in \N$ existiert, sodass für alle $r \geq R$ die Differentiale von und nach $E^r_{p,q}$ null sind und damit $E^\infty_{p,q} \coloneqq E^R_{p,q} \cong E^{R+1}_{p,q} \cong E^{R+2}_{p,q} \ldots$ \\
  Der \emph{Grenzwert} der SS ist die Unendlich-Seite $E^\infty \coloneqq \{ E^\infty_{p,q} \}_{p,q}$.
\end{defn}

\begin{nota}
  $E^r \Rightarrow E^\infty$
\end{nota}

\begin{defn}
  Eine SS \emph{degeneriert} auf Seite $R$, wenn $d^r_{p,q} \!=\! 0$ für alle $r \!\geq\! R$.
\end{defn}

\iffalse
\begin{bem}
  Das entspricht einer Art gleichmäßigen Konvergenz.
\end{bem}
\fi

\begin{bem}
  Viele Spektralsequenzen leben im ersten Quadranten, \dh{} $E^r_{p,q} = 0$ wenn $p < 0$ oder $q < 0$. Das impliziert, dass für $p$, $q$ fest und $r$ groß alle Differentiale von und nach $E^r_{p,q}$ aus dem ersten Quadranten heraus- oder hineinführen und damit Null sind.
  %Somit konvergieren solche Spektralsequenzen immer.
\end{bem}

\begin{defn}
  Ein \emph{exaktes Pärchen} $(A, E)$ in $\Aat$ ist gegeben durch Objekte $A, E \in \Ob(\Aat)$ und Morphismen wie folgt
  \begin{centertikzcd}
    A \arrow[rr, "i"] &&
    A \arrow[ld, "j"] \\
    & E \arrow[lu, "k"]
  \end{centertikzcd}
  sodass das Dreieck an jeder Ecke exakt ist.
\end{defn}

\begin{bem}
  Für das Differential $d \coloneqq j \circ k : E \to E$ gilt $d^2 = 0$.
\end{bem}

\begin{defn}
  Sei ein exaktes Pärchen $(A, E)$ gegeben. \\
  Dann gibt es ein \emph{abgeleitetes Pärchen} $(A', E')$
  \begin{centertikzcd}
    A' \arrow[rr, "i'"] &&
    A' \arrow[ld, "j'"] \\
    & E' \arrow[lu, "k'"]
  \end{centertikzcd}
  mit \quad
  \inlineitem{$E' \coloneqq \ker(d) / \im(d)$,} \quad
  \inlineitem{$A' \coloneqq i(A) \subset A$,} \\
  \inlineitem{$i' \coloneqq i|_{A'}$} \quad
  \inlineitem{$j'(i(a)) \coloneqq [j(a)] \in E'$} \quad
  \inlineitem{$k'([e]) \coloneqq k(e)$}
\end{defn}

% 1.1 im Hatcher
\begin{lem}
  Das abgeleitete Pärchen eines exakten Pärchens ist exakt.
\end{lem}

\begin{bem}
  Man erhält nun aus einem exakten Pärchen $(A^1, E^1)$ durch wiederholtes Ableiten eine Folge von exakten Pärchen $(A^r, E^r)_{r \in \N}$. Die $E^r$ bilden mit $d^r \!:\! E^r \!\to\! E^r$ eine Spektralseq. im folgenden Sinne:
\end{bem}

\begin{bem}
  Man kann auch die $r$-te Seite als einzelnes Obj. $E^r$ auffassen. Dann ist eine \emph{Spektralsequenz} gegeben durch Objekte $E^r$, $r \geq 1$, Differentiale $d^r : E^r \to E_r$ mit $d^r \circ d^r = 0$ und Isomorphismen $\alpha^r : H(E^r) \coloneqq \ker(d^r) / \im(d^r) \to E^{r+1}$.
\end{bem}

\begin{bem}
  Sei $\ldots \subseteq X_p \subseteq X_{p+1} \subseteq \ldots$ eine aufsteigende Filtrierung eines topologischen Raumes $X$. Man kann dann die Homologiegruppen (mit Koeffizienten implizit) übersichtlich in ein Raster schreiben:
  \begin{tikzcd}[row sep=0.2cm, column sep=0.25cm, font=\scriptsize]
    \arrow[d] && \arrow[d] \\
    \Emph{H_{n\!+\!1}(X_p)} \arrow[r] \arrow[d, Emph] & H_{n\!+\!1}(X_p, X_{p\!-\!1}) \arrow[r] & H_n(X_{p\!-\!1}) \arrow[r] \arrow[d] & H_n(X_{p\!-\!1}, X_{p\!-\!2}) \arrow[r] & \, \\
    \Emph{H_{n\!+\!1}(X_{p\!+\!1})} \arrow[r, Emph] \arrow[d] & \Emph{H_{n\!+\!1}(X_{p\!+\!1}, X_p)} \arrow[r, Emph] & \Emph{H_n(X_p)} \arrow[r] \arrow[d, Emph] & H_n(X_p, X_{p\!-\!1}) \arrow[r] & \, \\
    H_{n\!+\!1}(X_{p\!+\!2}) \arrow[r] \arrow[d] & H_{n\!+\!1}(X_{p\!+\!2}, X_{p\!+\!1}) \arrow[r] & \Emph{H_n(X_{p\!+\!1})} \arrow[r, Emph] \arrow[d] & \Emph{H_n(X_{p\!+\!1}, X_{p})} \arrow [r, Emph] & \, \\
    \, & \, & \, & \,
  \end{tikzcd}
  Die langen exakten Sequenzen von Raumpaaren liegen treppenstu- fenartig in diesem Raster.
  Man erhält aus den langen Morphismen wie in den \leS{} (rechts, rechts, runter) ein exaktes Pärchen $(A, E)$ mit $A^1_{n,p} \coloneqq H_n(X_p)$ und $E_{n,p}^1 \coloneqq H_n(X_p, X_{p-1})$.
  Beim Bilden des abgeleiteten Pärchens verschieben sich die exakten Sequenzen:
  \vspace{-8pt}
  \begin{centertikzcd}[row sep=0.2cm, column sep=0.45cm, font=\scriptsize]
    & \, \arrow[d] & \phantom{E^2_{n+1,p-1}} & \, \arrow[d] & \phantom{E^2_{n,p-2}} \\
    \arrow[r, Emph] & \Emph{A^2_{n+1,p}} \arrow[ru] \arrow[d, Emph] & \Emph{E^2_{n+1,p}} \arrow[r, Emph] & \Emph{A^2_{n,p-1}} \arrow[ru] \arrow[d, Emph] & \Emph{E^2_{n,p-1}} \arrow[r, Emph] & \, \\
    \arrow[r] & \Emph{A^2_{n+1,p+1}} \arrow[ru, Emph] \arrow[d] & E^2_{n+1,p+1} \arrow[r] & \Emph{A^2_{n,p}} \arrow[ru, Emph] \arrow[d] & E^2_{n,p} \arrow[r] & \, \\
    \arrow[r] & A^2_{n+1,p+2} \arrow[ru] \arrow[d] & E^2_{n+1,p+2} \arrow[r] & A^2_{n,p+1} \arrow[ru]  \arrow[d] & E^2_{n,p+1} \arrow[r] & \, \\
    & \phantom{A^2_{n+1,p+3}} \arrow[ru] & \, & \phantom{A^2_{n,p+1}} \arrow[ru] & \, & \, &
  \end{centertikzcd}
\end{bem}

\begin{defn}
  Eine \emph{Filtrierung} eines $A$-Moduls $M$ ist eine aufsteigende Folge $\nldots \subseteq F_p M \subseteq F_{p+1} M \subseteq \nldots$ von Untermodulen von $M$ mit $p \in \Z$, sodass $0 = \cap_p F_p M$ und $M = \cup_p F_p M$.
\end{defn}
% XXX: beschränkte Filtrierungen

% 1.2 im Hatcher
\begin{prop}
  Angenommen, in jeder $A^1$-Spalte sind alle bis auf endlich viele Morphismen Isomorphismen. Dann hat man in jeder Spalte stabile obere und untere Werte $A^1_{n, \mp \infty}$. Außerdem konvergiert dann die Spektralsequenz der $E^r_{n,p}$.
  \begin{itemize}
    \item Falls $A^1_{n,-\infty} = 0$ für alle $n$, dann ist $E^\infty_{n,p} \cong F^p_n / F^{p-1}_n$, wobei
    \[ F^p_n \coloneqq \im(A^1_{n,p} \to A^1_{n,\infty}) \subseteq A^1_{n,\infty} \]
    eine Filtrierung $\ldots \subseteq F^{p-1}_n \subseteq F^p_n \subseteq \ldots$ von $A^1_{n,\infty}$ ist.
    \item Falls $A^1_{n,\infty} = 0$ für alle $n$, dann ist $E^\infty_{n,p} \cong F^{n-1}_p / F^{n-1}_{p-1}$ mit
    \[ F^p_{n-1} \coloneqq \ker(A^1_{n-1,-\infty} \to A^1_{n-1,p}) \subseteq A^1_{n-1,-\infty}. \]
  \end{itemize}
\end{prop}

\begin{bem}
  Die erste Bedingung ist äquivalent dazu, dass in jeder $E^1$-Spalte nur endlich viele Objekte ungleich Null sind. \\
  Angenommen, die Filtrierung des Raums $X$ erfüllt $X_p = \emptyset$ für $p < 0$.
  Im Homologiesetting ist die Bed. erfüllt, wenn es für alle $n$ ein $p$ gibt, sodass $H_n(X_p) \nacong H_n(X_{p+1}) \nacong \ldots \nacong H_n(X)$ induziert durch Inklusion.
  Dann gilt $A^1_{n,\infty} \!=\! H_n(X; G)$ und $F^p_n \!=\! \im(H_n(X_p; G) \to H_n(X; G))$. \\
  Man sagt, die SS konvergiere gegen $H_*(X; G)$.
\end{bem}

\begin{bem}
  Oft ist $X$ ein CW-Komplex und die Filtrierung $X_p$ gegeben durch die $p$-Skelette von $X$. Dann ist $E^1_{n,p} \coloneqq H_n(X_p, X_{p-1}; G) = 0$ für $n < p$, da $(X_p, X_{p-1})$ $(p\!-\!1)$-zusammenhängend ist. Das ist der Grund für die zwei alternativen Notationen $E^*_{n,p}$ und $E^*_{p,q}$.
\end{bem}

\begin{bem}
  In Kohomologie gibt es eine SS mit $A^{n,p}_1 \coloneqq H^n(X_p)$ und $E^{n,p}_1 = H^n(X_p, X_{p-1})$. Für die Proposition benötigt man dann $H^n(X) \nacong \ldots \nacong H^n(X_{p+1}) \nacong H^n(X_p)$ durch Inklusion für alle $n$ und $p$ groß. Dann ist $E^{n,p}_\infty \cong F^n_p / F^n_{p+1}$ mit $F^n_p \!=\! \ker(H^n(X) \!\to\! H^n(X_{p-1}))$.
\end{bem}

\subsection{Die Leray-Serre-Spektralsequenz}

\begin{defn}
  Eine \emph{Serre-Faserung} ist eine stetige Abb. $p : E \to B$, die die Homotopieliftungseigenschaft für alle CW-Komplexe $A$ erfüllt, \dh{}
  für alle $H$, $H_0$ wie unten, sodass das Quadrat kommutiert, gibt es ein diagonales $\tilde{H}$, sodass die Dreiecke kommutieren:
  \begin{centertikzcd}[row sep=0.8cm, column sep=1cm]
    A \arrow[r, "H_0"] \arrow[d, hook, "i_0"] &
    E \arrow[d, "p"] \\
    A \times \I \arrow[r, "H"] \arrow[ur, "\exists\, \tilde{H}", dashed] &
    B
  \end{centertikzcd}
\end{defn}

\begin{lem}
  Die Homotopieliftungseig. ist genau dann für alle CW-Kom- plexe erfüllt, wenn sie für die Kuben $A = \I^n$ erfüllt ist.
\end{lem}

\begin{bem}
  Jeder stetige Weg $\gamma : \I \to B$ in $B$ induziert eine Homoto- pieäquivalenz $\gamma_* : p^{-1}(\gamma(0)) \to p^{-1}(\gamma(1))$ zwischen den Fasern über Anfangs- und Endpunkt.
  Wenn $B$ wegzshgd ist, so sind alle Fasern homotopieäquivalent und man notiert $F \to E \to B$ für die Faserung, wobei $F$ die Faser über einem beliebigen Punkt ist. Man erhält eine Wirkung der Fundamentalgr. $\pi_1(B)$ auf der Homologie $H_k(F)$ durch
  \[ \pi_1(B) \to \Aut(H_k(F)), \quad [(\gamma : \I \to B)] \mapsto (\gamma_* : F \to F)_* \]
\end{bem}

% 1.3 im Hatcher
\begin{thm}
  Sei $F \to X \to B$ eine Serre-Faserung, $B$ wegzshgd und $G$ eine ab. Gruppe. Angenommen, $\pi_1(B)$ wirkt trivial auf $H_*(F; G)$. \\
  Dann gibt es die \emph{(Leray-)Serre-Spektralsequenz} mit
  \[ E^2_{p,q} = H_p(B; H_q(F; G)), \]
  deren Eintrag $E^\infty_{p,q} = E^\infty_{p,n-p}$ der Quotient $F^p_n/F^{p-1}_n$ in einer Filtration
  $0 \subseteq F_n^0 \subseteq \ldots \subseteq F_n^n = H_n(X; G)$ von $H_n(X; G)$ ist.
\end{thm}
% XXX: Verallgemeinerung auf lokale Koeffizienten?

\begin{konstr}
  Für einen CW-Komplex $B$ und Faserung $p : X \to B$ mit Faser $F$ sei $X_p \coloneqq p^{-1}(B_p)$ das Urbild des $p$-Skeletts von $B$. \\
  Da $(X, X_p)$ $p$-zshgd ist, induziert $X_p \hookrightarrow X$ einen Isomorphismus $H_n(X_p; G) \cong H_n(X; G)$ für $n < p$.
  Man zeigt, dass für die von dieser Faserung von $X$ induzierte SS gilt: $E^2_{p,q} \cong H_p(B; H_q(F; G))$.
\end{konstr}

\begin{bem}
  Wenn $G$ ein Körper ist, so folgt $H_n(X; G) \cong \oplus_p E^\infty_{p,n-p}$.
\end{bem}

% Ausgelassen: Bei Faserbündeln: "pathspace construction"

\begin{thm}
  Sei $F \to X \xrightarrow{p} B$ eine Serre-Faserung, die die Bedingungen für Ex. der Serre-SS erfüllt, $B' \!\subset\! B$ ein Unterraum, $X' \!\coloneqq\! p^{-1}(B')$. \\
  Dann gibt es eine \emph{relative (Leray-)Serre-Spektralsequenz} mit
  \[ E^2_{p,q} = H_p(B, B'; H_q(F; G)), \]
  welche gegen $H_*(X, X'; G)$ konvergiert.
\end{thm}

\begin{bem}
  Sei eine Abbildung $(f, \tilde{f})$ von Faserungen, die die Voraus- setzungen für Existenz der Serre-SS erfüllen, wie folgt gegeben:
  \vspace{-4pt}
  \begin{centertikzcd}[row sep=0.35cm, column sep=0.6cm]
    F \arrow[r] \arrow[d] & X \arrow[r, "p"] \arrow[d, "\tilde{f}"] & B \arrow[d, "f"] \\
    F' \arrow[r] & X' \arrow[r, "p'"] & B'
  \end{centertikzcd}
  \vspace{-4pt}
  Dann gibt es ind. Morphismus $f_* : E \to E'$ der zugeh. Serre-SS, der für $r \to \infty$ gegen die Abb. $\tilde{f}_* : H_*(X; G) \to H_*(X'; G)$ konvergiert, welche mit den Filtrierungen $F^p_n$ und $F'^p_n$ verträglich ist.
  Außerdem entspricht der Morphismus bei $E^2_{p,q}$ der von $B \to B'$ und $F \to F'$ induzierten Abbildung $H_p(B; H_q(F, G)) \to H_p(B'; H_q(F', G))$. \\
  Die Zuordnung $(f, \tilde{f}) \mapsto f_*$ ist funktoriell.
\end{bem}

% 1.12 im Hatcher
\begin{prop}
  Sei eine Abbildung $(f, \tilde{f})$ von Faserungen wie oben, $R$ ein Hauptidealbereich.
  Wenn zwei der Abbildungen $F \to F'$, $B \to B'$ und $X \to X'$ Isomorphismen in Homologie mit $R$-Koeffizienten induzieren, so auch die dritte.
\end{prop}
% TODO: Algebraische Variante des "SS Comparison Theorems"

\begin{defn}
  Sei $F \to X \xrightarrow{p} B$ eine Serre-Faserung. Betrachte
  \[ H_n(B) \xrightarrow{i_*} H_n(B, b_0) \xleftarrow{p_*} H_n(X, F) \xrightarrow{\partial} H_{n-1}(F) \]
  Dabei ist $i_*$ ein Iso für $n > 0$, $p_*$ aber im Allgemeinen nicht. \\
  Die \emph{Transgression} ist die induzierte Abbildung
  \[ t_n : i_*^{-1}(\im(p_*)) \to H_{n-1}(F) / \partial(\ker(p_*)). \]
\end{defn}

\begin{bem}
  Die Transgression entspricht dem Verbindungsmorphismus $\pi_n(B) \to \pi_{n-1}(F)$ in der \leS{} von Homotopiegr. einer Faserung.
  Manchmal wird auch die additive Relation $R \subseteq H_n(B) \times H_{n-1}(F)$ als Transgression bezeichnet.
\end{bem}

% 1.13 im Hatcher
\begin{prop}
  Die Transgression ist gleich einem Differential der Serre-SS:
  \[ t_n = d_n : E^{n}_{n,0} \to E^n_{0,n-1}. \]
\end{prop}

% TODO: Lokale Koeffizienten

% 1.14 im Hatcher
\begin{thm}
  Sei $F \to E \to B$ eine Serre-Faserung, $B$ wegzshgd und $G$ eine ab. Gruppe. Angenommen, $\pi_1(B)$ wirkt trivial auf $H^*(F; G)$.
  Dann ex. die \emph{(Leray-)Serre-Spektralsequenz} für Kohomologie mit
  \[ E_2^{p,q} = H^p(B; H^q(F; G)), \]
  deren Eintrag $E_\infty^{p,n-p}$ der Quotient $F_p^n/F_{p+1}^n$ in einer Filtration
  $0 \subseteq F_n^n \subseteq \ldots \subseteq F_0^n = H^n(X; G)$ von $H^n(X; G)$ ist.
\end{thm}

\begin{lem}
  Sei $F \!\to\! E \!\to\! B$ eine Serre-Faserung, die die Bed. für Existenz der Serre-SS erfüllt und  $R$ ein Ring.
  Dann gibt bilineare Abbildungen
  \[ d^{(p,q,s,t)}_r : E^{p,q}_r \times E^{s,t}_r \to E^{p+s,q+t}_r, \enspace (x, y) \mapsto xy \]
  mit folgenden Eigenschaften:
  \begin{itemize}
    \item $d_r$ ist derivativ: $d_r(xy) = (d_r x) y + (-1)^{p+q} x (d_r y)$
    \item Die Abbildung $d_{r+1}$ ist gegeben durch
    \[
      d_{r+1} : E^{p,q}_{r+1} \times E^{s,t}_{r+1} \to E^{p+s,q+t}_{r+1}, \quad
      ([x], [y]) \mapsto [xy].
    \]
    Diese ist wohldefiniert wegen Derivativität.
    \item $d_2 : E_2^{p,q} \!\times\! E_2^{r,s} \!\to\! E_2^{p+s,q+t}$ ist das $(-1)^{qs}$-fache des Cup-Produkts
    \[ H^p(B; H^q(F; R)) \times H^s(B; H^t(F; R)) \to H^{p+s}(B; H^{q+t}(F; R)), \]
    wobei Koeffizienten mit dem Cup-Produkt von $H^*(F; R)$, $H^q(F; R) \times H^q(F; R) \to H^{q+t}(F; R)$, multipliziert werden.
    \item Das Cup-Produkt in $H^*(X; R)$ respektiert die Faserungen von $H^n(X; R)$ und schränkt daher ein zu Abb. $F_p^m \times F_s^n \to F_{p+s}^{m+n}$.
    Die induzierte Abbildung auf dem Quotienten $F^m_p/F^m_{p+1} \times F^n_s/F^n_{s+1} \to F^{m+n}_{p+s} / F^{m+n}_{p+s+1}$ entspricht $d_\infty : E_\infty^{p,m-p} \times E_\infty^{s,n-s} \to E_\infty^{p+s,m+n-p-s}$.
  \end{itemize}
\end{lem}

% TODO: Leray Spectral Sequence
% TODO: May Spectral Sequence
% TODO: Spektralsequenz eines Doppelkomplexes

\end{document}
