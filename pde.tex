\documentclass{cheat-sheet}

\pdfinfo{
  /Title (Zusammenfassung Topologie)
  /Author (Tim Baumann)
}

\newcommand{\Tau}{\mathcal{T}} % Großes Tau
\newcommand{\inte}{\mathop{\mathrm{int}}} % Inneres (interior)
\newcommand{\grad}{\mathrm{grad}} % Gradient
\newcommand{\dive}{\mathrm{div}} % Divergenz
\newcommand{\divergence}{\mathrm{div}} % Divergenz

% Abschnittsnummerierung einschalten (entgegen cheat-sheet.cls)
\makeatletter
  % Abstand von Nummerierung und Titel verringern
  \renewcommand*{\@seccntformat}[1]{\csname the#1\endcsname\hspace{0.2cm}}
\makeatother
\renewcommand{\thesection}{\arabic{section}.} % Punkt nach Nummer
\setcounter{secnumdepth}{1}

% Kleinere Klammern
\delimiterfactor=701

% Integral mit Strich durch, siehe
% http://www.tex.ac.uk/cgi-bin/texfaq2html?label=prinvalint
\def\Xint#1{\mathchoice
   {\XXint\displaystyle\textstyle{#1}}%
   {\XXint\textstyle\scriptstyle{#1}}%
   {\XXint\scriptstyle\scriptscriptstyle{#1}}%
   {\XXint\scriptscriptstyle\scriptscriptstyle{#1}}%
   \!\int}
\def\XXint#1#2#3{{\setbox0=\hbox{$#1{#2#3}{\int}$}
     \vcenter{\hbox{$#2#3$}}\kern-.5\wd0}}
%\def\ddashint{\Xint=}
\def\dashint{\Xint-}

% Mittelwerts-Integrale (mean value integrals)
\newcommand{\mymvint}[2]{{\textstyle \dashint\limits_{#1}^{#2}}}
\newcommand{\MVInt}[4]{\mymvint{#1}{#2} #3 \,\d #4}

\newcommand{\HM}{\mathcal{H}} % Hausdorff-Maß
\usepackage{bbm} % Für 1 mit Doppelstrich (Indikatorfunktion)
\newcommand{\ind}{\mathbbm{1}} % Indikatorfunktion
\newcommand{\dist}{\mathrm{dist}} % Entfernung (distance)

\begin{document}

\maketitle{Zusammenfassung Partielle DGLn}

% Vorlesung vom 8.4.2014

% Kapitel 1.
\section{Einleitung}

% Kapitel 1.1. Partielle Differentialgleichungen und klassische Lösungen

% Ausgelassen: Definition ODE

\begin{defn}
  Eine \emph{partielle Differentialgleichung} (PDGL) hat die Form
  \begin{align*}
    E(x, u(x), Du(x), ..., D^k u(x)) = 0 \quad \text{in $\Omega \subset \R^n$ offen}, \tag{$\star$}
  \end{align*}
  wobei $E : \Omega \times \R \times \R^n \times ... \times \R^{n^k} \to \R$ gegeben und $u : \Omega \to \R$ gesucht ist.
  Die höchste Ableitungsordnung von $u$, die in $E$ vorkommt, heißt \emph{Ordnung} der PDGL.
\end{defn}

% Nicht definiert: Multiindizes

\begin{defn}
  Eine PDGL von der Ordnung $k$ heißt
  \begin{itemize}
    \item \emph{linear}, falls sie folgende Form besitzt:
    \[ \sum_{\mathclap{\abs{\alpha} \leq k}} a_\alpha(x) D^{\alpha} u(x) - f(x) = 0 \]
    \item \emph{semilinear}, falls sie linear in der höchsten Ableitungsordnung ist, man sie also schreiben kann als
    \[ \sum_{\mathclap{\abs{\alpha} = k}} a_\alpha(x) D^{\alpha} u(x) + E_{k-1}(x, u(x), Du(x), ..., D^{k-1} u(x)) = 0. \]
    \item \emph{quasilinear}, falls sie sich schreiben lässt als
    \begin{align*}
      & \sum_{\mathclap{\abs{\alpha} = k}} a_\alpha(x, u(x), D u(x), ..., D^{k-1} u(x)) D^{\alpha} u(x)\\[-3pt]
      +\, & E_{k-1}(x, u(x), D u(x), ..., D^{k-1} u(x)) = 0.
    \end{align*}
    \item sonst \emph{voll nichtlinear}.
  \end{itemize}
\end{defn}

\begin{bem}
  $\{\text{ lineare PDGLn }\} \subsetneq \{\text{ semilineare PDGLn }\} \subsetneq \{\text{ quasilineare PDGLn }\} \subsetneq \{\text{ PDGLn }\}$
\end{bem}

% Ausgelassen: Schwierigkeitsfaustregel

% Typeinteilung für lineare PDGLn 2. Ordnung
\begin{defn}[Typeinteilung für lineare PDGLn 2. Ordnung]
  Seien $a_{ij}, b_i$, $c, f : \Omega \to \R$ ($i, j \in \{ 1, ..., n \}$) vorgegebene Fktn. auf $\Omega \subset \R^n$ offen.
  \begin{itemize}
    \item Die lineare PDGL
    \[ \sum_{1 \leq i, j \leq n} a_{ij}(x) D_i D_j u(x) + \sum_{1 \leq j \leq n} b_j(x) D_j u(x) + c(x) u(x) + f(x) = 0 \]
    heißt \emph{elliptisch}, falls die $(n \times n)$-Matrix $(a_{ij})_{1 \leq i,j \leq n}$ für alle $x \in \Omega$ positiv definit ist.
    \item Die lineare PDGL
    \[ D_1 D_1 u(x) - \sum_{\mathclap{2 \leq i, j \leq n}} a_{j}(x) D_i D_j u(x) + \sum_{\mathclap{1 \leq i \leq n}} b_i(x) D_i u(x) + c(x) u(x) + f(x) = 0 \]
    heißt \emph{hyperbolisch}, falls die $(n{-}1) \times (n{-}1)$-Matrix $(a_{ij})_{2 \leq i, j \leq n}$ für alle $x \in \Omega$ positiv definit ist.
    \item Die lineare PDGL
    \[ D_1 u(x) - \sum_{\mathclap{2 \leq i, j \leq n}} a_{ij}(x) D_i D_j u(x) + \sum_{\mathclap{2 \leq i \leq n}} b_i(x) D_i u(x) + c(x) u(x) + f(x) = 0 \]
    heißt \emph{parabolisch}, falls die $(n{-}1) \times (n{-}1)$-Matrix $(a_{ij})_{2 \leq i,j \leq n}$ für alle $x \in \Omega$ positiv definit ist.
  \end{itemize}
\end{defn}

% Modellfälle: Laplace-Gleichung, Wellengleichung, Wärmeleitungsgleichung

% Ausgelassen: Zielsetzung der Theorie der PDGLn

\begin{defn}
  Eine Funktion $u : \Omega \to \R$ heißt \emph{klassische Lösung}, falls $u \in \mathcal{C}^k(\Omega)$ und die Differentialgleichung ($\star$) überall in $\Omega$ erfüllt ist.
\end{defn}

% Kapitel 1.2. Einige Beispiele partieller Differentialgleichungen

% (ausgelassen)

% Vorlesung vom 10.4.2014

% Kapitel 2.
\section{Laplace- und Poisson-Gleichung}

\begin{nota}
  Seien $f : \R^n \to \R$ und $F = (F_1, ..., F_n)^T : \R^n \to \R^n$ Funktionen. Dann heißt
  \begin{itemize}
    \item $\dive F \coloneqq \sum_{i=1}^n D_i F_i : \R^n \to \R$ \emph{Divergenz} von $F$,
    \item $\grad f \coloneqq \nabla f \coloneqq (\partial_1 f, ..., \partial_n f)^T : \R^n \to \R^n$ \emph{Gradient} von $f$,
    \item $\Delta$ mit $\Delta f = \dive (\grad f) = \sum_{i=1}^n D_i D_i f$ \emph{Laplace-Operator}.
  \end{itemize}
\end{nota}

\begin{nota}
  Für $\Omega \subset \R^n$ schreibe
  \[
    V \Subset \Omega
    \quad \text{für} \quad
    \text{$V \subset \R^n$ mit $\overline{V}$ kompakt und $\overline{V} \subset \Omega^{\circ} $.}
  \]
\end{nota}

\begin{defn}
  Die \emph{Laplace-} bzw. \emph{Poisson-Gleichung} ist die Gleichung
  \[ \Delta u = 0 \quad \text{bzw.} \quad \Delta u = f \qquad \text{auf $\Omega \subset \R^n$}. \]
\end{defn}

% Kapitel 2.1. Vorbereitung

\begin{satz}[Transformationssatz]
  Sei $T : \Omega \to T(\Omega)$ für $\Omega \subset \R^n$ ein $\mathcal{C}^1$-Diffeo, dann gilt für $f : T(\Omega) \to \ER$
  \[ f \in L^1(T(\Omega)) \iff (f \circ T) \circ \abs{\det(DT)} \in L^1(\Omega) \quad \text{mit} \]
  \[ \Int{T(\Omega)}{}{f}{x} = \Int{\Omega}{}{(f \circ T) \cdot \abs{\det(DT)}}{x}. \]
\end{satz}

\begin{bsp}[Polarkoordinaten]
  Sei $f \in L^1(B_r(K))$. Dann ist $f$ auf fast jeder Sphäre $\partial B_\rho(K)$ für $\rho \in \cinterval{0}{r}$ integrierbar und es gilt
  \[ \Int{B_r(x)}{}{f(x)}{x} = \Int{0}{r}{\Int{\partial B_\rho(x_0)}{}{\!\!\!f}{S}}{\rho} \] % dH^{n-1}
\end{bsp}

\begin{satz}[Gauß]
  Sei $\Omega \subset \R^n$ beschränkt, offen mit $\mathcal{C}^1$-Rand $\partial \Omega$. Ist $F \in \mathcal{C}^0(\overline{\Omega}, \R^n) \cap \mathcal{C}^1(\Omega, \R^n)$ mit $\dive F \in L^1(\Omega)$, so gilt
  \[ \Int{\Omega}{}{\dive F}{x} = \Int{\partial \Omega}{}{(F \circ \nu)}{S}, \]
  wobei $\nu$ der äußere Einheitsnormalenvektor ist.
\end{satz}

% 1. Übungsblatt, 1. Aufgabe
\begin{kor}
  Sei $\Omega \subset \R^n$ beschränkt, offen mit $\mathcal{C}^1$-Rand $\partial \Omega$. Sind $f, g \in \mathcal{C}^1(\overline{\Omega})$, dann gilt die partielle Integrationsregel
  \[ \Int{\Omega}{}{D_i f g}{x} = - \Int{\Omega}{}{f D_i g}{x} + \Int{\partial \Omega}{}{f g \nu^i}{\HM^{n-1}} \]
  Sind $f, g \in \mathcal{C}^2(\overline{\Omega})$, dann gelten die Greenschen Formeln
  \begin{align*}
    \Int{\Omega}{}{Df \cdot Dg}{x} &= - \Int{\Omega}{}{f \Delta g}{x} + \Int{\Omega}{}{f D_{\nu} g}{\HM^{n-1}}\\
    \Int{\Omega}{}{(f \Delta g - g \Delta f)}{x} &= \Int{\partial \Omega}{}{(f D_{\nu} g - g D_{\nu} f)}{\HM^{n-1}}
  \end{align*}
\end{kor}

% 1. Übungsblatt, 2. Aufgabe: Differentiation von parameterabhängigen Integralen
\begin{prop}
  Sei $\Omega \subset \R^n$ messbar mit $\abs{\Omega} < \infty$, $I = \ointerval{a}{b} \subset \R$ und $f : \Omega \times I \to \R$. Angenommen,
  \begin{itemize}
    \item $f(x, \blank) \in \mathcal{C}^1(I)$ für fast alle $x \in \Omega$,
    \item $f(\blank, t) \in L^1(\Omega) \tfrac{\partial f}{\partial t}(\blank, t) \in L^1(\Omega)$ für alle $t \in I$ und
    \item für alle $t \in I$ gibt es $\epsilon > 0$ sodass $\ointerval{t-\epsilon}{t+\epsilon} \subset I$ und
    \[ \sup_{s \in \ointerval{t-\epsilon}{t+\epsilon}} \abs{\tfrac{\partial f}{\partial t}(\blank, s)} \in L^1(\Omega). \]
  \end{itemize}
  Dann ist die Abbildung
  \[
    g : I \to \R, \qquad
    t \mapsto \Int{\Omega}{}{f(x, t)}{x}
  \]
  wohldefiniert und stetig differenzierbar mit
  \[ \tfrac{\partial g}{\partial t}(t) = \Int{\Omega}{}{\tfrac{\partial f}{\partial t}(x,t)}{x}. \]
\end{prop}

\begin{bem}
  Die Voraussetzungen sind erfüllt, wenn $\Omega$ offen und beschränkt ist, $f(x, \blank) \in \mathcal{C}^1(I)$ für alle $x \in \Omega$ und $f, \tfrac{\partial f}{\partial t} \in \mathcal{C}(\overline{\Omega} \times I)$.
\end{bem}

\begin{nota}
  Bezeichne mit $\mathcal{L}^n$ das Lebesgue-Maß auf dem $\R^n$. Für messbare Teilmengen $A \subset \R^n$ schreibe $\abs{A} \coloneqq \mathcal{L}^n(A)$.
\end{nota}

\begin{bsp}
  Zwischen dem Volumen von Kugeln und Sphären im $\R^n$ bestehen folgende Zusammenhänge:
  \[
    \abs{B_r(0)} = r^n \cdot \abs{B_1(0)}
    \quad \text{und} \quad
    \abs{B_r(0)} = \tfrac{r}{n} \cdot \enspace\Int{\mathclap{\partial B_r(0)}}{}{\enspace1}{S}
  \]
\end{bsp}
\vspace{-12pt}
\begin{nota}
  $\omega_n \coloneqq \mathcal{L}^n(B_1(0)) = \frac{\pi^{\tfrac{n}{2}}}{\Gamma(\tfrac{n}{2} + 1)}$
\end{nota}

\begin{nota}
  Sei $f : \Omega / M \to \R$ integrierbar für $\Omega \subset \R^n$ messbar mit $\mathcal{L}^k(\Omega) \in \ointerval{0}{\infty}$ bzw. $M \subset \R^n$ eine $k$-dimensionale Untermannigfaltigkeit mit $\Int{M}{}{1}{S} \in \ointerval{0}{\infty}$
  \[
    \MVInt{\Omega}{}{f(x)}{x} \coloneqq \tfrac{1}{\abs{\Omega}} \Int{\Omega}{}{f(x)}{x}
    \quad \text{bzw.} \quad
    \MVInt{M}{}{f(x)}{x} \coloneqq \tfrac{1}{\abs{M}} \Int{M}{}{f(x)}{x}
  \]
  heißen \emph{Mittelwerte} von $f$ auf $\Omega$ bzw. $M$.
\end{nota}

\begin{defn}
  Ein \emph{Glättungskern} auf $\R^n$ ist eine nicht-negative, radialsym- metrische Funktion $\eta \in \mathcal{C}_0^\infty(B_1(0))$ mit $\Int{\R^n}{}{\eta}{x} = 1$.
\end{defn}

\begin{defn}
  Der \emph{Standardglättungskern} ist die Funktion
  \[ \eta(x) \coloneqq C \cdot \exp \left(\tfrac{1}{\abs{x}^2 - 1}\right) \cdot \ind_{B_1(0)}(x) \]
  mit Normierungskonstante $C$. Für $\epsilon > 0$ ist der dazugehörige skalierte Glättungskern gegeben durch
  \[ \eta_{\epsilon}(x) \coloneqq \epsilon^{-n} \eta(x/\eta). \]
  Alle Glättungskern-Eigenschaften bleiben bei Skalierung erhalten.
\end{defn}

\begin{nota}
  $\Omega_{\epsilon} \coloneqq \Set{ x \in \Omega }{ \dist(x, \partial \Omega) > \epsilon }$
\end{nota}

\begin{defn}
  Sei $\Omega \subset \R^n$ offen, $\epsilon > 0$. Für $f \in L_{\text{loc}}^1$ heißt die Funktion
  \[
    f_\epsilon : \Omega_{\epsilon} \to \R, \quad
    x \mapsto \eta_\epsilon * f(x) \coloneqq \Int{\mathclap{B_{\epsilon}(x)}}{}{\enspace\eta_{\epsilon}(x-y) f(y)}{y}
    \quad \text{\emph{$\epsilon$-Glättung} von $f$}
  \]
\end{defn}

\begin{satz}[Eigenschaften von Glättungen]
  Sei $\Omega \subset \R^n$ offen, $\epsilon > 0$ und $f \in L_{\text{loc}}^1(\Omega)$. Dann gilt
  \begin{itemize}
    \item Regularität: $f_{\epsilon} \in \mathcal{C}^\infty(\Omega_\epsilon)$ mit $D^{\alpha} f_\epsilon = (D^\alpha \eta_\epsilon) * f$ für beliebige Multiindizes $\alpha \in \N^n$.
    \item Ist $D_i f$ stetig auf $\Omega$, so gilt $D_i (f_\epsilon) = (D_i f)_\epsilon$ auf $\Omega_\epsilon$. % Vertauschbarkeit mit Ableitungen
    % Nächsten zwei Punkte: Erhaltung von Normen
    \item Falls $f \in \mathcal{C}^\alpha(\Omega)$ für ein $\alpha \in \ocinterval{0}{1}$, so gilt $f_\epsilon \in \mathcal{C}^\alpha(\Omega_\epsilon)$ mit derselben Hölderkonstante.
    \item Falls $f \in L^p(\Omega)$ für $p \in \cinterval{0}{\infty}$, so gilt $\norm{f_\epsilon}_{L^p(\Omega_\epsilon)} \leq \norm{f}_{L^p(\Omega)}$.
    % Nächste drei Punkte: Approximation
    \item $f_\epsilon \xrightarrow{\epsilon \to 0} f$ fast-überall in $\Omega$.
    \item Falls $f \in \mathcal{C}(\Omega)$, so konvergiert $f_\epsilon$ gleichmäßig gegen $f$ für $\epsilon \to 0$ auf kompakten Teilmengen von $\Omega$,
    \item Falls $f \in L_{\text{loc}}^p(\Omega)$ für $p \in \cointerval{1}{\infty}$, so gilt $f_\epsilon \xrightarrow{\epsilon \to 0} f$ in $L_{\text{loc}}^p(\Omega)$.
    \item Abschätzung der Approximationsgüte: Ist $Du \in L^p(\Omega)$, so gilt
    \[ \norm{f - f_\epsilon}_{L^p(\Omega_\epsilon)} \leq \epsilon \cdot \norm{Df}_{L^p{\Omega}}. \]
  \end{itemize}
\end{satz}

% Kapitel 2.2. Harmonische Funktionen

\iffalse
  Physikalische Motivation/Herleitung
  % siehe auch: Skript Schmidt 

  Beschreibung von Gleichgewichtszuständen von physikalischen Zuständen $u$ (Temperatur / Konzentration)

  Gleichgewicht: "`Nettofluss"' durch den Rand $V \subset \Omega$:
  Jedes (glatt berandete) Testvolumen verschwindet.

  $F \coloneqq Flussdichte von $u

  $0 = \Int{\partial V}{}{F \circ \nu}{S} \overset{\text{Gauss}}{=} \Int{V}{}{\dive F}{x}$

  Es folgt: $\dive F = 0$, da $V$ beliebig

  Typischerweise $F = -a Du$ ($a > 0$).

  Es folgt: $-a \Delta u = 0$

  In der Physik: Diffusion von Feldern, 1. Ficksche Gesetz, Wärmeleitung, Fouriersches Gesetz

  Definition von harmonischen Funktionen und der Fundamentallösung
\fi

\begin{defn}
  Sei $\Omega \subset \R^n$ offen, $u \in \mathcal{C}^2(\Omega)$. Man nennt $u$
  \begin{itemize}
    \item \emph{harmonisch}, falls $\Delta u = 0$ in $\Omega$ gilt.
    \item \emph{subharmonisch}, falls $\Delta u \geq 0$ in $\Omega$ gilt.
    \item \emph{superharmonisch}, falls $\Delta u \leq 0$ in $\Omega$ gilt.
  \end{itemize}
\end{defn}

\begin{bspe}
  \begin{itemize}
    \item Affine Funktionen sind harmonisch.
    \item Sei $A \in \R^{n \times n}$. Definiere $u(x) \coloneqq x \cdot Ax$. Dann gilt $\Delta u = \spur A$, also $\Delta u = 0 \iff \spur A = 0$.
    % Ausgelassen: Harmonische Funktionen aus Ansatz mit Trennung der Variablen
    \item Real- und Imaginärteil von holomorphen Fktn. sind harmonisch.
  \end{itemize}
\end{bspe}

\iffalse
  % Herleitung der Fundamentallösung: siehe Evans, PDE, Abschnitt 2.2.1 a) oder Schmidt, Abschnitt 2.1.3
  Konstruktion von rotationssymmetrischen harmonischen Funktionen, d.\,h.
  \[ \Delta u = 0 \]
  mit $u(x) = v(r)$ mit $v : \R \to \R$ und $r = \norm{x} = (x_1^2 + ... + x_n^2)^{1/2}$.
  Beachte (für $i \in \{ 1, ..., n \}, x \not= 0$):
  \begin{itemize}
    \item $D_i r = \frac{x_i}{(x_1^2 + ... + x_n^2)^{1/2}}$, also $\abs{Dr}^2 = \sum_{i=1}^n (D_i r)^2 = 1$
    \item $D_i D_i r = D_i (D_i r) = \frac{1}{r} - \frac{x_i x_i}{r^3}$
    \item $\Delta r = \sum_{i=1}^n (\frac{1}{r} - \frac{x_i^2}{r^3}) = \frac{n-1}{r}$
    \item $\Delta u = \sum_{i=1}^n D_i (v^i(r) D_i r) = v''(r) \sum_{i=1}^n (D_i r)^2 + v'(r) \sum_{i=1}^n D_i D_i r = v''(r) + \frac{n-1}{r} v'(r) = 0$
  \end{itemize}

  Für $v' \not= 0$ kann man diese ODE explizit lösen

  \[ (\log \abs{v'(r)})' = \frac{v''(r)}{v'(r)} = \frac{1-n}{r} = (1-n)(\log r)' = (\log r^{1-n})' \]

  Also (Integrieren, Exponentialfunktion anwenden): $v'(r) = c r^{1-n}$ für $c \in \R$.
  Somit $v(r) = c_1 \cdot \log r + c_2$, wenn $n = 2$
  Somit $v(r) = c_1 \cdot r^{2-n} + c_2$, wenn $n \geq 3$
\fi

\begin{defn}
  Die Funktion $\Phi : \R^n \setminus \{ 0 \} \to \R$, definiert durch
  \[
    \Phi(x) \coloneqq \begin{cases}
      - (2\pi)^{-1} \log \abs{x}, & \text{wenn $n = 2$}\\
      (n (n{-}2) \, \omega_n)^{-1} \abs{x}^{2-n}, & \text{wenn $n \geq 3$}
    \end{cases}
  \]
  heißt \emph{Fundamentallösung} der Laplacegleichung.
\end{defn}

\begin{bem}
  \begin{itemize}
    \item $\Phi$ ist radialsymmetrisch, d.\,h. für alle $x_1, x_2 \in \R^n \setminus \{ 0 \}$ mit $\norm{x_1} = \norm{x_2}$ gilt $\Phi(x_1) = \Phi(x_2)$.
    \item $\Phi$, $\abs{D\Phi} \in L^1(B_R(0))$ für alle $R > 0$ aber $\abs{D^2 \phi} \not\in L^1(B_1(0))$.
    \item Die Konstanten wurden so gewählt, dass gilt:
    \[ - \Int{\mathclap{\partial B_r(0)}}{}{D \Phi \cdot \nu}{\HM^{n-1}} = 1 \quad \text{für alle $r > 0$}. \]
  \end{itemize}
\end{bem}

% 2.1. (Schmidt: Lemma 2.3)
\begin{lem}
  Sei $\Omega \subset \R^n$ offen, $B_R(x_0) \subset \Omega$, $u \in \mathcal{C}^2(\Omega)$. Für
  \[
    \phi : \ointerval{0}{R} \to \R, \quad
    r \mapsto \enspace \MVInt{\mathclap{\partial B_r(x_0)}}{}{\enspace u}{\HM^{n-1}}
    \qquad \text{gilt dann}
  \]
  \begin{itemize}
    \miniitem{0.40 \linewidth}{$\lim_{r \to 0} \phi(r) = u(x_0)$}
    \miniitem{0.48 \linewidth}{$\phi'(r) = \tfrac{r}{n} \enspace\MVInt{\mathclap{B_r(x_0)}}{}{\enspace\Delta u(x)}{x}$}
  \end{itemize}
\end{lem}

% Vorlesung vom 15.4.2014

% 2.2.
\begin{kor}[Mittelwertseigenschaft]
  Sei $\Omega \subset \R^n$ offen, $B_r(x_0) \Subset \Omega$ und $u \in \mathcal{C}^2(\Omega)$. Dann gilt:
  \[
    0 = \Delta u \enspace\implies\enspace
    u(x_0) = \enspace\MVInt{\mathclap{\partial B_r(x_0)}}{}{\enspace u}{\HM^{n-1}}
    \quad \text{und} \quad
    u(x_0) = \enspace\MVInt{\mathclap{B_r(x_0)}}{}{\enspace u}{\HM^{n-1}}
  \]
  In diesen Gleichungen darf man $=$ durch $\leq$, $<$, $\geq$ oder $>$ ersetzen.
\end{kor}

% 2.3.
\begin{satz}
  Sei $\Omega \subset \R^n$ offen. Dann sind äquivalent:
  \begin{itemize}
    \item $u$ ist harmonisch, d.\,h. es gilt $\Delta u = 0$ in $\Omega$.
    \item $u$ erfüllt die sphärische Mittelwertseigenschaft, d.\,h. es gilt
    \[
      u(x_0) = \enspace\MVInt{\mathclap{\partial B_r(x_0)}}{}{\enspace u}{\HM^{n-1}}
      \qquad \text{für alle Kugeln $B_r(x_0) \Subset \Omega$.}
    \]
    \item $u$ erfüllt die Mittelwertseigenschaft auf Kugeln, d.\,h. es gilt
    \[
      u(x_0) = \enspace\MVInt{\mathclap{B_r(x_0)}}{}{\enspace u}{\HM^{n-1}}
      \qquad \text{für alle Kugeln $B_r(x_0) \Subset \Omega$.}
    \]
  \end{itemize}
\end{satz}

\begin{bem}
  Die Äquivalenz gilt auch unter den schwächeren Voraussetzungen $u \in \mathcal{C}(\Omega)$ oder $u \in L^1(\Omega)$.
\end{bem}

% 2.4.
\begin{satz}
  Sei $\Omega \subset \R^n$ offen, beschränkt und $u \in \mathcal{C}^2(\Omega) \cap \mathcal{C}^0(\overline{\Omega})$ subharmonisch in $\Omega$, d.\,h. $\Delta u \geq 0$ in $\Omega$. Dann gilt
  \begin{itemize}
    \item Das \emph{schwache Maximumsprinzip}: $\max_{\overline{\Omega}} u = \max_{\partial \Omega} u$
    \item Das \emph{starke Maximumsprinzip}: Ist $\Omega$ zusammenhängend und existiert $x_0 \in \Omega$ mit $u(x_0) = \max_{\overline{\Omega}} u$, so ist $u$ konstant.
  \end{itemize}
  % Entsprechende Maximumsprinzipien für superharmonische Funktionen
\end{satz}

% Ausgelassen: Definition "`zusammenhängend"'

\begin{bem}
  Sei $\Omega \subset \R^n$ beschränkt, offen, zusammenhängend und $u \in \mathcal{C}^2(\Omega) \cap \mathcal{C}(\overline{\Omega})$ harmonisch. Dann gilt
  \[ \min_{\partial \Omega} u < \max_{\partial \Omega} u \enspace\implies\enspace \min_{\partial \Omega} u < u < \max_{\partial \Omega} u \text{ auf $\Omega$}. \]
\end{bem}

\begin{kor}[Eindeutigkeit]
  Sei $\Omega \subset \R^n$ offen, beschränkt und $u, v \in \mathcal{C}^2(\Omega) \cap \mathcal{C}(\overline{\Omega})$. Dann ist $u = v$, falls gilt:
  \[
    \left\{ \begin{array}{ll}
      \Delta u = \Delta v & \text{in $\Omega$}\\
      u = v & \text{auf $\partial \Omega$}
    \end{array} \right.
  \]
\end{kor}

\begin{bem}[Stetige Abhängigkeit von Randwerten]
  Gilt lediglich $\Delta u = \Delta v$ in $\Omega$, aber nicht $u = v$ auf $\partial \Omega$, so gilt immerhin
  \[ \max_{\overline{\Omega}} \abs{u - v} = \max_{\partial \Omega} \abs{u - v}. \]
\end{bem}

% Ausgelassen: Bemerkung zur stetigen Abhängigkeit von den Randwerten

% 2.6.
\begin{satz}[Harnack-Ungleichung]
  Sei $\Omega \subset \R^n$ offen, $V \Subset \Omega$ offen, zusammenhängend. Dann gibt es eine Konstante $c = c(\Omega, V)$, sodass
  \[
    \sup_{V} u \leq c \cdot \inf_{V} u
    \qquad \text{für alle harmonischen Fktn. $u : \Omega \to \R_{\geq 0}$.}
  \]
  % Insbesondere sind alle Funktionswerte vergleichbar.
\end{satz}

% Vorlesung vom 24.4.2014

% Thema: Regularität

% 2.8.
\begin{satz}
  Sei $\Omega \subset \R^n$ offen und erfülle $u \in \mathcal{C}(\Omega)$ die Mittelwert- Eigenschaft auf Sphären, d.\,h.
  \[
    u(x_0) = \enspace\MVInt{\mathclap{\partial B_r(x_0)}}{}{\enspace u}{\HM^{n-1}}
    \quad \text{für alle Kugeln $B_r(x_0) \Subset \Omega$.}
  \]
  Dann gilt $u(x) = u_\epsilon(x)$ für alle $x \in \Omega$ und $\epsilon < \dist(x, \partial \Omega)$.\\
  Insbesondere ist $u \in \mathcal{C}^\infty(\Omega)$ und harmonisch.
\end{satz}
% Achtung: Keine Aussage des Satzes über Stetigkeit der Randwerte!

\begin{kor}
  Obiger Satz gilt auch, wenn $u \in \mathcal{C}(\Omega)$ die Mittelwert-Eigenschaft auf Kugeln erfüllt, d.\,h.
  \[
    u(x_0) = \enspace\MVInt{\mathclap{B_r(x_0)}}{}{\enspace u}{\HM^{n-1}}
    \quad \text{für alle Kugeln $B_r(x_0) \Subset \Omega$.}
  \]
\end{kor}

\begin{defn}
  Eine Folge von Funktionen $(f_n)_{n \in \N}$ auf einem topologischen Raum $X$ \emph{konvergiert lokal gleichmäßig} gegen $f : X \to \R$, falls es zu jedem Punkt $x \in X$ eine Umgebung $U_x$ von $x$ gibt, sodass $f_n$ auf $U_x$ gleichmäßig gegen $f$ konvergiert.
\end{defn}

\begin{kor}[Konvergenzsatz von Weierstraß]
  Sei $\Omega \subset \R^n$ offen, zusammenhängend und $(u_k)_{k \in \N}$ eine Folge harmonischer Funktionen auf $\Omega$, die lokal gleichmäßig gegen eine Funktion $u$ konvergiert.\\
  Dann ist $u$ harmonisch auf $\Omega$.
\end{kor}

\begin{kor}[Harnackscher Konvergenzsatz]
  Sei $\Omega \subset \R^n$ offen und $(u_k)_{k \in \N}$ eine monoton wachsende Folge harmonischer Funktionen auf $\Omega$. Gibt es ein $x_0 \in \Omega$, sodass $(u_k(x_0))_{k \in \N}$ beschränkt (und damit konvergent) ist, so konvergiert $(u_k)$ lokal gleichmäßig gegen eine harmonische Funktion auf $\Omega$.
\end{kor}

\begin{satz}[von Hermann Weyl]
  Sei $\Omega \subset \R^n$ offen und $u \in L_{\text{loc}}^1(\Omega)$ mit
  \[
    \Int{\Omega}{}{u \cdot \Delta \phi}{x} = 0
    \quad \text{für alle $\phi \in \mathcal{C}_0^\infty(\Omega)$.}
  \]
  Dann gibt es eine harmonische Funktion $\tilde{u} : \Omega \to \R$ mit $u(x) = \tilde{u}(x)$ für fast alle $x \in \Omega$.
\end{satz}

\begin{satz}[Innere Abschätzung für Ableitungen harmonischer Fktn]\mbox{}\\
  Sei $\Omega \subset \R^n$ offen und $u \in \mathcal{C}^2(\Omega)$ harmonisch. Dann gilt für jeden Multiindex $\alpha$ mit $\abs{\alpha} = k \in \N_0$ und jede Kugel $B_r(x_0) \Subset \Omega$:
  \[
    \abs{D^\alpha u(x_0)} \leq C(n, k) r^{-n-k} \norm{u}_{L^1(B_r(x_0))}
    \quad \text{mit } C(n, k) \coloneqq \tfrac{(2^{n+1} nk)^k}{\omega_n}.
  \]
\end{satz}

\begin{satz}[Liouville]
  Sei $u \in \mathcal{C}^2(\Omega)$ harmonisch.
  \begin{itemize}
    \item Ist $u$ beschränkt, so ist $u$ konstant.
    \item Gilt $\limsup_{\abs{x} \to \infty} \tfrac{\abs{u(x)}}{\abs{x}^{k+1}} = 0$, so ist $u$ ein Polynom, dessen Grad $\leq k$ ist.
  \end{itemize}
\end{satz}

\begin{defn}
  Sei $\Omega \subset \R^n$ offen. Eine Funktion $f : \Omega \to \R$ heißt \emph{analytisch} in $x \in \Omega$, falls $f$ sich lokal durch ihre Taylorreihe darstellen lässt, also ein $r \in \ointerval{0}{\dist(x, \partial \Omega)}$ existiert mit
  \[
    f(y) = \sum_{\alpha \in \N^n} \tfrac{1}{\alpha!} D^\alpha f(x) (y-x)^\alpha
    \qquad \text{für alle $y \in B_r(x)$.}
  \]
\end{defn}

\begin{satz}
  Sei $\Omega \subset \R^n$ offen und $u \in \mathcal{C}^2(\Omega)$. Wenn $u$ harmonisch ist, dann auch analytisch.
\end{satz}

% Vorlesung vom 29.4.2014

\begin{prob}
  Sei $\Omega \subset \R^n$ offen, (beschränkt), regulär und $f : \Omega \to \R$ und $g : \partial \Omega \to \R$ stetig. Gesucht ist $u : \overline{\Omega} \to \R$ mit
  \[
    (2.1) \left\{ \begin{array}{lll}
      - \Delta u &= f &\quad \text{in $\Omega \subset \R^n$}\\
      u &= g &\quad \text{auf $\partial \Omega$}
    \end{array} \right.
  \]
\end{prob}

% Bekannt: Lösungen dazu sind eindeutig

% 2.15.
\begin{satz}[Greensche Darstellungsformel]
  Sei $\Omega \subset \R^n$ beschränkt, offen mit $\mathcal{C}^1$-Rand und $h \in \mathcal{C}^2(\Omega) \cap \mathcal{C}^1(\overline{\Omega})$ mit $\Delta h {\in} L^1(\Omega)$. Es gilt für $x {\in} \Omega$:
  \begin{align*}
    h(x) = - \Int{\Omega}{}{\Phi(x-y) \Delta h(y)}{y} &+ \Int{\mathclap{\partial \Omega}}{}{\Phi(x {-} y) \cdot Dh(y) \cdot \nu}{S(y)}\\
    &- \Int{\mathclap{\partial \Omega}}{}{h(y) D_y \Phi(x{-}y) \cdot \nu}{S(y)}
  \end{align*}
\end{satz}

\begin{bem}
  Für Randpunkte $x \in \partial \Omega$ gilt:
  \begin{align*}
    \tfrac{1}{2} h(x) = - \Int{\Omega}{}{\Phi(x-y) \Delta h(y)}{y} &+ \Int{\mathclap{\partial \Omega}}{}{\Phi(x {-} y) \cdot Dh(y) \cdot \nu}{S(y)}\\
    &- \Int{\mathclap{\partial \Omega}}{}{h(y) D_y \Phi(x{-}y) \cdot \nu}{S(y)}
  \end{align*}
\end{bem}

% Ausgelassen: Physikalische Interpretation, Merkregel

\begin{kor}[Darstellungsformel für Lsgn in $\R^n$]
  Sei $f {\in} \mathcal{C}_0^2(\R^n)$, setze
  \[ u : \R^n \to \R, \qquad x \mapsto (\Phi * f)(x) \coloneqq \Int{\R^n}{}{\Phi(x-y) f(y)}{y}. \]
  Dann gilt: $u \in \mathcal{C}^2(\R^n)$ und $- \Delta u = f$ in $\R^n$.
\end{kor}

\begin{bem}
  \begin{itemize}
    \item Für $n = 2$ ist die Lösung potentiell unbeschränkt.
    \item Für $n \geq 3$ ist diese Lsg beschränkt und erfüllt $\enspace\lim_{\mathclap{\abs{x} \to \infty}} u(x) = 0$.
  \end{itemize}
\end{bem}

\begin{prop}
  Jede andere beschränkte Lösung von $- \Delta u = f$ auf $\R^n$ unterscheidet sich nur durch eine additive Konstante.
\end{prop}

\begin{defn}
  Sei $\Omega \subset \R^n$ offen. Eine \emph{Greensche Funktion} für $\Omega$ ist eine Funktion
  $G : \Set{ (x, y) \in \Omega \times \Omega }{ x \not= y } \to \R$, falls für alle $x \in \Omega$ gilt:
  \begin{itemize}
    \item Die \emph{Korrektorfunktion} $y \mapsto G(x, y) - \Phi(x - y)$ ist von der Klasse $\mathcal{C}^2(\Omega) \cap \mathcal{C}^1(\overline{\Omega})$ und ist harmonisch in $\Omega$. % (für $x=y$ forsetzbar)
    \item Die Funktion $G(x, \blank)$ hat Nullrandwerte auf $\partial \Omega$, d.\,h. es gilt $\lim_{y \to y_0} G(x, y) = 0$ für alle $y_0 \in \partial \Omega \cup \{ \infty \}$.
  \end{itemize}
\end{defn}

% Ausgelassen: Bemerkung zum Neumann-RWP und der Greenschen Funktion zweiter Art (mit verschwindender Normalenableitung)

\begin{bem}
  Die Funktion $G(x, \blank)$ ist in $\mathcal{C}^2(\Omega \setminus \{ x \}) \cap \mathcal{C}(\overline{\Omega} \setminus \{ x \})$ und hat die gleiche Singularität wie $y \mapsto \Phi(x - y)$.
\end{bem}

% TODO: 2. Bemerkung zur Korrektorfunktion verstehen und ergänzen

\begin{satz}
  Sei $\Omega \subset \R^n$ offen, beschränkt, mit $\mathcal{C}^1$-Rand $\partial \Omega$. Ist $u \in \mathcal{C}^2(\Omega) \cap \mathcal{C}^1(\overline{\Omega})$ eine Lösung von (2.1) und ist $G$ die Greensche Funktion für $\Omega$ (falls existent), dann gilt für alle $x \in \Omega$:
  \[ u(x) = \Int{\Omega}{}{G(x,y) f(y)}{y} - \Int{\partial \Omega}{}{g(y) \cdot D_y G(x,y) \cdot \nu}{S(y)}. \]
\end{satz}

% Vorlesung vom 6.5.2014

% 2.18
\begin{lem}
  Sei $\Omega \subset \R^n$ offen, $G$ die Greensche Funktion für $\Omega$ und $B_r(x) \Subset \Omega$. Für $f \in \mathcal{C}^2(\Omega) \cap \mathcal{C}^1(\overline{\Omega})$ gilt dann:
  \[ \lim_{\epsilon \to 0} \quad\enspace \Int{\mathclap{\partial B_{\epsilon}(x)}}{}{\left( G(x, y) Df(y) - f(y) D_y G(x, y) \right) \cdot \nu(y) }{\HM^{n-1}(y)} = f(x). \]
\end{lem}

% 2.19
\begin{satz} % Symmetrie der Greenschen Funktion
  Ist $G$ die Greensche Funktion zu $\Omega \subset \R^n$ offen, beschränkt mit $\mathcal{C}^1$-Rand $\partial \Omega$, so gilt $G(x, y) = G(y, x)$ für alle $x, y \in \Omega$ mit $x \not= y$.
\end{satz}

\begin{kor}
  Sei $G$ die Greensche Funktion zu $\Omega \subset \R^n$ offen, beschränkt mit $\mathcal{C}^1$-Rand $\partial \Omega$, so ist die Funktion $x \mapsto G(x, y)$ harmonisch auf $\Omega \setminus \{ y \}$.
\end{kor}

\begin{defn}
  Sei $B_r(a) \subset \R^n$ eine Kugel, $x \in \R^n \setminus \partial B_r(a)$. Dann heißt
  \[
    x^* \coloneqq a + r^2 \frac{x - a}{\norm{x - a}^2} \in \R^n \setminus \partial B_r(a) \quad
    \text{\emph{Spiegelungspunkt} von $x$.}
  \]
\end{defn}

\begin{bem}
  \begin{minipage}{0.12 \linewidth}Es gilt:\end{minipage}
  \begin{minipage}{0.63 \linewidth}
    \begin{itemize}
      \miniitem{0.65 \linewidth}{$\norm{x - a} \cdot \norm{x^* - a} = r^2$}
      \miniitem{0.3 \linewidth}{$(x^*)^* = x$}
    \end{itemize}
  \end{minipage}
  \begin{itemize}
    \item $\fa{y \in \partial B_r(a)} \norm{x^* - y}^2 = r^2 \norm{x - a}^{-2} \norm{y - x}^2$.
  \end{itemize}
\end{bem}

\begin{nota}
  %Sei $g : B_r(a) \times B_r(a) \setminus \Set{ (x, x) }{ x \in B_r(a) }$ def. durch
  Für $B_r(a) \subset \R$ sei $g : B_r(a) \times B_r(a) \to \R$ definiert durch
  \[ g(x, y) \coloneqq \begin{cases}
      - \Phi\left( \tfrac{\abs{x-a}}{r} (y - x^*) \right) & \text{für $x \in B_r(a) \setminus \{ a \}$}\\
      - \Phi(r e_1) & \text{für $x = a$.}
    \end{cases}
  \]
\end{nota}

\begin{prop}
  Für die Funktion $g$ gilt:
  \begin{itemize}
    \item $g(x, y) - \Phi(x - y) = 0$ für alle $y \in \partial B_r(a)$ und $x \in B_r(a)$.
    \item $y \mapsto g(x, y)$ ist glatt und harmonisch in $B_r(a)$ für alle $x \in B_r(a)$.
  \end{itemize}
\end{prop}

\begin{kor}
  Die Greensche Funktion für $B_r(a) \subset \R^n$ lautet
  \begin{align*}
    G_{B_r(a)}(x, y) &\coloneqq \Phi(x - y) + g(x, y)\\
    &= \begin{cases}
      \Phi(x - y) - \Phi\left(\tfrac{\abs{x-a}}{r} (y - x^*)\right) & \text{für $x \in B_r(a) \setminus \{ a \}$}\\
      \Phi(a - y) - \Phi(r e_1) & \text{für $x = a$.}
    \end{cases}
  \end{align*}
\end{kor}

\begin{defn}
  Der \emph{Poisson-Kern für die Kugel} $B_r(a) \subset \R^n$ ist
  \[ K_{B_r(a)}(x, y) \coloneqq \tfrac{1}{n \omega_n} \cdot \tfrac{r^2 - \abs{x - a}^2}{r \abs{x - y}^n}. \]
\end{defn}

\begin{satz}[Poisson-Integralformel für Kugeln]
  Sei $B_r(a) \subset \R^n$ und $g : \partial B_r(a)$ stetig.
  \begin{itemize}
    \item Für $u {\in} \mathcal{C}^2(B_r(a)) \cap \mathcal{C}(\overline{B_r(a)})$ harmonisch mit $u {=} g$ auf $\partial B_r(a)$ gilt
    \[
      u(x) = \Int{\mathclap{\partial B_r(a)}}{}{K_{B_r(a)}(x,y) g(y)}{\HM^{n-1}(y)}
      \quad \text{für alle $x \in B_r(a)$.} \tag{2.8}
    \]
    \item Umgekehrt definiert (2.8) die eindeutige harmonische Funktion $u \in \mathcal{C}^2(B_r(a)) \cap \mathcal{C}^1(\overline{B_r(a)})$ mit $u = g$ auf $\partial B_r(a)$.
  \end{itemize}
\end{satz}

% Ausgelassen: Bemerkung
% * Punkt 1 ist für $a=x$ die MW-Eigenschaft undermöglicht Herleitung einer speziellen Harnack-Ungleichung
% * Punkt 2 ist Satz mit expliziter Lösungsformel und wird als Satz von Schwarz bezeichnet.

\end{document}