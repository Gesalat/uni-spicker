\documentclass{cheat-sheet}

\pdfinfo{
  /Title (Zusammenfassung Homologische Algebra)
  /Author (Tim Baumann)
}

\usepackage{tikz}
%\usetikzlibrary{babel}
\usetikzlibrary{matrix,arrows,cd}
%\usepackage{tikz-cd}

% http://tex.stackexchange.com/questions/117732/tikz-and-babel-error
% Es ist schierer Wahnsinn, welche Hacks LaTeX benötigt!
\tikzset{
  every picture/.prefix style={
    execute at begin picture=\shorthandoff{"}
  }
}

\usepackage{pgffor} % \foreach-Schleifen
\usepackage{stmaryrd} % liefert \mapsfrom
\usepackage{stackrel}

% Kategorientheorie-Makros

% Konzepte
\DeclareMathOperator{\Ob}{Ob} % Objekte (einer Kategorie)
\DeclareMathOperator{\Mor}{Mor} % Morphismenmenge / -klasse
\DeclareMathOperator{\Hom}{Hom} % Homomorphisms
\DeclareMathOperator{\Nat}{Nat} % Natürliche Transformationen
\DeclareMathOperator{\dom}{dom} % Domain
\DeclareMathOperator{\codom}{codom} % Codomain
\newcommand{\op}{\mathrm{op}} % opposite category
\DeclareMathOperator{\Aut}{Aut} % Automorphismengruppe
\newcommand{\ladj}{\dashv} % Links-adjungiert (left-adjoint)
\newcommand{\Lim}{\lim} % Limes
\DeclareMathOperator{\colim}{colim} % Kolimes
\newcommand{\Colim}{\colim} % Kolimes
%\newcommand{\myint}[2]{{\textstyle \int\limits_{#1}^{#2}}}
\newcommand{\EndC}[2]{\myint{#1}{} #2} % Ende
\newcommand{\CoEndC}[2]{\myint{}{#1} #2} % Koende
\DeclareMathOperator{\Ran}{Ran} % Rechts-Kan-Erweiterung
\DeclareMathOperator{\Lan}{Lan} % Links-Kan-Erweiterung

% Platzhalter
\newcommand{\DiaTodo}{\fcolorbox{red}{white}{TODO: Diagramm einfügen!}}

% Konkrete Kategorien
\newcommand{\SetC}{\mathbf{Set}} % Kategorie der Mengen
\newcommand{\sSet}{\mathbf{sSet}} % Kategorie der simplizialen Mengen
\newcommand{\Top}{\mathbf{Top}} % Kategorie der topologischen Räume
\newcommand{\AbGrp}{\mathbf{Ab}} % Kategorie der abelschen Gruppen
\newcommand{\Grp}{\mathbf{Grp}} % Kategorie der Gruppen
\newcommand{\RMod}{\mathbf{R\text{-}Mod}} % Kategorie der R-Moduln
\newcommand{\Ouv}{\mathbf{Ouv}} % Kategorie der offenene Mengen eines topol. Raumes
\newcommand{\KHaus}{\mathbf{KHaus}} % Kategorie der kompakten Hausdorffräume
\newcommand{\CatC}{\mathbf{Cat}} % Kategorie der kleinen Kategorien
\newcommand{\Vect}{\mathbf{Vect}} % Kategorie der Vektorräume über einem Körper
\newcommand{\Alg}{\mathbf{Alg}} % Kategorie der Algebren über einem Körper/Ring
\newcommand{\VectFin}{\mathbf{Vect}_{\mathrm{fin}}} % Kategorie der endlichen Vektorräume über einem Körper
\newcommand{\kVect}{\text{$k$-$\Vect$}} % Kategorie der k-Vektorräume über einem Körper k
\newcommand{\kVectFin}{\text{$k$-$\VectFin$}} % Kategorie der endlichen k-Vektorräume über einem Körper k
\newcommand{\Mod}{\mathbf{Mod}} % Kategorie der Moduln über einem Ring
\newcommand{\Kom}{\mathbf{Kom}} % Kategorie der Komplexe in einer abelschen Kategorie
\newcommand{\Der}{\mathcal{D}} % abgeleitete Kategorie einer abelschen Kategorie
\newcommand{\kAlg}{k\text{-}\Alg} % Kategorie der k-Algebren

% Bezeichnungen für Variablen, die für Kategorien stehen
\newcommand{\Aat}{\mathcal{A}} % Category-A
\newcommand{\Bat}{\mathcal{B}} % Category-B
\newcommand{\Cat}{\mathcal{C}} % Category-C
\newcommand{\Dat}{\mathcal{D}} % Category-D
\newcommand{\Eat}{\mathcal{E}} % Category-E
\newcommand{\Fat}{\mathcal{F}} % Category-F
\newcommand{\Gat}{\mathcal{G}} % Category-G
\newcommand{\Iat}{\mathcal{I}} % Category-I (Indexkategorie)
\newcommand{\Jat}{\mathcal{J}} % Category-J (Indexkategorie)
\newcommand{\MatC}{\mathcal{M}} % Category-M
\newcommand{\Sit}{\mathcal{S}} % Situs-S
 % Kategorientheorie-Makros
% Garbentheorie-Makros

% Konzepte

% Konkrete Garben
\renewcommand{\O}{\mathcal{O}} % Strukturgarbe der stetigen Funktionen
\newcommand{\constSh}[1]{\underline{#1}} % konstante Garbe

% Kategorien von Garben
% Notation übernommen von http://stacks.math.columbia.edu/download/sheaves.pdf
\newcommand{\PShSet}{\mathbf{PSh}} % Prägarben von Mengen
\newcommand{\ShSet}{\mathbf{Sh}} % Garben von Mengen
\newcommand{\PShAb}{\mathbf{PAb}} % Prägarben von abelschen Gruppen
\newcommand{\ShAb}{\mathbf{Ab}} % Garben von abelschen Gruppen

% Bezeichnungen für Variablen, die für Garben stehen
\newcommand{\Fais}{\mathcal{F}} % Faisceau-F (Garbe auf französisch)
\newcommand{\Garb}{\mathcal{G}} % Garben-G
\newcommand{\Harb}{\mathcal{H}} % Garben-H
\newcommand{\Karb}{\mathcal{H}} % Garben-K (Kern)
\newcommand{\Carb}{\mathcal{H}} % Garben-C (Kokern)
\newcommand{\Iarb}{\mathcal{H}} % Garben-I (Image)
 % Garbentheorie-Makros

\newenvironment{centertikz}
  {\begin{center}\begin{tikzpicture}}
  {\end{tikzpicture}\end{center}}
\newenvironment{centertikzcd}
  {\begin{center}\begin{tikzcd}}
  {\end{tikzcd}\end{center}}

\newcommand{\nspace}[1]{\foreach \i in {1,...,#1}{ \! }} % Negativer Abstand
\DeclareMathOperator{\sk}{sk} % Skelett
\newcommand{\CC}[1]{{#1}_{\bullet}} % Kettenkomplex (chain complex)
\newcommand{\CCC}[1]{{#1}^{\bullet}} % Ko-Kettenkomplex (cochain complex)
\newcommand{\keS}{k.\,e.\,S.} % kurze exakte Sequenz
\newcommand{\leS}{l.\,e.\,S.} % lange exakte Sequenz
\newcommand{\Lg}{\mathfrak{g}} % Lie-g
\DeclareMathOperator{\Sing}{Sing} % Singuläre-Homologie-Funktor
\DeclareMathOperator{\Cov}{Cov} % Überlagerungen (coverings)
\DeclareMathOperator{\Gal}{Gal} % Galois-Gruppe
\newcommand{\ceil}[1]{\lceil #1 \rceil} % Aufrunden
\newcommand{\floor}[1]{\lfloor #1 \rfloor} % Abrunden
\newcommand{\Tau}{\mathcal{T}} % Großes Tau
\DeclareMathOperator{\Spec}{Spec} % Spektrum (eines Ringes)
\newcommand{\Pow}{\mathcal{P}} % Potenzmenge (powerset)
\newcommand{\Cont}{\mathcal{C}} % Menge der stetigen/diff'baren Funktionen
\newcommand{\NCat}{\mathcal{NC}} % Nerv der Kategorie C
\newcommand{\Nerve}{\mathcal{N}} % Nerv einer Kategorie
\DeclareMathOperator{\coker}{coker} % Kokern
\DeclareMathOperator{\coim}{coim} % Kobild
\newcommand{\clos}[1]{\overline{#1}} % topologischer Abschluss
\newcommand{\pt}{\{ \, \mathrm{pt} \, \}} % Einpunktiger Raum

% TODO
\newcommand{\AMod}{\text{$A$-$\Mod$}} % Kategorie der A-Linksmoduln
\newcommand{\ModA}{\text{$\Mod$-$A$}} % Kategorie der A-Rechtsmoduln

% Kleinere Klammern
\delimiterfactor=701

% TODO:
% * Diagrammjadgregeln
% * Das Bild eines Garbenmorphismus ist die Garbifizierung des Bildes des Morphismus aufgefasst als Prägarbenmorphismus
% * Serresche Quotientenkategorien (?)
% * "Direct image with compact support" und Adjungierter

\begin{document}

\maketitle{Zusammenfassung Homologische Algebra}

Dies ist eine übersetzte, korrigierte Zusammenfassung des Buches "`Methods of Homological Algebra"' von S.\,I.\,Gelfand und Y.\,I.\,Manin.

% Vorgezogen:
% II. Hauptbegriffe der Kategorientheorie

% II.1 Die Sprache der Kategorien und Funktoren

\section{Kategorientheorie}

\begin{bem}
  Die \href{http://timbaumann.info/uni-spicker/topo.pdf}{Topologie-Zusammenfassung} bietet eine Übersicht über Grundbegriffe der Kategorientheorie.
\end{bem}

% Bemerkung in II.6.9
\begin{konv}
  Man übersetzt ein Diagramm folgendermaßen in eine Proposition:
  Es wird über Objekte und über Morphismen, die als durchgezogener Pfeil dargestellt werden, allquantifiziert, sofern das Obj. oder der Morph. noch nicht eingeführt wurde.
  Die Behauptung ist dann die Existenz der gestrichelten Morphismen, die das Diagramm kommutativ machen. Wenn der Morphismus mit einem Ausrufezeichen markiert ist, so wird eindeutige Existenz gefordert.
\end{konv}

\begin{defn}
  Eine Kategorie $\Cat$ heißt \emph{lokal klein}, wenn $\Hom(X, Y)$ für alle $X, Y \in \Ob(\Cat)$ eine Menge (keine echte Klasse) ist.
  Sie heißt \emph{klein}, wenn auch die Klasse ihrer Objekte eine Menge ist.
  Sie heißt \emph{endlich}, wenn Objekt- und Hom-Mengen sogar nur endlich sind.
\end{defn}

\begin{defn}
  Funktoren $F : \Jat \to \Cat$ mit $\Jat$ klein heißen \emph{Diagramme} in $\Cat$.
\end{defn}

\begin{defn}
  Sei $\CatC$ die Kategorie mit kleinen Kategorien als Objekten und Funktoren als Morphismen.
\end{defn}

% Ausgelassen: II.1.1-II.1.7

% Ausgelassen: II.1.9-II.1.11

% II.1.12
\begin{defn}
  Eine Kategorie $\Dat$ heißt \emph{Unterkategorie} einer Kategorie $\Cat$ (notiert $\Dat \subseteq \Cat$), wenn für alle geeigneten $X$, $Y$, $f$, $g$ gilt:
  \[
    \Ob(\Dat) \subseteq \Ob(\Cat), \enspace
    \Hom_\Dat(X, Y) \subseteq \Hom_\Cat(X, Y) \enspace \text{und} \enspace
    f \circ_\Dat g = f \circ_\Cat g.
  \]
\end{defn}

\begin{defn}
  Eine Unterkategorie $\Dat \subseteq \Cat$ heißt \emph{voll}, wenn
  \[ \fa{X, Y \in \Ob(\Dat)} \Hom_\Dat(X, Y) = \Hom_\Cat(X, Y). \]
\end{defn}

\begin{defn}
  Ein Funktor $F : \Cat \to \Dat$ heißt \ldots
  \begin{itemize}
    \item \ldots{} \emph{treu}, wenn für alle $X, Y \in \Ob(\Cat)$ die Abbildung
    $F : \Hom_\Cat(X, Y) \to \Hom_\Dat(FX, FY)$
    injektiv ist.
    \item \ldots{} \emph{voll}, wenn diese Abb. für alle $X, Y \in \Ob(\Cat)$ surjektiv ist.
  \end{itemize}
\end{defn}

\begin{bem}
  Die Einbettung einer (vollen) Unterkategorie ist ein (voll-)treuer Funktor.
\end{bem}

\begin{defn}
  \begin{itemize}
    \item Ein Objekt $X \in \Ob(\Cat)$ heißt \emph{initiales Objekt}, falls für alle $Y \in \Ob(Y)$ genau ein Morphismus $f \in \Hom_\Cat(X, Y)$ existiert.
    \item Ein Objekt $Z \in \Ob(\Cat)$ heißt \emph{terminales Objekt}, falls für alle $Y \in \Ob(Y)$ genau ein Morphismus $f \in \Hom_\Cat(Y, Z)$ existiert.
  \end{itemize}
\end{defn}

\begin{defn}
  Eine Kategorie $\Cat$ heißt \emph{punktiert}, falls initiales und terminales Objekt in $\Cat$ existieren und zusammenfallen.
\end{defn}

\begin{bspe}
  $\AbGrp$ und die Kat. der punktierten top. Räume sind punktiert.
\end{bspe}

% Ausgelassen: II.1.13

% II.2 (Kategorien und Strukturen, Äquivalenz von Kategorien)

% Ausgelassen: II.2.1-II.2.4

\begin{defn}
  Ein Funktor $F : \Cat \to \Dat$ ist ein \emph{Kategorienäquivalenz}, falls es einen Funktor $G : \Dat \to \Cat$ mit $F \circ G \simeq \Id_\Dat$ und $G \circ F \simeq \Id_\Cat$ gibt. \\
  Die Funktoren $F$ und $G$ heißen dann zueinander \emph{quasiinvers} und die Kategorien $\Cat$ und $\Dat$ äquivalent.
\end{defn}

% Ausgelassen: II.2.6

% II.2.7
\begin{prop}
  $F : \Cat \to \Dat$ ist genau dann eine Kategorienäquivalenz, wenn:
  \inlineitem{$F$ ist volltreu,} \quad
  %\item Jedes Objekt $X \in \Ob(\Dat)$ ist isomorph zu einem Objekt $F(X)$, $X \in \Ob(\Cat)$.
  \inlineitem{$\fa{Y \in \Ob(\Dat)} \ex{X \in \Ob(\Cat)} Y \cong F(X)$}
\end{prop}

% Ausgelassen: II.2.8 (Bsp. Galois-Theorie)

% II.2.9
\begin{bsp}
  Sei $B$ ein lokal wegzshgder, semi-lokal einfach zshgder topol. Raum. Dann ist die Kategorie $\Cov(B)$ der Überlagerungen von $B$ äquivalent zur Kategorie $[\pi(B), \SetC]$ der mengenwertigen Funktoren auf dem Fundamentalgruppoid von $B$. Dabei ist
  \begin{align*}
    F : \Cov(B) & \to [\pi(B), \SetC], \quad F(p : \tilde{B} \to B) \coloneqq G_{\tilde{B},p}, \\
    G_{\tilde{B},p}(b \in B) & \coloneqq p^{-1}(b), \quad G_{\tilde{B},p}(\gamma : \I \to B)(\tilde{b} \in p^{-1}(\gamma(0))) \coloneqq \tilde{\gamma}(1), \\
    & \text{mit $\tilde{\gamma}$ Liftung von $\gamma$ mit $\tilde{\gamma}(0) = \tilde{b}$}.
  \end{align*}
\end{bsp}

% Ausgelassen: II.2.10 (Bsp. Kommutative Banach-Algebren)

% Ausgelassen: II.2.11 (Bsp. Pontryagin-Dualität)
% Ausgelassen: II.2.12 (Bemerkungen)

% II.2, exercises
\begin{defn}
  Zwei Ringe $A$ und $B$ heißen \emph{Morita-äquivalent}, wenn ihre Kategorien der (Links/Rechts)-Moduln äquivalent sind.
\end{defn}

% II.3: Strukturen und Kategorien: Darstellbare Funktoren

% Aus II.1.8
\begin{defn}
  Sei $\Cat$ eine Kategorie, $X \in \Ob(\Cat)$. Der \emph{kontravariante Hom-Funktor} $h_X : \Cat^\op \to \SetC$ ist definiert durch
  \[
    h_X(Y) \coloneqq \Hom(Y, X), \quad
    h_X(h : Y' \to Y)(g : Y \to X) \coloneqq g \circ h.
  \]
  Allgemeiner gibt es den Funktor $\Hom : \Cat^\op \times \Cat \to \SetC$ mit
  \[
    \Hom(h : Y' \to Y, f : X \to X')(g : Y \to X) \coloneqq f \circ g \circ h.
  \]
\end{defn}

% Aus II.3.1
\begin{nota}
  $\hat{\Cat} \coloneqq [\Cat^\op, \SetC]$
\end{nota}

% Aus II.3.1
\begin{defn}
  Ein Element $x \in X(Y) \!\coloneqq\! \Hom(Y, X)$ heißt \emph{$Y$-Element} von $X$.
\end{defn}

% II.3.2
\begin{defn}
  Ein Funktor $F \in \Ob(\hat\Cat)$ wird \emph{dargestellt} durch $X \in \Ob(\Cat)$, falls $F \cong h_X$.
  Er heißt \emph{darstellbar}, falls ein solches $X$ existiert.
\end{defn}

\begin{bsp}
  Sei $k$ ein Körper. Für jede $k$-Algebra $A$ ist dann
  \[ \Hom_{\SetC}(\Hom_{\kAlg}(k[X], A), A), \quad \varphi \mapsto \varphi(X) \]
  eine in $A$ natürliche Bijektion. Somit stellt $k[X] \in \Ob(\kAlg)$ den Vergissfunktor $V : \kAlg \to \SetC$ (ko-)dar.
\end{bsp}

\begin{defn}
  Die \emph{Yoneda-Einbettung} ist der Funktor
  \[ Y : \Cat \to \hat\Cat, \enspace X \mapsto h_X, \enspace \phi \mapsto (\phi \circ \blank : X(Y) \to X'(Y))_{Y \in \Ob(\Cat)}. \]
\end{defn}

\begin{lem}
  Sei $F \in \hat{\Cat}$, $Y \in \Cat$. Ist dann $s \in F(Y)$, so existiert genau eine natürliche Transformation $\eta : \Hom_\Cat(\blank, Y) \to F$ mit $\eta(Y)(\id_Y) = s$. \\
  Ist $\eta$ ein Isomorphismus, stellt also $Y$ den Funktor $F$ vermöge $\eta$ dar, so heißt $s$ die \emph{universelle Familie}.
\end{lem}

\begin{kor}[\emph{Yoneda-Lemma}]
  Es gibt es eine natürliche Bijektion
  \[
    \Hom_{\hat\Cat}(h_X, F) \cong F(X) \quad
    \text{für alle $X \in \Ob(\Cat)$, $F \in \hat{C}$.}
  \]
\end{kor}

\begin{kor}
  Die Yoneda-Einbettung ist volltreu und liefert eine Kat'en- Äquivalenz von $\Cat$ und der vollen Unterkat. der darstellb. Funkt. in $\hat\Cat$.
\end{kor}

\begin{kor}
  Stellen $Y$ und $Y'$ beide den Funktor $F$ dar (mittels natürlichen Transformationen $\alpha$, $\beta$), so existiert genau ein Isomorphismus $\varphi \in \Hom(Y, Y')$, sodass $\alpha = \beta \circ \Hom(\blank, \varphi)$.
\end{kor}

\begin{defn}
  Das \emph{Produkt} von $X, Y \in \Ob(\Cat)$ ist ein Obj. $Z \in \Ob(\Cat)$, das
  %den Funktor
  $F : \Cat^\op \to \SetC, \enspace U \mapsto X(U) \times Y(U), \enspace \phi \mapsto ((\blank \circ \phi) \times (\blank \circ \phi))$ darstellt.
\end{defn}

\begin{bem}
  Diese Definition ist äquivalent zur folgenden: \\
  Das Produkt von $X, Y$ ist ein Objekt $Z \in \Ob(\Cat)$ zusammen mit Morphismen
  $p_X : Z \to X$ und $p_Y : Z \to Y$,
  falls
  \vspace{-10pt}
  \begin{centertikz}
    \matrix (mat) [matrix of nodes, column sep=1cm, row sep=0.45cm]{
      & \node (Z') {$Z'$}; \\
      \node (X) {$X$}; &
      \node (Z) {$Z$}; &
      \node (Y) {$Y$.}; \\
    };
    \draw[->] (Z) to node [above] {$\,\,\,p_X$} (X);
    \draw[->] (Z) to node [above] {$p_Y\,\,\,$} (Y);
    \draw[->] (Z') to node [left] {} (X);
    \draw[->] (Z') to node [right] {} (Y);
    \draw[->,dashed] (Z') to node [right] {!} (Z);
  \end{centertikz}
  \vspace{-12pt}
\end{bem}

\begin{defn}
  Seien $\phi : X \to S$ und $\psi : Y \to S$ Abb. von Mengen. \\
  Das Faserprodukt von $X$ und $Y$ über $S$ ist
  \[ X \times_S Y \coloneqq \Set{ (x, y) \in X \times Y }{ \phi(x) = \psi(y) }. \]
\end{defn}

\begin{defn}
  Sei $\phi \in \Hom_\Cat(X, S)$ und $\psi \in \Hom_\Cat(Y, S)$. \\
  Das \emph{Faserprodukt} von $X$ und $Y$ über $S$ ist ein Obj. in $\Cat$, das den Funktor
  $F : \Cat^\op \to \SetC, \enspace U \mapsto X(U) \times_{S(U)} Y(U)$ darstellt.
\end{defn}

\begin{bem}
  % TODO: Slash-Richtung überprüfen
  Das Faserprodukt von $X$ und $Y$ über $S$ ist das Produkt von $X \xrightarrow{\phi} S$ und $Y \xrightarrow{\psi} S$ in der Scheibenkategorie $\Cat/S$.
\end{bem}

% Ausgelassen: II.3.6 (Pfeilumkehrung ~> Koprodukt, amalgamierte Summe)
% Ausgelassen: II.3.7 (Bsp: Tensor-Produkt von $A$-Algebren)

\begin{defn}
  Eine \emph{Gruppenstruktur} auf einem Objekt $X \in \Ob(\Cat)$ ist gegeben durch Gruppenstrukturen auf $\Hom(Y, X)$ für alle $Y \in \Ob(\Cat)$ und Gruppenmorphismen $\Hom(Y, X) \to \Hom(Y', X)$ für jeden Morphismus $\phi : Y' \to Y$ (die die offensichtlichen Axiome erfüllen).
\end{defn}

\begin{bem}
  Falls $\Cat$ ein term. Obj. $1$ und die Produkte $X \!\times\! X$ und $X \!\times\! X \!\times\! X$ besitzt, dann ist eine Gruppenstr. auf $X$ geg. durch Morphismen
  \[
    m : X \times X \to X \text{ (Mult.)}, \enspace
    i : X \to X \text{ (Inv.)}, \enspace
    e : 1 \to X \text{ (Einheit)},
  \]
  die die offensichtlichen Axiome erfüllen.
\end{bem}

% II.5.11e)
\begin{defn}
  Sei $\Cat$ eine Kategorie. Ein Morphismus $f \in \Hom_\Cat(X, Y)$ heißt
  \begin{itemize}
    \item \emph{Monomorphismus} ($f : X \hookrightarrow Y$), wenn $f$ linkskürzbar ist, \dh{}
    \[
      \fa{X' \in \Ob(\Cat)} \fa{g, h \in \Hom_\Cat(X', X)}
      f \circ g = f \circ h \implies g = h.
    \]
    \item \emph{Epimorphismus} ($f : Y \twoheadrightarrow Y$), wenn $f$ rechtskürzbar ist, \dh{}
    \[
      \fa{Y' \in \Ob(\Cat)} \fa{g, h \in \Hom_\Cat(Y, Y')}
      g \circ f = h \circ f \implies g = h.
    \]
  \end{itemize}
\end{defn}

\begin{defn}
  Sei $X \in \Ob(\Cat)$ ein Objekt. Auf der Klasse der Monomorphis- men $(i : U \to X) \in \Cat$ von einem Objekt $U \in \Ob(\Cat)$ nach $X$ ist durch
  \[ (U, i) \leq (U', i') \coloniff \ex{f : U' \to U} i' = i \circ f \]
  eine Präordnung definiert. Ein \emph{Unterobjekt} von $X$ ist eine Äquivalenzklasse dieser Präordnung, also von der Äq'relation
  \[ x \sim y \coloniff x \leq y \wedge y \leq x. \]
\end{defn}

% XXX: Gibt es dafür eine vernünftige deutsche Übersetzung?
\begin{defn}
  Eine Kategorie heißt \emph{well-powered}, wenn die Präordnung der Unterobjekte von jedem Obj. eine Menge (nicht nur eine Klasse) ist.
\end{defn}

% Ausgelassen: II.3.11,II.3.12 (Affine Gruppen-Schemata)
% Ausgelassen: II.3.13 (Cartier-Dualität)
% Ausgelassen: II.3.14 (Relative Gruppen-Schemata)

\subsection{(Ko-)Limiten}

% II.3.15 (Limiten)

\begin{defn}
  Seien $\Jat$, $\Cat$ Kat'en.
  Der \emph{Diagonal-Funktor} $\Delta \!:\! \Cat \!\to\! [\Jat, \Cat]$ ist
  \[
    (\Delta X)(J \in \Ob(\Jat)) \coloneqq X, \quad
    (\Delta X)(\phi) \coloneqq \id_X, \quad
    (\Delta f)_{J \in \Ob(\Jat)} \coloneqq f.
  \]
\end{defn}

% II.3.16
\begin{defn}
  Seien $\Jat$, $\Cat$ Kategorien, $\Jat$ klein. Der \emph{Limes} eines Funktors $F : \Jat \to \Cat$ ist ein Objekt $X \in \Ob(\Cat)$, das den Funktor
  \[
    G \in \hat\Cat, \quad
    G(Y) \coloneqq \Hom_{[\Jat, \Cat]}(\Delta Y, F), \quad
    G(f)(\eta) \coloneqq \eta \circ \Delta f
  \]
  darstellt. Man notiert $X = \Lim F = \Lim_{j \in \Jat} F(j)$.
\end{defn}

% Ausgelassen: Gleichung (II.5)

% II.3.17 (Die univ. Eigenschaft des Limes)
\begin{defn}
  Ein \emph{Kegel} o. \emph{Möchtegern-Limes} eines Funktors $F : \Jat \to \Cat$
  ist ein Objekt $X \in \Ob(\Cat)$ mit Projektionsabb. $f_J : X \to F(J)$
  für alle $J \in \Ob(\Jat)$, sodass
  $\fa{h \in \Hom_\Jat(J, I)} f_I = F(h) \circ f_J$. \\
  Ein Kokegel von $F$ ist ein Kegel von $F^\op : \Jat^\op \to \Cat^\op$.
\end{defn}

\begin{bem}
  Der Limes $X$ ist durch folgende \emph{universelle Eigenschaft} charakterisiert:
  Er ist ein terminales Objekt in der Kategorie der Kegel über $F$, \dh{} er ist ein Kegel und für jeden weiteren Kegel $X'$ gibt es genau ein $g \in \Hom_\Cat(X', X)$ mit
  $\fa{J \in \Ob(\Jat)} f'_J = f_J \circ g$.
\end{bem}

\begin{bem}
  Die univ. Eigenschaft zeigt: Der Limes ist funktoriell, \dh{} wenn in $\Cat$ alle $\Jat$-Limiten (\dh{} Limiten von Funktoren $\Jat \to \Cat$) existieren, dann gibt es einen Funktor $\Lim : [\Jat, \Cat] \to \Cat$.
\end{bem}

\begin{defn}
  Eine Kategorie $\Cat$ heißt \emph{filtriert}, falls für alle Funktoren $F : \Iat \to \Cat$ mit endl. Quellkat. $\Iat$ ein Kokegel von $F$ existiert. \\
  Eine Kategorie $\Cat$ heißt \emph{kofiltriert}, falls $\Cat^\op$ filtriert ist.
\end{defn}

\begin{defn}
  Eine \emph{(ko-)gerichtete Menge} ist eine Menge mit einer Präordnung, deren Präordnungskategorie (ko-)filtriert ist.
\end{defn}

\begin{defn}
  Der Limes eines Funktors $F : \Cat \to \Dat$ heißt
  \begin{itemize}
    \item \emph{projektiver} oder \emph{inverser Limes}, wenn $\Cat$ die Präordnungs- kategorie einer kogerichteten Menge ist. Man notiert $\varprojlim F$.
    \item \emph{filtriert}, wenn $\Cat$ kofiltriert ist.
  \end{itemize}
\end{defn}

% II.3.18
\begin{bem}
  Folgende Konzepte lassen sich als Spezialfall des Limes über eine spezielle Indexkategorie $\Jat$ auffassen:
  \begin{center}
    \begin{tabular}{ | r | l | }
      \hline
      Konzept & Indexkategorie $\Jat$ \\ \hline
      Terminales Objekt & $\emptyset$ (leere Kategorie) \\
      (binäres) Produkt & $\mathbf{2} \coloneqq \{ 0, 1 \}$ (kein nichttrivialer Morphismus) \\
      (endl.) \emph{Produkt} & (endliche) Menge, aufgefasst als Kategorie \\
      Faserprodukt & $1 \to 0 \leftarrow 2$ (zwei nichttriviale Morphismen) \\
      \emph{Differenzkern} &  $0 \rightrightarrows 1$ (zwei nichttriviale Morphismen) \\ \hline
    \end{tabular}
  \end{center}
\end{bem}

\begin{bem}
  Insbesondere sind terminale Objekte leere Produkte.
\end{bem}

\begin{lem}
  Sei $K$ ein Differenzkern von $(X \rightrightarrows Y) \in \Cat$. \\
  Dann ist der Morphismus $i : K \to X$ ein Monomorphismus.
\end{lem}

% II.3.19 (Das duale Konzept: Kolimiten)

\begin{defn}
  Sei $\Jat$ klein.
  Der \emph{Kolimes} eines Funktors $F : \Jat \to \Cat$ ist ein Objekt $X \in \Ob(\Cat^\op) = \Ob(\Cat)$, das den Funktor
  \[
    G \in \hat{\Cat^\op} = [\Cat, \SetC], \enspace
    G(Y) \coloneqq \Hom_{[\Jat, \Cat]}(F, \Delta Y), \enspace
    G(f)(\eta) \coloneqq \Delta f \circ \eta
  \]
  darstellt. Man notiert $X = \Colim F = \Colim_{j \in \Jat} F(j)$.
\end{defn}

\begin{bem}
  Der Kolimes von $F \!:\! \Jat \!\to\! \Cat$ ist der Limes von $F^\op \!:\! \Jat^\op \!\to\! \Cat^\op$.
\end{bem}

\begin{bem}
  Wenn in $\Cat$ alle $\Jat$-Kolimiten existieren, dann gibt es einen Funktor $\Colim : [\Jat, \Cat] \to \Cat$.
\end{bem}

% II.3.19
\begin{bem}
  Folgende Konzepte sind ein Spezialfall des Kolimes:
  \begin{center}
    \begin{tabular}{ | r | l | }
      \hline
      Konzept & Indexkategorie $\Jat$ \\ \hline
      Initiales Objekt & $\emptyset$ (leere Kategorie) \\
      (bin.) \emph{Koprodukt} & $\mathbf{2} \coloneqq \{ 0, 1 \}$ (kein nichttrivialer Morphismus) \\
      (endl.) Koprodukt & (endliche) Menge, aufgefasst als Kategorie \\
      \emph{Kofaserprodukt} & $1 \to 0 \leftarrow 2$ (zwei nichttriviale Morphismen) \\
      \emph{Kodifferenzkern} &  $0 \rightrightarrows 1$ (zwei nichttriviale Morphismen) \\ \hline
    \end{tabular}
  \end{center}
\end{bem}

\begin{bem}
  Initiale Objekte sind leere Koprodukte.
\end{bem}

\begin{defn}
  Der Kolimes eines Funktors $F : \Cat \to \Dat$ heißt
  \begin{itemize}
    \item \emph{induktiver} oder \emph{direkter Limes}, wenn $\Cat$ die Präordnungs- kategorie einer gerichteten Menge ist. Man notiert $\varinjlim F$.
    \item \emph{filtrierter Kolimes}, wenn $\Cat$ filtriert ist.
  \end{itemize}
\end{defn}

% nicht im Buch
\begin{defn}
  \begin{itemize}
    \item Eine Kategorie $\Cat$ heißt \emph{(ko-)vollständig}, wenn alle Diagramme $F : \Jat \to \Cat$ einen (Ko-)Limes in $\Cat$ besitzen.
    \item Eine Kategorie $\Cat$ heißt \emph{endlich (ko-)vollständig}, wenn alle endlichen (Ko-)Limiten in $\Cat$ existieren.
  \end{itemize}
\end{defn}

\begin{bspe}
  Vollständig sind: \enspace
  \inlineitem{$\SetC$,}
  \inlineitem{$\Grp$,}
  \inlineitem{$\AbGrp$,}
  \inlineitem{$\Top$,}
  \inlineitem{$k$-$\Vect$.}
\end{bspe}
% XXX: Beispiele für kovollständige Kategorien?

\begin{lem}
  Eine Kategorie enthält endliche Produkte, wenn sie ein terminales Objekt und binäre Produkte besitzt.
  Duales gilt für Koprodukte mit initialem statt terminalen Objekt.
\end{lem}

% II.3.20
\begin{satz}
  Angenommen, eine Kategorie $\Cat$ enthält (Ko-)Differenzkerne und (endliche) Produkte.
  Dann ist $\Cat$ (endlich) (ko-)vollständig.
\end{satz}

% Ausgelassen, da später sogar die Vollständigkeit von $\SetC$ erwähnt wird
\iffalse
  % II.3.21
  \begin{kor}
    In $\SetC$ existieren alle endlichen Limiten und Kolimiten.
  \end{kor}
\fi

% Nicht (dort) im Buch
\begin{bem}
  Angenommen, in $\Cat$ existieren alle $\Jat$-Limiten.
  Sei $\Iat$ eine bel. Kategorie.
  Dann ex. alle $\Jat$-Limiten in $[\Iat, \Cat]$ und die Limiten werden objektweise berechnet:
  Sei $F : \Jat \to [\Iat, \Cat]$ ein Funktor, dann ist
  \[
    (\Lim F)(I) = \Lim (F(\blank)(I)), \quad
    (\Lim F)(f) = \Lim (F(\blank)(f)).
  \]
\end{bem}

\begin{defn}
  Ein Funktor $F : \Cat \to \Dat$ heißt \emph{stetig}, wenn er \emph{Limiten bewahrt}, \dh{} für alle Funktoren $D : \Jat \to \Dat$ (mit $\Jat$ klein) mit $\Lim D \in \Ob(\Cat)$ ex. auch der Limes von $F \circ G$ in $\Dat$ und es gilt
  \[ \Lim (F \circ D) \cong F(\Lim D). \]
  Ein Funktor $F$ heißt \emph{kostetig}, wenn er \emph{Kolimiten bewahrt}.
\end{defn}

\begin{satz}
  Sei $F : I \times J \to \Cat$ ein Diagramm. Wenn einer der folgenden Limiten existiert, dann alle, und sie sind natürlich isomorph:
  \[ \lim_i \lim_j F(i, j) \cong \lim_j \lim_i F(i, j) \cong \lim_{i,j} F(i, j). \]
\end{satz}

\begin{samepage}
  \subsection{Adjunktionen}
\end{samepage}

\begin{defn}
  Ein Funktor $F : \Cat \to \Dat$ heißt \emph{linksadjungiert} zum Funktor $G : \Cat \to \Dat$, wenn es einen natürlichen Isomorphismus
  \[
    \Hom_\Dat(F(\blank), \blank) \cong \Hom_\Cat(\blank, G(\blank))
  \]
  gibt (dabei sind beide Seiten Funktoren $\Cat^\op \times \Dat \to \SetC$). \\
  Dann heißt $G$ auch \emph{rechtsadjungiert} zu $F$. Man notiert $F \ladj G$ oder sagt, es bestehe eine Adjunktion \enspace $F : \Cat \rightleftarrows \Dat : G$.
\end{defn}

\begin{bem}
  Sei $F : \Cat \to \Dat$ ein Funktor.
  Dann besitzt $F$ genau dann einen Rechtsadjungierten $G : \Dat \to \Cat$, wenn für alle $Y \in \Ob(\Dat)$ der Funktor
  \[
    \Cat^\op \to \SetC, \quad
    X \mapsto \Hom_\Dat(FX, Y), \quad
    f \mapsto (\blank \circ F(f))
  \]
  darstellbar ist, \dh{} es existiert $GY \in \Ob(\Cat)$ und Isomorphismen
  \[ a^Y_X : \Hom_\Dat(FX, Y) \xrightarrow{\cong} \Hom_\Cat(X, GY) \]
  mit $\fa{\phi \in \Hom(X', X)} a^Y_{X'}(\blank \circ F(\phi)) = a^Y_X(\blank) \circ \phi$. \\
  Dann ist $G$ auf Morphismen definiert durch
  \[ G(f \in \Hom_\Dat(Y, Y')) \coloneqq a^{Y'}_{GY} \left( f \circ \left( a^Y_{GY} \right)^{-1} (\id_{GY}) \right). \]
\end{bem}

\begin{bem}
  Sei $F : \Cat \to \Dat$ linksadjungiert zu $G : \Dat \to \Cat$. Setze
  \begin{align*}
    \eta_X & \coloneqq a_X^{FX}(\id_{FX}) : X \to GFX, \\
    \epsilon_Y & \coloneqq (a_{GY}^Y)^{-1}(\id_{GY}) : FGY \to Y.
  \end{align*}
  Dann sind $\eta : \Id_{\Cat} \to G \circ F$ (genannt \emph{Einheit}) und $\epsilon : F \circ G \to \Id_\Cat$ (genannt \emph{Koeinheit}) natürliche Transformationen und es gilt
  \[
    (G \xrightarrow{\eta G} GFG \xrightarrow{G \epsilon} G) = \id_G, \quad
    (F \xrightarrow{\epsilon F} FGF \xrightarrow{F \eta} F) = \id_F.
  \]
  Umgekehrt definieren zwei solche nat. Transf. $\eta$ und $\epsilon$, die diese Glei- chungen erfüllen, eine Adj. zwischen $F$ und $G$.
  Dabei ist $\eta_X$ univ. unter den Morphismen von $X \in \Ob(\Dat)$ zu einem Obj. der Form $GY$:
  Für alle $f \in \Hom_\Dat(X, GY)$ gibt es genau ein $h \in \Hom(FX, Y)$ mit $f = G(h) \circ \eta_X$, und zwar $h = (a_X^Y)^{-1}(f)$. Duales gilt für $\epsilon_Y$.
\end{bem}

\begin{lem}[Verknüpfung von Adjunktionen]\mbox{}\\
  Sei $F_1 \!:\! \Cat \to \Dat$ zu $G_1 \!:\! \Dat \to \Cat$ und $F_2 \!:\! \Dat \to \Eat$ zu $G_2 \!:\! \Eat \to \Dat$ linksadj. \\
  Dann ist $F_2 \circ F_1 : \Cat \to \Eat$ zu $G_1 \circ G_2 : \Eat \to \Cat$ linksadjungiert.
\end{lem}

\begin{lem}[Eindeutigkeit des adjungierten Funktors]\mbox{}\\
  \begin{itemize}
    \item Gelte $F \ladj G_1$ und $F \ladj G_2$. Dann sind $G_1$ und $G_2$ nat. isomorph.
    \item Gelte $F_1 \ladj G$ und $F_2 \ladj G$. Dann sind $F_1$ und $F_2$ nat. isomorph.
  \end{itemize}
\end{lem}

\begin{bem}
  Sei $F : \Cat \rightleftarrows \Dat : G$ eine Adjunktion und $\Jat$ klein. \\
  Dann gibt es eine ind. Adjunktion $(F \circ \blank) : [\Jat, \Cat] \rightleftarrows [\Jat, \Dat] : (G \circ \blank)$.
\end{bem}

\begin{bspe}
  \begin{itemize}
    % Übungsblatt 12, Aufgabe 4
    \item Angenommen, in $\Cat$ existieren $\Jat$-Limiten bzw. $\Jat$-Kolimiten. Dann gibt es eine Adj.
    $\Delta \ladj \Lim$
    bzw.
    $\Colim \ladj \Delta$.
    \item Sei $F : \SetC \to \Grp$ der Funktor, der die freie Gruppe über einer Menge bastelt und $V : \Grp \to \SetC$ der Vergiss- funktor. Dann gilt $F \ladj V$. Gleiches gilt für viele weitere "`freie"' Konstruktionen.
    \item Sei $\KHaus$ die Kat. der kompakten Hausdorffräume und $K : \Top \to \KHaus$ die Stone-Čech-Kompaktifizierung und $I : \KHaus \to \Top$ die Inklusion. Dann gilt $K \ladj I$.
  \end{itemize}
\end{bspe}

\begin{defn}
  Im Spezialfall, dass $\Cat$ und $\Dat$ Präordnungskategorien sind, wird ein Paar von adjungierten Funktoren (\dh{} monotonen Abbildungen) zwischen $\Cat$ und $\Dat$ auch \emph{Galoisverbindung} genannt.
\end{defn}

\begin{bspe}
  \begin{itemize}
    % Übungsblatt 12, Aufgabe 5
    \item $(\ceil{\blank} : \R \to \Z) \ladj (i : \Z \hookrightarrow \R) \ladj (\floor{\blank} : \R \to \Z)$
    \item Sei $L \supset K$ eine endl. Körpererweiterung.
    Für eine Zwischenerw. $L \!\supseteq\! M \!\supseteq\! K$ sei
    $\Gal(L, M) \!\coloneqq\! \{ f \!\in\! \Aut(L) | f|_M \!=\! \id_M \}$ die Galoisgruppe von $L$ über $M$.
    Dann ist
    \begin{align*}
      \{ \, \text{Untergruppen von } \Gal(L, K) \, \} & \leftrightarrow \{ \, \text{Zwischenerw. } L \!\supseteq\! M \!\supseteq\! K \, \} \\
      G & \mapsto \Set{x \in L}{\fa{\sigma \in G} \sigma(x) = x} \\
      \Gal(L, M) & \mapsfrom M
    \end{align*}
    eine Galoisverbindung (dabei sind Untergruppen durch Inklusion und Zwischenerweiterungen umgekehrt durch Inklusion geordnet).
  \end{itemize}
\end{bspe}

\begin{lem}
  Sei $F \ladj G$ eine Adjunktion. Dann gilt:
  \begin{itemize}
    \item $F$ bewahrt Kolimiten (\emph{LAPC}, left-adjoints preserve colimits).
    \item $G$ bewahrt Limiten (\emph{RAPL}, right-adjoints preserve limits).
  \end{itemize}
\end{lem}

\begin{beweis}[RAPL]
  Sei $\Jat$ eine kleine Indexkategorie.
  %Wir haben folgende Adjunktionen:
  Es gilt:
  \begin{align*}
    (F \circ \blank) \circ (\Delta : \Cat \to [\Jat, \Cat]) & \ladj (\Lim : [\Jat, \Cat] \to \Cat) \circ (G \circ \blank), \\
    (\Delta : \Dat \to [\Jat, \Dat]) \circ F & \ladj G \circ (\Lim : [\Jat, \Dat] \to \Dat).
  \end{align*}
  Da $(F \circ \blank) \circ \Delta \equiv \Delta \circ F$, folgt aus der Eindeutigkeit des Rechtsadjungierten
  $\Lim (G \circ D) \cong G(\Lim D)$ natürlich in $D$.
\end{beweis}

\begin{bem}
  Sei umgekehrt $G : \Dat \to \Cat$ ein stetiger Funktor. \\
  Unter gewissen Bedingungen an die Größe der Kategorien $\Cat$ und $\Dat$ besitzt dann $G$ einen Linksadjungierten:
\end{bem}

\begin{defn}
  \begin{itemize}
    \item Ein \emph{Koerzeuger} einer Kategorie $\Cat$ ist ein Objekt $S \in \Ob(\Cat)$, für das der Funktor $h_S : \Cat^\op \to \SetC$ treu ist.
    \item Eine \emph{koerzeugende Menge} von $\Cat$ ist eine Teil\textit{menge} $\mathcal{S}$ von $\Ob(\Cat)$, für die der Funktor $h_{\mathcal{S}} \coloneqq \prod_{S \in \mathcal{S}} h_S$ treu ist.
  \end{itemize}
\end{defn}

% nicht im Buch, Quelle: http://ncatlab.org/nlab/show/adjoint+functor+theorem
\begin{lem}
  Ein stetiger Funktor $G : \Dat \to \Cat$ hat einen Linksadj, wenn:
  \begin{itemize}
    \item General Adjoint Functor Theorem: $\Dat$ ist vollständig und lokal klein und $G$ erfüllt die Lösungsmengen-Bedingung:
    \begin{align*}
      & \fa{X \in \Ob(\Cat)} \ex{I \text{ Menge}} \ex{(f_i : X \to G(Y_i))_{i \in I}} \\
      & \fa{(g : X \to Z) \in \Cat} \ex{i \in I, \, h : G(Y_i) \to Z} g = h \circ f_i.
    \end{align*}
    \item Special Adjoint Functor Theorem (\emph{SAFT}): \\
    $\Dat$ ist vollständig, well-powered (damit automatisch lokal klein), besitzt eine kleine koerzeugende Menge und $\Cat$ ist lokal klein.
  \end{itemize}
\end{lem}

% nicht im Buch
\begin{defn}
  Eine \emph{monoidale Kategorie} $\Cat$ besitzt einen Funktor $\otimes : \Cat \times \Cat \to \Cat$ (genannt Tensorprodukt), ein Objekt $1 \in \Ob(\Cat)$ und natürliche Isomorphismen
  (die zwei Kohärenzbedingungen erfüllen)
  \[
    a_{X,Y,Z} : (X \otimes Y) \otimes Z \cong X \otimes (Y \otimes Z), \enspace
    \lambda_X : 1 \otimes X \cong X, \enspace
    \rho_X : X \otimes 1 \cong X.
  \]
\end{defn}

% II.4.23
\begin{defn}
  Sei $(\Cat, \otimes)$ eine monoidale Kategorie. Ein \emph{interner Hom-Funktor} ist ein Funktor $[\blank,\blank] : \Cat^\op \times \Cat \to \Cat$, für den gilt: für alle $X \in \Ob(\Cat)$ der Funktor $\blank \otimes X$ linksadjungiert zu $[X, \blank]$ ist, \dh{}
  \[ \Hom_\Cat(Y \otimes X, Z) \cong \Hom_\Cat(Y, [X, Z]). \]
\end{defn}

\begin{nota}
  $[X, Y] =: Y^X$ heißt auch \emph{Exponentialobjekt}.
\end{nota}

% nicht im Buch
\begin{defn}
  Eine monoidale Kategorie heißt \emph{kartesisch abgeschlossen}, wenn sie einen internen Hom-Funktor besitzt.
\end{defn}

\begin{bspe}
  $\SetC$, $\AbGrp$, $k$-$\Vect$ und $\CatC$ sind kartesisch abgeschlossen.
\end{bspe}


\section{Abelsche und additive Kategorien}

Sei $\Cat$ eine Kategorie.

% II.5.2
\begin{axiom}
  $\Cat$ erfüllt \textbf{A1}, wenn sie über $\AbGrp$ angereichtert ist, \dh{} die Morphismenmengen sind ab. Gruppen und die Verknüpfung
  \[ \circ : \Hom_\Cat(Y, Z) \times \Hom_\Cat(X, Y) \to \Hom_\Cat(X, Z) \]
  ist für alle $X, Y, Z \in \Ob(\Cat)$ bilinear.
\end{axiom}

% II.5.3
\begin{axiom}
  $\Cat$ erfüllt \textbf{A2}, wenn es ein \emph{Nullobjekt} $0 \in \Ob(\Cat)$ gibt mit
  \[ \Hom_\Cat(0, 0) = \text{Nullgruppe} = \{ \id_0 \}. \]
\end{axiom}

\begin{bem}
  Dann ist auch $\Hom_\Cat(X, 0) \!=\! \Hom(0, X) \!=\! 0$ für alle $X \!\in\! \Ob(\Cat)$.
  Somit ist $0$ initiales und terminales Objekt, und folglich $\Cat$ punktiert.
\end{bem}

% II.5.4
\begin{axiom}
  $\Cat$ erfüllt \textbf{A3}, wenn es für alle $X, Y \in \Ob(\Cat)$ ein Objekt $X \oplus Y \in \Ob(\Cat)$ (genannt \emph{direkte Summe}) und Morphismen
  \[ X \stackrel[i_x]{p_X}{\leftrightarrows} X \oplus Y \stackrel[i_Y]{p_Y}{\rightleftarrows} Y \]
  gibt mit
  \inlineitem{$p_X \circ i_X = \id_X$}, \enspace
  \inlineitem{$p_Y \circ i_Y = \id_Y$}, \enspace
  \inlineitem{$p_Y \circ i_X = 0$}, \enspace
  \inlineitem{$p_X \circ i_Y = 0$}, \enspace
  \inlineitem{$(i_X \circ p_X) + (i_Y \circ p_Y) = \id_{X \oplus Y}.$}
\end{axiom}

% II.5.5
\begin{bem}
  $X \oplus Y$ ist sowohl Produkt als auch Koprodukt von $X$ und $Y$.
\end{bem}

% II.5.6

% II.5.6
\begin{defn}
  Der \emph{Kern} $\ker \varphi$ eines Morphismus $\varphi \in \Hom_\Cat(X, Y)$ ist
  ein Paar $(K \in \Ob(\Cat), k \in \Hom_\Cat(K, X))$ mit
  sodass $\varphi \circ k = 0$, sodass es für alle $k' \in \Hom(K', X)$ mit $\varphi \circ k'$ einen eindeutigen Morphismus $h \in \Hom_\Cat(K', K)$ mit $k' = k \circ h$ gibt.
\end{defn}

\begin{bem}
  Die Definition des Kerns ist äquivalent zur folgenden: \\
  Der Kern von $\varphi$ ist das darstellende Obj. $K \!\in\! \Ob(\AbGrp)$ des Funktors
  \[
    \Cat^\op \to \AbGrp, \enspace
    Z \mapsto \ker(\varphi \circ \blank : X(Z) \to Y(Z)), \enspace
    f \mapsto (\blank \circ f).
  \]
\end{bem}

% Ausgelassen: II.5.7

% II.5.7
\begin{defn}
  Der \emph{Kokern} $\coker \varphi$ eines Morphismus $\varphi \in \Hom_\Cat(X, Y)$ ist ein Paar $(C \in \Ob(\Cat), c \in \Hom_\Cat(Y, C))$ mit $c \circ \varphi = 0$, sodass es für alle $c' \in \Hom_\Cat(Y, C')$ mit $c' \circ \varphi = 0$ einen eindeutigen Morphismus $h \in \Hom_\Cat(C, C')$ mit $c' = h \circ c$ gibt.
\end{defn}

% Ausgelassen: Definition über Doppeldualisierung

\begin{bem}
  Die Definition des Kokerns ist äquivalent zur folgenden: \\
  Der Kern von $\varphi$ ist ein Morphismus $c \in \Hom_\Cat(Y, C)$, sodass
  \[ 0 \to \Hom_\Cat(C, Z) \xrightarrow{c \circ \blank} \Hom_\Cat(Y, Z) \xrightarrow{\varphi \circ \blank} \Hom_\Cat(X, Z) \to 0 \]
  für alle $Z \in \Ob(\Cat)$ exakt ist.
\end{bem}

\begin{acht}
  Der Kokern ist {\em nicht} das darstellende Obj. des Funktors
  \[
    \Cat^\op \to \AbGrp, \enspace
    Z \mapsto \coker(\varphi \circ \blank : X(Z) \to Y(Z)), \enspace
    f \mapsto (\blank \circ f).
  \]
\end{acht}

\begin{bem}
  Kern und Kokern sind eindeutig bis auf Isomorphismus.
\end{bem}

\begin{defn}
  Sei $\varphi \in \Hom_\Cat(X, Y)$. Dann heißt
  \begin{itemize}
    \item $\im \varphi \coloneqq \ker(\coker \varphi)$ \emph{Bild} von $\varphi$,
    \item $\coim \varphi \coloneqq \coker(\ker \varphi)$ \emph{Kobild} von $\varphi$.
  \end{itemize}
\end{defn}

\begin{lem}
  Kerne sind monomorph, Kokerne epimorph.
\end{lem}

\begin{lem}
  Sei $(K, k)$ der Kern, $(C, c)$ der Kokern von $\varphi$. Dann gilt
  \begin{align*}
    \varphi \text{ ist ein Monomorphismus} & \iff K \cong 0, \\
    \varphi \text{ ist ein Epimorphismus} & \iff C \cong 0.
  \end{align*}
\end{lem}

% II.5.9
\begin{axiom}
  $\Cat$ erfüllt \textbf{A4}, wenn für alle $\varphi \in \Hom_\Cat(X, Y)$ eine Sequenz
  \[ K \xrightarrow{k} X \xrightarrow{i} I \xrightarrow{j} Y \xrightarrow{c} C \]
  existiert mit folgenden Eigenschaften: \inlineitem{$\varphi = j \circ i$}
  \begin{itemize}
    \item $(K, k)$ ist der Kern, $(C, c)$ der Kokern von $\varphi$,
    \item $(I, i)$ ist der Kokern von $k$, $(I, j)$ der Kern von $c$.
  \end{itemize}
  Diese Sequenz heißt \emph{kanonische Zerlegung} von $\varphi$.
\end{axiom}

% II.5.11a)
\begin{bem}
  Die kanonische Zerlegung ist eindeutig bis auf eindeutigen Iso.
\end{bem}

% II.5.11b)
\begin{bem}
  Angenommen, $\Cat$ besitzt Kerne und Kokerne. \\
  Dann gibt es für alle $\varphi \in \Hom_\Cat(X, Y)$ die Diagramme
  \[
    \ker \varphi \xrightarrow{k} X \xrightarrow{i} \coim \varphi, \quad
    \im \varphi \xrightarrow{j} Y \xrightarrow{c} \coker \varphi.
  \]
  Aus den univ. Eigenschaften von Kern u. Kokern folgt die Existenz eines Morphismus $l \in \Hom(\coim \varphi, \im \varphi)$ mit $j \circ l \circ i = \varphi$. \\
  Das Ax. \textbf{A4} gilt genau dann, wenn $l$ für alle $\varphi$ ein Isomorphismus ist.
\end{bem}

% Achtung: Anders als im Buch behauptet, ist $l$ i.A. nicht Epi- und Monomorphismus.
% Insbesondere ist die Aussage
%   "eine präabelsche Kategorie ist genau dann abelsch, wenn sie balanciert ist"
% falsch. Siehe dazu
%   http://mathoverflow.net/questions/41722/is-every-balanced-pre-abelian-category-abelian

% II.5.10
\begin{defn}
  \begin{itemize}
    \item Eine \emph{präadditive} Kategorie erfüllt das Axiom \textbf{A1}. \\
    \item Eine \emph{additive} Kategorie erfüllt die Axiome \textbf{A1}-\textbf{A3}. \\
    \item Eine \emph{präab.} Kategorie ist additiv und besitzt Kerne und Kokerne.
    \item Eine \emph{abelsche} Kategorie erfüllt die Axiome \textbf{A1}-\textbf{A4}.
  \end{itemize}
\end{defn}

\begin{bem}
  Die Axiome \textbf{A1}-\textbf{A4} sind selbstdual, \dh{} eine Kategorie $\Cat$ ist genau dann (prä-)additiv / (prä-)abelsch, wenn $\Cat^\op$ es auch ist.
\end{bem}

\begin{bspe}
  Ab. Kategorien sind: \enspace
  \inlineitem{$\AbGrp$,} \enspace
  \inlineitem{$R$-$\Mod$,} \enspace
  \inlineitem{$\PShAb(X)$.}
  % Ausgelassen: Kategorie der Koeffizientensysteme
\end{bspe}

\begin{defn}
  Eine Kategorie $\Cat$ heißt \emph{balanciert}, falls $\forall \, (f \!:\! X \!\to\! Y) \in \Cat$ gilt:
  \[ f \text{ ist epi und mono} \iff f \text{ ist ein Isomorphismus.} \]
\end{defn}

% II.5.11d)
\begin{prop}
  Abelsche Kategorien sind balanciert.
\end{prop}

% http://en.wikipedia.org/wiki/Normal_morphism
\begin{defn}
  \begin{itemize}
    \item Ein Mono- / Epimorphismus heißt \emph{normal} / \emph{konormal}, wenn er Kern / Kokern eines Morphismus ist.
    \item Eine präadd. Kategorie $\Aat$ heißt normal / konormal wenn jeder Mono- / Epimorphismus in $\Aat$ normal / konormal ist.
  \end{itemize}
\end{defn}

% II.6.3a)
\begin{lem}
  Sei $\Aat$ eine abelsche Kategorie.
  \begin{itemize}
    \item Sei $X \xrightarrow{\varphi} Y \in \Aat$ ein Monomorphismus und $(C, c)$ dessen Kokern. \\
    Dann ist $(X, \varphi)$ der Kern von $c$.
    \item Sei $X \xrightarrow{\varphi} Y \in \Aat$ ein Epimorphismus und $(K, k)$ dessen Kern. \\
    Dann ist $(Y, \varphi)$ der Kokern von $k$.
  \end{itemize}
\end{lem}

\begin{kor}
  Ab. Kategorien sind binormal, \dh{} normal und konormal.
\end{kor}

\begin{bem}
  Der (Ko-)Differenzkern von Morphismen $f, g \in \Hom_\Aat(X, Y)$ ist der (Ko-)Kern der Differenz $f - g \in \Hom(X, Y)$.
\end{bem}

% II.6, exercise 1
\begin{kor}
  Abelsche Kategorien sind endlich bivollständig.
\end{kor}



\section{Verklebedaten und simpliziale Mengen}

% §I.1 Triangulierte Räume

\begin{defn}
  \emph{Verklebedaten} sind gegeben durch einen Funktor
  \[ X : \Delta_{\text{strikt}}^\op \to \SetC. \]
  Dabei ist $\Delta_{\text{strikt}}$ die Kategorie mit den Mengen
  $[n] \coloneqq \{ 0, 1, \ldots, n \}$ für $n \in \N$ als Objekten und streng monotonen Abbildungen.
\end{defn}

\begin{nota}
  $X_{(n)} \coloneqq X([n])$ heißt Menge der $n$-Simplizes.
\end{nota}

\begin{defn}
  Das \emph{Standard-$n$-Simplex} $\Delta_n \subset \R^{n+1}$ ist die von den $(n{+}1)$ Standardbasisvektoren aufgespannte affinlineare Hülle. Eine streng monotone Abb $f : [n] \to [m]$ induziert durch Abbilden des $i$-ten Basisvektors auf den $f(i)$-ten eine Inklusion $\Delta_f : \Delta_n \to \Delta_m$, 
\end{defn}

\begin{defn}
  Die \emph{geometrische Realisierung} von Verklebedaten $X$ ist der topologische Raum
  \[ \abs{X} \coloneqq \left( \coprod_{n \in \N} (\Delta_n \times X_{(n)}) \right) / R \]
  Dabei ist $X_{(n)}$ diskret. Die Äquivalenzrelation $R$ wird erzeugt von
  \[
    (\Delta_f(t), x) \sim (t, X(f)(x)) \enspace
    \text{mit $t \in \Delta_m$, $x \in X_{(n)}$, $f : [m] {\to} [n]$ s.m.s.}
  \]
\end{defn}

% Ausgelassen: Proposition I.1.3

\begin{defn}
  Das \emph{$k$-Skelett} $\sk_k X$ von Verklebedaten $X$ ist definiert durch
  \[
    (\sk_k X)_{(n)} \coloneqq
    \Set{ x \in X_{(n)} }{ n \leq k }, \enspace
    %\begin{cases}
    %  X_{(n)}, & \text{falls $n \leq k$,} \\
    %  \emptyset, & \text{sonst.}
    %\end{cases}, \quad
    (\sk_k X)(f) \coloneqq X(f) \enspace \text{sofern möglich}
  \]
\end{defn}

% aus I.4.5
\begin{defn}
  Der \emph{Kegel} $CX$ über Verklebedaten $X$ ist definiert durch
  \begin{align*}
    (CX)_{(0)} & \coloneqq X_{(0)} \amalg \{ \star \}, \quad (CX)_{(n)} \coloneqq X_{(n)} \amalg (X_{(n-1)} \times \{ \star \}), \\
    (CX)(f)(x) & \coloneqq X(f)(x), \\
    (CX)(f)(x,*) &  \coloneqq \begin{cases}
      X(i \mapsto f(i) - 1)(x), & \text{wenn $f(0) > 0$,} \\
      (X(i \mapsto f(i{+}1) - 1)(x), *), & \text{wenn $f(0) = 0$.}
    \end{cases}
  \end{align*}
\end{defn}

% Ausgelassen: I.1.4 (Triangulierung des Produktes zweier Simplizes)

% §I.2 Simpliziale Mengen

\begin{defn}
  Eine \emph{simpliziale Menge} ist ein Funktor
  \[ X : \Delta^\op \to \SetC. \]
  Dabei ist $\Delta$ die Kategorie mit den Mengen
  $[n] \coloneqq \{ 0, 1, \ldots, n \}$ für $n \in \N$ als Objekten und monotonen Abbildungen.
\end{defn}

\begin{nota}
  $X_n \coloneqq X([n])$ heißt Menge der $n$-Simplizes.
\end{nota}

% vorgezogen: aus I.2.15
\begin{defn}
  Eine \emph{simpliziale Abbildung} zw. simpl. Mengen $X$ und $Y$ ist eine nat. Transformation zwischen den Funktoren $X, Y : \Delta^\op \to \SetC$.
\end{defn}

% vorgezogen: aus I.2.15
\begin{defn}
  Die Kategorie der simplizialen Mengen ist $\sSet \coloneqq [\Delta^\op, \SetC]$.
\end{defn}

\begin{defn}
  Die \emph{geometrische Realisierung} einer simplizialen Menge $X$ ist der topologische Raum
  \[ \abs{X} \coloneqq \left( \coprod_{n \in \N} (\Delta_n \times X_{n}) \right) / R \]
  %Dabei ist $\Delta_n \subset \R^{n+1}$ das Standard-$n$-Simplex und $X_{n}$ trägt die diskrete Topologie.
  Die Äquivalenzrelation $R$ wird dabei erzeugt von
  \[
    (\Delta_f(t), x) \sim (t, X(f)(x)) \enspace
    \text{mit $t \in \Delta_m$, $x \in X_n$ u. $f \in \Hom_\Delta([m], [n])$.}
  \]
\end{defn}

\begin{defn}
  Ein topologischer Raum heißt \emph{trianguliert}, wenn er die Realisierung von Verklebedaten ist.
\end{defn}

\begin{defn}
  Der \emph{Nerv} einer Überdeckung $X = \cup_{\alpha \in A} U_\alpha$ eines topologischen Raumes ist die simpliziale Menge
  \begin{align*}
    X_n & \coloneqq \Set{(\alpha_0, \nldots, \alpha_n) \in A^{n+1}}{ U_{\alpha_0} \cap \nldots \cap U_{\alpha_n} \not= \emptyset } \\
    X(f)(\alpha_0, \nldots, \alpha_n) & \coloneqq (\alpha_{f(0)}, \nldots, \alpha_{f(m)}) \quad \text{für } f : [m] \to [n].
  \end{align*}
\end{defn}

\begin{bem}
  Falls die Überdeckung lokal endlich ist und alle nichtleeren, endlichen Schnitte $U_{\alpha_1} \cap \ldots \cap U_{\alpha_n}$ zusammenziehbar sind, so ist die geom. Realisierung des Nerves der Überdeckung homotopieäq. zu $X$.
\end{bem}

% I.2.4
\begin{defn}
  Sei $Y$ ein topologischer Raum. Die simpliziale Menge $X$ der \emph{singulären Simplizes} in $Y$ ist
  \[
    X_n \coloneqq \{ \, \text{stetige Abb. } \sigma : \Delta_n \to Y \, \}, \quad
    X_n(f)(\sigma) \coloneqq \sigma \circ \Delta_f.
  \]
\end{defn}

\begin{bem}
  Diese Konstruktion stiftet eine Funktor $\Sing : \Top \to \sSet$. \\
  % nicht im Buch
  Es besteht die Adjunktion \enspace $\abs{\blank} : \sSet \rightleftarrows \Top : \Sing$.
\end{bem}

% I.2.5
\begin{defn}
  $\Delta[p]_n \coloneqq \{ \, g : [n] \to [p] \text{ monoton steigend} \, \}$, $\Delta[p](f)(g) \coloneqq g \circ f$
\end{defn}

% I.2.8
\begin{defn}
  Der \emph{klassifizierende Raum} einer Gruppe $G$ ist gegeben durch die Realisierung der simpl. Menge $BG$ mit $(BG)_n \coloneqq G^n$ und
  \begin{align*}
    BG(f : [m] \to [n])(g_1, \ldots, g_n) \coloneqq
    (h_1, \ldots, h_m), \quad h_i = \nspace{3} \prod_{j=f(i-1)+1}^{f(i)} \nspace{3} g_j.
    %(g_{f(0)+1} \cdot \ldots \cdot g_{f(1)}, g_{f(1)+1} \cdot \ldots \cdot g_{f(2)}, \ldots, g_{f(m-1)+1} \cdot \ldots \cdot g_{f(m)})
  \end{align*}
\end{defn}

% I.2.9
\begin{defn}
  Ein $n$-Simplex $x \in X_n$ heißt \emph{degeneriert}, falls eine monotone surjektive Abbildung $f : [n] \to [m]$, $n > m$ und ein Element $y \in X_m$ existiert mit $x = X(f)(y)$.
\end{defn}

% I.2.6
\begin{defn}
  Seien $X$ Verklebedaten. Wir konstruieren eine dazugehörende simpliziale Menge $\tilde{X}$ wie folgt:
  \[ \tilde{X}_n \coloneqq \Set{ (x, g) }{ x \in X_{(k)}, g : [n] \to [k] \text{ monoton und surjektiv} }, \]
  Für eine monotone Abbildung $f : [m] \to [n]$ und $(x, g) \in \tilde{X}_n$ schreiben wir zunächst $g \circ f = f_1 \circ f_2$ mit einer Injektion $f_1$ und einer Surjektion $f_2$ und setzen
  $\tilde{X}(f)(x, g) \coloneqq (X(f)(x), f_2)$.
\end{defn}

% I.2.7
\begin{prop}
  Eine simpliziale Menge $\tilde{X}$ kann genau dann aus (dann eindeutigen) Verklebedaten gewonnen werden, falls für alle nicht-degenerierten Simplizes $x \in \tilde{X}_n$ und streng monotonen Abbildungen $f : [m] \to [n]$ auch $\tilde{X}(f)(x) \in \tilde{X}_m$ nicht degeneriert ist.
\end{prop}

% Ausgelassen: I.2.10-12

% I.2.13
\begin{prop}
  Seien $X$ Verklebedaten, $\tilde{X}$ die entsprechende simpliziale Menge. Dann gilt $\abs{X} \approx \abs{\tilde{X}}$.
\end{prop}

% I.2.14
\begin{defn}
  Das \emph{$k$-Skelett} $\sk_k X$ einer simplizialen Menge $X$ ist geg. durch
  \[ (\sk_k X)_n \coloneqq \Set{ X(f)(x) }{ p \leq k, f : [n] \to [p] \text{ monoton}, x \in X_p }. \]
\end{defn}

\begin{defn}
  Eine simpliziale Menge $X$ hat \emph{Dimension} $n$, falls $X = \sk_n X$.
\end{defn}

\begin{prop}
  Geom. Realisierung ist ein Funktor $\abs{\blank} : \sSet \to \Top$.
\end{prop}

\begin{bspe}
  \begin{itemize}
    \item Eine Überdeckung $(U_\alpha)_{\alpha \in A}$ eines topologischen Raumes ist Verfeinerung von $(V_\beta)_{\beta \in B}$, wenn es eine Abbildung $\psi : A \to B$ gibt, sodass $U_\alpha \subset V_{\psi(\alpha)}$ für alle $\alpha \in A$. Dies induziert eine simpliziale Abb. zwischen den Nerven der Überdeckungen durch
    \[ F_n(\alpha_0, \ldots, \alpha_n) \coloneqq (\psi(\alpha_0), \ldots, \psi(\alpha_n)). \]
    % Ausgelassen: I.2.17
    \item Ein Gruppenhomomorphismus $\phi : G \to H$ stiftet eine Abbildung $BG \to BH$ zwischen den klassifizierenden Räumen durch
    \[ F(g_1, \ldots, g_n) \coloneqq (\phi(g_1), \ldots, \phi(g_n)). \]
  \end{itemize}
\end{bspe}

% §1.3 Simpliziale topologische Räume und das Eilenberg-Zilber-Theorem

% I.3.2
\begin{defn}
  Ein \emph{simplizialer topologischer Raum} ist ein Funktor
  \[ X : \Delta^\op \to \Top. \]
\end{defn}

\begin{bem}
  Die geometrische Realisierung eines simplizialen topol. Raumes ist definiert wie die einer simplizialen Menge mit dem Unterschied, dass $X_n$ im Allg. nicht die diskrete Topologie trägt.
\end{bem}

% I.3.3
\begin{defn}
  Eine \emph{bisimpliziale Menge} ist ein Funktor
  \[ X : \Delta^\op \times \Delta^\op \to \SetC. \]
\end{defn}

\begin{nota}
  $X_{nm} \coloneqq X([n],[m])$
\end{nota}

% I.3.4
\begin{bsp}
  Das \emph{direkte Produkt} von simplizialen Mengen $X$ und $Y$ ist die bisimpliziale Menge
  \[
    (X \times Y)_n \coloneqq X_n \times Y_n, \quad
    (X \times Y)(f, g)(x, y) \coloneqq (f(x), g(y)).
  \]
\end{bsp}

% I.3.5
\begin{defn}
  Die \emph{Diagonale} $DX$ einer bisimplizialen Menge $X$ ist die simpliziale Menge mit
  $(DX)_n \coloneqq X_{nn}$ und $DX(f) \coloneqq X(f, f)$.
\end{defn}

% I.3.6
\begin{defn}
  Sei $X$ eine bisimpliziale Menge.
  \begin{itemize}
    \item Setze $\abs{X}^D \coloneqq \abs{DX}$.
    \item Definiere einen simplizialen topologischen Raum $X^I$ durch
    \[ X^I_n \coloneqq \abs{X_{\bullet n}}, \quad X^I(g) \coloneqq \abs{X(\id, g)}. \]
    Setze $\abs{X}^{I,II} \coloneqq \abs{II,I}$.
    \item Definiere analog $\abs{X}^{II,I}$.
  \end{itemize}
\end{defn}

\begin{satz}[\emph{Eilenberg-Zilber}]
  $\abs{X}^D \cong \abs{X}^{I,II} \cong \abs{X}^{II,I}$ kanonisch.
\end{satz}

% II.4.20a)
\begin{defn}
  Der \emph{Nerv} $\NCat$ einer kleinen Kategorie $\Cat$ ist die simpl. Menge
  \begin{align*}
    & \NCat_n \coloneqq \left\{ \, \text{Diagramme } X_0 \xrightarrow{\varphi_1} X_1 \xrightarrow{\varphi_2} \ldots \xrightarrow{\varphi_n} X_n \text{ in $\Cat$} \, \right\} \\
    & \NCat(f : [m] \to [n])(X_0 \xrightarrow{\varphi_1} \ldots \xrightarrow{\varphi_n} X_n) \coloneqq 
    ( Y_0 \xrightarrow{\psi_1} \ldots \xrightarrow{\psi_m} Y_m) \\
    & \quad \text{mit} \enspace Y_i \coloneqq X_{f(i)}, \enspace \psi_i \coloneqq \varphi_{f(i)} \circ \nldots \circ \varphi_{f(i-1)+1}
    %& \left( X_{f(0)} \to \ldots \to X_{f(i)} \xrightarrow{\varphi_{f(i+1)} \ldots \circ \varphi_{f(i)+1}} X_{f(i+1)} \to \ldots \to X_{f(m)} \right)
  \end{align*}
\end{defn}

% II.4.21
\begin{bsp}
 % Der Nerv der Präordnungskat. mit Objekten $\{ 0, \ldots, n \}$ ist $\Delta[n]$.
 $\Delta[n] = \NCat(\text{Präordnungskategorie mit Objekten } \{ 0, \ldots, n \})$
\end{bsp}

% II.4.20b), II.4.22
\begin{bem}
  \begin{itemize}
    \item Der Nerv ist volltreuer Funktor $\NCat : \CatC \to \sSet$.
    \item Jede kleine Kat. kann aus ihrem Nerv zurückgewonnen werden.
    \item $\Nerve(\Cat \times \Dat) = D(\Nerve\Cat \times \Nerve\Dat)$
  \end{itemize}
\end{bem}

\begin{bem}
  Mit $X \!*\! Y \coloneqq D(X \!\times\! Y)$ ist $\sSet$ eine monoidale Kategorie.
\end{bem}

% II.4.24a)
\begin{prop}
  $\sSet$ ist kartesisch abgeschlossen. Dabei ist
  \[ [X, Y]_n = (Y^X)_n \coloneqq \Hom_\sSet(\Delta[n] * X, Y). \]
\end{prop}

% II.4.24b)
\begin{prop}
  Der Nervfunktor ist verträglich mit dem internen Hom: \\
  Seien $\Cat$, $\Dat$ kleine Kategorien. Dann ist
  \[ \Nerve([\Cat, \Dat]_{\CatC}) \!\cong\! [\Nerve\Cat, \Nerve\Dat]_{\sSet}. \]
\end{prop}

\section{Garben}

% I.5 (Garben)

% Ausgelassen: I.5.1 (Beispiele von Garben)

\begin{defn}
  \begin{itemize}
    \item Eine mengenwertige \emph{Prägarbe} $\Fais$ auf einem topol. Raum $X$ ist ein Funktor
    $\Fais : \Ouv(X)^\op \to \SetC$.
    Dabei ist $\Ouv(X)$ die Präordnungs-Kat. der off. Teilmengen von $X$ geordnet durch $\subseteq$.
    \item Allgemeiner ist eine $\Cat$-wertige Prägarbe ein Funktor $\Fais : \Ouv(X)^\op \to \Cat$ (z.\,B. $\Cat = \AbGrp, \RMod, \Top$).
    \item Ein Morphismus zwischen Prägarben $\Fais$ und $\Garb$ auf demselben topol. Raum ist eine natürliche Transformation zwischen $\Fais$ und $\Garb$.
  \end{itemize}
\end{defn}

\begin{nota}
  Sei $\Fais$ eine Prägarbe
  \begin{itemize}
    \item $\Gamma(U, \Fais) \coloneqq \Fais(U)$ heißt Menge der \emph{Schnitte} von $\Fais$ über $U$.
    \item $r_{UV} \coloneqq \Fais(V \subseteq U) : \Fais(U) \to \Fais(V)$ heißt \emph{Restriktionsabb}.
    \item $x|_V \coloneqq r_{UV}(x)$ für $V \subseteq U$ und $x \in \Fais(U)$ heißt \emph{Einschränkung} von $x$ auf $V$.
  \end{itemize}
\end{nota}

\begin{defn}
  Eine \emph{Garbe} auf einem topol. Raum $X$ ist eine Prägarbe $\Fais$, für die gilt:
  Für alle Familien $(U_i)_{i \in I}$ von offenen Teilmengen und Schnitten $(s_i \in \Fais(U_i))_{i \in I}$, die miteinander kompatibel sind, \dh{}
  \[ \fa{i, j \in I} s_i|_{U_i \cap U_j} = s_j|_{U_i \cap U_j}, \]
  gibt es genau einen Schnitt $s \in \Fais(\cup_{i \in I} U_i)$ mit $\fa{i \in I} s_i = s|_{U_i}$.\\
  Ein Morphismus zw. Garben ist ein Morphismen zw. den Prägarben.
\end{defn}

\begin{defn}
  Eine Prägarbe $\Fais$ auf einem topol. Raum $X$ heißt \emph{separiert}, wenn zwei Schnitte $s, t \in \Fais(U)$ auf einer offenen Teilmenge $U \subset X$ genau dann übereinstimmen, wenn sie lokal übereinstimmen, \dh{}
  \[ s = t \iff \fa{x \!\in\! U\!}\! \ex{V_x \subset U \text{ offene Umgebung von $x$}\!}\! s|_{V_x} = t|_{V_x}. \]
\end{defn}

\begin{bem}
  Das entspricht dem Eindeutigkeitsteil im Garbenaxiom.
\end{bem}

\begin{bem}
  Sei $\Fais$ eine (Prä-)Garbe auf $X$ und $U \subseteq X$ offen. Dann definiert $(\Fais|U)(V) \coloneqq \Fais(U \cap V)$ eine (Prä-)Garbe auf $U$.
\end{bem}

% Notation abgeguckt von http://stacks.math.columbia.edu/download/sheaves.pdf
\begin{nota}
  Die Kategorien der Prägarben von Mengen bzw. von abelschen Gruppen auf einem topol. Raum $X$ werden bezeichnet mit
  \[
    \PShSet(X) \coloneqq [\Ouv(X)^\op, \SetC]
    \quad \text{bzw.} \quad
    \PShAb(X) \coloneqq [\Ouv(X)^\op, \AbGrp].
  \]
  Die volle Unterkategorie der Garben ist
  \[
    \ShSet(X) \subset \PShSet(X)
    \quad \text{bzw.} \quad
    \ShAb(X) \subset \PShAb(X).
  \]
\end{nota}

% The Geometry of schemes, proposition I-12
\begin{lem}
  Sei $\mathcal{B}$ eine Basis der Topologie von $X$.
  \begin{itemize}
    \item Sei $(\Fais(U))_{U \in \mathcal{B}}$ eine Familie von Mengen bzw. ab. Gruppen und $r_{UV} : \Fais(U) \to \Fais(V)$ Einschränkungsabb. mit $r_{VW} \circ r_{UV} = r_{UW}$ für alle $U, V, W \in \mathcal{B}$ mit $W \subseteq V \subseteq U$.
    Angenommen, für alle Familien $(s_U \in \Fais(U))_{U \in \mathcal{C}}$ mit $\mathcal{C} \subset \mathcal{B}$ und $A \coloneqq {\bigcup}_{U \in \mathcal{C}} U \in \mathcal{B}$, sodass $r_{UW}(s_U) = r_{VW}(s_V)$ für alle $W \in \mathcal{B}$, $W \subset U \cap V$, gibt es genau ein $s_A \in \Fais(A)$ mit $r_{AU}(s_A) = s_U$ für alle $U \in \mathcal{C}$. \\
    Dann lässt sich $\Fais$ eindeutig zu einer Garbe auf $X$ fortsetzen.
    \item Seien $\Fais$ und $\Garb$ Garben auf $X$ und $(\varphi_U : \Fais(U) \to \Garb(U))_{U \in \mathcal{B}}$ eine Fam. von mit Einschränkung vertr. Abbildungen. Dann gibt es genau einen Garbenmorphismus $\varphi : \Fais \to \Garb$ mit $\varphi(U) = \varphi_U$.
  \end{itemize}
\end{lem}

% I.5.3 (Prägarben und Garben von strukturierten Mengen)

% Ausgelassen: Garbe von $O(U)$-Modulen, wobei $O$ selbst eine Ringprägarbe auf $X$ ist.

\begin{defn}
  Eine Sequenz $\Fais \to \Garb \to \Harb$ von (Prä-)Garben abelscher Gruppen auf $X$ heißt \emph{exakt} bei $\Garb$, falls für alle offenen $U \subset X$ die Sequenz $\Fais(U) \to \Garb(U) \to \Harb(U)$ exakt bei $\Garb(U)$ ist.
\end{defn}

\begin{defn}
  Sei $f : \Fais \to \Garb$ ein Morphismus von Prägarben auf $X$. Definiere Prägarben $\mathcal{K}$ und $\mathcal{C}$ auf $X$ durch
  \[
    \mathcal{K}(U) \coloneqq \ker (f_U : \Fais(U) \to \Garb(U)), \quad
    \mathcal{C}(U) \coloneqq \Garb(U) / \im (f_U).
  \]
\end{defn}

\begin{prop}
  Sei $f : \Fais \to \Garb$ sogar ein Morphismus von Garben. \\
  Dann ist auch $\mathcal{K}$ eine Garbe.
\end{prop}

\begin{acht}
  Aber $\mathcal{C}$ ist im Allgemeinen keine Garbe!
\end{acht}

% I.5.5 (Keime und Halme)

\begin{defn}
  Sei $\Fais$ eine Garbe auf $X$. Der \emph{Halm} von $\Fais$ in $x \in X$ ist
  \begin{align*}
    \Fais_y & \coloneqq \Set{(U, s)}{U \subseteq X \text{offen}, x \in U, s \in \Fais(U)} / {\sim}, \\
    (U, s) \sim (V, t) & \coloniff \ex{W \subset U \cap V \text{ offen}, y \in W} s|_W = t|_W.
  \end{align*}
\end{defn}

\begin{nota}
  $s_x \coloneqq [(U, s)]$ für $s \in \Fais(U)$ mit $x \in U$.
\end{nota}

\begin{sprech}
  Elemente $[t] \in \Fais_x$ heißen \emph{Keime} in $x$.
\end{sprech}

\begin{defn}
  Sei $\Fais$ eine Garbe auf $X$, $Z \subseteq X$ beliebig. Definiere
  \[ \Gamma(Z, \Fais) \coloneqq \Colim \Gamma(U, \Fais), \]
  wobei der Limes über alle offenen $U \subset X$ mit $Z \subseteq U$ läuft.
\end{defn}

\begin{beob}
  $\Fais_x = \Gamma(\{x\}, \Fais)$
\end{beob}

\begin{defn}
  Der \emph{Totalraum} $F$ einer Prägarbe $\Fais$ auf $X$ ist
  \[ F \coloneqq \coprod_{x \in X} \Fais_x \]
  mit der Topologie erzeugt durch die Mengen
  \[
    \Set{s_x}{ x \in U } \quad
    \text{für $U \subseteq X$ offen, $s \in \Fais(U)$.}
  \]
\end{defn}

\begin{bem}
  Mit dieser Topologie ist die Projektion $\pi : F \to X$ stetig und ein lokaler Homöomorphismus.
\end{bem}

\begin{defn}
  Sei $\Fais$ eine Prägarbe von Mengen auf $Y$. Die \emph{Garbifizierung} $\Fais^+$ von $\Fais$ ist die Garbe der stetigen Schnitte von $\pi : F \to X$, also
  \[ \Fais^+(U) \coloneqq \Set{f : U \to F}{\pi \circ f = (i : U \hookrightarrow X)}. \]
\end{defn}

\begin{bem}
  Garbifizierung ist ein Funktor $s : \PShSet(X) \!\to\! \ShSet(X), \enspace \Fais \!\mapsto\! \Fais^+$.
\end{bem}

\begin{prop}
  Es ex. ein kanonischer Morphismus $f : \Fais \to \Fais^+$ def. durch
  \[ s \in \Fais(U) \enspace \mapsto \enspace (x \mapsto s_x : U \to F). \]
  Wenn $\Fais$ schon eine Garbe ist, dann ist $f$ ein Isomorphismus.
\end{prop}

\begin{defn}
  Sei $\Fais$ eine Prägarbe auf einem topologischen Raum $X$. \\
  Eine Familie $(s_i \in \Fais(U_i))_{i \in I}$ von Schnitten auf offenen Teilmengen von $X$ heißt \emph{lokal kompatibel}, falls für alle $i, j \in I$ und $x \in U_i \cap U_j$ eine Umgebung $V \subset U_i \cap U_j$ von $x$ mit $s_i|_V = s_j|_V$ existiert.
\end{defn}

\begin{defn}
  Der \emph{Garbifizierungsfunktor} $s : \PShAb(X) \!\to\! \ShAb(X), \, \Fais \!\mapsto\! \Fais^+$ ist def. auf Prägarben abelscher Gruppen $\Fais$ und $U \subset X$ offen durch
  \begin{align*}
    s(\Fais)(U) \coloneqq \{ \, & \text{Familien $(s_i \in \Fais(U_i))_{i \in I}$ von lokal kompatiblen} \\[-2pt]
     & \text{Schnitten auf offenen Teilmengen mit $U = {\bigcup}_{i \in I} \, U_i$} \, \} / {\sim}
  \end{align*}\vspace{-16pt}
  \begin{align*}
    (s_i)_{i \in I} \sim (t_j)_{j \in J} \!\! \coloniff \, & \ex{\text{offene Überdeckung $(W_k)_{k \in K}$ von U}} \\
    & \fa{i, j, k\!}\! s_i|(U_i \!\cap\! V_j \!\cap\! W_k) = t_j|(U_i \!\cap\! V_j \!\cap\! W_k).
  \end{align*}
\end{defn}

% II.5.13
\begin{thm}
  Es besteht die Adjunktion \enspace $s : \PShAb(X) \rightleftarrows \ShAb(X) : i$. \\
  Die Koeinheit $\epsilon : s \circ i \to \Id_{\ShAb(X)}$ ist ein Isomorphismus.
\end{thm}

% Übungsblatt 16, Aufgabe 1
\begin{bem}[Universelle Eigenschaft der Garbifizierung]
  Sei $\alpha : \Fais \to \Garb$ ein Morphismus von Prägarben. Sei $\Garb$ sogar eine Garbe. Dann gibt es einen einen eindeutigen Morphismus $\alpha^+ : \Fais^+ \to \Garb$ mit $\alpha = \alpha^+ \circ \eta_\Fais$.
\end{bem}

% Ausgelassen: I.5.7 (Hauptklassen von Garben)

% I.5.8
\begin{defn}
  Sei $A$ eine Menge (oder ab. Gruppe, \ldots), $X$ ein topol. Raum.
  \begin{itemize}
    \item Die \emph{konstante Prägarbe} $\mathbf{A}$ mit Faser $A$ auf $X$ ist def. durch
    \[
      \mathbf{A}(U) \coloneqq A, \quad
      r_{UV} \coloneqq \id_A \quad
      \text{für alle $V \subseteq U \subseteq X$.}
    \]
    \item Die \emph{konstante Garbe} mit Faser $A$ ist die Garbifizierung $\constSh{A} \coloneqq A_X \coloneqq \mathbf{A}^+$ von $\mathbf{A}$.
  \end{itemize}
\end{defn}

% I.5.9
\begin{defn}
  Eine Garbe $\Fais$ auf $X$ heißt \emph{lokal konstant}, falls jeder Punkt in $X$ eine offene Umgebung $U$ besitzt, sodass $F|_U$ isomorph zu einer konstanten Garbe ist.
\end{defn}

\begin{defn}
  Seien $\Fais, \Garb$ (Prä-)Garben auf $X$. Dann ist auch
  \[ \Hom(\Fais, \Garb) : U \mapsto \Hom(\Fais|_U, \Garb|_U) \]
  eine (Prä-)Garbe auf $X$, die sogenannte \emph{Hom-(Prä-)Garbe}.
\end{defn}

\begin{defn}
   Eine Garbe $\Fais$ auf einem topologischen Raum $X$ heißt \ldots{}
  \begin{itemize}
    \item \ldots{} \emph{welk} (flabby, flasque), wenn die Einschränkungsabbildungen
    \[ \Gamma(X, \Fais) \to \Gamma(U, \Fais) \]
    für alle {\em offenenen} $U \subseteq X$ surjektiv sind.
    \item \ldots{} \emph{weich} (soft, mou), wenn die Einschränkungsabbildungen
    \[ \Gamma(X, \Fais) \to \Gamma(A, \Fais) \]
    für alle {\em abgeschlossenen} $A \subseteq X$ surjektiv sind.
  \end{itemize}
\end{defn}

% Übungsblatt 6, Aufgabe 6b)
\begin{lem}
  Welke Garben sind immer auch weich.
\end{lem}

\begin{defn}
   Eine Garbe $\Fais$ ab. Gruppen auf einem topol. Raum $X$ heißt \emph{fein} (fine, fin), wenn für je zwei disjunkte, abgeschlossene Teilmengen $A_1, A_2 \subseteq X$ ein Garbenmorphismus $\alpha : \Fais \to \Fais$ existiert, sodass $\alpha$ auf einer offenen Umgebung von $A_1$ Null und auf einer offenen Umgebung von $A_2$ die Identität ist.
\end{defn}

\begin{lem}
  \begin{itemize}
    \item Eine Garbe $\Fais$ ab. Gruppen auf einem parakompakten Hausdorffraum ist genau dann fein, wenn $\Hom(\Fais, \Fais)$ welk ist.
    \item Feine Garben auf parakompakten Hausdorffräumen sind weich.
  \end{itemize}
\end{lem}

\begin{samepage}
  \subsection{(Lokal) geringte Räume}
\end{samepage}

% II.4 Ein kategorieller Ansatz für die Konstruktion geometrischer Objekte

% II.4.1 (Drei geometrische Kategorien)
% (hier nur zwei)
\begin{defn}
  Eine topol. / glatte \emph{Mannigfaltigkeit} ist ein Paar $(M, \O_M)$, wobei $M$ ein Hausdorffraum und $\O_M$ eine Garbe auf $M$ ist, sodass jeder Punkt $x \in M$ eine offene Umgebung $U$ besitzt, sodass $\O_M|U$ isomorph zu einer Garbe von stetigen / glatten Funktionen auf einer offenen Teilmenge des $\R^n$ ist.
\end{defn}

% II.4.3 (Morphismen)
\begin{defn}
  Ein Morphismus $\Phi : (M, \O_M) \to (N, \O_N)$ zwischen topol. / glatten Mften. ist geg. durch eine stetige Abb. $\phi : M \to N$, sodass
  %für alle offenen $U \subset N$ und $f \in \Gamma(U, \O_N)$ auch $f \circ \phi \in \Gamma(\phi^{-1}(U), \O_M)$ ist.
  \[
    \fa{U \subseteq N \text{ offen}} \fa{f \in \Gamma(U, \O_N)} f \circ \phi \in \Gamma(\phi^{-1}(U), \O_M).
  \]
\end{defn}

% II.4.2 (Atlanten)
\begin{bem}
  Diese Definitionen sind äquivalent zu den üblichen Alte-Römer-Definitionen von Mannigfaltigkeiten über Atlanten.
\end{bem}

% II.4.5
\begin{defn}
  Ein \emph{geringter Raum} ist ein Paar $(M, \O_M)$, wobei $M$ ein topol. Raum und $\O_M$ eine Ringgarbe auf $M$ ist. \\
  Ein Morphismus $\Phi : (M, \O_M) \to (N, \O_N)$ zwischen geringten Räumen ist ein Paar $(\varphi, \theta)$, wobei $\varphi : M \to N$ stetig und
  \[ \theta = (\theta_U : \Gamma(U, \O_N) \to \Gamma(\varphi^{-1}(U), \O_M))_{U \opn N \text{ offen}} \]
  eine Familie von Ringhomomorphismen ist, sodass
  \[ \fa{V \opn U \opn N} \theta_U(\blank)|_{\varphi^{-1}(V)} = \theta_V(\blank|_V). \]
\end{defn}

\begin{sprech}
  $\O_M$ heißt \emph{Strukturgarbe}.
\end{sprech}

% TODO: Pushforward (?) von Garben vor dieser Stelle definieren!

\begin{bem}
  Man kann $\theta$ als Garbenmorph. $\theta : \O_N \to \varphi_\bullet(\O_M)$ auffassen.
\end{bem}

\begin{defn}
  Sei $(M, \O_M)$ ein geringter Raum. Eine \emph{(Prä-)Garbe von $\O_M$-Moduln} ist eine mengenwertige (Prä-)Garbe $\Fais$ auf $M$, sodass $\Fais(U)$ ein $\O_M(U)$-Modul für alle offenen $U \subset M$ ist. Desweiteren soll die Skalarmultiplikation mit Einschränkungen verträglich sein:
  \[ \fa{V \opn U \opn M} \fa{z \in \O_M(U), \, r \in \Fais(U)} (z \cdot r)|_V = (z|_V) \cdot (r|_V). \]
\end{defn}

\begin{bsp}
  Sei $(M, \O_M)$ eine glatte Mft. Sei $\mathcal{D}_M$ die Garbe der linearen, derivativen Operatoren, also
  \[ \mathcal{D}_M(U) \!\coloneqq\! \{ \, P \!:\! \O_M(U) \!\to\! \O_M(U), \, P \!=\! \sum f_I(z) \tfrac{\partial^I}{\partial z^I} \text{ in lok. Koord.} \, \}. \]
  Dann ist $(M, \mathcal{D}_M)$ ein geringter Raum. % TODO: ist er lokal geringt?
\end{bsp}

% II.4.4 (Strukturgarben vs. Garben von Funktionen)

% Ausgelassen: II.4.4 a) ("Super-regions in $\R^{m|n}$")

% II.4.4 b) (Affine Schemata)
% XXX: In eine noch zu erstellende Algebra/AG-Zusammenfassung auslagern?
\begin{defn}
  Sei $A$ ein komm. Ring. Das \emph{Spektrum} von $A$ ist
  \[ \Spec(A) \coloneqq \{ \, \text{Primideale $\mathfrak{p} \subsetneq A$} \, \} \]
  mit der sogenannten \emph{Zariski-Topologie} mit offenen Mengen
  \[
    \Tau \coloneqq \Set{D(S)}{S \subseteq A} \subset \Pow(\Spec(A)), \enspace
    D(S) \coloneqq \Set{\mathfrak{p} \in \Spec(A)}{S \not\subseteq \mathfrak{p}}.
  \]
  Die abgeschlossenen Mengen sind von der Form $V(S)$ für $S \subseteq A$ mit
  \[
    V(S) \coloneqq \Spec(A) \setminus D(S) = \Set{\mathfrak{p} \in \Spec(A)}{S \subseteq \mathfrak{p}}.
  \]
  Für $U \subseteq \Spec(A)$ offen sei $\Delta(U)$ das Komplement der Vereinigung der Ideale in $U$.
  Da $\Delta(U)$ multiplikativ abgeschlossen ist und $V \subseteq U \!\implies\! \Delta(V) \subseteq \Delta(U)$ gilt, können wir eine Prägarbe komm. Ringe $\O'$ auf $\Spec(A)$ wie folgt definieren:
  \[
    \O'(U) \coloneqq (\Delta(U))^{-1} A, \quad
    r_{UV}(\left[\tfrac{s}{t}\right]) \coloneqq \left[\tfrac{s}{t}\right].
  \]
  Sei $\O \coloneqq \O_{\Spec(A)} \coloneqq (\O')^+$ die Garbifizierung von $\O'$. \\
  Der geringte Raum $(\Spec(A), \O)$ heißt \emph{affines Schema} von $A$.
\end{defn}

\begin{bem}
  Sei $A$ ein Integritätsbereich. Dann ist das Nullideal $(0) \in \Spec(A)$ ein generischer Punkt, \dh{} $\clos{(0)} = \Spec(A)$.
\end{bem}

% Nicht im Buch
\begin{lem}
  $(\O_{\Spec(A)})_{\mathfrak{p}} \cong A_{\mathfrak{p}} \coloneqq \Delta(\mathfrak{p})^{-1} A$ für alle $\mathfrak{p} \in \Spec(A)$
\end{lem}

% TODO: Morphismen zwischen Schemata

% XXX: In eine noch zu erstellende Algebra-Zusammenfassung auslagern?
\begin{defn}
  Ein Ring $R$ heißt \emph{lokal}, wenn er eine der folgenden äquivalenten Bedingungen erfüllt:
  \begin{enumerate}
    \item Er besitzt genau ein maximales \enspace a) Linksideal \enspace b) Rechtsideal.
    %\item $\sum_{i=0}^n x_i$ ist eine Einheit $\implies$ ein $x_i$ ist eine Einheit.
    \item Wenn eine endliche (evtl. leere) Summe von Ringelementen eine Einheit ist, dann auch einer der Summanden (insb. gilt $0 \not= 1$).
    \item $\Spec(R)$ hat genau einen abgeschl. Punkt (das maximale Ideal).
  \end{enumerate}
\end{defn}

\begin{bem}
  In lok. Ringen stimmen max. Links- und Rechtsideal überein.
\end{bem}

\begin{defn}
  Seien $R$ und $S$ lokale Ringe mit max. Idealen $\mathfrak{m} \!\subset\! R$ und $\mathfrak{n} \!\subset\! S$. \\
  Ein \emph{lokaler Ringhomomorphismus} zwischen $R$ und $S$ ist ein Ringhomomorphismus $f : R \to S$ mit $f(\mathfrak{m}) \subseteq \mathfrak{n}$.
\end{defn}

% II.4.6
\begin{defn}
  Ein geringter Raum $(M, \O_M)$ heißt \emph{lokal geringt}, wenn die Fasern $\O_{M,x}$ für alle $x \in M$ lokale Ringe sind. \\
  Ein Morphismus $\Phi = (\varphi, \theta) \!:\! (M, \O_M) \to (N, \O_N)$ heißt \emph{Morph. von lokal geringten Räumen}, wenn für alle $x \in M$ die ind. Abb.
  \[ \theta_x : \O_{N,y} \to \O_{M,x} \]
  ein lokaler Homomorphismus von lokalen Ringen ist.
\end{defn}

% II.4.7 (Bemerkungen)
\begin{bspe}
  Affine Schemata und Mften sind lokal geringte Räume.
  % XXX: Gegenbeispiel
\end{bspe}

% Ausgelassen: II.4.8 (Superkommutativität)
% Ausgelassen: II.4.9 (Definition: lokal geringte Superräume)

% II.4.10
\begin{defn}
  Ein \emph{Schema} ist ein lokal geringter Raum $(S, \O_S)$, der lokal isomorph zum affinen Spektrum eines Ringes ist, \dh{} jedes $x \in S$ besitzt eine offene Umgebung $U$, sodass $(U, \O_S|U)$ als lokal geringter Raum isomorph zum affinen Schema eines komm. Ringes ist.
\end{defn}

% Ausgelassen: II.4.11 (Definition: Differentierbare Supermannigfaltigkeiten)

% Weiter vor gezogen
\subsection{Direktes und inverses Bild}

% II.6.15
\begin{defn}
  Sei $f : X \to Y$ eine stetige Abbildung zwischen topologischen Räumen.
  Das \emph{direkte Bild} (oder Pushforward) einer (Prä-)Garbe $\Fais$ auf $X$ ist die (Prä-)Garbe $f_\bullet(\Fais)$ auf $Y$ mit
  \[
    f_\bullet(\Fais)(U) \coloneqq \Fais(f^{-1}(U)) \quad
    \text{für $U \subset Y$ offen}.
  \]
\end{defn}

% II.6.16, ausgelassen b)
\begin{bem}
  \begin{itemize}
    \item $f_\bullet$ ist ein Funktor $f_\bullet : \MShSet(X) \to \MShSet(Y)$.
    \item $(\blank)_\bullet$ ist selbst funktoriell: $f_\bullet \circ g_\bullet = (f \circ g)_\bullet$, $\id_\bullet = \Id$.
    \item Die Konstruktion funktioniert für (Prä-)Garben von Mengen, ab. Gruppen, $A$-Linksmoduln und auch für $\O_X$-Modulgarben, wenn $f : (X, \O_X) \!\to\! (Y, \O_Y)$ ein Morphismus von geringten Räumen ist.
  \end{itemize}
\end{bem}

% TODO: funktioniert das inverse Bild auch für Prägarben? Gilt dann eine Adjunktionsbeziehung?

% II.6.17
\begin{defn}
  Sei $f : X \to Y$ eine stetige Abbildung zwischen topol. Räumen. \\
  Das \emph{inverse Bild} (Pullback) ist der Funktor $f^\bullet : \ShSet(Y) \to \ShSet(X)$, der für Garben $\Fais \in \ShSet(Y)$ und $U \subset X$ offen definiert ist durch
  \begin{align*}
    f^\bullet(\Fais)(U) \coloneqq \{ \, & \text{Familien $(s_x \in \Fais_{f(x)})_{x \in U}$ von Keimen, für die gilt:} \\[-2pt]
    & \text{Für alle $x \in X$ gibt es eine Umgebung $V \subset X$ von $f(x)$,} \\[-2pt]
    & \text{eine Umgebung $W \subset U \cap f^{-1}(V)$ von $x$ und einen} \\[-2pt]
    & \text{Schnitt $t \in \Fais(V)$ mit $\fa{x \in W} t_x = s_x$} \, \}.
  \end{align*}
\end{defn}

% II.6.17
\begin{prop}
  Sei $f : X \to Y$ eine stetige Abb. zwischen topol. Räumen. \\
  Dann besteht die Adjunktion \enspace $f^\bullet : \ShSet(Y) \rightleftarrows \ShSet(X) : f_\bullet$.
\end{prop}

% II.6.18 (Eigenschaften des inversen Bildes)
\begin{bem}
  \begin{itemize}
    \item $(\blank)^\bullet$ ist kontravar. funktoriell: $g^\bullet \circ f^\bullet = (f \circ g)^\bullet$, $\id^\bullet = \Id$.
    \item Alternative Definition: $f^\bullet(\Fais) \coloneqq (U \mapsto \Gamma(f(U), \Fais))^+$
    % Ausgelassen: Zweite alternative Definition
    \item Es gilt $f^\bullet(\Fais)_x = \Fais_{f(x)}$ für alle $x \in X$.
    \item Die Konstruktion funktioniert für (Prä-)Garben von Mengen, ab. Gruppen, $A$-Linksmoduln, aber {\em nicht} für $\O_X$-Modulgarben, wenn $f : (X, \O_X) \!\to\! (Y, \O_Y)$ ein Morphismus von geringten Räumen ist.
    \item Sei $f : X \to \pt$ und $\Fais$ die Garbe auf $\pt$ mit $\Fais(\pt) = A$. \\
    Dann ist $f^\bullet(A) \cong \constSh{A}$ die konstante Garbe auf $X$ mit Faser $A$.
  \end{itemize}
\end{bem}

% TODO: gilt $f^\bullet = (f^{-1})_\bullet$ für Homöomorphismen $f$?
% TODO: Konstruktion von $f^*$ für Modulgarben verstehen, aufschreiben


% II.4.12 (Wie können wir Topologie in der Sprache der Kategorien beschreiben?)

\begin{samepage}

\subsection{Garben auf Siten}

\end{samepage}

% II.4.13 (Teile ausgelassen)
\begin{defn}
  Sei $\Sit$ eine Kategorie. Ein \emph{Sieb} auf $U \in \Ob(\Sit)$ ist eine Menge $\Phi = \Set{\varphi_i \in \Hom_\Sit(U_i, U)}{i \in I}$ von Morphismen nach $U$, sodass gilt:
  \[ \fa{V \in \Ob(\Sit), \, i \in I, \, \psi \in \Hom_\Sit(V, U_i)} \varphi_i \circ \psi \in \Phi. \]
\end{defn}

\begin{bem}
  Sei $\Phi$ ein Sieb auf $U$, $f \in \Hom(V, U)$. Dann ist
  \[ f^*(\Phi) \coloneqq \Set{\varphi \in \Hom_\Sit(W, V)}{W \in \Ob(\Sit), f \circ \varphi \in \Phi} \]
  ein Sieb auf $V$, die \emph{Einschränkung} von $\Phi$ auf $V$ (über $f$).
\end{bem}

% II.4.14
\begin{defn}
  Eine \emph{Grothendieck-Topologie} auf einer Kategorie $\Sit$ ist gegeben durch eine Menge $C(U)$ von Sieben auf $U$ für jedes $U \in \Ob(\Sit)$ (den sogenannten \emph{überdeckenden Sieben}), sodass gilt:
  \begin{itemize}
    \item Für alle $U \in \Ob(\Sit)$ ist das Sieb aller Abb. nach $U$ in $C(U)$.
    \item Die Einschränkung $f^*(\Phi)$ eines Siebes $\Phi \in C(U)$ über $f \in \Hom_\Sit(V, U)$ ist in $C(U)$.
    \item Die Überdeckungseigenschaft lässt sich lokal testen: \\
    Für $\Psi$ ein bel. Sieb auf $U$ und $\Phi \in C(U)$ überdeckend. Angenom- men, für alle $(\varphi_i : U_i \to U) \in \Phi$ ist die Einschränkung von $\Psi$ über $\varphi_i$ überdeckend, also $\varphi_i^*(\Psi) \in C(U_i)$. Dann ist auch $\Psi \in C(U)$.
  \end{itemize}
\end{defn}

% II.4.15 (Garben)

\begin{defn}
  Ein \emph{Situs} ist eine Kategorie $\Sit$ mit Grothendieck-Topologie.
\end{defn}

\begin{defn}
  Sei $X$ ein topol. Raum. Dann ist $\Ouv(X)$ ein Situs mit
  \[ C(U) \coloneqq \{ \, \text{von offenen Überdeckungen von $U$ erzeugte Siebe} \, \}. \]
\end{defn}

\begin{defn}
  \begin{itemize}
    %\item Eine \emph{Prägarbe} von Mengen auf einem Situs $\Sit$ ist ein kontravarianter Funktor $\Garb : \Sit^\op \to \SetC$.
    \item Eine $\mathcal{C}$-wertige Prägarbe ist ein Funktor $\Fais : \Sit^\op \to \Cat$ \\
    (z.\,B. $\Cat = \SetC, \AbGrp, \RMod, \Top$).
    \item Ein Morphismus zwischen Prägarben $\Fais$ und $\Garb$ auf demselben Situs ist eine natürliche Transformation zwischen $\Fais$ und $\Garb$.
  \end{itemize}
\end{defn}

\begin{defn}
  Eine \emph{Garbe} auf einem Situs $\Sit$ ist eine Prägarbe $\Fais$, für die gilt:
  Für alle überdeckenden Siebe $\Phi \in C(U)$ und Familien von Schnitten $(s_\varphi \in \Fais(V))_{(\varphi : V \to U) \in \Phi}$, die miteinander vertr. sind, \dh{}
  \[ \fa{(\varphi : V \to U) \in \Phi} \fa{\psi : W \to V} s_{\varphi \circ \psi} = s_\varphi|_W \coloneqq \Fais(\psi)(s_\varphi), \]
  gibt es genau einen Schnitt $s \in \Fais(U)$ mit
  \[ \fa{(\varphi : V \to U) \in \Phi} s_\varphi = s|_V \coloneqq \Fais(\varphi)(s). \]
  Ein Morphismus zw. Garben ist ein Morphismen zw. den Prägarben.
\end{defn}

\begin{bem}
  Die Notationen und Sprechw. für (Prä-)Garben auf topol. Räumen werden auch für Garben auf Siten verwendet.
  Man notiert auch $s|_V \coloneqq \Garb(f)(s)$ für die Einschränkung eines Schnittes $s \in \Garb(U)$ über $f \in \Hom_\Sit(V, U)$, wohlwissend, dass sie auch von $f$ abhängt.
\end{bem}

% II.4.16
\begin{bsp}
  Sei $G$ eine Gruppe und $\Sit_G$ die Kategorie der Mengen mit $G$-Wirkung und äquivarianten Abbildungen. Man nennt ein Sieb $\Phi$ über $U \in \Ob(\Sit_G)$ überdeckend, wenn
  $U = \cup_{(\varphi : V \to U) \in \Phi} \,\, \varphi(V)$. \\
  %\[ U = \bigcup_{(\varphi : V \to U) \in \Phi} \varphi(V). \]
  Sei $\ShSet_G$ die Kategorie der Garben von Mengen auf dem Situs $\Sit_G$. \\
  Sei $G_l \coloneqq G \in \Ob(\Sit_G)$ mit der Linkswirkung $g.h \coloneqq gh$. \\
  Es gibt einen Funktor $\alpha : \ShSet_G \to \Sit_G$ mit $\alpha(F) \coloneqq F(G_l) \in \Ob(\Sit_G)$, wobei $G$ auf $F(G_l)$ durch $g.x \coloneqq F(h \mapsto hg)(x)$ für $x \in F(G_l)$ wirkt.
\end{bsp}

% II.4.17
\begin{prop}
  $\alpha : \ShSet_G \to \Sit_G$ ist eine Kategorienäquivalenz.
\end{prop}

% Ausgelassen: II.4.18 (Lemma)
% Ausgelassen: II.4.19 (Bemerkungen)


% I.4 Homologie und Kohomologie
% (gleich auf abelsche Kategorien verallgemeinert)

\section{Komplexe und (Ko-)Homologie}

Sei $\Aat$ eine abelsche Kategorie.

% I.4.3
\begin{defn}
  \begin{itemize}
    \item Ein \emph{Kettenkomplex} $\CC{C}$ ist eine Folge $(C_n)_{n \in \N}$ von Obj. aus $\Aat$ und Morphismen $\partial_n : C_n \to C_{n-1}$ mit $\partial_{n-1} \circ \partial_n = 0$.
    \item Ein \emph{Kokettenkomplex} $\CCC{C}$ ist eine Folge $(C^n)_{n \in \N}$ von Objekten aus $\Aat$ und Morphismen $\delta^n : C^n \to C^{n+1}$ mit $\delta^{n+1} \circ \delta^n = 0$.
  \end{itemize}
\end{defn}

% I.4.4
\begin{defn}
  Sei $\CC{C}$ ein Kettenkomplex.
  \begin{itemize}
    \item $C_n$ heißt Objekt der \emph{$n$-Ketten},
    \item $\partial_n : C_n \to C_{n-1}$ heißt \emph{Randabbildung} oder \emph{Differential},
    \item $Z_n(\CC{C}) \coloneqq \ker \partial_n \hookrightarrow C_n(\CC{C})$ heißt Objekt der \emph{$n$-Zykel},
    \item $B_n(\CC{C}) \coloneqq \im \partial_{n+1} \hookrightarrow Z_n(\CC{C})$ heißt Objekt der \emph{$n$-Ränder}.
    %\item $H_n(\CC{C}) \coloneqq Z_n(\CC{C}) / B_n(\CC{C})$ heißt \emph{$n$-te Homologiegruppe}.
  \end{itemize}
  Analog nennt man für einen Kokettenkomplex $\CCC{C}$
  \begin{itemize}
    \miniitem{0.46 \linewidth}{$\delta^n$ \emph{Korandabbildung},}
    \miniitem{0.41 \linewidth}{$C^n$ \emph{$n$-Koketten},}
    \miniitem{0.46 \linewidth}{$Z^n \coloneqq \ker \delta^n$ \emph{$n$-Kozykel},}
    \miniitem{0.45 \linewidth}{$B^n \coloneqq \im \delta^{n-1}$ \emph{$n$-Koränder}.}
    %\item $H^n(\CCC{C}) \coloneqq Z^n(\CCC{C}) / B^n(\CCC{C})$ $n$-te \emph{Kohomologiegruppe}.
  \end{itemize}
\end{defn}


% II.6.3: Äquivalenz zweier Definitionen von $H^n(\CC{C})$
% Dies sollte das letzte Lemma vor II.6.2 bleiben!
\begin{lem}
  Sei $X \xrightarrow{f} Y \xrightarrow{g} Z \in \Aat$ mit $g \circ f = 0$. Sei $(K, k)$ der Kern von $g$ und $(C, c)$ der Kokern von $f$. Deren universelle Eigenschaften induzieren Morphismen $a : X \to K$ und $b : C \to Z$ wie folgt:
  \vspace{-8pt}
  \begin{centertikz}
    \matrix (mat) [matrix of nodes, column sep=1cm, row sep=0.45cm]{
      & \node (C) {$C$}; \\
      \node (X) {$X$}; &
      \node (Y) {$Y$}; &
      \node (Z) {$Z$}; \\
      & \node (K) {$K$}; \\
    };
    \draw[->] (C) to node [above] {$b$} (Z);
    \draw[->] (Y) to node [right] {$c$} (C);
    \draw[->] (X) to node [above] {$f$} (Y);
    \draw[->] (Y) to node [above] {$g$} (Z);
    \draw[->] (X) to node [below] {$a$} (K);
    \draw[->] (K) to node [right] {$k$} (Y);
  \end{centertikz}
  \vspace{-10pt}
  Dann gibt es einen kanonischen Isomorphismus $\coker a \cong \ker b$.
\end{lem}

% II.6.2
\begin{defn}
  Die \emph{$n$-te Homologie} $H_n(\CC{C})$ eines Kettenkomplexes $\CC{C}$ aus $\Aat$ ist der Kokern der Abbildung $a_n : C_{n+1} \to \ker \partial_n$, die durch die universelle Eigenschaft des Kerns induziert wird. \\
\end{defn}

\begin{bem}
  \begin{itemize}
    \item Das letzte Lemma besagt, dass $H_n(\CC{C})$ isomorph zum Kern der Abbildung $b_n : \coker \partial_{n+1} \to C_{n-1}$, die durch die universelle Eigenschaft des Kokerns induziert wird, ist.
    \item Äquivalent ist $H_n(\CC{C})$ der Kokern der induzierten Abbildung $B_n(\CC{C}) \hookrightarrow Z_n(\CC{C})$, kurz geschrieben $H_n(\CC{C}) \cong Z_n(\CC{C}) / B_n(\CC{C})$.
  \end{itemize}
\end{bem}

\begin{defn}
  Analog ist die \emph{$n$-te Kohomologie} eines Kokomplexes $\CCC{C}$
  \begin{align*}
  H^n(\CCC{C}) \coloneqq & \coker (a^n : C^{n-1} \to \ker \delta^n) \\
  \cong & \ker (b^n : \coker \delta^{n-1} \to C^{n+1}) \cong Z^n / B^n.
  \end{align*}
\end{defn}

% II.6.2
\begin{defn}
  Ein (Ko-)Kettenkomplex heißt \emph{exakt} oder \emph{azyklisch}, wenn
  \[
    \fa{n \in \Z} H_n(\CC{C}) \cong 0 \quad
    (\text{bzw. } H^n(\CC{C}) \cong 0).
  \]
\end{defn}

% III.2.3
\begin{defn}
  Ein (Ko-)Kettenkomplex heißt \emph{zyklisch} wenn seine Differentiale alle Null sind.
\end{defn}

% Vorgezogen: I.6.5 (Morphismen von Komplexen)
\begin{defn}
  Eine Morphismus $f : \CC{C} \to \CC{D}$ (bzw. $f : \CCC{C} \to \CCC{D}$) zwischen (Ko-)Kettenkomplexen ist eine Familie von Homomorphismen
  \[
    (f_n : C_n \to D_n)_{n \in \N} \quad
    \text{(bzw. $(f^n : C^n \to D^n)_{n \in \N}$),}
  \]
  die mit den Randabbildungen verträglich sind, \dh{}
  \[
    f_{n-1} \circ \partial^C_n = \partial^D_n \circ f_n \quad
    \text{(bzw. $f^{n+1} \circ \delta_C^n = \delta_D^n \circ f^n$)} \quad
    \text{für alle $n$.}
  \]
\end{defn}

% Aus III.1.1, III.2.3
\begin{defn}
  Die \emph{Kategorie der (Ko-)Komplexe} in $\Aat$ ist $\Kom(\Aat)$. \\
  Die Unterkat. der zyklischen Komplexe ist $\Kom_0(\Aat) \subset \Kom(\Aat)$.
\end{defn}

\begin{prop}
  (Ko-)Homologie ist ein Funktor $H_n, H^n : \Kom(\Aat) \to \Aat$
  beziehungsweise $H_\bullet, H^\bullet : \Kom(\Aat) \to \Kom_0(\Aat)$.
\end{prop}

% I.7.9 (Homotope Abbildungen von Komplexen)
\begin{defn}
  Eine \emph{Kettenhomotopie} zw. Morphismen $f, g : \CC{C} \to \CC{D}$ von Kettenkomplexen ist eine Folge von Homomorphismen $k_n : C_n \to D_{n+1}$ mit
  $\fa{n \in \N} \partial^D_{n+1} \circ k_n + k_{n-1} \circ \partial^C_n = f_n - g_n$.
\end{defn}

\begin{nota}
  $f \simeq g$ für kettenhomotope $f, g : \CC{C} \to \CC{D}$
\end{nota}

% III.1.2b) (verallgemeinert)
\begin{bem}
  Kettenhomotopien lassen sich verknüpfen:
  \[ f_1 \simeq f_2, \enspace g_1 \simeq g_2 \implies g_1 \circ f_1 \simeq g_2 \circ f_2 \]
\end{bem}

% I.7.10, III.1.2c)
\begin{lem}
  Kettenhomotope Abb. ind. dieselbe Abb. in (Ko-)Homologie:
  Seien $f, g : \CC{C} \to \CC{D}$ kettenhomotop. Dann gilt
  $H_\bullet(f) = H_\bullet(g)$.
  %\[ H_n(f) = H_n(g) \quad \text{für alle $n \in \N$.} \]
\end{lem}

\begin{samepage}
  \subsection{(Ko-)Homologie von simplizialen Mengen und topologischen Räumen}
\end{samepage}

% I.4.1
\begin{defn}
  Sei $X$ eine simpl. Menge. Sei $C_n(X)$ die von den $n$-Simplizes $X_n$ erzeugte abelsche Gruppe (\dh{} die Gruppe der endl. formalen Linearkombinationen mit Koeffizienten in $\Z$). Sei $\delta_n^i : [n{-}1] \to [n]$ diejenige streng monotone Abb. mit $i \not\in \im \delta_n^i$. Definiere
  \[
    \partial_n : C_n(X) \to C_{n-1}(X), \quad
    \sum_{\sigma \in X_n} \lambda_\sigma \cdot \sigma \, \mapsto \nspace{3} \sum_{\sigma \in X_n} \lambda_\sigma \sum_{i=0}^n (-1)^i X(\partial_n^i)(\sigma).
  \]
\end{defn}

% I.4.2
\begin{prop}
  $(C_\bullet(X),\partial_\bullet)$ ist ein Kettenkomplex (\dh{} $\partial_{n-1} \circ \partial_n = 0$)
\end{prop}

\begin{defn}
  Sei $X$ eine simpliziale Menge und $A$ eine abelsche Gruppe.
  \begin{itemize}
    \item Der \emph{Komplex} $(C_\bullet(X; A), \partial_\bullet)$ von $X$ \emph{mit Koeff.} in $A$ ist
    \[
      C_n(X; A) \coloneqq C_n(X) \otimes_{\Z} A, \enspace
      \partial_n \coloneqq \partial_n \otimes \id : C_n(X; A) \to C_{n-1}(X; A).
    \]
    \item Der \emph{Kokomplex} $(C^\bullet(X; A), \delta^\bullet)$ von $X$ mit \emph{Koeff.} in $A$ ist
    \begin{align*}
      & C^n(X; A) \coloneqq \Hom(C^n(X), A), \\
      \delta^n : \,\, & C^n(X; A) \to C^{n+1}(X; A), \enspace f \mapsto f \circ \delta_{n+1}.
    \end{align*}
  \end{itemize}
\end{defn}

\begin{defn}
  Für Verklebedaten $X$ ist der zugeh. (Ko-)Kettenkomplex (mit Koeffizienten) genauso definiert wie für simpliziale Mengen.
\end{defn}

\begin{beob}
  $C_n(X; \Z) = C_n(X)$
\end{beob}

\begin{nota}
  Sei $X$ eine simpliziale Menge. Setze
  \begin{itemize}
    \miniitem{0.48 \linewidth}{$H_n(X) \coloneqq H_n(C_\bullet(X))$,}
    \miniitem{0.48 \linewidth}{$H^n(X) \coloneqq H^n(C^\bullet(X; \Z))$,}
    \miniitem{0.48 \linewidth}{$H_n(X; A) \coloneqq H_n(C_\bullet(X; A))$,}
    \miniitem{0.48 \linewidth}{$H^n(X; A) \coloneqq H^n(C^\bullet(X; A))$.}
  \end{itemize}
\end{nota}

% I.4.5 (Geometrie von Ketten)
\begin{prop}
  Für jede simpl. Menge $X$ ex. ein kanonischer Isomorphismus
  \[ H_0(X, \Z) \cong \text{freie ab. Gr. erzeugt von Zshgskomponenten von $\abs{X}$}. \]
\end{prop}

\begin{prop}
  Sei $CX$ der Kegel über Verklebedaten $X$. Es gilt
  \[ H_0(CX) = \Z, \enspace H_{>0}(CX) = 0. \]
\end{prop}

% Ausgelassen: I.4.6 Geometrie von Koketten :-)

% I.4.7 Koeffizientensysteme

% I.4.8
\begin{defn}
  Sei $X$ eine simpliziale Menge.
  \begin{itemize}
    \item Ein \emph{homol. Koeffizientensystem} $\mathcal{A}$ auf $X$ ist ein Funktor
    \[ \mathcal{A} : (1 \downarrow X) \to \AbGrp. \]
    Dabei ist $1 : \mathbf{1} \to \SetC$ der Funktor, der konstant $\{ \star \}$ ist (und $\mathbf{1}$ die Kategorie mit einem Objekt und einem Morphismus).\\
    Expliziter besteht ein Koeffizientensystem aus einer abelschen Gruppe $\mathcal{A}_\sigma$ für jedes $n$-Simplex $\sigma \in X_n$ und Abbildungen $\mathcal{A}(f, \sigma) : \mathcal{A}_\sigma \to \mathcal{A}_{X(f)(\sigma)}$ für alle $\sigma \in X_n$, $f \in \Hom_{\Delta}([m], [n])$ mit
    \[
      \mathcal{A}(\id, \sigma) = \id, \quad
      \mathcal{A}(f \circ g, \sigma) = \mathcal{A}(g, X(f)(\sigma)) \circ \mathcal{A}(f, \sigma).
    \]
    \item Ein \emph{kohomol. Koeffizientensystem} $\mathcal{B}$ auf $X$ ist ein Funktor
    \[ \mathcal{B} : (1 \downarrow X)^\op \to \AbGrp. \]
    % I.7.9
    \item Ein Morphismus zw. (ko-)homologischen Koeffizientensystemen auf derselben simpl. Menge ist eine natürliche Transformation.
  \end{itemize}
\end{defn}

% I.4.9 (Bemerkungen und Beispiele)

% Ausgelassen: a) Konstante Koeffizientensysteme, c) Koeffizientensystem auf BG
\begin{bsp}
  Sei $Y$ ein topologischer Raum, $(U_\alpha)_{\alpha \in A}$ eine offene Überdeckung und $X$ deren Nerv. Dann definiert
  \begin{align*}
    \mathcal{F}_{\alpha_0, \ldots, \alpha_n} & \coloneqq \{ U_{\alpha_0} \cap \ldots \cap U_{\alpha_n} \to \R \text{ stetig} \}, \\
    \mathcal{F}(f, (\alpha_0, \ldots, \alpha_n))(\phi) & \coloneqq \text{passende Einschränkung von $\phi$}.
  \end{align*}
  ein kohomologisches Koeffizientensystem auf $X$.
\end{bsp}

% I.4.10 (Homologie und Kohomologie mit einem Koeffizientensystem)

\begin{defn}
  Sei $\mathcal{A}$ ein homologisches Koeffizientensystem auf einer simplizialen Menge $X$. Wir setzen
  \[ C_n(X; \mathcal{A}) \coloneqq \{ \text{ formale endl. Linearkomb. } \nspace{3} \sum_{\sigma \in X_n} \lambda_\sigma \cdot \sigma \text{ mit } \lambda_\sigma \in \mathcal{A}_\sigma \, \} \]
  und definieren $\partial_n : C_n(X; \mathcal{A}) \to C_{n-1}(X; \mathcal{A})$ durch
  \[ \sum_{\sigma \in X_n} \lambda_\sigma \cdot \sigma \enspace \mapsto \sum_{\sigma \in X_n} \sum_{i=0}^n \enspace (-1)^i \mathcal{A}(\partial_n^i, \sigma)(\lambda_\sigma) \cdot X(\partial_n^i)(\sigma). \]
  Die Homologiegruppen des so def. Kettenkomplexes $C_\bullet(X; \mathcal{A})$ heißen \emph{Homologiegruppen} von $X$ \emph{mit Koeffizienten in $\mathcal{A}$}.
\end{defn}

\begin{defn}
  Sei $\mathcal{B}$ ein kohomologisches Koeffizientensystem auf einer simplizialen Menge $X$. Wir setzen
  \[ C^n(X; \mathcal{B}) \coloneqq \{ \text{ Funktionen } f : (\sigma \in X_n) \to \mathcal{B}_\sigma \, \} \]
  und definieren $\delta_n : C^n(X; \mathcal{B}) \to C_{n+1}(X; \mathcal{B})$ durch
  \[ \delta^n(f)(\sigma) \coloneqq \sum_{i=0}^{n+1} (-1)^i \mathcal{B}(\partial_{n+1}^i, \sigma)(f(X(\partial_{n+1}^i)(\sigma))). \]
  Die Kohomologiegruppen des so def. Kokettenkomplexes $C^\bullet(X; \mathcal{B})$ heißen \emph{Kohomologiegruppen} von $X$ \emph{mit Koeffizienten in $\mathcal{B}$}.
\end{defn}

% I.4.11

\begin{bsp}
  Sei $Y$ ein topologischer Raum, $U = (U_\alpha)_{\alpha \in A}$, $X$ und $\mathcal{F}$ wie im letzten Beispiel. Die Homologiegruppen $H^n(X, \mathcal{F})$ werden \emph{Čech-Kohomologiegruppen} der Garbe der stetigen Funktionen auf $Y$ bzgl. der Überdeckung $U$ genannt.
\end{bsp}
% Ausgelassen: Beispiel c)

\begin{defn}
  Sei $Y$ ein topol. Raum und $X$ dessen simpl. Menge der singulären Simplizes.
  Die Homologiegruppen von $\CC{C}(X; A)$ heißen \emph{singuläre Homologiegruppen} $H_n(Y; A)$ von $Y$ mit Koeff. $A$.
\end{defn}

% I.7.11a)
\begin{prop}
  Seien $\phi, \psi : X \to Y$ homotope Abbildungen zwischen topologischen Räumen. Dann sind die induzierten Abbildungen $\phi_*, \psi_* : \CC{C}(X; A) \to \CC{C}(Y; A)$ kettenhomotop.
\end{prop}

% I.7.12
\begin{kor}
  Homotopieäquivalente Räume besitzen isomorphe singuläre Homologiegruppen.
\end{kor}


\subsection{Weitere Beispiele für (Ko-)Homologie}

% I.7 (Komplexe)

% I.7.1 (Wo kommen Komplexe her?)

% I.7.2
\begin{defn}
  Eine \emph{simpl. ab. Gruppe} ist ein Funktor
  $A : \Delta^\op \to \AbGrp$.
\end{defn}

% I.7.3
\begin{defn}
  Sei $A$ eine simpliziale abelsche Gruppe. \\
  Dann ist $(A_\bullet, \partial)$ ein Kettenkomplex mit
  \[
    \partial_n : A_n \to A_{n-1}, \quad
    a \mapsto \sum_{i=0}^n (-1)^i A(\partial_n^i)(a).
  \]
\end{defn}

\begin{defn}
  Eine \emph{kosimpl. ab. Gruppe} ist ein Funktor
  $A : \Delta \to \AbGrp$.
\end{defn}

\begin{defn}
  Sei $A$ eine kosimpliziale abelsche Gruppe. \\
  Dann ist $(A^\bullet, \delta)$ ein Kokettenkomplex mit
  \[
    \delta^n : A^n \to A^{n+1}, \quad
    a \mapsto \sum_{i=0}^n (-1)^i A(\partial_{n+1}^i)(a).
  \]
\end{defn}

% I.7.4 (Der Čech-Komplex)

\begin{defn}
  Sei $Y$ ein topol. Raum, $(U_\alpha)_{\alpha \in A}$ eine (nicht unbedingt offene) Überdeckung von $Y$ und $\Fais$ eine Garbe ab. Gruppen auf $Y$. Die kosimpliziale abelsche Gruppe $\check{C}(U, \Fais)$ der \emph{Čech-Koketten} ist
  \begin{align*}
    \check{C}^m(U, \Fais) \coloneqq \nspace{4} \prod_{\alpha_0, \ldots, \alpha_m \in A} \nspace{4} \Fais(U_{\alpha_0} \cap \ldots \cap U_{\alpha_m}), \\
    \check{C}(U, \Fais)(f : [m] \to [n])((f_{\alpha_0, \ldots ,\alpha_m})_{\alpha_0, \ldots, \alpha_m}) \coloneqq \\
    (f_{g(0), \ldots, g(m)}|U_{\alpha_0} \cap \ldots \cap U_{\alpha_n})_{\alpha_0, \nldots, \alpha_n}.
  \end{align*}
\end{defn}

\begin{bem}
  Die Randabb. im zugeh. Kokettenkomplex ist gegeben durch
  \[ (\delta^n \phi)_{\alpha_0, \ldots, \alpha_{n+1}} \coloneqq \sum_{i=0}^{n+1} (-1)^i \phi_{\alpha_0, \ldots, \hat{\alpha_i}, \ldots, \alpha_{n+1}}. \]
\end{bem}

\begin{defn}
  Die Kohomologiegruppen dieses Komplexes heißen \emph{Čech-Homologiegruppen} von $\Fais$ bzgl. der Überdeckung $(U_\alpha)_{\alpha \in A}$.
\end{defn}

\begin{bem}
  $\check{H}(U, \Fais) \cong \Gamma(X, \Fais)$ hängt nicht von der Überdeckung ab.
\end{bem}

% Ausgelassen: I.7.6 (Homologie und Kohomologie von Gruppen)

% I.7.7 (Der de-Rham-Komplex)
\begin{defn}
  Sei $M$ eine $\Cont^\infty$-Mft, $\Omega^k(M)$ das $C^\infty(M)$-Modul der $k$-Formen auf $M$. Die \emph{äußere Ableitung} $\d : \Omega^k(M) \to \Omega^{k+1}(M)$ ist in lokalen Koordinaten $(x^1, \ldots, x^n)$ definiert durch
  \[ \d \left( \sum_{\abs{I} = k} f_I \d x^I \right) = \sum_{\abs{I} = k} \sum_{i=1}^n \frac{\partial f_I}{\partial x^i} \d x^i \wedge \d x^I. \]
  Die Kohomologiegruppen des so definierten Komplexes $\Omega^\bullet(M)$ heißen \emph{De-Rham-Kohomologiegruppen}.
\end{defn}

% I.7.11b)
\begin{prop}
  Seien $\phi, \psi : M \to N$ zwei glatt homotope Abbildungen von $\Cont^\infty$-Mften. Dann sind $\phi^*, \psi^* : \Omega^\bullet(N) \to \Omega^\bullet(M)$ kettenhomotop.
\end{prop}

% I.7.8 (Homologie und Kohomologie einer Lie-Algebra)

\begin{defn}
  Sei $\Lg$ eine Lie-Algebra und $A$ ein $\Lg$-Modul. Setze $C^k(\Lg, A) \coloneqq L(\wedge^k \Lg, A)$ und definiere $\d : C^k(\Lg, A) \to C^{k+1}(\Lg, A)$ durch eine allgemeine Cartan-Formel
  \begin{align*}
    (\d c)(g_1, \nldots, g_{k+1}) \coloneqq & \nspace{7} \sum_{1 \leq j < l \leq k+1} \nspace{7} (-1)^{j+l-1} c([g_j, g_l], g_1, \nldots, \hat{g_j}, \nldots, \hat{g_l}, \nldots, g_{k+1}) \\
    & + \sum_{j=1}^{k+1} (-1)^j g_j c(g_1, \nldots, \hat{g_j}, \ldots, g_{k+1}).
  \end{align*}
  Die Kohomologiegruppen des so definierten Kokettenkomplexes werden mit $H^\bullet(\Lg, A)$ bezeichnet.
\end{defn}


\subsection{Exakte Sequenzen}

% I.6 (Die lange Sequenz)

% Ausgelassen: I.6.1 (Homologie als Funktion in zwei Variablen)

% I.6.2 (Exakte Sequenzen)

\begin{defn}
  Eine (lange) \emph{exakte Sequenz} ist ein (Ko-)Kettenkomplex mit verschwindender Homologie.
\end{defn}

\begin{defn}
  Eine \emph{kurze ex. Sequenz} (\keS{}) ist eine ex. Seq. der Form
  \[ 0 \to A \to B \to C \to 0. \]
\end{defn}

% II.6.3d)
\begin{lem}
  Jede \keS{} $0 \to X \xrightarrow{f} Y \xrightarrow{g} Z \to 0$ in einer abelschen Kategorie ist kanonisch isomorph zur Sequenz
  \[ 0 \to \ker g \to Y \to \coker f \to 0. \]
\end{lem}

\begin{defn}
  Sei $0 \to A \to B \to C \to 0 \in \Aat$ eine \keS{}
  Die Sequenz heißt \emph{spaltend}, falls sie isomorph zur \keS{} $0 \to A \to A \oplus C \to C$ ist.
\end{defn}

\begin{prop}
  Für eine \keS{} $0 \to A \xrightarrow{f} B \xrightarrow{g} C \to 0 \in \Aat$ sind äquivalent:
  \begin{itemize}
    \item Die Sequenz spaltet.
    \item Es existiert eine Retraktion $r : B \to A$ mit $r \circ f = \id_A$.
    \item Es existiert ein Schnitt $s : C \to B$ mit $g \circ s = \id_C$.
  \end{itemize}
\end{prop}

% III.2.3
\begin{defn}
  Die abelsche Kategorie $\Aat$ heißt \emph{halbeinfach}, wenn alle kurzen exakten Sequenzen in $\Aat$ spalten.
\end{defn}

% III.2.3
\begin{bsp}
  Die Kategorie der VR über einem Körper ist halbeinfach.
\end{bsp}

\begin{bem}
  Jede (lange) exakte Sequenz lässt sich in \keS{} zerlegen:
  % Mit der neuesten Version von tikz-cd (in Entwicklung) sind die Pfeile
  % angeblich nicht mehr so windschief: siehe Kommentar auf
  % http://tex.stackexchange.com/questions/130780/tikz-cd-how-can-i-arrange-diagonal-arrows-parallel
  % --> TODO: auf neue Version warten, rekompilieren, diesen Kommentar entfernen
  \begin{centertikzcd}[column sep=0.2cm, row sep=0.3cm]
    0 \arrow[rd] && 0 && 0 \arrow[rd] && 0 \\
    & C_{n+1} \arrow[rd] \arrow[ru] &&&& C_{n-1} \arrow[rd] \arrow[ru] \\
    \nldots \to L_{n+1} \arrow[ru] \arrow[rr, "\partial_{n+1}"] &&
    L_n \arrow[rd] \arrow[rr, "\partial_{n}"] &&
    L_{n-1} \arrow[ru] \arrow[rr, "\partial_{n-1}"] &&
    L_{n-2} \to \nldots \\
    &&& C_n \arrow[ru] \arrow[rd] \\
    && 0 \arrow[ru] && 0
  \end{centertikzcd}
  Dabei ist $C_i \coloneqq \ker \partial_{i-1} \cong \coker \partial_{i+1}$. Umgekehrt lässt sich aus solch diagonal verknüpften kurzen exakten Sequenzen eine \leS{} bauen.
\end{bem}

% I.6.5
\begin{defn}
  Eine Sequenz $0 \to \CCC{A} \to \CCC{B} \to \CCC{C} \to 0$ von Komplexen heißt \emph{exakt}, wenn für alle $n$ die Seq. $0 \to A_n \to B_n \to C_n \to 0$ exakt ist.
\end{defn}

% Ausgelassen: I.6.7 (Konstruktion der Randabbildung)

% 1.6.8
\begin{prop}
  Eine kurze exakte Sequenz
  $0 \to \CCC{A} \xrightarrow{i^\bullet} \CCC{B} \xrightarrow{p^\bullet} \CCC{C} \to 0$
  von Kokettenkomplexen induziert eine lange exakte Sequenz
  \[ \nldots \to H^n(\CCC{A}) \xrightarrow{H^n(i^\bullet)} H^n(\CCC{B}) \xrightarrow{H^n(p^\bullet)} H^n(\CCC{C}) \xrightarrow{\delta^n} H^{n+1}(\CCC{A}) \to \nldots \]
\end{prop}

% I.6.6
\begin{lem}
  Sei $0 \to A \to B \to C \to 0$ eine \keS{} ab. Gruppen und $X$ eine simpl. Menge.
  Dann sind ebenfalls exakt:
  \begin{align*}
    0 \to \CC{C}(X; A) \to \CC{C}(X; B) \to \CC{C}(X; C) \to 0, \\
    0 \to \CCC{C}(X; A) \to \CCC{C}(X; B) \to \CCC{C}(X; C) \to 0.
  \end{align*}
\end{lem}

% I.6.3
\begin{kor}
  Sei $0 \to A \to B \to C \to 0$ eine \keS{} ab. Gruppen und $X$ eine simpl. Menge. Dann existieren lange exakte Sequenzen
  \begin{align*}
    \ldots \to H_n(X; A) \to H_n(X; B) \to H_n(C) \to H_{n-1}(X; A) \to \ldots \\
    \ldots \to H^n(X; A) \to H^n(X; B) \to H^n(C) \to H^{n+1}(X; A) \to \ldots
  \end{align*}
\end{kor}

% Ausgelassen: I.6.4 (Bemerkungen)

% I.6.9 (Verallgemeinerung auf ein Koeffizientensystem)

\begin{defn}
  Eine Sequenz $0 \to \mathcal{B}' \to \mathcal{B} \to \mathcal{B}'' \to 0$ von (ko-)homologischen Koeffizientensystemen auf einer simpl. Menge $X$ heißt \emph{exakt}, falls
  \[
    0 \to \mathcal{B}'_\sigma \to \mathcal{B}_\sigma \to \mathcal{B}''_\sigma \to 0 \qquad
    \text{für alle $\sigma \in X_n$ exakt ist.}
  \]
\end{defn}

\begin{lem}
  Eine kurze exakte Sequenz $0 \to \mathcal{B}' \to \mathcal{B} \to \mathcal{B}'' \to 0$ von (ko-)homologischen Koeff'systemen induziert kurze ex. Sequenzen
  \begin{align*}
    0 \to \CC{C}(X; \mathcal{B}') \to \CC{C}(X; \mathcal{B}) \to \CC{C}(X; \mathcal{B}'') \to 0, \\
    0 \to \CCC{C}(X; \mathcal{B}') \to \CCC{C}(X; \mathcal{B}) \to \CCC{C}(X; \mathcal{B}'') \to 0
  \end{align*}
  und damit auch entsprechende lange exakte Sequenzen.
\end{lem}

% Ausgelassen: Gegenbeispiele für A4
% II.5.17: Filtrierte abelsche Gruppen
% II.5.18: Topologische abelsche Gruppen

\begin{lem}[Viererlemmata]
  Sei folgendes kommutatives Diagramm in einer abelschen Kategorie mit exakten Zeilen gegeben:
  \vspace{-8pt}
  \begin{centertikz}
    \matrix (mat) [matrix of nodes, column sep=1cm, row sep=0.45cm]{
      \node (A) {$A$}; &
      \node (B) {$B$}; &
      \node (C) {$C$}; &
      \node (D) {$D$}; \\
      \node (A') {$A'$}; &
      \node (B') {$B'$}; &
      \node (C') {$C'$}; &
      \node (D') {$D'$}; \\
    };
    \draw[->] (A) to node {} (B);
    \draw[->] (B) to node {} (C);
    \draw[->] (C) to node {} (D);
    \draw[->] (A') to node {} (B');
    \draw[->] (B') to node {} (C');
    \draw[->] (C') to node {} (D');
    \draw[->>] (A) to node [right] {$\alpha$} (A');
    \draw[->] (B) to node [right] {$\beta$} (B');
    \draw[->] (C) to node [right] {$\gamma$} (C');
    \draw[right hook->] (D) to node [right] {$\delta$} (D');
  \end{centertikz}
  \vspace{-10pt}
  Sei $\alpha$ epimorph und $\delta$ monomorph.\\
  \inlineitem{Ist $\gamma$ epimorph, so auch $\beta$.} \enspace
  \inlineitem{Ist $\beta$ monomorph, so auch $\gamma$.}
\end{lem}

\begin{bem}
  Die Aussagen der beiden Viererlemmata sind zueinander dual.
\end{bem}

% aus AlgTopo-Zusammenfassung
\begin{kor}[\emph{Fünferlemma}]
  Sei folgendes kommutatives Diagramm in einer abelschen Kategorie mit exakten Zeilen gegeben:
  \vspace{-8pt}
  \begin{centertikz}
    \matrix (mat) [matrix of nodes, column sep=1cm, row sep=0.45cm]{
      \node (A) {$A$}; &
      \node (B) {$B$}; &
      \node (C) {$C$}; &
      \node (D) {$D$}; &
      \node (E) {$E$}; \\
      \node (A') {$A'$}; &
      \node (B') {$B'$}; &
      \node (C') {$C'$}; &
      \node (D') {$D'$}; &
      \node (E') {$E'$}; \\
    };
    \draw[->] (A) to node {} (B);
    \draw[->] (B) to node {} (C);
    \draw[->] (C) to node {} (D);
    \draw[->] (D) to node {} (E);
    \draw[->] (A') to node {} (B');
    \draw[->] (B') to node {} (C');
    \draw[->] (C') to node {} (D');
    \draw[->] (D') to node {} (E');
    \draw[->] (A) to node [right] {$\alpha$} (A');
    \draw[->] (B) to node [right] {$\beta$} (B');
    \draw[->] (C) to node [right] {$\gamma$} (C');
    \draw[->] (D) to node [right] {$\delta$} (D');
    \draw[->] (E) to node [right] {$\epsilon$} (E');
  \end{centertikz}
  \vspace{-10pt}
  Sind $\alpha$, $\beta$, $\delta$ und $\epsilon$ Isomorphismen, dann auch $\gamma$.
\end{kor}

\begin{lem}[\emph{Schlangenlemma}]
  Sei folgendes kommutatives Diagramm in einer abelschen Kategorie mit exakten Zeilen gegeben:
  \vspace{-8pt}
  \begin{centertikz}
    \matrix (mat) [matrix of nodes, column sep=1cm, row sep=0.45cm]{
      &
      \node (A) {$A$}; &
      \node (B) {$B$}; &
      \node (C) {$C$}; &
      \node (O) {$0$}; \\
      \node (O') {$0$}; &
      \node (A') {$A'$}; &
      \node (B') {$B'$}; &
      \node (C') {$C'$}; \\
    };
    \draw[->] (A) to node {} (B);
    \draw[->] (B) to node {} (C);
    \draw[->] (C) to node {} (O);
    \draw[->] (O') to node {} (A');
    \draw[->] (A') to node {} (B');
    \draw[->] (B') to node {} (C');
    \draw[->] (A) to node [right] {$\alpha$} (A');
    \draw[->] (B) to node [right] {$\beta$} (B');
    \draw[->] (C) to node [right] {$\gamma$} (C');
  \end{centertikz}
  \vspace{-10pt}
  Dann gibt es einen Verbindungshomomorphismus $\delta : \ker \gamma \to \coker \alpha$, mit dem folgende Sequenz exakt ist:
  \[ \ker \alpha \to \ker \beta \to \ker \gamma \xrightarrow{\delta} \coker \alpha \to \coker \beta \to \coker \gamma. \]
\end{lem}

% Übungsblatt 18, Aufgabe 3
\begin{defn}
  Die \emph{K-Theorie} $K(\Aat)$ einer abelschen Kategorie $\Aat$ ist die abelsche Gruppe (bzw. das Klassen-Äquivalent einer ab. Gruppe) erzeugt von $\Ob(\Aat)$ modulo der Äquivalenzrelation erzeugt von $X + Z = Y$ für alle kurzen ex. Seq. $0 \to X \to Y \to Z \to 0$ in $\Aat$.
\end{defn}

% Übungsblatt 18, Aufgabe 3b), c)
\begin{bspe}
  \inlineitem{$K(\kVectFin) \cong \Z$} \quad
  \inlineitem{$K(\kVect) \cong 0$}
\end{bspe}

\subsection{Exakte Sequenzen von Garben}
% II.5.12 (Garben und Prägarben)

% Ausgelassen: Bsp II.5.14

% II.5.16
\begin{lem}
  Sei $\varphi : \Fais \to \Garb$ ein Morphismus von Prägarben,
  $(K, k)$ dessen Kern und $(C, c)$ dessen Kokern.
  Dann ist $(sK, sk)$ der Kern und $(sC, sc)$ der Kokern von $s \varphi : s \Fais \to s \Garb$.
\end{lem}

% II.5.15
\begin{prop}
  Seien $\Fais$ und $\Garb$ Garben über einem topologischen Raum $X$, $\varphi : \Fais \to \Garb$ ein Morphismus von Garben und
  \[ \Karb \xrightarrow{k} i \Fais \xrightarrow{i} \Iarb \xrightarrow{j} i \Garb \xrightarrow{c} \Carb \ \]
  dessen kanonische Zerlegung von $i \varphi$ in $\PShAb(X)$. Dann ist
  \[ s \Karb \xrightarrow{s(k)} si \Fais \cong \Fais \xrightarrow{s(i)} s \Iarb \xrightarrow{s(j)} \Garb \cong si \Garb \xrightarrow{s(c)} s \Carb \ \]
  eine kanonische Zerlegung von $si \varphi \cong \varphi$.
\end{prop}

\begin{kor}
  $\ShAb(X)$ ist eine abelsche Kategorie.
\end{kor}

% Übungsblatt 6, Aufgabe 3
\begin{bem}
  Sei $\Fais \xrightarrow{f} \Garb \xrightarrow{g} \Harb$ eine Sequenz von Prägarben abelscher Gruppen auf einem topologischen Raum $X$.
  \begin{itemize}
    \item Die Sequenz ist eine exakte Sequenz von Prägarben, wenn für alle offenen $U \subseteq X$ gilt: $\im f_U = \ker g_U$.
    \item Seien $\Fais$, $\Garb$, $\Harb$ sogar Garben. Dann ist die Seq. eine ex. Seq. von Garben, wenn für alle offenen $U \subseteq X$ gilt, dass $\im f_U \subseteq \ker g_U$ und jeder Schnitt $t \in \ker g_U$ lokal Urbilder besitzt, \dh{} es existiert eine offene Überdeckung $U = {\bigcup}_{i \in I} U_i$ und eine Familie von Schnitten $(s_i \in \Fais(U_i))_{i \in I}$ mit 
    $\fa{i \in I} f_{U_i}(s_i) = t|_{U_i}$.
  \end{itemize}
\end{bem}

% Übungsblatt 6, Aufgabe 3a)-d)
\begin{lem}
  \begin{itemize}
    \item Eine Sequenz $\Fais \to \Garb \to \Harb$ von Garben ab. Gruppen ist genau dann exakt, wenn sie halmweise exakt ist, \dh{} für alle $x \in X$ ist die induzierte Sequenz $\Fais_x \to \Garb_x \to \Harb_x$ exakt.
    \item Wenn eine Sequenz von Garben aufgefasst als Sequenz von Prägarben exakt ist, dann ist sie es auch als Sequenz von Garben.
    \item Sei $0 \to \Fais \to \Garb \to \Harb \to 0$ eine $\keS{}$ von Prägarben auf einem topol. Raum und $\Fais$, $\Harb$ sogar Garben. Dann ist auch $\Garb$ eine Garbe.
  \end{itemize}
\end{lem}

% Übungsblatt 6, Aufgabe 5b),5c)
\begin{lem}
  Sei $0 \to \Fais \to \Garb \to \Harb \to \O \in \ShAb(X)$ eine \keS{} von Garben abelscher Gruppen auf einem topologischen Raum $X$.
  \begin{itemize}
    \item Sei $\Fais$ welk und $U \subseteq X$ offen. Dann ist auch
    \[
      0 \to \Gamma(U, \Fais) \to \Gamma(U, \Garb) \to \Gamma(U, \Harb) \to 0
      \quad \text{exakt.}
    \]
    \item Sind $\Fais$ und $\Garb$ welk, dann auch $\Harb$.
  \end{itemize}
\end{lem}

% Übungsblatt 6, Aufgabe 6c),6d)
\begin{lem}
  Sei $0 \to \Fais \to \Garb \to \Harb \to \O \in \ShAb(X)$ eine \keS{} von Garben abelscher Gruppen auf einem parakompakten Hausdorffraum $X$.
  \begin{itemize}
    \item Sei $\Fais$ weich und $A \subseteq X$ abgeschlossen. Dann ist auch
    \[
      0 \to \Gamma(A, \Fais) \to \Gamma(A, \Garb) \to \Gamma(A, \Harb) \to 0
      \quad \text{exakt.}
    \]
    \item Sind $\Fais$ und $\Garb$ weich, dann auch $\Harb$.
  \end{itemize}
\end{lem}

% (etwas später als im Buch)
% Kapitel II.6 Funktoren zw. abelschen Kategorien
\section{Funktoren zwischen abelschen Kategorien}

% II.6.1
\begin{defn}
  Ein Funktor $F \!:\! \Aat \!\to\! \Bat$ zw. additiven Kategorien heißt \emph{additiv}, falls für alle $X, Y \!\in\! \Ob(\Aat)$ die Abb. $F \!:\! \Hom(X, Y) \to \Hom(FX, FY)$ ein Morphismus von abelschen Gruppen ist.
\end{defn}

% Ausgelassen: Beispiel

% II.6.4
\begin{defn}
  Ein additiver Funktor $F : \Aat \to \Bat$ zw. ab. Kategorien heißt \\
  a) \emph{exakt}, b) \emph{links-exakt}, c) \emph{rechts-exakt}, falls für alle \keS{}
  $0 \to X \xrightarrow{f} Y \xrightarrow{g} Z \to 0$
  aus $\Aat$ auch folgende Seq. in $\Bat$ exakt ist:
  \begin{alignat*}{3}
    \text{a)} \enspace & 0 \to &&\, FX \xrightarrow{Ff} FY \xrightarrow{Fg} FZ \to 0 \\[-5pt]
    \text{b)} \enspace & 0 \to &&\, FX \xrightarrow{\phantom{Ff}} FY \xrightarrow{\phantom{Fg}} FZ \\[-5pt]
    \text{c)} \enspace & &&\, FX \xrightarrow{\phantom{Ff}} FY \xrightarrow{\phantom{Fg}} FZ \to 0.
  \end{alignat*}
\end{defn}

\begin{defn}
  Sei $F : \Aat \to \Bat$ ein additiver Funktor zwischen abelschen Kategorien und $0 \to X \to Y \to Z \to 0$ eine zerfallende \keS{} in $\Aat$. \\
  Dann ist auch $0 \to FX \to FY \to FZ \to 0$ eine zerfallende \keS{}
\end{defn}

% II.6.7
\begin{prop}
  Sei $X$ ein topologischer Raum, $U \subseteq X$ offen. Der Funktor
  \[ \Gamma(U, \blank) : \ShAb(X) \to \AbGrp, \enspace \Fais \mapsto \Gamma(U, \Fais) \]
  ist links-exakt.
\end{prop}

\begin{bem}
  Die Prop gilt auch für Garben abelscher Gruppen auf Siten.
\end{bem}

% II.6.5
\begin{prop}
  Sei $\Aat$ eine abelsche Kategorie. Dann sind die Funktoren
  \begin{alignat*}{5}
    & \Hom_\Aat(X, \blank) : \Aat \to \AbGrp, \enspace && Y \mapsto \Hom_\Aat(X, Y) \qquad && \text{($X \in \Ob(\Aat)$ fest)}, \\
    & \Hom_\Aat(\blank, Y) : \Aat^\op \to \AbGrp, \enspace && X \mapsto \Hom_\Aat(X, Y) \qquad && \text{($Y \in \Ob(\Aat)$ fest)}
  \end{alignat*}
  beide links-exakt.
\end{prop}

% II.6.8
\begin{defn}
  Sei $\Aat$ eine abelsche Kategorie. Ein Objekt
  \begin{itemize}
    \item $X \in \Ob(\Aat)$ heißt \emph{projektiv}, falls $\Hom_\Aat(X, \blank)$ exakt ist.
    \item $Y \in \Ob(\Aat)$ heißt \emph{injektiv}, falls $\Hom_\Aat(\blank, Y)$ exakt ist.
  \end{itemize}
\end{defn}

\begin{bem}
  Ein Objekt $X \in \Ob(\Aat)$ ist genau dann projektiv, wenn $X \in \Ob(\Aat^\op)$ injektiv ist und umgekehrt.
\end{bem}

% II.6.6
\begin{prop}
  Sei $A$ ein Ring, $\AMod$ und $\ModA$ die Kategorien der $A$-Links- bzw. $A$-Rechtsmoduln.
  Dann sind die Funktoren
  \begin{align*}
    \AMod \to \AbGrp, \enspace & Y \mapsto X \otimes_A Y \qquad \text{($X \in \Ob(\ModA)$ fest)}, \\
    \ModA \to \AbGrp, \enspace & X \mapsto X \otimes_A Y \qquad \text{($Y \in \Ob(\AMod)$ fest)}
  \end{align*}
  beide rechts-exakt.
\end{prop}

% II.6.8
\begin{defn}
  Sei $A$ ein Ring. Ein Modul $X \!\in\! \Ob \ModA$ / $Y \!\in\! \Ob \AMod$
  heißt \emph{flach}, falls der Funktor $Y \mapsto X \otimes_A Y$ / $X \mapsto X \otimes_A Y$ exakt ist.
\end{defn}

\begin{konv}
  Falls in einem Diagramm in einer Zeile das Nullobjekt vorkommt, so wird diese Zeile als exakt angenommen.
\end{konv}

% II.6.9
\begin{bem}
  Sei $\Aat$ eine abelsche Kategorie. Ein Objekt $X \in \Ob(\Aat)$ bzw. $Y \in \Ob(\Aat)$ ist genau dann projektiv bzw. injektiv, wenn
  \vspace{-10pt}
  \begin{center}
    \begin{minipage}{100pt}
      \begin{center}
        \begin{tikzpicture}
          \matrix (mat) [matrix of nodes, column sep=1cm, row sep=0.45cm]{
            \node (X) {$X$}; \\
            \node (Y) {$Y$}; &
            \node (Y') {$Y'$}; &
            \node (O) {$0$}; \\
          };
          \draw[->] (X) to node [above] {$\varphi$} (Y');
          \draw[->,dashed] (X) to node [left] {$\psi$} (Y);
          \draw[->>] (Y) to node [above] {$\pi$} (Y');
          \draw[->] (Y') to node {} (O);
        \end{tikzpicture} \\
        (Projektivitätsdiagramm)
      \end{center}
    \end{minipage}
    \hspace{8pt}
    \begin{minipage}[t]{10pt}
      \vspace{-10pt}
      bzw.
    \end{minipage}
    \hspace{12pt}
    \begin{minipage}{100pt}
      \begin{center}
        \begin{tikzpicture}
          \matrix (mat) [matrix of nodes, column sep=1cm, row sep=0.45cm]{
            \node (Y) {$Y$}; \\
            \node (X) {$X$}; &
            \node (X') {$X'$}; &
            \node (O) {$0$.}; \\
          };
          \draw[->] (X') to node [above] {$\varphi$} (Y);
          \draw[->,dashed] (X) to node [left] {$\psi$} (Y);
          \draw[left hook->] (X') to node [above] {$i$} (X);
          \draw[->] (O) to node {} (X');
        \end{tikzpicture} \\
        (Injektivitätsdiagramm)
      \end{center}
    \end{minipage}
  \end{center}
\end{bem}

% II.6.10 (Projektive und freie Moduln)
\begin{lem}
  Ein $A$-Modul $X$ ist genau dann projektiv, wenn er ein direkter Summand eines freien $A$-Moduls ist, \dh{} wenn ein \\
  $A$-Modul $Y$ existiert, sodass $X \oplus Y$ frei ist.
\end{lem}

% II.6.11 (Injektive Moduln und Teilbarkeit), http://ncatlab.org/nlab/show/Baer%27s+criterion
\begin{lem}[Baer-Kriterium]
  Ein $A$-Linksmodul $X$ ist genau dann injektiv, wenn für alle $A$-Linksideale $I \subset X$ und Modul-Morphismen $q : I \to Q$ eine Fortsetzung $\tilde{q} : Q \to Q$ mit $q = \tilde{q}|_I$ existiert.
\end{lem}

% II.6.11, http://ncatlab.org/nlab/show/injective+module#injective_modules__abelian_groups
\begin{lem}
  Eine ab. Gruppe $X$ ist genau dann als $\Z$-Modul injektiv, wenn man in $A$ durch ganze Zahlen teilen kann, \dh{}
  \[ \fa{a \in X, n \in \Z \setminus \{ 0 \}} \ex{b \in X} nb = a. \]
\end{lem}

\begin{bspe}
  Es sind injektive abelsche Gruppen: $\R$, $\Q$, $\Q/\Z$, $\Q_{(p)}/\Z$
\end{bspe}

% II.6.12 (Flache Moduln und Relationen)
\begin{defn}
  Sei $X$ ein $A$-Rechtsmodul. Eine \emph{Relation in $A$} von Elementen $x_1, \ldots, x_n \in X$ ist ein Tupel $(a_1, \ldots, a_n) \in A^n$ mit $\sum x_i a_i = 0 \in X$. \\
  Allgemeiner ist eine \emph{Relation in einem $A$-Linksmodul $Y$} von $x_1, \nldots, x_n \in X$ ein Tupel $(y_1, \nldots, y_n) \in Y^n$ mit $\sum x_i \otimes y_i \!=\! 0 \in X \!\otimes\! Y$.
\end{defn}

\begin{bem}
  Sei $X$ ein $A$-Rechts-, $Y$ ein $A$-Linksmodul und $x_1, \nldots, x_n \in X$. \\
  Für $j = 1, \ldots, m$ sei $y^{(j)} \in Y$ beliebig und $(a_1^{(j)}, \nldots, a_n^{(j)}) \in A^n$ eine Relation von den $x_i$ in $A$.
  Dann ist $(y_1, \nldots, y_n) \in Y^n$ mit $y_i \coloneqq \sum a_i^{(j)} y^{(j)}$ eine Relation von den $x_i$ in $Y$.
\end{bem}

% II.6.12 & II.6.13 (Flache und projektive Moduln)
\begin{lem}
  Für einen $A$-Linksmodul $Y$ sind äquivalent:
  \begin{itemize}
    \item $Y$ ist flach (\dh{} $X \mapsto X \otimes_A Y$ ist exakt).
    \item Alle Relationen von Elementen $x_1, \ldots, x_n \in X$ in einem $A$-Rechts- modul $X$ erhält man durch die eben beschriebene Konstruktion.
    \item $Y$ ist filtrierter Kolimes von freien Moduln (Lazards Kriterium).
  \end{itemize}
\end{lem}

% II.6.13
\begin{lem}
  \begin{itemize}
    \item Freie Moduln sind flach.
    \item Direkte Summanden von flachen Moduln sind flach.
    \item Induktive Limiten von flachen Moduln sind flach.
  \end{itemize}
\end{lem}

% TODO: Abschnitt II.6.14 "Azyklische Objekte eines Situs"

% Übungsblatt 17, Aufgabe 2
\begin{lem}
  Sei $F : \Aat \to \Cat$ ein add. Funktor zw. ab. Kategorien. Dann gilt:
  \begin{align*}
    \text{$F$ ist linksexakt} & \, \iff \text{$F$ bewahrt endliche Limiten} \\
    \text{$F$ ist rechtsexakt} & \, \iff \text{$F$ bewahrt endliche Kolimiten}
  \end{align*}
\end{lem}

% II.6.20
\begin{kor}
  Mit RAPL bzw. LAPC folgt: Rechtsadjungierte additive Funktoren sind linksexakt, linksadjungierte rechtsexakt.
\end{kor}

\begin{bem}
  %Es folgt die Rechtsexaktheit von $f_\bullet$ und die Linksexaktheit von $f^\bullet$. Es gilt sogar:
  Es folgt: $f_\bullet$ ist rechts- und $f^\bullet$ linksexakt. Es gilt sogar:
\end{bem}

\begin{lem}
  Das inverse Bild $f^\bullet$ ist exakt.
\end{lem}

% III. Abgeleitete Kategorien und abgeleitete Funktoren

% III.1 Komplexe als verallgemeinerte Objekte

\section{Abgeleitete Kategorien}

Sei $\Aat$ eine abelsche Kategorie.

% Ausgelassen: Hilberts Begriff der "syzygies"

% III.1.1
\begin{defn}
  Eine \emph{projektive Auflösung} $\CC{F} \xrightarrow{\epsilon} E$ eines Obj. $E \in \Ob(\Aat)$ ist ein Kettenkomplex $\CC{F}$ bestehend aus projektiven Objekten in $\Aat$ und ein Augmentierungsmorphismus $\epsilon : F_0 \to E$, sodass
  \[ \ldots \to F_3 \to F_2 \to F_1 \to F_0 \xrightarrow{\epsilon} E \to 0 \]
  azyklisch ist. Entsprechend besteht bei einer \emph{freien Auflösung} eines Obj. aus $\Aat \!=\! \AMod$ der Komplex $\CC{F}$ aus freien $A$-Moduln. \\
  Eine \emph{injektive Auflösung} von $E$ ist eine proj. Auflösung in $\Aat^\op$.
\end{defn}

% III.1.3
\begin{lem}
  Seien $\CC{P} \xrightarrow{\epsilon_X} X$ und $\CC{Q} \xrightarrow{\epsilon_Y} Y$ projektive Auflösungen von Objekten $X, Y \in \Ob(\Aat)$ und $f \in \Hom_\Aat(X, Y)$ ein Morphismus.
  \begin{itemize}
    \item Dann existiert ein Morphismus von Auflösungen $R(f) : \CC{P} \to \CC{Q}$ der $f$ fortsetzt, \dh{} $R(f)$ ist ein Morphismen von Komplexen und es gilt $\epsilon_Y \circ R(f)_0 = f \circ \epsilon_X$.
    \item Die Fortsetzung ist eindeutig bis auf Kettenhomotopie.
  \end{itemize}
\end{lem}

% III.1.4a)
\begin{kor}
  Seien $\CC{P} \to X$ und $\CC{P} \to X$ projektive Auflösungen desselben Obj. $X \in \Ob(\Aat)$. Dann gibt es Morphismen von Auflösungen $f : \CC{P} \to \CC{Q}$ und $g : \CC{Q} \to \CC{P}$ mit $f \circ g \simeq \id$ und $g \circ f \simeq \id$.
\end{kor}

% Ausgelassen: III.1.4b)

% III.1.4c)
\begin{bem}
  Duale Aussagen gelten für injektive Auflösungen.
\end{bem}

% III.1.5
\begin{defn}
  Ein Morphismus $f : \CC{K} \to \CC{L}$ zwischen Kettenkomplexen heißt \emph{Quasiisomorphismus} (Qis), falls $H_\bullet(f) : H_\bullet(K) \to H_\bullet(L)$ ein Isomorphismus ist.
\end{defn}

% III.1.6
\begin{bem}
  \begin{itemize}
    \item Zwei proj. Auflösungen desselben Obj. sind quasiisomorph.
    \item Ein Komplex $\CC{K}$ ist genau dann azyklisch, wenn $\CC{K} \to \CC{0}$ ($\CC{0}$ ist der Nullkomplex) ein Quasiisomorphismus ist.
    \item Jede projektive Auflösung $\CC{P} \xrightarrow{\epsilon} X$ induziert einen Qis
    \vspace{-8pt}
    \begin{centertikz}
      \matrix (mat) [matrix of nodes, column sep=1cm, row sep=0.45cm]{
        \node (A) {$\phantom{A}\ldots$}; &
        \node (B) {$P_2$}; &
        \node (C) {$P_1$}; &
        \node (D) {$P_0$}; &
        \node (E) {$0$}; &
        \node (F) {$\ldots\phantom{A}$}; \\
        \node (A') {$\phantom{A}\ldots$}; &
        \node (B') {$0$}; &
        \node (C') {$0$}; &
        \node (D') {$X$}; &
        \node (E') {$0$}; &
        \node (F') {$\ldots\phantom{A}$}; \\
      };
      \draw[->] (A) to node {} (B);
      \draw[->] (B) to node {} (C);
      \draw[->] (C) to node {} (D);
      \draw[->] (D) to node {} (E);
      \draw[->] (E) to node {} (F);
      \draw[->] (A') to node {} (B');
      \draw[->] (B') to node {} (C');
      \draw[->] (C') to node {} (D');
      \draw[->] (D') to node {} (E');
      \draw[->] (E') to node {} (F');
      %\draw[->] (A) to node [right] {$\alpha$} (A');
      \draw[->] (B) to node {} (B');
      \draw[->] (C) to node {} (C');
      \draw[->] (D) to node [right] {$\epsilon$} (D');
      \draw[->] (E) to node {} (E');
    \end{centertikz}
    %\vspace{-10pt}
  \end{itemize}
\end{bem}

% Ausgelassen: Großer Ausblick in III.1.6 (Was is die abgeleitete Kategorie?)

% III.2 Abgeleitete Kategorien und Lokalisierung

% III.2.2
\begin{defn}
  Die \emph{Lokalisierung} einer Kat. $\Cat$ an einer Klasse $S \subset \Mor(\Cat)$ von Morphismen ist die Kat. $\Cat[S^{-1}]$ mit $\Ob(\Cat[S^{-1}]) \coloneqq \Ob(\Cat)$ und
  \begin{align*}
    \Hom_{\Cat[S^{-1}]}(X, Y) \coloneqq \{ \, & \text{Ketten in $\Cat$, bestehend aus Morphismen aus $\Cat$} \\[-4pt]
    & \text{und formalen Inversen von Morphismen in $S$,} \\[-2pt]
    & \text{die bei $X$ beginnen und bei $Y$ enden} \, \} / \sim.
  \end{align*}
  Dabei wird die Kongruenzrelation ${\sim}$ erzeugt von
  \begin{align*}
    (X \xrightarrow{f} Y \xrightarrow{g} Z) \sim (X \xrightarrow{g \circ f} Z), \quad
    & (X \xrightarrow{\id} X) \sim \text{leere Kette}, \\
    (S \xrightarrow{s} T \xleftarrow{s} S) \sim (S \xrightarrow{\id} S), \quad
    & (T \xleftarrow{s} S \xrightarrow{s} T) \sim (T \xrightarrow{\id} T)
  \end{align*}
  für Morphismen $(X \xrightarrow{f} Y), (Y \xrightarrow{g} Z) \in \Mor(\Cat)$ und $(S \xrightarrow{s} T) \in S$.
\end{defn}

% III.2.2
\begin{bem}
  Es gibt einen kanonischen Funktor $Q : \Cat \to \Cat[S^{-1}]$, der die Morphismen aus $S$ auf Isomorphismen abbildet.
\end{bem}

% III.2.2
\begin{lem}
  Die Lokalisierung erfüllt folgende universelle Eigenschaft: \\
  Jeder andere Funktor $F : \Cat \to \Dat$, der die Morphismen aus $S$ auf Isomorphismen abbildet, faktorisiert eindeutig über $Q$, \dh{} es gibt gibt genau einen Funktor $G : \Cat[S^{-1}] \to \Dat$ mit $F = G \circ Q$.
\end{lem}

\begin{bspe}
  \begin{itemize}
    % Übungsblatt 18, Aufgabe 2
    \item Die Lokalisierung der Kategorie der metrischen Räume an der Klasse der Bilipschitzabbildungen mit dichtem Bild ist äquivalent zur Kategorie der vollständigen metrischen Räume.
    % Übungsblatt 17, Aufgabe 4
    \item Die Lokalisierung der Kategorie der Garben auf einem topol. Raum $X$ an der Klasse der Morphismen, die halmweise Isomor- phismen sind, ist äquivalent zur Kategorie der Garben auf $X$.
  \end{itemize}
\end{bspe}

% III.2.1
\begin{defn}
  Die \emph{abgeleitete Kategorie} $\Der(\Aat)$ der abelschen Kategorie $\Aat$ ist die Lokalisierung $\Kom(\Aat)[S^{-1}]$ von $\Kom(\Aat)$ an der Klasse $S \subset \Mor(\Kom(\Aat))$ der Quasiisomorphismen.
\end{defn}

% III.2.3
\begin{bem}
  Der Homologiefunktor $H_\bullet : \Kom(\Aat) \to \Kom_0(\Aat)$ bildet Quasiisomorphismen auf Isomorphismen ab. Somit gibt es einen Funktor $k : \Der(\Aat) \to \Kom_0(\Aat)$ mit $H_\bullet = k \circ Q$.
\end{bem}

% III.2.4
\begin{prop}
  Wenn die abelsche Kategorie $\Aat$ halbeinfach ist, dann ist $k : \Der(\Aat) \to \Kom_0(\Aat)$ eine Kategorienäquivalenz.
\end{prop}

% III.2.5 (Varianten)
\begin{defn}
  Die Kategorien der (nach links/rechts) beschränkten Ketten- komplexe sind die vollen Unterkategorien von $\Kom(\Aat)$ mit
  \begin{align*}
    \Ob(\Kom^+(\Aat)) & \coloneqq \Set{K_\bullet \in \Ob(\Kom(\Aat))}{\ex{i_0 \!\in\! \Z\!}\! \fa{i \leq i_0\!}\! K_i \!=\! 0}, \\
    \Ob(\Kom^-(\Aat)) & \coloneqq \Set{K_\bullet \in \Ob(\Kom(\Aat))}{\ex{i_0 \!\in\! \Z\!}\! \fa{i \geq i_0\!}\! K_i \!=\! 0}, \\
    \Ob(\Kom^b(\Aat)) & \coloneqq \Ob(\Kom^+(\Aat)) \cap \Ob(\Kom^+(\Aat)).
  \end{align*}
\end{defn}

% III.2.5 (Varianten)
\begin{defn}
  Die Lokalisierung dieser Kategorien an der Klasse der Quasi- isomorphismen wird mit $\Der^+(\Aat)$, $\Der^-(\Aat)$ bzw. $\Der^b(\Aat)$ bezeichnet.
\end{defn}

% III.2.5 (Varianten)
\begin{bem}
  Das Problem mit der bisherigen Definition der abgeleiteten Kategorie ist, dass es schwierig ist, sie gut zu verstehen, da ihre Morphismenmengen schwer zu begreifen sind.
\end{bem}

% III.2.6
\begin{defn}
  Sei $\Cat$ eine beliebige Kategorie.
  Eine Klasse $S \subset \Mor(\Cat)$ von Morphismen heißt \emph{(links-)lokalisierend}, wenn
  \begin{enumerate}
    \item $S$ ist abgeschlossen unter Kompositionen: $\id_X \in S$ für alle $X \in \Ob(\Cat)$ und $s \circ t \in S$ für alle passenden $s, t \in S$.
    \item Erweiterungsbedingung: Für alle $(X \xrightarrow{f} Y \xleftarrow{s \in S} Z) \in \Cat$ gibt es $W \!\in\! \Ob(\Cat)$, $(W \xrightarrow{s' \in S} X)$ und $(W \xrightarrow{f'} Z) \in \Cat$ mit $f \circ s' \!=\! s \circ f'$.
    \item Für parallele Morphismen $f, g \in \Hom_\Cat(X, Y)$ gilt
    \[ \ex{s \in S} s \circ f = s \circ g \implies \ex{t \in S} f \circ t = g \circ t. \]
  \end{enumerate}
\end{defn}

% selbst bewiesen
\begin{bem}
  Man kann zeigen: Wenn in Bedingung 2 auch $f \in S$ gilt, dann findet man ein passendes $f'$ auch in $S$.
\end{bem}

% III.2.7a)
\begin{bem}
  Aus der Bedingung 2 ergibt sich eine vereinfachte Darstellung der Elemente von $\Hom_{\Cat[S^{-1}]}(X, Y)$, wenn $S$ lokalisierend ist: Nämlich kann man in den formalen Ketten alle künstlichen Inversen auf die linke Seite verfrachten. \\
  Die Bedingung 1 erlaubt das Zusammenfassen von künstl. Inversen.
\end{bem}

% III.2.7b)
\begin{bem}
  Die Klasse der Qis ist i.\,A. (leider) nicht lokalisierend. :-(
\end{bem}

% III.2.8
\begin{lem}
  Sei $\Cat$ eine beliebige Kategorie, $S \subset \Mor(\Cat)$ lokalisierend. \\
  Dann gibt es folgende äquivalente Konstruktion von $\Cat[S^{-1}]$:
  \begin{itemize}
    \item $\Ob(\Cat[S^{-1}]) \coloneqq \Ob(\Cat)$
    \item Morphismen von $X$ nach $Y$ in $\Cat[S^{-1}]$ sind Äquivalenzklassen von \emph{Dächern}, das sind Diagramme der Form
    \begin{centertikzcd}[column sep=1.4cm, row sep=0.3cm]
      & \arrow{ld}[swap]{s \in S} X' \arrow{rd}{f} \\
      X && Y
    \end{centertikzcd}
    \item Zwei Dächer (mit Spitzen $X'$ und $X''$) heißen äquivalent, wenn es ein kommutatives Diagramm von folgender Form gibt:
    \begin{centertikzcd}[column sep=1.4cm, row sep=0.3cm]
      && \arrow{ld}[swap]{r \in S} X''' \arrow{rd}{h} \\
      & |[alias=XP]| X' \arrow[rrrd, swap, "f"] &&
      \arrow[llld, crossing over, "t \in S"] X'' \arrow{rd}{g} \\
      X \arrow[from=XP, swap, "s \in S"] &&&& Y
    \end{centertikzcd}
    Man kann Dächer auch als Brüche mit Nenner $s \in S$ und Zähler $f$ auffassen. Die Äquivalenzrelation macht es möglich, parallele Dächer auf einen gemeinsamen Nenner zu bringen.
    \item Der Identitätsmorphismus auf $X$ ist das Dach $X \xleftarrow{\id_X} X \xrightarrow{\id_X} X$.
    \item Mit Bedingung 2a kann man Morphismen verknüpfen:
    \begin{centertikzcd}[column sep=1.4cm, row sep=0.3cm]
      && X'' \arrow[ld, dashed, swap, "t' \in S"] \arrow[rd, dashed, "f'"] \\
      & X' \arrow[ld, swap, "s \in S"] \arrow[rd, "f"] &&
      Y' \arrow[ld, swap, "t \in S"] \arrow[rd, "g"] \\
      X && Y && Z
    \end{centertikzcd}
    \item Der Funktor $Q : \Cat \to \Cat[S^{-1}]$ ist gegeben durch
    \[ (X \xrightarrow{f} Y) \mapsto [ X \xleftarrow{\id_X} X \xrightarrow{f} Y ]. \]
    \item Diese Definition der Lokalisierung erfüllt dieselbe universelle Eigenschaft und ist daher äquivalent zur bisherigen.
  \end{itemize}
\end{lem}

% Übungsblatt 18, Aufgabe 5b)
\begin{bem}
  Wenn $\Cat$ präadditiv ist, dann auch $\Cat[S^{-1}]$: Man addiert Dächer, indem man sie zunächst auf einen gemeinsamen Nenner bringt und dann die Zähler (die rechten Morphismen) addiert.
\end{bem}

% III.2.9
\begin{bem}
  Statt Linksdächern kann man auch Rechtsdächer der Form $(X \xrightarrow{f} X' \xleftarrow{s \in S} Y)$ verwenden. Dazu muss $S$ rechtslokalisierend sein, \dh{} die dualen Bedingungen zu 1-3 erfüllen.
\end{bem}

% III.2.10
\begin{prop}
  Sei $\Cat$ eine beliebige Kategorie, $S \subset \Mor(\Cat)$ lokalisierend und $\Bat \subset \Cat$ eine volle Unterkategorie, sodass $S_\Bat \coloneqq S \cap \Mor(\Bat)$ lokalisierend in $\Bat$ ist.
  Der kanonische Funktor $\Bat[S_\Bat^{-1}] \to \Cat[S^{-1}]$ ist volltreu, wenn eine der folgenden Bedingungen erfüllt ist:
  \begin{itemize}
    \item Für alle $(X' \xrightarrow{s \in S} X)$ mit $X \in \Ob(\Bat)$ gibt es einen Morphismus $(X'' \xrightarrow{f} X') \in \Cat$ mit $s \circ f \in S$ und $X'' \in \Ob(\Bat)$.
    \item Die duale Bedingung.
  \end{itemize}
\end{prop}

% TODO: III.3 (Dreiecke als verallgemeinerte exakte Tripel)

\end{document}
