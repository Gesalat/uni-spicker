\documentclass{cheat-sheet}

\pdfinfo{
  /Title (Zusammenfassung Homologische Algebra)
  /Author (Tim Baumann)
}

\usepackage{pgffor} % \foreach-Schleifen

\newcommand{\nspace}[1]{\foreach \i in {1,...,#1}{ \! }} % Negativer Abstand
\newcommand{\SetC}{\mathbf{Set}} % Kategorie der Mengen
\newcommand{\sSet}{\mathbf{sSet}} % Kategorie der simplizialen Mengen
\newcommand{\Top}{\mathbf{Top}} % Kategorie der topologischen Räume
\newcommand{\AbGrp}{\mathbf{AbGrp}} % Kategorie der abelschen Gruppen
\newcommand{\RMod}{\mathbf{R\text{-}Mod}} % Kategorie der R-Moduln
\newcommand{\Ouv}{\mathbf{Ouv}} % Kategorie der offenene Mengen eines topol. Raumes
\newcommand{\op}{\mathrm{op}} % opposite category
\DeclareMathOperator{\sk}{sk} % Skelett
\newcommand{\CC}[1]{{#1}_{\bullet}} % Kettenkomplex (chain complex)
\newcommand{\CCC}[1]{{#1}^{\bullet}} % Ko-Kettenkomplex (cochain complex)
\DeclareMathOperator{\Hom}{Hom} % Homomorphisms
\newcommand{\Fais}{\mathcal{F}} % Faisceau-F (Garbe auf französisch)
\newcommand{\Garb}{\mathcal{G}} % Garben-G
\newcommand{\Harb}{\mathcal{H}} % Garben-H
\newcommand{\keS}{k.\,e.\,S.} % kurze exakte Sequenz
\newcommand{\leS}{l.\,e.\,S.} % lange exakte Sequenz
\newcommand{\Lg}{\mathfrak{g}} % Lie-g
\DeclareMathOperator{\Sing}{Sing} % Homomorphisms
\newcommand{\Cat}{\mathcal{C}} % Category-C
\newcommand{\Dat}{\mathcal{D}} % Category-D
\DeclareMathOperator{\Ob}{Ob} % Objects (of a category)
\newcommand{\inlineitem}[1]{\textbullet \enspace #1} % List-Item im Fließtext


% Kleinere Klammern
\delimiterfactor=701


\begin{document}

\maketitle{Zusammenfassung Homologische Algebra}

% Basierend auf dem Buch "Methods of Homological Algebra" von Gelfand und Manin

% Vorgezogen:
% II. Hauptbegriffe der Kategorientheorie

% II.1 Die Sprache der Kategorien und Funktoren

\section{Kategorientheorie}

\begin{bem}
  Die \href{http://timbaumann.info/uni-spicker/topo.pdf}{Topologie-Zusammenfassung} bietet eine Übersicht über Grundbegriffe der Kategorientheorie.
\end{bem}

% Ausgelassen: II.1.1-II.1.7

% Aus II.1.8
\begin{defn}
  Sei $\Cat$ eine Kategorie, $X \in \Ob(\Cat)$. Der \emph{kovariante Hom-Funktor} $h_X : \Cat \to \SetC$ ist definiert durch
  \[
    h_X(Y) \coloneqq \Hom(X, Y), \quad
    h_X(f : Y \to Y')(g : X \to Y) \coloneqq f \circ g.
  \]
  Allgemeiner gibt es den Funktor $\Hom : \Cat^\op \times \Cat \to \SetC$ mit
  \[
    \Hom(h : X' \to X, f : Y \to Y')(g : X \to Y) \coloneqq f \circ g \circ h.
  \]
\end{defn}

% Ausgelassen: II.1.9-II.1.11

% II.1.12
\begin{defn}
  Eine Kategorie $\Dat$ heißt \emph{Unterkategorie} einer Kategorie $\Cat$ (notiert $\Dat \subseteq \Cat$), wenn für alle geeigneten $X$, $Y$, $f$, $g$ gilt:
  \[
    \Ob(\Dat) \subseteq \Ob(\Cat), \enspace
    \Hom_\Dat(X, Y) \subseteq \Hom_\Cat(X, Y) \enspace \text{und} \enspace
    f \circ_\Dat g = f \circ_\Cat g.
  \]
\end{defn}

\begin{defn}
  Eine Unterkategorie $\Dat \subseteq \Cat$ heißt \emph{voll}, wenn
  \[ \fa{X, Y \in \Ob(\Dat)} \Hom_\Dat(X, Y) = \Hom_\Cat(X, Y). \]
\end{defn}

\begin{defn}
  Ein Funktor $F : \Cat \to \Dat$ heißt \ldots
  \begin{itemize}
    \item \ldots{} \emph{treu}, wenn für alle $X, Y \in \Ob(\Cat)$ die Abbildung
    \[ F : \Hom_\Cat(X, Y) \to \Hom_\Dat(FX, FY) \]
    injektiv ist.
    \item \ldots{} \emph{voll}, wenn diese Abb. für alle $X, Y \in \Ob(\Cat)$ surjektiv ist.
  \end{itemize}
\end{defn}

\begin{bem}
  Die Einbettung einer Unterkategorie ist ein volltreuer Funktor.
\end{bem}

\begin{defn}
  \begin{itemize}
    \item Ein Objekt $X \in \Ob(\Cat)$ heißt \emph{initiales Objekt}, falls für alle $Y \in \Ob(Y)$ genau ein Morphismus $f \in \Hom_\Cat(X, Y)$ existiert.
    \item Ein Objekt $Z \in \Ob(\Cat)$ heißt \emph{terminales Objekt}, falls für alle $Y \in \Ob(Y)$ genau ein Morphismus $f \in \Hom_\Cat(Y, Z)$ existiert.
  \end{itemize}
\end{defn}

% Ausgelassen: II.1.13

% II.2 (Kategorien und Strukturen, Äquivalenz von Kategorien)

% Ausgelassen: II.2.1-II.2.4

\begin{defn}
  Ein Funktor $F : \Cat \to \Dat$ ist ein \emph{Kategorienäquivalenz}, falls es einen Funktor $G : \Dat \to \Cat$ mit $F \circ G \simeq \Id_\Dat$ und $G \circ F \simeq \Id_\Cat$ gibt. \\
  Die Funktoren $F$ und $G$ heißen dann zueinander \emph{quasiinvers}.
\end{defn}

\begin{defn}
  Zwei Kategorien $\Cat$, $\Dat$ heißen zueinander \emph{äquivalent}, wenn es eine Kategorienäquivalenz $F : \Cat \to \Dat$ gibt.
\end{defn}

% Ausgelassen: II.2.6

% II.2.7
\begin{prop}
  $F : \Cat \to \Dat$ ist genau dann eine Kategorienäquivalenz, wenn:
  \inlineitem{$F$ ist volltreu,} \quad
  %\item Jedes Objekt $X \in \Ob(\Dat)$ ist isomorph zu einem Objekt $F(X)$, $X \in \Ob(\Cat)$.
  \inlineitem{$\fa{Y \in \Ob(\Dat)} \ex{X \in \Ob(\Cat)} Y \cong F(X)$}
\end{prop}

% Ausgelassen: II.2.8 (Galois-Theorie)

\section{Simpliziale Mengen}

% §I.1 Triangulierte Räume

\begin{defn}
  \emph{Verklebedaten} sind gegeben durch einen Funktor
  \[ X : \Delta_{\text{strikt}}^\op \to \SetC. \]
  Dabei ist $\Delta_{\text{strikt}}$ die Kategorie mit den Mengen
  $[n] \coloneqq \{ 0, 1, \ldots, n \}$ für $n \in \N$ als Objekten und streng monotonen Abbildungen.
\end{defn}

\begin{nota}
  $X_{(n)} \coloneqq X([n])$ heißt Menge der $n$-Simplizes.
\end{nota}

\begin{defn}
  Das \emph{Standard-$n$-Simplex} $\Delta_n \subset \R^{n+1}$ ist die von den $(n{+}1)$ Standardbasisvektoren aufgespannte affinlineare Hülle. Eine streng monotone Abb $f : [n] \to [m]$ induziert durch Abbilden des $i$-ten Basisvektors auf den $f(i)$-ten eine Inklusion $\Delta_f : \Delta_n \to \Delta_m$, 
\end{defn}

\begin{defn}
  Die \emph{geometrische Realisierung} von Verklebedaten $X$ ist der topologische Raum
  \[ \abs{X} \coloneqq \left( \coprod_{n \in \N} (\Delta_n \times X_{(n)}) \right) / R \]
  Dabei ist $X_{(n)}$ diskret. Die Äquivalenzrelation $R$ wird erzeugt von
  \[
    (\Delta_f(t), x) \sim (t, X(f)(x)) \enspace
    \text{mit $t \in \Delta_m$, $x \in X_{(n)}$, $f : [m] {\to} [n]$ s.m.s.}
  \]
\end{defn}

% Ausgelassen: Proposition I.1.3

\begin{defn}
  Das \emph{$k$-Skelett} $\sk_k X$ von Verklebedaten $X$ ist definiert durch
  \[
    (\sk_k X)_{(n)} \coloneqq
    \Set{ x \in X_{(n)} }{ n \leq k }, \enspace
    %\begin{cases}
    %  X_{(n)}, & \text{falls $n \leq k$,} \\
    %  \emptyset, & \text{sonst.}
    %\end{cases}, \quad
    (\sk_k X)(f) \coloneqq X(f) \enspace \text{sofern möglich}
  \]
\end{defn}

% Ausgelassen: I.1.4 (Triangulierung des Produktes zweier Simplizes)

% §I.2 Simpliziale Mengen

\begin{defn}
  Eine \emph{simpliziale Menge} ist ein Funktor
  \[ X : \Delta^\op \to \SetC. \]
  Dabei ist $\Delta$ die Kategorie mit den Mengen
  $[n] \coloneqq \{ 0, 1, \ldots, n \}$ für $n \in \N$ als Objekten und monotonen Abbildungen.
\end{defn}

\begin{nota}
  $X_n \coloneqq X([n])$ heißt Menge der $n$-Simplizes.
\end{nota}

% vorgezogen: aus I.2.15
\begin{defn}
  Eine \emph{simpliziale Abbildung} zwischen simplizialen Mengen $X$ und $Y$ ist eine natürliche Transformation zwischen den beiden Funktoren $X, Y : \Delta^\op \to \SetC$.
\end{defn}

% vorgezogen: aus I.2.15
\begin{defn}
  Die Kategorie der simplizialen Mengen ist die Funktorkategorie $\sSet \coloneqq [\Delta^\op, \SetC]$.
\end{defn}

\begin{defn}
  Die \emph{geometrische Realisierung} einer simplizialen Menge $X$ ist der topologische Raum
  \[ \abs{X} \coloneqq \left( \coprod_{n \in \N} (\Delta_n \times X_{n}) \right) / R \]
  %Dabei ist $\Delta_n \subset \R^{n+1}$ das Standard-$n$-Simplex und $X_{n}$ trägt die diskrete Topologie.
  Die Äquivalenzrelation $R$ wird dabei erzeugt von
  \[
    (\Delta_f(t), x) \sim (t, X(f)(x)) \enspace
    \text{mit $t \in \Delta_m$, $x \in X_n$ u. $f \in \Hom_\Delta([m], [n])$.}
  \]
\end{defn}

\begin{defn}
  Ein topologischer Raum heißt \emph{trianguliert}, wenn er die Realisierung von Verklebedaten ist.
\end{defn}

\begin{defn}
  Der \emph{Nerv} einer Überdeckung $X = \cup_{\alpha \in A} U_\alpha$ eines topologischen Raumes ist die simpliziale Menge
  \begin{align*}
    X_n & \coloneqq \Set{(\alpha_0, \nldots, \alpha_n) \in A^{n+1}}{ U_{\alpha_0} \cap \nldots \cap U_{\alpha_n} \not= \emptyset } \\
    X(f)(\alpha_0, \nldots, \alpha_n) & \coloneqq (\alpha_{f(0)}, \nldots, \alpha_{f(m)}) \quad \text{für } f : [m] \to [n].
  \end{align*}
\end{defn}

\begin{bem}
  Falls die Überdeckung lokal endlich ist und alle nichtleeren, endlichen Schnitte $U_{\alpha_1} \cap \ldots \cap U_{\alpha_n}$ zusammenziehbar sind, so ist die geom. Realisierung des Nerves der Überdeckung homotopieäq. zu $X$.
\end{bem}

% I.2.4
\begin{defn}
  Sei $Y$ ein topol. Raum. Die simpliziale Menge $X$ der \emph{singulären Simplizes} in $Y$ ist
  \[
    X_n \coloneqq \{ \, \text{stetige Abb. } \sigma : \Delta_n \to Y \, \}, \quad
    X_n(f)(\sigma) \coloneqq \sigma \circ \Delta_f.
  \]
\end{defn}

\begin{bem}
  Diese Konstruktion stiftet eine Funktor $\Sing : \Top \to \sSet$.
\end{bem}

% I.2.5
\begin{defn}
  $\Delta[p]_n \coloneqq \{ \, g : [n] \to [p] \text{ monoton steigend} \, \}$, $\Delta[p](f)(g) \coloneqq g \circ f$
\end{defn}

% I.2.8
\begin{defn}
  Der \emph{klassifizierende Raum} einer Gruppe $G$ ist gegeben durch die Realisierung der simpl. Menge $BG$ mit $(BG)_n \coloneqq G^n$ und
  \begin{align*}
    BG(f : [m] \to [n])(g_1, \ldots, g_n) \coloneqq
    (h_1, \ldots, h_m), \quad h_i = \nspace{3} \prod_{j=f(i-1)+1}^{f(i)} \nspace{3} g_j.
    %(g_{f(0)+1} \cdot \ldots \cdot g_{f(1)}, g_{f(1)+1} \cdot \ldots \cdot g_{f(2)}, \ldots, g_{f(m-1)+1} \cdot \ldots \cdot g_{f(m)})
  \end{align*}
\end{defn}

% I.2.9
\begin{defn}
  Ein $n$-Simplex $x \in X_n$ heißt \emph{degeneriert}, falls eine monotone surjektive Abbildung $f : [n] \to [m]$, $n > m$ und ein Element $y \in X_m$ existiert mit $x = X(f)(y)$.
\end{defn}

% I.2.6
\begin{defn}
  Seien $X$ Verklebedaten. Wir konstruieren eine dazugehörende simpliziale Menge $\tilde{X}$ wie folgt:
  \[ \tilde{X}_n \coloneqq \Set{ (x, g) }{ x \in X_{(k)}, g : [n] \to [k] \text{ monoton und surjektiv} }, \]
  Für eine monotone Abbildung $f : [m] \to [n]$ und $(x, g) \in \tilde{X}_n$ schreiben wir zunächst $g \circ f = f_1 \circ f_2$ mit einer Injektion $f_1$ und einer Surjektion $f_2$ und setzen
  $\tilde{X}(f)(x, g) \coloneqq (X(f)(x), f_2)$.
\end{defn}

% I.2.7
\begin{prop}
  Eine simpliziale Menge $\tilde{X}$ kann genau dann aus (dann eindeutigen) Verklebedaten gewonnen werden, falls für alle nicht-degenerierten Simplizes $x \in \tilde{X}_n$ und streng monotonen Abbildungen $f : [m] \to [n]$ auch $\tilde{X}(f)(x) \in \tilde{X}_m$ nicht degeneriert ist.
\end{prop}

% Ausgelassen: I.2.10-12

% I.2.13
\begin{prop}
  Seien $X$ Verklebedaten, $\tilde{X}$ die entsprechende simpliziale Menge. Dann gilt $\abs{X} \approx \abs{\tilde{X}}$.
\end{prop}

% I.2.14
\begin{defn}
  Das \emph{$k$-Skelett} $\sk_k X$ einer simplizialen Menge $X$ ist geg. durch
  \[ (\sk_k X)_n \coloneqq \Set{ X(f)(x) }{ p \leq k, f : [n] \to [p] \text{ monoton}, x \in X_p }. \]
\end{defn}

\begin{defn}
  Eine simpliziale Menge $X$ hat \emph{Dimension} $n$, falls $X = \sk_n X$.
\end{defn}

\begin{prop}
  Geom. Realisierung ist ein Funktor $\abs{\blank} : \sSet \to \Top$.
\end{prop}

\begin{bspe}
  \begin{itemize}
    \item Eine Überdeckung $(U_\alpha)_{\alpha \in A}$ eines topologischen Raumes ist Verfeinerung von $(V_\beta)_{\beta \in B}$, wenn es eine Abbildung $\psi : A \to B$ gibt, sodass $U_\alpha \subset V_{\psi(\alpha)}$ für alle $\alpha \in A$. Dies induziert eine simpliziale Abb. zwischen den Nerven der Überdeckungen durch
    \[ F_n(\alpha_0, \ldots, \alpha_n) \coloneqq (\psi(\alpha_0), \ldots, \psi(\alpha_n)). \]
    % Ausgelassen: I.2.17
    \item Ein Gruppenhomomorphismus $\phi : G \to H$ stiftet eine Abbildung $BG \to BH$ zwischen den klassifizierenden Räumen durch
    \[ F(g_1, \ldots, g_n) \coloneqq (\phi(g_1), \ldots, \phi(g_n)). \]
  \end{itemize}
\end{bspe}

% §1.3 Simpliziale topologische Räume und das Eilenberg-Zilber-Theorem

% I.3.2
\begin{defn}
  Ein \emph{simplizialer topologischer Raum} ist ein Funktor
  \[ X : \Delta^\op \to \Top. \]
\end{defn}

\begin{bem}
  Die geometrische Realisierung eines simplizialen topol. Raumes ist definiert wie die einer simplizialen Menge mit dem Unterschied, dass $X_n$ im Allg. nicht die diskrete Topologie trägt.
\end{bem}

% I.3.3
\begin{defn}
  Eine \emph{bisimpliziale Menge} ist ein Funktor
  \[ X : \Delta^\op \times \Delta^\op \to \SetC. \]
\end{defn}

\begin{nota}
  $X_{nm} \coloneqq X([n],[m])$
\end{nota}

% I.3.4
\begin{bsp}
  Das \emph{direkte Produkt} von simplizialen Mengen $X$ und $Y$ ist die bisimpliziale Menge
  \[
    (X \times Y)_n \coloneqq X_n \times Y_n, \quad
    (X \times Y)(f, g)(x, y) \coloneqq (f(x), g(y)).
  \]
\end{bsp}

\begin{samepage}

% I.3.5
\begin{defn}
  Die \emph{Diagonale} $DX$ einer bisimplizialen Menge $X$ ist die simpliziale Menge mit
  $(DX)_n \coloneqq X_{nn}$ und $DX(f) \coloneqq X(f, f)$.
\end{defn}

% I.3.6
\begin{defn}
  Sei $X$ eine bisimpliziale Menge.
  \begin{itemize}
    \item Setze $\abs{X}^D \coloneqq \abs{DX}$.
    \item Definiere einen simplizialen topologischen Raum $X^I$ durch
    \[ X^I_n \coloneqq \abs{X_{\bullet n}}, \quad X^I(g) \coloneqq \abs{X(\id, g)}. \]
    Setze $\abs{X}^{I,II} \coloneqq \abs{II,I}$.
    \item Definiere analog $\abs{X}^{II,I}$.
  \end{itemize}
\end{defn}

\begin{satz}[\emph{Eilenberg-Zilber}]
  $\abs{X}^D \cong \abs{X}^{I,II} \cong \abs{X}^{II,I}$ kanonisch.
\end{satz}

\section{Garben}

\end{samepage}

% I.5 (Garben)

% Ausgelassen: I.5.1 (Beispiele von Garben)

\begin{defn}
  \begin{itemize}
    \item Eine mengenwertige \emph{Prägarbe} $\Fais$ auf einem topol. Raum $X$ ist ein Funktor
    $\Fais : \Ouv(X)^\op \to \SetC$.
    Dabei ist $\Ouv(X)$ die Präordnungs-Kategorie der offenen Teilmengen von $X$ geordnet durch Inklusion.
    \item Allgemeiner ist eine $\mathcal{C}$-wertige Prägarbe ein Funktor $\Fais : \Ouv(X)^\op \to \mathcal{C}$ (z.\,B. $\mathcal{C} = \AbGrp, \RMod, \Top$).
    \item Ein Morphismus zwischen Prägarben $\Fais$ und $\Garb$ auf demselben topol. Raum ist eine natürliche Transformation zwischen $\Fais$ und $\Garb$.
  \end{itemize}
\end{defn}

\begin{nota}
  Sei $\Fais$ eine Prägarbe
  \begin{itemize}
    \item $\Gamma(U, \Fais) \coloneqq \Fais(U)$ heißt Menge der \emph{Schnitte} von $\Fais$ über $U$.
    \item $r_{UV} \coloneqq \Fais(V \subseteq U) : \Fais(U) \to \Fais(V)$ heißt \emph{Restriktionsabb}.
    \item $x|_V \coloneqq r_{UV}(x)$ für $V \subseteq U$ und $x \in \Fais(U)$ heißt \emph{Einschränkung} von $x$ auf $V$.
  \end{itemize}
\end{nota}

\begin{defn}
  Eine \emph{Garbe} auf einem topol. Raum $X$ ist eine Prägarbe $\Fais$, für die gilt:
  Für alle Familien $(U_i)_{i \in I}$ von offenen Teilmengen und Schnitten $(s_i \in \Fais(U_i))_{i \in I}$, die miteinander verträglich sind, \dh{}
  \[ \fa{i, j \in I} s_i|_{U_i \cap U_j} = s_j|_{U_i \cap U_j}, \]
  gibt es genau einen Schnitt $s \in \Fais(\cup_{i \in I} U_i)$ mit $\fa{i \in I} s_i = s|_{U_i}$.\\
  Ein Morphismus zw. Garben ist ein Morphismen zw. den Prägarben.
\end{defn}

\begin{bem}
  Sei $\Fais$ eine (Prä-)Garbe auf $X$ und $U \subseteq X$ offen. Dann definiert $(\Fais|U)(V) \coloneqq \Fais(U \cap V)$ eine (Prä-)Garbe auf $U$.
\end{bem}

% I.5.3 (Prägarben und Garben von strukturierten Mengen)

% Ausgelassen: Garbe von $O(U)$-Modulen, wobei $O$ selbst eine Ringprägarbe auf $X$ ist.

\begin{defn}
  Eine Sequenz $\Fais \to \Garb \to \Harb$ von (Prä-)Garben abelscher Gruppen auf $X$ heißt \emph{exakt} bei $\Garb$, falls für alle offenen $U \subset X$ die Sequenz $\Fais(U) \to \Garb(U) \to \Harb(U)$ exakt bei $\Garb(U)$ ist.
\end{defn}

\begin{defn}
  Sei $f : \Fais \to \Garb$ ein Morphismus von Prägarben auf $X$. Definiere Prägarben $\mathcal{K}$ und $\mathcal{C}$ auf $X$ durch
  \[
    \mathcal{K}(U) \coloneqq \ker (f_U : \Fais(U) \to \Garb(U)), \quad
    \mathcal{C}(U) \coloneqq \Garb(U) / \im (f_U).
  \]
\end{defn}

\begin{prop}
  Sei $f : \Fais \to \Garb$ sogar ein Morphismus von Garben. \\
  Dann ist auch $\mathcal{K}$ eine Garbe.
\end{prop}

\begin{acht}
  Aber $\mathcal{C}$ ist im Allgemeinen keine Garbe!
\end{acht}

% I.5.5 (Keime und Halme)

\begin{defn}
  Sei $\Fais$ eine Garbe auf $Y$. Der \emph{Halm} von $\Fais$ in $y \in Y$ ist
  \begin{align*}
    \Fais_y & \coloneqq \Set{(U, s)}{U \subseteq Y \text{offen}, y \in U, s \in \Fais(U)} / {\sim}, \\
    (U, s) \sim (V, t) & \coloniff \ex{W \subset U \cap V \text{ offen}, y \in W} s|_W = t|_W.
  \end{align*}
\end{defn}

\begin{nota}
  $s_y \coloneqq [(U, s)]$ für $s \in \Fais(U)$ mit $y \in U$.
\end{nota}

\begin{sprech}
  Elemente $[t] \in \Fais_y$ heißen \emph{Keime} in $y$.
\end{sprech}

\begin{defn}
  Sei $\Fais$ eine Garbe auf $Y$, $Z \subseteq Y$ beliebig. Definiere
  \[ \Gamma(Z, \Fais) \coloneqq \varinjlim \Gamma(U, \Fais), \]
  wobei der Limes über alle offenen $U \subset X$ mit $Z \subseteq U$ läuft.
\end{defn}

\begin{beob}
  $\Fais_y = \Gamma(\{y\}, \Fais)$
\end{beob}

\begin{defn}
  Der \emph{Totalraum} $F$ einer Prägarbe $\Fais$ auf $Y$ ist
  \[ F \coloneqq \coprod_{y \in Y} \Fais_y \]
  mit der Topologie erzeugt durch die Mengen
  \[
    \Set{s_y}{ y \in U } \quad
    \text{für $U \subseteq X$ offen, $s \in \Fais(U)$.}
  \]
\end{defn}

\begin{bem}
  Mit dieser Topologie ist die Projektion $\pi : F \to Y$ stetig und ein lokaler Homöomorphismus.
\end{bem}

\begin{defn}
  Sei $\Fais$ eine Prägarbe auf $Y$. Die \emph{Garbifizierung} $\Fais^+$ von $\Fais$ ist die Garbe der stetigen Schnitte von $\pi : F \to Y$, also
  \[ \Fais^+(U) \coloneqq \Set{f : U \to F}{\pi \circ f = (i : U \hookrightarrow Y)}. \]
\end{defn}

\begin{prop}
  Es ex. ein kanonischer Morphismus $f : \Fais \to \Fais^+$ def. durch
  \[ s \in \Fais(U) \enspace \mapsto \enspace (y \mapsto s_y : U \to F). \]
  Wenn $\Fais$ schon eine Garbe ist, dann ist $f$ ein Isomorphismus.
\end{prop}

% Ausgelassen: I.5.7 (Hauptklassen von Garben)

% I.5.8
\begin{defn}
  Sei $A$ eine Menge (oder ab. Gruppe, \ldots), $Y$ ein topol. Raum.
  \begin{itemize}
    \item Die \emph{konstante Prägarbe} $\mathbf{A}$ mit Faser $A$ auf $Y$ ist def. durch
    \[
      \mathbf{A}(U) \coloneqq A, \quad
      r_{UV} \coloneqq \id_A \quad
      \text{für alle $V \subseteq U \subseteq Y$.}
    \]
    \item Die \emph{konstante Garbe} mit Faser $A$ ist die Garbifizierung $\mathcal{A} = \mathbf{A}^+$ von $\mathbf{A}$.
  \end{itemize}
\end{defn}

\begin{samepage}

% I.5.9
\begin{defn}
  Eine Garbe $\Fais$ auf $Y$ heißt \emph{lokal konstant}, falls jeder offene Punkt in $Y$ eine offene Umgebung $U$ besitzt, sodass $F|U$ isomorph zu einer konstanten Garbe ist.
\end{defn}

\begin{defn}
   Eine Garbe $\Fais$ auf einem topologischen Raum $Y$ heißt \ldots{}
  \begin{itemize}
    \item \ldots{} \emph{welk} (flabby, flasque), wenn die Einschränkungsabbildungen
    \[ \Gamma(Y, \Fais) \to \Gamma(U, \Fais) \]
    für alle {\em offenenen} $U \subseteq Y$ surjektiv sind.
    \item \ldots{} \emph{weich} (soft, mou), wenn die Einschränkungsabbildungen
    \[ \Gamma(Y, \Fais) \to \Gamma(A, \Fais) \]
    für alle {\em abgeschlossenen} $A \subseteq Y$ surjektiv sind.
  \end{itemize}
\end{defn}

\begin{defn}
   Eine Garbe $\Fais$ ab. Gruppen auf einem topol. Raum $Y$ heißt \emph{fein} (fine, fin), wenn für je zwei disjunkte, abgeschlossene Teilmengen $A_1, A_2 \subseteq Y$ ein Garbenmorphismus $\alpha : \Fais \to \Fais$ existiert, sodass $\alpha$ auf einer offenen Umgebung von $A_1$ Null und auf einer offenen Umgebung von $A_2$ die Identität ist.
\end{defn}

\section{Komplexe und (Ko-)Homologie}

\end{samepage}

% I.4 Homologie und Kohomologie

% I.4.3
\begin{defn}
  \begin{itemize}
    \item Ein \emph{Kettenkomplex} $\CC{C}$ ist eine Folge $(C_n)_{n \in \N}$ von abelschen Gruppen und Gruppenhomomorphismen $\partial_n : C_n \to C_{n-1}$ mit der Eigenschaft $\partial_{n-1} \circ \partial_n = 0$.
    \item Ein \emph{Kokettenkomplex} $\CCC{C}$ ist eine Folge $(C^n)_{n \in \N}$ von abelschen Gruppen und Gruppenhomomorphismen $\delta^n : C^n \to C^{n+1}$ mit der Eigenschaft $\delta^{n+1} \circ \delta^n = 0$.
  \end{itemize}
\end{defn}

% I.4.4
\begin{defn}
  Sei $\CC{C}$ ein Kettenkomplex.
  \begin{itemize}
    \item $C_n$ heißt Gruppe der \emph{$n$-Ketten},
    \item $\partial : C_n \to C_{n-1}$ heißt \emph{Randabbildung},
    \item $Z_n(\CC{C}) \coloneqq \ker \partial_n \subset C_n(\CC{C})$ heißt Gruppe der \emph{$n$-Zykel},
    \item $B_n(\CC{C}) \coloneqq \im \partial_{n+1} \subset Z_n(\CC{C})$ heißt Gruppe der \emph{$n$-Ränder},
    \item $H_n(\CC{C}) \coloneqq Z_n(\CC{C}) / B_n(\CC{C})$ heißt \emph{$n$-te Homologiegruppe}.
  \end{itemize}
  Analog nennt man für einen Kokettenkomplex $\CCC{C}$
  \begin{itemize}
    \miniitem{0.46 \linewidth}{$\delta^n$ \emph{Korandabbildung},}
    \miniitem{0.41 \linewidth}{$C^n$ \emph{$n$-Koketten},}
    \miniitem{0.46 \linewidth}{$Z^n \coloneqq \ker \delta^n$ \emph{$n$-Kozykel},}
    \miniitem{0.45 \linewidth}{$B^n \coloneqq \im \delta^{n-1}$ \emph{$n$-Koränder},}
    \item $H^n(\CCC{C}) \coloneqq Z^n(\CCC{C}) / B^n(\CCC{C})$ $n$-te \emph{Kohomologiegruppe}.
  \end{itemize}
\end{defn}

% Vorgezogen: I.6.5 (Morphismen von Komplexen)
\begin{defn}
  Eine Morphismus $f : \CC{C} \to \CC{D}$ (bzw. $f : \CCC{C} \to \CCC{D}$) zwischen (Ko-)Kettenkomplexen ist eine Familie von Abbildungen
  \[
    (f_n : C_n \to D_n)_{n \in \N} \quad
    \text{(bzw. $(f^n : C^n \to D^n)_{n \in \N}$),}
  \]
  die mit den Randabbildungen verträglich sind, \dh{}
  \[
    f_{n-1} \circ \partial^C_n = \partial^D_n \circ f_n \quad
    \text{(bzw. $f^{n+1} \circ \delta_C^n = \delta_D^n \circ f^n$)} \quad
    \text{für alle $n$.}
  \]
\end{defn}

\begin{prop}
  $H_n$ (bzw. $H^n$) ist ein Funktor von der Kategorie der (Ko-)Kettenkomplexe in die Kategorie der abelschen Gruppen.
\end{prop}

% I.4.1
\begin{defn}
  Sei $X$ eine simpl. Menge. Sei $C_n(X)$ die von den $n$-Simplizes $X_n$ erzeugte abelsche Gruppe (\dh{} die Gruppe der endl. formalen Linearkombinationen mit Koeffizienten in $\Z$). Sei $\delta_n^i : [n{-}1] \to [n]$ diejenige streng monotone Abb. mit $i \not\in \im \delta_n^i$. Definiere
  \[
    \partial_n : C_n(X) \to C_{n-1}(X), \quad
    \sum_{\sigma \in X_n} \lambda_\sigma \cdot \sigma \, \mapsto \nspace{3} \sum_{\sigma \in X_n} \lambda_\sigma \sum_{i=0}^n (-1)^i X(\partial_n^i)(\sigma).
  \]
\end{defn}

% I.4.2
\begin{prop}
  $(C_\bullet(X),\partial_\bullet)$ ist ein Kettenkomplex (\dh{} $\partial_{n-1} \circ \partial_n = 0$)
\end{prop}

\begin{defn}
  Sei $X$ eine simpl. Menge und $A$ eine ab. Gruppe. Dann ist \ldots
  \begin{itemize}
    \item \ldots{} der \emph{Kettenkomplex} $(C_\bullet(X; A), \partial_\bullet)$ von $X$ \emph{mit Koeffizienten} in $A$ definiert durch
    \[
      C_n(X; A) \coloneqq C_n(X) \otimes_{\Z} A, \enspace
      \partial_n \coloneqq \partial_n \otimes \id : C_n(X; A) \to C_{n-1}(X; A).
    \]
    \item \ldots{} der \emph{Kokettenkomplex} $(C^\bullet(X; A), \delta^\bullet)$ von $X$ mit \emph{Koeffizienten} in $A$ definiert durch
    \begin{align*}
      & C^n(X; A) \coloneqq \Hom(C^n(X), A), \\
      \delta^n : \,\, & C^n(X; A) \to C^{n+1}(X; A), \enspace f \mapsto f \circ \delta_{n+1}.
    \end{align*}
  \end{itemize}
\end{defn}

\begin{beob}
  $C_n(X; \Z) = C_n(X)$
\end{beob}

\begin{nota}
  Sei $X$ eine simpliziale Menge. Setze
  \begin{itemize}
    \miniitem{0.48 \linewidth}{$H_n(X) \coloneqq H_n(C_\bullet(X))$,}
    \miniitem{0.48 \linewidth}{$H^n(X) \coloneqq H^n(C^\bullet(X; \Z))$,}
    \miniitem{0.48 \linewidth}{$H_n(X; A) \coloneqq H_n(C_\bullet(X; A))$,}
    \miniitem{0.48 \linewidth}{$H^n(X; A) \coloneqq H^n(C^\bullet(X; A))$.}
  \end{itemize}
\end{nota}

% I.4.5 (Geometrie von Ketten)
\begin{prop}
  Für jede simpl. Menge $X$ ex. ein kanonischer Isomorphismus
  \[ H_0(X, \Z) \cong \text{freie ab. Gr. erzeugt von Zshgskomponenten von $\abs{X}$}. \]
\end{prop}

\begin{defn}
  Der \emph{Kegel} $CX$ über Verklebedaten $X$ ist definiert durch
  \begin{align*}
    (CX)_{(0)} & \coloneqq X_{(0)} \amalg \{ \star \}, \quad (CX)_{(n)} \coloneqq X_{(n)} \amalg (X_{(n-1)} \times \{ \star \}), \\
    (CX)(f)(x) & \coloneqq X(f)(x), \\
    (CX)(f)(x,*) &  \coloneqq \begin{cases}
      X(i \mapsto f(i) - 1)(x), & \text{wenn $f(0) > 0$,} \\
      (X(i \mapsto f(i{+}1) - 1)(x), *), & \text{wenn $f(0) = 0$.}
    \end{cases}
  \end{align*}
\end{defn}

\begin{defn}
  Für Verklebedaten ist der zugeh. (Ko-)Kettenkomplex (mit Koeffizienten) genauso definiert wie für simpliziale Mengen.
\end{defn}

\begin{prop}
  $H_0(CX) = \Z$, $H_{>0}(CX) = 0$
\end{prop}

% Ausgelassen: I.4.6 Geometrie von Koketten :-)

% I.4.7 Koeffizientensysteme

% I.4.8
\begin{defn}
  Sei $X$ eine simpliziale Menge.
  \begin{itemize}
    \item Ein \emph{homol. Koeffizientensystem} $\mathcal{A}$ auf $X$ ist ein Funktor
    \[ \mathcal{A} : (1 \downarrow X) \to \AbGrp. \]
    Dabei ist $1 : \mathbf{1} \to \SetC$ der Funktor, der konstant $\{ \star \}$ ist (und $\mathbf{1}$ die Kategorie mit einem Objekt und einem Morphismus).\\
    Expliziter besteht ein Koeffizientensystem aus einer abelschen Gruppe $\mathcal{A}_\sigma$ für jedes $n$-Simplex $\sigma \in X_n$ und Abbildungen $\mathcal{A}(f, \sigma) : \mathcal{A}_\sigma \to \mathcal{A}_{X(f)(\sigma)}$ für alle $\sigma \in X_n$, $f \in \Hom_{\Delta}([m], [n])$ mit
    \[
      \mathcal{A}(\id, \sigma) = \id, \quad
      \mathcal{A}(f \circ g, \sigma) = \mathcal{A}(g, X(f)(\sigma)) \circ \mathcal{A}(f, \sigma).
    \]
    \item Ein \emph{kohomol. Koeffizientensystem} $\mathcal{B}$ auf $X$ ist ein Funktor
    \[ \mathcal{B} : (1 \downarrow X)^\op \to \AbGrp. \]
    % I.7.9
    \item Ein Morphismus zw. (ko-)homologischen Koeffizientensystemen auf derselben simpl. Menge ist eine natürliche Transformation.
  \end{itemize}
\end{defn}

% I.4.9 (Bemerkungen und Beispiele)

% Ausgelassen: a) Konstante Koeffizientensysteme, c) Koeffizientensystem auf BG
\begin{bsp}
  Sei $Y$ ein topol. Raum, $(U_\alpha)_{\alpha \in A}$ eine offene Überdeckung und $X$ deren Nerv. Dann definiert
  \begin{align*}
    \mathcal{F}_{\alpha_0, \ldots, \alpha_n} & \coloneqq \{ U_{\alpha_0} \cap \ldots \cap U_{\alpha_n} \to \R \text{ stetig} \}, \\
    \mathcal{F}(f, (\alpha_0, \ldots, \alpha_n))(\phi) & \coloneqq \text{passende Einschränkung von $\phi$}.
  \end{align*}
  ein kohomologisches Koeffizientensystem auf $X$.
\end{bsp}

% I.4.10 (Homologie und Kohomologie mit einem Koeffizientensystem)

\begin{defn}
  Sei $\mathcal{A}$ ein homologisches Koeffizientensystem auf einer simplizialen Menge $X$. Wir setzen
  \[ C_n(X; \mathcal{A}) \coloneqq \{ \text{ formale endl. Linearkomb. } \nspace{3} \sum_{\sigma \in X_n} \lambda_\sigma \cdot \sigma \text{ mit } \lambda_\sigma \in \mathcal{A}_\sigma \, \} \]
  und definieren $\partial_n : C_n(X; \mathcal{A}) \to C_{n-1}(X; \mathcal{A})$ durch
  \[ \sum_{\sigma \in X_n} \lambda_\sigma \cdot \sigma \enspace \mapsto \sum_{\sigma \in X_n} \sum_{i=0}^n \enspace (-1)^i \mathcal{A}(\partial_n^i, \sigma)(\lambda_\sigma) \cdot X(\partial_n^i)(\sigma). \]
  Die Homologiegruppen des so def. Kettenkomplexes $C_\bullet(X; \mathcal{A})$ heißen \emph{Homologiegruppen} von $X$ \emph{mit Koeffizienten in $\mathcal{A}$}.
\end{defn}

\begin{defn}
  Sei $\mathcal{B}$ ein kohomologisches Koeffizientensystem auf einer simplizialen Menge $X$. Wir setzen
  \[ C^n(X; \mathcal{B}) \coloneqq \{ \text{ Funktionen } f : (\sigma \in X_n) \to \mathcal{B}_\sigma \, \} \]
  und definieren $\delta_n : C^n(X; \mathcal{B}) \to C_{n+1}(X; \mathcal{B})$ durch
  \[ \delta^n(f)(\sigma) \coloneqq \sum_{i=0}^{n+1} (-1)^i \mathcal{B}(\partial_{n+1}^i, \sigma)(f(X(\partial_{n+1}^i)(\sigma))). \]
  Die Kohomologiegruppen des so def. Kokettenkomplexes $C^\bullet(X; \mathcal{B})$ heißen \emph{Kohomologiegruppen} von $X$ \emph{mit Koeffizienten in $\mathcal{B}$}.
\end{defn}

% I.4.11

\begin{bsp}
  Sei $Y$ ein topol. Raum, $U = (U_\alpha)_{\alpha \in A}$, $X$ und $\mathcal{F}$ wie im letzten Beispiel. Die Homologiegruppen $H^n(X, \mathcal{F})$ werden Čech-Kohomologiegruppen der Garbe der stetigen Funktionen auf $Y$ bzgl. der Überdeckung $U$ genannt.
\end{bsp}
% Ausgelassen: Beispiel c)

% I.6 (Die lange Sequenz)

% Ausgelassen: I.6.1 (Homologie als Funktion in zwei Variablen)

% I.6.2 (Exakte Sequenzen)

\begin{defn}
  Eine (lange) \emph{exakte Sequenz} ab. Gruppen ist ein (Ko-)Kettenkomplex mit verschwindenden Homologiegruppen, \dh{}
  \[ \im \partial_n = \ker \partial_{n-1} \quad \text{für alle $n$.} \]
\end{defn}

\begin{defn}
  Eine \emph{kurze ex. Sequenz} (\keS{}) ist eine ex. Seq. der Form
  \[ 0 \to A \to B \to C \to 0. \]
\end{defn}

\begin{defn}
  Sei $0 \to A \to B \to C \to 0$ eine \keS{} in einer abelschen Kategorie $\mathcal{A}$. Die Sequenz heißt \emph{spaltend}, falls sie isomorph zur \keS{} $0 \to A \to A \oplus C \to C$ ist.
\end{defn}

\begin{prop}
  Für eine Sequenz $0 \to A \xrightarrow{f} B \xrightarrow{g} C \to 0$ sind äquivalent:
  \begin{itemize}
    \item Die Sequenz spaltet.
    \item Es existiert eine Retraktion $r : B \to A$ mit $r \circ f = \id_A$.
    \item Es existiert ein Schnitt $s : C \to B$ mit $g \circ s = \id_C$.
  \end{itemize}
\end{prop}

% I.6.5
\begin{defn}
  Eine Sequenz $0 \to \CCC{A} \to \CCC{B} \to \CCC{C} \to 0$ von Komplexen heißt \emph{exakt}, wenn für alle $n$ die Seq. $0 \to A_n \to B_n \to C_n \to 0$ exakt ist.
\end{defn}

% Ausgelassen: I.6.7 (Konstruktion der Randabbildung)

% 1.6.8
\begin{prop}
  Eine kurze exakte Sequenz
  $0 \to \CCC{A} \xrightarrow{i^\bullet} \CCC{B} \xrightarrow{p^\bullet} \CCC{C} \to 0$
  von Kokettenkomplexen induziert eine lange exakte Sequenz
  \[ \nldots \to H^n(\CCC{A}) \xrightarrow{H^n(i^\bullet)} H^n(\CCC{B}) \xrightarrow{H^n(p^\bullet)} H^n(\CCC{C}) \xrightarrow{\delta^n} H^{n+1}(\CCC{A}) \to \nldots \]
\end{prop}

% I.6.6
\begin{lem}
  Sei $0 \to A \to B \to C \to 0$ eine \keS{} ab. Gruppen und $X$ eine simpl. Menge.
  Dann sind ebenfalls exakt:
  \begin{align*}
    0 \to \CC{C}(X; A) \to \CC{C}(X; B) \to \CC{C}(X; C) \to 0, \\
    0 \to \CCC{C}(X; A) \to \CCC{C}(X; B) \to \CCC{C}(X; C) \to 0.
  \end{align*}
\end{lem}

% I.6.3
\begin{kor}
  Sei $0 \to A \to B \to C \to 0$ eine \keS{} ab. Gruppen und $X$ eine simpl. Menge. Dann existieren lange exakte Sequenzen
  \begin{align*}
    \ldots \to H_n(X; A) \to H_n(X; B) \to H_n(C) \to H_{n-1}(X; A) \to \ldots \\
    \ldots \to H^n(X; A) \to H^n(X; B) \to H^n(C) \to H^{n+1}(X; A) \to \ldots
  \end{align*}
\end{kor}

% Ausgelassen: I.6.4 (Bemerkungen)

% I.6.9 (Verallgemeinerung auf ein Koeffizientensystem)

\begin{defn}
  Eine Sequenz $0 \to \mathcal{B}' \to \mathcal{B} \to \mathcal{B}'' \to 0$ von (ko-)homologischen Koeffizientensystemen auf einer simpl. Menge $X$ heißt \emph{exakt}, falls
  \[
    0 \to \mathcal{B}'_\sigma \to \mathcal{B}_\sigma \to \mathcal{B}''_\sigma \to 0 \qquad
    \text{für alle $\sigma \in X_n$ exakt ist.}
  \]
\end{defn}

\begin{lem}
  Eine kurze exakte Sequenz $0 \to \mathcal{B}' \to \mathcal{B} \to \mathcal{B}'' \to 0$ von (ko-)homologischen Koeff'systemen induziert kurze ex. Sequenzen
  \begin{align*}
    0 \to \CC{C}(X; \mathcal{B}') \to \CC{C}(X; \mathcal{B}) \to \CC{C}(X; \mathcal{B}'') \to 0, \\
    0 \to \CCC{C}(X; \mathcal{B}') \to \CCC{C}(X; \mathcal{B}) \to \CCC{C}(X; \mathcal{B}'') \to 0
  \end{align*}
  und damit auch entsprechende lange exakte Sequenzen.
\end{lem}

% I.7 (Komplexe)

% I.7.1 (Wo kommen Komplexe her?)

% I.7.2
\begin{defn}
  Eine \emph{simpl. ab. Gruppe} ist ein Funktor
  $A : \Delta^\op \to \AbGrp$.
\end{defn}

% I.7.3
\begin{defn}
  Sei $A$ eine simpliziale abelsche Gruppe. \\
  Dann ist $(A_\bullet, \partial)$ ein Kettenkomplex mit
  \[
    \partial_n : A_n \to A_{n-1}, \quad
    a \mapsto \sum_{i=0}^n (-1)^i A(\partial_n^i)(a).
  \]
\end{defn}

\begin{defn}
  Eine \emph{kosimpl. ab. Gruppe} ist ein Funktor
  $A : \Delta \to \AbGrp$.
\end{defn}

\begin{defn}
  Sei $A$ eine kosimpliziale abelsche Gruppe. \\
  Dann ist $(A^\bullet, \delta)$ ein Kokettenkomplex mit
  \[
    \delta^n : A^n \to A^{n+1}, \quad
    a \mapsto \sum_{i=0}^n (-1)^i A(\partial_{n+1}^i)(a).
  \]
\end{defn}

% I.7.4 (Der Čech-Komplex)

\begin{defn}
  Sei $Y$ ein topol. Raum, $(U_\alpha)_{\alpha \in A}$ eine (nicht unbedingt offene) Überdeckung von $Y$ und $\Fais$ eine Garbe ab. Gruppen auf $Y$. Die kosimpliziale abelsche Gruppe $\check{C}(U, \Fais)$ der \emph{Čech-Koketten} ist
  \begin{align*}
    \check{C}^m(U, \Fais) \coloneqq \nspace{4} \prod_{\alpha_0, \ldots, \alpha_m \in A} \nspace{4} \Fais(U_{\alpha_0} \cap \ldots \cap U_{\alpha_m}), \\
    \check{C}(U, \Fais)(f : [m] \to [n])((f_{\alpha_0, \ldots ,\alpha_m})_{\alpha_0, \ldots, \alpha_m}) \coloneqq \\
    (f_{g(0), \ldots, g(m)}|U_{\alpha_0} \cap \ldots \cap U_{\alpha_n})_{\alpha_0, \nldots, \alpha_n}.
  \end{align*}
\end{defn}

\begin{bem}
  Die Randabb. im zugeh. Kokettenkomplex ist gegeben durch
  \[ (\delta^n \phi)_{\alpha_0, \ldots, \alpha_{n+1}} \coloneqq \sum_{i=0}^{n+1} (-1)^i \phi_{\alpha_0, \ldots, \hat{\alpha_i}, \ldots, \alpha_{n+1}}. \]
\end{bem}

\begin{defn}
  Die Kohomologiegruppen dieses Komplexes heißen \emph{Čech-Homologiegruppen} von $\Fais$ bzgl. der Überdeckung $(U_\alpha)_{\alpha \in A}$.
\end{defn}

\begin{bem}
  $\check{H}(U, \Fais) \cong \Gamma(X, \Fais)$ hängt nicht von der Überdeckung ab.
\end{bem}

\begin{defn}
  Sei $Y$ ein topol. Raum und $X$ dessen simpl. Menge der singulären Simplizes.
  Die Homologiegruppen von $\CC{C}(X; A)$ heißen \emph{singuläre Homologiegruppen} $H_n(Y; A)$ von $Y$ mit Koeff. $A$.
\end{defn}

% Ausgelassen: I.7.6 (Homologie und Kohomologie von Gruppen)

% I.7.7 (Der de-Rham-Komplex)
\begin{defn}
  Sei $M$ eine $\mathcal{C}^\infty$-Mft, $\Omega^k(M)$ das $C^\infty(M)$-Modul der $k$-Formen auf $M$. Die \emph{äußere Ableitung} $\d : \Omega^k(M) \to \Omega^{k+1}(M)$ ist in lokalen Koordinaten $(x^1, \ldots, x^n)$ definiert durch
  \[ \d \left( \sum_{\abs{I} = k} f_I \d x^I \right) = \sum_{\abs{I} = k} \sum_{i=1}^n \frac{\partial f_I}{\partial x^i} \d x^i \wedge \d x^I. \]
  Die Kohomologiegruppen des so definierten Komplexes $\Omega^\bullet(M)$ heißen \emph{De-Rham-Kohomologiegruppen}.
\end{defn}

% I.7.8 (Homologie und Kohomologie einer Lie-Algebra)

\begin{defn}
  Sei $\Lg$ eine Lie-Algebra und $A$ ein $\Lg$-Modul. Setze $C^k(\Lg, A) \coloneqq L(\wedge^k \Lg, A)$ und definiere $\d : C^k(\Lg, A) \to C^{k+1}(\Lg, A)$ durch eine allgemeine Cartan-Formel
  \begin{align*}
    (\d c)(g_1, \nldots, g_{k+1}) \coloneqq & \nspace{4} \sum_{1 \leq j < l \leq k+1} \nspace{4} (-1)^{j+l-1} c([g_j, g_l], g_1, \nldots, \hat{g_j}, \nldots, \hat{g_l}, \nldots, g_{k+1}) \\
    & + \sum_{j=1}^{k+1} (-1)^j g_j c(g_1, \nldots, \hat{g_j}, \ldots, g_{k+1}).
  \end{align*}
  Die Kohomologiegruppen des so definierten Kokettenkomplexes werden mit $H^\bullet(\Lg, A)$ bezeichnet.
\end{defn}

% I.7.9 (Homotope Abbildungen von Komplexen)
\begin{defn}
  Eine \emph{Kettenhomotopie} zw. Morphismen $f, g : \CC{C} \to \CC{D}$ von Kettenkomplexen ist eine Folge von Homomorphismen $k_n : C_n \to D_{n+1}$ mit
  $\fa{n \in \N} \partial^D_{n+1} \circ k_n + k_{n-1} \circ \partial^C_n = f_n - g_n$.
\end{defn}

% I.7.10
\begin{lem}
  Seien $f, g : \CC{C} \to \CC{D}$ kettenhomotop. Dann gilt
  \[ H_n(f) = H_n(g) \quad \text{für alle $n \in \N$.} \]
\end{lem}

% I.7.11
\begin{prop}
  \begin{itemize}
    \item Seien $\phi, \psi : X \to Y$ homotope Abbildungen zwischen topologischen Räumen. Dann sind die induzierten Abbildungen $\phi_*, \psi_* : \CC{C}(X; A) \to \CC{C}(Y; A)$ kettenhomotop.
    \item Seien $\phi, \psi : M \to N$ zwei glatt homotope Abbildungen von $\mathcal{C}^\infty$-Mften. Dann sind $\phi^*, \psi^* : \Omega^\bullet(N) \to \Omega^\bullet(M)$.
  \end{itemize}
\end{prop}

% I.7.12
\begin{kor}
  Homotopieäquivalente Räume besitzen isomorphe singuläre Homologiegruppen.
\end{kor}

\end{document}
