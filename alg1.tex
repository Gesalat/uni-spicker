\documentclass{cheat-sheet}

\pdfinfo{
  /Title (Zusammenfassung Algebra 1)
  /Author (Tim Baumann)
}

\usepackage{mathabx} % \divides
\newcommand{\K}{\mathbb{K}} % Körper

% Kleinere Klammern
\delimiterfactor=701

\usepackage{relsize}
\let\myBinom\binom
\renewcommand{\binom}[2]{\mathsmaller{\myBinom{#1}{#2}}}

\begin{document}

\maketitle{Zusammenfassung Algebra 1}

% Buchempfehlungen:
% * Rotman: Galois Theory
% * H. Edwards: Galois Theory
% * Van der Warden: Algebra 1

% Kapitel 1. Gleichungen

% Kapitel 1.1. "`Algebra"'

% t 31.5.1832 Évariste Galois (1811-1832)

\begin{defn}
  Ein \emph{Polynom} mit Unbestimmter $X$ hat die Form
  \[ f(X) = a_0 X^n + a_1 X^{n-1} + ... + a_{n-1} X + a_n. \]
\end{defn}

\begin{defn}
  Falls oben $a_0 \not= 0$ gilt, so ist $\partial f = n$ der \emph{Grad} des Polynoms.
\end{defn}

% Bemerkung: Polynome gibt es auch zu mehreren Unbestimmten ($X,Y,Z$ oder $X_1,X_2,X_3$)

\begin{defn}
  \begin{itemize}
    \item Eine \emph{Linearkombination} ist ein Polynom der Form
    \[ f(X_1, ..., X_n) = a_1 X_1 + ... + a_n X_n. \]
    \item Ein \emph{Monom} hat die Gestalt $f(x) = b x^k$.
  \end{itemize}
\end{defn}

% Algebra = Studium der Polynome und ihrer Nullstellenmengen

% Ausgelassen: Mitternachtsformel

% Kapitel 1.2. Das gemeinsame Maß

% Pythagoras, ca. 580-540 v. Chr.: "Alles ist Zahl"
% Zahlen sind nicht nur zum Zählen, sondern auch zum Vergleichen da!

% Hippasos von Metapont: Stimmt das? Vielleicht bricht das Verfahren niemals ab?

\begin{alg}[Euklid]
  Seien $a, b \in \R$ mit $a {>} b {>} 0$ gegeben. Schreibe
  \[ a = k \cdot b + r \]
  mit $k \in \N$ und $r < b$. Wiederhole diesen Schritt mit $(a, b) \coloneqq (b, r)$, falls $r \not= 0$.
\end{alg}

\begin{defn}
  Ein \emph{gemeinsames Maß} zweier Zahlen $a, b \in \R$ ist eine Zahl $c \in \R$, sodass es $k, l \in \Z$ mit $a = k \cdot c$ und $b = l \cdot c$ gibt.
\end{defn}

\begin{bem}
  Zwei Zahlen haben genau dann ein gemeinsames Maß, wenn der euklidische Algorithmus, angewandt auf diese Zahlen, abbricht.
\end{bem}

\begin{defn}
  Zwei Zahlen $a, b \in \R$, die kein gemeinsames Maß besitzen, heißen \emph{inkommensurabel}. Ihr Verhältnis ist dann \emph{irrational}.
\end{defn}

\begin{satz}
  Die Längen der Seite und der Diagonalen eines regelmäßigen Fünfecks sind zueinander inkommensurabel.
\end{satz}

\begin{defn}
  Der \emph{goldene Schnitt} ist die Zahl
  \[ \Phi \coloneqq \frac{1 + \sqrt{5}}{2} \approx 1.618. \]
\end{defn}

\begin{bem}
  Der goldene Schnitt ist Lösung der Polynomgleichung
  \[ X^2 - X - 1 = 0. \]
\end{bem}

% Ausgelassen: Archimedisches Axiom

% Kapitel 1.3. Die binomische Formel

\begin{defn}
  Ein \emph{Binom} ist ein Ausdruck der Form $(a+b)^n$ mit $n \in \N$.
\end{defn}

\begin{defn}
  Für $n \in \N$ und $k \leq n$ schreibe $\binom{n}{k} \coloneqq \tfrac{n!}{k! \cdot (n-k)!}$.
\end{defn}

\begin{satz}
  Es gilt $(a+b)^n = \sum_{k=0}^n \binom{n}{k} a^k b^{n-k}$ für alle $n \in \N$.
\end{satz}

% Kapitel 1.4. Die Tschirnhaus-Transformation

\begin{verf}[Tschirnhaus-Transformation]
  Sei eine Polynomgleichung der Form
  \[ x^n + a_1 x^{n-1} + ... + a_{n-1} x + a_n = 0 \]
  gegeben. Substituiere $x \coloneqq \tilde{x} - \tfrac{a_1}{n}$. Dann hat die neue Gleichung keinen $x^{n-1}$-Term. Lösungen der beiden Gleichungen können durch Addieren bzw. Subtrahieren von $\tfrac{a_1}{n}$ ineinander überführt werden.
\end{verf}

\begin{kor}
  Beim Lösen von Polynomgleichungen kann man also annehmen, dass kein $x^{n-1}$-Term vorhanden ist.
\end{kor}

\begin{kor}[Mitternachtsformel]
  Die Polynomgleichung zweiten Grades $x^2 + ax + b = 0$ wird gelöst durch
  \[ x = - \tfrac{a}{2} \pm \tfrac{1}{2} \sqrt{a^2 - 4b}. \]
\end{kor}

% Kapitel 1.5. Die kubische Gleichung

% Name: Cardanosche Formel
\begin{satz}
  Eine Nullstelle der kubischen Gleichung $x^3 + ax - b = 0$ ist gegeben durch
  \[
    x = \sqrt[3]{\tfrac{b}{2} + \sqrt{D}} + \sqrt[3]{\tfrac{b}{2} - \sqrt{D}}
    \quad \text{mit } D \coloneqq \left(\tfrac{a}{3}\right)^3 + \left(\tfrac{b}{2}\right)^2.
  \]
\end{satz}

\begin{prob}
  Was, wenn in der Quadratwurzel eine neg. Zahl steht?
\end{prob}

% Kapitel 1.6. Die imaginären Zahlen

% Ausgelassen: Motivation der imaginären Zahlen mittels Cardanos Formel

\begin{defn}
  Für die \emph{imaginäre Zahl} $i$ gilt: $i^2 = -1$. Die \emph{komplexen Zahlen} $\C$ sind Zahlen der Form $x + y i$ mit $x, y \in \R$.\\
  Es gelten die Rechenregeln
  \begin{align*}
    (x + yi) \pm (u + vi) &= (x+u) \pm (y+v)i\\
    (x + yi) \cdot (u + vi) &= (xu - yv) + (xv + yu)i\\
    \frac{1}{x + yi} &= \frac{x}{x^2 + y^2} + \frac{-y}{x^2 + y^2} i
  \end{align*}
\end{defn}

% Ausgelassen: Begriff des Körpers

\begin{defn}
  Für eine komplexe Zahl $z = x + yi$ mit $x, y \in \R$ heißen
  \[
    \Re(z) \coloneqq x \text{ \emph{Realteil}}
    \quad \text{und} \quad
    \Im(z) \coloneqq y \text{ \emph{Imaginärteil}}.
  \]
\end{defn}

\begin{defn}
  Die Operation $x + yi \mapsto x - yi$ heißt \emph{komplexe Konjugation}. Man notiert sie mit einem Querstrich, also $z \mapsto \overline{z}$ für $z \in \C$.
\end{defn}

\begin{bem}
  Die komplexe Konjugation ist verträglich mit Addition und Multiplikation und sogar ein Körperautomorphismus.
\end{bem}

\begin{defn}
  Der \emph{Betrag} einer komplexen Zahl $z = x + yi$ ist
  \[ \abs{z} \coloneqq \sqrt{x^2 + y^2} = \sqrt{z \overline{z}}. \]
\end{defn}

\begin{satz}
  Für komplexe Zahlen $z, w \in \C$ gilt
  \begin{itemize}
    \miniitem{0.54\linewidth}{$\abs{z + w} \leq \abs{z} + \abs{w}$ \enspace ($\triangle$-Ungl)}
    \miniitem{0.44\linewidth}{$\abs{z} \cdot \abs{w} = \abs{z \cdot w}$}
  \end{itemize}
\end{satz}

% Kapitel 1.8. Die Exponentialfunktion

\begin{defn}
  Die \emph{Exponentialfunktion} ist die Abbildung
  \[ \exp : \C \to \C, \quad x \mapsto \sum_{k=0}^\infty \tfrac{x^k}{k!}. \]
\end{defn}

\begin{satz}
  Für alle $x, y \in \C$ gilt $\exp(x + y) = \exp(x) \cdot \exp(y)$.
\end{satz}

\begin{defn}
  Die \emph{Eulersche Zahl} ist die Zahl
  \[ e \coloneqq \exp(1) = \sum_{k=0}^\infty \tfrac{1}{k!} \approx 2,718. \]
\end{defn}

\begin{nota}
  Schreibe $e^y \coloneqq \exp(y)$ für alle $y \in \C$.
\end{nota}

\begin{prop}
  Für alle $t \in \R$ gilt $\abs{e^{ti}} = 1$.
\end{prop}

\begin{prop}
  Es gilt für alle $t \in \R$:
  \begin{itemize}
    \miniitem{0.18 \linewidth}{$e^{2\pi i} = 1$}
    \miniitem{0.18 \linewidth}{$e^{\pi i} = -1$}
    \miniitem{0.28 \linewidth}{$e^{(2\pi + t)i} = e^{ti}$}
    \miniitem{0.32 \linewidth}{$e^{ti} = \cos(t) + i \sin(t)$}
  \end{itemize}
\end{prop}

\begin{bem}
  Jede komplexe Zahl $z \in \C$ lässt sich als $z = \abs{z} \cdot e^{si}$ mit $s \in \cointerval{0}{2\pi}$ darstellen.
  Mit $w = \abs{w} \cdot e^{ti}$ gilt $z \cdot w = (\abs{z} \cdot \abs{w}) \cdot e^{i(s+t)}$.
\end{bem}

\begin{defn}
  Für $z = \abs{z} \cdot e^{ti} \in \C$ und $n \in \N$ heißen die Zahlen
  \[ \sqrt[n]{\abs{z}} e^{(t + k 2 \pi)i/n} \]
  für $k \in \{ 0, ..., n-1 \}$ \emph{$n$-te Wurzel} von $z$.
\end{defn}

\begin{defn}
  Die \emph{$n$-ten Einheitswurzeln} sind die Zahlen
  \[ \zeta_k \coloneqq e^{2 \pi ik/n} \quad \text{für $k = 0, ..., n-1$}. \]
\end{defn}

% Kapitel 1.9. Der "`Fundamentalsatz der Algebra"'

\begin{satz}
  Jedes normierte Polynom vom Grad $n \in \N$
  \[ f(x) = x^n + a_1 x^{n-1} + ... + a_n \]
  mit Koeffizienten $a_1, ..., a_n \in \C$ hat eine Nullstelle in $\C$.
\end{satz}

% Kapitel 1.10. Alle Nullstellen

\begin{defn}
  Ein \emph{Monoid} ist ein Tupel $(M, \cdot, e)$ bestehend aus einer Menge $M$ mit einer Verknüpfung $\cdot : M \times M \to M$ und einem \emph{neutralen Element} $e \in M$, sodass gilt:
  \begin{itemize}
    \item $\fa{x, y, z \in G} (x \cdot y) \cdot z = x \cdot (y \cdot z)$ \pright{Assoziativität}
    \item $\fa{g \in G} e \cdot g = g = g \cdot e$ \pright{Neutralität}
  \end{itemize}
\end{defn}

\begin{defn}
  Eine \emph{Gruppe} ist ein Tupel $(G, \cdot, e)$ bestehend aus einer Menge $G$ mit einer Verknüpfung $\cdot : G \times G \to G$ und einem \emph{neutralen Element} $e \in G$ zusammen mit einer Inversion $\blank^{-1} : G \to G$, sodass:
  \begin{itemize}
    \item $\fa{x, y, z \in G} (x \cdot y) \cdot z = x \cdot (y \cdot z)$ \pright{Assoziativität}
    \item $\fa{g \in G} e \cdot g = g = g \cdot e$ \pright{Neutralität}
    \item $\fa{g \in G} g \cdot g^{-1} = g^{-1} \cdot g = e$
  \end{itemize}
\end{defn}

\begin{defn}
  Ein \emph{Ring} ist ein Tupel $(R, +, \cdot, 0, 1)$ bestehend aus einer Menge $R$, zwei Verknüpfungen $+, \cdot : R \times R \to R$ und zwei Elementen $0, 1 \in R$, sodass
  \begin{itemize}
    \item $(R, +, 0)$ eine Gruppe bildet,
    \item $(R, \cdot, 1)$ einen Monoid bildet und
    \item folgende Distributivgesetze für alle $a, b, c \in R$ erfüllt sind:
    \[
      (a + b) \cdot c = a \cdot c + b \cdot c, \qquad
      a \cdot (b + c) = a \cdot b + a \cdot c.
    \]
  \end{itemize}
\end{defn}

\begin{defn}
  Ein \emph{Körper} ist ein Tupel $(\K, +, \cdot, 0, 1)$ bestehend aus einer Menge $\K$, zwei Verknüpfungen $+, \cdot : \K \times \K \to \K$ und zwei Elementen $0, 1 \in \K$, sodass
  \begin{itemize}
    \item $(\K, +, 0)$ eine Gruppe bildet,
    \item $(\K \setminus \{ 0 \}, \cdot, 1)$ eine Gruppe bildet und
    \item folgende Distributivgesetze für alle $a, b, c \in R$ erfüllt sind:
    \[
      (a + b) \cdot c = a \cdot c + b \cdot c, \qquad
      a \cdot (b + c) = a \cdot b + a \cdot c.
    \]
  \end{itemize}
\end{defn}

\begin{bem}
  Jeder Körper ist auch ein Ring.
\end{bem}

\begin{nota}
  $\K[x] \coloneqq \{ \text{Polynome mit Koeffizienten in $\K$} \}$
\end{nota}

\begin{bem}
  Die Menge aller Polynome über $\K$ bildet einen Ring.
\end{bem}

\begin{defn}
  In einem Ring $R$ teilt ein Element $g \in R$ ein anderes Element $f \in R$, geschrieben $g \divides f$, falls es ein $h \in R$ mit $g \cdot h = f$ gibt.
\end{defn}

\begin{bem}
  Ein Ring, in dem Division mit Rest möglich ist (z.\,B. der Polynomring oder $\Z$), wird \emph{euklidischer Ring} genannt. In solchen Ringen kann man den euklidischen Algorithmus ausführen.
\end{bem}

\begin{satz}
  Ist $x_0 \in \K$ eine Nullstelle des Polynoms $f \in \K[x]$, dann gilt $(X - x_0) \divides f$, genauer $f = (x{-}x_0) \cdot g$ für ein $g {\in} \K[x]$ mit $\partial g = \partial f {-} 1$.
\end{satz}

\begin{kor}
  Ein Polynom $f \in \K[x]$ vom Grad $n \geq 1$ hat höchstens $n$ Nullstellen.
\end{kor}

\begin{kor}
  Wenn $\K$ unendlich viele Elemente hat, sind die Koeffi- zienten von jedem $f \in \K[x]$ durch die Fkt. $f$ eindeutig bestimmt.
\end{kor}

\begin{satz}[Hauptsatz der Algebra]
  Jedes Polynom $f \in \C[x]$ ist Produkt von Polynomen vom Grad $1$, sogenannten Linearfaktoren, also
  \[
    f = a \cdot (x - x_1) \cdot ... \cdot (x - x_n)
    \quad \text{mit $a, x_1, ..., x_n \in \C$}.
  \]
\end{satz}

\begin{bem}
  Die Zahlen $x_1, ..., x_n$ müssen nicht alle verschieden sein.
\end{bem}

\begin{defn}
  Die Anzahl der Vorkommen einer Nullstelle $x_i$ in obiger Produktdarstellung heißt \emph{Vielfachheit} der Nullstelle.
\end{defn}

\begin{defn}
  Die \emph{Ableitung} des Polynoms
  \[ f(x) = a_0 x^n + a_1 x^{n-1} + ... + a_{n-1} x + a_0 \in \K[x] \]
  ist das Polynom
  \[ f'(x) = n a_0 x^{n-1} + (n{-}1) a1 x^{n-2} + ... + a_{n-1}. \]
\end{defn}

\begin{bem}
  Sei $x_i$ eine $k$-fache Nullstelle von $f \in \C[x]$. Dann ist $x_i$ auch eine $(k{-}1)$-fache Nullstelle von $f'$.
\end{bem}

\begin{defn}
  Ein Körper $\K$ mit der Eigenschaft, dass jedes Polynom $f \in \K[x]$ in Linearfaktoren zerfällt, heißt \emph{algebraisch abgeschlossen}.
\end{defn}

\begin{defn}
  Eine Zahl $c \in \C$ heißt \emph{algebraisch}, wenn es ein Polynom $f \in \Q[x]$, $f \not= 0$ mit $f(c) = 0$ gibt.
\end{defn}

\begin{bem}
  Man kann zeigen, dass die Menge der algebraischen Zahlen ein abzählbarer, algebraisch abgeschlossener Körper ist.
\end{bem}

% Kapitel 1.11. Der Wurzelsatz von Vieta

\begin{defn}
  Die \emph{elementarsymmetrischen Funktionen} in $x_1, ..., x_n$ sind die Polynome
  \[
    e_k(x_1, ..., x_n) \coloneqq \sum_{\mathclap{j_1 < ... < j_n}} x_{j_1} \cdot ... \cdot x_{j_n}
    \quad \text{für $1 \leq k \leq n$.}
  \]
\end{defn}

\begin{bem}
  Bezeichne mit $e_j$ für $1 \leq j \leq n$ die elementarsymmetrischen Funktionen in den Variablen $x_1, ..., x_n$, mit $\tilde{e}_i$ für $1 \leq i < n$ die elementarsymmetrischen Funktionen in den Variablen $x_1, ..., x_{n-1}$. Dann gelten die Rekursionsgleichungen
  \[
    e_1 = x^n + \tilde{e}_1, \qquad
    e_i = \tilde{e}_i + x_n \cdot \tilde{e}_{i-1}, \qquad
    e_n = x_n \cdot \tilde{e}_{n-1}.
  \]
\end{bem}

\begin{satz}[Vieta]
  Sei $f \in \K[X]$ ein normiertes Polynom, das über $\K$ in Linearfaktoren zerfällt, also
  \[ f(x) = x^n + a_1 x^{n-1} + ... + a_n = (x-x_1) \cdot ... \cdot (x-x_n), \]
  dann gilt $a_j = (-1)^j e_j(x_1, ..., x_n)$ für alle $1 \leq j \leq n$.
\end{satz}

\begin{defn}
  Eine \emph{Permutation} der Zahlen $\{ 1, ..., n \}$ ist eine Bijektion
  \[ \sigma : \{ 1, ..., n \} \to \{ 1, ..., n \}. \]
\end{defn}

\begin{defn}
  Ein Polynom $f \in \K[x_1, ..., x_n]$ heißt \emph{symmetrisch}, falls für alle $x_1, ..., x_n \in \K$ und Permutationen $\sigma$ gilt:
  \[ f(x_1, ..., x_n) = (\sigma f)(x_1, ..., x_n) \coloneqq f(x_{\sigma(1)}, ...., x_{\sigma(n)}) \]
\end{defn}

% Ausgelassen: Definition Umkehrabbildung

\begin{satz}[Hauptsatz über symmetrische Polynome]
  Jedes symme- trische Polynom $f(\vec{x})$ mit $\vec{x} = (x_1, ..., x_n)$ lässt sich als Polynom in den elementarsymmetrischen Polynomen $e_1(\vec{x}), ..., e_n(\vec{x})$ darstellen.
\end{satz}

\begin{kor}
  Sind $x_1, ..., x_n$ die Wurzeln eines normierten Polynoms $f(x) = x^n + a_1 x^{n-1} + ... + a_n$, dann gilt für jedes symmetrische Polynom $s \in \K[y_1, ..., y_n]$: $s(x_1, ..., x_n)$ ist ein Polynomausdruck in den Koeffizienten $a_1 ..., a_n$ und damit aus diesen Zahlen berechenbar.
\end{kor}

\begin{defn}
  Die \emph{Diskriminante} eines Polynoms $f = (x - x_1) \cdot ... \cdot (x - x_n)$ ist der Ausdruck
  \[ \Delta (\vec{x}) \coloneqq \pm \prod_{i \not= j} (x_i - x_j). \]
  Da dieser Polynomausdruck symmetrisch ist, lässt er sich in den Koeffizienten des Polynoms $f$ darstellen.
\end{defn}

\begin{bsp}
  Die Diskriminante des quadratischen Polynoms $f(x) = x^2 - ax + b$ ist $- \Delta = a^2 - 4 b$, die des kubischen Polynoms $g(x) = x^3 - a x^2 + b x - c$ ist $\Delta = a^2 b^2 - 4 a^3 c - 4 b^3 + 18 abc - 27 c^3$.
\end{bsp}

% Kapitel 1.13. Lagrangesche Resolventen

\end{document}