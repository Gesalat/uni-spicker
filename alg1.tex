\documentclass{cheat-sheet}

\pdfinfo{
  /Title (Zusammenfassung Algebra 1)
  /Author (Tim Baumann)
}

\usepackage{mathabx} % \divides
\newcommand{\K}{\mathbb{K}} % Körper

% Kleinere Klammern
\delimiterfactor=701

\usepackage{relsize}
\let\myBinom\binom
\renewcommand{\binom}[2]{\mathsmaller{\myBinom{#1}{#2}}}
\DeclareMathOperator{\ggT}{ggT} % größter gemeinsamer Teiler
\newcommand{\Mod}[1]{\ (\text{mod}\ #1)} % http://tex.stackexchange.com/questions/137073/writing-mod-in-congruence-problems-without-leading-space
\newcommand{\F}{\mathbb{F}} % Symbol für Restklassenkörper
\renewcommand{\L}{\mathbb{L}} % Variable (für Körper verwendet)
\DeclareMathOperator{\Aut}{Aut} % Automorphismengruppe
\DeclareMathOperator{\Iso}{Iso} % Isomorphismenmenge
\DeclareMathOperator{\Gal}{Gal} % Galoisgruppe

\begin{document}

\maketitle{Zusammenfassung Algebra 1}

% Buchempfehlungen:
% * Rotman: Galois Theory
% * H. Edwards: Galois Theory
% * Van der Warden: Algebra 1

% Kapitel 1. Gleichungen

% Kapitel 1.1. "`Algebra"'

% t 31.5.1832 Évariste Galois (1811-1832)

\begin{defn}
  Ein \emph{Polynom} mit Unbestimmter $X$ hat die Form
  \[ f(X) = a_0 X^n + a_1 X^{n-1} + ... + a_{n-1} X + a_n. \]
\end{defn}

\begin{defn}
  Falls oben $a_0 \not= 0$ gilt, so ist $\partial f = n$ der \emph{Grad} des Polynoms.
\end{defn}

% Bemerkung: Polynome gibt es auch zu mehreren Unbestimmten ($X,Y,Z$ oder $X_1,X_2,X_3$)

\begin{defn}
  \begin{itemize}
    \item Eine \emph{Linearkombination} ist ein Polynom der Form
    \[ f(X_1, ..., X_n) = a_1 X_1 + ... + a_n X_n. \]
    \item Ein \emph{Monom} hat die Gestalt $f(x) = b x^k$.
  \end{itemize}
\end{defn}

% Algebra = Studium der Polynome und ihrer Nullstellenmengen

% Ausgelassen: Mitternachtsformel

% Kapitel 1.2. Das gemeinsame Maß

% Pythagoras, ca. 580-540 v. Chr.: "Alles ist Zahl"
% Zahlen sind nicht nur zum Zählen, sondern auch zum Vergleichen da!

% Hippasos von Metapont: Stimmt das? Vielleicht bricht das Verfahren niemals ab?

\begin{alg}[Euklid]
  Seien $a, b \in \R$ mit $a {>} b {>} 0$ gegeben. Schreibe
  \[ a = k \cdot b + r \]
  mit $k \in \N$ und $r < b$. Wiederhole diesen Schritt mit $(a, b) \coloneqq (b, r)$, falls $r \not= 0$.
\end{alg}

\begin{defn}
  Ein \emph{gemeinsames Maß} zweier Zahlen $a, b \in \R$ ist eine Zahl $c \in \R$, sodass es $k, l \in \Z$ mit $a = k \cdot c$ und $b = l \cdot c$ gibt.
\end{defn}

\begin{bem}
  Zwei Zahlen haben genau dann ein gemeinsames Maß, wenn der euklidische Algorithmus, angewandt auf diese Zahlen, abbricht.
\end{bem}

\begin{defn}
  Zwei Zahlen $a, b \in \R$, die kein gemeinsames Maß besitzen, heißen \emph{inkommensurabel}. Ihr Verhältnis ist dann \emph{irrational}.
\end{defn}

\begin{satz}
  Die Längen der Seite und der Diagonalen eines regelmäßigen Fünfecks sind zueinander inkommensurabel.
\end{satz}

\begin{defn}
  Der \emph{goldene Schnitt} ist die Zahl
  \[ \Phi \coloneqq \frac{1 + \sqrt{5}}{2} \approx 1.618. \]
\end{defn}

\begin{bem}
  Der goldene Schnitt ist Lösung der Polynomgleichung
  \[ X^2 - X - 1 = 0. \]
\end{bem}

% Ausgelassen: Archimedisches Axiom

% Kapitel 1.3. Die binomische Formel

\begin{defn}
  Ein \emph{Binom} ist ein Ausdruck der Form $(a+b)^n$ mit $n \in \N$.
\end{defn}

\begin{defn}
  Für $n \in \N$ und $k \leq n$ schreibe $\binom{n}{k} \coloneqq \tfrac{n!}{k! \cdot (n-k)!}$.
\end{defn}

\begin{satz}
  Es gilt $(a+b)^n = \sum_{k=0}^n \binom{n}{k} a^k b^{n-k}$ für alle $n \in \N$.
\end{satz}

% Kapitel 1.4. Die Tschirnhaus-Transformation

\begin{verf}[Tschirnhaus-Transformation]
  Sei eine Polynomgleichung der Form
  \[ x^n + a_1 x^{n-1} + ... + a_{n-1} x + a_n = 0 \]
  gegeben. Substituiere $x \coloneqq \tilde{x} - \tfrac{a_1}{n}$. Dann hat die neue Gleichung keinen $x^{n-1}$-Term. Lösungen der beiden Gleichungen können durch Addieren bzw. Subtrahieren von $\tfrac{a_1}{n}$ ineinander überführt werden.
\end{verf}

\begin{kor}
  Beim Lösen von Polynomgleichungen kann man also annehmen, dass kein $x^{n-1}$-Term vorhanden ist.
\end{kor}

\begin{kor}[Mitternachtsformel]
  Die Polynomgleichung zweiten Grades $x^2 + ax + b = 0$ wird gelöst durch
  \[ x = - \tfrac{a}{2} \pm \tfrac{1}{2} \sqrt{a^2 - 4b}. \]
\end{kor}

% Kapitel 1.5. Die kubische Gleichung

% Name: Cardanosche Formel
\begin{satz}
  Eine Nullstelle der kubischen Gleichung $x^3 + ax - b = 0$ ist gegeben durch
  \[
    x = \sqrt[3]{\tfrac{b}{2} + \sqrt{D}} + \sqrt[3]{\tfrac{b}{2} - \sqrt{D}}
    \quad \text{mit } D \coloneqq \left(\tfrac{a}{3}\right)^3 + \left(\tfrac{b}{2}\right)^2.
  \]
\end{satz}

\begin{prob}
  Was, wenn in der Quadratwurzel eine neg. Zahl steht?
\end{prob}

% Kapitel 1.6. Die imaginären Zahlen

% Ausgelassen: Motivation der imaginären Zahlen mittels Cardanos Formel

\begin{defn}
  Für die \emph{imaginäre Zahl} $i$ gilt: $i^2 = -1$. Die \emph{komplexen Zahlen} $\C$ sind Zahlen der Form $x + y i$ mit $x, y \in \R$.\\
  Es gelten die Rechenregeln
  \begin{align*}
    (x + yi) \pm (u + vi) &= (x+u) \pm (y+v)i\\
    (x + yi) \cdot (u + vi) &= (xu - yv) + (xv + yu)i\\
    \frac{1}{x + yi} &= \frac{x}{x^2 + y^2} + \frac{-y}{x^2 + y^2} i
  \end{align*}
\end{defn}

% Ausgelassen: Begriff des Körpers

\begin{defn}
  Für eine komplexe Zahl $z = x + yi$ mit $x, y \in \R$ heißen
  \[
    \Re(z) \coloneqq x \text{ \emph{Realteil}}
    \quad \text{und} \quad
    \Im(z) \coloneqq y \text{ \emph{Imaginärteil}}.
  \]
\end{defn}

\begin{defn}
  Die Operation $x + yi \mapsto x - yi$ heißt \emph{komplexe Konjugation}. Man notiert sie mit einem Querstrich, also $z \mapsto \overline{z}$ für $z \in \C$.
\end{defn}

\begin{bem}
  Die komplexe Konjugation ist verträglich mit Addition und Multiplikation und sogar ein Körperautomorphismus.
\end{bem}

\begin{defn}
  Der \emph{Betrag} einer komplexen Zahl $z = x + yi$ ist
  \[ \abs{z} \coloneqq \sqrt{x^2 + y^2} = \sqrt{z \overline{z}}. \]
\end{defn}

\begin{satz}
  Für komplexe Zahlen $z, w \in \C$ gilt
  \begin{itemize}
    \miniitem{0.54\linewidth}{$\abs{z + w} \leq \abs{z} + \abs{w}$ \enspace ($\triangle$-Ungl)}
    \miniitem{0.44\linewidth}{$\abs{z} \cdot \abs{w} = \abs{z \cdot w}$}
  \end{itemize}
\end{satz}

% Kapitel 1.8. Die Exponentialfunktion

\begin{defn}
  Die \emph{Exponentialfunktion} ist die Abbildung
  \[ \exp : \C \to \C, \quad x \mapsto \sum_{k=0}^\infty \tfrac{x^k}{k!}. \]
\end{defn}

\begin{satz}
  Für alle $x, y \in \C$ gilt $\exp(x + y) = \exp(x) \cdot \exp(y)$.
\end{satz}

\begin{defn}
  Die \emph{Eulersche Zahl} ist die Zahl
  \[ e \coloneqq \exp(1) = \sum_{k=0}^\infty \tfrac{1}{k!} \approx 2,718. \]
\end{defn}

\begin{nota}
  Schreibe $e^y \coloneqq \exp(y)$ für alle $y \in \C$.
\end{nota}

\begin{prop}
  Für alle $t \in \R$ gilt $\abs{e^{ti}} = 1$.
\end{prop}

\begin{prop}
  Es gilt für alle $t \in \R$:
  \begin{itemize}
    \miniitem{0.18 \linewidth}{$e^{2\pi i} = 1$}
    \miniitem{0.18 \linewidth}{$e^{\pi i} = -1$}
    \miniitem{0.28 \linewidth}{$e^{(2\pi + t)i} = e^{ti}$}
    \miniitem{0.32 \linewidth}{$e^{ti} = \cos(t) + i \sin(t)$}
  \end{itemize}
\end{prop}

\begin{bem}
  Jede komplexe Zahl $z \in \C$ lässt sich als $z = \abs{z} \cdot e^{si}$ mit $s \in \cointerval{0}{2\pi}$ darstellen.
  Mit $w = \abs{w} \cdot e^{ti}$ gilt $z \cdot w = (\abs{z} \cdot \abs{w}) \cdot e^{i(s+t)}$.
\end{bem}

\begin{defn}
  Für $z = \abs{z} \cdot e^{ti} \in \C$ und $n \in \N$ heißen die Zahlen
  \[ \sqrt[n]{\abs{z}} e^{(t + k 2 \pi)i/n} \]
  für $k \in \{ 0, ..., n-1 \}$ \emph{$n$-te Wurzel} von $z$.
\end{defn}

\begin{defn}
  Die \emph{$n$-ten Einheitswurzeln} sind die Zahlen
  \[ \zeta_k \coloneqq e^{2 \pi ik/n} \quad \text{für $k = 0, ..., n-1$}. \]
\end{defn}

% Kapitel 1.9. Der "`Fundamentalsatz der Algebra"'

\begin{satz}
  Jedes normierte Polynom vom Grad $n \in \N$
  \[ f(x) = x^n + a_1 x^{n-1} + ... + a_n \]
  mit Koeffizienten $a_1, ..., a_n \in \C$ hat eine Nullstelle in $\C$.
\end{satz}

% Kapitel 1.10. Alle Nullstellen

\begin{defn}
  Ein \emph{Monoid} ist ein Tupel $(M, \cdot, e)$ bestehend aus einer Menge $M$ mit einer Verknüpfung $\cdot : M \times M \to M$ und einem \emph{neutralen Element} $e \in M$, sodass gilt:
  \begin{itemize}
    \item $\fa{x, y, z \in G} (x \cdot y) \cdot z = x \cdot (y \cdot z)$ \pright{Assoziativität}
    \item $\fa{g \in G} e \cdot g = g = g \cdot e$ \pright{Neutralität}
  \end{itemize}
\end{defn}

\begin{defn}
  Eine \emph{Gruppe} ist ein Tupel $(G, \cdot, e)$ bestehend aus einer Menge $G$ mit einer Verknüpfung $\cdot : G \times G \to G$ und einem \emph{neutralen Element} $e \in G$ zusammen mit einer Inversion $\blank^{-1} : G \to G$, sodass:
  \begin{itemize}
    \item $\fa{x, y, z \in G} (x \cdot y) \cdot z = x \cdot (y \cdot z)$ \pright{Assoziativität}
    \item $\fa{g \in G} e \cdot g = g = g \cdot e$ \pright{Neutralität}
    \item $\fa{g \in G} g \cdot g^{-1} = g^{-1} \cdot g = e$
  \end{itemize}
\end{defn}

\begin{defn}
  Eine Gruppe $G$ heißt \emph{abelsch}, wenn sie kommutativ ist, d.\,h. es gilt $ab = ba$ für alle $a, b \in G$.
\end{defn}

\begin{defn}
  Ein \emph{Ring} ist ein Tupel $(R, +, \cdot, 0, 1)$ bestehend aus einer Menge $R$, zwei Verknüpfungen $+, \cdot : R \times R \to R$ und zwei Elementen $0, 1 \in R$, sodass
  \begin{itemize}
    \item $(R, +, 0)$ eine Gruppe bildet,
    \item $(R, \cdot, 1)$ einen Monoid bildet und
    \item folgende Distributivgesetze für alle $a, b, c \in R$ erfüllt sind:
    \[
      (a + b) \cdot c = a \cdot c + b \cdot c, \qquad
      a \cdot (b + c) = a \cdot b + a \cdot c.
    \]
  \end{itemize}
\end{defn}

\begin{defn}
  Ein \emph{Körper} ist ein Tupel $(\K, +, \cdot, 0, 1)$ bestehend aus einer Menge $\K$, zwei Verknüpfungen $+, \cdot : \K \times \K \to \K$ und zwei Elementen $0, 1 \in \K$, sodass
  \begin{itemize}
    \item $(\K, +, 0)$ eine Gruppe bildet,
    \item $(\K \setminus \{ 0 \}, \cdot, 1)$ eine Gruppe bildet und
    \item folgende Distributivgesetze für alle $a, b, c \in R$ erfüllt sind:
    \[
      (a + b) \cdot c = a \cdot c + b \cdot c, \qquad
      a \cdot (b + c) = a \cdot b + a \cdot c.
    \]
  \end{itemize}
\end{defn}

\begin{bem}
  Jeder Körper ist auch ein Ring.
\end{bem}

\begin{nota}
  $\K[x] \coloneqq \{ \text{Polynome mit Koeffizienten in $\K$} \}$
\end{nota}

\begin{bem}
  Die Menge aller Polynome über $\K$ bildet einen Ring.
\end{bem}

\begin{defn}
  In einem Ring $R$ teilt ein Element $g \in R$ ein anderes Element $f \in R$, geschrieben $g \divides f$, falls es ein $h \in R$ mit $g \cdot h = f$ gibt.
\end{defn}

\begin{bem}
  Ein Ring, in dem Division mit Rest möglich ist (z.\,B. der Polynomring oder $\Z$), wird \emph{euklidischer Ring} genannt. In solchen Ringen kann man den euklidischen Algorithmus ausführen.
\end{bem}

\begin{satz}
  Ist $x_0 \in \K$ eine Nullstelle des Polynoms $f \in \K[x]$, dann gilt $(X - x_0) \divides f$, genauer $f = (x{-}x_0) \cdot g$ für ein $g {\in} \K[x]$ mit $\partial g = \partial f {-} 1$.
\end{satz}

\begin{kor}
  Ein Polynom $f \in \K[x]$ vom Grad $n \geq 1$ hat höchstens $n$ Nullstellen.
\end{kor}

\begin{kor}
  Wenn $\K$ unendlich viele Elemente hat, sind die Koeffi- zienten von jedem $f \in \K[x]$ durch die Fkt. $f$ eindeutig bestimmt.
\end{kor}

\begin{satz}[Hauptsatz der Algebra]
  Jedes Polynom $f \in \C[x]$ ist Produkt von Polynomen vom Grad $1$, sogenannten Linearfaktoren, also
  \[
    f = a \cdot (x - x_1) \cdot ... \cdot (x - x_n)
    \quad \text{mit $a, x_1, ..., x_n \in \C$}.
  \]
\end{satz}

\begin{bem}
  Die Zahlen $x_1, ..., x_n$ müssen nicht alle verschieden sein.
\end{bem}

\begin{defn}
  Die Anzahl der Vorkommen einer Nullstelle $x_i$ in obiger Produktdarstellung heißt \emph{Vielfachheit} der Nullstelle.
\end{defn}

\begin{defn}
  Die \emph{Ableitung} des Polynoms
  \[ f(x) = a_0 x^n + a_1 x^{n-1} + ... + a_{n-1} x + a_0 \in \K[x] \]
  ist das Polynom
  \[ f'(x) = n a_0 x^{n-1} + (n{-}1) a1 x^{n-2} + ... + a_{n-1}. \]
\end{defn}

\begin{bem}
  Sei $x_i$ eine $k$-fache Nullstelle von $f \in \C[x]$. Dann ist $x_i$ auch eine $(k{-}1)$-fache Nullstelle von $f'$.
\end{bem}

\begin{defn}
  Ein Körper $\K$ mit der Eigenschaft, dass jedes Polynom $f \in \K[x]$ in Linearfaktoren zerfällt, heißt \emph{algebraisch abgeschlossen}.
\end{defn}

\begin{defn}
  Eine Zahl $c \in \C$ heißt \emph{algebraisch}, wenn es ein Polynom $f \in \Q[x]$, $f \not= 0$ mit $f(c) = 0$ gibt.
\end{defn}

\begin{bem}
  Man kann zeigen, dass die Menge der algebraischen Zahlen ein abzählbarer, algebraisch abgeschlossener Körper ist.
\end{bem}

% Kapitel 1.11. Der Wurzelsatz von Vieta

\begin{defn}
  Die \emph{elementarsymmetrischen Funktionen} in $x_1, ..., x_n$ sind die Polynome
  \[
    e_k(x_1, ..., x_n) \coloneqq \sum_{\mathclap{j_1 < ... < j_k}} x_{j_1} \cdot ... \cdot x_{j_k}
    \quad \text{für $1 \leq k \leq n$.}
  \]
\end{defn}

\begin{bem}
  Bezeichne mit $e_j$ für $1 \leq j \leq n$ die elementarsymmetrischen Funktionen in den Variablen $x_1, ..., x_n$, mit $\tilde{e}_i$ für $1 \leq i < n$ die elementarsymmetrischen Funktionen in den Variablen $x_1, ..., x_{n-1}$. Dann gelten die Rekursionsgleichungen
  \[
    e_1 = x^n + \tilde{e}_1, \qquad
    e_i = \tilde{e}_i + x_n \cdot \tilde{e}_{i-1}, \qquad
    e_n = x_n \cdot \tilde{e}_{n-1}.
  \]
\end{bem}

\begin{satz}[Vieta]
  Sei $f \in \K[X]$ ein normiertes Polynom, das über $\K$ in Linearfaktoren zerfällt, also
  \[ f(x) = x^n + a_1 x^{n-1} + ... + a_n = (x-x_1) \cdot ... \cdot (x-x_n), \]
  dann gilt $a_j = (-1)^j e_j(x_1, ..., x_n)$ für alle $1 \leq j \leq n$.
\end{satz}

\begin{defn}
  Eine \emph{Permutation} der Zahlen $\{ 1, ..., n \}$ ist eine Bijektion
  \[ \sigma : \{ 1, ..., n \} \to \{ 1, ..., n \}. \]
  Die Menge dieser Permutationen heißt \emph{symmetrische Gruppe} $S_n$.
\end{defn}

\begin{defn}
  Ein Polynom $f \in \K[x_1, ..., x_n]$ heißt \emph{symmetrisch}, falls für alle $x_1, ..., x_n \in \K$ und Permutationen $\sigma$ gilt:
  \[ f(x_1, ..., x_n) = (\sigma f)(x_1, ..., x_n) \coloneqq f(x_{\sigma(1)}, ...., x_{\sigma(n)}) \]
\end{defn}

% Ausgelassen: Definition Umkehrabbildung

\begin{satz}[Hauptsatz über symmetrische Polynome]
  Jedes symme- trische Polynom $f(\vec{x})$ mit $\vec{x} = (x_1, ..., x_n)$ lässt sich als Polynom in den elementarsymmetrischen Polynomen $e_1(\vec{x}), ..., e_n(\vec{x})$ darstellen.
\end{satz}

\begin{kor}
  Sind $x_1, ..., x_n$ die Wurzeln eines normierten Polynoms $f(x) = x^n + a_1 x^{n-1} + ... + a_n$, dann gilt für jedes symmetrische Polynom $s \in \K[y_1, ..., y_n]$: $s(x_1, ..., x_n)$ ist ein Polynomausdruck in den Koeffizienten $a_1 ..., a_n$ und damit aus diesen Zahlen berechenbar.
\end{kor}

\begin{defn}
  Die \emph{Diskriminante} eines Polynoms $f = (x - x_1) \cdot ... \cdot (x - x_n)$ ist der Ausdruck
  \[ \Delta (\vec{x}) \coloneqq \pm \prod_{i \not= j} (x_i - x_j). \]
  Da dieser Polynomausdruck symmetrisch ist, lässt er sich in den Koeffizienten des Polynoms $f$ darstellen.
\end{defn}

\begin{bsp}
  Die Diskriminante des quadratischen Polynoms $f(x) = x^2 - ax + b$ ist $- \Delta = a^2 - 4 b$, die des kubischen Polynoms $g(x) = x^3 - a x^2 + b x - c$ ist $\Delta = a^2 b^2 - 4 a^3 c - 4 b^3 + 18 abc - 27 c^3$.
\end{bsp}

% Kapitel 1.13. Lagrangesche Resolventen

% Ausgelassen: Ausführliche Erklärung, Rechnung

\begin{defn}
  Seien $\omega$ eine $n$-te Einheitswurzel, d.\,h. $\omega^n = 1$ und $x_1, ..., x_n$ die Nullstellen von $x^n + a_1 x^{n-1} + ... + a_n = 0$, dann heißt
  \[
    u_{\omega} \coloneqq x_n + \omega x_{n-1} + ... + \omega^{n-1} x_1 \quad
    \text{\emph{Lagrangesche Resolvente}.}
  \]
\end{defn}

\begin{bem}
  Es gilt $\sigma u_{\omega} = \omega u_{\omega}$ für $\sigma = (123 \cdots n)$.
\end{bem}

% Kapitel 1.14. Die Lösung der quartischen Gleichung

% Ausgelassen

% Kapitel 1.15. Gruppen

\begin{defn}
  Ein \emph{Gruppen-Homomorphismus} zwischen $(G, *_G)$ und $(H, *_H)$ ist eine Abbildung $\phi : G \to H$, sodass für alle $g, h \in G$ gilt:
  \begin{itemize}
    \miniitem{0.6 \linewidth}{$\phi(g *_G h) = \phi(g) *_{H} \phi(h)$}
    \miniitem{0.37 \linewidth}{$\phi(g)^{-1} = \phi(g^{-1})$}
  \end{itemize}
\end{defn}

\begin{defn}
  Ein \emph{Gruppen-Isomorphismus} ist ein bijektiver Gruppen-Homomorphismus. Die Umkehrabbildung ist automatisch ebenfalls ein Gruppen-Isomorphismus.
\end{defn}

\begin{defn}
  Zwei Gruppen $G$ und $H$ heißen \emph{isomorph} (notiert $G \cong H$), wenn es einen Gruppenisomorphismus zwischen ihnen gibt. Dann ist die Umkehrabbildung ebenfalls ein Gruppenisomorphismus.
\end{defn}

\begin{bspe}
  \begin{itemize}
    \item $(\Z, +, 0)$ ist eine kommutative Gruppe.
    \item Die Menge der $n$-ten Einheitswurzeln bilden eine Gruppe $(\Omega_n, \cdot, 1)$ mit $\Omega_n \coloneqq \Set{ e^{2i \pi k/n} }{ 0 \leq k \leq n - 1 }$
  \end{itemize}
\end{bspe}

\begin{defn}
  Eine \emph{Untergruppe} einer Gruppe $(G, *, e)$ ist eine Teilmenge $H \subset G$, für die $(H, *|_{H \times H}, e)$ selbst eine Gruppe ist, d.\,h. es gilt
  \begin{itemize}
    \miniitem{0.15 \linewidth}{$e \in H$}
    \miniitem{0.45 \linewidth}{$\fa{h, h' \in H} h * h' \in H$}
    \miniitem{0.35 \linewidth}{$\fa{h \in H} h^{-1} \in H$.}
  \end{itemize}
\end{defn}

% Ausgelassen: Beispiel: Symmetriegruppe einer Figur

% Kapitel 1.16. Gruppenwirkungen

\begin{defn}
  Eine \emph{Wirkung (Operation)} einer Gruppe $(G, *, e)$ auf einer Menge $X$ ist ein Gruppenhomomorphismus $\phi : G \to \mathrm{Aut}(X)$, wobei $\mathrm{Aut}(X)$ die Menge der Bijektionen von $X$ nach $X$ bezeichnet bzw. äquiv. eine Abb. $\phi : G \times X \to X, (g, x) \mapsto gx \coloneqq \phi(g, x)$, für die gilt:
  \begin{itemize}
    \miniitem{0.3 \linewidth}{$\phi(e, \blank) = \id_{X}$,}
    \miniitem{0.6 \linewidth}{$\fa{g, h \in G} \phi(g, \blank) \circ \phi(h, \blank) = \phi(g * h, \blank)$.}
  \end{itemize}
\end{defn}

\begin{defn}
  Für jede Gruppenwirkung $\phi$ von $G$ auf $X$ und jedes Element $x \in X$ ist $G_x \coloneqq \Set{ g \in G }{ gx = x }$ eine Untergruppe von $G$, die \emph{Standgruppe} oder \emph{Stabilisator} von $x$ unter $\phi$.
\end{defn}

\begin{defn}
  Eine Gruppenwirkung $\phi$ von $G$ auf $X$ heißt \emph{transitiv}, falls es für alle $x_1, x_2 \in X$ ein $g \in G$ mit $\phi(g, x_1) = x_2$ gibt.
\end{defn}

\begin{defn}
  Für $x \in X$ heißt $Gx \coloneqq \Set{ gx }{ g \in G }$ \emph{Orbit} oder \emph{Bahn} von $x$.
\end{defn}

\begin{bem}
  Für alle $g \in G$ und $x \in X$ gilt: $Gx = G(gx)$.
\end{bem}

\begin{bem}
  Für alle $x' = gx \in Gx$ für ein $g \in G$ gilt $G_x \cong G_{x'}$, genauer
  $G_{x'} = g G_x g^{-1}$.
\end{bem}

\begin{satz}
  Für eine endliche Gruppe $G$, eine Menge $X$ mit Gruppenwirkung $\phi : G \times X \to X$ gilt: $\abs{Gx} = \frac{\abs{G}}{\abs{G_x}}$.
\end{satz}

\begin{defn}
  Für eine Untergruppe $H \subset G$ und $g \in G$ heißt
  \begin{itemize}
    \item $gH \coloneqq \Set{ gh }{ h \in H }$ \emph{Linksnebenklasse} von $H$,
    \item $Hg \coloneqq \Set{ hg }{ h \in H }$ \emph{Rechtsnebenklasse} von $H$.
  \end{itemize}
\end{defn}

\begin{defn}
  Ein \emph{Normalteiler} einer Gruppe $(G, *, e)$ ist eine Untergruppe $H$, die die folgenden äquivalenten Bedingungen erfüllt:
  \begin{itemize}
    \miniitem{0.99 \linewidth}{Links- und Rechtsnebenklassen sind gleich: $\fa{g \in G\!}\!gH = Hg$}
    \miniitem{0.48 \linewidth}{$\fa{g \in G} g H g^{-1} = H$}
    \miniitem{0.48 \linewidth}{$\fa{g \in G, h \in H} ghg^{-1} \in H$}
  \end{itemize}
\end{defn}

% Kapitel 1.17. Die alternierende Gruppe

\begin{defn}
  Seien $i, j \in \{ 1, ..., n \}$ mit $i \not= j$. Dann ist die \emph{Transposition} von $i$ und $j$ die Abbildung, die $i$ und $j$ vertauscht, also
  \[
    (ij) : \{ 1, ..., n \} \to \{ 1, ..., n \}, \quad k \mapsto \begin{cases}
      j, & \text{falls $k = i$,}\\
      i, & \text{falls $k = j$,}\\
      k, & \text{sonst}
    \end{cases}
  \]
\end{defn}

\begin{bem}
  Jede Permutation kann als Komposition von Transpositionen geschrieben werden.
\end{bem}

\begin{defn}
  Ein \emph{Fehlstand} einer Permutation $\sigma$ auf $\{ 1, ..., n \}$ ist ein Zahlenpaar $(i, j)$ mit $i < j$ und $\sigma(i) > \sigma(j)$.
\end{defn}

\begin{defn}
  Zwei Zahlen $a, b \in \Z$ haben gleiche \emph{Parität}, falls $a \equiv b \pmod{2}$, also $a - b$ gerade ist.
\end{defn}

\begin{prop}
  Die Anzahl der Fehlstände einer Permutation $\sigma$ hat die gleiche Parität wie die Anzahl der Transpositionen in einer Darstellung von $\sigma$ als Komposition von Transpositionen.
\end{prop}

\begin{defn}
  Die Untergruppe $A_n \subset S_n$ der symmetrischen Gruppe, die aus allen Transpositionen mit gerader Anzahl an Fehlstellungen besteht, heißt \emph{Alternierende Gruppe}.
\end{defn}

% Kapitel 1.18. Die Galoisgruppe

\begin{defn}
  Ein Polynom $f \in \K[x]$ heißt \emph{separabel}, falls alle Nullstellen voneinander verschieden sind.
\end{defn}

\begin{bem}
  Ein Polynom vom Grad $\geq 1$ ist genau dann separabel, falls seine Diskriminante ungleich $0$ ist.
\end{bem}

\begin{defn}
  Sei $f \in \K[x]$ ein separables Polynom mit Nullstellen $x_1, ..., x_n$.
  \begin{itemize}
    \item Eine \emph{algebraische Relation} zwischen den Nullstellen über $\K$ ist ein Polynom $f \in \K[x_1, ..., x_n]$ mit $f(\alpha_1, ..., \alpha_n) = 0$.
    \item Die \emph{Galoisgruppe} $G$ von $f$ ist die Gruppe aller $n$-stelligen Permutationen, die alle algebraischen Relationen erhalten, d.\,h.
    \begin{align*}
      G \coloneqq \{ \sigma \in S_n \mid & \, \fa{f \in \K[x_1, ..., x_n]} f(\alpha_1, ..., \alpha_n) = 0 \\
      & \implies (\sigma f)(\alpha_1, ..., \alpha_n) = 0 \}.
    \end{align*}
  \end{itemize}
\end{defn}

\begin{lem}
  Sei $f \in \K[x]$ reduzibel, d.\,h. $f = gh$ für zwei nicht-konstante Polynome $g, h \in \K[x]$. Dann wirkt die Galois nicht transitiv.
\end{lem}

\begin{bsp}
  Sei $f \in \K[x]$ separabel mit Diskriminante $\Delta$. Angenommen,
  \[ D = \sqrt{\Delta} = \prod_{i < j} (x_i - x_j) \in \K. \]
  Dann gilt $G \subset A_n$ für die Galoisgruppe $G$ von $f$, da Transpositionen gerade das Vorzeichen von $\sqrt{\Delta}$ vertauschen.
\end{bsp}

\begin{samepage}

\begin{bsp}
  Die Galoisgruppe des Polynoms $f(x) = x^n - 1$, dessen Nullstellen \emph{$n$-te Einheitswurzeln} genannt werden, ist
  \[ G = \Set{ m \mapsto k \cdot m \Mod{n} }{ k \in \{ 1, ..., n{-}1 \}, \ggT(k, n) = 1 }. \]
\end{bsp}

% Ausgelassen: Galoisgruppe von $f(x) = x^n - 1$

% Kapitel 1.19. Die Lösung der quintischen Gleichung

\begin{bem}
  Der \emph{Ikosaeder} ist der platonische Körper, der $20$ gleichseitige Dreiecke als Seitenflächen, $12$ Eckpunkte und $30$ Kanten besitzt. Seine Drehgruppe ist die $A_5$.
\end{bem}

% Kapitel 2. Körpererweiterungen
\section{Körpererweiterungen}

\end{samepage}

% Kapitel 2.20. Körper

\begin{lem}
  Sei $p$ eine Primzahl. Dann ist jedes Element $n \not\equiv 0 \Mod{p}$ invertierbar, d.\,h. es gibt $m \in \Z$ mit $n \cdot m \equiv 1 \Mod{p}$.
\end{lem}

\begin{defn}
  Sei $p \in \N$ eine Primzahl. Dann bilden die Restklassen modulo $p$ einen Körper $\F_p \coloneqq \Z / (p \Z)$.
\end{defn}

\begin{defn}
  Ein Körper $\K$ hat \emph{Charakteristik} $p \in \N \setminus \{ 0 \}$, falls $1 + ... + 1 = 0$ ($p$-Mal die $1$) in $\K$ gilt. Falls es kein solches $p$ gibt, so hat der Körper Charakteristik $0$.
\end{defn}

\begin{defn}
  Sei $R$ ein Ring. Ein \emph{Ideal} in $R$ ist eine Teilmenge $I \subset R$, für die gilt: $\fa{r \in R, i \in I} ri = i$ ("`Magneteigenschaft"').
\end{defn}

\begin{defn}
  Ein Ideal $I \subset R$ heißt \emph{maximal}, falls es kein Ideal $J \subsetneq R$ mit $I \subsetneq J$ gibt.
\end{defn}

\begin{bem}
  Für jede Primzahl $p$ ist $p \Z \coloneqq \Set{ pz }{ z \in \Z }$ ein maximales Ideal.
\end{bem}

% Kapitel 2.21. Körpererweiterungen

\begin{defn}
  Ein \emph{Teilkörper} eines Körpers $\L$ ist eine Teilmenge $\K \subset \L$ mit $\{ 0, 1 \} \subset \K$, die unter Multiplikation, Addition und multiplikativer und additiver Inversenbildung abgeschlossen ist.
\end{defn}

\begin{defn}
  Sei $\K \subset \C$ ein Teilkörper und $\alpha_1, ..., \alpha_n \in \C$. Dann ist die Körpererweiterung $\K(\alpha_1, ..., \alpha_n)$ der Körper, der aus $\K$ durch Hinzufügen ("`Adjungieren"') von $\alpha_1, ..., \alpha_n$ und allen durch Multiplikation, Addition und Inversenbildung entstehenden Zahlen besteht.
\end{defn}

\begin{defn}
  Eine Körpererweiterung $\L \supset \K$ heißt \emph{endlich}, falls es Elemente $\alpha_1, ..., \alpha_k \in \L$ gibt, sodass jede Zahl $y \in \K$ eindeutig als Linearkombination $y = \lambda_1 \alpha_1 + ... + \lambda_k \alpha_k$ mit $\lambda_1, ..., \lambda_k \in \K$ geschrieben werden kann. Die Zahl $[\L : \K] \coloneqq k$ wird \emph{Grad} der Körpererweiterung genannt.
\end{defn}

\begin{bem}
  Der Schnitt von beliebig vielen Teilkörpern ist ein Teilkörper. Man kann also $\K(\alpha_1, ..., \alpha_n)$ auch als kleinsten Teilkörper von $\C$, der die Menge $\K \cup \{ \alpha_1, ...., \alpha_n \}$ enthält, beschreiben.
\end{bem}

\begin{defn}
  Sei $f \in \K[x]$ und $\alpha$ eine Nullstelle von $f$. Das Polynom $f$ heißt \emph{Minimalpolynom} von $\alpha$, falls für alle Polynome $g \in \K[x]$ gilt: $g(\alpha) = 0 \implies f \divides g$ (insbesondere).
\end{defn}

\begin{defn}
  Ein normiertes Polynom $f \in \K[x]$ heißt \emph{irreduzibel} über $\K$, falls es keine Zerlegung von $f$ als $f = gh$ mit normierten $g, h \in \K[x]$ und $g \not= 1$, $h \not= 1$ gibt.
\end{defn}

\begin{defn}
  Sei $f \in \K[x]$ normiert, irreduzibel und $\alpha$ eine Nullstelle von $f$. Dann heißt $f$ \emph{Minimalpolynom} von $\alpha$.
\end{defn}

% Selbst formuliert.
\begin{bem}
  Sei $\alpha$ eine Nullstelle eines Polynoms $f \in \Q[x]$. Dann existiert ein eindeutiges, irreduzibles Polynom $g \in \K[x]$ mit $g(\alpha) = 0$.
\end{bem}

\begin{satz}
  Sei $\alpha$ eine Nullstelle eines irreduziblen Polynoms $f \in \K[x]$. Dann ist $\{ 1, \alpha, ..., \alpha^{n-1} \}$ eine Basis von $\K(\alpha)$.
\end{satz}

\begin{kor}
  $\K(\alpha) = \K[\alpha]$ und $[\K(\alpha) : \K] = n$.
\end{kor}

% Kapitel 2.22. Konstruktionen mit Zirkel und Lineal

\begin{defn}
  Wir betrachten die Zahlenebene $\C$. Eine Zahl $z \in \C$ heißt aus vorgegebenen Zahlen $\alpha_1, ... \alpha_k$ \emph{konstruierbar}, wenn $z$ der Schnitt zweier Kreise, zweier Geraden oder einer Geraden und eines Kreises ist, wobei wir nur solche Geraden betrachten, die durch zwei vor- gegebenen Zahlen laufen und solche Kreise, die als Mittelpunkt eine vorgegebene Zahl haben und durch eine vorgegebene Zahl laufen.
\end{defn}

\begin{defn}
  Eine Zahl $z \in \C$ heißt (in $k-2$ Schritten) \emph{konstruierbar}, falls es eine Zahlenfolge $z_0 = 0, z_1 = 1, z_2, ..., z_k = k$ gibt, sodass $z_m$ aus $z_0, ..., z_{m-1}$ für alle $m \geq 3$ konstruierbar ist.
\end{defn}

\begin{defn}
  Sei $x$ aus $0, 1, \alpha_1, ..., \alpha_k$ konstruierbar. Dann gilt
  \[ [ \Q(\alpha_1, ..., \alpha_k, x) : \Q(\alpha_1, ..., \alpha_k) ] \in \{ 1, 2 \}. \]
\end{defn}

\begin{lem}
  Sind $\K \subset \K' \subset \K''$ endliche Körpererweiterungen, so gilt
  \[ [ \K'' : \K ] = [ \K'' : \K' ] \cdot [ \K' : \K ]. \]
\end{lem}

\begin{satz}
  Sei $\alpha \in \C$ in $n$ Schritten konstruierbar. Dann gilt
  \[
    [ \Q(\alpha) : \Q ] = 2^k
    \enspace \text{für ein } k \leq n.
  \]
\end{satz}

% Kapitel 2.23. Würfelverdopplung und Winkeldreiteilung

\begin{satz}
  $\sqrt[3]{2}$ ist keine konstruierbare Zahl.
\end{satz}

\begin{satz}
  Die Zahl $e^{\pi/9 i}$ ist nicht konstruierbar. Folglich kann der Winkel $\pi / 3 = 60^{\circ}$ nicht gedrittelt werden.
\end{satz}

% Kapitel 2.24. Konstruktion regelmäßiger Vielecke

\begin{lem}
  Ist $p = qr$ ein Produkt teilerfremder Zahlen. Dann ist das regelmäßige $p$-Eck genau dann konstruierbar, wenn das regelmäßige $q$-Eck und das regelmäßige $r$-Eck konstruierbar ist.
\end{lem}

\begin{lem}
  Sei $p \in \N$ prim. Dann ist das $p$-te Kreisteilungspolynom
  \[ f(x) = \frac{x^p - 1}{x-1} = x^{p-1} + x^{p-2} + ... + x + 1 \]
  irreduzibel. Sei $\zeta \in \C$ mit $f(\zeta) = 0$. Dann gilt $[ \Q(\zeta) : Q ] = p-1$.
\end{lem}

\begin{lem}
  Sei $2^s + 1$ eine Primzahl. Dann ist $s$ eine Zweierpotenz.
\end{lem}

\begin{bem}
  Zahlen der Form $2^{2^r} + 1$ heißen \emph{Fermatzahlen}. Die ersten $5$ Fermatzahlen
  $3$, $5$, $17$, $257$, $65537$ sind Primzahlen, die sechste nicht.
\end{bem}

\begin{satz}
  Sei $p$ eine Primzahl. Wenn das $p$-Eck konstruierbar ist, dann ist $p$ eine Fermatzahl.
\end{satz}

\begin{satz}[Gauß]
  Das $17$-Eck ist konstruierbar.
\end{satz}

% Kapitel 2.25. Irreduzibilität über $\Z$

\begin{defn}
  Ein Polynom $f \in \Z[x]$ heißt \emph{reduzibel} über $\Z$, falls es Polynome $g, h \in \Z[x]$ mit $g \not= 1$, $h \not= 1$ und $f = gh$ gibt.
\end{defn}

\begin{defn}
  Sei $p$ eine Primzahl. Ein Polynom $f \in \F_p[x]$ heißt \emph{reduzibel} über $\F_p$, wenn es nichtkonstante $g, h \in \F_p[x]$ mit $f = gh$ gibt.
\end{defn}

\begin{defn}
  Angenommen, ein normiertes Polynom $f \in \Z[x]$ ist reduzibel über $\Z$. Dann ist auch $f$ aufgefasst als $f \in \F_p[x]$ reduzibel.
\end{defn}

\begin{satz}[Eisenstein]
  Sei $p$ eine Primzahl und $f = a_0 x^n + a_1 x^{n-1} + ... + a_n \in \Z[x]$ mit $p \divides a_i$ für $i \in \{ 1, ..., n \}$, aber $p \not\divides a_0$ und $p^2 \not\divides a_n$. Dann ist $f$ irreduzibel über $\Z$.
\end{satz}

\begin{bem}
  Oft kann man das Eisenstein-Kriterium für $f \in \Z[x]$ nicht direkt anwenden. Dann kann man $g(x) \coloneqq f(x + k)$ für eine Zahl $k \in \Z$ betrachten. Dann ist $f$ genau dann über $\Z$ irreduzibel, wenn $g$ es auch ist. Man kann also versuchen, $k$ so zu wählen, dass man das Eisenstein-Kriterium auf $g$ anwenden kann.
\end{bem}

% Kapitel 2.26. Irreduzibilität über $\Q$

\begin{defn}
  Ein Polynom $f = a_0 x^n + ... + a_n \in \Z[x]$ heißt \emph{primitiv}, falls $\ggT(a_0, ..., a_n) = 1$.
\end{defn}

\begin{lem}
  % Das Produkt primitiver Polynome ist primitiv.
  Sind $g, h \in \Z[x]$ primitiv, dann ist es auch $gh$.
\end{lem}

\begin{satz}
  Wenn $f \in \Z[x] \subset \Q[x]$ über $\Q$ reduzibel ist, also $f = gh$ mit $g, h \in \Q[x]$, dann auch über $\Z$, genauer $f = g_0 h_0$ für $g_0, h_0 \in \Z[x]$, wobei $g_0$ und $h_0$ rationale Vielfache von $g$ bzw. $h$ sind.
\end{satz}

\begin{kor}
  Die Nullstellen von normierten ganzzahligen Polynomen sind ganzzahlig oder irrational.
\end{kor}

% Kapitel 2.27. Die Galoisgruppe einer Körpererweiterung

\begin{defn}
  Ein Homomorphismus zwischen Körpern $\K$ und $\K'$ ist eine Abbildung $f : \K \to \K'$, sodass für alle $a, b \in \K$ gilt:
  \[
    f(0) = 0, \quad f(1) = 1, \quad f(a + b) = f(a) + f(b), \quad f(a \cdot b) = f(a) \cdot f(b).
  \]
\end{defn}

\begin{bem}
  Die Verknüpfung von Körperhomomorphismen ist ein Körperhomomorphismus. Falls $f$ bijektiv ist, dann ist auch $f^{-1}$ ein Körperautomorphismus und $f$ heißt Körperisomorphismus.  Wenn zusätzlich $\K = \K'$ ist, so heißt $f$ Körperautomorphismus.
\end{bem}

\begin{nota}
  $\Aut(\K) \coloneqq \Set{ \sigma : \K \to \K }{ \sigma \text{ Körperautomorphismus} }$
\end{nota}

\begin{defn}
  Die \emph{Galoisgruppe} einer Körpererweiterung $\L \supset \K$ ist
  \[ G = \Gal(\L, \K) \coloneqq \Set{\sigma \in \Aut(\L) }{ \sigma|_\K = \id_{\K}. } \]
\end{defn}

\begin{defn}
  Sei $f \in \K[x]$ separabel mit Nullstellen $x_1, ..., x_n$. Dann heißt $\L = \K(x_1, ..., x_n)$ \emph{Zerfällungskörper} von $f$ über $\K$. Die Körpererweiterung $\L$ wird dann \emph{normal} genannt.
\end{defn}

\begin{lem}
  Sei $f \in \K[x]$ und $N(f, \L) \coloneqq \Set{ x \in \L }{ f(x) = 0 }$.\\
  Dann gilt für alle $\sigma \in \Gal(\L, \K)$:
  \begin{itemize}
    \miniitem{0.52 \linewidth}{$f(\sigma x) = \sigma f(x)$ für jedes $x \in \L$}
    \miniitem{0.46 \linewidth}{$\sigma N(f, \L) = N(f, \L)$}
  \end{itemize}
\end{lem}

\begin{bem}
  Sei $f \in \K[x]$ separabel mit Nullstellen $N(f) \coloneqq x_1, ..., x_n$ und $\L = \K(x_1, ..., x_n)$. Wegen Punkt 2 wirkt die Galoisgruppe $\Gal(\L, \K)$ auf der Nullstellenmenge $N(f) = N(f, \L)$. Wenn man die Nullstellen durchnummeriert, erhält man eine Abbildung $\phi : \Gal(\L, \K) \to S_n$, sodass $\fa{j} x_{\phi(\sigma)(j)} = \sigma(x_j)$.
\end{bem}

\begin{lem}
  Sei $\L = \K(x_1, ..., x_n)$ wie in der Bemerkung. Dann ist $\phi : G {\to} S_n$ injektiv und identifiziert $G$ mit einer Untergruppe von $S_n$.
\end{lem}

\begin{satz}
  Sei $f \in \K[x]$ separabel mit Nullstellen $x_1, ..., x_n$ und $\L \coloneqq \K(x_1, ..., x_n)$ der Zerfällungskörper von $f$. Sei
  \[ R \coloneqq \Set{ h \in \K[x_1, ..., x_n] }{ h(x_1, ..., x_n) = 0 } \]
  die Menge der algebraischen Relationen zwischen $x_1, ..., x_n$. Dann ist
  \[
    \phi(G) = \Set{ \sigma \in S_n }{ \fa{H \in R} \sigma H \in R }.
  \]
\end{satz}

\begin{bem}
  Folglich entspricht die Galoisgruppe des Zerfällungskörpers von $f$ über dem Grundkörper der vorher definierten Galoisgruppe von $f$.
\end{bem}

% Kapitel 2.18. Fortsetzung von Körper-Isomorphismen

\begin{satz}
  Sei $\sigma : \K \to \tilde{\K}$ ein Körperisomorphismus, das Polynom $f \in \K[x]$ separabel mit Zerfällungskörper $\L$ sowie $\tilde{f} = \sigma f$ mit Zerfällungskörper $\tilde{\L}$. Dann gilt
  \[ \abs{\Set{ \hat{\sigma} \in \Iso(\L, \tilde{L}) }{ \hat{\sigma}|_{\K} = \sigma }} = [\L : \K]. \]
\end{satz}

\begin{kor}
  Ist $\L \supset \K$ der Zerfällungskörper eines separablen Polynoms $f \in \K[x]$, so gilt
  \[ \abs{\Gal(\L, \K)} = [\L : \K]. \]
\end{kor}

% Kapitel 2.29. Auflösbarkeit durch Radikale

% Ausgelassen: Motivation Turm von Radikalerweiterungen

\begin{bem}
  Angenommen, die Nullstellen eines separablen Polynoms $f \in \K[x]$ lassen sich durch einen Wurzelausdruck angeben. Dann muss es Reihen von Zahlen $\alpha_1, ..., \alpha_s \in \L$ und $n_1, ..., n_s \in \N$ und Körpererweiterungen $\K_0 \subset ... \subset \K_s$ geben mit
  \[
    \K_0 = \Q, \quad
    \K_{j+1} = \K_j(\alpha_j) \text{ mit } \alpha_j^{n_j} \in \K_j, \quad
    \K_s = \L.
  \]
  Wir sagen, $\L$ komme durch Adjunktion von Wurzeln zustande.
\end{bem}

\begin{lem}
  Sind $\K \subset \K' \subset \L$ endliche Körpererweiterungen, so ist $\Gal(\L, \K')$ eine Untergruppe von $\Gal(\L, \K)$. Sei $\K'$ der Zerfällungs- körper des separablen Polynoms $g \in \K[x]$. Dann lassen die Elemente von $\Gal(\L, \K)$ den Körper $\K$ invariant (d.\,h. $\sigma(\K') \subset \K'$ für alle $\sigma \in \Gal(\L, \K)$) und $\Gal(\L, \K')$ ist ein Normalteiler von $\Gal(\L, \K)$.
\end{lem}

\begin{lem}
  Seien $\K \subset \K'$ und $\K' \subset \L$ endliche, normale Körper- erweiterungen. Dann gilt
  \[ \Gal(\K', \K) = \Gal(\L, \K) / \Gal(\L, \K'). \]
\end{lem}

\begin{lem}
  Sei $\rho : G \to H$ ein Gruppenhomomorphismus. Dann ist $K \coloneqq \ker(\rho) \subset G$ ein Normalteiler von $G$. Wenn $\rho$ surjektiv ist, dann definiert $\rho$ einen Isomorphismus
  \[ \overline{\rho} : G / K \to H, \quad gK \mapsto \rho(g) \]
\end{lem}

\begin{defn}
  Eine Gruppe $G$ heißt \emph{auflösbar}, wenn es eine absteigende Reihe von Untergruppen
  \[ G = G_0 \supset G_1 \supset \supset ... \supset G_r = \{ \id \} \]
  gibt, sodass $G_{j+1}$ ein Normalteiler in $G_j$ und $G_{j} / G_{j+1}$ für alle $j = 1, ..., r$ abelsch ist.
\end{defn}

\begin{satz}
  Sei $f \in \K[x]$ separabel mit Zerfällungskörper $\L$. Dieser komme durch Adjunktion von $r$ Wurzeln der Grade $n_1, ..., n_r$ zustande. Der Grundkörper $\K$ möge alle $n_j$-ten Einheitswurzeln für $j = 1, ..., n$ enthalten. Dann ist die Gruppe $\Gal(\L, \K)$ auflösbar.
\end{satz}

\begin{bem}
  Man kann auf die Voraussetzung verzichten, dass die $n_j$-ten Einheitswurzeln in $\K$ enthalten sind.
\end{bem}

\end{document}