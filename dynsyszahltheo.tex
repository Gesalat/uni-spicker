\documentclass{cheat-sheet}

\pdfinfo{
  /Title (Dynamische Systeme und Zahlentheorie)
  /Author (Tim Baumann)
}

\usepackage{mathabx} % \divides
\usepackage{bbm} % Für 1 mit Doppelstrich (Indikatorfunktion)
\usepackage{stmaryrd} % \lightning
\usepackage{pgffor} % \foreach-Schleifen

\newcommand{\nspace}[1]{\foreach \i in {1,...,#1}{ \! }} % Negativer Abstand
\DeclareMathOperator{\Aut}{Aut} % Automorphismengruppe
\DeclareMathOperator{\End}{End} % Endomorphismenmonoid
\newcommand{\AutEnd}{\Aut\!/\!\End} % Automorphismengruppe bzw. Endomorphismenmonoid
\DeclareMathOperator{\Iso}{Iso} % Isometriegruppe
\newcommand{\clos}[1]{\overline{#1}} % topologischer Abschluss
\DeclareMathOperator{\inte}{int} % Inneres (interior)
\newcommand{\Bor}{\mathcal{B}} % ein Maß (B für Borel)
\newcommand{\meS}{m.\,e.\,S.} % maßerhaltendes System
\newcommand{\meST}{$(X, \Bor, \mu, T)$} % maßerhaltendes-System-Tupel
\newcommand{\ind}{\mathbbm{1}} % Indikatorfunktion
\newcommand{\Lin}{\mathcal{L}} % lineare Funktionale
\newcommand{\Cont}{\mathcal{C}} % Menge der stetigen/diff'baren Funktionen
\newcommand{\Alg}{\mathfrak{A}} % (Mengen-)Algebra
\newcommand{\Meas}{\mathcal{M}} % Menge der Maße
\newcommand{\Boun}{\mathcal{B}} % Menge der beschränkten (bounded) Operatoren
\newcommand{\scp}[2]{\langle #1 , #2 \rangle} % Skalarprodukt

\begin{document}

\maketitle{Dyn. Systeme in der Zahlentheorie}

% TODO:
% Szemerédis Theorem
% Rados Theorem
% "Fast-Periodizität"
% "proximality"
% Schur und Brauers Resultat (SB)
% Hindmans Resultat (NH)
% Stabilität:
%   * Im Sinne von Poisson
%   * Im Sinne von Lagrange
%   * Im Sinne von Lyapunov

Dies ist eine übersetzte Zusammenfassung der ersten Kapitel des Buches "`Recurrence in Ergodic Theory and Combinatorial Number Theory"' von Harry Furstenberg.

% Teil I. Wiederkehr in dynamischen Systemen

% Kapitel 1. Wiederkehr und gleichmäßige Wiederkehr in kompakten Räumen
\section{1. (Gleichmäßige) Wiederkehr}

% 1.1. Dynamische Systeme und wiederkehrende Punkte

\begin{defn}
  Ein \emph{dynamisches System} ist ein Paar $(X, G)$ bestehend aus einem kompakten metrischen Raum $X$ und einer Gruppe oder einem Monoid $G$ mit Wirkung
  $\varphi : G \to \AutEnd(X), \, g \mapsto T_g, \, T_g(x) \coloneqq g.x$.
\end{defn}

\begin{defn}
  Ein \emph{Untersystem} eines dynamischen Systems $(X, G)$ ist eine Teilmenge $Z \subseteq X$ mit $T_g(Z) \subseteq Z$ für alle $g \in G$.
\end{defn}

\begin{bem}
  Falls $G \!=\! \Z$ oder $M \!=\! \N$, dann bezeichnen wir mit $T \coloneqq T_1$ den Erzeuger der Aktion und nennen $(X, T)$ ein \emph{zykl. System}.
\end{bem}

% Def 1.1
\begin{defn}
  Sei $X$ ein topologischer Raum, $T : X \to X$ stetig. \\
  Ein Punkt $x \in X$ heißt \emph{wiederkehrend}, falls für für alle Umgebungen $V \subset X$ von $x$ ein $n \geq 1$ existiert mit $T^n(x) \in V$.
\end{defn}

\begin{bem}
  Sei $X$ sogar ein metrischer Raum, $x \in X$ wiederkehrend. \\
  Dann gibt es eine Folge $(n_k)_{k \in \N}$ mit $d(T^{n_k}(x), x) \to 0$ für $k \to \infty$.
\end{bem}

% nicht dort im Buch
\begin{defn}
  Sei $X$ ein topologischer Raum, $T : X \to X$ stetig. Dann heißt
  \[ Q(x) \coloneqq \clos{\Set{T^n x}{n \geq 1}} \subseteq X \]
  \emph{abgeschlossener Vorwärtsorbit} von $x \in X$.
\end{defn}

% nicht explizit im Buch
\begin{lem}
  \begin{itemize}
    \item $x \in X$ ist wiederkehrend $\iff$ $x \in Q(x)$
    \item $x \in Q(y) \implies T(x) \in Q(y) \iff Q(x) \subseteq Q(y)$
    \item Die Relation $xRy \!\!\coloniff\!\! x \in Q(y)$ ist transitiv.
  \end{itemize}
\end{lem}

\begin{thm}[\emph{Birkhoff}]
  Sei $X$ ein kompakter topologischer Raum und $T : X \to X$ stetig.
  Dann gibt es einen wiederkehrenden Punkt $x \in X$.
\end{thm}
% XXX: Bemerkung: Beweis benutzt Auswahlaxiom?

% Defn 1.2
\begin{defn}
  Sei $K$ eine kompakte Gruppe, $a \in K$ und $T(x) \coloneqq ax$. Dann heißt $(K, T)$ ein \emph{Kronecker-System}.
\end{defn}

% Thm 1.2
\begin{thm}
  In einem Kronecker-System sind alle $x \in K$ wiederkehrend.
\end{thm}

% 1.2 Automorphismen und Homomorphismen von Dynamischen Systemen, Faktoren und Erweiterungen

\begin{defn}
  Ein Homomorphismus zwischen zwei dyn. Systemen $(X, G)$ und $(X', G)$ (zweimal die gleiche Gruppe oder Monoid $G$) ist eine $G$-äquivariante stetige Abbildung $\phi : X \to X'$.
\end{defn}

% Defn 1.3
\begin{defn}
  Ein dyn. System $(Y, G)$ ist \emph{Faktor} eines dyn. System $(X, G)$, wenn es einen surjektiven Homomorphismus $(X, G) \to (Y, G)$ gibt. \\
  Man nennt $(X, G)$ dann eine \emph{Erweiterung} von $(Y, G)$.
\end{defn}

% XXX: Auslassen?
\begin{bem}
  Sei $\phi : X \to Y$ surjektiv. Dann kann man $Y$ mit der Menge der Fasern von $\phi$ identifizieren.
\end{bem}

% Thm 1.3
\begin{thm}
  Sei $\phi : (X, T) \to (Y, T)$ ein Morphismus von zyklischen Systemen.
  Wenn $x \in X$ wiederkehrend ist, dann auch $\phi(x)$. \\
  Allgemeiner: $x \in Q(y) \implies \phi(x) \in Q(\phi(y))$
\end{thm}

% Defn 1.4
\begin{defn}
  Sei $(Y, T : Y \to Y)$ ein zyklisches System, $K$ eine kompakte Gruppe und $\psi : Y \to K$ stetig. Setze
  \[
    X \coloneqq Y \times K, \quad
    T : X \to X, \enspace (y, k) \mapsto (Ty, \psi(y)k).
  \]
  Das System $(X, T)$ wird \emph{Gruppenerweiterung} von $(Y, T)$ mit $K$ oder \emph{Schiefprodukt} von $(Y, T)$ mit $K$ genannt.
\end{defn}

\begin{bem}
  Die Gr. $K$ wirkt auf $(X, T) = (Y \!\times\! K, T)$ durch Rechtstransl.:
  \[
    R : K \to \Aut(X), \enspace k \mapsto R_k, \quad
    R_k(y,k') \coloneqq (y,k'k).
  \]
  Die Homöomorphismen $R_k$ kommutieren mit $T$, sind also Automorphismen des dynamischen Systems $(X, T)$.
\end{bem}

% Thm 1.4
\begin{thm}
  Sei $(X \!=\! Y \!\times\! K, T)$ eine Gruppenerw. von $(Y, T)$ und $y_0 \in Y$ wiederkehrend.
  Dann sind die Pkte $\Set{(y_0, k)}{k \in K}$ wiederkehrend.
\end{thm}

% Fußnote 1 Seite 22
\begin{bem}
  Durch Erweiterung mit der zykl. Gr. $\Z_m$ kann man zeigen:
\end{bem}

% Fußnote 1 Seite 22
\begin{prop}
  Ist $x \in X$ in $(X, T)$ wiederkehrend, dann auch in $(X, T^m)$.
\end{prop}

\begin{bsp}
  Sei $T \coloneqq \R / \Z$ und $\alpha \in \R$. Dann ist das System
  \[ (T^2, (\theta, \phi) \mapsto (\theta + \alpha, \phi + 2 \theta + \alpha)) \]
  eine Gruppenerweiterung des Kronecker-Systems $(T, \theta \mapsto \theta + \alpha)$. \\
  Somit sind alle Punkte des Torus $T^2$ wiederkehrend. \\
  Aus der Wiederkehr des Punktes $(0, 0)$ erhält man:
\end{bsp}

% Thm 1.5
\begin{prop}
  Für jedes $\alpha \in \R$ und $\epsilon > 0$ gibt es eine ganzzahlige Lsg der diophantischen Ungleichung
  $\abs{\alpha n^2 - m} < \epsilon$.
\end{prop}

\begin{bem}
  Durch Verallgemeinerung auf den $d$-dim Torus zeigt man:
\end{bem}

% Thm 1.6
\begin{prop}
  Sei $p(X) \in \R[X]$ mit $p(0) = 0$. Dann gibt es für alle $\epsilon > 0$ eine Lsg der diophantischen Ungleichung
  $\abs{p(n) - m} < \epsilon$, $n > 0$.
\end{prop}

% Defn
\begin{defn}
  Sei $M$ ein topol. Raum und $K \subseteq \Iso(M)$ kompakt. \\
  Sei $(Y, T)$ ein zykl. System und $\psi : Y \to K$ stetig. Setze
  \[
    X \coloneqq Y \!\times\! M, \quad
    T : X \to X, \enspace (y, u) \mapsto (Ty, \psi(y)u).
  \]
  Das System $(X, T)$ heißt \emph{isometrische Erweiterung} von $(Y, T)$.
\end{defn}

% Prop 1.7
\begin{prop}
  Sei $(X, T)$ eine isom. Erweiterung von $(Y, T)$. \\
  Dann ist $X = \cup X_\alpha$, wobei $X_\alpha$ abgeschlossene $T$-invariante Teilmengen von $X$ sind, sodass das System $(X_\alpha, T|_{X_\alpha})$ Faktor einer Gruppenerweiterung von $(Y, T)$ ist.
\end{prop}

% Thm 1.8
\begin{prop}
  Sei $(X, T)$ eine isom. Erweiterung von $(Y, T)$ und $y_0 \in Y$ wiederkehrend.
  Dann sind die Pkte $\Set{(y, m)}{m \in M}$ wiederkehrend.
\end{prop}

% 1.3. Wiederkehrende Punkte von Bebutov-Systemen

% Defn 1.6
\begin{defn}
  Sei $G$ eine abz. Gruppe/Monoid und $\Lambda$ ein kompakter metr. Raum.
  Sei $\Omega \coloneqq \Lambda^G \cong \prod \Lambda$ der kompakte metrisierbare Raum der Funktionen von $G$ nach $\Lambda$. Die \emph{reguläre Wirkung} von $G$ auf $\Omega$ ist
  \[
    G \mapsto \AutEnd(\Omega), \enspace g \mapsto T_g, \quad
    T_g(\omega)(g') \coloneqq \omega(g'g).
  \]
  Ein \emph{Bebutov-System} ist ein Untersystem von $(\Omega, G)$.
\end{defn}

\begin{bem}
  Sei $\{ g_1, g_2, \ldots \} = G$ eine Abzählung von $G$. \\
  Dann ist eine Metrik auf $\Omega$ definiert durch
  \[ d(\omega, \omega') \coloneqq \sum 2^{-n} d(\omega(g_n), \omega'(g_n)). \]
\end{bem}

\begin{defn}
  Für $\omega_0 \in \Omega$ ist der Abschluss des Orbits von $\omega_0$,
  \[ X_{\omega_0} \coloneqq \clos{\Set{T_g(\omega_0)}{g \in G}}, \]
  $G$-invariant. Das dynamische System $(X_{\omega_0}, G)$ wird das von $\omega_0$ \emph{erzeugte Bebutov-System} genannt.
\end{defn}

\begin{defn}
  Ein \emph{symbolischer Fluss} ist ein Bebutov-System mit endlichem $\Lambda$ und $G \in \{ \N, \Z \}$.
  Die Elemente von $\Omega$ sind dann unendliche/doppelt-unendliche Folgen von Elementen von $\Lambda$. \\
  Man bezeichnet $\Lambda$ dann als \emph{Alphabet}.
\end{defn}

% Ausgelassen: Theorem 1.9 über die Beziehung von fast-periodischen Funktionen (im Sinne von Bohr) mit der Theorie der Kronecker-Systeme

\begin{defn}
  Ein \emph{Wort} über $\Lambda$ ist eine endl. Sequenz von Elementen aus $\Lambda$. \\
  Die Länge $\abs{w}$ eines Wortes ist die Länge der Sequenz.
\end{defn}

% Prop 1.10
\begin{prop}
  Für eine Sequenz $\omega \in \Lambda^\N$ sind äquivalent:
  \begin{itemize}
    \item $\omega$ ist wiederkehrend.
    \item Jedes Wort in $\omega$ kommt ein 2.\,Mal an einer anderen Pos. in $\omega$ vor.
    \item Jedes Wort aus $\omega$ kommt an unendlich oft in $\omega$ vor.
  \end{itemize}
\end{prop}

\begin{bem}
  Ein wiederkehrendes Wort $\omega \in \Lambda^\N$ hat die allgemeine Form
  \[ \omega = [(aw^{(1)}a)w^{(2)}(aw^{(1)}a)]w^{(3)}[(aw^{(1)}a)w^{(2)}(aw^{(1)}a)]\ldots \]
  mit $a \in \Lambda$ und Wörtern $w^{(1)}, w^{(2)}, \ldots$.
  Damit kann man zeigen:
\end{bem}

\begin{defn}
  Eine \emph{Partition} einer Menge ist eine Darstellung dieser Menge als Vereinigung disjunkter Teilmengen.
\end{defn}

\begin{lem}[\emph{Hilbert}]
  Sei $\N = B_1 \cup \ldots \cup B_q$ eine Partition von $\N$ und $l \in \N_{>0}$ beliebig.
  Schreibe
  \[ P(x_1, \ldots, x_l) \coloneqq \Set{ x_{i_1} + \ldots + x_{i_k} }{0 \leq k \leq l, \, 1 \leq i_1 < \ldots i_k \leq l}. \]
  Dann gibt es $m_1 \leq m_2 \leq \ldots \leq m_l$, sodass unendlich viele Translationen von $P(m_1, \ldots, m_l)$ in demselben $B_j$ enthalten sind.
\end{lem}

\begin{bem}
  Sei $(X, T)$ ein zykl. System und $f : X \to \Lambda$ stetig. Dann ist
  \[
    (X, T) \to (\Lambda^\N, T), \quad
    x \mapsto (f(x), f(Tx), f(T^2x), \ldots)
  \]
  ein Homomorphismus zyklischer Systeme.

\end{bem}

% Thm 1.11
\begin{thm}
  Seien $\Lambda_1$, $\Lambda_2$ komp. Räume und $\phi : \Lambda_1 \to \Lambda_2$ eine Abbildung. \\
  Für $\omega \in \Lambda_1^\N$ definiere $\omega' \in \Lambda_2^\N$ durch $\omega'(n) \coloneqq f(\omega(n))$. \\
  Falls $\omega$ wiederkehrend ist und zusätzlich $f$ in allen Punkten $\omega(n)$ stetig ist, dann ist auch $\omega'$ wiederkehrend.
\end{thm}

% Ausgelassen: Beispiel

% Prop 1.12
\begin{prop}
  Sei $K$ eine komp. Gruppe und $\xi \in K^\N$ wiederkehrend. Dann ist $\eta \!\in\! K^\N$ definiert durch $\eta(n) \coloneqq \xi(n) \xi(n-1) \cdots \xi(1)$ wiederkehrend.
\end{prop}

% Ausgelassen: Beispiel

% 1.4. Gleichmäßige Wiederkehr und minimale Systeme

% Defn 1.7
\begin{defn}
  Eine Teilmenge $S$ einer abelschen topologischen Gruppe / eines Monoids heißt $G$ \emph{syndetisch}, wenn eine kompakte Menge $K \subset G$ existiert, sodass $\fa{g \in G} \ex{k \in K} gk \in S$.
\end{defn}

% TODO: gute deutsche Übersetzung
\begin{bem}
  Eine Teilmenge $\{ s_1 < s_2 < \ldots \} = S \subset \N$ ist genau dann syndetisch, wenn die Größe $s_i - s_{i-1}$ der "`Lücken"' zw. Elementen aus $S$ beschränkt ist.
  Solche Mengen heißen auch \emph{relativ dicht}.
\end{bem}

% Defn 1.8
\begin{defn}
  Sei $(X, G)$ ein dyn. System. Ein Punkt $x \in X$ heißt \emph{gleichmäßig wiederkehrend}, falls für alle Umgebungen $V \subset X$ von $x$ die Menge $\Set{g \in G}{g.x \in V}$ syndetisch ist.
\end{defn}

% Defn 1.9
\begin{defn}
  Ein dyn. System $(X, G)$ heißt \emph{minimal}, wenn es keine echte abgeschl. Teilmenge von $X$ gibt, die inv. unter der $G$-Wirkung ist.
\end{defn}

% Lem 1.13 und 1.14
\begin{lem}
  Sei $(X, G)$ ein dyn System. Es sind äquivalent: \\
  \inlineitem{$(X, G)$ ist minimal} \quad
  \inlineitem{$\fa{x \in X}$ der Orbit $Gx$ ist dicht in $X$} \\
  \inlineitem{$\fa{\emptyset \neq V \subset X \text{ offen}} \ex{\text{endlich viele Elemente } g_1, \ldots, g_n \in G}$}
  \[ g_1^{-1} V \cup \ldots \cup g_n^{-1} V = X. \]
\end{lem}

% Thm 1.15
\begin{thm}
  Sei $(X, G)$ ein minimales dynamisches System. \\
  Dann sind alle $x \in X$ gleichmäßig wiederkehrend.
\end{thm}

\begin{bem}
  Aus Zorns Lemma folgt: Jedes dyn. System besitzt ein minimales Untersystem. Es folgt:
  % XXX: Warum wird im Buch extra erwähnt, dass $X$ kompakt ist? Das ist doch in der Definition eines dyn. Systems enthalten?
\end{bem}

% Thm 1.16
\begin{thm}
  Jedes dyn. System hat einen gleichm. wiederkehrenden Pkt.
\end{thm}

% Thm 1.17, etwas erweitert
\begin{thm}
  Sei $(X, G)$ ein dyn. System, $x \in X$. Dann sind äquivalent: \\
  \inlineitem{$x$ ist glm. wiederkehrend.} \enspace
  \inlineitem{Das Untersystem $\clos{Gx}$ ist minimal.}
\end{thm}

% Thm 1.18
\begin{thm}
  In einem Kronecker-System ist jeder Pkt glm. wiederkehrend.
\end{thm}

% Thm 1.19 und 1.20
\begin{thm}
  Sei $(X, T)$ eine Gruppenerw. oder isometrische Erweiterung von $(Y, T)$ mit Projektion $\pi : (X, T) \to (Y, T)$ und $y_0 \in Y$ glm. wiederkehrend.
  Dann sind die Pkte $\pi^{-1}(y_0)$ glm. wiederkehrend.
\end{thm}

\begin{bem}
  Es folgt durch Betr. eines dyn. Systems auf dem $k$-dim Torus:
\end{bem}

% Thm 1.21
\begin{thm}
  Seien $p_1(X), \ldots, p_k(X) \in \R[X]$ Polynome. \\
  Für alle $\epsilon > 0$ ist die Teilmenge der nat. Zahlen, die
  \[
    \abs{e^{2 \pi i p_1(n) - e^{2 \pi i p_1(0)}}} < \epsilon, \ldots,
    \abs{e^{2 \pi i p_k(n) - e^{2 \pi i p_k(0)}}} < \epsilon
  \]
  gleichzeitig erfüllen, syndetisch.
\end{thm}

% 1.5 "Substitution Minimal Sets and Uniform Recurrence in Bebutov Systems"

% Prop 1.22
\begin{prop}
  Sei $\Lambda$ ein endl. Alphabet. Ein Punkt $\omega \in \Lambda^\N$ oder $\omega \in \Lambda^\Z$ ist genau dann glm. wiederkehrend, wenn für jedes Wort in $\omega$ die Menge der Positionen, an denen dieses Wort auftaucht, syndetisch ist.
\end{prop}

\begin{bem}
  Eine wiederkehrende Sequenz $\omega$ in der allgemeinen Form
  \[ \omega = [(aw^{(1)}a)w^{(2)}(aw^{(1)}a)]w^{(3)}[(aw^{(1)}a)w^{(2)}(aw^{(1)}a)]\ldots \]
  ist glm. wiederkehrend, wenn die Länge der $w^{(n)}$ beschränkt ist. \\
  Es existieren also nichtperiodische, glm. wiederkehrende Sequenzen.
\end{bem}

\begin{defn}
  Ein \emph{Vokabular} ist eine Teilmenge $V$ aller Wörter über einem Alphabet $\Lambda$, für die gilt:
  \begin{itemize}
    \item Jedes Teilwort eines Wortes aus $V$ ist ebenfalls in $V$.
    \item Jedes Wort in $V$ ist Teilwort eines längeren Wortes aus $V$.
  \end{itemize}
\end{defn}

\begin{defn}
  Sei $V$ ein Vokabular. Sei dann $S(V) \subset \Lambda^\N$ die abgeschl., trans- lations-inv. Menge aller Sequenzen, die nur Wörter aus $V$ enthalten.
\end{defn}

\begin{lem}
  Sei $V$ ein Vokabular, das folgende Bedingung erfüllt: Für alle $l \in \N$ gibt es ein $L \in \N$, sodass für alle Wörter $w \in V$ der Länge $\abs{w} = l$ gilt: $w$ ist in jedem Wort $v \in V$ der Länge $\abs{v} = L$ enthalten. \\
  Dann sind alle Sequenzen in $S(V)$ glm. wiederkehrend.
\end{lem}

\begin{bem}
  Sei $\Lambda = \{ a_1, \ldots, a_r \}$ und $w_1, \ldots, w_r$ Wörter über $\Lambda$, die jeweils alle Buchstaben aus $\Lambda$ enthalten. Sei $V_1 \coloneqq \Lambda$ die Menge der Wörter der Länge 1 über $\Lambda$ und induktiv $V_n$ die Menge der Wörter, die aus einem Wort $w \in V_{n-1}$ durch simultane Substitution
  \[ a_1 \to w_1, \ldots, a_r \to w_r \]
  hervorgehen und deren Teilwörter. Das Vokabular $V \coloneqq \cup_{n \in \N} V_n$ heißt dann \emph{substitution minimal set}. Alle Sequenzen in $S(V)$ sind gleichmäßig wiederkehrend.
\end{bem}

\begin{bem}
  Seien $d_1, d_2, \ldots \in \N$ mit $d_n \divides d_{n+1}$ für alle $n$.
  Schreibe nun $\Z$ als disjunkte Vereinigung
  \[
    \Z = \bigsqcup_{k=1}^\infty (d_k \Z + a_k) \quad
    \text{mit $a_1, a_2, \ldots \in \Z$.}
  \]
  Sei $\Lambda$ kompakt und $(\lambda_i)_{i \in \N}$ eine Folge in $\Lambda$. Setze
  $\omega(n) \coloneqq \lambda_k$, falls $n \in d_k \Z + a_k$.
  Wenn nun $[-N, N] \subset \sqcup_{k=1}^l (d_k \Z + a_k)$, dann gilt für alle $n \in [-N, N], q \in \Z$: $\omega(n) = \omega(n + q \cdot d_l)$. Somit tritt jedes Wort in $\omega$ periodisch auf und $\omega$ ist glm. wiederkehrend.
\end{bem}

% 1.6. Kombinatorische Anwendungen

% Defn 1.10
\begin{defn}
  Eine Teilmenge $R \subset \N$ oder $R \subset \Z$ heißt \emph{dick}, wenn sie Intervalle $\cinterval{a_n}{a_n + n}$ beliebiger Länge enthält.
\end{defn}

\begin{bem}
  Eine Menge ist genau dann syndetisch, wenn ihr Schnitt mit jeder dicken Menge nichtleer ist.
  Eine Menge ist genau dann dick, wenn ihr Schnitt mit jeder syndetischen Menge nichtleer ist.
\end{bem}

% Defn 1.11
\begin{defn}
  Eine Teilmenge $A \subset \N$ oder $A \subset \Z$ heißt \emph{stückw. syndetisch}, wenn sie Schnitt einer dicken und einer syndetischen Menge ist.
\end{defn}

% Ausgelassen: Thm 1.23, da nur Spezialfall von 1.24

% Thm 1.24
\begin{thm}
  Sei $B \subseteq \N$ oder $B \subseteq \Z$ stückw. syndetisch, $B = B_1 \cup \nldots \cup B_q$ eine Partition.
  Dann ist auch eine Menge $B_i$ stückweise syndetisch.
\end{thm}

% 1.7. Mehr diophantische Approximation

\begin{bem}
  Seien $\tau_1, \ldots, \tau_n$ die kanonischen Erzeuger von $H_1(T^n) \cong \Z^n$.
\end{bem}

% Lem 1.25
\begin{lem}
  Habe $T : T^d \to T^d$ die Form
  \[
    T(\theta_1, \ldots, \theta_d) \coloneqq
    (\theta_1 + \alpha, \theta_2 + f_1(\theta_1), \ldots, \theta_d + f_{d-1}(\theta_1, \ldots, \theta_{d-1}))
  \]
  mit $\alpha$ irrational, $f_i : T^i \!\to\! T$ stetig mit $(f_i)_*(\tau_i) \!\neq\! 0$ für $i \!=\! 1, \nldots, d{-}1$. \\
  Dann ist $(T^d, T)$ ein minimales dynamisches System.
\end{lem}

% Thm 1.26
\begin{thm}
  Sei $p(X) \in \R[X]$ mit mind. einem irrationalen Koeffizienten. \\
  Dann gibt es $\forall \, \epsilon > 0$ eine Lsg der Ungleichung $\abs{p(n) - m} < \epsilon$.
\end{thm}

% 1.8 Nicht wandernde Transformationen und Wiederkehr

\begin{defn}
  Sei $X$ ein kompakter metrischer Raum und $T : X \to X$ stetig. \\
  Eine Teilmenge $A \subset X$ heißt \emph{wandernd}, wenn die Urbilder $T^{-1}(A)$, \nldots{}, $T^{-n}(A)$, \nldots{} disjunkt von $A$ (und damit auch voneinander) sind.
\end{defn}

\begin{defn}
  Ein dyn. System $(X, T)$ heißt \emph{nicht wandernd}, wenn keine offene, nichtleere Menge $A \subset X$ wandernd ist.
\end{defn}

% Ausgelassen: Hinreichende Bedingung: $X$ ist Support eines inv. endl. Maßes

\begin{defn}
  Eine Teilmenge $A \!\subset\! X$  heißt \emph{nirgends dicht}, falls $\inte(\clos{A}) = \emptyset$.
\end{defn}

\begin{defn}
  Eine Teilmenge $A \!\subset\! X$ heißt \emph{mager}, wenn sie Vereinigung abzählbar vieler nirgends dichter Mengen ist.
\end{defn}

% Thm 1.27
\begin{thm}
  Sei $(X, T)$ nicht wandernd. Dann ist die Menge der nicht wiederkehrenden Punkte in $X$ mager.
\end{thm}

% Ausgelassen: Lem 1.28 über halbstetige Funktionen


% Kapitel 2. Van der Waerden's Theorem
\begin{samepage}
  \section{2. Van der Waerdens Theorem}
\end{samepage}

% Ausgelassen: Gegenbeispiele: Was kann schief laufen, wenn die Abbildungen nicht kommutieren?

% 2.1. Bowen's Lemma und homogene Mengen

% Lem 2.1
\begin{lem} %[Bowen]
  Sei $X$ ein komp. metr. Raum, $T \!:\! X \!\to\! X$ stetig und $A \!\subset\! X$. \\
  Angenommen, $\fa{\epsilon > 0} \fa{x \in A} \ex{y \in Y} \ex{n \geq 1} d(T^n y, x) < \epsilon$. \\
  Dann $\fa{\epsilon > 0} \ex{z \in A} \ex{n \geq 1} d(T^n z, z) < \epsilon$.
  %Sei $A \!\subset\! X$, sodass für alle $x \in A$ und $\epsilon > 0$ ein $y \in A$ und $n \geq 1$ existiert mit $d(T^n y, x) < \epsilon$.
  %Dann gibt es für alle $\epsilon > 0$ ein $z \in A$ und $n \geq 1$ mit $d(z, T^n z) < \epsilon$.
\end{lem}

% Defn 2.1
\begin{defn}
  Sei $X$ ein komp. Raum, $T : X \to X$ stetig. \\
  Eine abgeschlossene Teilmenge $A \subseteq X$ heißt \emph{homogen} bzgl. $T$, wenn es eine Untergruppe $G \!<\! \Aut(X)$ gibt mit $\fa{S \in G} ST \!=\! TS$ und $S(A) \!=\! A$, sodass das dyn. System $(A, G)$ minimal ist.
\end{defn}

\begin{bsp}
  Sei $(X, T) = (Y \!\times\! K, T)$ eine Gruppenerweiterung von $(Y, T)$. \\
  Dann sind die Fasern $\Set{(y_0, k)}{k \in K}$ für alle $y_0 \in Y$ homogen.
\end{bsp}

% Defn 2.2
\begin{defn}
  Eine abgeschl. Teilmenge $A \subset X$ eines komp. metr. Raumes heißt \emph{wiederkehrend} für eine Transformation $T : X \to X$, falls für alle $\epsilon \!>\! 0$ und $x \!\in\! A$ ein $y \!\in\! A$ und $n \!\geq\! 1$ existiert, sodass $d(T^n y, x) \!<\! \epsilon$.
\end{defn}

% Lem 2.2
\begin{lem}
  Sei $A \subset X$ homogen bzgl. $T : X \to X$.
  Angenommen, $\fa{\epsilon \!>\! 0} \ex{x, y \!\in\! A, \, n \!\geq\! 1} d(T^n y, x) \!<\! \epsilon$. 
  Dann ist $A$ wiederkehrend.
\end{lem}

\begin{bsp}
  Sei $(X, T) \!=\! (Y \!\times\! K, T)$ eine Gruppenerw. von $(Y, T)$ und $y_0 \in Y$ wiederkehrend.
  Dann ist die Faser $\{ y_0 \} \times K$ wiederkehrend.
\end{bsp}

% Lem 2.3
\begin{lem}
  Sei $A \subset X$ homogen und wiederkehrend bzgl. $T : X \to X$. \\
  Dann enthält $A$ einen wiederkehrenden Punkt von $(X, T)$.
\end{lem}

% Ausgelassen: Lem 2.3 angewandt auf Gruppenerweiterungen
% Ausgelassen: Prop 2.4 da nur Kombination der Lemmata 2.2 und 2.3

% 2.2 "The Multiple Birkhoff Recurrence Theorem"

% Ausgelassen: Prop 2.5, da nur Spezialfall und Zwischenschritt zum Beweis von 2.6

% Thm 2.6
\begin{thm}[Multiple Birkhoff Recurrence, \emph{MBR}]\mbox{}\\
  Sei $X$ ein komp. metr. Raum und $T_1, \ldots, T_l : X \to X$ komm. stetige Abbildungen.
  Dann gibt es einen Punkt $x \!\in\! X$ und eine Folge $(n_k)_{k \in \N}$ mit $n_k \to \infty$ und $T_i^{n_k} x \to x$ simultan für alle $i = 1, \ldots, l$ bei $k \to \infty$.
\end{thm}

% 2.3. Das mehrdimensionale van-der-Waerden-Theorem

\begin{defn}
  Eine \emph{arithmetische Sequenz} der Länge $N \in \N$ ist eine Teilmenge $A \subset \N$ der Form $A = \Set{a + ib}{i = 1, \nldots, N}$ mit $a \!\in\! \N$, $b \!\geq\! 1$.
\end{defn}

% Thm 2.7
\begin{thm}[\emph{Grünwald}]
  Sei $\N^m = B_1 \cup \ldots \cup B_q$ eine Partition von $\N^m$. \\
  Dann hat eine Menge $B_j$ die Eigenschaft, dass für alle endl. Teil- mengen $F \subset \N^m$ ein $a \in \N^m$ und ein $b \in \N$ existiert mit $bF + a \subset B_j$.
\end{thm}

\begin{bem}
  Dieses Thm ist eine Verallgemeinerung auf $>\!1$ Dim. von:
\end{bem}

\begin{thm}[\emph{van der Waerden}]
  Sei $\N = B_1 \cup \ldots \cup B_q$ eine Partition. \\
  Dann enthält ein $B_j$ arithmetische Sequenzen beliebiger Länge.
\end{thm}

% Ausgelassen: 2.4 Umformulierungen und Anwendungen: Diophantinische Approximation
% Ausgelassen: 2.5 Verfeinerung: IP-Mengen


% Teil II. Wiederkehr in maßerhaltenden Systemen

% Kapitel 3. Invariante Maße auf auf kompakten Räumen

% 3.1 Maßerhaltende Systeme und Poincaré-Wiederkehr
\section{3.1. Dynamische Systeme auf Maßräumen}

\begin{bem}
  Sei im Folgenden $(X, \Bor, \mu)$ ein Maßraum mit einem topol. Raum $X$, einer $\sigma$-Algebra $\Bor$ auf $X$ und einem endlichen normierten Maß $\mu : \Bor \to \cinterval{0}{1}$ mit $\mu(X) = 1$.
\end{bem}

\begin{nota}
  $L^p(X, \Bor, \mu) \coloneqq \{ \text{Äq'klassen $p$-integrierbarer Fktn $X \!\to\! \R$} \}$ heißt \emph{Lebesgue-Raum} ($1 \leq p < \infty$). Für $p = \infty$ setze
  $L^\infty(X, \Bor, \mu) \coloneqq \{ \text{Äq'klassen wes. beschränkter Fktn $X \!\to\! \R$} \}$. \\
  Es werden dabei Fktn identifiziert, die fast-überall übereinstimmen.
\end{nota}

\begin{bem}
  $L^p(X, \Bor, \mu)$ ist ein Banachraum für $1 \leq p \leq \infty$.
\end{bem}

\begin{nota}
  $E(f) \coloneqq \Int{X}{}{f}{\mu}$ für $f \in L^1(X, \Bor, \mu)$.
\end{nota}

\begin{defn}
  Eine messbare Abbildung $T : X \to X$ heißt \emph{maßerhaltend}, wenn $\fa{B \in \Bor} \mu(B) = \mu(T^{-1} B)$.
  Das Tupel \meST{} heißt \emph{maßerhaltendes System} (\meS{}).
\end{defn}

\begin{lem}
  $T$ ist maßerhaltend $\!\iff\!$ $\fa{f \!\in\! L^1(X, \Bor, \mu)\!}\! E(f) \!=\! E(f \!\circ\! T)$
\end{lem}

% Defn 3.1
\begin{defn}
  Ein maßerh. System \meST{} heißt \emph{ergodisch}, wenn
  \[ \fa{B \in \Bor} T^{-1} B = B \implies \mu(B) \in \{ 0, \mu(X) \}. \]
  Man sagt auch, $T$ sei eine ergodische Transformation.
\end{defn}

% Thm 3.1
\begin{thm}[\emph{Birkhoffscher Ergodensatz}]\mbox{}\\
  Sei \meST{} ein \meS{} und $f \in L^1(X, \Bor, \mu)$. Dann gilt
  \[
    f_n(x) \coloneqq \tfrac{1}{n+1} \cdot \left( f(x) + f(Tx) + \ldots + f(T^n x) \right) \xrightarrow{n \to \infty} \overline{f}(x)
  \]
  für fast-alle $x \in X$, wobei $\overline{f}$ $T$-invariant ist, \dh{} $\overline{f} = \overline{f} \circ T$. \\
  Falls $f \!\in\! L^p(X, \Bor, \mu)$ mit $1 \leq p < \infty$, dann $f_n \to \overline{f}$ in $L^p(X, \Bor, \mu)$.
\end{thm}

\begin{bem}
  Es folgt $E(f) = E(\overline{f})$. Wenn $T$ ergodisch ist, dann ist $\overline{f}$ fast-überall konstant $E(f)$, also $f_n(x) \to E(f)$ für fast-alle $x \in X$.
\end{bem}

% Thm 3.2
\begin{thm}[\emph{Poincaré}]
  Sei \meST{} ein \meS{} und $A \in \Bor$ mit $\mu(A) > 0$.
  Dann gibt es ein $n \geq 1$ mit $\mu(A \cap T^{-n} A) > 0$.
\end{thm}

\begin{beweis}
  Sei $f = \ind_A$ die Indikatorfunktion von $A$. \\
  Angenommen, die Behauptung ist falsch. Dann gilt
  \[
    \Int{}{}{f(T^n x)f(T^m x)}{\mu(x)} = 0 \quad
    \text{für $n \neq m$.}
  \]
  Es folgt durch $2 \times$ Übergang zum Grenzwert $E(\overline{f}^2) = 0$, also
  \[ 0 = E(\overline{f}) = E(f) = \mu(A) > 0 \enspace \lightning \]
\end{beweis}

% Thm 3.3
\begin{thm}
  Sei \meST{} ein \meS{}, wobei $X$ ein separabler metrischer Raum ist, dessen offenen Mengen messbar sind. \\
  Dann sind fast-alle Punkte in $X$ wiederkehrend.
\end{thm}

% Nicht im Buch, aber für Frühlingsschule wichtig
\section{Etwas Maßtheorie \& Funktionalanalysis}

\begin{defn}
  Sei $k \in \{ \R, \C \}$.
  Ein \emph{topologischer $k$-Vektorraum} $V$ ist ein $k$-VR mit einer Topologie bzgl. der die Addition $+ : V \times V \to V$ und die Skalarmultiplikation $\cdot : k \times V \to V$ stetig sind.
\end{defn}

\begin{defn}
  Sei $k \in \{ \R, \C \}$ und $V$ ein $k$-VR.
  Eine Teilmenge $A \subset V$ heißt
  \begin{itemize}
    \item \emph{absorbierend}, wenn
    $\fa{v \!\in\! V\!}\! \ex{r \!>\! 0\!}\! \fa{\alpha \!\in\! k\!}\! \abs{\alpha} \!<\! r \!\implies\! \alpha v \!\in\! A$.
    \item \emph{ausgewogen}, wenn $\fa{v \in A} \fa{\alpha \in k} \abs{\alpha} \leq 1 \implies \alpha v \in A$.
    \item \emph{absolutkonvex}, wenn $A$ ausgewogen und konvex ist.
  \end{itemize}
\end{defn}

\begin{lem}
  Eine Teilmenge $A \!\subset\! V$ ist genau dann absolutkonvex, wenn
  \[ \fa{v, w \in A} \fa{\lambda, \mu \in k} \abs{\lambda} + \abs{\mu} \leq 1 \implies \lambda v + \mu w \in A. \]
\end{lem}

\begin{defn}
  Ein topol. $k$-VR $V$ heißt \emph{lokalkonvex}, wenn jede Umgebung von $0 \!\in\! V$ eine offene, absorbierende, absolutkonv. Teilmenge besitzt.
\end{defn}

\begin{bsp}
  Normierte Räume sind lokalkonvex.
\end{bsp}

\begin{defn}
  Sei $V$ ein $k$-VR, $K \subset V$ konvex. \\
  Ein Punkt $x \in K$ heißt \emph{Extremalpunkt}, wenn
  \[ \fa{\lambda, \mu \in \overline{B_1(0)} \!\subset\! k, \, y, z \in V} \lambda \!+\! \mu \!=\! 1 \wedge \lambda y \!+\! \mu z \!=\! x \implies x \!=\! y \!=\! z. \]
  \iffalse
  \begin{align*}
    & \fa{y_i, \ldots, y_k \in K, \, \lambda_1, \ldots, \lambda_k \in \cinterval{0}{1}} \\
    & \lambda_1 + \nldots + \lambda_k = 1 \,\wedge\, \lambda_1 y_1 + \nldots + \lambda_k y_k = x \implies y_1 \!=\! \nldots \!=\! y_k = x.
  \end{align*}
  \fi
\end{defn}

\begin{lem}
  Sei $V$ ein topol. VR, $A \subset V$. Ist $A$ konvex, dann auch $\clos{A}$.
  %Der Abschluss einer konvexen Teilmenge eines topol. VR ist ebenfalls konvex.
\end{lem}

\begin{satz}[\emph{Krein-Milman}]
  Sei $V$ ein lokalkonvexer Raum, $K \subset V$ kompakt und konvex. Dann ist $K$ gleich dem Abschluss der von den Extremalpunkten von $K$ aufgespannten konvexen Hülle.
\end{satz}

\begin{defn}
  Sei $X$ ein topologischer Raum, $\Bor$ eine $\sigma$-Algebra auf $X$. \\
  Ein Maß $\mu$ auf $(X, \Bor)$ heißt
  \begin{itemize}
    \item \emph{von innen regulär}, wenn für alle $A \in \Bor$ gilt:
    \[ \mu(A) = \sup \Set{\mu(F)}{F \subseteq X \text{ kompakt und messbar, } F \subseteq A}. \]
    \item \emph{von außen regulär}, wenn für alle $A \in \Bor$ gilt:
    \[ \mu(A) = \inf \Set{\mu(F)}{F \subseteq X \text{ offen und messbar, } F \supseteq A}. \]
    \item \emph{regulär}, wenn es von innen und außen regulär ist.
  \end{itemize}
\end{defn}

\begin{defn}
  Ein topol. Raum $X$ heißt \emph{$\sigma$-kompakt}, wenn $X$ Vereinigung höchstens abzählbar vieler kompakter Teilmengen ist.
\end{defn}

% TODO: Muss man hier noch Stetigkeit der Funktionale voraussetzen? Oder wird das schon irgendwie durch die Monotonie impliziert?
% Thm 2.14 in "Complex and Real Analyisis" von Walter Rudin
\begin{thm}[\emph{Riesz}]
  Sei $X$ ein lokalkompakter Hausdorffraum, dessen offene Teilmengen $\sigma$-kompakt sind, $\Bor$ die Borel-$\sigma$-Algebra von $X$,
  \begin{align*}
    \Cont_0(X) \coloneqq \{ \, & \text{stetige Funktionen $X \to \R$ mit kompaktem Träger} \, \}, \\
    \Lin \coloneqq \{ \, & \text{lineare Funktionale $L : \Cont_0(X) \to \R$} \, \}, \\
    \Lin' \coloneqq \{ \, & L \in \Lin \enspace|\enspace \fa{f \in C_0(X)} f \geq 0 \implies L(f) \geq 0 \, \} \\
    \Meas' \coloneqq \{ \, & \text{reguläre Maße $\mu : \Bor \to \cinterval{0}{\infty}$ auf $(X, \Bor)$ mit} \\
    & \text{$\mu(K) < \infty$ für alle kompakten $K \subset X$} \, \}
  \end{align*}
  Dann ist folgende Abbildung eine Bijektion:
  \[ R : \Meas' \to \Lin', \quad \mu \mapsto \left( L_\mu : C_0(X) \to \R, \enspace f \mapsto \Int{X}{}{f}{\mu} \right) \]
\end{thm}

% Thm 2.18 in "Complex and Real Analyisis" von Walter Rudin
\begin{thm}
  Sei $X$ ein lokalkompakter Hausdorffraum, dessen offene Teilmengen $\sigma$-kompakt sind.
  Sei $\mu$ ein Maß auf der Borel-$\sigma$-Algebra von $X$ mit $\mu(K) \!<\! \infty$ für alle kompakten $K \!\subset\! X$. Dann ist $\mu$ regulär.
\end{thm}

% Aus der Stochi-Zusammenfassung, abgeändert
\begin{defn}
  Seien $\mu$ und $\nu$ Maße auf dem messbaren Raum $(\Omega, \Alg)$. \\
  Dann heißt $\nu$ heißt \emph{absolut stetig} bezüglich $\mu$ (notiert $\nu \ll \mu$), falls
  \[ \mu(A) = 0 \implies \nu(A) = 0 \quad \text{ für alle } A \in \Alg. \]
\end{defn}

\begin{defn}
  Ein Maß $\mu$ auf einem messb. Raum $(\Omega, \Alg)$ heißt \emph{$\sigma$-endlich}, wenn es eine Folge $(A_n)$ in $\Alg$ mit $\mu(A_n) < \infty$ und $\Omega = \cup_{n \in \N} A_n$ gibt.
\end{defn}

\begin{thm}[\emph{Radon-Nikodým}]
  Seien $\mu$ und $\nu$ Maße auf dem messbaren Raum $(\Omega, \Alg)$, $\mu$ $\sigma$-endlich und $\nu$ absolut stetig bzgl. $\mu$ ($\nu \ll \mu$).
  Dann gibt es eine fast-überall eindeutige messb. Funktion $f : \Omega \to \R_{\geq 0}$ mit
  \[
    \nu(A) = \Int{\Omega}{}{f}{\mu} \quad
    \text{für alle $A \in \Alg$}.
  \]
  Ist $\nu$ endlich, so ist $f \in L^1(\Omega, \Alg, \mu)$.
\end{thm}

\begin{defn}
  Sei $X$ ein topol. Raum, $\Bor$ die Borel-$\sigma$-Algebra.
  Ein Maß $\mu$ auf $(X, \Bor)$ heißt \emph{Borelmaß}, wenn für alle $x \in X$ eine offene Umgebung $U$ von $x$ existiert mit $\mu(U) < \infty$.
\end{defn}

\begin{bem}
  Falls $X$ lokalkompakt ist, dann ist $\mu$ genau dann ein Borelmaß, wenn $\mu(K) < \infty$ für alle Kompakta $K \subseteq X$.
\end{bem}

% Von http://de.wikipedia.org/wiki/Haarsches_Ma%C3%9F
% XXX: Wird hier noch Hausdorff benötigt?
\begin{defn}
  Ein \emph{Haar-Maß} auf einer topol. Gr. $G$ ist ein reguläres Borel- maß auf $(G, \Bor)$, das linksinv. ist, d.h. $\fa{g \!\in\! G, \, B \!\in\! \Bor\!}\! \mu(B) \!=\! \mu(gB)$.
\end{defn}

\begin{thm}
  Sei $G$ eine lokalkompakte topol. Gruppe.
  Dann gibt es bis auf multiplikative Konstante genau ein nichttriviales Haar-Maß.
\end{thm}

% 3.2 Invariante Maße auf kompakten Räumen
\begin{samepage}
  \section{3.2 Invariante Maße auf komp. Räumen}
\end{samepage}

\begin{bem}
  Sei im Folgenden $X$ ein kompakter metrischer Raum und $\Bor$ die Borel-$\sigma$-Algebra auf $X$ und $T : X \to X$ stetig.
\end{bem}

% Aus Ana3-Zusammenfassung
\begin{defn}
  Sei $(\Omega, \Alg, \mu)$ ein Maßraum und $(\Omega', \Alg')$ ein messbarer Raum und $f : \Omega \to \Omega'$ eine messbare Abbildung, dann ist
  \[ \mu' = f_*(\mu) = \mu \circ f^{-1} : \Alg' \to \cinterval{0}{\infty}, \quad A' \mapsto \mu(f^{-1}(A')) \]
  ein Maß auf $(\Omega', \Alg')$, genannt das \emph{Bildmaß} von $f$.
\end{defn}

\begin{defn}
  Setze $\Meas \coloneqq \{ \, \text{Maße $\mu$ auf $(X, \Bor)$ mit $\mu(X) = 1$} \, \}$.
  Sei $\mathcal{S} \subset \Meas$ die Menge der Maße $\mu$, die unter $T$ \emph{invariant} sind, \dh{} $T_* \mu = \mu$.
\end{defn}

\begin{kor}
  Die Abb. $R$ aus dem Satz von Riesz liefert eine Bij.
  \[ \Meas \xleftrightarrow{1:1} \Set{L \in \Lin'}{L(x \mapsto 1) = 1} \]
\end{kor}

\begin{bem}
  Man kann $\Meas$ also als Teilmenge von $\mathcal{L}$ auffassen.
  Wenn $\Lin'$ die Schwach-*-Topologie trägt, dann trägt $\Meas$ die Teilmengentopol. mit
  \[
    \mu_n \xrightarrow{n \to \infty} \mu \text{ in $\Meas$} \coloniff
    \fa{f \in C_0(X)} \Int{}{}{f}{\mu_n} \xrightarrow{n \to \infty} \Int{}{}{f}{\mu}.
  \]
  Mit dieser Topologie ist $\Meas \subset \Lin'$ kompakt und konvex.
\end{bem}

\begin{bem}
  Die Menge $\mathcal{S}$ ist nicht leer: Sei $\nu \in \Meas$ beliebig. Setze
  \[ \mu_n \coloneqq \tfrac{1}{n+1} \left( \nu + T_* \nu + T^2_* \nu + \ldots + T^n_* \nu \right). \]
  Der Grenzwert von $(\mu_n)_{n \in \N}$ in $\Meas$ ist dann in $\mathcal{S}$.
\end{bem}

\begin{bem}
  $\mathcal{S} \subset \Lin$ ist kompakt und konvex. Aus dem Satz von Krein- Milman folgt: $\mathcal{S}$ ist der Abschluss der konvexen Hülle seiner Extre- malpkte (die insb. existieren). Diese sind wie folgt charakterisiert:
\end{bem}

\begin{prop}
  Ein Maß $\mu \in \mathcal{S}$ ist genau dann ein Extremalpunkt von $\mathcal{S}$, wenn \meST{} ein ergodisches System ist.
\end{prop}

\begin{bem}
  Folglich heißen Extremalpunkte von $\mathcal{S}$ \emph{ergodische Maße}.
\end{bem}

% Defn 3.2
\begin{defn}
  Das dynamische System $(X, T)$ heißt \emph{eindeutig ergodisch}, wenn es genau ein invariantes Maß $\mu \in \mathcal{S}$ gibt.
\end{defn}

\begin{bem}
  Aus der Prop. folgt, dass dieses eind. Maß dann ergodisch ist.
\end{bem}

% Thm 3.5
\begin{thm}
  Sei $(X, T)$ ein eind. ergodisches System, $\mu \!\in\! \mathcal{S}$ das $T$-inv. Maß. \\
  Dann gilt für alle $f \in \Cont(X)$: $f_n \xrightarrow{n \to \infty} \Int{}{}{f}{\mu}$ glm. in $X$.
\end{thm}

% Defn 3.3
\begin{defn}
  Sei $X$ ein separabler metrischer Raum, $\Bor$ die Borel-$\sigma$-Algebra. \\
  Der \emph{Träger} eines Maßes auf $(X, \Bor)$ ist das Komplement aller offenen Mengen mit Maß Null.
\end{defn}

\begin{bem}
  Der Träger eines $T$-invarianten Maßes ist $T$-invariant.
\end{bem}

% Prop 3.6
\begin{prop}
  Sei $T$ eindeutig ergodisch auf $X$ ist und $X' \subset X$ der Träger des inv. Maßes. Dann ist $(X', T)$ minimal.
\end{prop}

% Defn 3.4
\begin{defn}
  $x_0 \!\in\! X$ heißt \emph{generischer Pkt} des \meS{} \meST{}, wenn
  \[ \fa{f \in \Cont(X)} f_n(x_0) \xrightarrow{n \to \infty} \Int{X}{}{f}{\mu}. \]
\end{defn}

% Prop 3.7
\begin{prop}
  Sei $\mu$ ein ergodisches Maß auf $(X, T)$. Dann sind $\mu$-fast-alle Punkte von $X$ generische Punkte von $\mu$.
\end{prop}

% Prop 3.8
\begin{prop}
  Das System $(X, T)$ ist eindeutig ergodisch, wenn jeder Punkt $x \in X$ generisch für ein inv. Maß $\mu_x \in \mathcal{S}$ ist.
\end{prop}

% Ausgelassen: Sei $(X, T)$ eindeutig ergodisch, $X'$ der Träger des eindeutig ergodischen Maßes, dann ist jeder Punkt in $X'$ glm. wiederkehrend (Beweis mit Techniken dieses Kapitels)

% Defn 3.5
\begin{defn}
  Sei $\mu$ ein invariantes Maß von $(X, T)$. Ein Punkt $x_0 \in X$ heißt \emph{quasi-generischer Punkt} von $\mu$, falls Folgen $(a_k)_{k \in \N}$ und $(b_k)_{k \in \N}$ natürlicher Zahlen existieren mit $a_k \leq b_k$, \enspace $b_k - a_k \xrightarrow{k \to \infty} \infty$ und
  \[ \fa{f \in \Cont(X)} \frac{1}{b_k - a_k + 1} \sum_{n=a_k}^{b_k} f(T^n x_0) \xrightarrow{k \to \infty} \Int{X}{}{f}{\mu}. \]
\end{defn}

% Prop 3.9
\begin{prop}
  Sei $(X, T)$ ein zykl. System, $x_0 \!\in\! X$ und $X' \coloneqq \clos{\Set{T^n x_0}{n \geq 0}}$. Wenn $\mu \in \mathcal{M}(X')$ ein ergodisches Maß bzgl. $T_{X'}$ ist, dann ist $x_0$ quasi-generisch für $\mu$.
\end{prop}

% 3.3. Gruppenerweiterungen, eindeutige Ergodizität und Gleichverteilung
\begin{samepage}
  \section{3.3. Gruppenerweiterungen, eindeutige Ergodizität und Gleichverteilung}
\end{samepage}

\begin{defn}
  Sei $(Y, T)$ ein dyn. System, $\nu$ ein $T$-inv. Maß auf $(Y, \Bor)$, $G$ eine kompakte Gruppe und $\phi : Y \!\to\! G$ stetig.
  Sei $\mu_G$ ein Haar-Maß auf $G$. \\
  Dann ist das Produktmaß $\nu \!\times\! \mu_G$ ein inv. Maß auf der Gruppener- weiterung $(Y \!\times\! G, T)$, es ist also $(Y \!\times\! G, \Bor_{Y \times G}, \nu \!\times\! \mu_G, T)$ ein \meS{}.
\end{defn}

% Prop 3.10
\begin{prop}
  Sei $(Y, T)$ eind. ergodisch und $(Y \!\times\! G, \Bor_{Y \times G}, \nu \!\times\! \mu_G, T)$ ergodisch. Dann ist auch $(Y \!\times\! G, T)$ eindeutig ergodisch.
\end{prop}

% XXX: Ein bisschen was zu Fourier-Analysis

% Prop 3.11 und 3.12 kombiniert
\begin{prop}
  Sei $\alpha \in \R$ irrational und $(b_{ij})_{1 \leq j < i \leq d}$ ganze Zahlen mit $b_{j+1}{j} \neq 0$ für $d > j \geq 1$ und $T : T^d \to T^d$ definiert durch
  \[
    T(\theta_1, \nldots, \theta_d) \coloneqq (\theta_1 + \alpha, \theta_2 + b_{21} \theta_1, \nldots, \theta_d + b_{d1} \theta_1 + \nldots + b_{d,d-1} \theta_{d-1}).
  \]
  Dann ist $(T^d, T)$ eind. ergodisch bzgl. dem Haar-Maß auf $T^d$.
\end{prop}

% Aus http://de.wikipedia.org/wiki/Gleichverteilung_modulo_1
\begin{defn}
  Sei $(x_n)_{n \in \N}$ eine Folge reeller Zahlen, $p : \R \to S^1, \, z \mapsto e^{2i\pi z}$ und $\mu$ das Haar-Maß auf $S^1$.
  Die Folge heißt \emph{gleichverteilt modulo 1}, falls für alle offenen $U \subset S^1$ gilt:
  \[ \frac{\abs{U \cap \Set{p(x_i)}{i=1,\nldots,n}}}{\abs{n}} \xrightarrow{n \to \infty} \frac{\mu(U)}{\mu(S^1)}. \]
\end{defn}

\begin{lem}
  Die Folge $(x_n)$ ist genau dann gleichverteilt, wenn
  \[
    \tfrac{1}{n} \sum_{i=1}^n f(p(x_i)) \xrightarrow{n \to \infty} \tfrac{1}{\mu(S^1)} \Int{S^1}{}{f}{\mu} \qquad
    \text{$\forall$ stetige $f : S^1 \to \R$.}
  \]
\end{lem}

% XXX: bessere Übersetzung
\begin{defn}
  Die Folge $(x_n)$ heißt \emph{wohlverteilt modulo 1}, wenn
  \[ \frac{\abs{U \cap \Set{p(x_i)}{i=1+M,\nldots,n+M}}}{\abs{n}} \xrightarrow{n \to \infty} \frac{\mu(U)}{\mu(S^1)}. \]
  gleichmäßig für alle $M \in \N$ gilt.
\end{defn}

% Thm 3.13, verschärft von gleichverteilt zu wohlverteilt (siehe Bemerkungen nach dem Theorem)
\begin{thm}[\emph{Weyl}]
  Sei $p(X) \in \R[X]$ ein Polynom mit mind. einem irrationalen Koeffizienten, der nicht der konstante Term ist. \\
  Dann ist die Folge $(p(n))_{n \in \N}$ wohlverteilt modulo 1.
\end{thm}

\begin{samepage}
  \section{3.4. Unitäre Op'en und Poincaré-Folgen}
\end{samepage}

\begin{defn}
  Ein $\C$-VR mit Skalarprodukt $\scp{\blank}{\blank}$ heißt Prähilbertraum. \\
  Er heißt \emph{Hilbertraum}, wenn er vollständig bzgl. der ind. Norm ist.
\end{defn}

\begin{defn}
  Ein \emph{linearer beschränkter Operator} auf einem Hilbertraum $H$ ist eine lineare Abbildung $T : H \to H$ mit
  \[ \sup_{\norm{x} \leq 1} \norm{Tx} < \infty \]
\end{defn}

\begin{nota}
  $\Boun(H) \coloneqq \{ \, \text{lineare, beschränkte Operatoren auf $H$} \, \}$
\end{nota}

\begin{thm}
  Für jeden Operator $T \!\in\! \Boun(H)$ gibt es genau ein $T^* \!\in\! \Boun(H)$ mit
  \[ \fa{x, y \in H} \scp{Tx}{y} = \scp{x}{T^* y}. \]
\end{thm}

\begin{defn}
  Der Operator $T^*$ heißt \emph{adjungierter Operator} von $T$.
\end{defn}

\begin{defn}
  Sei $H$ ein Hilbertraum.
  Ein Operator $T \in \Boun(H)$ heißt \\
  \inlineitem{\emph{normal}, wenn $T^* T \!=\! T T^*$,} \enspace
  \inlineitem{\emph{unitär}, wenn $T^* T \!=\! T T^* \!=\! \Id_H$.}
\end{defn}

% TODO: Spektrum, Maß auf Spektrum, Spektralsatz

% Lem 3.14
\begin{lem}
  Sei $H$ ein Hilbertraum, $T \in \Boun(H)$ ein unitärer Operator, $p(X) \in \Q[X]$ ein Polynom mit $\fa{n \in \Z} p(n) \in \Z$. Setze
  \begin{align*}
    H_\theta & \coloneqq \Set{x \in H}{Tx = e^{2 \pi i \theta}} \quad
    \text{für $\theta \in \cointerval{0}{1}$,} \\
    H_{\text{rat}} & \coloneqq \nspace{3} \bigoplus_{q \in \Q \cap \cointerval{0}{1}} \nspace{3} H_q.
  \end{align*}
  Sei $P_{\text{rat}} : H \to H_{\text{rat}}$ die orth. Projektion. Wenn $P_{\text{rat}} x = 0$, dann
  \[ \tfrac{1}{N + 1} \sum_{n=0}^N T^{p(n)} x \xrightarrow{N \to \infty} 0. \]
\end{lem}

% Prop 3.15
\begin{prop}
  Sei $p(X) \in \Q[X]$ mit $\fa{n \in \Z} p(n) \in \Z$ und $p(0) = 0$.
  Sei $T \in \Boun(H)$ ein unitärer Operator auf einem Hilbertraum $H$ und $x \in H$ mit $\scp{T^{p(n)}}{x} = 0$ für alle $n \geq 1$. Dann ist $x$ orthogonal zu allen Eigenvektoren deren zugeh. Eigenwert eine Einheitswurzel ist.
\end{prop}

% XXX: "Poincaré sequence = Poincaré-Folge" ?
% Defn 3.6
\begin{defn}
  Eine Teilmenge $W \subset \Z$ heißt \emph{Poincaré-Folge}, wenn für alle maßerhaltenden Systeme \meST{} und $A \in \Bor$ mit $\mu(A) > 0$ gilt:
  \[ \ex{n \in W \setminus \{ 0 \}} \mu(T^{-n} A \cap A) > 0. \]
\end{defn}

\begin{bsp}
  Dicke Mengen sind Poincaré-Folgen.
\end{bsp}

% Thm 3.16
\begin{thm}
  Sei $p(X) \in \Q[X]$ mit $\fa{n \in \Z} p(n) \in \Z$ und $p(0) = 0$. \\
  Dann ist $\Set{p(n)}{n \geq 1}$ eine Poincaré-Folge.
\end{thm}

\begin{bem}
  Ist $W \!\subset\! \Z$ eine Poincaré-Folge, dann ist $(W \cap m \Z) \setminus \{ 0 \} \neq \emptyset$ für alle $m \geq 1$.
  Denn: Wenn \meST{} ein maßerhaltendes System ist, dann auch $(X, \Bor, \mu, T^m)$.
  % XXX: Wenn $p(X) \in \Q[X]$ mit $p(\Z) \subset \Z$, dann ist $p(\Z)$ nur dann eine Poincaré-Folge, wenn $0 \in p(\Z)$.
\end{bem}

\begin{samepage}
  \section{3.5. Invariante Maße auf symb. Flüssen}
\end{samepage}

\begin{bem}
  Sei im Folgenden $\Lambda$ ein endliches Alphabet und $T : \Lambda^\Z \to \Lambda^\Z$ der Verschiebeoperator.
\end{bem}

% Defn 3.7
\begin{defn}
  Sei $S \subset \N$ bzw. $S \subset \Z$. Die \emph{obere Banach-Dichte} von $S$ ist
  \[ BD^*(S) \coloneqq \limsup_{n \to \infty} \enspace \tfrac{1}{n+1} \cdot \max_{m \in \Z} \, \abs{S \cap \cinterval{m}{m+n}}. \]
\end{defn}

% Ausgelassen: untere Banach-Dichte; obere, untere Dichte

\begin{defn}
  Man sagt, ein Symbol $a \in \Lambda$ trete in $\xi \!\in\! \Lambda^\Z$ mit positiver oberer Banach-Dichte auf, wenn $BD^*(\xi^{-1}(a)) > 0$.
\end{defn}

% Lem 3.17
\begin{lem}
  Sei $\xi \in \Lambda^\Z$ und $X \coloneqq \clos{\Set{T^n \xi}{n \in \Z}}$ dessen abgeschl. Orbit.
  Sei $a \in \Lambda$ ein Symbol und $A(a) \coloneqq \Set{\omega \in X}{\omega(0) = a}$.
  Dann tritt $a$ genau dann mit positiver Banach-Dichte in $\xi$ auf, wenn ein $T$-inv. Maß $\mu$ auf $X$ mit $\mu(A(a)) > 0$ existiert.
\end{lem}

\begin{nota}
  $S_{\text{diff}} \coloneqq \Set{x - y}{x, y \in S}$ für $S \subset \Z$.
\end{nota}

% Thm 3.18
\begin{thm}
  Sei $W \!\subset\! \Z$ eine Poincaré-Folge und habe $S \!\subset\! \Z$ positive obere Banach-Dichte.
  Dann gilt
  $W \cap S_{\text{diff}} \setminus \{ 0 \} \neq \emptyset$.
\end{thm}

% Thm 3.19 a)
\begin{kor}
  $S_{\text{diff}}$ ist syndetisch.
\end{kor}

% Bemerkung am Ende des Kapitels
\begin{bem}
  Man kann zeigen: Wenn $S_1, \ldots, S_k \subset \Z$ alle positive obere Banach-Dichte besitzen, dann ist $S_{1,\text{diff}} \cap \ldots \cap S_{k,\text{diff}}$ syndetisch.
\end{bem}

% Thm 3.19 b)
\begin{thm}[\emph{Sárközy}]
  Habe $S \subset \Z$ positive obere Banach-Dichte, sei $p[X] \in \Q[X]$ mit $\fa{n \in \Z} p(n) \in \Z$ und $0 \in p(\Z)$.
  Dann existiert eine Lsg der Gleichung $x - y = p(z)$ in Variablen $x, y \in S$, $x \neq y$, $z \in \Z$.
\end{thm}

\begin{samepage}
  \section{4.3. Von Neumannscher Ergodensatz und generische Maße}
\end{samepage}

\begin{defn}
  Sei $H$ ein Hilbertraum. Eine Folge $(T_n)_{n \in \N}$ von Operatoren in $\Boun(H)$ konvergiert gegen $T \in \Boun(H)$
  \begin{itemize}
    \item in der \emph{schwachen Operatortopologie} (WOT), wenn
    \[ \fa{x, y \in H} \scp{T_n x}{y} \xrightarrow{n \to \infty} \scp{T x}{y}. \]
    \item in der \emph{starken Operatortopologie} (SOT), wenn
    \[ \fa{x \in H} T_n x \xrightarrow{n \to \infty} T x. \]
  \end{itemize}
\end{defn}

% Thm 4.13.
\begin{thm}
  Sei $H$ ein Hilbertraum, $T \!\in\! \Boun(H)$ unitär, $H_T \!\coloneqq\! \{ x {\in} H | Tx \!=\! x \}$. Sei $P_T : H \to H$ die orth. Projektion auf $H_T$. Dann konvergiert
  \[
    \tfrac{1}{n+1} (\Id + T + T^2 + \ldots + T^n) \xrightarrow{n \to \infty} P_T \quad
    \text{in der starken Operatortopol.}
  \]
\end{thm}

\begin{bem}
  Sei $T$ eine maßerh. Transformation auf einem Maßraum $(X, \Bor, \mu)$. Dann induziert $T$ einen unitären Operator $T$ auf dem Hilbertraum $L^2(X, \Bor, \mu)$ durch $T(f) \coloneqq f \circ T$. Aus dem Theorem folgt, dass für jede Funktion $f \in L^2(X, \Bor, \mu)$ die Folge der ergodischen Mittel $f_n$ in $L^2$ gegen $P_T f$ konvergiert.
\end{bem}

% Defn 4.4
\begin{defn}
  Sei $\nu$ ein Maß auf $(X, \Bor)$ und $\Alg$ eine Algebra von beschränkten $\Bor$-messbaren Funktionen $X \to \C$, die abgeschlossen unter komplexer Konjugation ist. Das Maß $\nu$ heißt \emph{generisches Maß} von \meST{} bzgl. $\Alg$, wenn für alle $f \in \Alg$ gilt:
  \[ \Int{X}{}{f_n}{\nu} \xrightarrow{n \to \infty} \Int{X}{}{f}{\mu}. \]
\end{defn}

\begin{bem}
  Aus dem Ergodensatz folgt: $\mu$ ist selbst ein generisches Maß von \meST{} bzgl. $\Alg \coloneqq \{\, \text{beschränkte $\Bor$-messb. Fktn $X \to \C$} \,\}$.
\end{bem}

\begin{bem}
  Sei $X$ ein komp. metr. Raum und $T : X \to X$ stetig.
  Sei $x_0$ ein generischer Punkt eines Maßes $\mu$ auf $X$.
  Dann ist das Punktmaß $\delta_{x_0}$ ein generisches Maß von \meST{} bzgl. $\Alg \coloneqq \Cont(X)$.
\end{bem}

\begin{bem}
  Angenommen, es gibt eine Teilmenge $B \subset \Alg$ von $T$-inv. Fktn, die dicht in $L^2(X, \Bor, \mu)_T \coloneqq \Set{f \in L^2(X, \Bor, \mu)}{f \!=\! f \!\circ\! T}$ liegt. Es gilt
  \[
    \fa{f \in B}
    \Int{X}{}{\abs{f}^2}{\mu} = \lim_{N \to \infty} \tfrac{1}{N+1} \Int{X}{}{T^n \abs{f}^2}{\nu} = \Int{X}{}{\abs{f}^2}{\nu}.
  \]
  Man kann also eine dichte Teilmenge von $L^2(X, \Bor, \mu)_T$ isometrisch mit einer Teilmenge in $L^2(X, \Bor, \nu)$ identifizieren und so $L^2(X, \Bor, \mu)$ als Teilmenge von $L^2(X, \Bor, \nu)$ auffassen.
\end{bem}

% Thm 4.14
\begin{thm}
  Sei $\nu$ ein generisches Maß von \meST{} bzgl. $\Alg$. Enthalte $\Alg$ eine Teilmenge von $T$-invarianten Funktionen, die dicht in $L^2(X, \Bor, \mu)_T$ liegt. Dann gilt
  \[
    \fa{f \in \Alg}
    \frac{1}{N+1} \sum_{n=0}^N T^n f \xrightarrow{N \to \infty} P_T f \quad
    \text{in } L^2(X, \Bor, \nu).
  \]
\end{thm}

\end{document}