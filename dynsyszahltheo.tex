\documentclass{cheat-sheet}

\pdfinfo{
  /Title (Dynamische Systeme und Zahlentheorie)
  /Author (Tim Baumann)
}

\DeclareMathOperator{\Aut}{Aut} % Automorphismengruppe
\DeclareMathOperator{\End}{End} % Endomorphismenmonoid
\newcommand{\AutEnd}{\Aut\!/\!\End} % Automorphismengruppe bzw. Endomorphismenmonoid
\DeclareMathOperator{\Iso}{Iso} % Isometriegruppe
\newcommand{\clos}[1]{\overline{#1}} % topologischer Abschluss

\begin{document}

\maketitle{Dyn. Systeme in der Zahlentheorie}

% TODO:
% Van der Waerdens Theorem
% Szemerédis Theorem
% Poincarés Wiederkehrsatz
% Rados Theorem
% "Fast-Periodizität"
% gleichförmig wiederkehrend
% "proximality"
% Schur und Brauers Resultat (SB)
% Hindmans Resultat (NH)
% Grünwalds Resultat (G)
% Hilberts Resultat (DH)
% Theorem (MBR = "multiple Birkhoff Recurrence")
% Stabilität:
%   * Im Sinne von Poisson
%   * Im Sinne von Lagrange
%   * Im Sinne von Lyapunov

% Kapitel 1. Wiederkehr und gleichmäßige Wiederkehr in kompakten Räumen

% 1.1. Dynamische Systeme und wiederkehrende Punkte

\begin{defn}
  Ein \emph{dynamisches System} ist ein Paar $(X, G)$ bestehend aus einem kompakten metrischen Raum $X$ und einer Gruppe oder einem Monoid $G$ mit Wirkung
  $\varphi : G \to \AutEnd(X), \, g \mapsto T_g$.
\end{defn}

\begin{defn}
  Ein \emph{Untersystem} eines dynamischen Systems $(X, G)$ ist eine Teilmenge $Z \subseteq X$ mit $T_g(Z) \subseteq Z$ für alle $g \in G$.
\end{defn}

\begin{bem}
  Falls $G \!=\! \Z$ oder $M \!=\! \N$, dann bezeichnen wir mit $T \coloneqq T_1$ den Erzeuger der Aktion und nennen $(X, T)$ ein \emph{zykl. System}.
\end{bem}

% Def 1.1
\begin{defn}
  Sei $X$ ein topol. Raum, $T : X \to X$ stetig.
  Ein Punkt $x \in X$ heißt \emph{wiederkehrend}, falls für für alle Umgebungen $V \subset X$ von $x$ ein $n \geq 1$ existiert mit $T^n(x) \in V$.
\end{defn}

\begin{bem}
  Sei $X$ sogar ein metrischer Raum, $x \in X$ wiederkehrend. \\
  Dann gibt es eine Folge $(n_k)$ mit $d(T^{n_k}(x), x) \to 0$ für $k \to \infty$.
\end{bem}

% nicht dort im Buch
\begin{defn}
  Sei $X$ ein topol. Raum, $T : X \to X$ stetig. Dann heißt
  \[ Q(x) \coloneqq \clos{\Set{T^n x}{n \geq 1}} \subseteq X \]
  \emph{abgeschlossener Vorwärtsorbit} von $x \in X$.
\end{defn}

% nicht explizit im Buch
\begin{lem}
  \begin{itemize}
    \item $x \in X$ ist wiederkehrend $\iff$ $x \in Q(x)$
    \item $x \in Q(y) \implies T(x) \in Q(y) \iff Q(x) \subseteq Q(y)$
    \item Die Relation $xRy \!\!\coloniff\!\! x \in Q(y)$ ist transitiv.
  \end{itemize}
\end{lem}

\begin{thm}
  Sei $X$ ein kompakter topol. Raum, $T : X \to X$ stetig. \\
  Dann gibt es einen wiederkehrenden Punkt $x \in X$.
\end{thm}
% XXX: Bemerkung: Beweis benutzt Auswahlaxiom?

% Defn 1.2
\begin{defn}
  Sei $K$ eine kompakte Gruppe, $a \in K$ und $T(x) \coloneqq ax$. Dann heißt $(K, T)$ ein \emph{Kronecker-System}.
\end{defn}

% Thm 1.2
\begin{thm}
  In einem Kronecker-System sind alle $x \in K$ wiederkehrend.
\end{thm}

% 1.2 Automorphismen und Homomorphismen von Dynamischen Systemen, Faktoren und Erweiterungen

\begin{defn}
  Ein Homomorphismus zwischen zwei dyn. Systemen $(X, G)$ und $(X', G)$ (zweimal die gleiche Gruppe oder Monoid $G$) ist eine $G$-äquivariante stetige Abbildung $\phi : X \to X'$.
\end{defn}

% Defn 1.3
\begin{defn}
  Ein dyn. System $(Y, G)$ ist \emph{Faktor} eines dyn. System $(X, G)$, wenn es einen surjektiven Homomorphismus $(X, G) \to (Y, G)$ gibt. \\
  Man nennt $(X, G)$ dann eine \emph{Erweiterung} von $(Y, G)$.
\end{defn}

% XXX: Auslassen?
\begin{bem}
  Sei $\phi : X \to Y$ surjektiv. Dann kann man $Y$ mit der Menge der Fasern von $\phi$ identifizieren.
\end{bem}

% Thm 1.3
\begin{thm}
  Sei $\phi : (X, T) \to (Y, T)$ ein Morphismus von zyklischen Systemen.
  Wenn $x \in X$ wiederkehrend ist, dann auch $\phi(x)$. \\
  Allgemeiner: $x \in Q(y) \implies \phi(x) \in Q(\phi(y))$
\end{thm}

% Defn 1.4
\begin{defn}
  Sei $(Y, T : Y \to Y)$ ein zyklisches System, $K$ eine kompakte Gruppe und $\psi : Y \to K$ stetig. Setze
  \[
    X \coloneqq Y \times K, \quad
    T : X \to X, \enspace (y, k) \mapsto (Ty, \psi(y)k).
  \]
  Das System $(X, T)$ wird \emph{Gruppenerweiterung} von $(Y, T)$ mit $K$ oder \emph{Schiefprodukt} von $(Y, T)$ mit $K$ genannt.
\end{defn}

\begin{bem}
  Die Gr. $K$ wirkt auf $(X, T) = (Y \!\times\! K, T)$ durch Rechtstransl.:
  \[
    R : K \to \Aut(X), \enspace k \mapsto R_k, \quad
    R_k(y,k') \coloneqq (y,k'k).
  \]
  Die Homöomorphismen $R_k$ kommutieren mit $T$, sind also Automorphismen des dyn. Systems $(X, T)$.
\end{bem}

% Thm 1.4
\begin{thm}
  Sei $(X \!=\! Y \!\times\! K, T)$ eine Gruppenerw. von $(Y, T)$ und $y_0 \in Y$ wiederkehrend. Dann sind die Pkte $\Set{(y_0, k)}{k \in K}$ wiederkehrend.
\end{thm}

% Fußnote 1 Seite 22
\begin{bem}
  Durch Erweiterung mit der zykl. Gr. $\Z_m$ kann man zeigen:
\end{bem}

% Fußnote 1 Seite 22
\begin{prop}
  Ist $x \in X$ in $(X, T)$ wiederkehrend, dann auch in $(X, T^m)$.
\end{prop}

\begin{bsp}
  Sei $T \coloneqq \R / \Z$ und $\alpha \in \R$. Dann ist das System
  \[ (T^2, (\theta, \phi) \mapsto (\theta + \alpha, \phi + 2 \theta + \alpha)) \]
  eine Gruppenerweiterung des Kronecker-Systems $(T, \theta \mapsto \theta + \alpha)$. \\
  Somit sind alle Punkte des Torus $T^2$ wiederkehrend. \\
  Aus der Wiederkehr des Punktes $(0, 0)$ erhält man:
\end{bsp}

% Thm 1.5
\begin{prop}
  Für jedes $\alpha \in \R$ und $\epsilon > 0$ gibt es eine ganzzahlige Lsg der diophantinischen Ungleichung
  $\abs{\alpha n^2 - m} < \epsilon$.
\end{prop}

\begin{bem}
  Durch Verallgemeinerung auf den $d$-dim Torus zeigt man:
\end{bem}

% Thm 1.6
\begin{prop}
  Sei $p(X) \in \R[X]$ mit $p(0) = 0$. Dann gibt es für alle $\epsilon > 0$ eine Lsg der diophantinischen Ungleichung
  $\abs{p(n) - m} < \epsilon$, $n > 0$.
\end{prop}

% Defn
\begin{defn}
  Sei $M$ ein topol. Raum und $K \subseteq \Iso(M)$ kompakt. \\
  Sei $(Y, T)$ ein zykl. System und $\psi : Y \to K$ stetig. Setze
  \[
    X \coloneqq Y \!\times\! M, \quad
    T : X \to X, \enspace (y, u) \mapsto (Ty, \psi(y)u).
  \]
  Das System $(X, T)$ heißt \emph{isometrische Erweiterung} von $(Y, T)$.
\end{defn}

% Thm 1.8
\begin{prop}
  Sei $(X, T)$ eine isom. Erweiterung von $(Y, T)$ und $y_0 \in Y$ wiederkehrend.
  Dann sind die Pkte $\Set{(y, m)}{m \in M}$ wiederkehrend.
\end{prop}

% 1.3. Wiederkehrende Punkte von Bebutov-Systemen

% Defn 1.6
\begin{defn}
  Sei $G$ eine abz. Gruppe/Monoid und $\Lambda$ ein kompakter metr. Raum.
  Sei $\Omega \coloneqq \Lambda^G \cong \prod \Lambda$ der kompakte metrisierbare Raum der Funktionen von $G$ nach $\Lambda$. Die \emph{reguläre Wirkung} von $G$ auf $\Omega$ ist
  \[
    G \mapsto \AutEnd(\Omega), \enspace g \mapsto T_g, \quad
    T_g(\omega)(g') \coloneqq \omega(g'g).
  \]
  Ein \emph{Bebutov-System} ist ein Untersystem von $(\Omega, G)$.
\end{defn}

\begin{bem}
  Sei $\{ g_1, g_2, \ldots \} = G$ eine Abzählung von $G$. \\
  Dann ist eine Metrik auf $\Omega$ definiert durch
  \[ d(\omega, \omega') \coloneqq \sum 2^{-n} d(\omega(g_n), \omega'(g_n)). \]
\end{bem}

\begin{defn}
  Für $\omega_0 \in \Omega$ ist der Abschluss des Orbits von $\omega_0$,
  \[ X_{\omega_0} \coloneqq \clos{\Set{T_g(\omega_0)}{g \in G}}, \]
  $G$-invariant. Das dynamische System $(X_{\omega_0}, G)$ wird das von $\omega_0$ \emph{erzeugte Bebutov-System} genannt.
\end{defn}

\begin{defn}
  Ein \emph{symbolischer Fluss} ist ein Bebutov-System mit endlichem $\Lambda$ und $G \in \{ \N, \Z \}$.
  Die Elemente von $\Omega$ sind dann unendliche/doppelt-unendliche Folgen von Elementen von $\Lambda$. \\
  Man bezeichnet $\Lambda$ dann als \emph{Alphabet}.
\end{defn}

% Ausgelassen: Theorem 1.9 über die Beziehung von fast-periodischen Funktionen (im Sinne von Bohr) mit der Theorie der Kronecker-Systeme

\begin{defn}
  Ein \emph{Wort} über $\Lambda$ ist eine endl. Sequenz von Elementen aus $\Lambda$. \\
  Die Länge $\abs{w}$ eines Wortes ist die Länge der Sequenz.
\end{defn}

% Prop 1.10
\begin{prop}
  Für eine Sequenz $\omega \in \Lambda^\N$ sind äquivalent:
  \begin{itemize}
    \item $\omega$ ist wiederkehrend.
    \item Jedes Wort in $\omega$ kommt ein 2.\,Mal an einer anderen Pos. in $\omega$ vor.
    \item Jedes Wort aus $\omega$ kommt an unendlich oft in $\omega$ vor.
  \end{itemize}
\end{prop}

\begin{bem}
  Ein wiederkehrendes Wort $\omega \in \Lambda^\N$ hat die allgemeine Form
  \[ \omega = [(aw^{(1)}a)w^{(2)}(aw^{(1)}a)]w^{(3)}[(aw^{(1)}a)w^{(2)}(aw^{(1)}a)]\ldots \]
  mit $a \in \Lambda$ und Wörtern $w^{(1)}, w^{(2)}, \ldots$.
  Damit kann man zeigen:
\end{bem}

\begin{lem}[Hilbert]
  Sei $\N = B_1 \cup B_2 \cup \ldots \cup B_q$ eine Partition von $\N$ und $l \in \N_{>0}$ beliebig.
  Schreibe
  \[ P(x_1, \ldots, x_l) \coloneqq \Set{ x_{i_1} + \ldots + x_{i_k} }{0 \leq k \leq l, \, 1 \leq i_1 < \ldots i_k \leq l}. \]
  Dann gibt es $m_1 \leq m_2 \leq \ldots \leq m_l$, sodass unendlich viele Translationen von $P(m_1, \ldots, m_l)$ in demselben $B_j$ enthalten sind.
\end{lem}

\begin{bem}
  Sei $(X, T)$ ein zykl. System und $f : X \to \Lambda$ stetig. Dann ist
  \[
    (X, T) \to (\Lambda^\N, T), \quad
    x \mapsto (f(x), f(Tx), f(T^2x), \ldots)
  \]
  ein Homomorphismus zyklischer Systeme.

\end{bem}

% Thm 1.11
\begin{thm}
  Seien $\Lambda_1$, $\Lambda_2$ komp. Räume und $\phi : \Lambda_1 \to \Lambda_2$ eine Abbildung. \\
  Für $\omega \in \Lambda_1^\N$ definiere $\omega' \in \Lambda_2^\N$ durch $\omega'(n) \coloneqq f(\omega(n))$. \\
  Falls $\omega$ wiederkehrend ist und zusätzlich $f$ in allen Punkten $\omega(n)$ stetig ist, dann ist auch $\omega'$ wiederkehrend.
\end{thm}

% Ausgelassen: Beispiel

% Prop 1.12
\begin{prop}
  Sei $K$ eine komp. Gruppe und $\xi \in K^\N$ wiederkehrend. Dann ist $\eta \!\in\! K^\N$ definiert durch $\eta(n) \coloneqq \xi(n) \xi(n-1) \cdots \xi(1)$ wiederkehrend.
\end{prop}

% Ausgelassen: Beispiel

% 1.4. Gleichmäßige Wiederkehr und minimale Systeme

\end{document}