% Kategorientheorie-Makros

% Konzepte
\DeclareMathOperator{\Ob}{Ob} % Objekte (einer Kategorie)
\DeclareMathOperator{\Mor}{Mor} % Morphismenmenge / -klasse
\DeclareMathOperator{\Hom}{Hom} % Homomorphisms
\DeclareMathOperator{\dom}{dom} % Domain
\DeclareMathOperator{\codom}{codom} % Codomain
\newcommand{\op}{\mathrm{op}} % opposite category
\DeclareMathOperator{\Aut}{Aut} % Automorphismengruppe
\newcommand{\ladj}{\dashv} % Links-adjungiert (left-adjoint)

% Konkrete Kategorien
\newcommand{\SetC}{\mathbf{Set}} % Kategorie der Mengen
\newcommand{\sSet}{\mathbf{sSet}} % Kategorie der simplizialen Mengen
\newcommand{\Top}{\mathbf{Top}} % Kategorie der topologischen Räume
\newcommand{\AbGrp}{\mathbf{Ab}} % Kategorie der abelschen Gruppen
\newcommand{\Grp}{\mathbf{Grp}} % Kategorie der Gruppen
\newcommand{\RMod}{\mathbf{R\text{-}Mod}} % Kategorie der R-Moduln
\newcommand{\Ouv}{\mathbf{Ouv}} % Kategorie der offenene Mengen eines topol. Raumes
\newcommand{\KHaus}{\mathbf{KHaus}} % Kategorie der kompakten Hausdorffräume
\newcommand{\Sh}{\mathbf{Sh}} % Kategorie der Garben (sheaves)
\newcommand{\PSh}{\mathbf{PSh}} % Kategorie der Prägarben (abelscher Gruppen)
\newcommand{\CatC}{\mathbf{Cat}} % Kategorie der kleinen Kategorien
\newcommand{\Vect}{\mathbf{Vect}} % Kategorie der Vektorräume über einem Körper
\newcommand{\Mod}{\mathbf{Mod}} % Kategorie der Moduln über einem Ring

% Bezeichnungen für Variablen, die für Kategorien stehen
\newcommand{\Aat}{\mathcal{A}} % Category-A
\newcommand{\Bat}{\mathcal{A}} % Category-A
\newcommand{\Cat}{\mathcal{C}} % Category-C
\newcommand{\Dat}{\mathcal{D}} % Category-D
\newcommand{\Eat}{\mathcal{E}} % Category-E
\newcommand{\Iat}{\mathcal{I}} % Category-I (Indexkategorie)
\newcommand{\Jat}{\mathcal{J}} % Category-J (Indexkategorie)
\newcommand{\Sit}{\mathcal{S}} % Situs-S
