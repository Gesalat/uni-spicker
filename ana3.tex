\documentclass{cheat-sheet}

\usepackage{bbm} % Für 1 mit Doppelstrich (Indikatorfunktion)
\usepackage{amsopn}
\usepackage{setspace}

\newcommand{\PS}{\mathcal{P}} % Potenzmenge
\newcommand{\PSO}{\PS(\Omega)} % Potenzmenge
\newcommand{\Alg}{\mathfrak{A}} % (Mengen-)Algebra
\newcommand{\Ring}{\mathfrak{R}} % (Mengen-)Ring
\newcommand{\Dyn}{\mathfrak{D}} % Dynkin-System
\newcommand{\Bor}{\mathfrak{B}} % Borel
\newcommand{\E}{\mathbb{E}} % Elementare Funktion
\newcommand{\Leb}{\mathcal{L}} % Lebesgue
\newcommand{\ind}{\mathbbm{1}} % Indikatorfunktion
\newcommand{\fue}{\overset{\text{f.ü.}}} % fast überall
\newcommand{\Atlas}{\mathcal{A}} % Atlas

% Integrale
\newcommand{\Intdf}[2]{{\textstyle \int\limits_{#1}} #2} % Differentialformintegral
\newcommand{\IntO}[2]{\Int{\Omega}{}{#1}{#2}} % Integral über \Omega
\newcommand{\IntOmu}[1]{\Int{\Omega}{}{#1}{\mu}} % Integral über \Omega bzgl. \mu

% Kleinere Klammern
\delimiterfactor=701

\usepackage{relsize}
\let\myBinom\binom
\renewcommand{\binom}[2]{\mathsmaller{\myBinom{#1}{#2}}}

\pdfinfo{
  /Title (Zusammenfassung Analysis 3)
  /Author (Tim Baumann)
}

\begin{document}


\maketitle{Zusammenfassung Analysis \rom{3}}

\section{Maßtheorie}

\begin{prob}[Schwaches Maßproblem]
  Gesucht: Abbildung $\mu : \PS(\R^n) \to [\R, \infty]$ mit folgenden Eigenschaften:
  \begin{itemize}
    \item Normierung: $\mu([0, 1]^n) = 1$
    \item Endliche Additivität: Sind $A, B \subset \R^n$ disjunkt, so gilt $\mu(A \cup B) = \mu(A) + \mu(B)$.
    \item Bewegungsinvarianz: Für eine euklidische Bewegung $f : \R^n \to \R^n$ und $A \subset \R^n$ gilt $\mu(f(A)) = \mu(A)$.
  \end{itemize}
\end{prob}

\begin{satz}[Hausdorff]
  Das schwache Maßproblem ist für $n \geq 3$ nicht lösbar.
\end{satz}

\begin{satz}[Banach]
  Das schwache Maßproblem ist für $n = 1, 2$ lösbar, aber nicht eindeutig lösbar.
\end{satz}

\begin{prob}[Starkes Maßproblem]
  Gesucht ist eine Abbildung $\mu : \PS(\R^n) \to [0, \infty]$ wie im schwachen Maßproblem, die anstelle der endlichen Additivität die Eigenschaft der $\sigma$-Additivität besitzt:
  \begin{itemize}
    \item Für eine Folge $(A_n)_{n \in \N}$ pw. disjunkter Teilmengen des $\R^n$ ist
      \[ \mu\left(\bigcup_{n \in \N} A_n\right) = \sum_{n=0}^\infty \mu(A_n). \]
  \end{itemize}
\end{prob}

\begin{satz}
  Das starke Maßproblem besitzt keine Lösung.
\end{satz}

\begin{nota}
  Sei im Folgenden $\Omega$ eine Menge.
\end{nota}

\begin{defn}
  $\Ring \subset \PSO$ heißt \emph{Ring}, wenn für alle $A, B \in \Ring$ gilt:
  \begin{itemize}
    \miniitem{0.13 \linewidth}{$\emptyset \in \Ring$}
    \miniitem{0.85 \linewidth}{Abgeschlossenheit unter Differenzen: $A \setminus B \in \Ring$}
    \item Abgeschlossenheit unter endlichen Vereinigungen: $A \cup B \in \Ring$
  \end{itemize}
\end{defn}

\begin{defn}
  $\Alg \subset \PSO$ heißt \emph{Algebra}, wenn für alle $A, B \in \Alg$ gilt:
  \begin{itemize}
    \miniitem{0.13 \linewidth}{$\emptyset \in \Alg$}
    \miniitem{0.85 \linewidth}{Abgeschlossenheit unter Komplementen: $A^c = \Omega \setminus A \in \Alg$}
    \item Abgeschlossenheit unter endlichen Vereinigungen: $A \cup B \in \Alg$
  \end{itemize}
\end{defn}

\begin{defn}
  Eine Algebra $\Alg \subset \PSO)$ heißt \emph{$\sigma$-Algebra}, wenn $\Alg$ unter abzählbaren Vereinigungen abgeschlossen ist, d.\,h. für jede Folge $(A_n)_{n \in \N}$ in $\Alg$ gilt $\bigcup_{n \in \N} A_n \in \Alg$.
\end{defn}

\begin{bem}
  \begin{itemize}
    \item Jede Algebra ist auch ein Ring.
    \item Ein Ring $\Ring \subset \PSO$ ist auch unter endlichen Schnitten abgeschlossen, da $A \cap B = A \setminus (B \setminus A) \in \Ring$
    \item Ein Ring $\Ring \subset \PSO$ ist genau dann eine Algebra, wenn $\Omega \in \Ring$
    \item Eine $\sigma$-Algebra $\Alg \subset \PSO$ ist auch unter abzählbaren Schnitten abgeschlossen: Sei $(A_n)_{n \in \N}$ eine Folge in $\Alg$, dann gilt
      \[ \bigcap_{n \in \N} A_n = \left( \bigcup_{n \in \N} (A_n)^c \right)^c \in \Alg \]
  \end{itemize}
\end{bem}

\begin{nota}
  Sei im Folgenden $\Ring \subset \PSO$ ein Ring.
\end{nota}

% Lemma 1.10 (Urbildalgebra)

\begin{satz}
  Sei $(A_i)_{i \in I}$ eine Familie von Ringen / Algebren / $\sigma$-Algebren über $\Omega$. Dann ist auch $\cap_{i \in I} A_i$ ein Ring / eine Algebra / eine $\sigma$-Algebra über $\Omega$.
\end{satz}

\begin{defn}
  Sei $E \subset \PSO$. Setze
  \begin{align*}
    \mathcal{R}(E) &\coloneqq \Set{ \Ring \subset \PSO }{ E \subset \Ring, \Ring \text{ Ring} } \text{ und} \\
    \mathcal{A}(E) &\coloneqq \Set{ \Alg \subset \PSO }{ E \subset \Alg, \Alg \text{ $\sigma$-Algebra} }.
  \end{align*}
  Dann heißen $\quad\Ring(E) \coloneqq \enspace \bigcap_{\mathclap{\Ring \in \mathcal{R}(E)}} \enspace \Ring, \quad \Alg(E)  \coloneqq \enspace \bigcap_{\mathclap{\Alg  \in \mathcal{A}(E)}} \enspace \Alg$\\
  von $E$ \emph{erzeugter Ring} bzw. von $E$ \emph{erzeugte $\sigma$-Algebra}.
\end{defn}

\begin{defn}
  Ist $(\Omega, \mathcal{O})$ ein topologischer Raum, dann heißt $\Bor = \Bor(\Omega, \mathcal{O}) \coloneqq \Alg(\mathcal{O})$ \emph{Borelsche $\sigma$-Algebra} von $(\Omega, \mathcal{O})$.
\end{defn}

\begin{bem}
  Die Borelsche $\sigma$-Algebra $\Bor(\R)$ wird auch erzeugt von $\Set{I \subset \R }{ I \text{ Intervall } }$. Dabei spielt es keine Rolle, ob man nur geschlossene, nur offene, nur nach einer Seite halboffene Intervalle oder gar nur Intervalle obiger Art mit Endpunkten in $\Q$ zulässt.
\end{bem}

\begin{defn}
  Eine Funktion $\mu : \Ring \to [0, \infty]$ heißt \emph{Inhalt} auf $\Ring$, falls
  \begin{itemize}
    \miniitem{0.2 \linewidth}{$\mu(\emptyset) = 0$}
    \miniitem{0.75 \linewidth}{$\mu(A \sqcup B) = \mu(A) + \mu(B)$ für disjunkte $A, B \in \Ring$.}
  \end{itemize}
\end{defn}

\begin{defn}
  Ein Inhalt $\mu : \Ring \to [0, \infty]$ heißt \emph{Prämaß} auf $\Ring$, wenn $\mu$ $\sigma$-additiv ist, d.\,h. wenn für jede Folge $(A_n)_{n \in \N}$ paarweise disjunkter Elemente von $\Ring$ mit $\bigsqcup_{n \in \N} A_n \in \Ring$ gilt:
  \[ \mu\left(\bigsqcup_{n \in \N} A_n\right) = \sum_{n=0}^\infty \mu(A_n) \]
\end{defn}

\begin{defn}
  Ein \emph{Maß} ist ein Prämaß auf einer $\sigma$-Algebra.
\end{defn}

\begin{satz}
  Für einen Inhalt $\mu$ auf $\Ring$ gilt für alle $A, B \in \Ring$:
  \begin{itemize}
    \item $\mu(A \cup B) + \mu(A \cap B) = \mu(A) + \mu(B)$
    \item $A \subset B \implies \mu(A) \leq \mu(B)$ \pright{Monotonie}
    \item Aus $A \subset B$ und $\mu(B) < \infty$ folgt $\mu(B \setminus A) = \mu(B) - \mu(A)$
    \item Für $A_1, ..., A_n \in \Ring$ ist $\mu\left(\bigcup_{i = 1}^n A_i \right) \leq \sum_{i = 1}^n \mu(A_i)$ \pright{Subadditivität}
    \item Ist $(A_n)_{n \in \N}$ eine Folge disjunkter Elemente aus $\Ring$, sodass $\bigsqcup_{n \in \N} A_n \in \Ring$, so gilt $\mu(\bigsqcup_{n \in \N} A_n) \geq \sum_{n=0}^\infty \mu(A_n)$.
  \end{itemize}
\end{satz}

\begin{defn}
  Ein Inhalt / Maß auf einem Ring~$\Ring$ / einer $\sigma$-Algebra~$\Alg$ heißt \emph{endlich}, falls $\mu(A) < \infty$ für alle $A \in \Ring$ bzw. $A \in \Alg$.
\end{defn}

\begin{satz}
  Ein Maß auf einer $\sigma$-Algebra~$\Alg$ ist $\sigma$-subadditiv, d.\,h. es gilt
  \[ \mu(\bigcup_{n \in \N} A_n) \leq \sum_{n=0}^\infty \mu(A_n) \quad \text{für alle Folgen $(A_n)_{n \in \N}$ in $\Alg$}. \]
\end{satz}

\begin{defn}
  Sei $A \subset \Omega$. Dann heißt die Abbildung
  \[ \chi_A = \ind_A : \Omega \to \R, \quad \omega \mapsto \abs{\Set{ \star }{ \omega \in A }} = \begin{cases} 1, & \text{ falls } \omega \in A \\ 0, & \text{ falls } \omega \not\in A \end{cases} \]
  \emph{Indikatorfunktion} oder \emph{charakteristische Funktion} von $A$.
\end{defn}

\begin{defn}
  Eine Folge $(A_n)_{n \in \N}$ \emph{konvergiert} gegen $A \subset \Omega$, notiert $\lim_{n \to \infty} A_n = A$, wenn $(\ind_{A_n})_{n \in \N}$ punktweise gegen $\ind_A$ konvergiert.
\end{defn}

\begin{defn}
  Für eine Folge $(A_n)_{n \in \N}$ in $\PS(\Omega)$ heißen
  \begin{align*}
    \limsup_{n \to \infty} A_n &\coloneqq \Set{ \omega \in \Omega }{ \omega \text{ liegt in unendlich vielen } A_n } \\
    \liminf_{n \to \infty} A_n &\coloneqq \Set{ \omega \in \Omega }{ \omega \text{ liegt in allen bis auf endlich vielen } A_n }
  \end{align*}
  \emph{Limes Superior} bzw. \emph{Limes Inferior} der Folge $A_n$. Es gilt
  \[
    \limsup_{n \to \infty} A_n = \bigcap_{n = 0}^\infty \bigcup_{k = n}^\infty A_n, \quad
    \liminf_{n \to \infty} A_n = \bigcup_{n = 0}^\infty \bigcap_{k = n}^\infty A_n.
  \]
\end{defn}

% TODO: Lemmas zum Limes Inferior, Limes Superior

\begin{satz}
  Es gilt $\lim_{\mathclap{n \to \infty}} A_n = A \iff \liminf_{n \to \infty} A_n = \limsup_{n \to \infty} A_n = A$.
\end{satz}

\begin{defn}
  Eine Folge $(A_n)_{n \in \N}$ in $\PSO$ heißt
  \begin{itemize}
    \item \emph{monoton wachsend}, wenn für alle $n \in \N$ gilt $A_n \subset A_{n+1}$,
    \item \emph{monoton fallend}, wenn für alle $n \in \N$ gilt $A_n \supset A_{n+1}$.
  \end{itemize}
\end{defn}

\begin{satz}
  Sei $(A_n)_{n \in \N}$ eine Folge in $\PSO$.
  \begin{itemize}
    \item Ist $(A_n)$ monoton wachsend, so gilt $\lim_{\mathclap{n \to \infty}} A_n = \bigcup_{\mathclap{n \in \N}} A_n$.
    \item Ist $(A_n)$ monoton fallend, so gilt $\lim_{\mathclap{n \to \infty}} A_n = \bigcap_{\mathclap{n \in \N}} A_n$.
  \end{itemize}
\end{satz}

\begin{satz}
  Sei $\mu$ ein Inhalt auf $\Ring \subset \PSO$. Wir betrachten die Aussagen:

  \begin{enumerate}[label=(\roman*),leftmargin=2em]
    \item $\mu$ ist ein Prämaß auf $\Ring$.
    \item Stetigkeit von unten: Für jede monoton wachsende Folge $(A_n)_{n \in \N}$ in $\Ring$ mit $A \coloneqq \lim_{n \to \infty} A_n = \bigcup_{n = 0}^\infty A_n \in \Ring$ gilt $\lim_{n \to \infty} \mu(A_n) = \mu(A)$.
    \item Stetigkeit von oben: Für jede monoton fallende Folge $(A_n)_{n \in \N}$ in $\Ring$ mit $\mu(A_0) < \infty$ und $A \coloneqq \lim_{n \to \infty} A_n = \bigcap_{n = 0}^\infty A_n \in \Ring$ gilt $\lim_{n \to \infty} \mu(A_n) = \mu(A)$.
    \item Für jede monoton fallende Folge $(A_n)_{n \in \N}$ in $\Ring$ mit $\mu(A_0) < \infty$ und $\lim_{n \to \infty} A_n = \bigcap_{n = 0}^\infty A_n = \emptyset$ gilt $\lim_{n \to \infty} \mu(A_n) = 0$.
  \end{enumerate}

  Dann gilt (i) $\iff$ (ii) $\implies$ (iii) $\iff$ (iv).\\
  Falls $\mu$ endlich ist, so gilt auch (iii) $\implies$ (ii).
\end{satz}

\begin{satz}
  Sei $\mu$ ein Maß auf einer $\sigma$-Algebra $\Alg \subset \PSO$. Dann gilt:
  \begin{itemize}
    \item Für eine Folge $(A_n)_{n \in \N}$ in $\Alg$ gilt $\mu\left(\liminf_{n \to \infty} A_n\right) \leq \liminf_{n \to \infty}(\mu(A_n))$.
    \item Sei $(A_n)_{n \in \N}$ eine Folge in $\Alg$, sodass es ein $N \in \N$ gibt mit $\mu\left(\bigcup_{n = N}^\infty A_n \right) < \infty$, dann gilt $\mu\left(\limsup_{n \to \infty} A_n \right) \geq \limsup_{n \to \infty} \mu(A_n)$.
    \item Sei $\mu$ endlich und $(A_n)_{n \in \N}$ eine Folge in $\Alg$, dann gilt
    \[ \mu\left(\liminf_{n \to \infty} A_n\right) \leq \liminf_{n \to \infty} \mu(A_n) \leq \limsup_{n \to \infty} \mu(A_n) \leq \mu\left(\limsup_{n \to \infty} A_n\right). \]
    \item Sei $\mu$ endlich und $(A_n)_{n \in \N}$ eine gegen $A$ konvergente Folge in $\Alg$, dann gilt $A \in \Alg$ und $\mu(A) = \lim_{n \to \infty} \mu(A_n)$.
  \end{itemize}
\end{satz}

\begin{defn}
  Ein Inhalt auf einem Ring $\Ring \subset \PSO$ heißt \emph{$\sigma$-endlich}, wenn gilt: Es gibt eine Folge $(S_n)_{n \in \N}$ in $\Ring$, sodass
  \[
      \Omega = \bigcup_{n \in \N} S_n
      \quad \text{und} \quad
      \mu(S_n) < \infty \text{ für alle } n \in \N.
  \]
\end{defn}

\begin{defn}
  Eine Funktion $f : \Omega \to \overline{\R} = \R \cup \{ \pm \infty \}$ wird \emph{numerische Funktion} genannt.
\end{defn}

\begin{defn}
  Eine numerische Funktion $\mu^* : \PSO \to \overline{\R}$ heißt \emph{äußeres Maß} auf $\Omega$, wenn gilt:
  \begin{itemize}
    \miniitem{0.3 \linewidth}{$\mu^*(\emptyset) = 0$}
    \miniitem{0.68 \linewidth}{$A \subset B \implies \mu^*(A) \leq \mu^*(B)$ \pright{Monotonie}}
    \item Für eine Folge $(A_n)_{n \in \N}$ in $\mathcal{P}(\Omega)$ gilt $\mu^*\left(\bigcup_{n \in \N} A_n \right) \leq \sum_{n = 0}^\infty \mu^*(A_n)$. %\hfill ($\sigma$-Subadditivität)
  \end{itemize}
\end{defn}

\begin{bem}
  Wegen $\mu^*(\emptyset) = 0$ und der Monotonie nimmt ein äußeres Maß nur Werte in $[0, \infty]$ an.
\end{bem}

\begin{defn}
  Eine Teilmenge $A \subset \Omega$ heißt \emph{$\mu^*$-messbar}, falls
  \[ \mu^*(Q) = \mu^*(Q \cap A) + \mu^*(Q \setminus A) \quad \text{für alle } Q \subset \Omega. \]
\end{defn}

\begin{satz}[Carathéodory]
  Für ein äußeres Maß $\mu^* : \PSO \to [0, \infty]$ ist
  \begin{itemize}
    \item $\Alg^* \coloneqq \Set{ A \subset \Omega }{ A \text{ ist $\mu^*$-messbar } }$ eine $\sigma$-Algebra und
    \item $\mu^*|_{\Alg^*}$ ist ein Maß auf $\Alg^*$.
  \end{itemize}
\end{satz}

\begin{satz}[\emph{Fortsetzungssatz}]
  Sei $\mu$ ein Prämaß auf einem Ring $\Ring$, dann gibt es ein Maß $\tilde{\mu}$ auf der von $\Ring$ erzeugten $\sigma$-Algebra $\Alg(\Ring)$ mit $\tilde{\mu}|_\Ring = \mu$. Falls $\mu$ $\sigma$-endlich, so ist $\tilde{\mu}$ eindeutig bestimmt.
\end{satz}

\begin{bem}
  Im Beweis wird ein äußeres Maß auf $\Omega$ so definiert:
  \begin{align*}
    \mathfrak{U}(Q) &\coloneqq \left\{ (A_n)_{n \in \N} \,\middle|\, Q \subset \bigcup_{n = 0}^\infty A_n \text{ und } A_n \text{ Folge in } \Ring \right\}, \\
    \mu^*(Q) &\coloneqq \inf \left( \left\{ \sum_{i = 0}^\infty \mu(A_n) \,\middle|\, (A_n)_{n \in \N} \in \mathfrak{U}(Q) \right\} \cup \{ \infty \} \right).
  \end{align*}
  Das äußere Maß $\mu^*$ eingeschränkt auf $\Alg^* \supset \Alg(\Ring)$ ist ein Maß.
\end{bem}

\begin{satz}
  Sei $(\Omega, \Alg, \mu)$ ein $\sigma$-endlicher Maßraum und $\mathcal{E}$ ein Erzeuger von $\Alg$, der unter Schnitten abgeschlossen ist. Es gebe eine Folge $(E_n)_{n \in \N}$ mit $E_n \uparrow \Omega$ und $\mu(E_n) < \infty$ für jedes $n \in \N$. Dann ist $\mu$ durch die Werte auf $\mathcal{E}$ eindeutig festgelegt.
\end{satz}

\begin{samepage}

\subsection{Das Lebesgue-Borel-Maß}

\begin{nota}
  Für $a = (a_1, ..., a_n)$ und $b = (b_1, ..., b_n)$ schreibe
  \begin{itemize}
    \item $a \lhd b$, falls $a_j < b_j$ für alle $j = 1, ..., n$.
    \item $a \unlhd b$, falls $a_j \leq b_j$ für alle $j = 1, ..., n$.
  \end{itemize}
\end{nota}

\end{samepage}

\begin{defn}
  Für $a, b \in \R^n$ heißen
  \[
    \left]a, b\right[\,\coloneqq \Set{ x \in \R^n }{ a \lhd x \lhd b }, \quad
    \mu(\left]a, b\right[) \coloneqq \prod_{j = 1}^{n} (b_j - a_j)
  \]
  \emph{Elementarquader} und \emph{Elementarinhalt}. Sei im Folgenden $\mathcal{E}$ die Menge aller Elementarquader.
\end{defn}

\begin{satz}
  Für alle $A \in \Ring(\mathcal{E})$ gibt es paarweise disjunkte Elementarquader $Q_1, ..., Q_p \in \mathcal{E}$ sodass $A = Q_1 \sqcup ... \sqcup Q_p$.
\end{satz}

\begin{defn}
  Für $A \in \Ring(\mathcal{E})$ setze $\mu(A) \coloneqq \textstyle\sum_{i = 1}^p \mu(Q_i)$, wenn $A = Q_1 \sqcup ... \sqcup Q_p$ für paarweise disjunkte $Q_1, ..., Q_p$.
\end{defn}

\begin{satz}
  $\mu$ definiert ein Prämaß auf $\Ring(\mathcal{E})$, genannt das \emph{Lebesgue-Borel-Prämaß} auf $\R^n$.
\end{satz}

\begin{defn}
  Die eindeutige (da $\mu$ $\sigma$-endlich) Fortsetzung $\tilde{\mu}$ von $\mu$ auf $\Alg(\mathcal{E}) = \Bor(\R^n)$ wird \emph{Lebesgue-Borel-Maß} genannt.
\end{defn}

\begin{bem}
  Nur das Lebesgue-Borel-Maß ist ein Maß auf $\Bor(\R^n)$, welches jedem Elementarquader seinen Elementarinhalt zuordnet.
\end{bem}

\begin{defn}
  Sei $\mu$ ein Maß auf einer $\sigma$-Algebra $\Alg \subset \PSO$. Eine Menge $N \subset \Omega$ heißt ($\mu$)-\emph{Nullmenge}, wenn es $A \in \Alg$ gibt mit $N \subset A$ und $\mu(A) = 0$. Die Menge aller Nullmengen ist $\mathfrak{N}_\mu \subset \PSO$.
\end{defn}

\begin{defn}
  Sei $\mu$ das Lebesgue- Borel-Maß auf $\Bor(\R^n)$. Dann heißt die von $\Bor(\R^n)$ und den ent\-sprech\-en\-den Nullmengen erzeugte $\sigma$-Algebra $\tilde{\Alg}_\mu$ \emph{Lebesguesche $\sigma$-Algebra}, notiert $\mathfrak{L}(\R^n)$, und das fortgesetzte Maß \emph{Lebesgue-Maß}.
\end{defn}

\begin{defn}
  Sei $\Omega$ eine Menge und $\Alg \subset \PSO$ eine $\sigma$-Algebra auf $\Omega$, sowie ggf. $\mu$ ein Maß auf $\Alg$. Dann heißt
  \begin{itemize}
    \item das Tupel $(\Omega, \Alg)$ \emph{messbarer Raum},
    \item das Tripel $(\Omega, \Alg, \mu)$ \emph{Maßraum}.
  \end{itemize}
\end{defn}

\begin{defn}
  Seien $(\Omega, \Alg)$ und $(\Omega', \Alg')$ zwei messbare Räume. Eine Abbildung $f : \Omega \to \Omega'$ heißt \emph{messbar} oder genauer $(\Alg, \Alg')$-messbar, wenn für alle $A' \in \Omega'$ gilt $f^{-1}(A') \in \Omega$ oder, kürzer, $f^{-1}(\Alg') \subset \Alg$.
\end{defn}

\begin{bem}
  Die messbaren Räume bilden eine Kategorie mit messbaren Abbildungen als Morphismen, d.\,h. die Identitäts- abbildung von einem messbaren Raum zu sich selbst ist messbar und die Verkettung zweier messbarer Abbildungen ist messbar.
\end{bem}

% Aufgabe 1.55
\begin{satz}
  \begin{itemize}
    \item Seien $(\Omega, \Alg)$ ein messbarer Raum, $\Omega'$ eine Menge und $f : \Omega \to \Omega'$ eine Abbildung. Die größte $\sigma$-Algebra auf $\Omega'$, sodass $f$ messbar ist, ist dann $\Alg' \coloneqq \Set{ A' \subset \Omega' }{ f^{-1}(A') \in \Alg }$.
    \item Ist $\Omega$ eine Menge und $(\Omega', \Alg')$ ein messbarer Raum sowie $f : \Omega \to \Omega'$ eine Abbildung. Dann ist $f^{-1}(\Alg')$ eine $\sigma$-Algebra.
    \item Seien $I$ eine Indexmenge, $\Omega$ eine Menge, $(\Omega_i, \Alg_i), i \in I$ messbare Räume und $f_i : \Omega \to \Omega_i$ Abbildungen, dann ist
    \[ \Alg \coloneqq \Alg\left( \bigcup_{i \in I} f_i^{-1}(\Alg_i) \right) \]
    die kleinste $\sigma$-Algebra auf $\Omega$, sodass alle Abbildungen $f_i$, $i \in I$, messbar sind. Diese $\sigma$-Algebra wird die von der Familie $\Set{ f_i }{i \in I}$ \emph{erzeugte $\sigma$-Algebra} genannt.
  \end{itemize}
\end{satz}

\begin{satz}
  Sei $f : \Omega \to \Omega'$ eine Abbildung und $\mathcal{E}' \subset \mathcal{P}(\Omega')$, dann ist
  \[ \Alg(f^{-1}(\mathcal{E}')) = f^{-1}(\Alg(\mathcal{E}')). \]
\end{satz}

\begin{satz}
  Sei $(\Omega, \Alg)$ ein messbarer Raum und $f : \Omega \to \Omega'$ eine Abbildung, sowie $\mathcal{E'} \subset \mathcal{P}(\Omega')$. Dann gilt:
  \[ f \text{ ist } (\Alg, \Alg(\mathcal{E}')) \text{-messbar} \iff f^{-1}(\mathcal{E}') \subset \Alg \]
\end{satz}

\begin{satz}
  Seien $(\Omega, \mathcal{O})$ und $(\Omega', \mathcal{O}')$ zwei topologische Räume und $\Alg \coloneqq \Alg(\mathcal{O})$ bzw. $\Alg' \coloneqq \Alg(\mathcal{O}')$ die dazugehörigen Borelschen $\sigma$-Algebren. Dann ist jede stetige Abbildung $f : \Omega \to \Omega'$ $(\Alg, \Alg')$-messbar.
\end{satz}

% Bildchen zum Projektionssatz?

\begin{satz}[Projektionssatz]
  Seien $I$ eine Indexmenge, $(\Omega_0, \Alg_0)$ sowie $(\Omega_i, \Alg_i)$, $i \in I$ messbare Räume und $\Omega$ eine Menge. Seien $g_i : \Omega \to \Omega_i, i \in I$ und $f : \Omega_0 \to \Omega$ Abbildungen. Wir setzen $\Alg \coloneqq \Alg\left( \bigcup_{i \in I} g_i^{-1}(\Alg_i) \right)$. Dann sind folgende Aussagen äquivalent:
  \begin{itemize}
    \item $f$ ist $(\Alg_0, \Alg)$-messbar.
    \item Für alle $i \in I$ sind die Abbildungen $g_i \circ f$ $(\Alg_0, \Alg_i)$-messbar.
  \end{itemize}
\end{satz}

\begin{satz}
  Sei $(\Omega, \Alg, \mu)$ ein Maßraum und $(\Omega', \Alg')$ ein messbarer Raum und $f : \Omega \to \Omega'$ eine messbare Abbildung, dann ist
  \[ \mu' = f_*(\mu) = \mu \circ f^{-1} : \Alg' \to [0, \infty], \quad A' \mapsto \mu(f^{-1}(A')) \]
  ein Maß auf $(\Omega', \Alg')$, genannt das \emph{Bildmaß} von $f$.
\end{satz}

\begin{bem}
  Sei $(\Omega, \Alg, \mu)$ ein Maßraum, $(\Omega', \Alg')$ und $(\Omega'', \Alg'')$ messbare Räume und $f : \Omega' \to \Omega''$, $g : \Omega \to \Omega'$ messbare Abbildungen, dann gilt $(f \circ g)_* \mu = f_*(g_* \mu)$.
\end{bem}

\begin{defn}
  Die $\sigma$-Algebra der Borelmengen auf $\ER = \R \cup \{ \pm \infty \}$ ist
  \[ \Bor(\ER) = \Set{ A, A \cup \{ +\infty \}, A \cup \{ -\infty \}, A \cup \{ \pm \infty \} }{ A \in \Bor(\R) }. \]
\end{defn}

\begin{satz}
  $\Bor(\ER) = \Alg(\Set{ [a, \infty] }{ a \in \R })$
\end{satz}

\begin{nota}
  Seien $f, g : \Omega \to \overline{\R}$ zwei numerische Funktionen. Setze
    \[ \{ f \leq g \} \coloneqq \Set{ \omega \in \Omega }{ f(\omega) \leq g(\omega) } \subset \Omega \]
  und definiere analog $\{ f < g \}$, $\{ f \geq g \}$, $\{ f > g \}$, $\{ f = g \}$, $\{ f \not= g \}$.
\end{nota}

\begin{satz}
  Für eine numerische Fkt. $f : (\Omega, \Alg) \to (\ER, \overline{\Bor})$ sind äquivalent:
  \begin{itemize}
    \miniitem{0.27 \linewidth}{$f$ ist messbar}
    \miniitem{0.7 \linewidth}{$\fa{a \in \R} \{ f \geq a \} = f^{-1}([a, \infty]) \in \Alg$}
    \begin{multicols}{2}
      \item $\fa{a \in \R} \{ f > a \} \in \Alg$
      \item $\fa{a \in \R} \{ f \leq a \} \in \Alg$
    \end{multicols}
    \item $\fa{a \in \R} \{ f < a \} \in \Alg$
  \end{itemize}
\end{satz}

\begin{satz}
  Für zwei numerische Funktionen $f, g : (\Omega, \Alg) \to (\ER, \overline{\Bor})$ gilt:
  \begin{multicols}{3}
    \begin{itemize}
      \item $\{ f < g \} \in \Alg$
      \item $\{ f \leq g \} \in \Alg$
      \item $\{ f > g \} \in \Alg$
      \item $\{ f \geq g \} \in \Alg$
      \item $\{ f = g \} \in \Alg$
      \item $\{ f \not= g \} \in \Alg$
    \end{itemize}
  \end{multicols}
\end{satz}

\begin{satz}
  Seien $f, g : (\Omega, \Alg) \to (\ER, \overline{\Bor})$ messbare numerische Funktionen und $\lambda, \mu \in \R$. Dann auch messbar (\ddag: falls $0 \not\in \mathrm{Bild}(f)$):
  \begin{multicols}{5}
    \begin{itemize}
      \item $\lambda \cdot f$
      \item $f + \mu \cdot g$
      \item $f \cdot g$
      \item $\tfrac{1}{f}$ (\ddag)
      \item $\tfrac{g}{f}$ (\ddag)
    \end{itemize}
  \end{multicols}
\end{satz}

\begin{satz}
  Seien $f_n : (\Omega, \Alg) \to (\ER, \overline{\Bor}), n \in \N$ messbare numerische Funktionen, dann auch messbar:
  \begin{multicols}{4}
    \begin{itemize}
      \item $\sup_{n \in \N} f_n$
      \item $\inf_{n \in \N} f_n$
      \item $\liminf_{n \in \N} f_n$
      \item $\limsup_{n \in \N} f_n$
    \end{itemize}
  \end{multicols}
  \vspace{4pt}
  Dabei werden Infimum, Supremum, usw. punktweise gebildet.
\end{satz}

\begin{satz}
  Seien $f_1, ..., f_n : (\Omega, \Alg) \to (\ER, \overline{\Bor})$ messbare numerische Fktn., dann sind auch $\max(f_1, ..., f_n)$ und $\min(f_1, ..., f_n)$ messbar.
\end{satz}

\begin{defn}
  Für $f : \Omega \to \ER$ heißen die Funktionen
  \begin{itemize}
    \item $\left|f\right| \coloneqq \max(f, -f) : \Omega \to [0, \infty]$ \emph{Betrag} von $f$
    \item $f^+ \coloneqq \,\,\,\, \max(f, 0) : \Omega \to [0, \infty]$ \emph{Positivteil} von $f$
    \item $f^- \coloneqq -\min(f, 0) : \Omega \to [0, \infty]$ \emph{Negativteil} von $f$
  \end{itemize}
\end{defn}

\begin{bem}
  $f = f^+ - f^-$ und $\left|f\right| = f^+ + f^-$
\end{bem}

\begin{satz}
  Falls $f : (\Omega, \Alg) \to (\ER, \overline\Bor)$ messbar, dann auch $\left|f\right|$, $f^+$ und $f^-$.
\end{satz}

\subsection{Das Lebesguesche Integral}

\begin{defn}
  Eine Funktion $f : (\Omega, \Alg) \to (\R, \Bor)$ heißt \emph{einfache Funktion} oder \emph{Elementarfunktion} auf $(\Omega, \Alg)$, wenn gilt:
  \begin{multicols}{3}
    \begin{itemize}
      \item $f$ ist messbar
      \item $f(\Omega) \subset \left[0, \infty\right[$
      \item $f(\Omega)$ ist endlich
    \end{itemize}
  \end{multicols}
  Die Menge aller einfachen Funktionen auf $(\Omega, \Alg)$ ist $\E(\Omega, \Alg)$.
\end{defn}

\begin{defn}
  Sei $f \in \E(\Omega, \Alg)$ und $\Omega = A_1 \sqcup ... \sqcup A_k$ eine disjunkte Vereinigung von Mengen mit $A_j \in \Alg$ für alle $j = 1, ..., k$, sodass $f(A_j) = \{ y_j \}$, dann heißt die Darstellung
  \[ f = \sum_{j=1}^k y_j \cdot \ind_{A_j} \quad \text{\emph{kanonische Darstellung}.} \]
\end{defn}

\begin{bem}
  Die kanonische Darstellung ist nicht eindeutig.
\end{bem}

\begin{satz}
  Seien $f, g \in \E(\Omega, \Alg)$ und $a \geq 0$. Dann auch in $\E(\Omega, \Alg)$:
  \begin{multicols}{5}
    \begin{itemize}
      \item $f + g$
      \item $f \cdot g$
      \item $\max(f, g)$
      \item $\min(f, g)$
      \item $a \cdot f$
    \end{itemize}
  \end{multicols}
\end{satz}

\begin{defn}
  Sei $f \in \E(\Omega, \Alg)$ und $\textstyle f = \sum_{j=1}^k y_j \ind_{A_j}$ eine kanonische Darstellung von $f$. Sei ferner $\mu$ ein Maß auf $\Alg$. Dann heißt die Größe
  \[ \IntOmu{f} \coloneqq \sum_{j=1}^k y_j \mu(A_j) \quad \text{\emph{Lebesgue-Integral} von $f$ bzgl. $\mu$.} \]
\end{defn}

\begin{bem}
  Obige Größe ist wohldefiniert, d.\,h. unabhängig von der kanonischen Darstellung.
\end{bem}

\begin{satz}
  Seien $f, g \in \E(\Omega, \Alg)$, $\mu$ ein Maß auf $\Alg$ und $\alpha \geq 0$, dann gilt
  \begin{itemize}
    \item $\IntOmu{(\alpha \cdot f + g)} = \alpha \cdot \IntOmu{f} + \IntOmu{g}$ \pright{Linearität}
    \item Falls $g \leq f$, dann $\IntOmu{g} \leq \IntOmu{f}$ \pright{Monotonie}
  \end{itemize}
\end{satz}

\begin{satz}
  Angenommen, die Funktionen $f_n \in \E(\Omega, \Alg, \mu), n \in \N$ bilden eine monoton wachsende Funktionenfolge und für $g \in \E(\Omega, \Alg)$ gilt $g \leq \sup_{n \in \N} f_n$, dann gilt $\textstyle \IntOmu{g} \leq \sup\limits_{n \in \N} \, \IntOmu{f_n}$.
\end{satz}

\begin{kor}
  Seien $f_n, g_n \in \E(\Omega, \Alg), n \in \N$ und die Funktionenfolgen $f_n$ und $g_n$ monoton wachsend mit $\sup_{n \in \N} f_n = \sup_{n \in \N} g_n$. Dann gilt
  \[ \sup_{n \in \N} \IntOmu{f_n} = \sup_{n \in \N} \IntOmu{g_n}. \]
\end{kor}

\begin{defn}
  Sei $\overline{\E}(\Omega, \Alg)$ die Menge aller Funktionen $f : \Omega \to \ER$, die Grenzfunktionen (pktw. Konvergenz) monoton wachsender Funktionenfolgen in $\E(\Omega, \Alg)$ sind.
\end{defn}

\begin{defn}
  Für eine Funktion $f \in \overline{\E}(\Omega, \Alg)$ (d.\,h. es existiert eine Folge $(g_n)_{n \in \N}$ in $\E(\Omega, \Alg)$ mit $f = \sup_{n \in \N} g_n$) und ein Maß $\mu$ auf $\Alg$ heißt
  \[ \IntOmu{f} \coloneqq \sup_{n \in \N} \IntOmu{g_n} \quad \text{\emph{Lebesgue-Integral} von $f$ bzgl. $\mu$.} \]
\end{defn}

\begin{satz}
  $\overline{\E}(\Omega, \Alg) = \Set{ f : (\Omega, \Alg) \to (\ER, \Bor) }{ f \text{ messbar und } f \geq 0 }$
\end{satz}

\begin{satz}
  Die Eigenschaften des Integrals für einfache Funktionen (Linearität, Monotonie) übertragen sich auf das Lebesgue-Integral.
\end{satz}

\begin{satz}[Satz von der monotonen Konvergenz]
  Sei $(f_n)_{n \in \N}$ eine monoton wachsende Folge von Funktionen in $\overline{\E}(\Omega, \Alg)$, dann gilt für $f \coloneqq \lim_{n \to \infty} f_n = \sup_{n \in \N} f_n \in \overline{\E}(\Omega, \Alg)$ und jedes Maß $\mu$ auf $\Alg$:
  \[ \lim_{n \to \infty} \IntOmu f_n = \sup_{n \in \N} \IntOmu{f_n} = \IntOmu{f} \]
\end{satz}

\begin{bem}
  Die Aussage ist für monoton fallende Fktn. i.\,A. falsch.
\end{bem}

\begin{defn}
  Eine messbare Funktion $f : (\Omega, \Alg) \to (\ER, \overline{\Bor})$ heißt \emph{integrierbar} bzw. $\mu$-integrierbar (im Sinne von Lebesgue), falls
  \[ \IntOmu{f^+} < \infty \quad \text{und} \quad \IntOmu{f^-} < \infty. \]
  In diesem Fall definieren wir das \emph{Lebesgue-Integral} von $f$ als
  \[ \IntOmu{f} \coloneqq \IntOmu{f^+} - \IntOmu{f^-}. \]
\end{defn}

\begin{nota}
  $\Leb^1(\Omega, \Alg, \mu) = \Leb^1(\mu)$ bezeichnet die Menge der $\mu$-integrierbaren Funktionen auf $\Omega$.
\end{nota}

\begin{satz}
  Für eine messbare Fkt. $f : (\Omega, \Alg, \mu) \to (\ER, \overline{\Bor})$ sind äquivalent:
  \begin{itemize}
    \begin{multicols}{2}
      \item $f \in \Leb^1(\Omega, \Alg, \mu)$.
      \item $f^+, f^- \in \Leb^1(\Omega, \Alg, \mu)$.
      \item $\left|f\right| \in \Leb^1(\Omega, \Alg, \mu)$.
      \item $\exists\,g \in \Leb^1(\Omega, \Alg, \mu)$ mit $\left|f\right| \leq g$.
    \end{multicols}
    \item Es gibt nicht negative $u, v \in \Leb^1(\Omega, \Alg, \mu)$ mit $f = u - v$.
  \end{itemize}
  Im letzten Fall gilt $\IntOmu{f} = \IntOmu{u} - \IntOmu{v}$.
\end{satz}

\begin{satz}
  \begin{itemize}
    \item $\Leb^1(\Omega, \Alg, \mu)$ ist ein $\R$-VR und die Abbildung
    \[ {\textstyle \int} : \Leb^1(\Omega, \Alg, \mu) \to \R, \quad f \mapsto \IntOmu{f} \quad \text{ist linear.} \]
    \item $f, g \in \Leb^1(\Omega, \Alg, \mu) \implies \max(f, g), \, \min(f, g) \in \Leb^1(\Omega, \Alg, \mu)$
    \item $f, g \in \Leb^1(\Omega, \Alg, \mu), \, f \leq g \implies \IntOmu{f} \leq \IntOmu{g}$ \pright{Monotonie}
    \item $\abs{\IntOmu{f}} \leq \IntOmu{\abs{f}}$ für alle $f \in \Leb^1(\Omega, \Alg, \mu)$ \pright{$\triangle$-Ungleichung}
  \end{itemize}
\end{satz}

\begin{defn}
  Sei $(\Omega, \Alg, \mu)$ ein Maßraum, $A \in \Alg$ und $f \in \overline{\E}(\Omega, \Alg, \mu)$ oder $f \in \Leb^1(\Omega, \Alg, \mu)$. Dann ist das \emph{$\mu$-Integral von $f$ über $A$}
  \[ \Int{A}{}{f}{\mu} = \IntOmu{(\ind_A \cdot f)}. \]
\end{defn}

\begin{defn}
  Ein Maßraum $(\Omega, \Alg, \mu)$ heißt \emph{vollständig}, wenn jede Nullmenge $N \subset \Omega$ in $\Alg$ liegt.
\end{defn}

\begin{defn}
  Sei $(\Omega, \Alg, \mu)$ ein Maßraum. Setze
  \begin{align*}
    \tilde{\mathfrak{N}}_\mu &\coloneqq \Set{N \subset \Omega}{N \text{ ist $\mu$-Nullmenge }}, \\
    \tilde{\Alg}_\mu &\coloneqq \Set{ A \cup N }{ A \in \Alg, \, N \in \tilde{\mathfrak{N}}_\mu }.
  \end{align*}
  Dann ist $\tilde{\Alg}_\mu$ eine $\sigma$-Algebra und mit $\tilde{\mu}(A \cup N) \coloneqq \mu(A)$ ist $(\Omega, \tilde{\Alg}_\mu, \tilde{\mu})$ ein Maßraum, genannt \emph{Vervollständigung} von $(\Omega, \Alg, \mu)$.
\end{defn}

\begin{defn}
  Sei $(\Omega, \Alg, \mu)$ ein Maßraum und $E(\omega)$ eine Aussage für alle $\omega \in \Omega$. Man sagt, $E$ ist \emph{($\mu$)-fast-überall wahr}, wenn $\Set{\omega \in \Omega}{\neg E(\omega)}$ eine Nullmenge ist.

  Zwei Funktionen $f, g : \Omega \to X$ heißen \emph{($\mu$-)fast-überall gleich}, notiert $f \overset{\text{f.ü.}}= g$, wenn $\Set{\omega \in \Omega}{f(\omega) \not= g(\omega)}$ eine Nullmenge ist.

  Eine Funktion $f : \Omega \to \ER$ heißt \emph{($\mu$)-fast-überall} endlich, wenn $\Set{\omega \in \Omega}{ f(\omega) = \infty}$ eine Nullmenge ist.
\end{defn}

\begin{bem}
  Das Cantorsche Diskontinuum ist eine Menge $C \subset [0, 1]$, $C \in \Bor$, welche die bemerkenswerte Eigenschaft hat, dass sie gleichzeitig überabzählbar ist und Maß $0$ besitzt. Da außerdem $\Bor \cong \R$ gilt, folgt $\PS(C) \cong \PS(\R) \not\cong \R \cong \Bor$. Somit gibt es eine Nullmenge $N \subset C$, die nicht in $\Bor$ liegt. Es folgt:
\end{bem}

\begin{satz}
  Der Maßraum $(\R, \Bor, \mu)$ ist nicht vollständig.
\end{satz}

\begin{defn}
  Sei $(\R^n, \Bor_L^n, \lambda)$ die Vervollständigung von $(\R^n, \Bor^n, \mu$, dann heißt $\Bor_L$ die \emph{Lebesguesche $\sigma$-Algebra} und $\lambda$ das \emph{Lebesgue-Maß} auf $\R^n$ (analog: $(\ER, \overline{\Bor}, \lambda)$).
\end{defn}

\begin{satz}
  Sei $f \in \overline{\E}(\Omega, \Alg, \mu)$, dann gilt $\IntOmu{f} = 0 \iff f \overset{\text{f.ü.}}= 0$.
\end{satz}

\vspace{-10pt}

\begin{satz}
  Seien $f, g : (\Omega, \Alg, \mu) \to (\ER, \overline{\Bor})$ messbar mit $f \fue= g$, dann gilt:
  \begin{itemize}
    \item Wenn $f, g \in \overline{\E}(\Omega, \Alg)$, dann $\IntOmu{f} = \IntOmu{g}$.
    \item Wenn $f \in \Leb^1(\Omega, \Alg, \mu)$, dann $g \in \Leb^1(\Omega, \Alg, \mu)$ mit $\IntOmu{f} = \IntOmu{g}$.
  \end{itemize}
\end{satz}

\begin{satz}
  Sei $f : (\Omega, \Alg, \mu) \to (\ER, \overline{\Bor})$ eine messbare Fkt. und $g \in \Leb^1(\Omega, \Alg, \mu), g \geq 0$. Wenn $f \fue\leq g$, dann gilt $f \in \Leb^1(\Omega, \Alg, \mu)$.
\end{satz}

% \begin{doublespace} ... \end{doublespace} für einheitlichen Zeilenabstand.
% Sonst Kraut und Rüben mit den ganzen Inline-Integralen.
\begin{satz}[Lemma von Fatou]\begin{doublespace}
  Sei $(f_n)_{n \in \N}$ eine Funktionenfolge mit $f_n$ $\mu$-integrierbar und $f_n \overset{\text{f.ü.}}\geq 0$. Dann $\IntOmu{(\liminf_{n \to \infty} f_n)} \leq \liminf_{n \to \infty} \IntOmu{fn}$.
\end{doublespace}\end{satz}

\vspace{-20pt}

\begin{satz}
  \begin{doublespace}
    Sei $(f_n)_{n \in \N}$ Folge messbarer Fkt. $f_n : (\Omega, \Alg, \mu) \to (\ER, \overline{\Bor})$ und $g \in \Leb^1(\Omega, \Alg, \mu), g \geq 0$, sodass $\fa{n \in \N} \abs{f_n} \overset{\text{f.ü.}}\leq g$. Dann:\\[-16pt]
  \end{doublespace}
  \begin{align*}
    \IntOmu{(\liminf_{n \to \infty} (f_n))}
    &\leq \liminf_{n \to \infty} (\IntOmu{f_n}) \,\,\leq\\[-5pt]
    &\leq \limsup_{n \to \infty} (\IntOmu{f_n})
    \leq \IntOmu{(\limsup_{n \to \infty} f_n)}.
  \end{align*}
\end{satz}

\begin{satz}[von der majorisierten Konvergenz]\begin{doublespace}
  Sei $g \in \Leb^1(\Omega, \Alg, \mu), g \geq 0$. Sei $(f_n)_{n \in \N}$ Folge messbarer Fkt. $f_n : (\Omega, \Alg, \mu) \to (\ER, \overline{\Bor})$ mit $\left|f_n\right| \overset{\text{f.ü.}}\leq g$ (Majorisierung).
  Sei ferner $f : \Omega \to \ER$ $(\Alg, \overline{\Bor})$-messbar mit $f_n \xrightarrow[n \to \infty]{\text{f.ü.}} f$, d.\,h. $\Set{ \omega \in \Omega }{ \lim_{n \to \infty} f_n(\omega) = f(\omega) \text{ falsch} }$ ist Nullmenge. Dann ist $f$ integrierbar mit $\IntOmu{f} = \lim_{n \to \infty} \IntOmu{f_n}$.
\end{doublespace}\end{satz}

\vspace{-20pt}

\begin{satz}
  \begin{doublespace}
    Sei $f \in \overline{\E}(\Omega, \Alg, \mu)$ bzw. $f \in \Leb^1(\Omega, \Alg, \mu)$, $(A_n)_{n \in \N}$ Folge in $\Alg$, $A_n \cap A_m = \emptyset$ für $n \not= m$, $A = \bigsqcup_{n=1}^\infty A_n$. Dann:\\[-10pt]
  \end{doublespace}
  \[ \Int{A}{}{f}{\mu} \coloneqq \IntOmu{f \cdot \ind_A} = \sum_{n=1}^\infty \left( \Int{A_n}{}{f}{\mu} \right). \]
\end{satz}

\begin{satz}\begin{doublespace}
  Seien $f, f_j : (\Omega, \Alg, \mu) \to (\R, \Bor), j \in \N$ messbare Funktionen, $g : (\Omega, \Alg, \mu) \to (\R, \Bor)$ integrierbar, sodass $\left|\sum_{j=1}^n f_j\right| \overset{\text{f.ü.}}\leq g \, \forall n \in \N$ und\\[-5pt]
  $f \fue= \sum_{n=1}^\infty f_j$. Dann sind $f, f_j$ integrierbar mit $\IntOmu{f} = \sum_{j=1}^\infty \IntOmu{f_j}$.
\end{doublespace}\end{satz}

\vspace{-20pt}

\begin{satz}[Ableiten unter Integral]
  Seien $a, b \in \R$ mit $a < b$, $(\Omega, \Alg, \mu)$ ein Maßraum und sei $f : \left]a, b\right[ \times \Omega \to (\R, \Bor)$ eine Funktion, sodass:
  \begin{itemize}
    \item Für alle $t \in \left]a, b\right[$ ist die Abbildung $f(t, -) : \Omega \to \R$ $\mu$-integrierbar.
    \item Für alle $\omega \in \Omega$ ist die Abbildung $f(-, \omega) : \left]a, b\right[ \to \R$ diff'bar.
    \item Es gibt eine Funktion $g \in \Leb^1(\Omega, \Alg, \mu)$ mit $g \geq 0$, sodass für alle $t \in \left]a, b\right[$ und fast alle $\omega \in \Omega$ gilt: $\left|f(-, \omega)'(t)\right| \leq g(\omega)$.
  \end{itemize}
  Dann ist die Funktion $F : \left]a, b\right[ \to \R, t \mapsto \IntOmu{f_t}$ differenzierbar mit\\[-3pt]
  $F'(t) = \IntOmu{h_t}$, wobei $h_t : \Omega \to \R, \, \omega \mapsto f(-, \omega)'(t)$.
\end{satz}

\begin{samepage}

\begin{satz}
  Sei $f \in \overline{\E}(\Omega, \Alg, \mu)$. Dann ist die Abbildung
  \[ f \mu : \Alg \to [0, \infty], \quad A \mapsto \Int{A}{}{f}{\mu} \]
  ein Maß, genannt \emph{Maß mit der Dichte $f$ bzgl. $\mu$} oder \emph{Stieltjes-Maß} zu $f$.
\end{satz}

\subsection{Zusammenhang mit dem Riemann-Integral}

\end{samepage}

\begin{defn}
  Eine \emph{Zerlegung} eines Intervalls $[a, b]$ ist eine geordnete endliche Teilmenge $\{ a = a_0 < a_1 < ... < a_k = b\} \subset [a, b]$.
\end{defn}

\begin{nota}
  Die Menge aller Zerlegungen von $[a, b]$ ist $\mathcal{Z}([a, b])$.
\end{nota}

\begin{defn}
  Die \emph{Feinheit} einer Zerlegung $\{ a_0, ..., a_n \} \in \mathcal{Z}([a, b])$ ist
  \[ \left|Z\right| \coloneqq \max \Set{ x_j - x_{j-1} }{ j \in \{ 1, ..., n \} }. \]
\end{defn}

\begin{defn}
  Für eine beschränkte Funktion $f : [a, b] \to \R$ und eine Zerlegung $Z = \{ a_0, ..., a_n \} \in \mathcal{Z}([a, b])$ bezeichnen
  \begin{align*}
    O(f, Z) &\coloneqq \sum_{j=1}^n (\sup \Set{ f(x) }{ x \in [x_{j-1}, x_j] })(x_j - x_{j-1}),\\
    U(f, Z) &\coloneqq \sum_{j=1}^n (\inf \Set{ f(x) }{ x \in [x_{j-1}, x_j] })(x_j - x_{j-1})
  \end{align*}
  die \emph{(Darbouxschen) Ober- und Untersummen} von $f$ bzgl. $Z$.
\end{defn}

\begin{nota}
  \begin{align*}\\[-24pt]
    O_*(f) &\coloneqq \inf \Set{ O(f, Z) }{ Z \in \mathcal{Z}([a, b]) }\\
    U^*(f) &\coloneqq \sup \Set{ U(f, Z) }{ Z \in \mathcal{Z}([a, b]) }\\
  \end{align*}
\end{nota}

\begin{defn}
  Eine beschränkte Funktion $f : [a, b] \to \R$ heißt \emph{Riemann-integrierbar}, wenn $O_*(f) = U^*(f)$. In diesem Fall heißt
  \[ \Int{a}{b}{f(x)}{x} \coloneqq O_*(f) = U^*(f) \quad \text{\emph{Riemann-Integral} von $f$.} \]
\end{defn}

\begin{nota}
  Sei $(Z_k)_{k \in \N}$ eine Folge in $\mathcal{Z}([a, b])$ mit $Z_k = \{ a_0^k, a_1^k, ..., a^k_{n_k} \}$. Für eine beschränkte Funktion $f : [a, b] \to \R$ definieren wir $f^k, f_k, f^*, f_* : [a, b] \to \R$ durch
  \begin{align*}
    f^k &= \sup f([a, a_1^k]) \cdot \ind_{[a, a_1^k]} + \sum_{j=2}^{n_k} \sup f([a_{j-1}^k, a_j^k]) \cdot \ind_{]a_{j-1}^k, a_j^k]},\\
    f_k &= \inf \, f([a, a_1^k]) \cdot \ind_{[a, a_1^k]} + \sum_{j=2}^{n_k} \inf \, f([a_{j-1}^k, a_j^k]) \cdot \ind_{]a_{j-1}^k, a_j^k]}\\
    f^*(x) &= \liminf_{y \to x}\,\, f(y) = \lim_{\epsilon \downarrow 0} \inf\,\, \Set{ f(y) }{ y \in [x-\epsilon, x+\epsilon] \cap [a, b]}\\
    f^*(x) &= \limsup_{y \to x} f(y) = \lim_{\epsilon \downarrow 0} \sup \, \Set{ f(y) }{ y \in [x-\epsilon, x+\epsilon] \cap [a, b]}
  \end{align*}
\end{nota}

\begin{bem}
  Es gilt: $f_* \leq f \leq f^*$ und $f_*(x_0) = f(x_0) = f^*(x_0)$ für $x_0 \in [a, b]$ genau dann, wenn $f$ in $x_0$ stetig ist.
\end{bem}

\begin{satz}
  Sei $f : [a, b] \to \R$ beschränkt und $(Z_k)_{k \in \N}$ eine Folge in $\mathcal{Z}([a, b])$ mit $\lim_{n \to \infty} \left|Z_k\right| = 0$. Dann gilt:
  \begin{itemize}
    \item Sei $R = \bigcup_{k=1}^\infty \bigcup_{j=1}^{n_k} \{ a_j^k \}$ die Vereinigung aller Zerlegungen $Z_k, k \in \N$. Für alle $x \in [a, b] \setminus R$ gilt dann
    \[
      \lim_{k \to \infty} f^k(x) = f^*(x)
      \quad \text{und} \quad
      \lim_{k \to \infty} f_k(x) = f_*(x).
    \]
    \item Die Funktionen $f^*$ und $f_*$ sind Borel-messbar und integrierbar bzgl. des Borel-Maßes $\mu$ mit
    \[ \Int{[a, b]}{}{f^*}{\mu} = O_*(f) \quad \text{und} \quad \Int{[a, b]}{}{f_*}{\mu} = O^*(f). \]
  \end{itemize}
\end{satz}

\begin{satz}
  Sei $f : [a, b] \to \R$ beschränkt. Dann sind äquivalent:
  \begin{itemize}
    \item $f$ ist Riemann-integrierbar.
    \item $f$ ist fast-überall stetig (im Sinne des Lebesgue-Borel-Maßes).
  \end{itemize}
\end{satz}

\begin{satz}
  Ist eine beschränkte Funktion $f : [a, b] \to \R$ Riemann-integrierbar, so ist sie auch auf $[a, b]$ Lebesgue-integrierbar bzgl. dem Lebesgue-Maß $\lambda$ und es gilt
  \[ \Int{a}{b}{f(x)}{x} = \Int{[a, b]}{}{f}{\lambda}. \]
\end{satz}

\begin{samepage}
\begin{satz}
  Sei $I$ ein Intervall und $f : I \to \R$ über jedem kompakten Teilintervall von $I$ Riemann-integrierbar. Dann sind äquivalent:
  \begin{itemize}
    \item $\left|f\right|$ ist auf $I$ uneigentlich Riemann-integrierbar.
    \item $f$ ist auf $I$ Lebesgue-integrierbar.
  \end{itemize}
  Falls eine der Bedingungen erfüllt ist, so stimmt das Riemann- Integral von $f$ auf $I$ mit dem Lebesgue-Integral von $f$ auf $I$ überein.
\end{satz}


\subsection{Miscellanea}
\end{samepage}

\begin{satz}
  Sei $f : [a, b] \to \R$ Lebesgue-integrierbar. Dann ist $F : [a, b] \to \R, t \mapsto \Int{[a, t]}{}{f}{\lambda}$ stetig.
\end{satz}

\begin{satz}
  Sei $f : [a, b] \to \R$ Lebesgue-integrierbar. Wenn $\forall \, t \in [a, b]$ gilt: $\Int{[a, t]}{}{f}{\lambda} = F(t) = 0$, dann $f \fue= 0$.
\end{satz}

\begin{nota}
  Sei $f : \R \to \R$ eine Abbildung, dann setzen wir
  \begin{align*}
    C(f) &= \Set{ x \in \R }{ f \text{ stetig in } x } \text{ und}\\
    D(f) &= \Set{ x \in \R }{ f \text{ unstetig in } x } = \R \setminus C(f).
  \end{align*}
\end{nota}

\begin{defn}
  Sei $A \subset \R$, $A$ heißt
  \begin{itemize}
    \item \emph{$G_{\delta}$-Menge}, wenn gilt: $A = \cap_{n \in \N}\,O_n, \, O_n \opn \R\ \forall\,n \in \N$
    \item \emph{$F_{\sigma}$-Menge}, wenn gilt: $A = \cup_{n \in \N}\,F_n, \,\, F_n \, \cls \R\ \forall\,n \in \N$
  \end{itemize}
\end{defn}

\begin{bem}
  $A $ ist $G_\delta$-Menge $\iff$ $A^C$ ist $F_\sigma$-Menge.
\end{bem}

\begin{satz}[Young]
  Sei $f : \R \to \R$ eine beliebige Abbildung. Dann ist $C(f)$ eine $G_\delta$-Menge (und somit $D(f)$ eine $F_\sigma$-Menge).
\end{satz}

\begin{kor}
  Es gibt keine Abbildung $f : \R \to \R$ mit $D(f) = \R \setminus \Q$.
\end{kor}

\begin{defn}
  Ein Maß $\mu$ auf $\Bor(\R^d)$ heißt \emph{translationsinvariant}, wenn für jedes $v \in \R^d$ gilt $(T_v)_* \mu = \mu$, wobei $T_v : \R^d \to \R^d$, $x \mapsto x + v$ die Translation um den Vektor $v$ bezeichnet.
\end{defn}

\begin{nota}
  Bezeichne mit $\mu_{LB}$ das Borel-Lebesgue-Maß auf $\R^d$.
\end{nota}

\begin{nota}
  Der Einheitswürfel im $\R^d$ ist $W_1 \coloneqq \left](0, ..., 0), (1, ..., 1)\right]$.
\end{nota}

\begin{satz}
  Ist $\mu$ ein translationsinvariantes Maß auf $\Bor(\R^d)$ mit $\alpha \coloneqq \mu(W_1) < \infty$, dann gilt $\mu = \alpha \cdot \mu_{LB}$.
\end{satz}

\begin{satz}
  Sei $A \in \mathrm{GL}_d(\R) = \Set{ A \in \R^{d \times d} }{ \det A \not= 0 }$, dann gilt
  \[ A_* \mu_{LB} = \tfrac{1}{\left|\det(A)\right|} \cdot \mu_{LB}. \]
\end{satz}

\begin{satz}
  Das Lebesgue-Borel-Maß $\mu_{LB}$ ist invariant unter Transformationen in $\mathrm{SL}_n(\R)$. Ferner ist $\mu_{LB}$ invariant unter Euklidischen Bewegungen.
\end{satz}

\begin{satz}[Kurt Hensel]
  Sei $\Phi : \mathrm{GL}_n(\R) \to (\R \setminus \{0\}, \cdot)$ ein Gruppenhomomorphismus, dann gibt es einen Gruppen- automorphismus $\phi : (\R \setminus \{0\}, \cdot) \to (\R \setminus \{0\}, \cdot)$, sodass $\Phi = \phi \circ \det$.
\end{satz}

\begin{satz}
  Sei $(\Omega, \Alg, \mu)$ ein Maßraum und $h \in \overline{\E}(\Omega, \Alg)$. Eine messbare Funktion $f : \Omega \to \overline{\R}$ ist genau dann $h\mu$-integrierbar, wenn $(f \cdot h)$ $\mu$-integrierbar ist. In diesem Fall gilt
  \[ \IntO{f}{(h \mu)} = \IntOmu{f \cdot h}. \]
  Obige Gleichung ist auch erfüllt, wenn lediglich $f \in \overline{\E}(\Omega, \Alg)$ gilt.
\end{satz}

\begin{bem}
  Somit ist $g (h \mu) = (g \cdot h) \mu$.
\end{bem}

\begin{satz}
  Sei $U, \widetilde{U} \opn \R^d$, $\phi : U \to \widetilde{u}$ ein $\mathcal{C}^1$-Diffeomorphismus, dann gilt:
    \[ \phi_*^{-1} \mu_{LB}|_{\widetilde{U}} = \underbrace{\left|\det(D \phi)\right|}_{U \to \R_{> 0} \text{ stetig}} \mu_{LB}|_U \]
\end{satz}

% Lemma
\begin{satz}
  Sei $U, \widetilde{U} \opn \R^d$, $\phi : U \to \widetilde{u}$ ein $\mathcal{C}^1$-Diffeomorphismus und $Q = \left]a, b\right[ \subset U$ Elementarquader mit $a \lhd b$, dann gilt
  \[ \mu_{LB}(Q) \cdot \inf_{q \in Q} \left|\det D \phi(q)\right| \leq \mu_{LB}(\phi(Q)) \leq \mu_{LB}(Q) \cdot \sup_{q \in Q} \left|\det(D(\phi(q)))\right|. \]
\end{satz}

\begin{satz}[Transformationssatz]
  Sei $U, \widetilde{U} \opn \R^d$ und sei $\phi : U \to \widetilde{U}$ ein $\mathcal{C}^1$-Diffeomorphismus. Dann ist eine Funktion $f : \widetilde{U} \to \ER$ genau dann auf $\widetilde{U}$ Lebesgue-Borel-integrierbar, wenn $(f \circ \phi) \cdot \left|\det(D\phi)\right| : U \to \ER$ auf $U$ Lebesgue-Borel-interierbar ist. In diesem Fall gilt
  \[ \Int{U}{}{(f \circ \phi) \cdot \left|\det(D\phi)\right|}{\mu_{LB}} = \Int{\phi(U)}{}{f}{\mu_{LB}} = \Int{\widetilde{U}}{}{f}{\mu_{LB}}. \]
  Obige Gleichung ist auch erfüllt, wenn lediglich $f \in \overline{\E}(\widetilde{U}, \Bor(\widetilde{U}))$ gilt.
\end{satz}

\begin{bem}
  Im Transformationssatz kann man "`Lebesgue-Borel"' durch "`Lebesgue"' ersetzen.
\end{bem}


% Produktstrukturen

% Ziel: $(\Omega_j, \Alg_j, \mu_j)$ Maßräume $j = 1, ..., n$.
% Betrachte: \Omega_1 \times ... \times \Omega_n = \Omega
% 1. Definiere $\sigma$-Algebra auf $\omega$, die die $\Alg_j$ "`respektiert"'.
% 2. Definiere Maß auf $\widetilde{A}$, welches von $\mu_j$ "`induziert"' wird.
% 
% Genauer:
% Zu 1.: Wir haben kanonische Projektionen
%   \Pi_j : \Omega = \Omega_1 \times ... \times \Omega_n \to \Omega_j
%           (\omega_1, ..., \omega_n) \mapsto \omega_j

\begin{defn}
  Seien $(\Omega_j, \Alg_j, \mu_j)$ Maßräume für $j = 1, ..., n$. Die kleinste $\sigma$-Algebra $\Alg$ auf $\sigma$, sodass alle $\Pi_j, j = 1, ..., n$ $(\Alg, \Alg_j)$-messbar sind, heißt \emph{Produkt} der $\sigma$-Algebren $\Alg_1, ..., \Alg_n$, notiert $\Alg =: \Alg_1 \otimes ... \otimes \Alg_n$.
\end{defn}

\begin{satz}
  Sei $\mathcal{E}_j$ Erzeugendensystem von $\Alg_j, j = 1, ..., n$, d.\,h. $\Alg(\mathcal{E}_j) = \Alg_j$. Annahme: Für alle $j \in \{ 1, ..., n \}$ gibt es eine monoton gegen $\Omega_j$ wachsende Folge $(E^j_k)_{k \in \N}$ in $\mathcal{E}_j$. Dann ist
  \begin{align*}
    \Alg_1 \otimes ... \otimes \Alg_n &= \Alg(\mathcal{E}_1 \times ... \times \mathcal{E}_n) \text{ mit }\\
    \mathcal{E}_1 \times ... \times \mathcal{E}_n &= \Set{ E_1 \times ... \times E_n }{ E_j \in \mathcal{E}_j, j = 1, ..., n }.
  \end{align*}
\end{satz}

% Korollar
\begin{satz}
  $\Bor(\R^n) = \underbrace{\Bor(\R) \otimes ... \otimes \Bor(\R)}_{\text{$n$-mal}}$.
\end{satz}

% Ziel: Maß \mu auf \Omega mit
% $\mu(E_1 \times ... \times E_n) = \mu_1(E_1) \cdot ... \cdot \mu_n(E_n)$ für alle $E_j \in \mathcal{E}_j$.

\begin{satz}[Eindeutigkeit von Produktmaßen]
  Seien $(\Omega_j, \Alg_j, \mu_j)$ Maßräume und $E_j$ ein Erzeugendensystem von $\Alg_j$ für $j = 1, ..., n$. Angenommen, $E_j$ ist stabil unter Schnitten und $\exists (E_k^{(j)})_{k \in \N} \uparrow \Omega_j$ mit $\mu_j(E_k^{(j)}) < \infty$ für alle $j$.
  Dann gibt es höchstens ein Maß $\nu : \Alg_1 \otimes ... \otimes \Alg_n \to [0, \infty]$, sodass für alle $E_j \in \mathcal{E}_j, j \in \{ 1, ..., n \}$ gilt:
  \[ \nu(E_1 \times ... \times E_n) = \mu_1(E_1) \cdot ... \cdot \mu_n(E_n). \]
\end{satz}

\begin{defn}
  Sei $\Omega$ eine Menge. Eine Teilmenge $\Dyn \subset \Omega$ heißt \emph{Dynkin-System}, wenn gilt:
  \begin{itemize}
    \miniitem{0.2 \linewidth}{$\Omega \in \Dyn$}
    \miniitem{0.7 \linewidth}{$D \in \Dyn \implies D^C = \Omega \setminus D \in \Dyn$}
    \item $(D_n)_{n \in \N}$ Folge pw. disjunkter Mengen in $\Dyn$, dann: $\bigcup_{n \in \N} D_n \in \Dyn$
  \end{itemize}
\end{defn}

\iffalse
\begin{aufg}
  Zeigen Sie:
  \begin{itemize}
    \item Die zweite Forderung kann durch die Forderung
    \[ (D_1, D_2 \in \mathfrak{D}, D_2 \subset D_1) \implies (D_1 \setminus D_2 \in \mathfrak{D}) \]
    \item Es sind äquivalent:
      \begin{itemize}
        \item $\Dyn$ ist ein unter $\bigcap$ stabiles Dynkin-System
        \item $\Dyn$ ist $\sigma$-Algebra
      \end{itemize}
    \item $\mathcal{E} \subset \PSO, \mathcal{E}$ stabil unter $\bigcap$, dann: $\Dyn(\mathcal{E}) = \Alg(\mathcal{E})$, wobei $\Dyn(\mathcal{E})$ das von $\mathcal{E}$ erzeugte Dynkin-System bezeichnet.
  \end{itemize}
\end{aufg}
\fi

\begin{nota}
  Seien $\Omega_1, \Omega_2$ Mengen, $\Omega \subset \Omega_1 \otimes \Omega_2, \omega_1 \in \Omega_1, \omega_2 \in \Omega_2$
  \begin{align*}
    Q_{\omega_1} &\coloneqq \Set{ \omega_2 \in \Omega_2 }{ (\omega_1, \omega_2) \in Q } = \pi_2(\pi_1^{-1}(\{ \omega_1 \}) \cap Q)\\
    Q_{\omega_2} &\coloneqq \Set{ \omega_1 \in \Omega_1 }{ (\omega_1, \omega_2) \in Q } = \pi_1(\pi_2^{-1}(\{ \omega_2 \}) \cap Q)
  \end{align*}
\end{nota}

\begin{satz} % Lemma
  $Q \subset \Alg_1 \otimes \Alg_2, \omega_1 \in \Omega_1, \omega_2 \in \Omega_2$
  $\implies Q_{\omega_1} \in \Alg_2, Q_{\omega_2} \in \Alg_1$.
\end{satz}

\begin{satz}(Cavalieri 1) % Proposition
  Seien $(\Omega_1, \Alg_1, \mu_1)$ und $(\Omega_2, \Alg_2, \mu_2)$ $\sigma$-endliche Maßräume, $Q \in \Alg_1 \otimes \Alg_2$. Dann:
  \begin{itemize}
    \item $h_Q^1 : \Omega_1 \to [0, \infty], \enspace \omega_1 \mapsto \mu_2(Q_{\omega_1})$ \enspace ist $(\Alg_1, \overline{\Bor})$-messbar.
    \item $h_Q^2 : \Omega_2 \to [0, \infty], \enspace \omega_2 \mapsto \mu_1(Q_{\omega_2})$ \enspace ist $(\Alg_2, \overline{\Bor})$-messbar.
  \end{itemize}
\end{satz}

\begin{satz}[Existenz von Produktmaßen]
  Die Abbildungen
  \begin{align*}
    \nu_1 : \Alg_1 \otimes \Alg_2 \to [0, \infty], \quad & Q \mapsto \Int{\Omega_1}{}{\mu_2(Q \omega_1)}{\mu_1} \\
    \nu_2 : \Alg_2 \otimes \Alg_1 \to [0, \infty], \quad & Q \mapsto \Int{\Omega_2}{}{\mu_1(Q \omega_2)}{\mu_2}
  \end{align*}
  sind Maße und es gilt für alle $A_1 \in \Alg_1$ und $A_2 \in \Alg_2$
  \[ \nu_1(A_1 \times A_2) = \mu_1(A_1) \cdot \mu_2(A_2) = \nu_2(A_1 \times A_2) \]
  und somit $\nu_1 = \nu_2$. Dieses Maß $\mu_1 \otimes \mu_2 \coloneqq \nu_1 = \nu_2$ heißt \emph{Produktmaß} von $\mu_1$ und $\mu_2$.
\end{satz}

% Name: Schnitte
\begin{nota}
  Für $f : \Omega_1 \times \Omega_2 \to \ER$ und $\omega_1 \in \Omega_1, \omega_2 \in \Omega_2$ schreibe
  \begin{align*}
    f_{\omega_1} : \Omega_2 \to \ER, \, \omega_2 \mapsto f(\omega_1, \omega_2), \qquad
    f_{\omega_2} : \Omega_1 \to \ER, \, \omega_1 \mapsto f(\omega_1, \omega_2)
  \end{align*}
\end{nota}

\begin{lem}
  Angenommen, $f : \Omega_1 \times \Omega_2 \to \ER$ ist $(\Alg_1 \otimes \Alg_2, \overline{\Bor})$-messbar. Dann ist auch für alle $\omega_1 \in \Omega_1$ die Abbildung $f_{\omega_1}$ $(\Alg_2, \overline{\Bor})$-messbar und für alle $\omega_2 \in \Omega_2$ die Abbildung $f_{\omega_2}$ $(\Alg_1, \overline{\Bor})$-messbar.
\end{lem}

\begin{satz}[Tonelli]
  Sei $f \in \overline{\E}(\Omega_1 \times \Omega_2, \Alg_1 \otimes \Alg_2)$, dann:
  \begin{itemize}
    \item $\Omega_2 \to [0, \infty], \enspace \omega_2 \mapsto \Int{\Omega_1}{}{f_{\omega_2}}{\mu_1}$ \enspace ist $(\Alg_2, \overline{\Bor})$-messbar,
    \item $\Omega_1 \to [0, \infty], \enspace \omega_1 \mapsto \Int{\Omega_2}{}{f_{\omega_1}}{\mu_2}$ \enspace ist $(\Alg_1, \overline{\Bor})$-messbar,
    \item \quad $\Int{\mathclap{\Omega_1 \otimes \Omega_2}}{}{f}{(\mu_1 \otimes \mu_2)} = \Int{\Omega_1}{}{(\Int{\Omega_2}{}{f_{\omega_1}}{\mu_2})}{\mu_1} = \Int{\Omega_2}{}{(\Int{\Omega_1}{}{f_{\omega_2}}{\mu_1})}{\mu_2}$.
  \end{itemize}
\end{satz}

\begin{satz}[Fubini]
  Sei $f : \Omega_1 \times \Omega_2 \to \ER$ $(\mu_1 \otimes \mu_2)$-integrierbar. Dann ist für $\mu_1$-fast-alle $\omega_1 \in \Omega_1$ der Schnitt $f_{\omega_1}$ $\mu_2$-integrierbar, und die $\mu_1$-fast-überall definierte Funktion $\omega_1 \mapsto \Int{\Omega_2}{}{f_{\omega_1}}{\mu_2}$ ist $\mu_1$-integrierbar. Analoges gilt mit $1$ und $2$ vertauscht. Es gilt:
  \[ \Int{\mathclap{\Omega_1 \otimes \Omega_2}}{}{f}{(\mu_1 \otimes \mu_2)} = \Int{\Omega_1}{}{(\Int{\Omega_2}{}{f_{\omega_1}}{\mu_2})}{\mu_1} = \Int{\Omega_2}{}{(\Int{\Omega_1}{}{f_{\omega_2}}{\mu_1})}{\mu_2}. \]
\end{satz}

\begin{samepage}

\section{Differentialformen}

\begin{nota}
  Sei im Folgenden $V$ ein $n$-dimensionaler $\R$-Vektorraum.
\end{nota}

\subsection{Alternierende Multilinearformen}

\begin{defn}
  Eine \emph{alternierende $k$-Form} auf $V$ ist eine Abb.
  \[ \omega : \underbrace{V \times ... \times V}_{\text{$k$-fach}} \to \R \quad \text{mit} \]
  \begin{itemize}
    \item $\omega$ ist \emph{multilinear}, d.\,h. linear in jedem Argument, d.\,h. für alle $l \in \{ 1, ..., k \}$ und $v_1, ..., v_{l-1}, v_{l+1}, ..., v_k \in V$ ist
    \[ \omega(v_1, ..., v_{l-1}, -, v_{l+1}, ..., v_k) \in \mathrm{Hom}(V, \R). \]
    \item Falls $v_j = v_l$ für $j < l$, dann ist $\omega(v_1, ..., v_j, ..., v_l, ..., v_k) = 0$.
  \end{itemize}
\end{defn}

\begin{bsp}
  Die Determinante ist eine alternierende $n$-Form auf $\R^n$.
\end{bsp}

\begin{nota}
  $\Lambda^k V^* \coloneqq \{ \text{$k$-Formen auf $V$} \}$ für $k \in \N^*$
\end{nota}

\begin{bem}
  $\Lambda^1 V^* = V^*$, $\Lambda^0 V^* \coloneqq \mathbb{K} = \R$
\end{bem}

\begin{lem}
  Sei $\omega \in \Lambda V^*, \sigma \in S_k$, dann gilt:
  \[ \omega(v_{\sigma(1)}, ..., v_{\sigma(k)}) = \mathrm{sgn}(\sigma) \cdot \omega(v_1, ..., v_k). \]
\end{lem}

\end{samepage}

\iffalse
% Definitionsmenge für das Dachprodukt
\begin{nota}
  % TODO: \bigoplus ist zu groß, \oplus etwas zu klein
  $\Lambda V^* \coloneqq \oplus_{k=1}^n \Lambda^k V^*$
\end{nota}
\fi

% Dach-Produkt (wedge-Produkt)

\begin{defn}
  Für $\phi_1, ..., \phi_k \in \Lambda^1 V^* = V^*$ ist das \emph{Dachprodukt} $\phi_1 \wedge ... \wedge \phi_k \in \Lambda^k V^*$ definiert durch
  \begin{align*}
    \phi_1 \wedge ... \wedge \phi_k &: V \times ... \times V \to \R \\
    (v_1, ..., v_k) &\mapsto \det \begin{pmatrix}
      \phi_1(v_1) & \phi_1(v_2) & ... & \phi_1(v_k) \\
      \phi_2(v_1) & \phi_2(v_2) & ... & \phi_2(v_k) \\
      \vdots & \vdots & \ddots & \vdots \\
      \phi_k(v_1) & \phi_k(v_2) & ... & \phi_k(v_k)
    \end{pmatrix}
  \end{align*}
\end{defn}

\begin{eign}
  \begin{itemize}
    \item Das Dachprodukt von Elementen aus $V^*$ ist in jedem Argument linear.
    \item Für $\sigma \in S_k$ gilt $\phi_{\sigma(1)} \wedge ... \wedge \phi_{\sigma(k)} = \mathrm{sgn}(\sigma) \cdot (\phi_1 \wedge ... \wedge \phi_k)$.
  \end{itemize}
\end{eign}

\begin{prop}
  \begin{itemize}
    \item Ist $\{ \phi_1, ..., \phi_n \}$ eine Basis von $V^*$, dann ist $\Set{ \phi_{j_1} \wedge ... \wedge \phi_{j_k} }{ 1 \leq j_1 < j_2 < ... < j_k \leq n }$ eine Basis von $\Lambda^k V^*$.

    \vspace{4pt}

    \begin{multicols}{2}
      \item $\dim(\Lambda^k V^*) = {n \choose k}$
      \item $\Lambda^k V^* = \{ 0 \}$ für $k > n$
    \end{multicols}
  \end{itemize}
\end{prop}

\begin{samepage}

\begin{prop}
  Seien $\phi_1, ..., \phi_k \in V^*$ und $A = (a_{jl}) \in \R^{k \times k}$ gegeben.
  Dann gilt für $\varphi_j \coloneqq \sum_{l=1}^k a_{jl} \phi_l \in V^*, j = 1, ..., k$:
  \[ \varphi_1 \wedge ... \wedge \varphi_k = \det(A) \cdot (\phi_1 \wedge ... \wedge \phi_k). \]
\end{prop}

\begin{satz}
  Seien $k, l, m \in \N^*$. Dann gilt:
  \begin{itemize}
    \item Es gibt eine eindeutig bestimmte bilineare Abbildung
    \[ (\Lambda^k V^*) \times (\Lambda^l V^*) \to \Lambda^{k+l} V^*, \quad (\omega, \widetilde{\omega}) \mapsto \omega \wedge \widetilde{\omega}, \]
    sodass für $\omega = \phi_1 \wedge ... \wedge \phi_k$ und $\widetilde{\omega} = \widetilde{\phi}_1 \wedge ... \wedge \widetilde{\phi}_l$ ($\phi_j, \widetilde{\phi}_i \in V^*$) gilt:
    \[ (\phi_1 \wedge \cdots \wedge \phi_k) \wedge (\widetilde{\phi}_1 \wedge ... \wedge \widetilde{\phi}_l) = \phi_1 \wedge ... \wedge \phi_k \wedge \widetilde{\phi}_1 \wedge ... \wedge \widetilde{\phi}_l. \]
    \item Sei $\{ \phi_1 , ..., \phi_k \}$ eine Basis von $V^*$, dann gilt für $\omega = \sum_{\mathclap{i_1 < ... < i_k}} \, a_{i_1 \cdots i_k} (\phi_{i_1} \wedge ... \wedge \phi_{i_k})$ und $\widetilde{\omega} = \sum_{\mathclap{j_1 < ... < j_k}} \, \widetilde{a}_{j_1 ... j_k} (\phi_{j_1} \wedge ... \wedge \phi_{j_k})$:
    \[ \omega \wedge \widetilde{\omega} = \sum_{\mathclap{\substack{i_1 < ... < i_k \\ j_1 < ... < j_l}}} \, (a_{i_1 \cdots i_k} \cdot \widetilde{a}_{j_1 \cdots j_l}) \cdot (\phi_{i_1} \wedge ... \wedge \phi_{i_k} \wedge \phi_{j_1} \wedge ... \wedge \phi_{j_l}) \]
  \end{itemize}
\end{satz}


\subsection{Differentialformen}

\end{samepage}

% Einsformen

\begin{nota}
  Sei im Folgenden $u \in U \opn \R^n$.

  Setze $T_u U \coloneqq \{ u \} \times \R^n = \Set{ (u, V) }{ V \in \R^n } \cong \R^n$
\end{nota}

\begin{bem}
  $T_u U$ ist ein $\R$-Vektorraum mit
  \begin{multicols}{2}
    \begin{itemize}
      \item $(u, V) + (u, W) = (u, V + W)$,
      \item $\lambda (u, V) = (u, \lambda V)$.
    \end{itemize}
  \end{multicols}
\end{bem}

\begin{bem}
  Für $U_1, U_2 \opn \R^n$, $u \in U_1 \cap U_2$ gilt $T_u U_1 = T_u U_2$.
\end{bem}

\begin{defn}
  \begin{itemize}
    \item \emph{Tangentialbündel} an $U \opn \R^n$: $TU = \bigsqcup_{\mathclap{u \in U}} T_u U$
    \item \emph{Dualraum} von $T_u U$: $T_u^* = \Set{ \alpha : T_u U \to \R }{ \alpha \text{ linear } }$
    \item \emph{Kotangentialbündel} an $U$: $T^* U = \bigsqcup_{\mathclap{u \in U}} T_u^* U$
    \item \emph{Einsform} (Differentialform von Grad 1, Pfaffsche Form) auf $U$:
\[ \omega : U \to T^*U \quad \text{mit} \quad \omega(u) \in T_u^*U \]
  \end{itemize}
\end{defn}


\begin{bsp}
  Sei $f : U \to \R$ total diff'bar, dann heißt die Einsform
  \[ \d f : U \to T^*U, \quad u \mapsto (u, V) \mapsto D_u f (V) \quad \text{\emph{totales Differential}.} \]
\end{bsp}

\begin{nota}
  $x_j : U \to \R, \quad (u_1, ..., u_n) \mapsto u_j$ % (Projektion auf $j$-te Koordinate)
\end{nota}

\begin{bem}
  $\d x_j (v_1, ..., v_n) = v_j$
\end{bem}

% Damit ist $(\d x_1(u), ..., \d x_j(u))$ eine Dualbasis von $((u, e_1), ..., (u, e_n))$

\iffalse % Allgemeiner aufgeschrieben für $k$-Formen
\begin{beobachtung}
  Für jede Einsform $\omega$ auf $U$ gibt es $n$ Abbildungen $a_j : U \to \R$, $j = 1, ..., n$, sodass $\omega = \sum_{j=1}^n a_j \d x_j$.
\end{beobachtung}

\begin{defn}
  Eine Einsform $\omega = \sum_{j=1}^n a_j \d x_j$ heißt stetig/diff'bar/$\mathcal{C}^k$, wenn für alle $j \in \{ 1, ..., n \}$: $a_j$ ist stetig/total diff'bar/$\mathcal{C}^k$.
\end{defn}
\fi

% Differentialformen höherer Ordnung

\begin{defn}
  Eine \emph{$k$-Form} auf $U$, $k \in \N^*$, ist eine Abbildung
  \[ \omega : U \to \bigsqcup_{\mathclap{u \in U}} \Lambda^k T_u^* U \quad \text{mit} \quad \omega(u) \in \Lambda^k T_u^* U \text{ für alle } u \in U. \]
  Eine \emph{$0$-Form} ist eine Abbildung $\omega : U \to \R$.
\end{defn}

% $\{ \d x_{j_1}(u) \wedge ... \wedge \d x_{j_k}(u) \mid 1 \leq j_1 < ... < j_k \leq n \}$
% Basis von $\Lambda^k T_u^* U$

\begin{beobachtung}
  Sei $\omega$ eine $k$-Form auf $U$, dann gibt es $\binom{n}{k}$ Funktionen $f_{j_1 \cdots j_k} : U \to \R$, $1 \leq j_1 < ... < j_k \leq n$, sodass
  \[ \omega = \sum_{\mathclap{j_1 < ... < j_k}} f_{j_1 \cdots j_k} \d x_{j_1} \wedge ... \wedge \d x_{j_k}. \]
\end{beobachtung}

\begin{defn}
  Die $k$-Form $\omega$ auf $U$ heißt stetig/diff'bar/$\mathcal{C}^k$, wenn alle $\binom{n}{k}$ Abbildungen $f_{j_1 \cdots j_k} : U \to \R$ stetig/total diff'bar/$\mathcal{C}^k$ sind.
\end{defn}

\begin{beobachtung}
  \begin{itemize}
    \item $\{ k\text{-Form auf } U \}$ ist Modul über $\{ f : U \to \R \}$
    % $(\omega_1 + \omega_2)(u) = \omega_1(u) + \omega_2(u)$
    % $(f \cdot \omega)(u) = f(u) \cdot \omega(u)$
    \item Für eine $k$-Form $\omega$ und eine $l$-Form $\eta$ ist $\omega \wedge \eta$ definiert durch $(\omega \wedge \eta)(u) \coloneqq \omega(u) \wedge \eta(u)$ für $u \in U$ eine $(k{+}l)$-Form auf $U$.
  \end{itemize}
\end{beobachtung}

\begin{defn}
  Sei $\omega = \sum_{\mathclap{j_1 < ... < j_k}} f_{j_1 \cdots j_k} (\d x_{j_1} \wedge ... \wedge \d x_{j_k})$ eine diff'bare $k$-Form auf $U$, dann heißt die $(k{+}1)$-Form
  \[ \d \omega \coloneqq \sum_{\mathclap{j_1 < ... < j_k}} \d f_{j_1 \cdots j_k} \wedge \d x_{j_1} \wedge ... \wedge \d x_{j_k} \quad \text{\emph{äußere Ableitung}}. \]
\end{defn}

% Herleitung:
\iffalse
\begin{bsp}
  Für $k = 1$: $\omega = \sum_{j = 1}^n f_j \d x_j$
  $\d \omega = \sum_{j=1}^n \d f_j \wedge \d x_j = \sum_{j=1}^n (\sum_{k=1}^n \frac{\partial f_j}{\partial x_k} \d x_k) \wedge \d x_j = \sum_{j=1}^k (\frac{\partial f_j}{\partial x_k} - \frac{\partial f_k}{\partial x_j}) \d x_k \wedge \d x_j$.

  $k = n -1$: $\mathrm{dim}(\Lambda^{n-1} V^*) = \binom{n}{n-1} = n = \mathrm{dim}(V)$
  $\Set{ (-1)^{j-1} \d x_1(u) \wedge ... \wedge \widehat{\d x_j(u)} \wedge ... \wedge \d x_n(u) }{ 1 \leq j \leq n }$ ist für alle $u \in U$ eine Basis von $\Lambda^{n-1} T_u^* U$.

  $\omega = \sum_{j=1}^n (-1)^{j-1} f_j \d x_1 \wedge ... \wedge \widehat{\d x_j} \wedge ... \wedge \d x_n$
  $\d \omega = \sum_{j=1}^n (-1)^{j-1} \d f_j \wedge \d x_1 \wedge ... \wedge \widehat{\d x_j} \wedge ... \wedge \d x_n = \begin{cases} 0, & \text{falls} j \not= k \\ \d x_1 \wedge ... \wedge \d x_n & j = k \end{cases}$
\end{bsp}
\fi

\begin{bem}
  \begin{itemize}
    \item Für eine diff'bare Einsform $\omega = \sum_{j=1}^n f_j dx_j$ auf $U$ gilt
    \[ \d \omega = \sum_{j<l} \left( \frac{\partial f_l}{\partial x_j} - \frac{\partial f_j}{\partial x_l} \right) \d x_j \wedge \d x_l. \]
    \item Eine diff'bare $(n{-}1)$-Form $\omega$ auf $U$ können wir schreiben als
    \[ \omega = \sum_{j=1}^n (-1)^{j-1} f_j \d x_1 \wedge ... \wedge \widehat{\d x_j} \wedge ... \wedge x_n \]
    mit total diff'baren Funktionen $f_j : U \to \R$. Dann ist
    \[ \d \omega = \left( \sum_{j=1}^n \frac{\partial f_j}{\partial x_j} \right) \d x_1 \wedge ... \wedge \d x_n. \]
  \end{itemize}
\end{bem}

\begin{satz}%[Eigenschaften der äußeren Ableitung]
  \begin{itemize}
    \item $\d (\omega \wedge \eta) = \d \omega \wedge \eta + (-1)^k \omega \wedge \d \eta$
    \begin{multicols}{2}
      \item $\d(\lambda \omega_1 + \omega_2) = \lambda \d \omega_1 + \d \omega_2$
      \item $\d (\d \omega) = 0$, falls $\omega$ $\mathcal{C}^2$ ist
    \end{multicols}
  \end{itemize}
\end{satz}

\begin{defn}
  Sei $U \opn \R^n$, $\widetilde{U} \opn \R^m$, $\phi : \widetilde{U} \to U$ total diff'bar und $\omega$ eine $k$-Form auf $U$. Die $k$-Form $\phi^* \omega$ auf $\widetilde{U}$, welche durch
  \[ (\phi^* \omega (\widetilde{u})) (X_1, ..., X_k) = (\omega (\phi(\widetilde{u})))(D_{\widetilde{u}} \phi (X_1), ..., D_{\widetilde{u}} \phi (X_k))  \]
  für alle $\widetilde{u} \in \widetilde{U}$, $X_1, ..., X_k \in T_{\widetilde{u}} \widetilde{U}$ definiert ist, heißt \emph{Rücktransport} von $\omega$ über $\phi$.
\end{defn}

\begin{anmerkung}
  Sei $\phi : \widetilde{U} \to U$ total diff'bar. Sei $\widetilde{u} \in \widetilde{U}$, dann ist
  \[ D_{\widetilde{u}} \phi : T_{\widetilde{u}} \widetilde{U} \to T_{\phi(\widetilde{u})} U \quad \text{linear}. \]
\end{anmerkung}

\begin{satz}
  % TODO: Voraussetzungen verschönern

  Sei $\check{U} \opn \R^d$, $\widetilde{U} \opn \R^m$, $U \opn \R^n$ und $\psi : \check{U} \to \widetilde{U}$ und $\phi : \widetilde{U} \to U$ total diff'bar.
  Seien $\omega, \omega_1, \omega_2$ $k$-Formen auf $U$, $\eta$ ein $l$-Form auf $U$ und $\lambda \in \R$. Dann gilt

  \begin{itemize}
    \item $\phi^*(\lambda \omega_1 + \omega_2) = \lambda \phi^*(\omega_1) + \phi^*(\omega_2)$ \pright{Linearität}
    \item $\phi^*(\omega \wedge \eta) = (\phi^* \omega) \wedge (\phi^* \eta)$ \pright{Verträglichkeit mit $\wedge$}
    \item $\psi^*(\phi^* \omega) = (\phi \circ \psi)^* \omega$ \pright{Funktorialität}
    \item $\d (\varphi^* \omega) = \phi^* (\d \omega)$, falls $\omega$ diff'bar und $\phi$ eine $\mathcal{C}^2$-Abb. ist
    \item Wenn $\omega = \sum_{j_1 < ... < j_k} f_{j_1 \cdots f_k} \d x_{j_1} \wedge ... \wedge \d x_{j_k} : U \to \R$, dann gilt:
    \[ \phi^* \omega = \sum_{j_1 < ... < j_k} (f_{j_1 \cdots j_k} \circ \phi) \d \phi_{j_1} \wedge ... \wedge \d \phi_{j_k} \]
  \end{itemize}
\end{satz}

\begin{defn}
  \begin{itemize}
    \item Für $k \geq 1$ heißt eine $k$-Form auf $U$ \emph{exakt}, wenn es eine diff'bare $(k{-}1)$-Form $\eta$ auf $U$ gibt, sodass $\omega = \d \eta$.
    \item Eine diff'bare $k$-Form $\omega$ auf $U$ heißt \emph{geschlossen}, wenn $\d \omega = 0$.
  \end{itemize}
\end{defn}

\begin{beobachtung}
  Jede diff'bare exakte $k$-Form auf $U$ ist geschlossen.
\end{beobachtung}

% Frage: Gilt auch die Umkehrung?

\begin{defn}
  Eine Teilmenge $U \subset \R^n$ heißt \emph{sternförmig}, falls es einen Punkt $u_0 \in U$ gibt, sodass für alle anderen Punkte $u \in U$ die Verbindungsstrecke von $u_0$ nach $u$ in $U$ liegt.
\end{defn}

\begin{lem}[Poincaré]
  Ist $U$ sternförmig, dann ist jede geschlossene, stetig diff'bare $k$-Form mit $k \geq 1$ auch exakt.
\end{lem}

\begin{bem}
  Statt Sternförmigkeit kann man auch nur Zusammenziehbarkeit fordern.
\end{bem}

% In der Sprache der de Rham Kohomologie besagt das Lemma von Poincaré, dass die de Rham Kohomologiegruppen von Grad $\geq 1$ von sternförmigen (oder allgemeiner zusammenziehbaren) Teilmenge $U \opn \R^n$ alle verschwinden.

\begin{lem}
  Sei $U \opn \R^n$ und $V \opn R^{n{+}1} = \R \times \R^n$, sodass der Zylinder $[0,1] \times U$ in $V$ liegt. Sei $\sigma$ eine geschlossene stetig diff'bare $k$-Form auf $V$ mit $k \geq 1$ und sei $\varphi_r : U \to V, \enspace u \mapsto (r, u)$ für $r \in \{ 0, 1 \}$.
  %die Abbildungen, welche $U$ mit den "`Deckeln des Zylinders"' $[0,1] \times U \subet V$ identifizieren.
  Dann gibt es eine stetig diff'bare $(k{-}1)$-Form $\eta$ auf $U$ mit $\varphi_1^* \sigma - \varphi_2^* \sigma = \d \eta$.
\end{lem}

% 20.1.2014

% Überschrift: Vektoranalyis

% $U \opn \R^n$, $f : U \to \R$ $\mathcal{C}^1$-Funktion, $F = \begin{pmatrix} F_1 \\ \vdots \\ F_n \end{pmatrix} : U \to \R^n$ $\mathcal{C}^1$-Vektorfeld

% $\mathrm{grad} f : U \to \R^n, \quad u \mapsto \begin{pmatrix} \tfrac{\partial f}{\partial x_1} (u) \\ \vdots \tfrac{\partial f}{\partial x_n} (u) \end{pmatrix}$ heißt \emph{Gradient} von $f$

\emph{Divergenz von $F$}
$\mathrm{div} F : U \to \R, \quad u \mapsto \sum_{j=1}^n \frac{\partial F_j}{\partial x_j} (u)$

\emph{Rotation von $F$}
$\mathrm{rot} F : U \to \R^3, \quad u \mapsto \begin{pmatrix}
  \frac{\partial F_3}{\partial x_2} (u) - \frac{\partial F_2}{\partial x_3} (u) \\
  \frac{\partial F_1}{\partial x_3} (u) - \frac{\partial F_3}{\partial x_1} (u) \\
  \frac{\partial F_2}{\partial x_1} (u) - \frac{\partial F_1}{\partial x_2} (u)
\end{pmatrix}$

Physiker-Notation:

Nabla-Operator $\nabla = \begin{pmatrix} \frac{\partial}{\partial x_1} \\ \vdots \\ \frac{\partial}{\partial x_n} \end{pmatrix}$ % TODO: Pfeil!

$\mathrm{grad} f = \nabla f$ % Pfeil!

$\mathrm{div} F = \nabla \cdot F$ % Pfeile!
"`Skalarprodukt-Analogon"'
 
in $\R^3$: $\mathrm{rot} F = \nabla \times F$ % Pfeile!
"`Vektorprodukt-Analogon"'

Vektoranalysis im $\R^3$ mit Physiker-Notation

$\d s = \begin{pmatrix} \d x_1 \\ \d x_2 \\ \d x_3 \end{pmatrix}$ % Pfeil über s

$\d S = \begin{pmatrix} \d x_2 \wedge \d x_3 \\ \d x_3 \wedge \d x_1 \\ \d x_1 \wedge \d x_2 \end{pmatrix}$ % Pfeil über S

$\d V = \d x_1 \wedge \d x_2 \wedge \d x_3$

In Analogon zum Skalarprodukt in $U \opn \R^3$

$0$-Form auf $U$: $f : U \to \R$

$1$-Form auf $U$: $\omega = \sum_{j=1}^3 f_j \d x_j = F \cdot \d s$, $F = \begin{pmatrix} f_1 \\ f_2 \\ f_3 \end{pmatrix} : U \to \R^3$

$2$-Form auf $U$: $\omega = g_1 \d x_2 \wedge \d x_3 + g_2 \d x_3 + \d x_1 + g_3 \d x_1 \wedge x_2 =: G \cdot \d S$, $G = \begin{pmatrix} g_1 \\ g_2 \\ g_3 \end{pmatrix}$

$3$-Form auf $U$: $\omega = g \cdot \d V$, $g : U \to \R$

Annahme: $f, g, F, G$ seien $\mathcal{C}^1$, dann gilt:

\begin{itemize}
  \item $\d f = \sum_{j=1}^3 \frac{\partial f}{\partial x_j} \d x_j = \mathrm{grad}(f) \cdot \d s$
  \item $\d (F \cdot \d s) = \d (\sum_{j=1}^3 f_j \d x_j) = \sum_{j=1}^3 \d (f_j \d x_j) = \sum_{j=1}^3 \sum_{k=1}^3 \frac{\partial f}{\partial x_k} \d x_k \wedge \d x_j = ... = (\frac{\partial f_3}{\partial x_2} - \frac{\partial f_2}{\partial x_2}) \d x_2 \wedge \d x_3 + (\frac{\partial f_3}{\partial x_1} - \frac{f_1}{x_3}) \d x_1 \wedge \d x_3 + (\frac{\partial f_2}{\partial x_1} - \frac{f_1}{x_2}) \d x_1 \wedge \d x_2$
  \item $\d (G \d S) = \d (G_1 \d x_2 \wedge \ x_3 + G_2 \d x_3 \wedge \d x_1 + G_3 \d x_1 \wedge \d x_2) = (\sum_{j=1}^3 \frac{\partial G_j}{\partial x_j}) \d x_1 \wedge \d x_2 \wedge \d x_3 = (\mathrm{div} G) \d V$
\end{itemize}

% Aufgabe:

Sei $f : U \to \R$ ein $\mathcal{C}^2$-Vektorfeld, $F : U \to \R^3$ $\mathcal{C}^2$, dann gilt:
$\mathrm{rot} (\mathrm{grad} f) = 0$, $\mathrm{div}(\mathrm{rot} F) = 0$

% Aufgabe:

Sei $U$ sternförmig (oder allgemeiner zusammenziehbar). Zu zeigen:

\begin{itemize}
  \item Ist $f : U \to \R^3$ ein $\mathcal{C}^1$-Vektorfeld mit $\mathrm{rot} F = 0$, dann gibt es $f : U \to \R$ ($\mathcal{C}^1$) mit $F = \mathrm{grad} f$.
  \item Ist $G : U \to \R^3$ ein $\mathcal{C}^1$-Vektorfeld mit $\mathrm{div} G = 0$, dann gibt es $\widehat{F} : U \to \R^3$ ($\mathcal{C}^1$) mit $G = \mathrm{rot} \widehat{F}$.
\end{itemize}


% 3.2. Integration von Differentialformen

\begin{defn}
  Sei $U \opn \R^n$, $\omega = f \d x_1 \wedge ... \wedge \d x_n$ eine $n$-Form, wobei $f : U \to \R$ Lebesgue-integrierbar. Sei $C \subset U$ eine Borel-Menge, dann
  \[ \Int{C}{}{\omega}{} \coloneqq \Int{C}{}{f}{\lambda_n} \in \R \]
\end{defn}

% Übungsaufgabe:

$\phi : \widetilde{U} \to U$ sei $\mathcal{C}^1$-Abbildung, $U, \widetilde{U} \opn \R^n$, dann
$\phi^* \omega = ((f \circ \phi) \cdot \det(D\phi)) \d x_1 \wedge ... \wedge \d x_n$

Transformationsregel für Integrale von $n$-Formen

Sei $\phi : \widetilde{U} \to U$, $\widetilde{U}, U \opn \R^n$. Sei $\omega = f \d x_1 \wedge ... \wedge \d x_n$ eine $n$-Form auf $U$. Dann gilt für ein Kompaktum $C \subset \widetilde{U}$:
\[ \Int{C}{}{\phi^* \omega}{} = \Int{C}{}{(f \circ \phi) \cdot \det(D\phi) \d x_1 \wedge ... \wedge \d x_n}{} = \Int{C}{}{(f \circ \phi) \det(D\phi)}{\d \lambda_n}. \]
Wenn $\det(D\phi(\widetilde{u})) > 0$ für alle $\widetilde{u} \in U$ (also $\phi$ orientierungserhaltend):
\[ \Int{C}{}{\phi^* \omega}{} = \Int{C}{}{(f \circ \phi) \cdot \abs{\det{D\phi}}} = \Int{\phi(C)}{}{f}{\d \lambda_n} = \Int{\phi(C)}{}{\omega}{} \]
Wenn $\det(D\phi(\widetilde{u})) < 0$ für alle $\widetilde{u} \in U$ (also $\phi$ orientierungsumkehrend):
\[ \Int{C}{}{\phi^* \omega}{} = - \Int{\phi(C)}{}{\omega}{} \]

Teilung der Eins:

Hutfunktion: $g : \R \to \R, \quad x \mapsto \begin{cases} \exp(\frac{1}{t^2 - 1}), & \abs{x} < 0 \\ 0, \abs{x} \geq 1 \end{cases}$ ist $\mathcal{C}^2$

Hutsummen: $G : \R \to \R, \quad x \mapsto \sum_{k \in \Z} g(x-k)$

normalisierter Hut für $k \in \Z$: $h_k : \R \to \R, \quad x \mapsto \frac{g(x-k)}{G(x)}$ ist $\mathcal{C}^\infty$, dann $\mathrm{supp}(h_k) = \left[ k-1, k+1 \right]$

Tolle Eigenschaft: $\sum_{k \in \Z} h_k(x) = \sum_{k \in \Z} \frac{g(x-k)}{G(x)} = \frac{1}{G(x)} \sum_{k \in \Z} g(x-k) = 1$ für alle $x \in \R$

Also $\{ h_k \}$ bilden \emph{Teilung der Eins}

In $\R^n$: Sei $q = (q_1, ..., q_n) \in \Z^n$, $\epsilon > 0$

$\d \alpha_{q,\epsilon} : \R^n \to \R, (x_1, ..., x_n) \mapsto \prod_{j=1}^{n} h_{q_j} \cdot (\frac{x_j}{\epsilon})$

Es gilt:

\begin{itemize}
  \item $\d \alpha_{q, \epsilon}$ ist $\mathcal{C}^\infty$
  \item $\mathrm{supp} \alpha_{q,\epsilon} = \Set{ (x_1, ..., x_n) \in \R^n }{ \abs{x_j - \epsilon q_j} \leq \epsilon \text{ für alle } j \in \{ 1, ..., n \} }$
  \item $\sum_{q \in \Z^n} \alpha_{q, \epsilon} = 1$ (Übung)
\end{itemize}


% Vorlesung vom 22.1.2014

% Kapitel 2.4. Integration von Differentialformen auf Untermannigfaltigkeiten

\begin{defn}
  Eine $m$-dimensionale \emph{Untermannigfaltigkeit} (UMF) des $\R^n$ ist eine Teilmenge $M \subset \R^n$ mit:
  \begin{itemize}
    \item $\fa{x \in M} \ex{\widetilde{U} \opn \R^n, x \in \widetilde{U}} \ex{\widetilde{V} \opn \R^n} \ex{\widetilde{\phi} : \widetilde{U} \to \widetilde{V}}$ Diffeomorphismus, sodass gilt:
    \[ \widetilde{\phi}(M \cap \widetilde{U}) = \widetilde{V} \cap \underbrace{\Set{ (x_1, ..., x_m, 0, ..., 0) }{ x_1, ..., x_m \in \R }}_{\R^m \subset \R^m \oplus \R^{n-m} \cong \R^n} \]
  \end{itemize}
\end{defn}

\begin{nomenklatur}
  $\widetilde{\phi} : \widetilde{U} \to \widetilde{V}$ heißt \emph{Untermannigfaltigkeit-Karte}, notiert $(\widetilde{\phi}, \widetilde{U})$
\end{nomenklatur}

\begin{defn}
  $\widetilde{\Atlas} = \Set{ \underbrace{(\widetilde{\phi}_i, \widetilde{U}_i)}_{\text{UMF-Karten von $M$}} }{ i \in I }$ mit $\underbrace{M \subset \bigcup_{i \in I} \widetilde{U}_i}_{\text{Überdeckungseigenschaft}}$ heißt \emph{UMF-Atlas} von $M$.
\end{defn}

\begin{beob}
  $\widetilde{\phi} : \widetilde{U} \to \widetilde{V}$ ist UMF-Karte von $M$, dann $M \cup \widetilde{U} \opn M$ (bzgl. Relativtopologie), $\widetilde{V} \cap \R^m \opn \R^m$
\end{beob}

\begin{defn}
  Der Homöomorphismus $\phi \coloneqq \widetilde{\phi}|_{U} : U \to V$ heißt \emph{Karte von $M$}.
\end{defn}

\begin{defn}
  Ist $\widetilde{A} = \Set{ (\widetilde{\phi}_i, \widetilde{U}_i) }{ i \in I }$ ein UMF-Atlas, dann heißt $\Atlas = \Set{ (\phi_i, U_i) }{ i \in I }$ \emph{Atlas von $M$}.
\end{defn}

\begin{defn}
  Ein Atlas $\Atlas = \Set{ (\phi_i, U_i) }{ i \in I }$ heißt \emph{orientiert}, wenn für alle $i, j \in I$ die Abbildung $\phi_j \circ \phi_i^{-1} |_{\phi_i(U_i \cap U_j)} : \phi(U_i \cap U_j) \to \phi_j(U_i \cap U_j)$ orientierungserhaltend sind, wenn $U_i \cap U_j \not= \emptyset$.
  % Definition mit Funktionaldeterminante?
\end{defn}

\begin{defn}
  \begin{itemize}
    \item Eine UMF $M$ von $\R^n$ heißt \emph{orientierbar}, wenn $M$ einen orientierten Atlas besitzt.
    \item Zwei orientierte Atlanten $\Atlas_1, \Atlas_2$ von $M$ heißen gleichorientiert, wenn $\Atlas = \Atlas_1 \cup \Atlas_2$ ein orientierter Atlas ist.
    \item Eine \emph{Orientierung} auf einer orientierbaren UMF $M$ ist eine Äquivalenzklasse bzgl. der Äquivalenzrelation $\Atlas_1 \sim \Atlas_2 :\iff \Atlas_1, \Atlas_2$ gleichorientiert auf der Menge aller orientierten Atlanten.
    \item $M$ orientierbare UMF, $[A]$ Orientierung von $M$, dann heißt $(M, [\Atlas])$ \emph{orientierte UMF}.
    \item $\Atlas'$ orientierter Atlas von $(M, [\Atlas])$ heißt \emph{positiv orientiert}, wenn $\Atlas' \in [\Atlas]$.
  \end{itemize}
\end{defn}

Gegeben $(M, [\Atlas])$ orientierte UMF, $\Atlas = \Set{ (\phi_i, U_i) }{ i \in I }$ ein positiv orientierter Atlas,
\[ \tau : \R^m \to \R^m, \quad (x_1, ..., x_m) \mapsto (x_1, ..., x_{m-1}, - x_m) \]
orientierungsumkehrend. Dann ist auch $\Atlas' = \Set{ (\phi_i', U_i) }{ i \in I }$ mit $\phi_i' = \tau \circ \phi_i : U_i \to \tau(V_i)$
ein orientierter Atlas von $M$, aber $\Atlas' \not\in [A]$.
Notation: $-[\Atlas] \coloneqq [\Atlas']$ die zu $[\Atlas]$ entgegengesetzte Orientierung.

Integration von $k$-Formen auf $k$-dimensionalen UMFen

Gegeben:
\begin{itemize}
  \item $\widehat{U} \opn \R^n$
  \item $M \subset \widehat{U} \subset \R^n$ $k$-dimensionale UMF
  \item $\omega$ $k$-Form auf $U$
  \item $C \subset M$ kompakt
  \item $[\Atlas]$ Orientierung von $M$
\end{itemize}

Gesucht: $\Int{(C, [\Atlas])}{}{\omega}{}$

Fall 1: $\ex{\phi : U \to V}$ positiv orientierte Karte mit $C \subset U$, dann setze:

% TODO: LaTeX-Makro für Integral von Differentialformen

\[ \Int{(C, [\Atlas])}{}{\omega}{} \coloneqq \Int{\phi(C)}{}{(\phi^{-1})^* \omega} \]

Unabhängigkeit von Karte:

Sei $\phi' : U' \to V'$ positiv orientierte Karte mit $C \subset U'$, o.\,E. $U' = U$.

$\iota \coloneqq \phi \circ \phi'^{-1}$

\[ \Int{\phi(C)}{}{(\phi^{-1})^* \omega} = \Int{\phi(C)}{}{(\phi'^{-1} \circ \iota^{-1})^* \omega}{} = \Int{\phi(C)}{}{\iota^{-1*} (\phi'^{-1*} \omega)}{} = \Int{\phi'(C)}{}{\phi'^{-1*} \omega}{} \]

% Vorlesung vom 27.1.2013

Fall 2: Es existiert keine positiv orientierte Karte $\phi : U \to V$ mit $C \subset U$:

Wähle (endliche) Teilung der Eins, d.\,h.

$\alpha_i : C \to \R$, $i = 1, ..., r$ mit

\begin{itemize}
  \item $\sum_{i=1}^r \alpha_i(x) = 1$ für alle $x \in C$
  \item zu demem $i \in \{ 1, ..., r\}$ gibt es eine positiv orientierte Karte $\phi_i : U_i \to V_i$ mit $\supp(\alpha_i) \subset U_i$.
\end{itemize}

Dann: $\Int{C}{}{\omega}{} = \Int{(C, [\Atlas])}{}{\omega}{} \coloneqq \sum_{j=1}^r \Int{\phi_j(\supp(\alpha_j) \cap C)}{}{(\alpha_j \circ \phi_j^{-1}) \cdot (\phi_j^{-1}) \omega}{}$

Übungsaufgabe: Unabhängigkeit von Teilung der Eins und Kartenwahl

% Kapitel 2.5. Satz von Stokes

\begin{satz}[Stokes]
  Gegeben sei
  \begin{itemize}
    \item $M \subset \R^n$ $k$-dimensionale Untermannigfaltigkeit
    \item $\widehat{U} \opn \R^n$, $M \subset \widehat{U}$
    \item $\omega$ stetig diff'bare $(k{-}1)$-Form auf $\widehat{U}$.
    \item $C \subset M$ Kompaktum mit glattem Rand $\partial_M C$.
  \end{itemize}
  Dann gilt (bzgl. der induzierten Orientierung auf $\partial_M C$)
  \[ \Int{C}{}{\d \omega}{} = \Int{\partial_M C}{}{\omega}{} \]
\end{satz}

% Erklärung

Halbraum:

\[ H^k = \Set{ (x_1, ..., x_k) \subset \R^k }{ x_1 \leq 0 } \]

Rand von $H^k$:

\[ \partial H^k = \Set{ (0, x_2, ..., x_k) \in \R^k }{ x_2, ..., x_k \in \R } \]

Beobachtung: $\partial H^k$ ist $(k{-}1)$-dimesionale UMF von $\R^k$.
UMF-Karte:

\begin{align*}
  \widetilde{\phi} : \R^k \to \R^k, \quad (x_1, ..., x_k) \mapsto (x_2, ..., x_k, x_1)
\end{align*}

Karte $phi = \mathrm{proj}_{\R^{k{-}1}} \circ \widetilde{\phi}|_{\partial H^k} : \partial H_k \to \R^{k{-}1}$, $(0, x2, ..., x_k) \mapsto (x_2, ..., x_k)$
orientierter Atlas $\Atlas = \{ (\phi, \partial H_k) \}$ und $(\partial H_k, [\Atlas])$ ist eine orientierte UMF

% Bemerkung:
% \[ \phi^{-1} : \R^{k{-}1} \to \R^k, \quad (t_1, ..., x_{k-1}) \mapsto (0, t_1, ..., t_{k-1}) \]

\begin{defn}
  Sei $C \subset M \subset \R^n$, wobei $C$ kompakt und $M$ eine $k$-dimensionale UMF.

  Ein Punkt $x \in M$ heißt \emph{Randpunkt} von $C$ relativ zu $M$, wenn gilt:
  \[ \fa{\underbrace{U \opn M}_{\text{bzgl. der Relativtopologie}} \text{ mit } x \in U} U \cap C \not= \emptyset \enspace \text{und} \enspace U \cap (M \setminus C) \not= \emptyset. \]
\end{defn}

% Ausgelassen: Definition Relativtopologie

\begin{nota}
  $\partial_M C \coloneqq \{ \text{Randpunkte von $C$ relativ zu $M$} \}$
\end{nota}

\begin{defn}
  $C$ hat glatten Rand in $M$, wenn gilt:
  \[ \fa{x \in \partial_M C} \ex{\text{UMF-Karte } \widetilde{\phi} : \widetilde{U} \to \widetilde{V} \text{ mit } x \in \widetilde{U}} \]
  für $\phi \coloneqq \mathrm{proj}_{\R^k} \circ \widetilde{\phi}|_U : U \to V$ (Homöo) gilt:
  \begin{itemize}
    \item $\phi(U \cap C) = V \cap H_k$
    \item $\phi(U \cap \partial_M C) = V \cap \partial H_k$
  \end{itemize}
  Eine UMF-Karte $(\widetilde{\phi}, \widetilde{U})$ (bzw. eine Karte $(\phi, U)$), die diese Bedingungen erfüllt, heißt \emph{Rand-adaptiert}.
\end{defn}

% Übungsaufgabe:
\begin{lem}
  $C \subset M$ Kompaktum mit glattem Rand. Dann gilt:
  \begin{itemize}
    \item  Ist $\Atlas = \Set{ (\widetilde{\phi}_i, \widetilde{U}_i) }{ i \in I }$ ein UMF-Atlas von $M$ aus Rand-adaptierten Karten, dann ist
    \[ \Atlas' = \Set{ (\phi_i', \widetilde{U}_i) }{ i \in I } \]
    \[ \phi_i' = \rho \circ \widetilde{\phi} \]
    \[ \rho : \R^n \to \R^n, \quad (x_1, ..., x_n) \mapsto (x_2, ..., x_k, x_1, x_{k+1}, ..., x_n) \]
    ein UMF-Atlas von $\partial_M C$. Insbesondere ist $\partial_M C$ eine $(k{-}1)$-dimensionale UMF.
    \item Ist $M$ orientiert, dann gibt es einen positiv orientierten, Rand-adaptierten Atlas $\Atlas = \Set{ (\widetilde{phi}_i, \widetilde{U}_i) }{ i \in I }$ von $M$. Sodann ist $\Atlas'$ ein orientierter UMF-Atlas von $\partial_m C$. ($[\Atlas']$ definiert eine Orientierung auf $\partial_M C$.)
  \end{itemize}
\end{lem}

% Vorbereitungslemma:

\begin{lem}
  Sei $\Omega$ eine stetig diff'bare $(k{-}1)$-Form auf $\R^k$ mit kompaktem Träger. Dann gilt:
  \[ \Intdf{H_k}{\d \omega} = \Intdf{\partial H_k}{\omega}. \]
\end{lem}

\end{document}
