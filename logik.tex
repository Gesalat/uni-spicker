\documentclass{cheat-sheet}

\pdfinfo{
  /Title (Zusammenfassung Logik für Informatiker)
  /Author (Tim Baumann)
}

\usepackage{MnSymbol} % \leftmodels, \rightmodels
%\usepackage{turnstile} % \turnstile and variants
\usepackage{bussproofs} % natural deduction type-setting

\newcommand{\Ibm}{I, \beta \models}
\newcommand{\Ibnm}{I, \beta \not\models}
\newcommand{\Iba}[1]{(#1)_{I,\beta}} % I-Beta-Auswertung
\newcommand{\lra}{\leftrightarrow}
\newcommand{\For}{\mathrm{For}}
\newcommand{\Ax}{\mathrm{Ax}}
\newcommand{\eqmodels}{\leftmodels \rightmodels}

\begin{document}

\maketitle{Zusammenfassung Logik für Informatiker}

\section{Prädikatenlogik erster Stufe}

% Kapitel 1.1. Syntax

% Ausgelassen: Definition Signatur, Definition Term
% Ausgelassen: Herleitung, Induktion über Herleitungen
% Ausgelassen: Definition Formel

% Kapitel 1.2. Semantik

% Ausgelassen: Definition Belegung und Interpretation

\begin{nota}
  Sei $\beta : \{ x_0, x_1, ... \} \to D_I$ eine Belegung zu einer Interpretation $I$, $x$ eine Variable und $d \in D_I$. Dann setze
  \[ \beta_x^d : \{ x_0, x_1, ... \} \to D_I, \quad y \mapsto \begin{cases} d, & \text{falls } x = y \\ \beta(y), & \text{sonst} \end{cases} \]
\end{nota}

% Ausgelassen: Definition Auswertung

% Ausgelassen: Erfüllung von Formeln durch Interpretation und Belegung

\begin{defn}
  Eine Interpretation $I$ und eine Belegung $\beta$ erfüllen eine eine Formel $F$, geschrieben $I, \beta \models F$, falls
  \begin{alignat*}{3}
    & \Ibm (t_1 = t_2) &&\coloniff&& \Iba{t_1} = \Iba{t_2}\\
    & \Ibm P(t_1, ..., t_n) &&\coloniff&& P^I(\Iba{t_1}, ..., \Iba{t_n}) \\
    & \Ibm \neg A &&\coloniff&& \Ibnm A \\
    & \Ibm A \wedge B &&\coloniff&& (\Ibm A) \wedge (\Ibm B) \\
    & \Ibm A \vee B &&\coloniff&& (\Ibm A) \vee (\Ibm B) \\
    & \Ibm A \to B &&\coloniff&& (\Ibnm A) \vee (\Ibm B) \\
    & \Ibm A \lra B &&\coloniff&& \quad\, ((\Ibnm A) \wedge (\Ibnm B))\\
    & && && \vee ((\Ibm A) \wedge (\Ibm B)) \\
    & \Ibm \fa{x} A &&\coloniff&& \fa{d \in D_I} I, \beta_x^d \models A \\
    & \Ibm \ex{x} A &&\coloniff&& \ex{d \in D_I} I, \beta_x^d \models A
  \end{alignat*}
\end{defn}

% Ausgelassen: Einführung von Wahrheitstafeln

\begin{prop}
  Es gilt für alle Interpretationen $I$, Belegungen $\beta$ und Formeln $A, B$:
  \begin{alignat*}{2}
    & \Ibm A &&\iff \Ibnm \neg A \iff \Ibm \neg\neg A\\
    & \Ibm A \wedge B &&\iff \Ibm \neg (A \to \neg B)\\
    & \Ibm A \vee B &&\iff \Ibm \neg A \to B\\
    & \Ibm A \lra B &&\iff \Ibm (A \to B) \wedge (B \to A)\\
    & \Ibm \ex{x} A &&\iff \Ibm \neg \fa{x} \neg A
  \end{alignat*}
\end{prop}

% Kapitel 1.3. Modelle und Folgerbarkeit

\begin{defn}
  Seien $A \in \For$, $M \subset \For$ und $I$ eine Interpretation. Dann heißt $I$ ein \emph{Modell} von $A$ bzw. $M$, falls
  \begin{align*}
    I \models A &\coloniff \text{für alle Belegungen $\beta$ gilt } \Ibm A,\\
    I \models M &\coloniff \fa{F \in M} I \models F.
  \end{align*}
\end{defn}

\begin{nota}
  Für $M \subset \For$, eine Interpretation $I$ und eine Belegung $\beta$ schreiben wir:
  \[ \Ibm M \coloniff \fa{F \in M} \Ibm F \]
\end{nota}

\begin{defn}
  Seien $A, B \subset \For$. Man sagt, $B$ \emph{folgt} aus $A$ (geschrieben $A \models B$), falls für alle Interpretationen $I$ und Belegungen $\beta$ gilt:
  \[ \Ibm A \implies \Ibm B. \]
  Falls $A \models B$ und $B \models A$ gilt, so heißen $A$ und $B$ \emph{logisch äquivalent}, geschrieben $A \eqmodels B$.
\end{defn}

\begin{nota}
  $A_1, ..., A_n \models A \coloniff \{ A_1, ..., A_n \} \models A$
\end{nota}

\begin{satz}
  Für alle Interpretationen $I$ und $n \in \N$ gilt:
  \[ I \models \{ A_1, ..., A_n \} \iff I \models A_1 \wedge ... \wedge A_n \]
\end{satz}

\begin{satz}
  Für alle $A, B \in \For$ und $M \subset \For$ gilt:
  \[ M \models A \to B \iff M \cup \{A\} \models B \]
\end{satz}

\begin{defn}
  Eine Formel $A \in \For$ heißt \emph{Tautologie} oder \emph{(allgemein-) gültig} (geschrieben $\models A$), falls $I \models A$ für alle Interpretationen $I$ gilt.
\end{defn}

\begin{defn}
  Eine Formel $A \in \For$ heißt \emph{erfüllbar}, wenn es eine Interpretation $I$ und eine Belegung $\beta$ mit $\Ibm A$ gibt. Falls es dies nicht gibt, so heißt $A$ \emph{unerfüllbar}.
\end{defn}

\begin{satz}
  Für $A \in \For$ gilt:
  \begin{itemize}
    \miniitem{0.45\linewidth}{$\models A \implies A \text{ ist erfüllbar}$}
    \miniitem{0.35\linewidth}{$\models A \iff \emptyset \models A$}
  \end{itemize}
\end{satz}

\begin{satz}
  Sei $A \in \For$ und $M \subset \For$. Dann gilt $M \models A$ genau dann, wenn $M \cup \{ \neg A \}$ unerfüllbar ist. Insbesondere ist $A$ genau dann gültig, wenn $\{ \neg A \}$ unerfüllbar ist.
\end{satz}

% Kapitel 1.4. Teil-Interpretationen

% Ausgelassen: Definition Teil-Interpretation, Teil-Modell

\begin{defn}
  \emph{Universelle Formeln} sind Formeln, die sich nach den folgenden Regeln herleiten lassen:

  \begin{minipage}{0.32\linewidth}
    \begin{prooftree}
      \AxiomC{$A$ ist quantorenfrei}
      \UnaryInfC{$A$}
    \end{prooftree}
  \end{minipage}
  \begin{minipage}{0.21\linewidth}
    \begin{prooftree}
      \AxiomC{$A$}
      \AxiomC{$B$}
      \BinaryInfC{$A \wedge B$}
    \end{prooftree}
  \end{minipage}
  \begin{minipage}{0.21\linewidth}
    \begin{prooftree}
      \AxiomC{$A$}
      \AxiomC{$B$}
      \BinaryInfC{$A \vee B$}
    \end{prooftree}
  \end{minipage}
  \begin{minipage}{0.21\linewidth}
    \begin{prooftree}
      \AxiomC{$A$}
      \UnaryInfC{$\fa{x} A$}
    \end{prooftree}
  \end{minipage}
\end{defn}

% Ausgelassen: 

\section{Aussagenlogik}

% Kapitel 2.1. Allgemeines, Wahrheitstafeln

% Ausgelassen: Begriff Atom

\begin{prop}
  Sei $I$ eine Teil-Interpretation zu $J$, $\beta$ eine Belegung zu $I$ und $A$ eine universelle Formel. Dann gilt:
  \[ J, \beta \models A \implies I, \beta \models A. \]
\end{prop}

% Ausgelassen: Entscheidbarkeit des Wahrheitsgehalts mit Wahrheitstafeln
% Ausgelassen: Satz 2.3
% Ausgelassen: Strukturelle Definition von Ersetzung von $p_0$ durch $A$
% Ausgelassen: Lange Liste von aussagenlogischen Äquivalenzen (Kommutativität, Idempotenz, Distributivität, etc.)

\begin{defn}
  Für $p \in \mathcal{P}^0$ heißen die Ausdrücke $p$ und $\neg p$ \emph{Literale}. Eine Disjunktion von Literalen heißt \emph{Klausel}. Eine Formel ist in \emph{konjunktiver Normalform (KNF)}, wenn sie eine Konjunktion von Klauseln ist.
\end{defn}

% Ausgelassen: Disjunktive Normalform

\begin{prob}[\emph{SAT}]
  Gegeben sei eine Formel in konjunktiver Normalform. Frage: Ist diese Formel erfüllbar?
\end{prob}

\begin{defn}
  Eine Formel ist in \emph{Negationsnormalform (NNF)}, wenn Negationen nur unmittelbar vor Atomen stehen.
\end{defn}

% Ausgelassen: Verfahren, um Formel in NNF und dann in KNF zu überführen

% Kapitel 2.2. Der Hilbert-Kalkül

% Ausgelassen: Definition von Korrektheit und Vollständigkeit von Kalkülen

\begin{defn}
  Der \emph{Hilbert-Kalkül} besteht aus den Axiomen
  \begin{align*}
    \Ax_1 &\coloneqq \Set{ A \to (B \to A) }{ A, B \in \For }\\
    \Ax_2 &\coloneqq \Set{ (A \to (B \to C)) \to ((A \to B) \to (A \to C)) }{ A, B, C \in \For }\\
    \Ax_3 &\coloneqq \Set{ (A \to (B \to C)) \to ((A \to B) \to (A \to C)) }{ A, B, C \in \For }
  \end{align*}
  und der Schlussregel \emph{Modus Ponens (MP)}
  \begin{prooftree}
    \AxiomC{$A$}
    \AxiomC{$A \to B$}
    \BinaryInfC{$B$}
  \end{prooftree}
\end{defn}

\begin{defn}
  Eine Formel $F \in \For$ ist aus $M \subset \For$ \emph{H-herleitbar}, notiert $M \vdash_H A$, wenn es eine Folge $A_1, ..., A_n$ in $\For$ gibt mit $A_n = A$, sodass für alle $i \in \{ 1, ..., n \}$ gilt:
  \[
    A_i \in \Ax_1 \cup \Ax_2 \cup \Ax_3 \cup M
    \quad \text{oder} \quad
    \ex{j, k < i} A_j = A_k \to A_i.
  \]
\end{defn}

\begin{defn}
  $A \in \For$ heißt \emph{herleitbar}, notiert $\vdash A$, falls $\emptyset \vdash A$ gilt.
\end{defn}

\begin{beob}
  Präfixe und Verkettungen von Herleitungen sind ebenfalls Herleitungen.
\end{beob}

\begin{prop}
  \begin{itemize}
    \item Aus $M \vdash A$ und $M \vdash A \to B$ folgt $M \vdash B$.
    \item Aus $M \vdash \neg A \to \neg B$ folgt $M \vdash B \to A$.
  \end{itemize}
\end{prop}

\begin{satz}[Deduktionstheorem]
  $M \vdash A \to B \iff M \cup \{ A \} \vdash B$
\end{satz}

% 2.9.
\begin{satz}
  Für alle $A, B, C \in \For$ gilt:
  \begin{itemize}
    \miniitem{0.65 \linewidth}{$\vdash (A \to B) \to ((B \to C) \to (A \to C))$}
    \miniitem{0.32 \linewidth}{$\vdash \neg A \to (A \to B)$}
    \miniitem{0.32 \linewidth}{$\vdash \neg\neg A \to A$}
    \miniitem{0.32 \linewidth}{$\vdash A \to \neg\neg A$}
    \miniitem{0.32 \linewidth}{$\vdash (\neg A \to A) \to A$}
  \end{itemize}
\end{satz}

% 2.10.
\begin{prop}
  Es gilt:
  
  \begin{center}
    \begin{minipage}{0.4\linewidth}
      \begin{prooftree}
        \AxiomC{$A \to B$}
        \AxiomC{$B \to C$}
        \BinaryInfC{$A \to C$}
      \end{prooftree}
    \end{minipage}
    \begin{minipage}{0.15\linewidth}
      \begin{prooftree}
        \AxiomC{$\neg\neg A$}
        \UnaryInfC{$A$}
      \end{prooftree}
    \end{minipage}
  \end{center}
\end{prop}

% Kapitel 2.3. Korrektheit und Vollständigkeit

% 2.11.
\begin{satz}[Korrektheitssatz]
  Sei $A \in \For$ und $M \subset \For$. Dann gilt
  \[ M \vdash A \implies M \models A. \]
\end{satz}

\begin{defn}
  $M \subset \For$ heißt \emph{konsistent}, wenn für kein $A \in \For$ zugleich $M \vdash A$ und $M \vdash \neg A$ gilt.
\end{defn}

\begin{lem}
  \begin{itemize}
    \item Ist $M$ inkonsistent, so gilt $M \vdash B$ für alle $B \in \For$.
    \item Für $A \in \For$ gilt: $M \not\vdash A \implies M \cup \{ A \}$ ist konsistent.
  \end{itemize}
\end{lem}

% 2.14.
\begin{lem}[Modell-Lemma]
  Jede konsistente Menge ist erfüllbar, d.\,h. sie besitzt ein Modell.
\end{lem}

% 2.15.
\begin{satz}[Vollständigkeitssatz]
  Sei $A \in \For$ und $M \subset \For$. Dann gilt
  \[ M \models A \implies M \vdash A. \]
\end{satz}

% Ausgelassen: Rest von Zusammenfassung 2.16.
\begin{prop}
  Sei $M \subset \For$. Dann ist $M$ genau dann erfüllbar, wenn $M$ konsistent ist.
\end{prop}

% 2.17.
\begin{satz}[Endlichkeits- bzw. Kompaktheitssatz] Sei $A \in \For$, $M \subset \For$.
  \begin{itemize}
    \item Dann gilt $M \models A$ genau dann, wenn es eine endliche Teilmenge $M' \subset M$ mit $M' \models A$ gibt.
    \item Dann ist $M$ genau dann erfüllbar, wenn jede endliche Teilmenge von $M$ erfüllbar ist.
  \end{itemize}
\end{satz}

\section{Hilbert-Kalkül für Prädikatenlogik}

\section{Weitere Beweisverfahren}

\section{Zusicherungskalkül}

\section{Temporale Logik}

\section{Modale Logik}

\end{document}