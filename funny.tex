\documentclass{cheat-sheet}

%\newcommand{\PS}{\mathcal{P}} % Powerset
%\newcommand{\PSO}{\PS(\Omega)} % Powerset
%\newcommand{\Alg}{\mathfrak{A}}
%\newcommand{\Ring}{\mathfrak{R}}
\newcommand{\K}{\mathbb{K}}
\newcommand{\supp}{\mathrm{supp}}
\newcommand{\Bor}{\mathfrak{B}} % Borel
\newcommand{\Leb}{\mathcal{L}} % Lebesgue
\newcommand{\fue}{\overset{\text{f.ü.}}} % fast überall
\newcommand{\dist}{\mathrm{dist}} % Entfernung

\newcommand{\IntO}[2]{\Int{\Omega}{}{#1}{#2}} % Integral über \Omega

\newcommand{\convWith}[1]{\xrightarrow{#1 \to \infty}} % konvergiert für #1 gegen unendlich gegen
\newcommand{\convWeaklyWith}[1]{\xrightharpoonup{#1 \to \infty}} % konvergiert schwach für #1 gegen unendlich gegen
\newcommand{\convWeaklyStarWith}[1]{\xrightharpoonup[*]{#1 \to \infty}} % konvergiert schwach für #1 gegen unendlich gegen

\pdfinfo{
  /Title (Zusammenfassung Funktionalanalysis)
  /Author (Tim Baumann)
}

\begin{document}

\maketitle{Zusammenfassung Funktionalanalysis}

\begin{nota}
  Sei im Folgenden $\K \in \{ \R, \C \}$.
\end{nota}

% TODO: Sei im Folgenden Y ein Banachraum?

\begin{defn}
  Ein \emph{Prä-Hilbertraum} ist ein $\K$-Vektorraum mit einem Skalarprodukt $\langle \cdot , \cdot \rangle$.
\end{defn}

\begin{defn}
  Sei $V$ ein $\K$-Vektorraum. Eine \emph{Fréchet-Metrik} ist eine Funktion $\rho : V \to \R_{\geq 0}$, sodass für $x, y \in V$ gilt:
  \begin{itemize}
    \item $\rho(x) = \rho(-x)$
    \item $\rho(x) = 0 \iff x = 0$
    \item $\rho(x + y) \leq \rho(x) + \rho(y)$
  \end{itemize}
\end{defn}

\begin{defn}
  Sei $(X, d)$ ein metrischer Raum und $A_1, A_2 \subset X$, so ist
  \[ \mathrm{dist}(A_1, A_2) \coloneqq \inf \Set{ d(x,y) }{ x \in A_1, y \in A_2 } \]
  der \emph{Abstand} zwischen $A_1$ und $A_2$.
\end{defn}

\begin{defn}
  Ein \emph{topologischer Raum} ist ein paar $(X, \tau)$, wobei $X$ eine Menge und $\tau \subset \mathcal{P}(X)$ ein System von offenen Mengen, sodass gilt:
  \begin{itemize}
    \item $\emptyset \in \tau$
    \item $\tilde\tau \subset \tau \implies \bigcup_{U \in \tilde\tau} U \in \tau$
    \item $U_1, U_2 \in \tau \implies U_1 \cap U_2 \in \tau$
  \end{itemize}
\end{defn}

\begin{defn}
  Ein topologischer Raum $(X, \tau)$ heißt \emph{Haussdorff-Raum}, wenn folgendes Trennungsaxiom erfüllt ist:
  \[ \forall\, x_1, x_2 \in X : \exists\,U_1, U_2 \in \tau : x_1 \in U_1 \wedge x_2 \in U_2 \wedge U_1 \cap U_2 = \emptyset \]
\end{defn}

\begin{defn}
  Sei $(X, \tau)$ ein topologischer Raum. Eine Menge $A \subset X$ heißt abgeschlossen, falls $X \setminus A \in \tau$, also das Komplement offen ist.
\end{defn}

\begin{defn}
  Sei $(X, \tau)$ ein topologischer Raum und $A \subset X$. Dann heißen
  \begin{align*}
    A^{\circ} &\coloneqq \Set{ x \in X }{ \exists\,U \in \tau \text{ mit } x \in U \text{ und } U \subset A } \\
    \overline{A} &\coloneqq \Set{ x \in X }{ \forall\,U \in \tau \text{ mit } x \in U \text{ gilt } U \cap A \not= \emptyset }
  \end{align*}
  \emph{Abschluss} bzw. \emph{Inneres} von $A$.
\end{defn}

\begin{defn}
  Ist $(X, \tau)$ ein topologischer Raum und $A \subset X$, dann ist auch $(A, \tau_A)$ ein topologischer Raum mit der \em{Relativtopologie} $\tau_A \coloneqq \{ U \cap A \,|\,U \in \tau \}$.
\end{defn}

\begin{defn}
  Sei $(X, \tau)$ ein topologischer Raum. Eine Teilmenge $A \subset X$ heißt \emph{dicht} in $X$, falls $\overline{A} = X$.
\end{defn}

\begin{defn}
  Ein topologischer Raum $(X, \tau)$ heißt \emph{separabel}, falls $X$ eine abzählbare dichte Teilmenge enthält. Eine Teilmenge $A \subset X$ heißt separabel, falls $(A, \tau_A)$ separabel ist.
\end{defn}

\begin{defn}
  Seien $\tau_1, \tau_2$ zwei Topologien auf einer Menge $X$. Dann heißt $\tau_2$ \emph{stärker} (oder feiner) als $\tau_1$ bzw. $\tau_1$ \emph{schwächer} (oder gröber) als $\tau_2$, falls $\tau_1 \subset \tau_2$.
\end{defn}

\begin{defn}
  Seien $d_1$ und $d_2$ Metriken auf einer Menge $X$ und $\tau_1$ und $\tau_2$ die induzierten Topologien. Dann heißt $d_1$ stärker als $d_2$, falls $\tau_1$ stärker ist als $\tau_2$.
\end{defn}

\begin{satz}
  Sind $\| \cdot \|_1$ und $\| \cdot \|_2$ zwei Normen auf dem $\K$-Vektorraum $X$. Dann gilt:
  \begin{itemize}
    \item $\| \cdot \|_2 \text{ ist stärker als } \| \cdot \|_1 \iff \exists\,C > 0 : \forall\,x \in X : \| x \|_1 \leq C \| x \|_2$
    \item $\| \cdot \|_1 \text{ und } \| \cdot \|_2 \text{ sind äquivalent } \iff \exists\,c, C > 0 : \forall\,x \in X : c \| x \|_1 \leq \| x \|_2 \leq C \| x \|_1$
  \end{itemize}
\end{satz}

\begin{defn}
  Die \emph{$p$-Norm} auf dem $\K^n$ ist definiert als
  \begin{align*}
    \| x \|_p &\coloneqq \left( \sum_{i = 1}^n |x_j|^p \right)^{\frac{1}{p}} \text{ für } 1 \leq p < \infty \\
    \| x \|_{\infty} &\coloneqq \| x \|_max \coloneqq \max_{1 \leq i \leq n} |x_i|.
  \end{align*}
\end{defn}

\begin{bem}
  Alle $p$-Normen sind zueinander äquivalent.
\end{bem}

\begin{defn}
  Seien $S \subset X$ eine Menge, $(X, \tau_X)$ und $(Y, \tau_Y)$ Hausdorff-Räume sowie $x_0 \in S$. Eine Funktion $f : S \to Y$ heißt \emph{stetig} in $x_0$, falls gilt:
  \[ \forall\,V \in \tau_Y : f(x_0) \in V \implies \exists\,U \in \tau_X \text{ mit } x_0 \in U \wedge f(U \cap S) \subset V \]
  Ist $X = S$, so heißt $f : X \to Y$ stetige Abbildung, falls $f$ stetig in allen Punkten $x_0 \in X$ ist, d.\,h. $V \in \tau_Y \implies f^{-1}(V) \in \tau_X$.
\end{defn}

\begin{bem}
  In metrischen Räumen ist diese Definition äquivalent zur üblichen Folgendefinition.
\end{bem}

\begin{defn}
  Sei $(X, d)$ ein metrischer Raum. Eine Folge $(x_k)_{k \in \N}$ heißt \emph{Cauchy-Folge}, falls $d(x_k, x_l) \convWith{k, l} 0$. Ein Punkt $x \in X$ heißt \emph{Häufungspunkt} der Folge, falls es eine Teilfolge $(x_{k_i})_{i \in \N}$ gibt mit $x_{k_i} - x \convWith{i} 0$.
\end{defn}

\begin{defn}
  Ein metrischer Raum $(X, d)$ heißt \emph{vollständig}, falls jede Cauchy-Folge in $X$ einen Häufungspunkt besitzt.
\end{defn}

\begin{defn}
  Ein normierter $\K$-Vektorraum heißt \emph{Banachraum}, falls er vollständig bzgl. der induzierten Metrik ist.
\end{defn}

\begin{defn}
  Ein Banachraum heißt \emph{Banach-Algebra}, falls er eine Algebra ist mit $\| x \cdot y \|_X \leq \| x \|_x \cdot \| y \|_X$.
\end{defn}

\begin{defn}
  Ein \emph{Hilbertraum} ist ein Prähilbertraum, der vollständig bzgl. der vom Skalarprodukt induzierten Norm ist.
\end{defn}

\begin{bem}
  Ein normierter Raum ist genau dann ein Prähilbertraum, falls die Parallelogrammidentität
  \[ \|x+y\|^2 + \|x-y\|^2 = 2 \|x\|^2 + 2 \|y\|^2 \]
  gilt. Folglich ist ein Banachraum genau dann ein Hilbertraum, falls die Parallelogrammidentität gilt.
\end{bem}

\begin{defn}
  Sei $\K^\N \coloneqq \Set{ (x_n)_{n \in \N} }{ \forall\,i \in \N : x_i \in \K }$ die Menge aller Folgen in $\K$. Mit der Fréchet-Metrik
    \[ \rho(x) \coloneqq \sum_{i = 1}^\infty 2^{-i} \frac{ |x_i| }{ 1 + |x_i| } < 1 \]
  wird der \emph{Folgenraum} $\K^\N$ zu einem Banachraum.
\end{defn}

\begin{satz}
  Sind $(x^k) = (x^k_i)_{i \in \N} \in \K^\N$ und $x = (x_i)_{i \in \N} \in \K^\N$, so gilt
  \[ \rho(x^k - x) \convWith{k} 0 \iff \forall\,i \in \N : x_i^k \convWith{k} x_i. \]
\end{satz}

\begin{defn}
  Die Norm
  \begin{align*}
    \| x \|_{\ell^p} &\coloneqq \left( \sum_{i=1}^\infty |x_i|^p \right)^{\frac{1}{p}} \in [0, \infty], \text{ für } 1 \leq p < \infty \\
    \| x \|_{\ell^\infty} &\coloneqq \sup_{i \in \N} |x_i| \in [0, \infty]
  \end{align*}
  heißt \emph{$\ell^p$-Norm} auf dem Raum $\ell^p(\K) \coloneqq \Set{ x \in \K^\N }{ \| x \|_{\ell^p} < \infty }$.
\end{defn}

\begin{satz}
  Der Raum $\ell^p(\K)$ ist vollständig, also ein Banachraum.
\end{satz}

\begin{bem}
  Im Fall $p = 2$ wird $\ell^2(\K)$ ein Hilbertraum mit dem Skalarprodukt $\langle x , y \rangle_{\ell^2} \coloneqq \sum_{i = 0}^\infty x_i \overline{y_i}$.
\end{bem}

\begin{defn}[Vervollständigung]
  Sei $(X, d)$ ein metrischer Raum. Betrachte die Menge $X^\N$ aller Folgen in $X$ und definiere
  \[ \tilde{X} \coloneqq \Set{ x \in X^\N }{ x \text{ ist Cauchy-Folge in } X }/\sim \]
  mit der Äquivalenzrelation
  \[ x \sim y \text{ in } \tilde{X} \iff d(x_j, y_j) \convWith{j} 0. \]
  Diese Menge wird mit der Metrik
  \[ \tilde{d}(x, y) \coloneqq \lim_{i \to \infty} d(x_i, y_i) \]
  zu einem vollständigen metrischen Raum. Die injektive Abbildung $J : X \to \tilde{X}$, welche $x \in X$ auf die konstante Folge $(x)_{i \in \N}$, ist isometrisch, d.\,h. sie erhält. Wir können also $X$ als einen dichten Unterraum von $\tilde{X}$ auffassen. Man nennt $\tilde{X}$ \emph{Vervollständigung} von~$X$.
\end{defn}

\begin{defn}[\emph{Raum der beschränkten Funktionen}]
  Sei $S$ eine Menge und $Y$ ein Banachraum über $\K$ mit Norm $y \mapsto |y|$. Dann ist
  \[ B(S; Y) \coloneqq \Set{ f : S \to Y }{ f(S) \text { ist eine beschränkte Teilmenge von } Y } \]
  die Menge der beschränkten Funktionen von $B$ nach $Y$. Diese Menge ist ein $\K$-Vektorraum und wird mit der Supremumsnorm $\| f \|_{B(S)} \coloneqq \sup_{x \in S} |f(x)|$ zu einem Banachraum.
\end{defn}

\begin{satz}
  Ist $(X, d)$ ein vollständiger metrischer Raum und $Y \subset X$ abgeschlossen, so ist auch $(Y, d)$ ein vollständiger metrischer Raum.
\end{satz}

\begin{defn}[\emph{Raum stetiger Funktionen auf einem Kompaktum}]
  Sei $S \subset \R^n$ beschränkt und abgeschlossen (d.\,h. kompakt) und $Y$ ein Banachraum über $\K$ mit Norm $y \mapsto |y|$, so ist
  \[ \mathcal{C}^0(S; Y) \coloneqq \mathcal{C}(S; Y) \coloneqq \Set{ f : S \to Y }{ f \text{ ist stetig } } \]
  die Menge der stetigen Funktionen von $S$ nach $Y$. Sie ist ein abgeschlossener Unterraum von $B(S; Y)$ mit der Norm $\| \cdot \|_{\mathcal{C}(S; Y)} = \| \cdot \|_{B(S; Y)}$, also ein Banachraum.
\end{defn}

\begin{bem}
  Für $Y = \K$ ist $\mathcal{C}^0(S; \K) = \mathcal{C}(S)$ eine kommutative Banach-Algebra mit dem Produkt $(f \cdot g)(x) \coloneqq f(x) \cdot g(x)$.
\end{bem}

\begin{defn}
  Eine Teilmenge $A \subset X$ heißt \emph{präkompakt}, falls es für jedes $\epsilon > 0$ eine Überdeckung von $A$ mit endlich vielen $\epsilon$-Kugeln $A \subset B_{\epsilon}(x_1) \cup ... \cup B_{\epsilon}(x_{n_\epsilon})$ mit $x_1, x_{n_\epsilon} \in X$ gibt.
\end{defn}

\begin{defn}
  Eine Teilmenge $A \subset X$ eines metrischen Raumes $(X, d)$ heißt \emph{kompakt}, falls eine der folgenden äquivalenten Bedinungen erfüllt ist:
  \begin{itemize}
    \item $A$ ist \emph{überdeckungskompakt}: Für jede Überdeckung $A \subset \bigcup_{i \in I} A_i$ mit $A_i \opn X$,  gibt es eine endl. Teilmenge $J \subset I$ mit $A \subset \bigcup_{i \in J} A_i$.
    \item $A$ ist \emph{folgenkompakt}: Jede Folge in $A$ besitzt eine konvergente Teilfolge mit Grenzwert in $A$.
    \item $(A, d|_A)$ ist vollständig und $A$ ist \emph{präkompakt}.
  \end{itemize}
\end{defn}

\begin{satz}
  Sei $(X, d)$ ein metrischer Raum. Dann gilt:
  \begin{itemize}
    \item $A$ präkompakt $\implies$ $A$ beschränkt,
    \item $A$ kompakt $\implies$ $A$ abgeschlossen und präkompakt,
    \item Falls $X$ vollständig, dann $A$ präkompakt $\iff$ $\overline{A}$ kompakt.
  \end{itemize}
\end{satz}

\begin{satz}
  Sei $A \subset \K^n$. Dann gilt:
  \begin{itemize}
    \item $A$ präkompakt $\iff$ $A$ beschränkt,
    \item $A$ kompakt $\iff$ $A$ abgeschlossen und beschränkt (Heine-Borel).
  \end{itemize}
\end{satz}

\begin{satz}
  Sei $(X, d)$ ein metrischer Raum und $A \subset X$ kompakt. Dann gibt es zu $x \in X$ ein $a \in A$ mit $d(x, a) = \dist(x, A)$.
\end{satz}

\begin{defn}
  Sei $S \subset \R^n$ und $(K_n)_{n \in \N}$ eine Folge kompakter Teilmengen des $\R^n$. Dann heißt $(K_n)$ eine \emph{Ausschöpfung} von $S$, falls
  \begin{itemize}
    \item $S = \cup_{n \in \N} K_n$,
    \item $\emptyset \not= K_i \subset K_{i+1} \subset S$ für alle $i \in \N$ und
    \item für alle $x \in S$ gibt es ein $\delta > 0$ und $i \in \N$, sodass $B_\delta(x) \subset K_i$.
  \end{itemize}
\end{defn}

\begin{bem}
  Zu $S \opn \R^n$ und $S \cls \R^n$ existiert eine Ausschöpfung.
\end{bem}

\begin{defn}[\emph{Raum stetiger Funktionen auf Menge mit Ausschöpfung}]
  Es sei $S \subset \R^n$ so, dass eine Ausschöpfung $(K_i)_{i \in \N}$ von $S$ existiert und $Y$ ein Banachraum. Dann bildet die Menge aller stetigen Funktionen
  \[ C^0(S; Y) \coloneqq \Set{ f : S \to Y }{ f \text{ ist stetig auf } S } \]
  einen $K$-Vektorraum und wird mit der Fréchet-Norm
  \[ \varrho(f) \coloneqq \sum_{i \in \N} 2^{-i} \frac{ \| f \|_{C^0(K_i)} }{ 1 + \| f \|_{C^0(K_i)} } \]
  zu einem vollständigen metrischen Raum.
\end{defn}

\begin{bem}
  \begin{itemize}
    \item Die von dieser Metrik erzeugte Topologie ist unabhängig von der Wahl der Ausschöpfung.
    \item Ist $S \subset \R^n$ kompakt, so stimmt die Topologie mit der von $\| \cdot \|_{B(s)}$ überein.
  \end{itemize}
\end{bem}

\begin{defn}
  Sei $S \subset \R^n$ und $Y$ ein Banachraum. Für $f : S \to Y$ heißt
  \[ \supp f \coloneqq \Set{ x \in S }{ f(x) \not= 0 } \]
  \emph{Träger} (engl. support) von $f$.
\end{defn}

\begin{defn}
  Sei $S \subset \R^n$ und $Y$ ein Banachraum. Dann ist
  \[ \mathcal{C}_0^0(S; Y) \coloneqq \Set{ f \in \mathcal{C}^0(S; Y) }{ \supp f \text{ ist kompakt in } S } \]
  die Menge der stetigen Fktn. mit kompaktem Träger von $S$ nach $Y$.
\end{defn}

\begin{defn}[\emph{Raum differenzierbarer Funktionen}]
  Sei $\Omega \subset \R^n$ offen und beschränkt und $m \in \N$. Dann ist die Menge der differenzierbaren Funktionen von $\Omega$ nach $Y$
  \begin{align*}
    \mathcal{C}^m(\overline\Omega, Y) \coloneqq \{ f : \Omega \to Y \,|\, & f \text{ ist $m$-mal stetig differenzierbar in $\Omega$ } \\
    & \text{ und für $k \leq m$ und } s_1, ..., s_k \in \{ 1, ..., n \} \\
    & \text{ ist $\partial_{s_1} ... \partial_{s_k} f$ auf $\overline\Omega$ stetig fortsetzbar } \}
  \end{align*}
  ein Vektorraum und mit folgender Norm ein Banachraum:
  \[ \| f \|_{\mathcal{C}^m(\overline\Omega)} = \sum_{|s| \leq m} \| \partial^s \|_{\mathcal{C}^0(\overline\Omega)} \]
\end{defn}

\begin{bem}
  In obiger Norm wird die Summe über alle $k$-fache partielle Ableitungen mit $k \leq m$ gebildet.
\end{bem}

% TODO: Hölder-Räume
% TODO: Unendlich differenzierbare Funktionen
% TODO: Lebesgue-Räume
% TODO: Sobolev-Räume
% TODO: Faltungen und Dirac-Folge
% TODO: Neumann-Reihe

\begin{satz}
  Sei $X$ ein normierter Raum und $Y \subset X$ ein abgeschlossener echter Teilraum. Für $0 < \Theta < 1$ (falls $X$ Hilbertraum, geht auch $\Theta = 1$) gibt es ein $x_{\Theta} \in X$ mit
  \[ \|x_0\| = 1 \quad \text{und} \Theta \leq \dist(x_{\Theta}, Y) \leq 1. \]
\end{satz}

\begin{satz}
  Für jeden normierten Raum $X$ gilt:
  \[ \overline{B_1(0)} \text{ kompakt } \iff \dim(X) < \infty. \]
\end{satz}

\begin{defn}
  Sei $S \subset \R^n$ kompakt, $Y$ ein Banachraum und $A \subset \mathcal{C}^0(S, Y)$. Dann heißt $A$ \emph{gleichgradig stetig}, falls
  \[ \sup_{f \in A} \left| f(x) - f(y) \right| \xrightarrow{\left| x - y \right| \to 0} 0. \]
\end{defn}

\begin{defn}[Arzelà-Ascoli]
  Sei $S \subset \R^n$ kompakt, $Y$ ein endlichdimensionaler Banachraum und $A \subset \mathcal{C}^0(S, Y)$. Dann gilt
  \[ A \text{ präkompakt } \iff A \text{ ist beschränkt und gleichgradig stetig. } \]
\end{defn}

\begin{satz}[Fundamentallemma der Variationsrechnung]
  Sei $\Omega \subset \R^n$ und $Y$ ein Banachraum. Für $g \in \Leb^1(\Omega, Y)$ sind dann äquivalent:
  \begin{itemize}
    \item Für alle $\xi \in \mathcal{C}_0^\infty$ gilt $\Int{\Omega}{}{(\xi \cdot g)}{x} = 0$.
    \item Für alle beschränkten $E \in \Bor(\Omega)$ mit $\overline{E} \subset \Omega$ gilt $\Int{E}{}{g}{x} = 0$.
    \item Es gilt $g \fue= 0$ in $\Omega$.
  \end{itemize}
\end{satz}

\begin{satz}
  Sei $T : X \to Y$ eine lineare Abbildung zwischen Vektorräumen $X$ und $Y$. Dann sind äquivalent:
  \begin{itemize}
    \begin{multicols}{3}
      \item $T$ ist stetig.
      \item $T$ ist stetig in $0$.
      \item $\sup_{\|x\| \leq 1} \|Tx\| < \infty$.
    \end{multicols}
    \item $\exists\,C > 0 : \forall x \in X : \|Tx\| \leq C \cdot \|x\|$.
  \end{itemize}
\end{satz}

\begin{defn}
  Seien $X, Y$ Vektorräume mit einer Topologie. Dann ist
  \[ \mathcal{L}(X, Y) = \Set{ T : X \to Y }{ X \text{ ist linear und stetig } } \]
  die Menge aller \emph{linearen Operatoren} zwischen $X$ und $Y$. Falls die Stetigkeit nicht nur topologisch, sondern bezüglich einer Norm gilt, so redet man von \emph{beschränkten Operatoren}.
\end{defn}

\begin{satz}
  Seien $X \not= \{0\}$, $Y \not= \{0\}$ Banachräume und $T, S \in \mathcal{L}(X, Y)$. Dann gilt: Falls $T$ invertierbar ist und $\|S-T\| < \tfrac{1}{\|T^{-1}\|}$, dann ist auch $S$ invertierbar.
\end{satz}

\begin{bem}
  Die Menge aller invertierbaren Operatoren in $\mathcal{L}(X, Y)$ ist somit eine offene Teilmenge.
\end{bem}

\begin{defn}
  Seien $X$ und $Y$ Banachräume über $\K$. Eine lineare Abbildung $T : X \to Y$ heißt \emph{kompakter (linearer) Operator}, falls eine der folgenden äquivalenten Bedingungen erfüllt ist:
  \begin{itemize}
    \begin{multicols}{2}
      \item $\overline{T(B_1(0))}$ ist kompakt.
      \item $T(B_1(0))$ ist präkompakt.
    \end{multicols}
    \item Für alle beschränkten $M \subset X$ ist $T(M) \subset Y$ präkompakt.
    \item Für jede beschränkte Folge $(x_n)_{n \in \N}$ in $X$ besitzt $(T x_n)_{n \in \N}$ eine in $Y$ konvergente Teilfolge.
  \end{itemize}
\end{defn}

\begin{defn}
  Sei $X$ ein Vektorraum über $\K$. Dann ist $X' \coloneqq \mathcal{L}(X, \K)$ der \emph{Dualraum} von $X$. Elemente von $X'$ werden \emph{lineare Funktionale} genannt.
  % Norm?
\end{defn}

\begin{satz}[Rieszscher Darstellungssatz]
  Ist $X$ ein Hilbertraum, so ist
  \begin{align*}
    J : X \to X', \quad x \mapsto y \mapsto (y, x)_X
  \end{align*}
  ein isometrischer konjugiert linearer Isomorphismus.
\end{satz}

\begin{satz}[Lax-Milgram]
  Sei $X$ ein Hilbertraum über $\K$ und $a : X \times X \to \K$ sesquilinear. Es gebe Konstanten $c_0$ und $C_0$ mit $0 < c_0 \leq C_0 < \infty$, sodass für alle $x, y \in X$ gilt:
  \begin{itemize}
    \item $\left| a(x, y) \right| \leq C_0 \cdot \|x\| \cdot \|y\|$ {\raggedright (Stetigkeit)}
    \item $Re a(x, x) \geq c_0 \cdot \|x\|^2$ {\raggedright (Koerzivität)}
  \end{itemize}
  Dann existiert genau eine Abbildung $A : X \to X$ mit
  \[ a(y, x) = (y, Ax) \text{ für alle $x, y \in X$. } \]
  Außerdem gilt: $A \in \mathcal{L}(X)$ ist ein invertierbarer Operator mit
  \[ \|A\| \leq C_0 \quad \text{und} \quad \|A^{-1}\| \leq \tfrac{1}{c_0}. \]
\end{satz}

\begin{satz}[Hahn-Banach]
  Sei $X$ ein $\R$-VR und
  \begin{itemize}
    \item $p : X \to \R$ sublinear, d.\,h. für alle $x, y \in X$ und $\alpha \in \R_{\geq 0}$ gelte
    \[ p(x+y) \leq p(x) + p(y) \quad \text{und} \quad p(\alpha x) = \alpha p(x), \]
    \item $f : Y \to \R$ linear auf einem Unterraum $Y \subset X$ und
    \item $f(x) \leq p(x)$ für $x \in Y$.
  \end{itemize}
  Dann gibt es eine lineare Abbildung $F : X \to \R$ mit
  \[ F(x) = f(x) \text{ für $x \in Y$} \quad \text{und} \quad F(x) \leq p(x) \text{ für } x \in X. \]
\end{satz}

\begin{satz}(Hahn-Banach für lineare Funktionale)
  Sei $X$ ein $\R$-VR, $Y \subset X$ ein Unterraum, $p : X \to \R$ linear und $f : Y \to \R$ linear, sodass $f(x) \leq p(x)$ für alle $x \in Y$. Dann existiert eine lineare Abbildung $F : X \to \R$ mit $f = F|_Y \text{ und } F \leq p$.
\end{satz}

% Korollar
\begin{satz}
  Sei $(X, \|\cdot\|_X)$ ein normierter $\K$-Vektorraum und $(Y, \|\cdot\|_Y)$ ein Unterraum. Dann gibt es zu $y \in Y'$ ein $x' \in X'$ mit $x'|_Y = y'$ und $\|x'\|_{X'} = \|y'\|_{Y'}$.
\end{satz}

\begin{satz}
  Sei $Y$ abgeschlossener Unterraum des normierten Raumes $X$ und $x_0 \in X \setminus Y$. Dann gibt es ein $x' \in X'$ mit $x'|_Y = 0$, $\|x'\|_{X'} = 1$, $\langle x', x_0 \rangle = \dist(x_0, Y)$.
\end{satz}

\begin{bem}
  Dann gibt es auch ein $x' \in X'$ mit $x'|_Y = 0$,
  \[ \|x'\|_{X'} = (\dist(x_0, Y))^{-1} \quad \text{und} \quad \langle x', x_0 \rangle = 1. \]
\end{bem}

% Bemerkung: Der Satz kann als Verallgemeinerung des Projektionssatzes für Hilberträume im linearen Fall aufgefasst werden: Ist $X$ Hilbertraum, so definiere $\langle x', x \rangle_{X' \times X} = (x | \frac{x_0 - Px_0}{\|x_0 - Px_0\|_X})_X$, wobei $P$ orthogonale Projektion auf $Y$ sei. Dann ist $x' = 0$ auf $Y$ und daher $\langle x', x_0 \rangle_{X' \times X} = \langle x', x_0 - Px_0 \rangle = \| x_0 - Px_0 \|_X$. Außerdem ist $\langle x', x \rangle \leq \|x\|_X$. Daher hat $x'$ die Eigenschaft wie im Satz.

% Korollar
\begin{satz}
  Seien $X$ normierter Raum und $x_0 \in X$. Dann gilt
  \begin{itemize}
    \item  Ist $x_0 \not= 0$, so gibt es $x_0' \in X'$ mit $\|x_0'\|_{X'} = 1$ und $\langle x_0', x_0 \rangle_{X' \times X} = \|x_0\|_X$.
    \item Ist $\langle x', x_0 \rangle_{X' \times X} = 0$ für alle $x' \in X'$, so ist $x_0 = 0$.
    \item Durch $Tx' = \langle x', x_0 \rangle_{X' \times X}$ für $x' \in X'$ ist ein $T \in \mathcal{L}(X', \K) = X''$, dem Bidualraum, definiert mit $\|T\| = \|x_0\|_X$.
  \end{itemize}
\end{satz}


% 5. Prinzip der gleichmäßigen Beschränktheit

\begin{satz}[Baire'scher Kategoriensatz]
  Es sei $X \not= \emptyset$ ein vollständiger metrischer Raum und $X = \bigcup_{k \in \N} A_k$ mit abgeschlossenen Mengen $A_k \subset X$. Dann gibt es ein $k_0 \in \N$ mit $\mathrm{int}(A_{k_0}) \not= \emptyset$.
\end{satz}

\begin{kor}
  Jede Basis eines $\infty$-dimensionalen Banachraumes ist überabzählbar.
\end{kor}

\begin{satz}[Prinzip der gleichmäßigen Beschränktheit]
  Es sei $X$ ein nichtleerer vollständiger metrischer Raum und $Y$ ein normierter Raum. Gegeben sei eine Menge von Funktionen $F \subset \mathcal{C}^0(X, Y)$ mit $\forall x \in X : \sup_{f \in F} \|f(x)\|_Y < \infty$. Dann gibt es ein $x_0 \in X$ und ein $\epsilon > 0$, sodass $\sup_{B_\epsilon(x_0)} \sup_{f \in F} \|f(x)\|_Y < \infty$.
\end{satz}

\begin{satz}[Banach-Steinhaus]
  Es sei $X$ ein Banachraum und $Y$ ein normierter Raum, $\mathcal{T} \subset \Leb(X, Y)$ mit $\forall x \in X : \sup_{T \in \mathcal{T}} \|Tx\|_Y < \infty$. DAnn ist $\mathcal{T}$ eine beschränkte Menge in $\Leb(X, Y)$, d.\,h. $\sup_{T \in \mathcal{T}} \|T\|_{\mathcal{L}(X, Y)}$.
\end{satz}

\begin{defn}
  Seien $X$ und $Y$ topologische Räume, so heißt eine Abbildung $f : X \to Y$ \emph{offen}, falls für alle offenen $U \opn X$ das Bild $f(U) \opn Y$ offen ist.
\end{defn}

\begin{bem}
  Ist $f$ bijektiv, so ist $f$ genau dann offen, wenn $f^{-1}$ stetig ist. Sind $X, Y$ normierte Räume und ist $T : X \to Y$ linear, so gilt: $T$ ist offen $\iff$ $\exists \delta > 0 : B_{\delta}(0) \subset T(B_1(0))$.
\end{bem}

\begin{satz}[von der offenen Abbildung]
  Seien $X, Y$ Banachräume und $T \in \mathcal{L}(X, Y)$. Dann ist $T$ genau dann surjektiv, wenn $T$ offen ist.
\end{satz}

\begin{satz}[von der inversen Abbildung]
  Seien $X, Y$ Banachräume und $T \in \mathcal{L}(X, Y)$ bijektiv, so ist $T^{-1}$ stetig, also $T^{-1} \in \mathcal{L}(Y, X)$.
\end{satz}

\begin{satz}[vom abgeschlossenen Graphen]
  Seien $X, Y$ Banachräume und $T : X \to Y$ linear. Dann ist $\mathrm{Graph}(T) = \Set{ (x, Tx) }{ x \in X }$ genau dann abgeschlossen, wenn $T$ stetig ist. Dabei ist $\mathrm{Graph}(T) \subset X \times Y$ mit der \emph{Graphennorm} $\|(x,y)\|_{X \times Y} = \|x\|_X + \|y\|_Y$.
\end{satz}

\begin{defn}
  Sei $X$ ein Banachraum.
  \begin{itemize}
    \item Eine Folge $(x_k)_{k \in \N}$ in $X$ \emph{konvergiert schwach} gegen $x \in X$ (notiert $x_k \convWeaklyWith{k} x$), falls für alle $x' \in X'$ gilt:
    \[ \langle x', x_k \rangle_{X' \times X} \convWith{k} \langle x', x \rangle_{X' \times X} \]
    \item Eine Folge $(x'_k)_{k \in \N}$ in $X'$ \emph{konvergiert schwach*} gegen $x' \in X'$ (notiert $x'_k \convWeaklyStarWith{k} x'$), falls für alle $x \in X$ gilt:
    \[ \langle x'_k, x \rangle_{X' \times X} \convWith{k} \langle x', x \rangle_{X' \times X} \]
    \item Analog sind \emph{schwache} und \emph{schwache* Cauchyfolgen} definiert.
    \item Eine Menge $M \subset X$ (bzw. $M \subset X'$) heißt \emph{schwach folgenkompakt} bzw. \emph{schwach* folgenkompakt}, falls jede Folge in der Menge $M$ eine schwach (bzw. schwach*) konvergente Teilfolge besitzt deren Grenzwert wieder in $M$ liegt.
  \end{itemize}
\end{defn}

\begin{bem}
  Der schwache bzw. schwache* Grenzwert einer Folge ist eindeutig bestimmt. Starke Konvergenz impliziert schwache Konvergenz.
\end{bem}

\begin{satz}
  Es gilt für $x, x_k \in X$, $x', x'_k \in X'$:
  \begin{align*}
    x_k \convWeaklyWith{k} x \,\, \text{ in } X \, \quad &\iff \quad J_x x_k \convWeaklyStarWith{k} J_x x \text{ in } X'' \\
    x'_k \convWeaklyWith{k} x' \text{ in } X' \quad &\implies \qquad \,\, x'_k \convWeaklyStarWith{k} x' \text{ in } X'
  \end{align*}
\end{satz}

\begin{lem}
  \begin{itemize}
    \item Aus $x'_k \convWeaklyStarWith{k} x'$ in $X'$ folgt $\|x'\|_{X'} \leq \liminf_{k \to \infty} \|x'_k\|_{X'}$, aus $x_k \convWeaklyWith{k} x$ in $X$ folgt $\|x\|_X \leq \liminf_{k \to \infty} \|x_k\|_X$.
    \item Schwach bzw. schwach* konvergente Folgen sind beschränkt.
    \item Aus $x_k \convWith{k} x$ in $X$ und $x'_k \convWeaklyStarWith{k} x'$ in $X'$ folgt $\langle x'_k, x_k \rangle_{X' \times X} \convWith{k} \langle x', x \rangle_{X' \times X}$. Dasselbe folgt mit $x_k \convWeaklyWith{k} x$ in $X$ und $x'_k \convWith{k} x'$ in $X'$.
  \end{itemize}
\end{lem}

\begin{acht}
  In der letzten Behauptung müssen wir vorraussetzen, dass mindestens eine Folge stark konvergiert. Für beidesmal schwache/schwache* Konvergenz ist die Aussage i.\,A. falsch.
\end{acht}

\iffalse
\begin{bsp}
  \begin{itemize}
    \item Sei $1 \leq p < \infty$ und $\tfrac{1}{p} + \tfrac{1}{p'} = 1$ und $\omega \subset \R^n$ beschränkt. Dann gilt für $f, f_k \in L^p(\Omega)$:
    \[ f_k \convWeaklyWith{k} f \text{ in } L^p(\Omega) \iff \forall g \in L^p(\Omega) : \IntO{f_k \cdot \overline{g}}{x} \convWith{k} \IntO{f \cdot \overline{g}}{x} \]
    \item Sei $\Omega \subset \R^n$ boffen und beschränkt und $1 \leq p \leq \infty$, $m \in \N_{> 0}$. Für $u, u_k \in W^{m,p}(\Omega)$ gilt dann:
    \[ u_k \convWeaklyWith{k} u \text{ in } W^{m,p}(\Omega) \iff \forall s, |s| \leq m : \partial^s u_k \convWeaklyWith{k} \partial^s u \text{ in } L^p(\Omega) \]
  \end{itemize}
\end{bsp}
\fi

\begin{satz}[Banach-Alaoglu]
  Sei $X$ ein separabler Banachraum. Dann ist die abgeschl. Einheitskugel $\overline{B_1(0)} \subset X'$ schwach* folgenkompakt.
\end{satz}

\begin{bsp}
  Sei $\Omega \subset \R^n$ beschränkt und offen. Dann ist $L^1(\Omega)$ separabel (Approximation durch Treppenfunktionen und der Satz besagt: Ist $(f_k)_{k \in \N}$ in $L^{\infty}(\Omega)$ beschränkt, so gibt es eine Teilfolge $(f_{k_l})_{l \in \N}$ und ein $f \in L^\infty(\Omega)$, sodass
  \[ \IntO{f_{k_l}{x} \cdot \overline{g}} \xrightarrow{l \to \infty} \IntO{f \cdot \overline{g}}{x} \quad \text{für alle $g \in L^1(\Omega)$} \]
\end{bsp}

\begin{bem}
  Schwach*-Konvergenz impliziert eine sogenannte Schwach*-Topologie in dem Sinne, dass man sagt, eine Folge $(x_k')_{k \in \N}$ in $X'$ ist bzgl. dieser Topologie konvergent, wenn sie punktweise für alle $x \in X$ konvergiert.
\end{bem}

\begin{defn}
  Sei $X$ ein Banachraum und $J_X$ die Isometrie bzgl. des Bidualraumes. Dann heißt $X$ \emph{reflexiv}, falls $J_X$ surjektiv ist.
\end{defn}

\begin{lem}
  \begin{itemize}
    \item Ist $X$ reflexiv, so stimmen schwache* und schwache konvergenz in $X'$ überein.
    \item Ist $X$ reflexiv, so ist jeder abgeschlossene Unterraum von $X$ reflexiv.
    \item Ist $T : X \to Y$ ein Isomorphismus, so gilt:
      \[ X \text{ reflexiv } \iff Y \text{ reflexiv } \]
    \item Es gilt: $X$ reflexiv $\iff$ $X'$ reflexiv.
  \end{itemize}
\end{lem}

\begin{lem}
  Für jeden Banachraum $X$ gilt: $X'$ separabel $\implies$ $X$ separabel.
\end{lem}

\begin{bem}
  Die Umkehrung gilt i.\,A. nicht! Gegenbeispiel: $X = L^1$.
\end{bem}

\begin{satz}[Eberlein-Shmulyan]
  Sei $X$ reflexiver Banachraum. Dann ist die abgeschlossene Einheitskugel $\overline{B_1(0)} \subset X$ schwach folgenkompakt.
\end{satz}

\begin{bsp}
  \begin{itemize}
    \item Hilberträume $X$ sind reflexiv (folgt direkt aus dem Riesz'schen Darstellungssatz; im Reellen $J_X = (R_X R_{X'})^{-1}$, wobei $R_X : X \to X'$ der zugehörige isomorphismus). Daher: Ist $(x_k)_{k \in \N}$ eine beschränkte Folge in $X$, so existiert eine Teilfolge $(x_{k_l})_{l \in \N}$ und $x \in X$, sodass
    \[ (y | x_{k_l})_X \xrightarrow{l \to \infty} (y | x)_X \]
    für alle $y \in X$.
    \item Sei $\Omega \subset \R^n$ beschränkt, $1 < p < \infty$, $\tfrac{1}{p} + \tfrac{1}{p'} = 1$. Dann ist $L^p(\Omega)$ reflexiv.
    \item $L^1$ und $L^\infty$ sind genau dann nicht reflexiv, wenn sie unendlich-dimensional sind.
  \end{itemize}
\end{bsp}

\begin{bem}
  Analog zur schwach*-Topologie kann man auch eine schwache Topologie einführen.
\end{bem}

% Minkowski-Funktional

% In Paragraph 2.1 haben wir gesehen, dass das Abstandsproblem zu konvexen, abgeschlossenen Mengen in allgemeinen Banachräumen nicht lösbar ist, in reflexiven aber doch, wie wir seher werden.

% Unterkapitel Minkowski-Funktional

\begin{satz}[Trennungssatz]
  Seien $X$ ein normierter Raum, $M \subset X$ nicht leer, abgeschlossen, konvex und $x_0 \in X \setminus M$. Dann gibt es ein $x' \in X'$ und ein $\alpha \in \R$ mit
  \[ Re \langle x', x_0 \rangle_{X' \times X} > \alpha \quad \text{und} \quad Re \langle x', x \rangle_{X' \times X} \leq \alpha \enspace \text{für } x \in M. \]
\end{satz}

% Geometrische Interpretation

% Häh?
% \begin{bem}
%  Es folgt $x' \not= 0$, also ist $\Set{ x \in X }{ Re \langle x', x \rangle_{X' \times X} = \alpha }$
% \end{bem}

\begin{satz}
  Sei $X$ ein normierter Raum, $M \subset X$ konvex und abgeschlossen. Dann ist $M$ schwach folgenabgeschlossen, d.\,h. sind $x_k, x \in X$ für $k \in \N$, so gilt
  \[ \forall \, k \in \N \,:\, x_k \in M, x_k \convWeaklyWith{k} x \text{ in } X \implies x \in M \]
\end{satz}

\begin{lem}[Mazur]
  Sei $X$ normierter Raum und $(x_k)_{k \in \N}$ Folge in $X$ mit $x_k \convWeaklyWith{k} x$. Dann gilt $x \in \mathrm{conv} \Set{ x_k }{ k \in \N }$
\end{lem}

\begin{satz}
  Sei $X$ ein reflexiver Banachraum und $M \subset X$ nicht leer, konvex, abgeschlossen. Dann gibt es zu $\tilde{x}$ ein $x \in M$ mit $\| x - \tilde{x} \| = \mathrm{dist}(\tilde{x}, M)$.
\end{satz}

% Der Satz über die schwache Folgenabgeschlossenheit konvexer abgeschlossener Mengen spielt eine wichtige Rolle für Existenzaussagen für partielle Differentialgleichungen mit Nebenbedingungen

% Sei dazu $\Omega \subset \R^n$ offen, beschränkt, zusammenhängend mit $\mathcal{C}^{0,1}$-Rand sowie $\K = \R$. Sei $a_{ij} = a_{ji} \in L^\infty(\Omega)$, $i,j = 1, ..., n$ gleichmäßig elliptisch, $f \in L^2(\Omega)$ sowie $E(u) = \Int{}{}{ \tfrac{1}{2} \sum_{i,j=1}^n \partial_i u a_{ij} \partial_j u + f u }{x}$, $E : W^{1,2}(\Omega) \to \R$.

% Für $M \subset W^{1,2}(\Omega)$ nicht leer, konvex, abgeschlossen mit $\forall \, u_0 \in M \,:\, \exists \, C_0 < \infty \,:\, \forall \, \xi \in \R \,:\, u_0 + \xi \in M \implies \abs{\xi} \leq C$ gilt dann (ohne Beweis)
% \begin{itemize}
%   \item $E$ besitzt auf $M$ ein absolutes Minimum
%   \item Die absoluten Minima von $E$ auf $M$ sind genau die Lösungen der Variationsungleichung: Finde $u \in M$, sodass $\Int{\Omega}{}{\sum_{i,j=1}^n \partial_i (u-v) a_{ij} \partial_j u + (u-v) f}{x}$ für alle $v \in M$.
%   \item Ist $M$ ein abgeschlossener affiner Unterraum $M = u_0 + M_0$ mit $M_0$ Unterraum, so ist die Variationsungleichung äquivalent zu: Finde $u \in M$, sodass $\Int{\Omega}{}{\sum_{i,j=1}^n \partial_i v a_{ij} u + v f}{x} = 0$ für alle $v \in M_0$
%   \item Hat $M$ die Eigenschaft $v \in M$, $\xi \in \R$ mit $v + \xi \in M$ $\implies \xi = 0$, so ist die Lösung eindeutig.
% \end{itemize}

\begin{bsp}
  \begin{itemize}
    \item Sei $M = W_0^{1,2}(\Omega)$. Dann ist die eindeutige Lösbarkeit des zugehörigen (schwachen) Dirichlet-Problems gesichert.
    \item Sei $M = \Set{ u \in W^{1,2}(\Omega) }{ \Int{\Omega}{}{u}{x} = 0 }$ und gelte $\Int{\Omega}{}{f}{x} = 0$. Dann sichern Punkt 3, 4 die eindeutige Lösbarkeit des zugehörigen Neumann-Problems.
    \item Seien $u_0, \psi_0 \in W^{1,2}(\Omega)$ gegeben und $u_0(x) \geq \phi_0(x)$ für fast alle $x \in \Omega$. Definiere $M = \Set{ v \in W^{1,2}(\Omega) }{ v = u_0 \text{ auf } \partial \Omega, v \geq \psi \text{ in } \Omega }$. Dann sichern die Punkte 1 bzw. 2 und 4 die eindeutige Existenz einer Lösung dieses Hindernis-Problems.
  \end{itemize}
\end{bsp}


% Kapitel 7. Endlich-dimensionale Approximation

% Für numerische Berechnungen ist die Approximation von Elementen oder Unterräumen unendlich-dimensionaler Räume mit endlich-dimensionaler Information von großer Relevanz. Eine abzählbare Approximation durch endlich-dimensionale Unterräume ist nur für separable Räume möglich:

\begin{lem}
  Ist $X$ $\infty$-dimensionaler Raum, so sind äquivalent:
  \begin{itemize}
    \item $X$ ist separabel
    \item $\exists X_n \subset X$ endlich-dim. Unterräume : $\forall \,n \in \N \,:\, X_n \subset X_{n{+}1}$ und $\bigcup_{n \in \N} X_n$ ist dicht in $X$.
    \item $\exists X_n \subset X$ endlich-dim. Unterräume : $E_n \cap E_m = \{ 0 \}$ für $n \not= m$ und $\bigcup_{n \in \N} (E_0 \oplus ... \oplus E_n)$ ist dicht in $X$.
    \item $\exists$ linear unabhängige Menge $\Set{ e_n }{ n \in \N }$ mit $\mathrm{span}\Set{e_n}{ n \in \N }$ ist dicht in $X$.
  \end{itemize}
\end{lem}

% Aussage (1) bedeutet, dass es zu $x \in X$ Punkte $x_n \in X$ gibt mit $x_n \xrightarrow{n \to \infty} x$
% Aussage (4) bedeutet, dass es zu $x \in X$ und $n \in \N$ fr $k = 0, ...., n$ Zahlen $\alpha_{n,k}$ gibt mit $\| x - \sum_{k=0}^n \alpha_{n,k} \|_X \xrightarrow{n \to \infty} 0$
% Falls die $\alpha_{n,k}$ unabhängig von $u$ gewählt werden können:

\begin{defn}
  Sei $X$ normierter Raum. Eine Folge $(x_n)_{n \in \N}$ heißt \emph{Schauder-Basis} von $X$, falls:
  \[ \forall \, x \in X \,:\, \exists \text{ eindeutige bestimmte } \alpha_k \in \K \,:\, \sum_{k=0}^n \alpha_n e_k \xrightarrow{n \to \infty} x \text{ in } X. \]
\end{defn}

% Sind $X^1$, $X^2$ Banachräume mit Schauderbasen $(e_k^1)_{k \in \N}$ und $(e_l^2)_{l \in \N}$ und $S \in \mathcal{L}(X^1, X^2)$, so haben $Se_k^1$ Darstellungen bzgl. $(e^2_l)_{l \in \N}$, d.\,h. es ex. eindeutig bestimmte $a_{k,l} \in \K$ mit $Se_k^1 = \sum_{l=1}^\infty a_{k,l} e_l^2$, also gilt für $x = \sum_{k=0}^\infty \alpha_k e_k^1$:
% \[ Sx = \sum_{k=0}^\infty \sum_{i=0}^\infty \alpha_k a_{k,l} e_l^2 \]

$S$ ist also eindeutig bestimmt durch die "`unendliche Matrix"' $(a_{k,l})_{k,l \in \N}$.

% Kapitel 7.1. Orthogonalsysteme

\begin{defn}
  Sei $X$ ein Prähilbertraum. Eine Folge $(e_k)_{k \in \N}$, $N \subset \N$ in $X$ heißt \emph{Orthogonalsystem}, falls $(e_k | e_l) = 0$ für $k \not= l$ und $e_k \not= 0$ für alle $k \in \N$ und \emph{Orthonormalsystem}, falls zusätzlich $\|e_k\| = 1$ für alle $k \in \N$ gilt.
\end{defn}

\begin{lem}[Besselsche Ungleichung]
  Sei $(e_k)_{k \in \N}$ ein (endliches) Orthonormalsystem des Prähilbertraumes $X$. Dann gilt für alle $x \in X$: $0 \leq \|x\|^2 - \sum_{k=0}^n \abs{(x|e_k)}^2 = \| x - \sum_{k=0}\infty (x | e_k) e_k \|^2 = \dist(x, \mathrm{span} \{ e_0, ..., e_n \})^2$.
\end{lem}

\begin{satz}
  Sei $(e_k)_{k \in \N}$ ein Orthonormalsystem des Prä-Hilbertraumes $X$. Dann sind äquivalent:
  \begin{itemize}
    \item $\mathrm{span} \Set{ e_k }{ k \in \N }$ liegt dicht in $X$
    \item $(e_k)_{k \in \N}$ ist eine Schauder-Basis von $X$.
    \item Für alle $x \in X$ $x = \sum_{k=0}^\infty (x | e_k) e_k$ (Darstellung)
    \item Für alle $x, y \in X$ gilt $(x|y) = \sum_{k=0}^\infty (x|e_k) \overline{(y|e_k)}$ (Parseval-Identität)
    \item Für alle $x \in X$ gilt $\| x \|^2 = \sum_{k=0}^\infty \abs{(x|e_k)}^2$
  \end{itemize}
\end{satz}

\begin{defn}
  Ist eine dieser Bedingungen erfüllt, nennen wir die $(e_k)_{k \in \N}$ \emph{Orthonormalbasis}.
\end{defn}

\begin{satz}
  Jeder $\infty$-dim. Hilbertraum über $\K$ ist genau dann $X$ separabel, wenn $X$ eine Orthonormalbasis besitzt.
\end{satz}

\begin{bem}
  In diesem Fall ist $X$ isometrisch isomorph zu $\mathcal{l}^l(\K)$ (Übergang zu Koeffizienten bzgl. Basis)
\end{bem}

\end{document}
