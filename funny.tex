\documentclass{cheat-sheet}

\newcommand{\K}{\mathbb{K}}
\newcommand{\Bor}{\mathfrak{B}} % Borel
\newcommand{\Leb}{\mathcal{L}} % Lebesgue
\newcommand{\fue}{\overset{\text{f.ü.}}} % fast überall
\newcommand{\dist}{\mathrm{dist}} % Entfernung (distance)
\newcommand{\diam}{\mathrm{diam}} % Durchmesser (diameter)
\newcommand{\codim}{\mathrm{codim}} % Kodimension
\newcommand{\scp}[2]{\left( #1 \!\mid\! #2 \right)} % Skalarprodukt
\newcommand{\dup}[2]{\langle #1 , #2 \rangle} % Duale Paarung
\newcommand{\inte}{\mathop{\mathrm{int}}} % Inneres (interior)
\newcommand{\clos}{\mathop{\mathrm{clos}}} % Abschluss (closure)
\newcommand{\bdry}{\mathop{\mathrm{bdry}}} % Rand (boundary)

\newcommand{\IntO}[2]{\Int{\Omega}{}{#1}{#2}} % Integral über \Omega

\newcommand{\convWith}[1]{\xrightarrow{#1 \to \infty}} % konvergiert für #1 gegen unendlich gegen
\newcommand{\convWeaklyWith}[1]{\xrightharpoonup{#1 \to \infty}} % konvergiert schwach für #1 gegen unendlich gegen
\newcommand{\convWeaklyStarWith}[1]{\xrightharpoonup[*]{#1 \to \infty}} % konvergiert schwach für #1 gegen unendlich gegen

\pdfinfo{
  /Title (Zusammenfassung Funktionalanalysis)
  /Author (Tim Baumann)
}

\begin{document}

\maketitle{Zusammenfassung Funktionalanalysis}

% Thema der Vorlesung:
% Kurz gesagt: Um unendlich-dimensionale Vektorräume und (lineare und stetige) Abbildungen zwischen solchen

% Beispiele:
% * Sei $\Omega \subset \R^n$ offen und beschränkt. Dann ist $\mathcal{C}(\overline{\Omega}) = \Set{ f : \Omega \to \R }{ f \text{ stetig} }$ mit der Norm $\norm{f}_\infty = \sup_{\omega \in \Omega} \abs{f(\omega)}$ ein Banachraum, d.\,h. ein vollständiger, normierter Vektorraum. Er ist allerdings nicht endlich-dimensional (ausgelassen: Begründung dafür).
% * Für eine lineare Abbildung zwischen endlich-dimensionalen Vektorräumen ist aus der linearen Algebra bekannt, dass sie injektiv genau dann sind, wenn sie surjektiv sind. Sind diese Vektorräume unendlich-dimensional ist dies i.\,A. falsch. Z.B. sei $C_* \coloneqq \Set{ (x_k)_{k \in \N} \text{ Folge in } \R }{ \ex{N \in \N} \fa{n \geq N} x_n = 0 }$ der Vektorraum der irgendwann abbrechenden Folgen. Der Shiftoperator
%   \[ T : C_* \to C_*, (x_0, x_1, x_2, ...) \mapsto (0, x_0, x_1, x_2, ...) \]
%   ist offensichtlich linear und injektiv, aber nicht surjektiv.
% * Sturm-Liouville-Problem (Details ausgelassen)

% Kapitel 1.

% Kapitel 1.1. Strukturen

\begin{nota}
  Sei im Folgenden $\K \in \{ \R, \C \}$.
\end{nota}

\begin{defn}
  Sei $X$ ein $\K$-Vektorraum. Eine \emph{Halbnorm} ist eine Abb. $\norm{\blank} : X \to \R, x \mapsto \norm{x}$, sodass für alle $x, y \in X$ und $\alpha \in \K$ gilt:
  \begin{itemize}
    \begin{multicols}{2}
      \item $\norm{x} \geq 0$ \enspace (Positivität)
      \item $\norm{\alpha x} = \abs{\alpha} \cdot \norm{x}$ \enspace (Homogenität)
    \end{multicols}
    \item $\norm{x + y} \leq \norm{x} + \norm{y}$ \enspace ($\triangle$-Ungleichung)
  \end{itemize}
  Eine \emph{Norm} ist eine Halbnorm, für die zusätzlich gilt:
  \[ \norm{x} = 0 \enspace \iff \enspace x = 0. \]
\end{defn}

\begin{defn}
  Sei $X$ ein $\K$-Vektorraum.
  \begin{itemize}
    \item Eine Abbildung $f : X \times X \to \K$ heißt \emph{Sesquilinearform}, wenn für alle $x, x_1, x_2, y, y_1, y_2 \in X$ und $\alpha \in \K$ gilt:
    \begin{align*}
      f(\alpha x_1 + x_2, y) &= \alpha f(x_1, y) + f(x_2, y) \tag*{(Linearität im 1. Arg)} \\
      f(x, \alpha y_1 + y_2) &= \overline{\alpha} f(x, y_1) + f(x, y_2) \tag*{(Antilinearität im 2. Arg)}
    \end{align*}
    \item Eine \emph{Hermitische Form} $f$ ist eine Sesquilinearform, für die gilt:
    \[ \fa{x, y \in X} f(x, y) = \overline{f(y, x)} \tag*{(Symmetrie)} \]
    Für alle $x \in X$ gilt dann $f(x, x) = \overline{f(x, x)}$, also ist $f(x, x)$ reell.
    \item Eine Sesquilinearform $f$ heißt \emph{positiv semidefinit}, falls $f(x, x) \geq 0$ für alle $x \in X$ gilt. Falls zusätzlich $f(x, x) = 0$ genau dann gilt, wenn $x = 0$, dann heißt $f$ \emph{positiv definit}.
    \item Ein \emph{Skalarprodukt} ist eine positiv definite Hermitische Form
    \[ \scp{\blank}{\blank} : X \times X \to \K, \quad (x, y) \mapsto \scp{x}{y}. \]
  \end{itemize}
\end{defn}

\begin{satz}
  Für eine positiv semidefinite Hermitische Form $\scp{\blank}{\blank}$ ist durch $x \mapsto \sqrt{\scp{x}{x}}$ eine Halbnorm definiert. Ist die Form auch positiv definit, also ein Skalarprodukt, handelt es sich dabei um eine Norm, die sogenannte \emph{induzierte Norm}.
\end{satz}

\begin{satz}
  Für ein Skalarprodukt $\scp{\blank}{\blank}$ auf einem $\K$-VR $X$ und die davon induzierte Norm gilt für alle $x, y \in X$:
  \begin{itemize}
    \item $\abs{\scp{x}{y}} \leq \norm{x} \cdot \norm{y}$ \pright{Cauchy-Schwarzsche Ungleichung}
    \item $\norm{x+y}^2 + \norm{x-y}^2 = 2 (\norm{x}^2 + \norm{y}^2)$ \pright{Parallelogrammidentität}
  \end{itemize}
  Gleichheit gilt bei CS genau dann, wenn $x$ und $y$ gleichgerichtet sind.
\end{satz}

\begin{defn}
  Ein $\K$-VR mit einer Norm heißt \emph{normierter Raum}, mit einem Skalarprodukt \emph{Prähilbertraum}.
\end{defn}

\begin{defn}
  Sei $X$ ein Prähilbertraum. Zwei Vektoren $x, y \in X$ heißen \emph{zueinander orthogonal}, notiert $x \perp y$, wenn $\scp{x}{y} = 0$.
\end{defn}

\begin{satz}
  Für zwei orthogonale Vektoren $x, y \in X$ gilt
  \[ \norm{x - y}^2 = \norm{x + y}^2 = \norm{x}^2 + \norm{y}^2. \tag*{(Pythagoras)} \]
\end{satz}

\begin{lem}
  Seien $Y$ und $Z$ Unterräume eines VR $X$, dann ist auch $Y + Z \coloneqq \Set{ y + z }{ y \in Y, z \in Z }$ ein Unterraum von $X$.
\end{lem}

\begin{defn}
  Für Unterräume $Y$ und $Z$ eines VR $X$ mit $Y \cap Z = \{ 0 \}$ heißt $Y \oplus Z \coloneqq Y + Z$ \emph{direkte Summe} von $Y$ und $Z$.
\end{defn}

\begin{defn}
  Zwei Unterräume $Y$ und $Z$ von $X$ heißen \emph{orthogonal}, notiert $Y \perp Z$, falls $\fa{y \in Y, z \in Z} y \perp z$.
\end{defn}

\begin{defn}
  Für einen $\K$-VR $X$ und einen Unterraum $Y \subset X$ heißt
  \[ Y^\perp \coloneqq \Set{ x \in X }{ \mathrm{span} \{ x \} \perp Y } \quad \text{\emph{orthog. Komplement} von $Y$}. \]
\end{defn}

% Ausgelassen: Restklassenbildung über Halbnorm

\begin{defn}
  Ein \emph{metrischer Raum} ist ein Paar $(X, d)$ mit einer Mange $X$ und einer \emph{Metrik} $d : X {\times} X \to \R$, d.\,h. für $x, y, z \in X$ gilt:
  \begin{itemize}
    \item $d(x, y) \geq 0$ \enspace und \enspace $d(x, y) = 0 \iff x = y$ \pright{Positivität}
    \miniitem{0.44 \linewidth}{$d(x, y) = d(y, x)$ (Symm.)}
    \miniitem{0.54 \linewidth}{$d(x, z) \leq d(x, y) + d(y, z)$ \enspace ($\triangle$-Ungl.)}
  \end{itemize}
\end{defn}

% Ausgelassen: Definition Halbmetrik (ohne Axiom $d(x, y) = 0 \iff x = y$)

\begin{defn}
  Sei $V$ ein $\K$-Vektorraum. Eine \emph{Fréchet-Metrik} ist eine Funktion $\rho : V \to \R_{\geq 0}$, sodass für alle $x, y \in V$ gilt:
  \begin{itemize}
    \miniitem{0.25 \linewidth}{$\rho(x) = \rho(-x)$}
    \miniitem{0.36 \linewidth}{$\rho(x) = 0 \iff x = 0$}
    \miniitem{0.36 \linewidth}{$\rho(x + y) \leq \rho(x) + \rho(y)$}
  \end{itemize}
\end{defn}

\begin{bsp}
  Auf dem $\R^n$ ist $x \mapsto \tfrac{\norm{x}}{1 + \norm{x}}$ eine Fréchet-Metrik.
\end{bsp}

\begin{defn}
  Sei $(X, d)$ ein metrischer Raum und $A, B \subset X$, so heißt
  \[ \dist(A_1, A_2) \coloneqq \inf \Set{ d(x,y) }{ x \in A_1, y \in A_2 } \quad \text{\emph{Abstand} zw. $A$ und $B$.} \]
\end{defn}

% Ausgelassen: Für $x \in X$ und $A \subset X$ schreibe $\dist(x, A) = \dist(\{ x \}, A)$.

\begin{bem}
  Für $A \subset X$ ist die Abbildung $x \mapsto \dist(x, A)$ Lipschitz-stetig mit Lipschitz-Konstante $\leq 1$.
\end{bem}

\begin{defn}
  Sei $(X, d)$ metrischer Raum, $A \subset X$, $\epsilon > 0$, dann heißt
  \[ B_\epsilon(A) \coloneqq \Set{ y \in X }{ \dist(\{ y \}, A) < A} \quad \text{\emph{$\epsilon$-Umgebung} von $A$.} \]
  Für $x \in X$ ist $B_\epsilon(x) \coloneqq B_\epsilon(\{ x \})$ die \emph{$\epsilon$-Kugel} um $x$.
\end{defn}

\begin{defn}
  Der Durchmesser von $A \subset X$ ist definiert durch
  \[ \diam(\emptyset) \coloneqq 0, \quad \diam(A) \coloneqq \sup \Set{ d(x, y) }{ x, y \in A } \text{ für } A \not= \emptyset. \]
\end{defn}

\begin{defn}
  $A {\subset} X$ mit $\diam(A) < \infty$ heißt \emph{beschränkt}.
\end{defn}

\begin{defn}
  Sei $(X, d)$ ein normierter Raum und $A \subset X$, dann heißt
  \begin{itemize}
    \item $\inte A \coloneqq A^\circ \coloneqq \Set{x \in X}{ \ex{\epsilon > 0} \! B_e(x) \subset A }$ das \emph{Innere} von $A$,
    \item $\clos A \coloneqq \overline{A} \coloneqq \Set{x \in X}{ \fa{\epsilon > 0} \! B_\epsilon(x) {\cap} A \not= \emptyset }$ \emph{Abschluss} von $A$,
    \item $\bdry A \coloneqq \partial A \coloneqq \overline{A} \setminus A^\circ$ \emph{Rand} von $A$,
    \item $A^c \coloneqq \complement A \coloneqq X \setminus A$ \emph{Komplement} von $A$.
  \end{itemize}
\end{defn}

\begin{defn}
  Eine Menge $A \subset X$ heißt \emph{offen}, falls $A = A^\circ$, und \emph{abgeschlossen}, falls $A = \overline{A}$.
\end{defn}

% Thema: Topologie

\begin{defn}
  Ein \emph{topologischer Raum} ist ein Paar $(X, \tau)$, wobei $X$ eine Menge und $\tau \subset \mathcal{P}(X)$ ein System von Teilmengen von $X$, den sogenannten \emph{offenen} Mengen, ist, sodass gilt:
  \vspace{-4pt}
  \begin{itemize}
    \miniitem{0.24 \linewidth}{$\emptyset \in \tau, X \in \tau$}
    \miniitem{0.34 \linewidth}{$\fa{\widetilde\tau \subset \tau} \bigcup^{\phantom{\mathclap{\widetilde\tau \subset U}}}_{\mathclap{U \in \widetilde\tau}} U \in \tau$}
    \miniitem{0.39 \linewidth}{$\fa{U_1, U_2 \in \tau} U_1 \cap U_2 \in \tau$}
  \end{itemize}
\end{defn}

\begin{defn}
  Sei $(X, \tau)$ ein topolischer Raum. Eine Menge $A \subset X$ heißt \emph{abgeschlossen}, wenn das Komplement offen ist, also $A^c \in \tau$.
\end{defn}

\begin{defn}
  Ein \emph{Hausdorff-Raum} ist ein topologischer Raum $(X, \tau)$, der folgendes Trennungsaxiom erfüllt:
  \[ \fa{x_1, x_2 \in X} \ex{U_1, U_2 \in \tau} x_1 \in U_1 \wedge x_2 \in U_2 \wedge U_1 \cap U_2 = \emptyset \]
\end{defn}

\begin{defn}
  Ist $(X, \tau)$ ein topologischer Raum und $A \subset X$, dann ist auch $(A, \tau_A)$ ein topologischer Raum mit der sogenannten \emph{Relativtopologie} $\tau_A \coloneqq \{ U \cap A \,|\,U \in \tau \}$.
\end{defn}

\begin{defn}
  Sei $(X, \tau)$ ein topol. Raum und $A \subset X$, dann heißt
  \begin{itemize}
    \item $A^\circ \coloneqq \Set{x \in X}{ \ex{U \in \tau} \! x \in U \text{ und } U \subset A }$ das \emph{Innere} von $A$,
    \item $\overline{A} \coloneqq \Set{x \in X}{ \fa{U \in \tau \text{ mit } x \in U} U \cap A \not= \emptyset }$ \emph{Abschluss} von $A$.
    % TODO: Rand?
  \end{itemize}
\end{defn}

\begin{defn}
  Sei $(X, d)$ ein metrischer Raum. Dann ist
  \[ (X, \tau) \quad \text{mit} \quad \tau \coloneqq \Set{A \subset X}{ \inte A = A } \]
  ein topol. Raum, wobei $\tau$ die von $d$ \emph{induzierte Topologie} heißt.
\end{defn}

\begin{bem}
  Die direkte Definitionen des Abschlusses, des Inneren, usw. für metrische Räume stimmen mit den Definitionen dieser Begriffe über die induzierte Topologie überein.
\end{bem}

\begin{defn}
  Sei $(X, \tau)$ ein topologischer Raum. Eine Teilmenge $A \subset X$ heißt \emph{dicht} in $X$, falls $\overline{A} = X$.
\end{defn}

\begin{defn}
  Ein topologischer Raum $(X, \tau)$ heißt \emph{separabel}, falls $X$ eine abzählbare dichte Teilmenge enthält. Eine Teilmenge $A \subset X$ heißt separabel, falls $(A, \tau_A)$ separabel ist.
\end{defn}

\begin{defn}
  Seien $\tau_1, \tau_2$ zwei Topologien auf einer Menge $X$. Dann heißt $\tau_2$ \emph{stärker} (oder feiner) als $\tau_1$ bzw. $\tau_1$ \emph{schwächer} (oder gröber) als $\tau_2$, falls $\tau_1 \subset \tau_2$.
\end{defn}

\begin{defn}
  Seien $d_1$ und $d_2$ Metriken auf einer Menge $X$ und $\tau_1$ und $\tau_2$ die induzierten Topologien. Dann heißt $d_1$ \emph{stärker} als $d_2$, falls $\tau_1$ stärker ist als $\tau_2$. Ist $\tau_1 = \tau_2$, so heißen $d_1$ und $d_2$ äquivalent.
\end{defn}

% Ausgelassen: Entsprechende Definition für Normen

\begin{satz}
  Seien $\norm{\blank}_1$ und $\norm{\blank}_2$ zwei Normen auf dem $\K$-VR $X$. Dann:
  \begin{itemize}
    \item $\norm{\blank}_2 \text{ ist stärker als } \norm{\blank}_1 \iff \ex{C > 0} \fa{x \in X} \norm{x}_1 \leq C \norm{x}_2$
    \item $\norm{\blank}_1 \text{ und } \norm{\blank}_2 \text{ sind äquivalent} \iff $\\
    $\ex{c, C > 0} \fa{x \in X} c \norm{x}_1 \leq \norm{x}_2 \leq C \norm{x}_1$
  \end{itemize}
\end{satz}

\begin{defn}
  Die \emph{$p$-Norm} auf dem $\K^n$ ist definiert für $p \in \left[ 1, \infty \right]$ als
  \[
    \norm{x}_p \coloneqq \left( \sum_{i = 1}^n \abs{x_j}^p \right)^{\frac{1}{p}} \text{ für } 1 \leq p < \infty, \quad
    \norm{x}_{\infty} \coloneqq \max_{1 \leq i \leq n} \abs{x_i}.
  \]
  % Ausgelassen: Alternative Bezeichung $\norm{\blank}_{\mathrm{max}}$ für $\norm{\blank}_\infty$
\end{defn}

\begin{bem}
  Alle $p$-Normen auf dem $\K^n$ sind zueinander äquivalent.
\end{bem}

% Ausgelassen: Die euklidische Norm und die Fréchet-Metrik $\rho(x) \coloneqq \frac{\norm{x}}{1 + \norm{x}}$ erzeugen im $\K^n$ diesselbe Topologie.

% Thema: Stetigkeit

\begin{defn}
  Seien $(X, \tau_X)$ und $(Y, \tau_Y)$ Hausdorff-Räume, $S \subset X$, sowie $x_0 \in S$. Eine Funktion $f : S \to Y$ heißt \emph{stetig} in $x_0$, falls gilt:
  \[ \fa{V \in \tau_Y} f(x_0) \in V \implies \ex{U \in \tau_X \text{ mit } x_0 \in U} f(U \cap S) \subset V \]
  Ist $X = S$, so heißt $f : X \to Y$ stetige Abbildung, falls $f$ stetig in allen Punkten $x_0 \in X$ ist. Das ist genau dann der Fall, wenn das Urbild offener Mengen offen ist, d.\,h. $\fa{V \in \tau_Y} f^{-1}(V) \in \tau_X$.
\end{defn}

\begin{bem}
  In metrischen Räumen ist diese Definition äquivalent zur üblichen Folgendefinition.
\end{bem}

\begin{defn}
  Sei $(X, d)$ ein metrischer Raum und $(x_k)_{k \in \N}$ eine Folge in $X$. Die Folge heißt \emph{Cauchy-Folge}, falls
  \[ d(x_k, x_l) \convWith{k, l} 0. \]
  Ein Punkt $x \in X$ heißt \emph{Häufungspunkt} der Folge, falls es eine Teilfolge $(x_{k_i})_{i \in \N}$ gibt mit $x_{k_i} - x \convWith{i} 0$.
\end{defn}

\begin{defn}
  Ein metrischer Raum $(X, d)$ heißt \emph{vollständig}, falls jede Cauchy-Folge in $X$ einen Häufungspunkt (den Grenzwert) hat.
\end{defn}

\begin{defn}
  \begin{itemize}
    \item Ein normierter $\K$-Vektorraum heißt \emph{Banachraum}, wenn er vollständig bezüglich der induzierten Metrik ist.
    \item Ein Banachraum $X$ heißt \emph{Banach-Algebra}, falls er eine Algebra ist mit $\norm{x \cdot y}_X \leq \norm{x}_X \cdot \norm{y}_X$.
    \item Ein \emph{Hilbertraum} ist ein Prähilbertraum, der vollständig bzgl. der vom Skalarprodukt induzierten Norm ist.
  \end{itemize}
\end{defn}

% Ausgelassen: Beispiel $\R$ ist ein Banachraum, genauso $\K^n$

\begin{bem}
  Ein normierter Raum $X$ ist genau dann ein Prähilbertraum, falls die Parallelogrammidentität
  \[ \fa{x,y \in X} \norm{x+y}^2 + \norm{x-y}^2 = 2 \left(\norm{x}^2 + \norm{y}^2\right) \]
  gilt. Folglich ist ein Banachraum genau dann ein Hilbertraum, falls die Parallelogrammidentität gilt.
\end{bem}

% Thema: Folgenräume

\begin{defn}
  Sei $\K^\N \coloneqq \{ (x_n)_{n \in \N} \text{ Folge in } \K \}$. Die Fréchet-Metrik
    \[ \rho(x) \coloneqq \sum_{i = 1}^\infty 2^{-i} \tfrac{ \abs{x_i} }{ 1 + \abs{x_i} } < 1 \]
  macht $\K^\N$ zu einem metrischen Raum, dem \emph{Folgenraum}.
\end{defn}

\begin{satz}
  Sei $(x^k)_{k \in \N}$ eine Folge in $\K^\N$ mit $x^k = (x^k_i)_{i \in \N}$ und $x = (x_i)_{i \in \N} \in \K^\N$, so gilt
  \[ \rho(x^k - x) \convWith{k} 0 \iff \fa{i \in \N} x_i^k \convWith{k} x_i. \]
\end{satz}

\begin{satz}
  Der Folgenraum $\K^\N$ ist vollständig.
\end{satz}

\begin{defn}
  Für $p \in \left[ 1, \infty \right]$ und $x = (x_i)_{i \in \N} \in \K^\N$ heißt die Norm
  \begin{align*}
    \norm{x}_{\ell^p} &\coloneqq \left( \sum_{i=1}^\infty \abs{x_i}^p \right)^{\frac{1}{p}} \!\in \left[ 0, \infty \right], \text{ für } 1 \leq p < \infty \\
    \norm{x}_{\ell^\infty} &\coloneqq \sup_{i \in \N} \abs{x_i} \in \left[0, \infty\right]
  \end{align*}
  \emph{$\ell^p$-Norm} auf dem Raum $\ell^p(\K) \coloneqq \Set{ x \in \K^\N }{ \norm{x}_{\ell^p} < \infty }$.
\end{defn}

\begin{satz}
  Der Raum $(\ell^p(\K), \norm{\blank}_{\ell^p})$ ist ein Banachraum.
\end{satz}

\begin{bem}
  Im Fall $p {=} 2$ ist $\ell^2(\K)$ ein Hilbertraum mit Skalar- produkt $\scp{x}{y}_{\ell^2} \coloneqq \sum_{i = 0}^\infty x_i \overline{y_i}$ für $x = (x_i)_{i \in \N}, \, y = (y_i)_{i \in \N} \in \ell^2(\K)$.
\end{bem}

% Thema: Vervollständigung

\begin{satz}[Vervollständigung]
  Sei $(X, d)$ ein metrischer Raum. Betrachte die Menge $X^\N$ aller Folgen in $X$ und definiere
  \[ \widetilde{X} \coloneqq \Set{ x \in X^\N }{ x \text{ ist Cauchy-Folge in } X }/\sim \]
  mit der Äquivalenzrelation $x \sim y \text{ in } \widetilde{X} \!\coloniff\! d(x_j, y_j) \convWith{j} 0$.
  Diese Menge wird mit der Metrik
  \[ \widetilde{d}(x, y) \coloneqq \lim_{i \to \infty} d(x_i, y_i) \]
  zu einem vollständigen metrischen Raum. Die injektive Abbildung $J : X \to \tilde{X}$, welche $x \in X$ auf die konstante Folge $(x)_{i \in \N}$ abbildet, ist isometrisch, d.\,h.
  $\fa{x, y \in X} \widetilde{d}(J(x), J(y)) = d(x, y)$.
  Wir können also $X$ als einen dichten Unterraum von $\widetilde{X}$ auffassen.
\end{satz}

\begin{defn}
  Man nennt $\widetilde{X}$ \emph{Vervollständigung} von~$X$.
\end{defn}

% Kapitel 1.2. Funktionenräume

% TODO: Sei im Folgenden Y ein Banachraum?
















\begin{defn}[\emph{Raum der beschränkten Funktionen}]
  Sei $S$ eine Menge und $Y$ ein Banachraum über $\K$ mit Norm $y \mapsto |y|$. Dann ist
  \[ B(S; Y) \coloneqq \Set{ f : S \to Y }{ f(S) \text { ist eine beschränkte Teilmenge von } Y } \]
  die Menge der beschränkten Funktionen von $B$ nach $Y$. Diese Menge ist ein $\K$-Vektorraum und wird mit der Supremumsnorm $\| f \|_{B(S)} \coloneqq \sup_{x \in S} |f(x)|$ zu einem Banachraum.
\end{defn}

\begin{satz}
  Ist $(X, d)$ ein vollständiger metrischer Raum und $Y \subset X$ abgeschlossen, so ist auch $(Y, d)$ ein vollständiger metrischer Raum.
\end{satz}

\begin{defn}[\emph{Raum stetiger Funktionen auf einem Kompaktum}]
  Sei $S \subset \R^n$ beschränkt und abgeschlossen (d.\,h. kompakt) und $Y$ ein Banachraum über $\K$ mit Norm $y \mapsto |y|$, so ist
  \[ \mathcal{C}^0(S; Y) \coloneqq \mathcal{C}(S; Y) \coloneqq \Set{ f : S \to Y }{ f \text{ ist stetig } } \]
  die Menge der stetigen Funktionen von $S$ nach $Y$. Sie ist ein abgeschlossener Unterraum von $B(S; Y)$ mit der Norm $\| \cdot \|_{\mathcal{C}(S; Y)} = \| \cdot \|_{B(S; Y)}$, also ein Banachraum.
\end{defn}

\begin{bem}
  Für $Y = \K$ ist $\mathcal{C}^0(S; \K) = \mathcal{C}(S)$ eine kommutative Banach-Algebra mit dem Produkt $(f \cdot g)(x) \coloneqq f(x) \cdot g(x)$.
\end{bem}

\begin{defn}
  Eine Teilmenge $A \subset X$ heißt \emph{präkompakt}, falls es für jedes $\epsilon > 0$ eine Überdeckung von $A$ mit endlich vielen $\epsilon$-Kugeln $A \subset B_{\epsilon}(x_1) \cup ... \cup B_{\epsilon}(x_{n_\epsilon})$ mit $x_1, x_{n_\epsilon} \in X$ gibt.
\end{defn}

\begin{defn}
  Eine Teilmenge $A \subset X$ eines metrischen Raumes $(X, d)$ heißt \emph{kompakt}, falls eine der folgenden äquivalenten Bedinungen erfüllt ist:
  \begin{itemize}
    \item $A$ ist \emph{überdeckungskompakt}: Für jede Überdeckung $A \subset \bigcup_{i \in I} A_i$ mit $A_i \opn X$,  gibt es eine endl. Teilmenge $J \subset I$ mit $A \subset \bigcup_{i \in J} A_i$.
    \item $A$ ist \emph{folgenkompakt}: Jede Folge in $A$ besitzt eine konvergente Teilfolge mit Grenzwert in $A$.
    \item $(A, d|_A)$ ist vollständig und $A$ ist \emph{präkompakt}.
  \end{itemize}
\end{defn}

\begin{satz}
  Sei $(X, d)$ ein metrischer Raum. Dann gilt:
  \begin{itemize}
    \item $A$ präkompakt $\implies$ $A$ beschränkt,
    \item $A$ kompakt $\implies$ $A$ abgeschlossen und präkompakt,
    \item Falls $X$ vollständig, dann $A$ präkompakt $\iff$ $\overline{A}$ kompakt.
  \end{itemize}
\end{satz}

\begin{satz}
  Sei $A \subset \K^n$. Dann gilt:
  \begin{itemize}
    \item $A$ präkompakt $\iff$ $A$ beschränkt,
    \item $A$ kompakt $\iff$ $A$ abgeschlossen und beschränkt (Heine-Borel).
  \end{itemize}
\end{satz}

\begin{satz}
  Sei $(X, d)$ ein metrischer Raum und $A \subset X$ kompakt. Dann gibt es zu $x \in X$ ein $a \in A$ mit $d(x, a) = \dist(x, A)$.
\end{satz}

\begin{defn}
  Sei $S \subset \R^n$ und $(K_n)_{n \in \N}$ eine Folge kompakter Teilmengen des $\R^n$. Dann heißt $(K_n)$ eine \emph{Ausschöpfung} von $S$, falls
  \begin{itemize}
    \item $S = \cup_{n \in \N} K_n$,
    \item $\emptyset \not= K_i \subset K_{i+1} \subset S$ für alle $i \in \N$ und
    \item für alle $x \in S$ gibt es ein $\delta > 0$ und $i \in \N$, sodass $B_\delta(x) \subset K_i$.
  \end{itemize}
\end{defn}

\begin{bem}
  Zu $S \opn \R^n$ und $S \cls \R^n$ existiert eine Ausschöpfung.
\end{bem}

\begin{defn}[\emph{Raum stetiger Funktionen auf Menge mit Ausschöpfung}]
  Es sei $S \subset \R^n$ so, dass eine Ausschöpfung $(K_i)_{i \in \N}$ von $S$ existiert und $Y$ ein Banachraum. Dann bildet die Menge aller stetigen Funktionen
  \[ C^0(S; Y) \coloneqq \Set{ f : S \to Y }{ f \text{ ist stetig auf } S } \]
  einen $K$-Vektorraum und wird mit der Fréchet-Norm
  \[ \varrho(f) \coloneqq \sum_{i \in \N} 2^{-i} \frac{ \| f \|_{C^0(K_i)} }{ 1 + \| f \|_{C^0(K_i)} } \]
  zu einem vollständigen metrischen Raum.
\end{defn}

\begin{bem}
  \begin{itemize}
    \item Die von dieser Metrik erzeugte Topologie ist unabhängig von der Wahl der Ausschöpfung.
    \item Ist $S \subset \R^n$ kompakt, so stimmt die Topologie mit der von $\| \cdot \|_{B(s)}$ überein.
  \end{itemize}
\end{bem}

\begin{defn}
  Sei $S \subset \R^n$ und $Y$ ein Banachraum. Für $f : S \to Y$ heißt
  \[ \supp f \coloneqq \Set{ x \in S }{ f(x) \not= 0 } \]
  \emph{Träger} (engl. support) von $f$.
\end{defn}

\begin{defn}
  Sei $S \subset \R^n$ und $Y$ ein Banachraum. Dann ist
  \[ \mathcal{C}_0^0(S; Y) \coloneqq \Set{ f \in \mathcal{C}^0(S; Y) }{ \supp f \text{ ist kompakt in } S } \]
  die Menge der stetigen Fktn. mit kompaktem Träger von $S$ nach $Y$.
\end{defn}

\begin{defn}[\emph{Raum differenzierbarer Funktionen}]
  Sei $\Omega \subset \R^n$ offen und beschränkt und $m \in \N$. Dann ist die Menge der differenzierbaren Funktionen von $\Omega$ nach $Y$
  \begin{align*}
    \mathcal{C}^m(\overline\Omega, Y) \coloneqq \{ f : \Omega \to Y \,|\, & f \text{ ist $m$-mal stetig differenzierbar in $\Omega$ } \\
    & \text{ und für $k \leq m$ und } s_1, ..., s_k \in \{ 1, ..., n \} \\
    & \text{ ist $\partial_{s_1} ... \partial_{s_k} f$ auf $\overline\Omega$ stetig fortsetzbar } \}
  \end{align*}
  ein Vektorraum und mit folgender Norm ein Banachraum:
  \[ \| f \|_{\mathcal{C}^m(\overline\Omega)} = \sum_{|s| \leq m} \| \partial^s \|_{\mathcal{C}^0(\overline\Omega)} \]
\end{defn}

\begin{bem}
  In obiger Norm wird die Summe über alle $k$-fache partielle Ableitungen mit $k \leq m$ gebildet.
\end{bem}

% TODO: Hölder-Räume
% TODO: Unendlich differenzierbare Funktionen
% TODO: Lebesgue-Räume
% TODO: Sobolev-Räume
% TODO: Faltungen und Dirac-Folge
% TODO: Neumann-Reihe

\begin{satz}
  Sei $X$ ein normierter Raum und $Y \subset X$ ein abgeschlossener echter Teilraum. Für $0 < \Theta < 1$ (falls $X$ Hilbertraum, geht auch $\Theta = 1$) gibt es ein $x_{\Theta} \in X$ mit
  \[ \|x_0\| = 1 \quad \text{und} \Theta \leq \dist(x_{\Theta}, Y) \leq 1. \]
\end{satz}

\begin{satz}
  Für jeden normierten Raum $X$ gilt:
  \[ \overline{B_1(0)} \text{ kompakt } \iff \dim(X) < \infty. \]
\end{satz}

\begin{defn}
  Sei $S \subset \R^n$ kompakt, $Y$ ein Banachraum und $A \subset \mathcal{C}^0(S, Y)$. Dann heißt $A$ \emph{gleichgradig stetig}, falls
  \[ \sup_{f \in A} \left| f(x) - f(y) \right| \xrightarrow{\left| x - y \right| \to 0} 0. \]
\end{defn}

\begin{defn}[Arzelà-Ascoli]
  Sei $S \subset \R^n$ kompakt, $Y$ ein endlichdimensionaler Banachraum und $A \subset \mathcal{C}^0(S, Y)$. Dann gilt
  \[ A \text{ präkompakt } \iff A \text{ ist beschränkt und gleichgradig stetig. } \]
\end{defn}

\begin{satz}[Fundamentallemma der Variationsrechnung]
  Sei $\Omega \subset \R^n$ und $Y$ ein Banachraum. Für $g \in \Leb^1(\Omega, Y)$ sind dann äquivalent:
  \begin{itemize}
    \item Für alle $\xi \in \mathcal{C}_0^\infty$ gilt $\Int{\Omega}{}{(\xi \cdot g)}{x} = 0$.
    \item Für alle beschränkten $E \in \Bor(\Omega)$ mit $\overline{E} \subset \Omega$ gilt $\Int{E}{}{g}{x} = 0$.
    \item Es gilt $g \fue= 0$ in $\Omega$.
  \end{itemize}
\end{satz}

\begin{satz}
  Sei $T : X \to Y$ eine lineare Abbildung zwischen Vektorräumen $X$ und $Y$. Dann sind äquivalent:
  \begin{itemize}
    \begin{multicols}{3}
      \item $T$ ist stetig.
      \item $T$ ist stetig in $0$.
      \item $\sup_{\|x\| \leq 1} \|Tx\| < \infty$.
    \end{multicols}
    \item $\ex{C > 0} \fa{x \in X} \|Tx\| \leq C \cdot \|x\|$.
  \end{itemize}
\end{satz}

\begin{defn}
  Seien $X, Y$ Vektorräume mit einer Topologie. Dann ist
  \[ \mathcal{L}(X, Y) = \Set{ T : X \to Y }{ X \text{ ist linear und stetig } } \]
  die Menge aller \emph{linearen Operatoren} zwischen $X$ und $Y$. Falls die Stetigkeit nicht nur topologisch, sondern bezüglich einer Norm gilt, so redet man von \emph{beschränkten Operatoren}.
\end{defn}

\begin{satz}
  Seien $X \not= \{0\}$, $Y \not= \{0\}$ Banachräume und $T, S \in \mathcal{L}(X, Y)$. Dann gilt: Falls $T$ invertierbar ist und $\|S-T\| < \tfrac{1}{\|T^{-1}\|}$, dann ist auch $S$ invertierbar.
\end{satz}

\begin{bem}
  Die Menge aller invertierbaren Operatoren in $\mathcal{L}(X, Y)$ ist somit eine offene Teilmenge.
\end{bem}

\begin{defn}
  Seien $X$ und $Y$ Banachräume über $\K$. Eine lineare Abbildung $T : X \to Y$ heißt \emph{kompakter (linearer) Operator}, falls eine der folgenden äquivalenten Bedingungen erfüllt ist:
  \begin{itemize}
    \begin{multicols}{2}
      \item $\overline{T(B_1(0))}$ ist kompakt.
      \item $T(B_1(0))$ ist präkompakt.
    \end{multicols}
    \item Für alle beschränkten $M \subset X$ ist $T(M) \subset Y$ präkompakt.
    \item Für jede beschränkte Folge $(x_n)_{n \in \N}$ in $X$ besitzt $(T x_n)_{n \in \N}$ eine in $Y$ konvergente Teilfolge.
  \end{itemize}
\end{defn}

\begin{defn}
  Sei $X$ ein Vektorraum über $\K$. Dann ist $X' \coloneqq \mathcal{L}(X, \K)$ der \emph{Dualraum} von $X$. Elemente von $X'$ werden \emph{lineare Funktionale} genannt.
  % Norm?
\end{defn}

\begin{satz}[Rieszscher Darstellungssatz]
  Ist $X$ ein Hilbertraum, so ist
  \begin{align*}
    J : X \to X', \quad x \mapsto y \mapsto (y, x)_X
  \end{align*}
  ein isometrischer konjugiert linearer Isomorphismus.
\end{satz}

\begin{satz}[Lax-Milgram]
  Sei $X$ ein Hilbertraum über $\K$ und $a : X \times X \to \K$ sesquilinear. Es gebe Konstanten $c_0$ und $C_0$ mit $0 < c_0 \leq C_0 < \infty$, sodass für alle $x, y \in X$ gilt:
  \begin{itemize}
    \item $\left| a(x, y) \right| \leq C_0 \cdot \|x\| \cdot \|y\|$ {\raggedright (Stetigkeit)}
    \item $Re a(x, x) \geq c_0 \cdot \|x\|^2$ {\raggedright (Koerzivität)}
  \end{itemize}
  Dann existiert genau eine Abbildung $A : X \to X$ mit
  \[ a(y, x) = (y, Ax) \text{ für alle $x, y \in X$. } \]
  Außerdem gilt: $A \in \mathcal{L}(X)$ ist ein invertierbarer Operator mit
  \[ \|A\| \leq C_0 \quad \text{und} \quad \|A^{-1}\| \leq \tfrac{1}{c_0}. \]
\end{satz}

\begin{satz}[Hahn-Banach]
  Sei $X$ ein $\R$-VR und
  \begin{itemize}
    \item $p : X \to \R$ sublinear, d.\,h. für alle $x, y \in X$ und $\alpha \in \R_{\geq 0}$ gelte
    \[ p(x+y) \leq p(x) + p(y) \quad \text{und} \quad p(\alpha x) = \alpha p(x), \]
    \item $f : Y \to \R$ linear auf einem Unterraum $Y \subset X$ und
    \item $f(x) \leq p(x)$ für $x \in Y$.
  \end{itemize}
  Dann gibt es eine lineare Abbildung $F : X \to \R$ mit
  \[ F(x) = f(x) \text{ für $x \in Y$} \quad \text{und} \quad F(x) \leq p(x) \text{ für } x \in X. \]
\end{satz}

\begin{satz}(Hahn-Banach für lineare Funktionale)
  Sei $X$ ein $\R$-VR, $Y \subset X$ ein Unterraum, $p : X \to \R$ linear und $f : Y \to \R$ linear, sodass $f(x) \leq p(x)$ für alle $x \in Y$. Dann existiert eine lineare Abbildung $F : X \to \R$ mit $f = F|_Y \text{ und } F \leq p$.
\end{satz}

% Korollar
\begin{satz}
  Sei $(X, \|\cdot\|_X)$ ein normierter $\K$-Vektorraum und $(Y, \|\cdot\|_Y)$ ein Unterraum. Dann gibt es zu $y \in Y'$ ein $x' \in X'$ mit $x'|_Y = y'$ und $\|x'\|_{X'} = \|y'\|_{Y'}$.
\end{satz}

\begin{satz}
  Sei $Y$ abgeschlossener Unterraum des normierten Raumes $X$ und $x_0 \in X \setminus Y$. Dann gibt es ein $x' \in X'$ mit $x'|_Y = 0$, $\|x'\|_{X'} = 1$, $\langle x', x_0 \rangle = \dist(x_0, Y)$.
\end{satz}

\begin{bem}
  Dann gibt es auch ein $x' \in X'$ mit $x'|_Y = 0$,
  \[ \|x'\|_{X'} = (\dist(x_0, Y))^{-1} \quad \text{und} \quad \langle x', x_0 \rangle = 1. \]
\end{bem}

% Bemerkung: Der Satz kann als Verallgemeinerung des Projektionssatzes für Hilberträume im linearen Fall aufgefasst werden: Ist $X$ Hilbertraum, so definiere $\langle x', x \rangle_{X' \times X} = (x | \frac{x_0 - Px_0}{\|x_0 - Px_0\|_X})_X$, wobei $P$ orthogonale Projektion auf $Y$ sei. Dann ist $x' = 0$ auf $Y$ und daher $\langle x', x_0 \rangle_{X' \times X} = \langle x', x_0 - Px_0 \rangle = \| x_0 - Px_0 \|_X$. Außerdem ist $\langle x', x \rangle \leq \|x\|_X$. Daher hat $x'$ die Eigenschaft wie im Satz.

% Korollar
\begin{satz}
  Seien $X$ normierter Raum und $x_0 \in X$. Dann gilt
  \begin{itemize}
    \item  Ist $x_0 \not= 0$, so gibt es $x_0' \in X'$ mit $\|x_0'\|_{X'} = 1$ und $\langle x_0', x_0 \rangle_{X' \times X} = \|x_0\|_X$.
    \item Ist $\langle x', x_0 \rangle_{X' \times X} = 0$ für alle $x' \in X'$, so ist $x_0 = 0$.
    \item Durch $Tx' = \langle x', x_0 \rangle_{X' \times X}$ für $x' \in X'$ ist ein $T \in \mathcal{L}(X', \K) = X''$, dem Bidualraum, definiert mit $\|T\| = \|x_0\|_X$.
  \end{itemize}
\end{satz}


% 5. Prinzip der gleichmäßigen Beschränktheit

\begin{satz}[Baire'scher Kategoriensatz]
  Es sei $X \not= \emptyset$ ein vollständiger metrischer Raum und $X = \bigcup_{k \in \N} A_k$ mit abgeschlossenen Mengen $A_k \subset X$. Dann gibt es ein $k_0 \in \N$ mit $\mathrm{int}(A_{k_0}) \not= \emptyset$.
\end{satz}

\begin{kor}
  Jede Basis eines $\infty$-dimensionalen Banachraumes ist überabzählbar.
\end{kor}

\begin{satz}[Prinzip der gleichmäßigen Beschränktheit]
  Es sei $X$ ein nichtleerer vollständiger metrischer Raum und $Y$ ein normierter Raum. Gegeben sei eine Menge von Funktionen $F \subset \mathcal{C}^0(X, Y)$ mit $\fa{x \in X} \sup_{f \in F} \|f(x)\|_Y < \infty$. Dann gibt es ein $x_0 \in X$ und ein $\epsilon > 0$, sodass $\sup_{B_\epsilon(x_0)} \sup_{f \in F} \|f(x)\|_Y < \infty$.
\end{satz}

\begin{satz}[Banach-Steinhaus]
  Es sei $X$ ein Banachraum und $Y$ ein normierter Raum, $\mathcal{T} \subset \Leb(X, Y)$ mit $\fa{x \in X} \sup_{T \in \mathcal{T}} \|Tx\|_Y < \infty$. DAnn ist $\mathcal{T}$ eine beschränkte Menge in $\Leb(X, Y)$, d.\,h. $\sup_{T \in \mathcal{T}} \|T\|_{\mathcal{L}(X, Y)}$.
\end{satz}

\begin{defn}
  Seien $X$ und $Y$ topologische Räume, so heißt eine Abbildung $f : X \to Y$ \emph{offen}, falls für alle offenen $U \opn X$ das Bild $f(U) \opn Y$ offen ist.
\end{defn}

\begin{bem}
  Ist $f$ bijektiv, so ist $f$ genau dann offen, wenn $f^{-1}$ stetig ist. Sind $X, Y$ normierte Räume und ist $T : X \to Y$ linear, so gilt: $T$ ist offen $\iff$ $\ex{\delta > 0} B_{\delta}(0) \subset T(B_1(0))$.
\end{bem}

\begin{satz}[von der offenen Abbildung]
  Seien $X, Y$ Banachräume und $T \in \mathcal{L}(X, Y)$. Dann ist $T$ genau dann surjektiv, wenn $T$ offen ist.
\end{satz}

\begin{satz}[von der inversen Abbildung]
  Seien $X, Y$ Banachräume und $T \in \mathcal{L}(X, Y)$ bijektiv, so ist $T^{-1}$ stetig, also $T^{-1} \in \mathcal{L}(Y, X)$.
\end{satz}

\begin{satz}[vom abgeschlossenen Graphen]
  Seien $X, Y$ Banachräume und $T : X \to Y$ linear. Dann ist $\mathrm{Graph}(T) = \Set{ (x, Tx) }{ x \in X }$ genau dann abgeschlossen, wenn $T$ stetig ist. Dabei ist $\mathrm{Graph}(T) \subset X \times Y$ mit der \emph{Graphennorm} $\|(x,y)\|_{X \times Y} = \|x\|_X + \|y\|_Y$.
\end{satz}

\begin{defn}
  Sei $X$ ein Banachraum.
  \begin{itemize}
    \item Eine Folge $(x_k)_{k \in \N}$ in $X$ \emph{konvergiert schwach} gegen $x \in X$ (notiert $x_k \convWeaklyWith{k} x$), falls für alle $x' \in X'$ gilt:
    \[ \langle x', x_k \rangle_{X' \times X} \convWith{k} \langle x', x \rangle_{X' \times X} \]
    \item Eine Folge $(x'_k)_{k \in \N}$ in $X'$ \emph{konvergiert schwach*} gegen $x' \in X'$ (notiert $x'_k \convWeaklyStarWith{k} x'$), falls für alle $x \in X$ gilt:
    \[ \langle x'_k, x \rangle_{X' \times X} \convWith{k} \langle x', x \rangle_{X' \times X} \]
    \item Analog sind \emph{schwache} und \emph{schwache* Cauchyfolgen} definiert.
    \item Eine Menge $M \subset X$ (bzw. $M \subset X'$) heißt \emph{schwach folgenkompakt} bzw. \emph{schwach* folgenkompakt}, falls jede Folge in der Menge $M$ eine schwach (bzw. schwach*) konvergente Teilfolge besitzt deren Grenzwert wieder in $M$ liegt.
  \end{itemize}
\end{defn}

\begin{bem}
  Der schwache bzw. schwache* Grenzwert einer Folge ist eindeutig bestimmt. Starke Konvergenz impliziert schwache Konvergenz.
\end{bem}

\begin{satz}
  Es gilt für $x, x_k \in X$, $x', x'_k \in X'$:
  \begin{align*}
    x_k \convWeaklyWith{k} x \,\, \text{ in } X \, \quad &\iff \quad J_x x_k \convWeaklyStarWith{k} J_x x \text{ in } X'' \\
    x'_k \convWeaklyWith{k} x' \text{ in } X' \quad &\implies \qquad \,\, x'_k \convWeaklyStarWith{k} x' \text{ in } X'
  \end{align*}
\end{satz}

\begin{lem}
  \begin{itemize}
    \item Aus $x'_k \convWeaklyStarWith{k} x'$ in $X'$ folgt $\|x'\|_{X'} \leq \liminf_{k \to \infty} \|x'_k\|_{X'}$, aus $x_k \convWeaklyWith{k} x$ in $X$ folgt $\|x\|_X \leq \liminf_{k \to \infty} \|x_k\|_X$.
    \item Schwach bzw. schwach* konvergente Folgen sind beschränkt.
    \item Aus $x_k \convWith{k} x$ in $X$ und $x'_k \convWeaklyStarWith{k} x'$ in $X'$ folgt $\langle x'_k, x_k \rangle_{X' \times X} \convWith{k} \langle x', x \rangle_{X' \times X}$. Dasselbe folgt mit $x_k \convWeaklyWith{k} x$ in $X$ und $x'_k \convWith{k} x'$ in $X'$.
  \end{itemize}
\end{lem}

\begin{acht}
  In der letzten Behauptung müssen wir vorraussetzen, dass mindestens eine Folge stark konvergiert. Für beidesmal schwache/schwache* Konvergenz ist die Aussage i.\,A. falsch.
\end{acht}

\iffalse
\begin{bsp}
  \begin{itemize}
    \item Sei $1 \leq p < \infty$ und $\tfrac{1}{p} + \tfrac{1}{p'} = 1$ und $\omega \subset \R^n$ beschränkt. Dann gilt für $f, f_k \in L^p(\Omega)$:
    \[ f_k \convWeaklyWith{k} f \text{ in } L^p(\Omega) \iff \fa{g \in L^p(\Omega)} \IntO{f_k \cdot \overline{g}}{x} \convWith{k} \IntO{f \cdot \overline{g}}{x} \]
    \item Sei $\Omega \subset \R^n$ boffen und beschränkt und $1 \leq p \leq \infty$, $m \in \N_{> 0}$. Für $u, u_k \in W^{m,p}(\Omega)$ gilt dann:
    \[ u_k \convWeaklyWith{k} u \text{ in } W^{m,p}(\Omega) \iff \fa{s, \abs{s} \leq m} \partial^s u_k \convWeaklyWith{k} \partial^s u \text{ in } L^p(\Omega) \]
  \end{itemize}
\end{bsp}
\fi

\begin{satz}[Banach-Alaoglu]
  Sei $X$ ein separabler Banachraum. Dann ist die abgeschl. Einheitskugel $\overline{B_1(0)} \subset X'$ schwach* folgenkompakt.
\end{satz}

\begin{bsp}
  Sei $\Omega \subset \R^n$ beschränkt und offen. Dann ist $L^1(\Omega)$ separabel (Approximation durch Treppenfunktionen und der Satz besagt: Ist $(f_k)_{k \in \N}$ in $L^{\infty}(\Omega)$ beschränkt, so gibt es eine Teilfolge $(f_{k_l})_{l \in \N}$ und ein $f \in L^\infty(\Omega)$, sodass
  \[ \IntO{f_{k_l}{x} \cdot \overline{g}} \xrightarrow{l \to \infty} \IntO{f \cdot \overline{g}}{x} \quad \text{für alle $g \in L^1(\Omega)$} \]
\end{bsp}

\begin{bem}
  Schwach*-Konvergenz impliziert eine sogenannte Schwach*-Topologie in dem Sinne, dass man sagt, eine Folge $(x_k')_{k \in \N}$ in $X'$ ist bzgl. dieser Topologie konvergent, wenn sie punktweise für alle $x \in X$ konvergiert.
\end{bem}

\begin{defn}
  Sei $X$ ein Banachraum und $J_X$ die Isometrie bzgl. des Bidualraumes. Dann heißt $X$ \emph{reflexiv}, falls $J_X$ surjektiv ist.
\end{defn}

\begin{lem}
  \begin{itemize}
    \item Ist $X$ reflexiv, so stimmen schwache* und schwache konvergenz in $X'$ überein.
    \item Ist $X$ reflexiv, so ist jeder abgeschlossene Unterraum von $X$ reflexiv.
    \item Ist $T : X \to Y$ ein Isomorphismus, so gilt:
      \[ X \text{ reflexiv } \iff Y \text{ reflexiv } \]
    \item Es gilt: $X$ reflexiv $\iff$ $X'$ reflexiv.
  \end{itemize}
\end{lem}

\begin{lem}
  Für jeden Banachraum $X$ gilt: $X'$ separabel $\implies$ $X$ separabel.
\end{lem}

\begin{bem}
  Die Umkehrung gilt i.\,A. nicht! Gegenbeispiel: $X = L^1$.
\end{bem}

\begin{satz}[Eberlein-Shmulyan]
  Sei $X$ reflexiver Banachraum. Dann ist die abgeschlossene Einheitskugel $\overline{B_1(0)} \subset X$ schwach folgenkompakt.
\end{satz}

\begin{bsp}
  \begin{itemize}
    \item Hilberträume $X$ sind reflexiv (folgt direkt aus dem Riesz'schen Darstellungssatz; im Reellen $J_X = (R_X R_{X'})^{-1}$, wobei $R_X : X \to X'$ der zugehörige isomorphismus). Daher: Ist $(x_k)_{k \in \N}$ eine beschränkte Folge in $X$, so existiert eine Teilfolge $(x_{k_l})_{l \in \N}$ und $x \in X$, sodass
    \[ (y | x_{k_l})_X \xrightarrow{l \to \infty} (y | x)_X \]
    für alle $y \in X$.
    \item Sei $\Omega \subset \R^n$ beschränkt, $1 < p < \infty$, $\tfrac{1}{p} + \tfrac{1}{p'} = 1$. Dann ist $L^p(\Omega)$ reflexiv.
    \item $L^1$ und $L^\infty$ sind genau dann nicht reflexiv, wenn sie unendlich-dimensional sind.
  \end{itemize}
\end{bsp}

\begin{bem}
  Analog zur schwach*-Topologie kann man auch eine schwache Topologie einführen.
\end{bem}

% Minkowski-Funktional

% In Abschnitt 2.1 haben wir gesehen, dass das Abstandsproblem zu konvexen, abgeschlossenen Mengen in allgemeinen Banachräumen nicht lösbar ist, in reflexiven aber doch, wie wir seher werden.

% Unterkapitel Minkowski-Funktional

\begin{satz}[Trennungssatz]
  Seien $X$ ein normierter Raum, $M \subset X$ nicht leer, abgeschlossen, konvex und $x_0 \in X \setminus M$. Dann gibt es ein $x' \in X'$ und ein $\alpha \in \R$ mit
  \[ Re \langle x', x_0 \rangle_{X' \times X} > \alpha \quad \text{und} \quad Re \langle x', x \rangle_{X' \times X} \leq \alpha \enspace \text{für } x \in M. \]
\end{satz}

% Geometrische Interpretation

% Häh?
% \begin{bem}
%  Es folgt $x' \not= 0$, also ist $\Set{ x \in X }{ Re \langle x', x \rangle_{X' \times X} = \alpha }$
% \end{bem}

\begin{satz}
  Sei $X$ ein normierter Raum, $M \subset X$ konvex und abgeschlossen. Dann ist $M$ schwach folgenabgeschlossen, d.\,h. sind $x_k, x \in X$ für $k \in \N$, so gilt
  \[ \fa{k \in \N} x_k \in M, x_k \convWeaklyWith{k} x \text{ in } X \implies x \in M \]
\end{satz}

\begin{lem}[Mazur]
  Sei $X$ normierter Raum und $(x_k)_{k \in \N}$ Folge in $X$ mit $x_k \convWeaklyWith{k} x$. Dann gilt $x \in \mathrm{conv} \Set{ x_k }{ k \in \N }$
\end{lem}

\begin{satz}
  Sei $X$ ein reflexiver Banachraum und $M \subset X$ nicht leer, konvex, abgeschlossen. Dann gibt es zu $\tilde{x}$ ein $x \in M$ mit $\| x - \tilde{x} \| = \mathrm{dist}(\tilde{x}, M)$.
\end{satz}

% Der Satz über die schwache Folgenabgeschlossenheit konvexer abgeschlossener Mengen spielt eine wichtige Rolle für Existenzaussagen für partielle Differentialgleichungen mit Nebenbedingungen

% Sei dazu $\Omega \subset \R^n$ offen, beschränkt, zusammenhängend mit $\mathcal{C}^{0,1}$-Rand sowie $\K = \R$. Sei $a_{ij} = a_{ji} \in L^\infty(\Omega)$, $i,j = 1, ..., n$ gleichmäßig elliptisch, $f \in L^2(\Omega)$ sowie $E(u) = \Int{}{}{ \tfrac{1}{2} \sum_{i,j=1}^n \partial_i u a_{ij} \partial_j u + f u }{x}$, $E : W^{1,2}(\Omega) \to \R$.

% Für $M \subset W^{1,2}(\Omega)$ nicht leer, konvex, abgeschlossen mit $\fa{u_0 \in M} \ex{C_0 < \infty} \fa{\xi \in \R} u_0 + \xi \in M \implies \abs{\xi} \leq C$ gilt dann (ohne Beweis)
% \begin{itemize}
%   \item $E$ besitzt auf $M$ ein absolutes Minimum
%   \item Die absoluten Minima von $E$ auf $M$ sind genau die Lösungen der Variationsungleichung: Finde $u \in M$, sodass $\Int{\Omega}{}{\sum_{i,j=1}^n \partial_i (u-v) a_{ij} \partial_j u + (u-v) f}{x}$ für alle $v \in M$.
%   \item Ist $M$ ein abgeschlossener affiner Unterraum $M = u_0 + M_0$ mit $M_0$ Unterraum, so ist die Variationsungleichung äquivalent zu: Finde $u \in M$, sodass $\Int{\Omega}{}{\sum_{i,j=1}^n \partial_i v a_{ij} u + v f}{x} = 0$ für alle $v \in M_0$
%   \item Hat $M$ die Eigenschaft $v \in M$, $\xi \in \R$ mit $v + \xi \in M$ $\implies \xi = 0$, so ist die Lösung eindeutig.
% \end{itemize}

\begin{bsp}
  \begin{itemize}
    \item Sei $M = W_0^{1,2}(\Omega)$. Dann ist die eindeutige Lösbarkeit des zugehörigen (schwachen) Dirichlet-Problems gesichert.
    \item Sei $M = \Set{ u \in W^{1,2}(\Omega) }{ \Int{\Omega}{}{u}{x} = 0 }$ und gelte $\Int{\Omega}{}{f}{x} = 0$. Dann sichern Punkt 3, 4 die eindeutige Lösbarkeit des zugehörigen Neumann-Problems.
    \item Seien $u_0, \psi_0 \in W^{1,2}(\Omega)$ gegeben und $u_0(x) \geq \phi_0(x)$ für fast alle $x \in \Omega$. Definiere $M = \Set{ v \in W^{1,2}(\Omega) }{ v = u_0 \text{ auf } \partial \Omega, v \geq \psi \text{ in } \Omega }$. Dann sichern die Punkte 1 bzw. 2 und 4 die eindeutige Existenz einer Lösung dieses Hindernis-Problems.
  \end{itemize}
\end{bsp}


% Kapitel 7. Endlich-dimensionale Approximation

% Für numerische Berechnungen ist die Approximation von Elementen oder Unterräumen unendlich-dimensionaler Räume mit endlich-dimensionaler Information von großer Relevanz. Eine abzählbare Approximation durch endlich-dimensionale Unterräume ist nur für separable Räume möglich:

\begin{lem}
  Ist $X$ $\infty$-dimensionaler Raum, so sind äquivalent:
  \begin{itemize}
    \item $X$ ist separabel
    \item $\exists X_n \subset X$ endlich-dim. Unterräume : $\fa{n \in \N} X_n \subset X_{n{+}1}$ und $\bigcup_{n \in \N} X_n$ ist dicht in $X$.
    \item $\exists X_n \subset X$ endlich-dim. Unterräume : $E_n \cap E_m = \{ 0 \}$ für $n \not= m$ und $\bigcup_{n \in \N} (E_0 \oplus ... \oplus E_n)$ ist dicht in $X$.
    \item $\exists$ linear unabhängige Menge $\Set{ e_n }{ n \in \N }$ mit $\mathrm{span}\Set{e_n}{ n \in \N }$ ist dicht in $X$.
  \end{itemize}
\end{lem}

% Aussage (1) bedeutet, dass es zu $x \in X$ Punkte $x_n \in X$ gibt mit $x_n \xrightarrow{n \to \infty} x$
% Aussage (4) bedeutet, dass es zu $x \in X$ und $n \in \N$ fr $k = 0, ...., n$ Zahlen $\alpha_{n,k}$ gibt mit $\| x - \sum_{k=0}^n \alpha_{n,k} \|_X \xrightarrow{n \to \infty} 0$
% Falls die $\alpha_{n,k}$ unabhängig von $u$ gewählt werden können:

\begin{defn}
  Sei $X$ normierter Raum. Eine Folge $(x_n)_{n \in \N}$ heißt \emph{Schauder-Basis} von $X$, falls:
  \[ \fa{x \in X} \ex{\text{eindeutige bestimmte } \alpha_k \in \K} \sum_{k=0}^n \alpha_n e_k \xrightarrow{n \to \infty} x \text{ in } X. \]
\end{defn}

% Sind $X^1$, $X^2$ Banachräume mit Schauderbasen $(e_k^1)_{k \in \N}$ und $(e_l^2)_{l \in \N}$ und $S \in \mathcal{L}(X^1, X^2)$, so haben $Se_k^1$ Darstellungen bzgl. $(e^2_l)_{l \in \N}$, d.\,h. es ex. eindeutig bestimmte $a_{k,l} \in \K$ mit $Se_k^1 = \sum_{l=1}^\infty a_{k,l} e_l^2$, also gilt für $x = \sum_{k=0}^\infty \alpha_k e_k^1$:
% \[ Sx = \sum_{k=0}^\infty \sum_{i=0}^\infty \alpha_k a_{k,l} e_l^2 \]

$S$ ist also eindeutig bestimmt durch die "`unendliche Matrix"' $(a_{k,l})_{k,l \in \N}$.

% Kapitel 7.1. Orthogonalsysteme

\begin{defn}
  Sei $X$ ein Prähilbertraum. Eine Folge $(e_k)_{k \in \N}$, $N \subset \N$ in $X$ heißt \emph{Orthogonalsystem}, falls $(e_k | e_l) = 0$ für $k \not= l$ und $e_k \not= 0$ für alle $k \in \N$ und \emph{Orthonormalsystem}, falls zusätzlich $\|e_k\| = 1$ für alle $k \in \N$ gilt.
\end{defn}

\begin{lem}[Besselsche Ungleichung]
  Sei $(e_k)_{k \in \N}$ ein (endliches) Orthonormalsystem des Prähilbertraumes $X$. Dann gilt für alle $x \in X$: $0 \leq \|x\|^2 - \sum_{k=0}^n \abs{(x|e_k)}^2 = \| x - \sum_{k=0}\infty (x | e_k) e_k \|^2 = \dist(x, \mathrm{span} \{ e_0, ..., e_n \})^2$.
\end{lem}

\begin{satz}
  Sei $(e_k)_{k \in \N}$ ein Orthonormalsystem des Prä-Hilbertraumes $X$. Dann sind äquivalent:
  \begin{itemize}
    \item $\mathrm{span} \Set{ e_k }{ k \in \N }$ liegt dicht in $X$
    \item $(e_k)_{k \in \N}$ ist eine Schauder-Basis von $X$.
    \item Für alle $x \in X$ $x = \sum_{k=0}^\infty (x | e_k) e_k$ (Darstellung)
    \item Für alle $x, y \in X$ gilt $(x|y) = \sum_{k=0}^\infty (x|e_k) \overline{(y|e_k)}$ (Parseval-Identität)
    \item Für alle $x \in X$ gilt $\| x \|^2 = \sum_{k=0}^\infty \abs{(x|e_k)}^2$
  \end{itemize}
\end{satz}

\begin{defn}
  Ist eine dieser Bedingungen erfüllt, nennen wir die $(e_k)_{k \in \N}$ \emph{Orthonormalbasis}.
\end{defn}

\begin{satz}
  Jeder $\infty$-dim. Hilbertraum über $\K$ ist genau dann $X$ separabel, wenn $X$ eine Orthonormalbasis besitzt.
\end{satz}

\begin{bem}
  In diesem Fall ist $X$ isometrisch isomorph zu $\mathcal{l}^2(\K)$ (Übergang zu Koeffizienten bzgl. Basis)
\end{bem}


% Vorlesung vom 21.1.2014

\begin{bsp}
  Betrachte $L^2(\left] -\pi, \pi \right[, \K)$. Dann ist durch $e_k(x) = \frac{1}{\sqrt{2\pi}} e^{ikx}$ für $k \in \Z$ eine Orthonormalbasis von $L^2(\left] -\pi, \pi \right[, \C)$ gegeben. Weiter ist durch $\widetilde{e}_0(x) = \frac{1}{\sqrt{2\pi}}$, $\widetilde{e}_{k}(x) = \frac{1}{\sqrt{2\pi}} \sin(kx)$ für $k > 0$ und $\widetilde{e}_k(x) = \frac{1}{\sqrt{2 \pi}} \cos(kx)$ für $k < 0$ eine ONB von $L^2(\left] -\pi, \pi \right[, \R)$ gegeben.
\end{bsp}

% Dazu nachzuweisen: Orthogonalsystem (Übungsaufgabe), Dichtheit (dazu folgender Beweis)

\begin{lem}
  Zu $f \in L^2(\left] -\pi, \pi \right[, \C)$ sei $P_n f = \sum_{\abs{k} \leq n} (f | e_k)_{L^2} e_k$ mit $e_k$ wie im Beispiel die \emph{Fourier-Summe} von $f$. Ist $f$ Lipschitz-stetig, gilt $f(x) = \lim_{n \to \infty} P_n f(x)$.
\end{lem}

% 7.2. Projektion

Die Fourier-Summe erlaubt die explizite Approximation von $f$ im Unterraum $X = \mathrm{span} \Set{e_k}{ \abs{k} \leq n }$. Allgemein führt man ein:

\begin{defn}
  Sei $Y$ Unterraum des Vektorraums $X$. Eine lineare Abbildung $P : X \to X$ heißt \emph{(lineare) Projektion auf $Y$}, falls $P^2 = P$ und $\mathrm{Bild}(P) = Y$.
\end{defn}

\begin{lem}
  \begin{itemize}
    \item $P$ ist Projektion auf $Y$ $\iff$ $P : X \to Y$ und $P = \Id$ auf $Y$.
    \item $P : X \to X$ ist Projektion $\implies$ $X = \mathrm{ker}(P) \oplus \mathrm{im}(P)$
    \item $P : X \to X$ ist Projektion $\implies$ $\Id - P$ ist Projektion und $\ker(\Id - P) = \mathrm{im}(P)$, $\mathrm{im}(\Id - P) = \mathrm{ker}(P)$.
    \item Zu jedem Unterraum $Y$ von $X$ gibt es eine Projektion auf $Y$.
  \end{itemize}
\end{lem}

% Bemerkung: Die Projektion in (4) ist i.\,A. nicht stetig; dazu in Kürze mehr

% Im Abschnitt 3 hatten wir für normierte Räume $X$ bereits die Menge der \emph{stetigen (linearen) Projektionen} $\mathcal{P}(X) = \Set{P \in \mathcal{L}(X)}{P^2 = P}$ eingeführt

\begin{lem}
  Für $P \in \mathcal{P}(X)$ gilt:
  \begin{itemize}
    \item $\mathrm{ker}(P)$ und $\mathrm{im}(P)$ sind abgeschlossen
    \item $\norm{P} \geq 1$ oder $\norm{P} = 0$
  \end{itemize}
\end{lem}

\begin{satz}[vom abgeschlossenen Komplement]
  Sei $X$ ein Banachraum. Gegeben sei ein abgeschlossener Unterraum $Y$ sowie ein Unterraum $Z$ mit $X = Y \oplus Z$. Dann gilt:
  \[ \ex{\text{stetige Projektion $P$ auf $Y$ mit $Z = \ker(P$}} \iff Z \text{ ist abgeschlossen} \]
\end{satz}

\begin{bem}
  Ist $Y$ abgeschlossener Unterraum eines Banachraumes $X$, so besitzt $Y$ ein abeschlossenes Komplement genau dann, wenn es eine stetige Projektion auf $Y$ gibt.
\end{bem}

Zwei wichtige Klassen von Unterräumen, die ein abgeschlossenes Komplement besitzen, sind endlich-dimensionale Unterräume beliebiger Banachräume sowie abgeschlossene Unterräume von Hilberträumen.


% Vorlesung vom 23.1.2014

\begin{satz}
  Sei $X$ ein normierter Vektorraum, $E$ ein $n$-dimensionaler Unterraum mit Basis $\Set{ e_i }{ i = 1, ..., n }$ und $Y$ ein abgeschlossener Unterraum mit $Y \cap E = \{ 0 \}$. Dann gilt:
  \begin{itemize}
    \item $\ex{e_1', ..., e_n' \in X'} e_j' = 0$ auf $Y$ und $\langle e_j', e_i \rangle = \delta_{ij}$.
    \item $\ex{\text{stetige Projektion $P$ auf $E$ mit $Y = \ker (P)$, nämlich $P_X = \sum_{j=1}^n \langle e_j', x \rangle e_j$}}$
  \end{itemize}
\end{satz}

\begin{lem}
  Ist $Y$ abgeschlossener Unterraum eines Hilbertraums $X$ und $P$ die orthogonale Projektion aus Abschnitt 2.1, so gilt
  \begin{itemize}
    \item $P \in \mathcal{P}(X)$
    \item $\mathrm{im}(P) = Y$ und $\ker (P) = Y^\perp$
    \item $X = Y \perp Y^\perp$
    \item Ist $Z \subset X$ Unterraum mit $X = Z \perp Y$, so gilt $Z = Y^\perp$.
  \end{itemize}
\end{lem}

Als Alternative zum Zugang in Abschnitt 2.1 lässt sich festhalten:

\begin{lem}
  Seien $X$ Hilbertraum und $P : X \to X$ linear. Dann sind äquivalent:
  \begin{itemize}
    \item $P$ ist die orthogonale Projektion auf $\mathrm{im}(P)$, d.\,h. $\fa{x,y \in X} \norm{x-Px} \leq \norm{x - Py}$
    \item $\fa{x,y \in X} (x-Px | Py) = 0$
    \item $P^2 = P$ und $\fa{x,y \in X} (Px | y) = (x | Py)$
    \item $P \in \mathcal{P}(X)$ mit $\norm{P} \leq 1$
  \end{itemize}
\end{lem}

Sei $X$ Banachraum und $X_n$ endlich-dimensionale Unterräume wie in (2) des ersten Lemmas des Kapitels. Dann gibt es nach Aussage (2) des obigen Satzes also $P_n \in \mathcal{P}(X)$ mit $X_n = \mathrm{im}(P_n)$. Eine stärkere Eigenschaft als (2) des ersten Lemmas ist:

(P1) $\fa{x \in X} P_n x \xrightarrow{n \to \infty} x$

(P1) impliziert nach dem Satz von Banach-Steinhaus $C = \sup_{n \in \N} \norm{P_n} < \infty$.

Wir forden noch:

(P2) $\fa{m, n} P_n \circ P_m = \P_{\min(n, m)}$

Man rechnet leicht nach, dass zu einer Folge $(P_n)_{n \in \N}$ mit (P1), (P2) mittels $Q_n \coloneqq P_n - P_{n-1}$ (wobei $P_1 = 0$) bzw. $P_n = \sum_{i=0}^n Q_i$ eine Folge $(Q_n)_{n \in \N}$ in $\mathcal{P}(X)$ mit

(Q1) $\fa{x \in X} \sum_{i=0}^n Q_i x \xrightarrow{n \to \infty} x$
(Q2) $\fa{m,n} Q_n \circ Q_m = \delta_{mn} Q_n$

Die Unterräume $E_n = \mathrm{im}(Q_n)$ erfüllen dann (3) aus dem ersten Lemma und (2) mit $X_n = E_0 \oplus ... \oplus E_n$.

% Beispiele
\begin{itemize}
  \item Ist $X$ Hilbertraum und $X = \overline{\bigcup_{n \in \N} X_n}$ mit $\mathrm{dim} X_n < \infty$, $X_n \subset X_{n+1}$, so sei $P_n$ die orthogonale Projektion auf $X_n$ und mit $X_{n+1} = X_n \perp E_n$ sei $Q_n$ die orthogonale Projektion auf $E_n$. Ist speziell $X_n = \mathrm{span} \Set{ e_i }{ 0 \leq i \leq n }$ mit einer ONB $(e_i)_{i \in \N}$, so ist
  \[ Q_n x = (x | e_n) e_n \quad \text{und} \quad P_n x = \sum_{i=0}^n (x|e_i) e_i \]
  \item Ist $(e_i)_{i \in \N}$ Schauder-Basis eines Banachraumes $X$, definiere die \emph{duale Basis} $(e_i')_i$ durch $e_i' = \alpha_i$ für $i \in \N$, falls $\sum_{i=1}^n \alpha_k e_k \xrightarrow{n \to \infty} x$. Man kann zeigen, dass für alle $i \in \N$ diese $e_i' \in X'$ eindeutig bestimmt sind. Damit ist
  \[ Q_n = \langle e_n', x \rangle e_n, \quad P_n x = \sum_{i=0}^n \langle e_i', x \rangle e_i \]
  \item Zerlege $[0, 1]$ in Punkte $M_n = \Set{ x_{n,i} }{ i = 0, ...., m_n }$ mit $0 = x_{n,0} < ... < x_{n,m} = 1$ und $h_n = \max_{i} \abs{x_{n_i,i} - x_{n_i,i-1}} \xrightarrow{n \to \infty} 0$ sowie $\fa{n \in \N} M_n \subset M_{n+1}$. Sei $A_{n_i,i} = (x_{n_i,i}, x_{n_i,i})$, $h_{n_i,i} = x_{n_i,i} - x_{n_i,i-1}$. Dann ist der Raum der stückweise konstanten Funktionen bzgl. dieser Zerlegung auf Level $n$:
  \[ X_n = \Set{ \sum_{i=1}^m \alpha_i \chi_{A_{n_i,i}} }{ \alpha_i \in \K }, \mathrm{dim}(X_n) = m_n \]
  Für $f \in L^1(\left] 0, 1 \right[)$ definiere $P_n f = \sum_{i=1}^{m_n} (\frac{1}{n_{n_i,i}} \Int{A_{n_i,i}}{}{f(s)}{s}) \chi_{A_{n_i,i}}$.
  Es ist $\mathrm{im}(P_1) = X_n$ und für die Standardzerlegung $x_{n_i,i} = i 2^{-n}$ ist $E_n = \mathrm{span} \Set{ e_{n_i} }{  1 \leq i \leq 2^{n-1} }$ mit $e_0 = \chi_{\left] 0, 1 \right[}, e_{n,i} = \chi_{A_{n,2i-1}} - \chi_{A_{n,2i}}$.
\end{itemize}


% Abschnitt 8. Kompakte Operatoren

Für normierte $\K$-Vektorräume $X, Y$ hatten wir im Abschnitt 3 die Menge der kompakten linearen Operatoren von $X$ nach $Y$

\[ \mathcal{K}(X, Y) = \Set{ T \in \mathcal{L}(X, Y) }{ \overline{T(B_1(0))} \text{ ist kompakt} } \]

Wir hatten aber schon festgestellt, dass wir, wenn $Y$ vollständig, "`$\overline{T(B_1(0))} \text{ ist kompakt}$"' durch "`$T(B_1(0)) \text{ ist präkompakt}$"' ersetzen können. Außerdem gilt:

\begin{lem}
  Seien $X, Y$ Banachräume über $\K$. Dann sind äquivalent:
  \begin{itemize}
    \item $T \in \mathcal{K}(X, Y)$
    \item $M \subset X$ beschränkt $\implies$ $T(M)$ ist präkompakt
    \item Für jede beschränkte Folge $(x_n)_{n \in \N}$ besitzt $(T x_n)_{n \in \N}$ eine in $Y$ konvergente Teilfolge.
  \end{itemize}
\end{lem}

% Vorlesung vom 28.1.2014

\begin{lem}
  Seien $X, Y$ Banachräume. Dann gilt:
  \begin{itemize}
    \item Für jede lineare Abbildung $T : X \to Y$ gilt: $T$ kompakt $\implies$ $T$ vollständig. Ist $X$ zudem reflexiv, gilt auch die Rückrichtung. % Eberleyn-Schmuljan
    \item $K(X, Y)$ ist abgeeschlossener Unterraum von $\mathcal{L}(X, Y)$
    \item Ist $T \in \L(X, Y)$ mit $\dim \mathrm{im}(T) < \infty$, so ist $T \in K(X, Y)$
    \item Ist $Y$ Hilbertraum, so gilt für $T \in \mathcal{L}(X, Y)$
    \[ T \in K(X, Y) \iff \ex{(T_n)_{n \in \N} \text{ Folge in } \mathcal{L}(X, Y) \text{ mit } \mathrm{im}(T_n) < \infty} \norm{T-T_n} \xrightarrow{n \to \infty} 0 \]
    \item Für $P \in \mathcal{P}(X)$ gilt: $P \in K(X) \iff \dim \mathrm{im}(P) < \infty$
  \end{itemize}
\end{lem}

% In Anwendungen, in denen Fixpunktsätze für kompakte Operatoren benutzt werden sollen, ist noch folgende Aussage sehr nützlich:

\begin{lem}
  Für $T_1 \in \mathcal{L}(X, Y)$ und $T_2 \in \mathcal{L}(X, Y)$ gilt:
  $T_1$ oder $T_2$ kompakt $\implies$ $T_2 T_1$ kompakt
\end{lem}

\iffalse
\begin{bspe}
  \begin{itemize}
    \item Sei $\Omega \opn \R^n$, beschränkt mit $\mathcal{C}^{0,1}$-Rand. Seien $m_1 > m_2 \in \N$ und $1 \leq p_1, p_2 < \infty$ sowie $m_1 - \tfrac{n}{p_1} > m_2 - \tfrac{n}{p_2}$. Dann ist die Einbettung $\Id : W^{m_1,p_1}(\Omega) \to W^{m_2,p_2}(\Omega)$ stetig und kompakt.
    \item Viele Integraloperatoren, vgl. z.\,B. ÜA24
  \end{itemize}
\end{bspe}
\fi

% Kapitel 9. Spektraltheorie

% Seien im Folgenden, sofern nicht anders spezifiziert, $X$ ein Banachraum über $\C$ und $T \in \mathcal{L}(X)$

\begin{defn}
  Die \emph{Resolventenmenge} von $T$ ist definiert als
  \[ \rho(T) \coloneqq \Set{ \lambda \in \C }{ \ker (\lambda \Id - T) = \{ 0 \} } \text{ und } \im (\lambda \Id - T) = X, \]
  das \emph{Spektrum} von $T$ durch $\sigma(T) \coloneqq \C \setminus \rho(T)$. Das Spektrum zerlegen wir in das \emph{Punktspektrum}
  \[ \sigma_p(T) \coloneqq \Set{ \lambda \in \sigma(T) }{ \ker(\lambda \Id - T) \not= \emptyset }, \]
  das \emph{kontinuierliche Spektrum}
  \[ \sigma_c(T) \coloneqq \Set{ \lambda \in \sigma(T) }{ \ker(\lambda \Id - T) = \{ 0 \} \text{ und } \im (\lambda \Id - T) \not= X, \text{ aber } \overline{\im (\lambda \Id - T) = X } }, \]
  sowie das \emph{Restspektrum} (Residualspektrum)
  \[ \sigma_r(T) \coloneqq \Set{ \lambda \in \sigma(T) }{ \ker(\lambda \Id - T) = \{ 0 \} \text{ und } \overline{\im(\lambda \Id - T)} \not= X }. \]
\end{defn}

Offenbar ist $\lambda \in \rho(T)$ genau dann, $\lambda \Id - T : X \to X$ bijektiv ist. Nach dem Satz von der inversen Abbildung ist dies äquivalent zur Existenz von
\[ R(\lambda, T) = (\lambda \Id - T)^{-1} \in \mathcal{L}(X), \]
der sogenannten \emph{Resolvente} von $T$ in $\lambda$. Als Funktion von $\lambda$ heißt sie auch \emph{Resolventenfunktion}. Weiterhin ist $\lambda \in \sigma_p(T)$ offenbar äquivalent zu $\ex{x \not= 0} Tx = \lambda x$, dann heißt $\lambda$ \emph{Eigenwert} und $x$ \emph{Eigenvektor} (oder \emph{Eigenfunktion}). Der Unterraum $\ker(\Id \lambda - T)$ ist der \emph{Eigenraum} von $T$ zum Eigenwert $\lambda$. Er ist $T$-invariant.

% TODO: Satz von der Neumannschen Reihe

\begin{satz}
  $\rho(T)$ ist offen und $\lambda \mapsto R(\lambda, T)$ ist eine komplex-analytische Abbildung von $\rho(T)$ nach $\mathcal{L}(X)$. Es gilt für $\lambda \in \rho(T)$: $\norm{R(\lambda, T)}^{-1} \leq \dist(\lambda, \rho(T))$
\end{satz}

\begin{satz}
  Das Spektrum $\sigma(T)$ ist kompakt und nichtleer (falls $X \not= \{ 0 \}$) mit
  \[ \sup_{\lambda \in \sigma(T)} = \lim_{m \to \infty} \norm{T^m}^{\tfrac{1}{m}} \leq \norm{T}. \]
  Der Wert heißt \emph{Spektralradius}.
\end{satz}

% Vorlesung vom 30.1.2014

\begin{lem}
  \begin{itemize}
    \item Ist $\dim X < \infty$, so ist $\sigma(T) = \sigma_p(T)$.
    \item Ist $\dim X = \infty$ und $T \in K(X)$, so ist $0 \in \sigma(T)$.
  \end{itemize}
\end{lem}

\begin{bem}
  Im Punkt 2 ist i.\,A. $0$ kein Eigenwert, also $0 \not\in \sigma_p(T)$.
\end{bem}

% Im Folgenden wollen wir das Punktspektrum genauer untersuchen.
% Betrachte dazu für $T \in \mathcal{L}(X)$ und $y \in X$ das Problem:
% Finde $\lambda \in \C$ und $x \in X$ mit $Tx - \lambda x = y$.
% Ist $\lambda \in \rho(T)$, so existiert eine eindeutig bestimmte Lösung $x$ dieser Gleichung.
% Ist $\lambda \in \sigma_p(T)$, so ist die Lösung, falls sie existiert, nicht eindeutig bestimmt,
% damit $A_{\lambda} = \lambda \Id - T$ die Addition eines Elements aus $\ker(A_{\lambda})$ eine weitere Lösung ergibt. Auf der anderen Seite muss $\lambda \in \mathrm{im}(A_\lambda)$ sein, damit Lösung existieren kann.

% Eine wichtige Klasse von Operatoren sind in diesem Zusammenhang:

\begin{defn}
  Eine Abbildung $A \in \mathcal{L}(X, Y)$ heißt \emph{Fredholm-Operator}, falls gilt:
  \begin{itemize}
    \item $\dim \ker(A) < \infty$
    \item $\im(A)$ ist abgeschlossen
    \item $\codim \im(A) < \infty$
  \end{itemize}
  Der \emph{Index} eines Fredholm-Operators ist $\mathrm{ind}(A) = \dim \ker (A) - \codim \im(A)$.
\end{defn}

\begin{bspe}
  \begin{itemize}
    \item Sei $X = W^{1,2}(\Omega)$, $Y = (W^{1,2}(\Omega))'$. Dann ist $A : W^{1,2}(\Omega) \to (W^{1,2}(\Omega))'$ definiert durch $\langle Au, v \rangle \coloneqq \Int{\Omega}{}{\sum_{i,j} \partial_i v \cdot a_{ij} \partial_j u}{x}$ für $u , v \in W^{1,2}(\Omega)$, der der schwache elliptische Differentialoperatoren mit Neumann-Randbedingungen. Aus Kapitel 4.1 und 6 wissen wir: Der Kern $\ker(A)$ besteht aus den konstanten Funktionen, also ist $\dim \ker(A) = 1$. Das Bild von $A$ ist $\im(A) = \Set{ F \in Y }{ \langle F, 1 \rangle_{W^{1,2}(\Omega)} = 0 }$, also abgeschlossen mit $\codim \im(A) = 1$. Es ist $Y = \im(A) \oplus \mathrm{span} \{ F_0 \}$, wenn $\langle F_0 , v \rangle = \Int{\Omega}{}{v}{x}$. Also ist $A$ ein Fredholm-Operator mit Index $0$.
    \item Für das homogene Dirichlet-Problem ist der Operator $A : W_0^{1,2}(\Omega) \to (W_0^{1,2}(\Omega))'$ ein Isomorphismus.
  \end{itemize}
\end{bspe}

Eine wichtige Klasse von Fredhom-Operatoren sind kompakte Störungen von $\Id$. Es gilt (ohne Beweis):

\begin{satz}
  Sei $T \in K(X)$. Dann gilt für $A = \Id - T$:
  \begin{itemize}
    \item $\dim \ker T < \infty$
    \item $\im(A)$ ist abgeschlossen
    \item $\ker A = \{ 0 \} \implies \im(A) = X$
    \item $\codim \im(A) = \dim \ker(A)$
  \end{itemize}
  Insbesondere ist $A$ also ein Fredholm-Operator mit Index $0$.
\end{satz}

\begin{satz}[Riesz-Schauder -- Spektralsatz für kompakte Operatoren]
  Für $T \in K(X)$ gilt:
  \begin{itemize}
    \item Die Menge $\sigma(T) \setminus \{ 0 \}$ besteht aus höchstens abzählbar vielen Elementen mit $0$ als einzig möglichem Häufungspunkt. Falls $\abs{\sigma(T)} = \infty$, ist $\overline{\sigma(T)} = \sigma_p(T) \cup \{ 0 \}$
    \item Für $\lambda \in \sigma(T) \setminus \{ 0 \}$ ist $1 \leq n_{\lambda} = \max \Set{n \in \N_{+}}{ \ker(\lambda \Id - T)^{n-1} \not= \ker(\lambda \Id - T)^n } < \infty$. Die Zahl $n_\lambda$ heißt \emph{Ordnung} von $\lambda$, $\dim (\ker (\lambda \Id - T))$ heißt \emph{Vielfachheit} von $\lambda$.
    \item Für $\lambda \in \sigma(T) \setminus \{ 0 \}$ gilt $X = \ker (\lambda \Id - T)^{n_\lambda} \oplus \im (\lambda \Id - T)^{n_\lambda}$
  \end{itemize}
  Beide Unterräume sind abgeschlossen und $T$-invariant und der \emph{charakteristische Unterraum} $\ker (\lambda \Id - T)^{n_\lambda}$ ist endlich-dimensional.
  \begin{itemize}
    \item  Für $\lambda \in \sigma(T) \setminus \{ 0 \}$ ist $\sigma(T|_{\im (\lambda \Id - T)^{n_\lambda}}) = \sigma(T) \setminus \{ \lambda \}$
    \item Ist $E_\lambda$ für $\lambda \in \sigma(T) \setminus \{ 0 \}$ die Projektion auf $\ker (\lambda \Id - T)^{n_\lambda}$ gemäß der Zerlegung in Punkt 3, so gilt $E_\lambda E_\mu = \delta_{\lambda \mu} E_\lambda$ für $\lambda, \mu \in \sigma(T) \setminus \{ 0 \}$
  \end{itemize}
\end{satz}

\end{document}