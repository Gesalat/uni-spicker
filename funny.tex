\documentclass{cheat-sheet}

\usepackage{bbm} % Für 1 mit Doppelstrich (Indikatorfunktion)

\newcommand{\K}{\mathbb{K}}
\newcommand{\Bor}{\mathfrak{B}} % Borel
\newcommand{\Leb}{\mathcal{L}} % Lebesgue
\newcommand{\LSO}{\mathcal{L}} % Menge der linearen stetigen Operatoren
\newcommand{\fue}{\overset{\text{f.ü.}}} % fast überall
\newcommand{\dist}{\mathrm{dist}} % Entfernung (distance)
\newcommand{\diam}{\mathrm{diam}} % Durchmesser (diameter)
\newcommand{\codim}{\mathrm{codim}} % Kodimension
\newcommand{\scp}[2]{\left( #1 \!\mid\! #2 \right)} % Skalarprodukt
\newcommand{\dup}[2]{\langle #1 , #2 \rangle} % Duale Paarung
\newcommand{\inte}{\mathop{\mathrm{int}}} % Inneres (interior)
\newcommand{\clos}{\mathop{\mathrm{clos}}} % Abschluss (closure)
\newcommand{\bdry}{\mathop{\mathrm{bdry}}} % Rand (boundary)
\newcommand{\conv}{\mathop{\mathrm{conv}}} % Konvexe Hülle
\newcommand{\Graph}{\mathop{\mathrm{Graph}}} % Graph
\newcommand{\Hoel}{\mathrm{\text{Höl}}} % Hölder
\newcommand{\Alg}{\mathfrak{A}} % (Mengen-)Algebra
\newcommand{\ind}{\mathbbm{1}} % Indikatorfunktion

% Abschnittsnummerierung einschalten (entgegen cheat-sheet.cls)
\makeatletter
  % Abstand von Nummerierung und Titel verringern
  \renewcommand*{\@seccntformat}[1]{\csname the#1\endcsname\hspace{0.2cm}}
\makeatother
\renewcommand{\thesection}{\arabic{section}.} % Punkt nach Nummer
\setcounter{secnumdepth}{1}

\renewcommand{\Re}{\operatorname{Re}} % Realteil

\newcommand{\IntO}[2]{\Int{\Omega}{}{#1}{#2}} % Integral über \Omega

\newcommand{\convWith}[1]{\xrightarrow{#1 \to \infty}} % konvergiert für #1 gegen unendlich gegen
\newcommand{\convWeaklyWith}[1]{\xrightharpoonup{#1 \to \infty}} % konvergiert schwach für #1 gegen unendlich gegen
\newcommand{\convWeaklyStarWith}[1]{\xrightharpoonup[*]{#1 \to \infty}} % konvergiert schwach für #1 gegen unendlich gegen

\pdfinfo{
  /Title (Zusammenfassung Funktionalanalysis)
  /Author (Tim Baumann)
}

\begin{document}

\maketitle{Zusammenfassung Funktionalanalysis}

% Thema der Vorlesung:
% Kurz gesagt: Um unendlich-dimensionale Vektorräume und (lineare und stetige) Abbildungen zwischen solchen

% Beispiele:
% * Sei $\Omega \subset \R^n$ offen und beschränkt. Dann ist $\mathcal{C}(\overline{\Omega}) = \Set{ f : \Omega \to \R }{ f \text{ stetig} }$ mit der Norm $\norm{f}_\infty = \sup_{\omega \in \Omega} \abs{f(\omega)}$ ein Banachraum, d.\,h. ein vollständiger, normierter Vektorraum. Er ist allerdings nicht endlich-dimensional (ausgelassen: Begründung dafür).
% * Für eine lineare Abbildung zwischen endlich-dimensionalen Vektorräumen ist aus der linearen Algebra bekannt, dass sie injektiv genau dann sind, wenn sie surjektiv sind. Sind diese Vektorräume unendlich-dimensional ist dies i.\,A. falsch. Z.B. sei $C_* \coloneqq \Set{ (x_k)_{k \in \N} \text{ Folge in } \R }{ \ex{N \in \N} \fa{n \geq N} x_n = 0 }$ der Vektorraum der irgendwann abbrechenden Folgen. Der Shiftoperator
%   \[ T : C_* \to C_*, (x_0, x_1, x_2, ...) \mapsto (0, x_0, x_1, x_2, ...) \]
%   ist offensichtlich linear und injektiv, aber nicht surjektiv.
% * Sturm-Liouville-Problem (Details ausgelassen)

% Kapitel 1. Strukturen und Funktionenräume

% Kapitel 1.1.
\section{Allgemeine Strukturen}

\begin{nota}
  Sei im Folgenden $\K \in \{ \R, \C \}$.
\end{nota}

\begin{defn}
  Sei $X$ ein $\K$-Vektorraum. Eine \emph{Halbnorm} ist eine Abb. $\norm{\blank} : X \to \R, x \mapsto \norm{x}$, sodass für alle $x, y \in X$ und $\alpha \in \K$ gilt:
  \begin{itemize}
    \begin{multicols}{2}
      \item $\norm{x} \geq 0$ \enspace (Positivität)
      \item $\norm{\alpha x} = \abs{\alpha} \cdot \norm{x}$ \enspace (Homogenität)
    \end{multicols}
    \item $\norm{x + y} \leq \norm{x} + \norm{y}$ \enspace ($\triangle$-Ungleichung)
  \end{itemize}
  Eine \emph{Norm} ist eine Halbnorm, für die zusätzlich gilt:
  \[ \norm{x} = 0 \enspace \iff \enspace x = 0. \]
\end{defn}

\begin{defn}
  Sei $X$ ein $\K$-Vektorraum.
  \begin{itemize}
    \item Eine Abbildung $f : X \times X \to \K$ heißt \emph{Sesquilinearform}, wenn für alle $x, x_1, x_2, y, y_1, y_2 \in X$ und $\alpha \in \K$ gilt:
    \begin{align*}
      f(\alpha x_1 + x_2, y) &= \alpha f(x_1, y) + f(x_2, y) \tag*{(Linearität im 1. Arg)} \\
      f(x, \alpha y_1 + y_2) &= \overline{\alpha} f(x, y_1) + f(x, y_2) \tag*{(Antilinearität im 2. Arg)}
    \end{align*}
    \item Eine \emph{Hermitische Form} $f$ ist eine Sesquilinearform, für die gilt:
    \[ \fa{x, y \in X} f(x, y) = \overline{f(y, x)} \tag*{(Symmetrie)} \]
    Für alle $x \in X$ gilt dann $f(x, x) = \overline{f(x, x)}$, also ist $f(x, x)$ reell.
    \item Eine Sesquilinearform $f$ heißt \emph{positiv semidefinit}, falls $f(x, x) \geq 0$ für alle $x \in X$ gilt. Falls zusätzlich $f(x, x) = 0$ genau dann gilt, wenn $x = 0$, dann heißt $f$ \emph{positiv definit}.
    \item Ein \emph{Skalarprodukt} ist eine positiv definite Hermitische Form
    \[ \scp{\blank}{\blank} : X \times X \to \K, \quad (x, y) \mapsto \scp{x}{y}. \]
  \end{itemize}
\end{defn}

\begin{satz}
  Für eine positiv semidefinite Hermitische Form $\scp{\blank}{\blank}$ ist durch $x \mapsto \sqrt{\scp{x}{x}}$ eine Halbnorm definiert. Ist die Form auch positiv definit, also ein Skalarprodukt, handelt es sich dabei um eine Norm, die sogenannte \emph{induzierte Norm}.
\end{satz}

\begin{satz}
  Für ein Skalarprodukt $\scp{\blank}{\blank}$ auf einem $\K$-VR $X$ und die davon induzierte Norm gilt für alle $x, y \in X$:
  \begin{itemize}
    \item $\abs{\scp{x}{y}} \leq \norm{x} \cdot \norm{y}$ \pright{Cauchy-Schwarzsche Ungleichung}
    \item $\norm{x+y}^2 + \norm{x-y}^2 = 2 (\norm{x}^2 + \norm{y}^2)$ \pright{Parallelogrammidentität}
  \end{itemize}
  Gleichheit gilt bei CS genau dann, wenn $x$ und $y$ gleichgerichtet sind.
\end{satz}

\begin{defn}
  Ein $\K$-VR mit einer Norm heißt \emph{normierter Raum}, mit einem Skalarprodukt \emph{Prähilbertraum}.
\end{defn}

\begin{satz}
  Die Norm und das Skalarprodukt sind stetig.
\end{satz}

\begin{defn}
  Sei $X$ ein Prähilbertraum. Zwei Vektoren $x, y \in X$ heißen \emph{zueinander orthogonal}, notiert $x \perp y$, wenn $\scp{x}{y} = 0$.
\end{defn}

\begin{satz}
  Für zwei orthogonale Vektoren $x, y \in X$ gilt
  \[ \norm{x - y}^2 = \norm{x + y}^2 = \norm{x}^2 + \norm{y}^2. \tag*{(Pythagoras)} \]
\end{satz}

\begin{lem}
  Seien $Y$ und $Z$ Unterräume eines VR $X$, dann ist auch $Y + Z \coloneqq \Set{ y + z }{ y \in Y, z \in Z }$ ein Unterraum von $X$.
\end{lem}

\begin{defn}
  Für Unterräume $Y$ und $Z$ eines VR $X$ mit $Y \cap Z = \{ 0 \}$ heißt $Y \oplus Z \coloneqq Y + Z$ \emph{direkte Summe} von $Y$ und $Z$.
\end{defn}

\begin{defn}
  Zwei Unterräume $Y$ und $Z$ von $X$ heißen \emph{orthogonal}, notiert $Y \perp Z$, falls $\fa{y \in Y, z \in Z} y \perp z$.
\end{defn}

\begin{defn}
  Für einen $\K$-VR $X$ und einen Unterraum $Y \subset X$ heißt
  \[ Y^\perp \coloneqq \Set{ x \in X }{ \mathrm{span} \{ x \} \perp Y } \quad \text{\emph{orthog. Komplement} von $Y$}. \]
\end{defn}

% Ausgelassen: Restklassenbildung über Halbnorm

\begin{defn}
  Ein \emph{metrischer Raum} ist ein Paar $(X, d)$ mit einer Mange $X$ und einer \emph{Metrik} $d : X {\times} X \to \R$, d.\,h. für $x, y, z \in X$ gilt:
  \begin{itemize}
    \item $d(x, y) \geq 0$ \enspace und \enspace $d(x, y) = 0 \iff x = y$ \pright{Positivität}
    \miniitem{0.44 \linewidth}{$d(x, y) = d(y, x)$ (Symm.)}
    \miniitem{0.54 \linewidth}{$d(x, z) \leq d(x, y) + d(y, z)$ \enspace ($\triangle$-Ungl.)}
  \end{itemize}
\end{defn}

% Ausgelassen: Definition Halbmetrik (ohne Axiom $d(x, y) = 0 \iff x = y$)

\begin{defn}
  Sei $V$ ein $\K$-Vektorraum. Eine \emph{Fréchet-Metrik} ist eine Funktion $\rho : V \to \R_{\geq 0}$, sodass für alle $x, y \in V$ gilt:
  \begin{itemize}
    \miniitem{0.25 \linewidth}{$\rho(x) = \rho(-x)$}
    \miniitem{0.36 \linewidth}{$\rho(x) = 0 \iff x = 0$}
    \miniitem{0.36 \linewidth}{$\rho(x + y) \leq \rho(x) + \rho(y)$}
  \end{itemize}
\end{defn}

\begin{bsp}
  Auf dem $\R^n$ ist $x \mapsto \tfrac{\norm{x}}{1 + \norm{x}}$ eine Fréchet-Metrik.
\end{bsp}

\begin{defn}
  Sei $(X, d)$ ein metrischer Raum und $A, B \subset X$, so heißt
  \[ \dist(A_1, A_2) \coloneqq \inf \Set{ d(x,y) }{ x \in A_1, y \in A_2 } \quad \text{\emph{Abstand} zw. $A$ und $B$.} \]
\end{defn}

% Ausgelassen: Für $x \in X$ und $A \subset X$ schreibe $\dist(x, A) = \dist(\{ x \}, A)$.

\begin{bem}
  Für $A \subset X$ ist die Abbildung $x \mapsto \dist(x, A)$ Lipschitz-stetig mit Lipschitz-Konstante $\leq 1$.
\end{bem}

\begin{defn}
  Sei $(X, d)$ metrischer Raum, $A \subset X$, $\epsilon > 0$, dann heißt
  \[ B_\epsilon(A) \coloneqq \Set{ y \in X }{ \dist(\{ y \}, A) < \epsilon} \quad \text{\emph{$\epsilon$-Umgebung} von $A$.} \]
  Für $x \in X$ ist $B_\epsilon(x) \coloneqq B_\epsilon(\{ x \})$ die \emph{$\epsilon$-Kugel} um $x$.
\end{defn}

\begin{defn}
  Der \emph{Durchmesser} von $A \subset X$ ist definiert durch
  \[ \diam(A) \coloneqq \sup \left( \Set{ d(x, y) }{ x, y \in A } \cup \{ 0 \} \right). \]
\end{defn}

\begin{defn}
  $A {\subset} X$ mit $\diam(A) < \infty$ heißt \emph{beschränkt}.
\end{defn}

\begin{defn}
  Sei $(X, d)$ ein normierter Raum und $A \subset X$, dann heißt
  \begin{itemize}
    \item $\inte A \coloneqq A^\circ \coloneqq \Set{x \in X}{ \ex{\epsilon > 0} \! B_e(x) \subset A }$ das \emph{Innere} von $A$,
    \item $\clos A \coloneqq \overline{A} \coloneqq \Set{x \in X}{ \fa{\epsilon > 0} \! B_\epsilon(x) {\cap} A \not= \emptyset }$ \emph{Abschluss} von $A$,
    \item $\bdry A \coloneqq \partial A \coloneqq \overline{A} \setminus A^\circ$ \emph{Rand} von $A$,
    \item $A^c \coloneqq \complement A \coloneqq X \setminus A$ \emph{Komplement} von $A$.
  \end{itemize}
\end{defn}

\begin{defn}
  Eine Menge $A \subset X$ heißt \emph{offen}, falls $A = A^\circ$, und \emph{abgeschlossen}, falls $A = \overline{A}$.
\end{defn}

% Thema: Topologie

\begin{defn}
  Ein \emph{topologischer Raum} ist ein Paar $(X, \tau)$, wobei $X$ eine Menge und $\tau \subset \mathcal{P}(X)$ ein System von Teilmengen von $X$, den sogenannten \emph{offenen} Mengen, ist, sodass gilt:
  \vspace{-4pt}
  \begin{itemize}
    \miniitem{0.24 \linewidth}{$\emptyset \in \tau, X \in \tau$}
    \miniitem{0.34 \linewidth}{$\fa{\widetilde\tau \subset \tau} \bigcup^{\phantom{\mathclap{\widetilde\tau \subset U}}}_{\mathclap{U \in \widetilde\tau}} U \in \tau$}
    \miniitem{0.39 \linewidth}{$\fa{U_1, U_2 \in \tau} U_1 \cap U_2 \in \tau$}
  \end{itemize}
\end{defn}

\begin{defn}
  Sei $(X, \tau)$ ein topolischer Raum. Eine Menge $A \subset X$ heißt \emph{abgeschlossen}, wenn das Komplement offen ist, also $A^c \in \tau$.
\end{defn}

\begin{defn}
  Ein \emph{Hausdorff-Raum} ist ein topologischer Raum $(X, \tau)$, der folgendes Trennungsaxiom erfüllt:
  \[ \fa{x_1, x_2 \in X} \ex{U_1, U_2 \in \tau} x_1 \in U_1 \wedge x_2 \in U_2 \wedge U_1 \cap U_2 = \emptyset \]
\end{defn}

\begin{defn}
  Ist $(X, \tau)$ ein topologischer Raum und $A \subset X$, dann ist auch $(A, \tau_A)$ ein topologischer Raum mit der sogenannten \emph{Relativtopologie} $\tau_A \coloneqq \{ U \cap A \,|\,U \in \tau \}$.
\end{defn}

\begin{defn}
  Sei $(X, \tau)$ ein topol. Raum und $A \subset X$, dann heißt
  \begin{itemize}
    \item $A^\circ \coloneqq \Set{x \in X}{ \ex{U \in \tau} \! x \in U \text{ und } U \subset A }$ das \emph{Innere} von $A$,
    \item $\overline{A} \coloneqq \Set{x \in X}{ \fa{U \in \tau \text{ mit } x \in U} U \cap A \not= \emptyset }$ \emph{Abschluss} von $A$.
    % TODO: Rand?
  \end{itemize}
\end{defn}

\begin{defn}
  Sei $(X, d)$ ein metrischer Raum. Dann ist
  \[ (X, \tau) \quad \text{mit} \quad \tau \coloneqq \Set{A \subset X}{ \inte A = A } \]
  ein topol. Raum, wobei $\tau$ die von $d$ \emph{induzierte Topologie} heißt.
\end{defn}

\begin{bem}
  Die direkte Definitionen des Abschlusses, des Inneren, usw. für metrische Räume stimmen mit den Definitionen dieser Begriffe über die induzierte Topologie überein.
\end{bem}

\begin{defn}
  Sei $(X, \tau)$ ein topologischer Raum. Eine Teilmenge $A \subset X$ heißt \emph{dicht} in $X$, falls $\overline{A} = X$.
\end{defn}

\begin{defn}
  Ein topologischer Raum $(X, \tau)$ heißt \emph{separabel}, falls $X$ eine abzählbare dichte Teilmenge enthält. Eine Teilmenge $A \subset X$ heißt separabel, falls $(A, \tau_A)$ separabel ist.
\end{defn}

\begin{defn}
  Seien $\tau_1, \tau_2$ zwei Topologien auf einer Menge $X$. Dann heißt $\tau_2$ \emph{stärker} (oder feiner) als $\tau_1$ bzw. $\tau_1$ \emph{schwächer} (oder gröber) als $\tau_2$, falls $\tau_1 \subset \tau_2$.
\end{defn}

\begin{defn}
  Seien $d_1$ und $d_2$ Metriken auf einer Menge $X$ und $\tau_1$ und $\tau_2$ die induzierten Topologien. Dann heißt $d_1$ \emph{stärker} als $d_2$, falls $\tau_1$ stärker ist als $\tau_2$. Ist $\tau_1 = \tau_2$, so heißen $d_1$ und $d_2$ äquivalent.
\end{defn}

% Ausgelassen: Entsprechende Definition für Normen

\begin{satz}
  Seien $\norm{\blank}_1$ und $\norm{\blank}_2$ zwei Normen auf dem $\K$-VR $X$. Dann:
  \begin{itemize}
    \item $\norm{\blank}_2 \text{ ist stärker als } \norm{\blank}_1 \iff \ex{C > 0} \fa{x \in X} \norm{x}_1 \leq C \norm{x}_2$
    \item $\norm{\blank}_1 \text{ und } \norm{\blank}_2 \text{ sind äquivalent} \iff $\\
    $\ex{c, C > 0} \fa{x \in X} c \norm{x}_1 \leq \norm{x}_2 \leq C \norm{x}_1$
  \end{itemize}
\end{satz}

\begin{defn}
  Die \emph{$p$-Norm} auf dem $\K^n$ ist definiert für $p \in \left[ 1, \infty \right]$ als
  \[
    \norm{x}_p \coloneqq \left( \sum_{i = 1}^n \abs{x_j}^p \right)^{\frac{1}{p}} \text{ für } 1 \leq p < \infty, \quad
    \norm{x}_{\infty} \coloneqq \max_{1 \leq i \leq n} \abs{x_i}.
  \]
  % Ausgelassen: Alternative Bezeichung $\norm{\blank}_{\mathrm{max}}$ für $\norm{\blank}_\infty$
\end{defn}

\begin{bem}
  Alle $p$-Normen auf dem $\K^n$ sind zueinander äquivalent.
\end{bem}

\begin{defn}
  Sei $p \in \left[ 1, \infty \right]$. Dann heißt die eindeutig bestimmte Zahl $p' \in \left[ 1, \infty \right]$ mit $\tfrac{1}{p} + \tfrac{1}{p'} = 1$ \emph{dualer Exponent} von $p$.
\end{defn}

% Ausgelassen: Die euklidische Norm und die Fréchet-Metrik $\rho(x) \coloneqq \frac{\norm{x}}{1 + \norm{x}}$ erzeugen im $\K^n$ diesselbe Topologie.

% Thema: Stetigkeit

\begin{defn}
  Seien $(X, \tau_X)$ und $(Y, \tau_Y)$ Hausdorff-Räume, $S \subset X$, sowie $x_0 \in S$. Eine Funktion $f : S \to Y$ heißt \emph{stetig} in $x_0$, falls gilt:
  \[ \fa{V \in \tau_Y} f(x_0) \in V \implies \ex{U \in \tau_X \text{ mit } x_0 \in U} f(U \cap S) \subset V \]
  Ist $X = S$, so heißt $f : X \to Y$ stetige Abbildung, falls $f$ stetig in allen Punkten $x_0 \in X$ ist. Das ist genau dann der Fall, wenn das Urbild offener Mengen offen ist, d.\,h. $\fa{V \in \tau_Y} f^{-1}(V) \in \tau_X$.
\end{defn}

\begin{bem}
  In metrischen Räumen ist diese Definition äquivalent zur üblichen Folgendefinition.
\end{bem}

\begin{defn}
  Sei $(X, d)$ ein metrischer Raum und $(x_k)_{k \in \N}$ eine Folge in $X$. Die Folge heißt \emph{Cauchy-Folge}, falls
  \[ d(x_k, x_l) \convWith{k, l} 0. \]
  Ein Punkt $x \in X$ heißt \emph{Häufungspunkt} der Folge, falls es eine Teilfolge $(x_{k_i})_{i \in \N}$ gibt mit $x_{k_i} - x \convWith{i} 0$.
\end{defn}

\begin{defn}
  Ein metrischer Raum $(X, d)$ heißt \emph{vollständig}, falls jede Cauchy-Folge in $X$ einen Häufungspunkt (den Grenzwert) hat.
\end{defn}

\begin{defn}
  \begin{itemize}
    \item Ein normierter $\K$-Vektorraum heißt \emph{Banachraum}, wenn er vollständig bezüglich der induzierten Metrik ist.
    \item Ein Banachraum $X$ heißt \emph{Banach-Algebra}, falls er eine Algebra ist mit $\norm{x \cdot y}_X \leq \norm{x}_X \cdot \norm{y}_X$.
    \item Ein \emph{Hilbertraum} ist ein Prähilbertraum, der vollständig bzgl. der vom Skalarprodukt induzierten Norm ist.
  \end{itemize}
\end{defn}

\begin{lem}
  Ist $(X, d)$ ein vollständiger metrischer Raum und $Y \subset X$ abgeschlossen, so ist auch $(Y, d|_Y)$ ein vollständiger metr. Raum.
\end{lem}

% Ausgelassen: Beispiel $\R$ ist ein Banachraum, genauso $\K^n$

\begin{bem}
  Ein normierter Raum $X$ ist genau dann ein Prähilbertraum, falls die Parallelogrammidentität
  \[ \fa{x,y \in X} \norm{x+y}^2 + \norm{x-y}^2 = 2 \left(\norm{x}^2 + \norm{y}^2\right) \]
  gilt. Folglich ist ein Banachraum genau dann ein Hilbertraum, falls die Parallelogrammidentität gilt.
\end{bem}

% Thema: Folgenräume

\begin{defn}
  Sei $\K^\N \coloneqq \{ (x_n)_{n \in \N} \text{ Folge in } \K \}$. Die Fréchet-Metrik
    \[ \rho(x) \coloneqq \sum_{i = 1}^\infty 2^{-i} \tfrac{ \abs{x_i} }{ 1 + \abs{x_i} } < 1 \]
  macht $\K^\N$ zu einem metrischen Raum, dem \emph{Folgenraum}.
\end{defn}

\begin{satz}
  Sei $(x^k)_{k \in \N}$ eine Folge in $\K^\N$ mit $x^k = (x^k_i)_{i \in \N}$ und $x = (x_i)_{i \in \N} \in \K^\N$, so gilt
  \[ \rho(x^k - x) \convWith{k} 0 \iff \fa{i \in \N} x_i^k \convWith{k} x_i. \]
\end{satz}

\begin{satz}
  Der Folgenraum $\K^\N$ ist vollständig.
\end{satz}

\begin{defn}
  Für $p \in \left[ 1, \infty \right]$ und $x = (x_i)_{i \in \N} \in \K^\N$ heißt die Norm
  \begin{align*}
    \norm{x}_{\ell^p} &\coloneqq \left( \sum_{i=1}^\infty \abs{x_i}^p \right)^{\frac{1}{p}} \!\in \left[ 0, \infty \right], \text{ für } 1 \leq p < \infty \\
    \norm{x}_{\ell^\infty} &\coloneqq \sup_{i \in \N} \abs{x_i} \in \left[0, \infty\right]
  \end{align*}
  \emph{$\ell^p$-Norm} auf dem Raum $\ell^p(\K) \coloneqq \Set{ x \in \K^\N }{ \norm{x}_{\ell^p} < \infty }$.
\end{defn}

\begin{satz}
  Der Raum $(\ell^p(\K), \norm{\blank}_{\ell^p})$ ist ein Banachraum.
\end{satz}

\begin{bem}
  Im Fall $p {=} 2$ ist $\ell^2(\K)$ ein Hilbertraum mit Skalar- produkt $\scp{x}{y}_{\ell^2} \coloneqq \sum_{i = 0}^\infty x_i \overline{y_i}$ für $x = (x_i)_{i \in \N}, \, y = (y_i)_{i \in \N} \in \ell^2(\K)$.
\end{bem}

% Thema: Vervollständigung

\begin{satz}[Vervollständigung]
  Sei $(X, d)$ ein metrischer Raum. Betrachte die Menge $X^\N$ aller Folgen in $X$ und definiere
  \[ \widetilde{X} \coloneqq \Set{ x \in X^\N }{ x \text{ ist Cauchy-Folge in } X }/\sim \]
  mit der Äquivalenzrelation $x \sim y \text{ in } \widetilde{X} \!\coloniff\! d(x_j, y_j) \convWith{j} 0$.
  Diese Menge wird mit der Metrik
  \[ \widetilde{d}(x, y) \coloneqq \lim_{\mathclap{i \to \infty}} d(x_i, y_i) \]
  zu einem vollständigen metrischen Raum. Die injektive Abbildung $J : X \to \tilde{X}$, welche $x \in X$ auf die konstante Folge $(x)_{i \in \N}$ abbildet, ist isometrisch, d.\,h.
  $\fa{x, y \in X} \widetilde{d}(J(x), J(y)) = d(x, y)$.
  Wir können also $X$ als einen dichten Unterraum von $\widetilde{X}$ auffassen.
\end{satz}

\begin{defn}
  Man nennt $\widetilde{X}$ \emph{Vervollständigung} von~$X$.
\end{defn}

% Kapitel 1.2.
\subsection{Funktionenräume}

\begin{nota}
  Sei im Folgenden $Y$ ein Banachraum.
\end{nota}

\begin{defn}
  Sei $S$ eine Menge. Dann ist
  \[ B(S, Y) \coloneqq \Set{ f : S \to Y }{ f(S) \text { ist beschränkt in } Y } \]
  der \emph{Raum der beschränkten Funktionen} von $B$ nach $Y$. Diese Menge ist ein $\K$-Vektorraum und wird mit der \emph{Supremumsnorm} $\norm{f}_{B(S)} \coloneqq \sup_{x \in S} \norm{f(x)}$ zu einem Banachraum.
\end{defn}

\begin{defn}
  % Ausgelassen: kompakt <=> beschränkt und abgeschlossen im $\R^n$
  Sei $S \subset \R^n$ kompakt, dann ist
  \[ \mathcal{C}^0(S, Y) \coloneqq \mathcal{C}(S, Y) \coloneqq \Set{ f : S \to Y }{ f \text{ ist stetig } } \]
  der \emph{Raum der stetigen Funktionen} von $S$ nach $Y$. Er ist ein abgeschlossener Unterraum von $B(S, Y)$ mit der Supremumsnorm, also ein Banachraum.
\end{defn}

% Weiter nach oben im Skript verschoben (passt hier nicht wirklich rein):
% Lemma: Abgeschlossene Unterräume vollständiger metrischer Räume sind vollständig

\begin{bem}
  Für $Y = \K$ ist $\mathcal{C}^0(S; \K) = \mathcal{C}(S)$ eine kommutative Banach-Algebra mit dem Produkt $(f \cdot g)(x) \coloneqq f(x) \cdot g(x)$.
\end{bem}

\begin{defn}
  Sei $S \subset \R^n$ und $(K_n)_{n \in \N}$ eine Folge kompakter Teilmengen des $\R^n$. Dann heißt $(K_n)$ eine \emph{Ausschöpfung} von $S$, falls:
  \begin{itemize}
    \item Für alle $x \in S$ gibt es ein $\delta > 0$ und $i \in \N$, sodass $B_\delta(x) \cap S \subset K_i$.
    \\[-4pt]
    \miniitem{0.3 \linewidth}{$S = \bigcup_{n \in \N}^{\phantom{n \in \N}} K_n$}
    \miniitem{0.67 \linewidth}{$\emptyset \not= K_i \subset K_{i+1} \subset S$ für alle $i \in \N$}
  \end{itemize}
\end{defn}

\begin{bem}
  Zu $S \opn \R^n$ und $S \cls \R^n$ existiert eine Ausschöpfung.
\end{bem}

\begin{defn}
  Es sei $S \subset \R^n$ so, dass eine Ausschöpfung $(K_i)_{i \in \N}$ von $S$ existiert. Dann ist
  \[ C^0(S; Y) \coloneqq \Set{ f : S \to Y }{ f \text{ ist stetig auf } S } \]
  der \emph{Raum der stetigen Funktionen} von $S$ nach $Y$. Er ist ein vollständiger metrischer Raum mit der Fréchet-Norm
  \[ \varrho(f) \coloneqq \sum_{i \in \N} 2^{-i} \tfrac{ \norm{f}_{C^0(K_i)} }{ 1 + \norm{f}_{C^0(K_i)} }. \]
\end{defn}

\begin{bem}
  \begin{itemize}
    \item Die von dieser Metrik erzeugte Topologie ist unabhängig von der Wahl der Ausschöpfung.
    \item Ist $S \subset \R^n$ kompakt, so stimmt die Topologie mit der von $\norm{\blank}_{B(S,Y)}$ überein.
    \item ist $S \subset \R^n$ offen, so gibt es auf $\mathcal{C}^0(S)$ keine Norm, die dieselbe Topologie wie die Fréchet-Metrik $\rho$ erzeugt.
  \end{itemize}
\end{bem}

\begin{defn}
  Sei $S \subset \R^n$. Für $f : S \to Y$ heißt
  \[ \supp f \coloneqq \overline{\Set{ x \in S }{ f(x) \not= 0 }} \subset \overline{S} \quad \text{\emph{Träger} von $f$.} \]
\end{defn}

\begin{defn}
  Sei $S \subset \R^n$ und $Y$ ein Banachraum. Dann ist
  \[ \mathcal{C}_0^0(S; Y) \coloneqq \Set{ f \in \mathcal{C}^0(S; Y) }{ \supp f \text{ ist kompakt in } \R^n } \]
  die Menge der \emph{stetigen Funktionen mit kompaktem Träger}. %von $S$ nach $Y$.
\end{defn}

\begin{defn}
  Sei $\Omega \subset \R^n$ offen und beschränkt und $m \in \N$. Dann ist
  \begin{align*}
    \mathcal{C}^m(\overline\Omega, Y) \coloneqq \{ f : \Omega \to Y \,|\, & f \text{ ist $m$-mal stetig differenzierbar in $\Omega$ } \\[-2pt]
    & \text{ und für $k \leq m$ und } s_1, ..., s_k \in \{ 1, ..., n \} \\[-2pt]
    & \text{ ist $\partial_{s_1} \cdots \partial_{s_k} f$ auf $\overline\Omega$ stetig fortsetzbar } \}
  \end{align*}
  der \emph{Raum der differenzierbaren Funktionen} von $\Omega$ nach $Y$ und mit folgender Norm ein Banachraum:
  \[ \norm{f}_{\mathcal{C}^m(\overline\Omega)} = \sum_{\mathclap{\abs{s} \leq m}} \, \norm{\partial^s f}_{\mathcal{C}^0(\overline\Omega)} \]
\end{defn}

\begin{bem}
  In obiger Norm wird die Summe über alle $k$-fache partielle Ableitungen mit $k \leq m$ gebildet.
\end{bem}

% Ausgelassen: Allgemeinere Definition eines Stetigkeitsmoduls

\begin{defn}
  Sei $S \subset \R^n$ und $f : S \to Y$. Für $\alpha \in \left] 0, 1 \right]$ heißt
  \[ \Hoel_\alpha(f, S) \coloneqq \sup_{x, y \in S} \tfrac{\norm{f(y)-f(x)}}{\norm{y-x}^\alpha} \in \left[ 0 , \infty \right] \]
  \emph{Hölder-Konstante} von $f$ auf $S$ zum Exponenten $\alpha$.\\
  Im Fall $\alpha {=} 1$ heißt $\mathrm{Lip}(f, S) \coloneqq \Hoel_1(f, s)$ \emph{Lipschitz-Konstante}.
\end{defn}

\begin{defn}
  Ist $\Omega$ offen und beschränkt und $m \in \N$, so ist
  \[ \mathcal{C}^{m,\alpha}(\overline{\Omega}, Y) \coloneqq \Set{ f \in \mathcal{C}^m(\overline{\Omega}, Y) }{ \fa{s \text{ mit } \abs{s} {=} m} \Hoel_\alpha(\partial^s f, \overline{\Omega}) < \infty } \]
  ein sogenannter \emph{Hölder-Raum}. Er ist ein Banachraum mit Norm
  \[ \norm{f}_{\mathcal{C}^{m,\alpha}} \coloneqq \sum_{\mathclap{\abs{s} \leq m}} \, \norm{\partial^s f}_{\mathcal{C}^0(\Omega)} + \sum_{\mathclap{\abs{s} = m}} \Hoel_\alpha(\partial^s f, \overline{\Omega}). \]
\end{defn}

\begin{defn}
  Funktionen aus $\mathcal{C}^{0,\alpha}(\overline{\Omega}, Y)$ heißen \emph{Hölder-stetig} (zum Exponenten $\alpha$), Funktionen aus $\mathcal{C}^{0,1}(\overline{\Omega}, Y)$ \emph{Lipschitz-stetig}.
\end{defn}

\begin{defn}
  Der \emph{Vektorraum der unendlich oft diff'baren Fktn} und dessen Unterraum der Fktn mit kompakten Träger sind
  \[
    \mathcal{C}^\infty(\Omega, Y) \coloneqq \, \bigcap_{\mathclap{m \in \N}} \, \mathcal{C}^m(\Omega, Y)
    \quad \text{bzw.} \quad
    \mathcal{C}_0^\infty(\Omega, Y) \coloneqq \, \bigcap_{\mathclap{m \in \N}} \, \mathcal{C}_0^m(\Omega, Y).
  \]
\end{defn}

% Ausgelassen: Boolsche Algebras, Prämaßräume, elementares Lebesgue-Maß, äußeres Maß, Eigenschaften, die $\mu$-fast-überall gelten, elementare Funktionen ("`Treppenfunktionen"'), kanonische Darstellung, elementares Lebesgue-Maß

\begin{defn}
  Sei $(\Omega, \Alg, \mu)$ ein Maßraum und $Y$ ein Banachraum. Eine Funktion $f : \Omega \to Y$ heißt \emph{elementare Funktion}, wenn $f$ die Form
  \[ f = \sum_{i=1}^n \ind_{E_i} b_i \quad \text{mit $n \in \N$, $E_1, ..., E_n \in \Alg$, $b_1, ..., b_n \in Y$} \]
  und $\mu(E_i) < \infty$ für $i = 1, ..., n$ besitzt. Für eine solche Funktion heißt
  \[ \Int{\Omega}{}{f}{\mu} = \sum_{i=1}^n \mu(E_i) b_i \quad \text{Bochner-Integral.} \]
  Eine messbare Funktion $f : \Omega \to Y$ heißt \emph{Bochner-integrierbar}, wenn es eine Folge $(f_n)_{n \in \N}$ elementarer Funktionen gibt, sodass
  \[ \Int{\Omega}{}{\norm{f - f_n}}{\mu} \convWith{n} 0, \]
  wobei links das gewöchnliche Lebesgue-Integral steht. Dann heißt
  \[ \Int{\Omega}{}{f}{\mu} \coloneqq \lim_{n \to \infty} \Int{\Omega}{}{f_n}{\mu} \quad \text{\emph{Bochner-Integral} von $f$.} \]
\end{defn}

\begin{nota}
  $L(\mu, Y) \coloneqq L(\mu) \coloneqq \Set{ f : \Omega \to Y }{ \text{$f$ Bochner-integrierbar} }$
\end{nota}

% Ausgelassen: Charakterisierung der Lebesgue-integrierbaren Funktionen

\begin{defn}
  Sei $(\Omega, \Alg, \mu)$ ein Maßraum und $(Y, d)$ ein metrischer Raum. Eine Abbildung $f : \Omega \to Y$ heißt \emph{$\mu$-messbar} ($\mu$-mb), wenn gilt:
  \begin{itemize}
    \item $\fa{U \opn Y \text{ offen}} f^{-1}(U) \in \Alg$
    \item Es gibt eine $\mu$-Nullmenge $N$, sodass $f(\Omega \setminus N)$ separabel ist.
  \end{itemize}
\end{defn}

\begin{satz}[Bochner-Kriterium]
  Für $f : \Omega \to Y$ gilt:
  \[ f \in L(\mu, Y) \iff f \text{ ist $\mu$-messbar und } \abs{f} \in L(\mu, \R). \]
\end{satz}

\begin{satz}[Majoranten-Kriterium]
  Sei $f : \Omega \to Y$ $\mu$-messbar und $g \in L(\mu, \R)$ mit $\norm{f} \leq g$ $\mu$-fast-überall. Dann ist $f \in L(\mu, Y)$.
\end{satz}

% Satz über die dominierte Konvergenz:
\begin{satz}
  Sei $f : \Omega \to Y$ $\mu$-messbar und $(f_n)_{n \in \N}$ eine Folge $\mu$-messbarer Funktionen von $\Omega$ nach $Y$. Angenommen, es gilt $\norm{f_n(\omega)} \leq g(\omega)$ für alle $n \in \N$ und $\mu$-fast-alle $\omega \in \Omega$ und ein $g \in L(\mu, \R)$. Dann gilt:
  \[ f_n \xrightarrow[\text{f.ü.}]{n \to \infty} f \enspace\implies\enspace f \in L(\Omega, Y) \text{ mit } \Int{\Omega}{}{f}{\mu} = \lim_{n \to \infty} \Int{\Omega}{}{f_n}{\mu}. \]
\end{satz}

\begin{defn}
  Sei $(\Omega, \Alg, \mu)$ ein Maßraum und $Y$ ein Banachraum. Dann heißt $f : \Omega \to Y$ \emph{wesentlich beschränkt}, falls
  \[ \sup_{\omega \in \Omega \setminus N} \norm{f(\omega)} < \infty \quad \text{für eine Nullmenge $N \subset \Omega$.} \]
\end{defn}

\begin{defn}
  Sei $(\Omega, \Alg, \mu)$ ein Maßraum, $Y$ ein Banachraum und $p \in \left[ 1, \infty \right[$.
  \begin{align*}
    L^p(\mu, Y) &\coloneqq \Set{ f : \Omega \to Y }{ \text{$f$ ist $\mu$-mb und $\norm{f}^p \in L(\mu, \R)$} } / \sim \\
    L^\infty(\mu, Y) &\coloneqq \Set{ f : \Omega \to Y }{ \text{$f$ ist $\mu$-mb und wes. beschr. bzgl. $\mu$} } / \sim
  \end{align*}
  heißen \emph{Lebesgue-Räume}. Dabei ist $f \sim g$, wenn $f$ und $g$ fast-überall übereinstimmen. Sie sind Banachräume mit Norm
  \[
    \norm{f}_{L^p} \coloneqq \left( \Int{\Omega}{}{\abs{f}^p}{\mu} \right)^{\tfrac{1}{p}},
    \quad
    \norm{f}_{L^\infty} \coloneqq \!\!\!\!\! \inf_{\substack{N \subset \Omega \\ N \text{ Nullmenge}}} \left( \sup_{\omega \in \Omega \setminus N} \norm{f(\omega)} \right).
  \]
\end{defn}

% Ausgelassen: Es gilt $L(\mu, Y) = L^1(\mu, Y)$

\begin{bem}
  Für $p {=} 2$ wird, falls $Y$ ein Hilbertraum ist, $L^2(\mu, Y)$ ebenfalls zu einem Hilbertraum mit Skalarprodukt
  \[ \scp{f}{g}_{L^2(\mu, Y)} \coloneqq \Int{\Omega}{}{\scp{f}{g}_Y}{\mu}. \]
\end{bem}

\begin{satz}
  Sei $n \in \N$, $q, p_1, ..., p_n \in \left[ 1, \infty \right]$ mit $\tfrac{1}{p_1} + ... + \tfrac{1}{p_n} = \tfrac{1}{q}$ und $f_i \in L^{p_i}(\mu, \K)$ für $i \in \{ 1, ..., n \}$. Dann ist $f_1 \cdot ... \cdot f_n \in L^q(\mu, \K)$ und es gilt die Hölder-Ungleichung $\norm{f_1 \cdot ... \cdot f_n}_{L^q} \leq \norm{f_1}_{L^{p_1}} \cdot ... \cdot \norm{f_n}_{L^{p_n}}$.
\end{satz}

\begin{bem}
  Das Majorantenkriterium sowie der Konvergenzsatz von Lebesgue übertragen sich direkt auf $L^p(\mu, Y)$ mit $y < \infty$.
\end{bem}

% TODO: Was ist ein Multiindex?

\begin{defn}
  Sei $\Omega \opn \R^n$ und $f, g : \Omega \to \R$. Falls für einen Multi-Index $s$
  \[
    \Int{\Omega}{}{\partial^s \zeta f}{\lambda_n} = (-1)^{\abs{s}} \cdot \Int{\Omega}{}{\zeta g}{\lambda_n} 
    \quad
    \text{für alle } \zeta \in \mathcal{C}_0^\infty(\Omega)
  \]
  gilt, so heißt $g$ die \emph{$s$-te schwache Ableitung} von $g$.
\end{defn}

% Ausgelassen: Einführung und Motivation der Sobolev-Räume als Vervollständigung von
% \[ X \coloneqq \Set{ f \in \mathcal{C}^\infty(\Omega) }{ \norm{f}_X < \infty } \quad \text{mit} \quad \norm{f}_X \coloneqq \sum_{\abs{s} \leq m} \norm{\partial^s f}_{L^p(\Omega)}. \]

\begin{defn}
  Sei $\Omega \opn \R^n$, $m \in \N$ und $p \in \left[ 1, \infty \right]$. Dann heißt
  \begin{align*}
    W^{m,p}(\Omega) \coloneqq \{ f \in L^p(\Omega) \mid \, &\text{$f$ hat für alle Multi-Indizes $s$ mit $\abs{s} \leq m$}\\
    &\text{eine schwache Ableitung $f^{(s)} \in L^p(\Omega)$} \}
  \end{align*}
  \emph{Sobolev-Raum} der Ordnung $m$ zum Exponenten $p$. Mit der Norm
  \[ \norm{f}_{W^{m,p}(\Omega)} \coloneqq \!\! \sum_{\abs{s} \leq m} \norm{f^{(s)}}_{L^p(\Omega)} \]
  wird $W^{m,p}(\Omega)$ für $p < \infty$ zum Banachraum.
\end{defn}

\begin{samepage}

% Ausgelassen: Beispiel $f(x) \coloneqq \abs{x}$ auf $\left] 0, 1 \right[$.

\begin{bem}
  Für $p \geq 2$ sind Sobolev-Funktionen i.\,A. nicht stetig!
\end{bem}

\begin{defn}
  Sei $\Omega \opn \R^n$, $m \in \N$ und $p \in \left[ 1, \infty \right[$. Dann heißt
  \begin{align*}
    W^{m,p}_0(\Omega) \coloneqq \{ f \in W^{m,p}(\Omega) \mid \, &\text{$\exists$ Folge $(f_k)_{k \in \N}$ in $\mathcal{C}_0^\infty(\Omega)$, sodass}\\
    &\norm{f - f_k}_{W^{m,p}(\Omega)} \to 0 \text{ für } k \to \infty \}
  \end{align*}
  \emph{Sobolev-Raum mit Null-Randwerten} der Ordnung $m$ zum Exponent $p$. Er ist ein abgeschlossener Unterraum von $W^{m,p}(\Omega)$.
\end{defn}

% Kapitel 2.
\section{Teilmengen von (Funktionen-)Räumen}

\end{samepage}

% Kapitel 2.1. Konvexe Teilmengen

\begin{defn}
  Sei $X$ ein $\K$-VR. Die \emph{konvexe Hülle} von $A \subset X$ ist
  \[ \conv(A) \coloneqq \Set{ \sum_{i=0}^k a_i x_i }{ k {\in} \N, x_1, ..., x_k {\in} A, a_1, ..., a_k {\in} \R_{> 0}, \sum_{i=1}^k a_i = 1 }. \]
  Die Menge $A$ heißt \emph{konvex}, wenn $A = \conv(A)$.
\end{defn}

\begin{defn}
  Ist $A \subset X$ konvex, so heißt $f : A \to \R \cup \{ \pm\infty \}$ \emph{konvexe Funktion}, falls für alle $x, y \in A$ und $t \in \left[ 0, 1 \right]$ gilt:
  \[ f((1{-}t)x + t y) \leq (1-t) f(x) + t f(y) \]
  Eine Funktion $g : A \to \R \cup \{ \pm \infty \}$ heißt konkav, falls $-g$ konvex ist.
\end{defn}

\begin{satz}
  Sei $X$ ein Hilbertraum und $A \subset X$ nicht leer, abgeschlossen und konvex. Dann gibt es genau eine Abbildung $P : X \to A$ mit
  \[ \norm{x - P(x)}_X = \dist(x, A) = \inf_{y \in A} \norm{x - y}_X \quad \text{für alle $x \in X$.} \]
  Für $x \in X$ ist eine äquivalente Charakterisierung von $P(x)$ durch
  \[ \Re \scp{x-P(x)}{a-P(x)}_X \leq 0 \quad \text{für alle $a \in A$} \]
  gegeben. Die Abbildung $P$ heißt \emph{orthogonale Projektion} auf $A$.
\end{satz}

\begin{defn}
  Sei $X$ ein $\K$-VR. $A \subset X$ heißt \emph{affiner Unterraum}, falls
  \[ (1{-}\alpha) x + \alpha y \in A \quad \text{für alle $x, y \in A$, $\alpha \in \K$.} \]
\end{defn}

\begin{satz}
  Ist im vorherigen Satz $A$ nicht leer, abgeschlossen und affiner Unterraum von $X$, dann ist $P$ affin linear, d.\,h.
  \[ P\left( (1{-}\alpha) x + \alpha y \right) = P \left( (1{-}\alpha) x \right) + P \left( \alpha y \right) \quad \text{für alle $x, y \in A$, $\alpha \in \K$.} \]
\end{satz}

\begin{satz}
  Ist im vorherigen Satz $A$ nicht leer und abgeschlossener Unterraum von $X$, dann ist $P$ linear und $\fa{x \in X} x - P(x) \perp A$.
\end{satz}

\begin{satz}[vom fast orthogonalen Komplement]
  Sei $X$ ein normierter Raum, $Y \subset X$ ein abgeschlossener echter Teilraum und $\theta \in \left] 0, 1 \right[$ (bzw. $\theta \in \left] 0, 1 \right]$, falls $X$ ein Hilbertraum). Dann gibt es ein $x_0 \in X$ mit $\norm{x_0} = 1$ und $\theta \leq \dist(x_0, Y) \leq 1$.
\end{satz}

% Kapitel 2.2. Kompakte Teilmengen

\begin{defn}
  Eine Teilmenge $A \subset X$ heißt \emph{präkompakt}, falls es für jedes $\epsilon > 0$ eine Überdeckung von $A$ mit endlich vielen $\epsilon$-Kugeln $A \subset B_{\epsilon}(x_1) \cup ... \cup B_{\epsilon}(x_{n_\epsilon})$ mit $x_1, ..., x_{n_\epsilon} \in X$ gibt.
\end{defn}

\begin{defn}
  Sei $A \subset X$ eine Menge. Eine Überdeckung von $A$ ist ein System von Teilmengen $\Set{ A_i \subset X }{ i \in I }$, sodass $A \subset \bigcup_{i \in I} A_i$.
\end{defn}

\begin{defn}
  Eine Teilmenge $A \subset X$ eines metrischen Raumes $(X, d)$ heißt \emph{kompakt}, falls eine der folgenden äquiv. Aussagen gilt:
  \begin{itemize}
    \item $A$ ist \emph{überdeckungskompakt}: Für jede offene Überdeckung $\Set{A_i \opn X}{i \in I}$ von $A$ gibt es eine endliche Teilmenge $J \subset I$, sodass $\Set{A_i}{i \in J}$ ebenfalls eine Überdeckung von $A$ ist.
    \item $A$ ist \emph{folgenkompakt}: Jede Folge in $A$ besitzt eine konvergente Teilfolge mit Grenzwert in $A$.
    \item $(A, d|_A)$ ist vollständig und $A$ ist präkompakt.
  \end{itemize}
\end{defn}

\begin{acht}
  In topologischen Räumen sind die obigen Begriffe i.\,A. nicht äquivalent.
\end{acht}

\begin{satz}
  Sei $(X, d)$ ein metrischer Raum und $A \subset X$. Dann gilt:
  \begin{itemize}
    \item $A$ präkompakt $\implies$ $A$ beschränkt,
    \item $A$ kompakt $\implies$ $A$ abgeschlossen und präkompakt,
    \item Falls $X$ vollständig, dann $A$ präkompakt $\iff$ $\overline{A}$ kompakt.
  \end{itemize}
\end{satz}

\begin{satz}
  Sei $A \subset \K^n$. Dann gilt: \enspace
  \begin{inparaitem}
    \item $A$ präkompakt $\iff$ $A$ beschränkt,
  \end{inparaitem}
  \begin{itemize}
    \item $A$ kompakt $\iff$ $A$ abgeschlossen und beschränkt (Heine-Borel).
  \end{itemize}
\end{satz}

\begin{lem}
  Jeder endlich-dimensionale Unterraum eines normierten Raumes ist vollständig und daher abgeschlossener Unterraum.
\end{lem}

\begin{lem}
  Sei $(X, d)$ ein metrischer Raum. Ist $Y \subset X$ und $(Y, d|_{Y \times Y})$ vollständig, so ist $Y$ abgeschlossen in $X$.
\end{lem}

\begin{satz}
  Für jeden normierten Raum $X$ gilt:
  \[ \overline{B_1(0)} \text{ kompakt } \iff \dim(X) < \infty. \]
\end{satz}

% Ausgelassen: Man kann $\overline{B_1(0)}$ durch jede andere abgeschlossene Kugel ersetzen.

\begin{satz}
  Sei $(X, d)$ ein metrischer Raum und $A \subset X$ kompakt. Dann gibt es zu $x \in X$ ein $a \in A$ mit $d(x, a) = \dist(x, A)$.
\end{satz}

\begin{defn}
  Sei $S \subset \R^n$ kompakt, $Y$ ein Banachraum und $A \subset \mathcal{C}^0(S, Y)$.
  $A$ heißt \emph{gleichgradig stetig}, falls $\sup_{f \in A} \norm{f(x) - f(y)} \xrightarrow[\norm{x - y} \to 0]{} 0$.
\end{defn}

\begin{satz}[Arzelà-Ascoli]
  Sei $S \subset \R^n$ kompakt, $Y$ ein endlich- dimensionaler Banachraum und $A \subset \mathcal{C}^0(S, Y)$. Dann gilt
  \[ A \text{ präkompakt } \iff A \text{ ist pktw. beschränkt und gleichgradig stetig. } \]
\end{satz}

% Ausgelassen: Bemerkung: Es gibt eine Erweiterung auf unendlich-dimensionales $Y$ (mit Zusatzbedingung)

% Ausgelassen: Beispiel: Teilmengen von Hölder-Räumen

% Ausgelassen: Etwas allgemeinere Definition von Faltung mit zweistelligem $f$

\begin{defn}
  Sei $\varphi \in L^1(\lambda_n, \R)$, $f \in L^p(\lambda_n, Y)$ mit $p \in \left[ 1, \infty \right]$. Dann heißt
  \[ (\varphi * f) : \R^n \to Y, \quad x \mapsto \Int{\R^n}{}{\varphi(x - y) \cdot f(y)}{y} \]
  \emph{Faltung} von $\varphi$ mit $f$. Es gilt $\varphi * f \in L^p(\lambda_n, Y)$.
\end{defn}

\begin{satz}
  Es gilt in diesem Fall die Faltungsabschätzung
  \[ \norm{\varphi * f}_{L^p(\lambda_n, Y)} \leq \norm{\varphi}_{L^1(\lambda_n, \R)} \cdot \norm{f}_{L^p(\lambda_n, Y)}. \]
\end{satz}

\begin{satz}
  $\supp (\varphi * f) \subset \Set{x + y}{x \in \supp(\varphi), y \in \supp(f)}$
\end{satz}

\begin{lem}
  Ist $\varphi \in \mathcal{C}_0^\infty(\lambda_n, \R)$, $f \in L^p(\lambda_n, \R)$, so ist $\varphi * f \in \mathcal{C}^\infty(\lambda_n, \R)$ und für einen beliebigen Multi-Index $s$ gilt: $\partial^s (\varphi * f) = (\partial^s \varphi) * f$.
\end{lem}

% Vielleicht weitere Eigenschaften der Faltung einfügen:
% * Symmetrie
% * Assoziativität

\begin{bem}
  $L^1(\lambda_n, \K)$ ist mit der Faltung eine Banach-Algebra.
\end{bem}

\begin{defn}
  Eine Folge $(\varphi_k)_{k \in \N}$ heißt (allgemeine) \emph{Dirac-Folge}, falls
  \begin{align*}
    \varphi_k \geq 0, \quad \Int{\R^n}{}{\varphi_k}{\lambda_n} = 1, \quad &\text{für alle $k \in \N$,}\\
    \Int{\mathclap{\R^n \setminus B_\rho(0)}}{}{\varphi_j}{\lambda_n} \convWith{j} 0 \quad &\text{für alle $\epsilon > 0$.}
  \end{align*}
\end{defn}

\begin{satz}
  Sei $\varphi \in L^1(\lambda^n, \R)$ mit $\varphi \geq 0$ und $\Int{\R^n}{}{\varphi}{\lambda_n} = 1$. Setze
  \[ \varphi_\epsilon : \R^n \to \R, \quad x \mapsto \epsilon^{-n} \cdot \varphi(\tfrac{x}{\epsilon}). \]
  Dann ist $(\varphi_\epsilon)_{\epsilon > 0}$ eine allgemeine Dirac-Folge.
\end{satz}

\begin{defn}
  Sei $\varphi \in \mathcal{C}^\infty_0(\R^n)$, sodass $\supp(\varphi) \subset B_1(0) \subset \R^n$. Dann heißt die Folge $(\varphi_\epsilon)_{\epsilon > 0}$ aus obigem Satz \emph{Standard-Dirac-Folge}.
\end{defn}

\begin{lem}
  Sei $p \in \left[ 1, \infty \right[$. Dann gilt für $f \in L^p(\lambda_n, Y)$:
  \begin{itemize}
    \item $\norm{f(\cdot + h) - f}_{L^p(\lambda_n)} \xrightarrow{\abs{h} \to 0} 0$ mit $h \in \R^n$.
    \item Ist $(\varphi_k)_{k \in \N}$ eine Dirac-Folge, so gilt $\varphi_k * f \convWith{k} f$ in $L^p(\lambda_n, Y)$.
  \end{itemize}
\end{lem}

\begin{satz}
  Sei $p \in \left[ 1, \infty \right[$, $\Omega \opn \R^n$ offen und $Y$ ein Banachraum. Dann ist $\mathcal{C}_0^\infty(\Omega, Y)$ dicht in $L^p(\Omega, Y)$.
\end{satz}

% Ausgelassen: Beweis-Idee: Approximation durch Faltung mit Dirac-Folge

\begin{satz}[M. Riesz]
  Sei $p \in \left[ 1, \infty \right[$ und $Y$ ein endlich-dimensionaler Banachraum. Dann ist $A {\subset} L^p(\lambda_n, Y)$ präkompakt genau dann, wenn
  \begin{itemize}
    \begin{multicols}{2}
      \item $\sup_{f \in A} \norm{f}_{L^p(\lambda_n, Y)} < \infty$,
      \item $\sup_{f \in A} \norm{f}_{L^p(\R^n \setminus B_R(0))} \convWith{R} 0$,
    \end{multicols}
    \item $\sup_{f \in A} \norm{f(\cdot + h) - f}_{L^p(\lambda_n, Y)} \xrightarrow{\abs{h} \to 0} 0$ mit $h \in \R^n$.
  \end{itemize}
\end{satz}

% Bemerkung: Der Satz gilt auch für $L^p(\left] 0, 1 \right[, \R)$, bzw. in ähnlicher Form für $L^p(\Omega, Y)$ mit offenem $\Omega \subset \R^n$.

\begin{samepage}

\begin{satz}[Fundamentallemma der Variationsrechnung]
  Sei $\Omega \subset \R^n$ und $Y$ ein Banachraum. Für $g \in L^1(\Omega, Y)$ sind dann äquivalent:
  \begin{itemize}
    % (Phantome, damit Zeilen auf gleicher Höhe stehen)
    \miniitem{0.6 \linewidth}{Für alle $\xi \in \mathcal{C}_0^\infty$ gilt $\Int{\Omega}{\phantom{\Omega}}{(\xi \cdot g)}{x} = 0$.}
    \miniitem{0.37 \linewidth}{Es gilt $g \overset{\text{f.ü.}}= \underset{\phantom{\Omega}}0$ in $\Omega$.}
    \item Für alle beschränkten $E \in \Bor(\Omega)$ mit $\overline{E} \subset \Omega$ gilt $\Int{E}{}{g}{x} = 0$.
  \end{itemize}
\end{satz}

\begin{lem}
  Für $p \in \left[ 1, \infty \right[$ ist $W^{m,p}(\Omega) \cap \mathcal{C}^\infty(\Omega)$ dicht in $W^{m,p}(\Omega)$.
\end{lem}

% Kapitel 3.
\section{Lineare Operatoren}

\end{samepage}

\begin{nota}
  Seien im Folgenden $X$, $Y$ und $Z$ normierte $\K$-VRe.
\end{nota}

\begin{nota}
  Für lineare Abb. $T : X \to Y$ und $S : Y \to Z$ schreibe
  \[ Tx \coloneqq T(x), \quad Sy \coloneqq S(y), \quad ST \coloneqq S \circ T. \]
\end{nota}

\begin{satz}
  Sei $T : X \to Y$ eine lineare Abbildung zwischen Vektorräumen $X$ und $Y$. Dann sind äquivalent:
  \begin{itemize}
    \item $\ex{C > 0} \fa{x \in X} \norm{Tx} \leq C \cdot \norm{x}$.
  \end{itemize}
  \vspace{-8pt}
  \begin{itemize}
    \miniitem{0.27 \linewidth}{$T$ ist stetig.}
    \miniitem{0.4 \linewidth}{$T$ ist stetig in $x_0 \in X$.}
    \miniitem{0.3 \linewidth}{$\sup_{\norm{x} \leq 1} \norm{Tx} < \overset{\phantom{\norm{x}}}\infty$.}
  \end{itemize}
\end{satz}

\begin{bem}
  Wenn $X$ endlich-dimensional ist, dann ist jede lineare Abbildung $T : X \to Y$ stetig.
\end{bem}

\begin{defn}
  Seien $X, Y$ Vektorräume mit einer Topologie. Dann heißt
  \[ \LSO(X, Y) \coloneqq \Set{ T : X \to Y }{ X \text{ ist linear und stetig } } \]
  Raum der \emph{linearen stetigen Operatoren} zw. $X$ und $Y$ mit Norm
  \[ \norm{T}_{\LSO(X, Y)} \coloneqq \sup_{\norm{x} \leq 1} \norm{Tx}. \]
  Falls die Stetigkeit nicht nur topologisch, sondern bezüglich einer Norm gilt, so redet man von \emph{linearen beschränkten Operatoren}.
\end{defn}

\begin{nota}
  $\LSO(X) \coloneqq \LSO(X, X)$
\end{nota}

\begin{bem}
  Die linearen stetigen Operatoren zwischen VR bilden eine Kategorie, das heißt insbesondere, dass die Identitätsabbildung von einem VR in sich selbst sowie die Verkettung zweier linearer stetiger Operatoren wieder linear und stetig ist.
\end{bem}

\begin{satz}
  \begin{itemize}
    \item Ist $Y$ ein Banachraum, dann auch $\LSO(X, Y)$.
    \item $\LSO(X)$ ist eine Banachalgebra (bzgl. $\circ$), falls $X$ Banachraum.
  \end{itemize}
\end{satz}

\begin{satz}
  Für $T \in \LSO(X, Y)$ und $x \in X$ gilt: $\norm{Tx}_Y \leq \norm{T}_{\LSO(X, Y)} \cdot \norm{x}_X$.
\end{satz}

% Ausgelassen: Beispiel: $A = \Set{ T|_{\overline{B_R(0)}} }{ T \in \LSO(X, Y), \norm{T} \leq M }$

\begin{defn}
  Der Raum $X' \coloneqq \LSO(X, \K)$ heißt \emph{Dualraum} von $X$. Die Elemente von $X'$ heißen \emph{lineare Funktionale}. Schreibe
  \[ \langle x', x \rangle_{X' {\times} X} \coloneqq x'(x) \quad \text{für $x' \in X'$ und $x \in X$ (duale Paarung)}. \]
  % Ausgelassen: Norm ist die von $\LSO(X, \K)$
\end{defn}

% Ausgelassen: Definition der Menge der kompakte lineare Operatoren (kommt später)
% Ausgelassen: Definition der Menge der linearen, stetigen Projektionen (kommt später)

\begin{defn}
  Sei $T \in \LSO(X, Y)$. Dann heißt
  \begin{align*}
    \ker T \coloneqq \Set{ x \in X }{ Tx = 0 } \quad &\text{\emph{Kern} von $T$,}\\
    \im T \coloneqq T(X) \coloneqq \Set{ Tx }{ x \in X } \quad &\text{\emph{Bild} von $T$.}
  \end{align*}
\end{defn}

\begin{bem}
  Aus der Stetigkeit von $T$ folgt, dass $\ker T$ ein abgeschlossener Unterraum von $X$ ist.
\end{bem}

% Ausgelassen: Vorgriff auf Satz von der inversen Abbildung

\begin{defn}
  Der \emph{adjungierter Operator} von $T \in \LSO(X, Y)$ ist
  \[ T' : Y' \to X', \quad y' \mapsto y' \circ T. \]
\end{defn}

\begin{satz}
  Es gilt $T' \in \Leb(Y', X')$ und $\norm{T'}_{\LSO(Y', X')} = \norm{T}_{\LSO(X, Y)}$.
\end{satz}

% Steht noch an anderer Stelle im Skript (Kapitel 4.2. Dualraum von $L^p$).
\iffalse
  \begin{bsp}
    Seien $p, q \in \left[ 1, \infty \right]$ mit $\tfrac{1}{p} + \tfrac{1}{q} = 1$ und $g \in L^q(S, \K)$. Dann ist ein Funktional $T_g \in (L^p(S, \K))'$ definiert durch
    \[ T_g : L^p(S, \K) \to \K, \quad f \mapsto \Int{S}{}{f \cdot \overline{g}}{\mu}. \]
  \end{bsp}
\fi

% Ausgelassen: Beispiel: Funktional auf $(W^{m,p}(S,\K))'$
% Ausgelassen: Beispiel: Operator, der mit Dirac-Folge faltet

\begin{satz}
  Sei $X$ Banachraum und $T \in \LSO(X)$ mit $\limsup_{m \to \infty} \,\norm{T^m}^{\tfrac{1}{m}} < 1$. Dann ist $(\Id - T)$ bijektiv und $(\Id - T)^{-1} \in \LSO(X)$ mit
  \[
    (\Id - T)^{-1} = \sum_{n=0}^\infty T^n
    \quad \text{(\emph{Neumann-Reihe}).}
  \]
\end{satz}

\begin{bem}
  Damit die Voraussetzung erfüllt ist, reicht $\norm{T} < 1$.
\end{bem}

\begin{satz}
  Seien $X \not= \{0\}$, $Y \not= \{0\}$ Banachräume und $T, S \in \mathcal{L}(X, Y)$. Falls $T$ invertierbar ist mit $\norm{S-T} < \norm{T^{-1}}^{-1}$, dann auch $S$.
\end{satz}

\begin{bem}
  Die Menge aller invertierbaren Operatoren in $\mathcal{L}(X, Y)$ ist somit eine offene Teilmenge.
\end{bem}

\begin{satz}
  Sei $f(z) = \sum_{n=0}^\infty a_n z^n$ eine Potenzreihe in $\K$ mit Konvergenz- radius $\rho > 0$ und $X$ ein Banachraum. Dann gilt für $T \in \LSO(X)$:
  \[ \limsup_{n \to \infty} \norm{T^m}^{\tfrac{1}{m}} < \rho \implies f(T) \coloneqq \sum_{n=0}^\infty a_n T^n \in \LSO(X). \]
\end{satz}

\begin{bsp}
  Die Exponentialfunktion auf einem Banachraum $X$ ist
  \[ \exp : \LSO(X) \to \LSO(X), \quad T \mapsto \sum_{n=0}^\infty \tfrac{1}{n!} T^n. \]
  Für $T, S \in \LSO(X)$ mit $TS = ST$ gilt $\exp(T+S) = \exp(T) \circ \exp(S)$.
\end{bsp}

% Ausgelassen: Beispiel Evolutionsgleichungen

\begin{bsp}
  Der Logarithmus auf einem Banachraum $X$ ist
  \[
    \log : \Set{ T \in \LSO(X) }{ \norm{\Id - T} < 1 } \to \LSO(X), \enspace T \mapsto - \sum_{n=1}^{\infty} \tfrac{1}{n} (\Id - T)^n.
  \]
\end{bsp}

\begin{defn}
  Sei $\Omega \opn \R^n$ und $p \in \left[ 1, \infty \right]$. Dann heißt
  \begin{align*}
    L_{\mathrm{loc}}^p(\Omega) \coloneqq \{ f : \Omega \to \K \mid \, &\text{Für alle präkompakten $D \subset \R^n$}\\[-4pt]
    &\text{mit $D \subset \Omega$ ist $f|_D \in L^p(D)$} \}.
  \end{align*}
  Raum der zur $p$-ten Potenz \emph{lokal in $\Omega$ integrierbaren Fktn}.
\end{defn}

\begin{bem}
  Analog ist $W_{\mathrm{loc}}^{m,p}$ definiert.
\end{bem}

\begin{defn}
  Sei $\Omega \opn \R^n$ und für Multi-Indizes $s$ mit $\abs{s} \leq m$ Funktionen $a_s : \Omega \to \K$ gegeben. Dann definiert
  \[ (Tf)(x) \coloneqq \sum_{\abs{s} \leq m} a_s(x) \cdot \partial^s f(x) \]
  einen \emph{linearen Differentialoperator} der Ordnung $m$ mit Koeffizienten $a_s$. Z.\,B. ist $T \in \LSO(\mathcal{C}^m(\Omega), \mathcal{C}^0(\Omega))$.
\end{defn}

% Kapitel 4.
\section{Lineare Funktionale}

\begin{satz}[Rieszscher Darstellungssatz]
  Ist $X$ ein Hilbertraum, so ist
  \begin{align*}
    J : X \to X', \quad x \mapsto y \mapsto \langle J(x), y \rangle_{X' {\times} X} \coloneqq \scp{y}{x}_X
  \end{align*}
  ein isometrischer konjugiert lin. Isomorphismus, d\,h. für $x, y \in X$ gilt:
  \begin{itemize}
    \miniitem{0.56 \linewidth}{$\fa{\alpha \in \K} J(\alpha x + y) = \overline{\alpha} Jx + Jy$}
    \miniitem{0.41 \linewidth}{$\norm{Jx}_{X'} = \norm{x}_X$}
  \end{itemize}
\end{satz}

% Eigentlich nur der Rieszsche Darstellungssatz in etwas anderen Worten
\begin{umformulierung}
  Zu jedem $x' \in X'$ gibt es genau ein $x_0 \in X$ mit
  \[
    \fa{x \in X} \langle x', x \rangle_{X' {\times} X} = \scp{x}{x_0}_X
    \quad \text{und} \quad
    \norm{x'}_{X'} = \norm{x_0}_{X}.
  \]
\end{umformulierung}

% Kommt im Skript hinter dem Satz von Lax-Milgram, als Reformulierung für den Operator $A$.
\begin{lem}
  Sei $X$ ein Hilbertraum und $A \in \LSO(X)$ koerziv, d.\,h.
  \[ \ex{c_0 > 0} \fa{x \in X} \Re \scp{x}{Ax}_X \geq c_0 \cdot \norm{x}_X^2. \]
  Dann ist $A$ invertierbar in $\LSO(X)$ und $\norm{A^{-1}} \leq \tfrac{1}{c_0}$.
\end{lem}

\begin{satz}[Lax-Milgram]
  Sei $X$ ein Hilbertraum über $\K$ und $a : X \times X \to \K$ sesquilinear. Es gebe Konstanten $c_0$ und $C_0$ mit $0 < c_0 \leq C_0 < \infty$, sodass für alle $x, y \in X$ gilt:
  \begin{itemize}
    \item $\abs{a(x, y)} \leq C_0 \cdot \norm{x}_X \cdot \norm{y}_X$ \pright{Stetigkeit}
    \item $\Re a(x, x) \geq c_0 \cdot \norm{x}_X^2$ \pright{Koerzivität}
  \end{itemize}
  Dann existiert genau eine Abbildung $A : X \to X$ mit
  \[ a(y, x) = \scp{y}{Ax}_X \quad \text{für alle $x, y \in X$.} \]
  Außerdem gilt: $A \in \LSO(X)$ ist ein invertierbarer Operator mit
  \[
    \norm{A}_{\LSO(X)} \leq C_0
    \quad \text{und} \quad
    \norm{A^{-1}}_{\LSO(X)} \leq \tfrac{1}{c_0}.
  \]
\end{satz}

\begin{kor}
  Sei $A \in \LSO(X)$ der Operator aus dem Satz von Lax- Milgram und $J_X$ die Isometrie aus dem Rieszschen Darstellungssatz. Zu $x' \in X'$ ist dann $x \coloneqq A^{-1} J^{-1} x'$ die eindeutige Lösung zu
  \[ a(y, x) = x'(y) \quad \text{für alle $y \in X$.} \]
  Es gilt die Stabilitätsaussage $\norm{x}_X \leq \tfrac{1}{c_0} \norm{x'}_{X'}$.
  Falls $a : X {\times} X \to \K$ ein Skalarprodukt ist, so ist $x$ das eindeutige Minimum von
  \[ E(y) \coloneqq \tfrac{1}{2} a(y, y) - \Re x'(y). \]
\end{kor}

% Kapitel 4.1. Anwendung auf elliptische Randwertprobleme

\begin{prob}
  Sei $\Omega \subset \R^n$ offen, beschränkt. Gesucht ist $u \in \mathcal{C}^2$ mit

  \[ - \sum_{i=1}^n \partial_i \left( \sum_{j=1}^n a_{ij} \partial_j u + h_i \right) + bu + f = 0, \]

  wobei $a_{ij}, h_i \in \mathcal{C}^1(\Omega)$ und $f, b \in \mathcal{C}^0(\Omega)$ gegeben sind.
  Es sei $a = (a_{ij})$ gleichmäßig elliptisch, d.\,h. es gibt $c_0 > 0$, sodass

  \[ \fa{x \in \Omega} \fa{\xi \in \R^n} \xi^T a(x) \xi = \sum_{i,j}^n \xi_i a_{ij}(x) \xi_j \geq c_0 \norm{\xi}^2. \]

  Außerdem soll eine der folgenden Randwertbedingungen erfüllt sein:

  \begin{itemize}
    \item \emph{Dirichlet-Randbedingung}: $u|_{\partial \Omega} = g$ für gegebenes $g \in \mathcal{C}^0(\partial \Omega)$. Gesucht ist dann $u \in \mathcal{C}^2(\Omega) \cap \mathcal{C}^0(\overline{\Omega})$.
    \item \emph{Neumann-Randbedingung}: Auf $\partial \Omega$ gilt für geg. $g \in \mathcal{C}^0(\overline{\Omega})$
    \[ - \sum_{i=1}^n \nu_i \left( \sum_{j=1}^n a_{ij} \partial_j u + h_i \right) = g, \]
    wobei $\partial \Omega$ ein $\mathcal{C}^1$-Rand mit äußerer Normalen $\nu = (\nu_1, ..., \nu_n)$ und $a_{ij}, h_i \in \mathcal{C}^0(\overline{\Omega})$. Gesucht ist dann $u \in \mathcal{C}^2(\Omega) \cap \mathcal{C}^1(\overline{\Omega})$.
  \end{itemize}
\end{prob}

\begin{defn}
  \begin{itemize}
    \item $u : \Omega \to \R$ heißt \emph{schwache Lösung des Dirichlet-RWP}, falls $u \in W^{1,2}_0(\Omega)$ und für alle Testfunktionen $\zeta \in W_0^{1,2}(\Omega)$ gilt
    \[ \Int{\Omega}{}{\left(\sum_{i=1}^n (\partial_i \zeta) \left( \sum_{j=1}^n a_{ij} \partial_j u + h_i \right) \right) + \zeta (bu + f)}{x} = 0. \]
    \item $u : \Omega \to \R$ heißt \emph{schwache Lösung des Neumann-RWP}, falls $u \in W^{1,2}(\Omega)$ und obige Gleichung für alle $W^{1,2}(\Omega)$ erfüllt ist.
  \end{itemize}
\end{defn}

\begin{satz}
  Sei $b_0 > 0$, sodass $b(x) \geq b_0$ für fast alle $x \in \Omega$. Dann gibt es genau eine schwache Lösung $u \in W^{1,2}(\Omega)$ des Neumann-Rand- problems und es gilt mit Konstante $C > 0$ von $f$ und $h$ unabhängig:
  \[ \norm{u}_{W^{1,2}(\Omega)} \leq C (\norm{h}_{L^2} + \norm{f}_{L^2}). \]
\end{satz}

\begin{lem}[Poincaré-Ungleichung]
  Ist $\Omega \subset \R^n$ offen und beschränkt, so gibt es eine Konstante $C_p$ (abhängig von $\Omega$), sodass
  \[ \fa{u \in W_0^{1,2}(\Omega)} \norm{u}_{L^2(\Omega)}^2 \leq C_p \norm{\nabla u}_{L^2(\Omega)} \]
\end{lem}

\begin{satz}
  Sei $b \geq 0$, dann gibt es genau eine schwache Lösung $u \in W_0^{1,2}(\Omega)$ des Dirichlet-Randproblems und es gilt mit einer Konstante $C > 0$ von $f$ und $h$ unabhängig:
  \[ \norm{u}_{W^{1,2}(\Omega)} \leq C (\norm{h}_{L^2} + \norm{f}_{L^2}). \]
\end{satz}


% Kapitel 4.2. Dualraum von $L^p$

\begin{satz}
  Seien $p \in \left[ 1, \infty \right[$ und $p'$ duale Exponenten und $\Omega \subset \R^n$ beschränkt. Dann definiert
  \[ J : L^p(\Omega) \to (L^{p'}(\Omega))', \quad f \mapsto g \mapsto \Int{\Omega}{}{g \cdot \overline{f}}{\lambda_n} \]
  einen konjugiert linearen isometrischen Isomorphismus. Für $p=2$ ist $J$ gerade der Isomorphismus aus dem Rieszschen Darstellungssatz.
\end{satz}

% Ausgelassen: Beweisidee mit Formulierung des Satzes von Radon-Nikodym


% Kapitel 4.3. Satz von Hahn-Banach

% Ausgelassen: Formulierung des Lemmas von Zorn

\begin{samepage}
\begin{satz}[Hahn-Banach]
  Sei $X$ ein $\R$-VR und
  \begin{itemize}
    \item $p : X \to \R$ sublinear, d.\,h. für alle $x, y \in X$ und $\alpha \in \R_{\geq 0}$ gelte
    \[ p(x+y) \leq p(x) + p(y) \quad \text{und} \quad p(\alpha x) = \alpha p(x), \]
    \item $f : Y \to \R$ linear auf einem Unterraum $Y \subset X$ und
    \item $f(x) \leq p(x)$ für $x \in Y$.
  \end{itemize}
  Dann gibt es eine lineare Abbildung $F : X \to \R$ mit
  \[ F(x) = f(x) \text{ für $x \in Y$} \quad \text{und} \quad F(x) \leq p(x) \text{ für } x \in X. \]
\end{satz}
\end{samepage}

% Dieser Satz steht so nicht im Skript! Keine Ahnung, wo ich den herhab.
\iffalse
\begin{satz}(Hahn-Banach für lineare Funktionale)
  Sei $X$ ein $\R$-VR, $Y \subset X$ ein Unterraum, $p : X \to \R$ linear und $f : Y \to \R$ linear, sodass $f(x) \leq p(x)$ für alle $x \in Y$. Dann existiert eine lineare Abbildung $F : X \to \R$ mit $f = F|_Y \text{ und } F \leq p$.
\end{satz}
\fi

% Korollar
\begin{kor}
  Sei $(X, \norm{\blank})$ ein normierter $\K$-Vektorraum und $Y$ ein Unterraum. Dann gibt es zu $y \in Y'$ ein $x' \in X'$ mit
  \[
    x'|_Y = y'
    \quad \text{und} \quad
    \norm{x'}_{X'} = \norm{y}_{Y'}.
  \]
\end{kor}

\begin{satz}
  Sei $Y$ abgeschlossener Unterraum des normierten Raumes $X$ und $x_0 \in X \setminus Y$. Dann gibt es ein $x' \in X'$ mit
  \[
    x'|_Y = 0, \quad
    \norm{x'}_{X'} = 1
    \quad \text{und} \quad
    \langle x', x_0 \rangle = \dist(x_0, Y).
  \]
\end{satz}

\begin{satz}
  Es gibt dann auch ein $x' \in X'$ mit
  \[
    x'|_Y = 0, \quad
    \norm{x'}_{X'} = (\dist(x_0, Y))^{-1}
    \quad \text{und} \quad
    \langle x', x_0 \rangle = 1.
  \]
\end{satz}

\begin{bem}
  Der Satz kann als Verallgemeinerung des Projektions- satzes für Hilberträume im linearen Fall aufgefasst werden: Ist $X$ Hilbertraum mit abgeschlossenem Unterraum $Y$, so definiere
  \[ \langle x', x \rangle_{X' {\times} X} \coloneqq \scp{x}{\tfrac{x_0 - Px_0}{\norm{x_0 - Px_0}_X}}_X, \]
  wobei $P$ die orthogonale Projektion auf $Y$ sei. Dann gilt:
  \[
    x'|_Y = 0, \quad
    %\langle x', x_0 \rangle_{X' {\times} X} = \langle x', x_0 - Px_0 \rangle = \norm{x_0 - Px_0}_X
    \langle x', x_0 \rangle_{X' {\times} X} = \norm{x_0 - Px_0}_X
    \quad \text{und} \quad
    \langle x', x \rangle \leq \norm{x}_X.
  \]
  %Daher hat $x'$ die Eigenschaft wie im Satz. 
\end{bem}

\begin{samepage}

\begin{kor}
  Sei $X$ ein normierter Raum und $x_0 \in X$. Dann gilt
  \begin{itemize}
    %\item  Ist $x_0 \not= 0$, so gibt es $x_0' \in X'$ mit $\norm{x_0'}_{X'} = 1$ und $\langle x_0', x_0 \rangle_{X' {\times} X} = \norm{x_0}_X$.
    \item  Ist $x_0 \not= 0$, so ex. $x_0' \in X'$ mit $\norm{x_0'} {=} 1$ und $\langle x_0', x_0 \rangle_{X' {\times} X} {=} \norm{x_0}_X$.
    \item Ist $\langle x', x_0 \rangle_{X' {\times} X} = 0$ für alle $x' \in X'$, so ist $x_0 = 0$.
    \item Durch $Tx' \coloneqq \langle x', x_0 \rangle_{X' {\times} X}$ für $x' \in X'$ ist $T \in \mathcal{L}(X', \K) = X''$, dem Bidualraum, definiert mit $\norm{T}_{X''} = \norm{x_0}_X$.
  \end{itemize}
\end{kor}

% Kapitel 5.
\section{Prinzip der gleichmäßigen Beschränktheit}

\end{samepage}

\begin{satz}[Baire'scher Kategoriensatz]
  Es sei $X \not= \emptyset$ ein vollständiger metrischer Raum und $(A_k)_{k \in \N}$ eine Folge abgeschlossener Mengen $A_k \cls X$, sodass $X = \bigcup_{\mathclap{k \in \N}} A_k$. Dann gibt es $k_0 \in \N$ mit $\inte(A_{k_0}) \not= \emptyset$.
\end{satz}

\begin{kor}
  Jede Basis eines $\infty$-dim. Banachraumes ist überabzählb.
\end{kor}

\begin{satz}[Prinzip der gleichmäßigen Beschränktheit]
  Es sei $X \not= \emptyset$ ein vollständiger metrischer Raum, $Y$ ein normierter Raum und $F \subset \mathcal{C}^0(X, Y)$ eine Menge von Funktionen. Dann gilt:
  \[
    \fa{x {\in} X} \! \sup_{f \in F} \norm{f(x)}_Y {<} \infty
    \implies
    \ex{x_0 {\in} X, \epsilon {>} 0} \!\!\!\!\! \sup_{x {\in} B_\epsilon(x_0)} \sup_{f \in F} \norm{f(x)} {<} \infty.
  \]
\end{satz}

\begin{satz}[Banach-Steinhaus]
  Es sei $X$ ein Banachraum, $Y$ ein normierter Raum und $\mathcal{T} \subset \LSO(X, Y)$. Dann gilt:
  \[
    \fa{x \in X} \sup_{T \in \mathcal{T}} \norm{Tx}_Y < \infty
    \enspace \implies \enspace
    \sup_{T \in \mathcal{T}} \norm{T}_{\LSO(X, Y)} < \infty.
  \]
\end{satz}

\begin{defn}
  Seien $X$ und $Y$ topologische Räume, so heißt $f : X \to Y$ \emph{offen}, falls für alle offenen $U \opn X$ das Bild $f(U) \subset Y$ offen ist.
\end{defn}

\begin{bem}
  Ist $f$ bijektiv, so ist $f$ genau dann offen, wenn $f^{-1}$ stetig ist. Sind $X, Y$ normierte Räume und ist $T : X \to Y$ linear, so gilt: $T$ ist offen $\iff$ $\ex{\delta > 0} B_{\delta}(0) \subset T(B_1(0))$.
\end{bem}

\begin{satz}[von der offenen Abbildung]
  Seien $X, Y$ Banachräume und $T \in \LSO(X, Y)$. Dann ist $T$ genau dann surjektiv, wenn $T$ offen ist.
\end{satz}

\begin{samepage}

\begin{satz}[von der inversen Abbildung]
  Seien $X, Y$ Banachräume und $T \in \LSO(X, Y)$ bijektiv, so ist $T^{-1}$ stetig, also $T^{-1} \in \LSO(Y, X)$.
\end{satz}

\begin{satz}[vom abgeschlossenen Graphen]
  Seien $X, Y$ Banachräume und $T : X \to Y$ linear. Dann ist $\Graph(T) = \Set{ (x, Tx) }{ x \in X }$ genau dann abgeschlossen, wenn $T$ stetig ist. Dabei ist $\Graph(T) \subset X \times Y$ mit der \emph{Graphennorm} $\norm{(x,y)}_{X \times Y} = \norm{x}_X + \norm{y}_Y$.
\end{satz}


% Kapitel 6.
\section{Schwache Konvergenz}

\end{samepage}

\begin{defn}
  Sei $X$ ein Banachraum.
  \begin{itemize}
    \item Eine Folge $(x_k)_{k \in \N}$ in $X$ \emph{konvergiert schwach} gegen $x \in X$ (notiert $x_k \convWeaklyWith{k} x$), falls für alle $x' \in X'$ gilt:
    \[ \langle x', x_k \rangle_{X' {\times} X} \convWith{k} \langle x', x \rangle_{X' {\times} X} \]
    \item Eine Folge $(x'_k)_{k \in \N}$ in $X'$ \emph{konvergiert schwach*} gegen $x' \in X'$ (notiert $x'_k \convWeaklyStarWith{k} x'$), falls für alle $x \in X$ gilt:
    \[ \langle x'_k, x \rangle_{X' {\times} X} \convWith{k} \langle x', x \rangle_{X' {\times} X} \]
    \item Analog sind \emph{schwache} und \emph{schwache* Cauchyfolgen} definiert.
    \item Eine Menge $M \subset X$ (bzw. $M \subset X'$) heißt \emph{schwach folgenkompakt} bzw. \emph{schwach* folgenkompakt}, falls jede Folge in der Menge $M$ eine schwach (bzw. schwach*) konvergente Teilfolge besitzt deren Grenzwert wieder in $M$ liegt.
  \end{itemize}
\end{defn}

\begin{bem}
  Der schwache bzw. schwache* Grenzwert einer Folge ist eindeutig. Aus starker Konvergenz folgt schwache Konvergenz.
\end{bem}

\begin{satz}
  Sei $X$ ein normierter $\K$-Vektorraum. Dann ist durch
  \[ J_X : X \to X'', \quad x \mapsto x' \mapsto \langle Jx , x' \rangle_{X'' {\times} X'} \coloneqq \langle x', x \rangle_{X' {\times} X} \]
  eine isometrische Abbildung $J_X \in \LSO(X, X'')$ definiert.
\end{satz}

\begin{satz}
  Es gilt für $x, x_k \in X$, $x', x'_k \in X'$:
  \begin{align*}
    x_k \convWeaklyWith{k} x \,\, \text{ in } X \, \quad &\iff \quad J_x x_k \convWeaklyStarWith{k} J_x x \text{ in } X'' \\
    x'_k \convWeaklyWith{k} x' \text{ in } X' \quad &\implies \qquad \,\, x'_k \convWeaklyStarWith{k} x' \text{ in } X'
  \end{align*}
\end{satz}

\begin{lem}
  \begin{itemize}
    \item Aus $x'_k \convWeaklyStarWith{k} x'$ in $X'$ folgt $\norm{x'}_{X'} \leq \liminf_{k \to \infty} \|x'_k\|_{X'}$, aus $x_k \convWeaklyWith{k} x$ in $X$ folgt $\|x\|_X \leq \liminf_{k \to \infty} \norm{x_k}_X$.
    \item Schwach bzw. schwach* konvergente Folgen sind beschränkt.
    \item Aus $x_k \convWith{k} x$ in $X$ und $x'_k \convWeaklyStarWith{k} x'$ in $X'$ folgt $\langle x'_k, x_k \rangle_{X' \times X} \convWith{k} \langle x', x \rangle_{X' \times X}$. Dasselbe folgt mit $x_k \convWeaklyWith{k} x$ in $X$ und $x'_k \convWith{k} x'$ in $X'$.
  \end{itemize}
\end{lem}

\begin{acht}
  In der letzten Behauptung müssen wir voraussetzen, dass mindestens eine Folge stark konvergiert. Für beidesmal schwache/schwache* Konvergenz ist die Aussage i.\,A. falsch.
\end{acht}

\begin{bsp}
  Seien $p \in \left[ 1, \infty \right[$, $p'$ duale Exponenten und $\Omega \subset \R^n$ beschränkt. Dann gilt für $f, f_k \in L^p(\Omega)$ (jeweils mit $k {\to} \infty$):
  \[ f_k \xrightharpoonup{\phantom{A}} f \text{ in } L^p(\Omega) \iff \fa{g \in L^{p'}(\Omega)} \IntO{f_k \cdot \overline{g}}{x} \xrightarrow{\phantom{A}} \IntO{f \cdot \overline{g}}{x} \]
\end{bsp}

\begin{bsp}
  Sei $\Omega \subset \R^n$ offen und beschränkt und $p \in \left[ 1, \infty \right]$, $m \in \N_{> 0}$. Für $u, u_k \in W^{m,p}(\Omega)$ gilt dann (jeweils mit $k {\to} \infty$):
  \[ u_k \xrightharpoonup{\phantom{A}} u \text{ in } W^{m,p}(\Omega) \iff \fa{s, \abs{s} \leq m} \partial^s u_k \xrightharpoonup{\phantom{A}} \partial^s u \text{ in } L^p(\Omega) \]
\end{bsp}

\begin{satz}[Banach-Alaoglu]
  Sei $X$ ein separabler Banachraum. Dann ist die abgeschl. Einheitskugel $\overline{B_1(0)} \subset X'$ schwach* folgenkompakt.
\end{satz}

% Ausgelassen: Beweisidee: Bolzano-Weierstraß, Diagonalisierung

\begin{bsp}
  Sei $\Omega \subset \R^n$ beschränkt und offen. Dann ist $L^1(\lambda_n)$ separabel (Approx. durch elem. Funktionen). Ist $(f_k)_{k \in \N}$ in $L^{\infty}(\Omega)$ beschränkt, so gibt es eine Teilfolge $(f_{k_l})_{l \in \N}$ und $f \in L^\infty(\Omega)$, sodass
  \[ \IntO{f_{k_l}{x} \cdot \overline{g}} \convWith{l} \IntO{f \cdot \overline{g}}{x} \quad \text{für alle $g \in L^1(\Omega)$}. \]
\end{bsp}

\begin{bem}
  Schwach*-Konvergenz impliziert eine sogenannte Schwach*-Topologie in dem Sinne, dass man sagt, eine Folge $(x_k')_{k \in \N}$ in $X'$ ist bzgl. dieser Topologie konvergent, wenn sie punktweise für alle $x \in X$ konvergiert.
\end{bem}

\begin{defn}
  Sei $X$ ein Banachraum und $J_X$ die Isometrie bzgl. des Bidualraumes. Dann heißt $X$ \emph{reflexiv}, falls $J_X$ surjektiv ist.
\end{defn}

\begin{lem}
  Sei $X$ ein Banachraum.
  \begin{itemize}
    \item Ist $X$ reflexiv, so stimmen schwache* und schwache Konvergenz in $X'$ überein, d.h. für eine Folge $(x'_n)_{n \in \N}$ in $X'$ und $x' \in X'$ gilt
    \[ x_n' \convWeaklyWith{n} x' \text{ in } X' \quad \iff \quad x_n' \convWeaklyStarWith{n} x' \text{ in } X'. \]
    \item Ist $X$ reflexiv, dann auch jeder abgeschlossene Unterraum von $X$.
    \item Ist $T : X \to Y$ ein Iso, so gilt: $X \text{ reflexiv} \iff Y \text{reflexiv}$.
    \item Es gilt: $X$ reflexiv $\iff$ $X'$ reflexiv.
  \end{itemize}
\end{lem}

% Vorlesung vom 9.1.2014

\begin{lem}
  Für jeden Banachraum $X$ gilt:
  \[ X' \text{ separabel} \implies X \text{ separabel.} \]
\end{lem}

\begin{bem}
  Die Umkehrung gilt i.\,A. nicht! Gegenbeispiel: $X \coloneqq L^1$.
\end{bem}

\begin{satz}[Eberlein-Shmulyan]
  Sei $X$ reflexiver Banachraum. Dann ist die abgeschlossene Einheitskugel $\overline{B_1(0)} \subset X$ schwach folgenkompakt.
\end{satz}

% Bemerkung: Dies gilt dann auch für jede andere abgeschlossene Kugel $\overline{B_R(x)}$.

\begin{bspe}
  \begin{itemize}
    \item Hilberträume $X$ sind reflexiv (Folgerung aus dem Rieszschen Darstellungssatz). Daher: Ist $(x_k)_{k \in \N}$ eine beschränkte Folge in $X$, so existiert eine Teilfolge $(x_{k_l})_{l \in \N}$ und $x \in X$, sodass
    \[
      \scp{y}{x_{k_l}}_X \convWith{l} \scp{y}{x}_X
      \quad \text{für alle $y \in X$.}
    \]
    \item Seien $p \in \left] 1, \infty \right[$, $p'$ duale Exponenten und $\Omega \subset \R^n$ beschränkt. Dann ist $L^p(\Omega)$ reflexiv.
    \item $L^1$ und $L^\infty$ sind genau dann nicht reflexiv, wenn sie unendlich-dimensional sind.
  \end{itemize}
\end{bspe}

\begin{bem}
  Analog zur schwach*-Topologie kann man auch eine schwache Topologie einführen.
\end{bem}

% In Abschnitt 2.1 haben wir gesehen, dass das Abstandsproblem zu konvexen, abgeschlossenen Mengen in allgemeinen Banachräumen nicht lösbar ist, in reflexiven aber doch, wie wir sehen werden.

% Kapitel 6.1. Minkowski-Funktional

\begin{satz}[Trennungssatz]
  Sei $X$ ein normierter Raum, $M \subset X$ nicht leer, abgeschlossen, konvex und $x_0 \in X \setminus M$. Dann gibt es ein $x' \in X'$ und ein $\alpha \in \R$ mit
  \[
    \Re \langle x', x_0 \rangle_{X' {\times} X} > \alpha
    \quad \text{und} \quad
    \fa{x \in M} \Re \langle x', x \rangle_{X' \times X} \leq \alpha.
  \]
\end{satz}

% Ausgelassen: Geometrische Interpretation

\begin{bem}
  Dann ist $\Set{ x \in X }{ \Re \langle x', x \rangle_{X' {\times} X} = \alpha }$ eine Hyperebene, die $M$ und $x_0$ trennt.
\end{bem}

% Vorlesung vom 14.1.2014

\begin{satz}
  Sei $X$ ein normierter Raum, $M \subset X$ abgeschlossen und konvex. Dann ist $M$ schwach folgenabgeschlossen, d.\,h. für jede Folge $(x_k)_{k \in N}$ in $M$ und $x \in X$ gilt:
  \[ x_k \convWeaklyWith{k} x \text{ in } X \enspace \implies \enspace x \in M. \]
\end{satz}

\begin{lem}[Mazur]
  Sei $X$ ein normierter Raum und $(x_k)_{k \in \N}$ eine Folge in $X$ mit $x_k \convWeaklyWith{k} x$. Dann gilt $x \in \overline{\conv \Set{ x_k }{ k \in \N }}$.
\end{lem}

\begin{satz}
  Sei $X$ ein reflexiver Banachraum und $M \subset X$ nicht leer, konvex und abgeschlossen. Dann gibt es zu jedem $\widetilde{x} \in X$ ein $x \in M$ mit $\norm{x - \tilde{x}} = \mathrm{dist}(\widetilde{x}, M)$.
\end{satz}

% Der Satz über die schwache Folgenabgeschlossenheit konvexer abgeschlossener Mengen spielt eine wichtige Rolle für Existenzaussagen für partielle Differentialgleichungen mit Nebenbedingungen

\begin{satz}

  Sei $\Omega \subset \R^n$ offen, beschränkt, zusammenhängend mit $\mathcal{C}^{0,1}$-Rand sowie $\K = \R$. Sei $a_{ij} = a_{ji} \in L^\infty(\Omega)$ für $i,j = 1, ..., n$, sodass $a = (a_{ij})$ gleichmäßig elliptisch ist und $f \in L^2(\Omega)$. Betrachte $E \in W^{1,2}(\Omega) \to \R$ gegeben durch
  \[
    E(u) = \Int{\Omega}{}{ \tfrac{1}{2} \sum_{i,j=1}^n \partial_i u a_{ij} \partial_j u + f u }{x}
  \]
  Für $M \subset W^{1,2}(\Omega)$ nicht leer, konvex und abgeschlossen mit
  \[ \fa{u_0 \in M} \ex{C_0 < \infty} \fa{\xi \in \R} u_0 + \xi \in M \implies \abs{\xi} \leq C_0 \]
  gilt dann: %(ohne Beweis)
  \begin{itemize}
    \item $E$ besitzt ein absolutes Minimum auf $M$.
    \item Die absoluten Minima von $E$ auf $M$ sind genau die Lösungen der Variationsungleichung: Finde $u \in M$, sodass
    \[ \fa{v \in M} \Int{\Omega}{}{\sum_{i,j=1}^n \partial_i (u-v) a_{ij} \partial_j u + (u-v) f}{x} \leq 0. \]
    \item Ist $M$ ein abgeschlossener affiner Unterraum $M = u_0 + M_0$ mit $M_0$ Unterraum, so ist die Variationsungleichung äquivalent zu: Finde $u \in M$, sodass $\Int{\Omega}{}{\sum_{i,j=1}^n \partial_i v a_{ij} u + v f}{x} = 0$ für alle $v \in M_0$.
    \item Hat $M$ die Eigenschaft
    \[ \fa{v \in M} \fa{\xi \in \R} v + \xi \in M \implies \xi = 0, \]
    so ist die Lösung eindeutig.
  \end{itemize}
\end{satz}

\begin{bspe}
  \begin{itemize}
    \item Sei $M = W_0^{1,2}(\Omega)$. Dann ist die eindeutige Lösbarkeit des zugehörigen (schwachen) Dirichlet-Problems gesichert.
    \item Sei $M = \Set{ u \in W^{1,2}(\Omega) }{ \Int{\Omega}{}{u}{x} = 0 }$ und gelte $\Int{\Omega}{}{f}{x} = 0$. Dann sichern Punkt 3, 4 die eindeutige Lösbarkeit des zugehörigen Neumann-Problems.
    \item Seien $u_0, \psi_0 \in W^{1,2}(\Omega)$ gegeben und $u_0(x) \geq \phi_0(x)$ für fast alle $x \in \Omega$. Definiere $M = \Set{ v \in W^{1,2}(\Omega) }{ v = u_0 \text{ auf } \partial \Omega, v \geq \psi \text{ in } \Omega }$. Dann sichern die Punkte 1 bzw. 2 und 4 die eindeutige Existenz einer Lösung dieses Hindernis-Problems.
  \end{itemize}
\end{bspe}


% Vorlesung vom 16.1.2014

% Kapitel 7.
\section{Endlich-dimensionale Approximation}

% Für numerische Berechnungen ist die Approximation von Elementen oder Unterräumen unendlich-dimensionaler Räume mit endlich-dimensionaler Information von großer Relevanz. Eine abzählbare Approximation durch endlich-dimensionale Unterräume ist nur für separable Räume möglich:

\begin{lem}
  Ist $X$ $\infty$-dimensionaler Raum, so sind äquivalent:
  \begin{itemize}
    \item $X$ ist separabel.
    \item Es gibt endlich-dim. Unterräume $\Set{X_n {\subset} X}{ n {\in} \N, \dim X_n {<} \infty }$ mit
    \[
      \fa{n \in \N} X_n \subset X_{n{+}1}
      \quad \text{und} \quad
      \bigcup_{\mathclap{n \in \N}} X_n \text{ dicht in $X$.}
    \]
    \item Es gibt endlich-dim. Unterräume $\Set{E_n {\subset} X}{ n {\in} \N, \dim E_n {<} \infty }$ mit
    \[
      \fa{n \not= m} E_n \cap E_m = \{0\}
      \quad \text{und} \quad
      \bigcup_{\mathclap{n \in \N}} \, (E_0 \oplus ... \oplus E_n) \text{ dicht in $X$.}
    \]
    \item Es gibt eine linear unabhängige Menge $\Set{ e_n }{ n \in \N }$, sodass $\mathrm{span}\Set{e_n}{ n \in \N }$ dicht in $X$ liegt.
  \end{itemize}
\end{lem}

% Aussage (1) bedeutet, dass es zu $x \in X$ Punkte $x_n \in X$ gibt mit $x_n \convWith{n} x$
% Aussage (4) bedeutet, dass es zu $x \in X$ und $n \in \N$ f+r $k = 0, ...., n$ Zahlen $\alpha_{n,k}$ gibt mit $\norm{x - \sum_{k=0}^n \alpha_{n,k}}_X \convWith{n} 0$
% Falls die $\alpha_{n,k}$ unabhängig von $u$ gewählt werden können:

\begin{defn}
  Sei $X$ normierter Raum. Eine Folge $(x_n)_{n \in \N}$ in $X$ heißt \emph{Schauder-Basis} von $X$, falls für alle $x \in X$ gilt:
  \[
    \ex{\text{eindeutige Folge $(\alpha_k)_{k \in \N}$ in $\K$}} \enspace
    \sum_{k=0}^n \alpha_k e_k \convWith{n} x \text{ in } X.
  \]
\end{defn}

\begin{bem}
  Seien $X$ und $\widetilde{X}$ Banachräume mit Schauderbasen $(e_k)_{k \in \N}$ bzw. $(\widetilde{e}_l)_{l \in \N}$ und $S \in \LSO(X, \widetilde{X})$. Dann gibt es eine eindeutig bestimmte "`unendliche Matrix"' $(s_{k,l})_{k,l \in \N}$, sodass
  \[
    \fa{x = \sum_{k=0}^\infty \alpha_k e_k \in X}
    \quad
    Sx = \sum_{k=0}^\infty \sum_{l=0}^\infty \alpha_k s_{k,l} \widetilde{e}_l.
  \]
\end{bem}

\begin{defn}
  Ist $(e_i)_{i \in \N}$ Schauder-Basis eines Banachraumes $X$, dann definiere die \emph{duale Basis} $(e_i')_{i \in \N}$ von $X'$ durch
  \[ e_i' : X \to \K, \quad \sum_{i=1}^\infty \alpha_k e_k \mapsto \alpha_i. \]
\end{defn}

% Kapitel 7.1. Orthogonalsysteme

\begin{defn}
  Sei $X$ ein Prähilbertraum.
  \begin{itemize}
    \item Eine Menge $\Set{ e_k {\in} X }{ k \in N }$, $N \subset \N$ heißt \emph{Orthogonalsystem}, falls $\scp{e_k}{e_l} = 0$ für $k \not= l$ und $e_k \not= 0$ für alle $k \in \N$ gilt.
    \item Falls zusätzlich $\norm{e_k} = 1$ für alle $k \in N$ gilt, so heißt die Menge \emph{Orthonormalsystem} (ONS).
  \end{itemize}
\end{defn}

\begin{lem}[Besselsche Ungleichung]
  Sei $(e_k)_{k \in \N}$ ein (endliches) ONS des Prähilbertraumes $X$. Dann gilt für alle $x \in X$ und $n \in N$:
  \begin{align*}
    0 \leq \norm{x}^2 - \sum_{k=0}^n \abs{\scp{x}{e_k}}^2 &= \norm{x - \sum_{k=0}^\infty \scp{x}{e_k} e_k}^2\\
    &= \dist(x, \mathrm{span} \{ e_0, ..., e_n \})^2.
  \end{align*}
\end{lem}

\begin{satz}
  Sei $(e_k)_{k \in \N}$ ONS des Prähilbertraumes $X$. Dann äquivalent:
  \begin{itemize}
    \begin{multicols}{2}
      \item $(e_k)_{k \in \N}$ ist Schauder-Basis.
      \item $\mathrm{span} \Set{ e_k }{ k \in \N }$ ist dicht in $X$.
    \end{multicols}
    \vspace{4pt}
    \begin{multicols}{2}
      \item $\fa{x \in X} x = \sum_{k=0}^\infty \scp{x}{e_k} e_k$
      \item $\fa{x \in X} \norm{x}^2 = \sum_{k=0}^\infty \abs{\scp{x}{e_k}}^2$
    \end{multicols}
    \item $\fa{x, y \in X} \scp{x}{y} = \sum_{k=0}^\infty \scp{x}{e_k} \overline{\scp{y}{e_k}}$ \pright{Parseval-Identität}
  \end{itemize}
\end{satz}

\begin{defn}
  Falls eine der Eigenschaften aus dem Satz erfüllt ist, dann heißt $(e_k)_{k \in \N}$ \emph{Orthonormalbasis} (ONB).
\end{defn}

\begin{satz}
  Jeder $\infty$-dimensionale Hilbertraum $X$ über $\K$ ist genau dann separabel, wenn $X$ eine Orthonormalbasis besitzt.
\end{satz}

\begin{beweisidee}
  Gram-Schmidtsches Orthonormalisierungsverfahren
\end{beweisidee}

\begin{bem}
  In diesem Fall ist $X$ isometrisch isomorph zu $\ell^2(\K)$ (durch Übergang zu Koeffizienten bzgl. Basis).
\end{bem}


% Vorlesung vom 21.1.2014

\begin{bsp}
  Eine ONB $(e_k)_{k \in \N}$ von $L^2(\left] -\pi, \pi \right[, \C)$ ist gegeben durch
  \[
    e_k : \left] -\pi, \pi \right[ \to \C, \quad x \mapsto \tfrac{1}{\sqrt{2\pi}} e^{ikx}. \tag*{(Fourier-Basis)}
  \]

  % Ausgelassen: Reeller Fall:
  % Weiter ist durch $\widetilde{e}_0(x) = \frac{1}{\sqrt{2\pi}}$, $\widetilde{e}_{k}(x) = \frac{1}{\sqrt{2\pi}} \sin(kx)$ für $k > 0$ und $\widetilde{e}_k(x) = \frac{1}{\sqrt{2 \pi}} \cos(kx)$ für $k < 0$ eine ONB von $L^2(\left] -\pi, \pi \right[, \R)$ gegeben.
\end{bsp}
% Dazu nachzuweisen: Orthogonalsystem (Übungsaufgabe), Dichtheit (dazu folgender Beweis)

\begin{lem}
  Zu $f \in L^2(\left] -\pi, \pi \right[, \C)$ sei mit $e_k$ wie im Beispiel
  \[ P_n f = \sum_{\abs{k} \leq n} \scp{f}{e_k}_{L^2} e_k \tag*{(\emph{Fourier-Summe})} \]
  Wenn $f$ Lipschitz-stetig ist, dann gilt $f(x) = \lim_{\mathclap{n \to \infty}} P_n f(x)$.
\end{lem}

\begin{samepage}

\begin{bem}
  Die Fourier-Summe erlaubt die explizite Approx. von $f$ im Unterraum $X = \mathrm{span} \Set{e_k}{ \abs{k} \leq n }$.
\end{bem}

% Allgemein führt man ein:

% Kapitel 7.2.
\subsection{Lineare Projektionen}

\end{samepage}

\begin{defn}
  Sei $Y$ Unterraum des VR $X$. Eine lineare Abb. $P : X \to X$ heißt \emph{(lineare) Projektion auf $Y$}, falls $P^2 = P$ und $\im(P) = Y$.
\end{defn}

\begin{lem}
  Für eine Projektion $P : X \to X$ gilt:
  \begin{itemize}
    \item $X = \ker(P) \oplus \im(P)$
    \item $(\Id {-} P)$ ist Projektion mit $\ker(\Id {-} P) \!=\! \im(P)$, $\im(\Id {-} P) \!=\! \ker(P)$.
  \end{itemize}
\end{lem}

\begin{lem}
  Sei $P : X \to X$ linear. Dann gilt
  \[
    \text{$P$ ist Projektion auf $Y$}
    \enspace \iff \enspace
    \im(P) \subset Y \text{ und } P|_Y = \Id
  \]
\end{lem}

\begin{bem}
  Zu jedem Unterraum $Y \subset X$ ex. eine Projektion auf $Y$.
\end{bem}
% Die Projektion in der Bemerkung ist i.\,A. nicht stetig; dazu in Kürze mehr

% Im Skript eigentlich schon in Abschnitt 3 hatten wir für normierte Räume $X$ bereits die Menge der \emph{stetigen (linearen) Projektionen} $\mathcal{P}(X) = $ eingeführt
\begin{defn}
  Sei $X$ ein normierter Raum. Dann heißt
  \[ \mathcal{P} \coloneqq \Set{P \in \LSO(X)}{P^2 = P}. \]
  Menge der \emph{stetigen (linearen) Projektionen} auf $X$.
\end{defn}

\begin{lem}
  Für $P \in \mathcal{P}(X)$ gilt:
  \begin{itemize}
    \miniitem{0.62 \linewidth}{$\ker(P)$ und $\im(P)$ sind abgeschlossen}
    \miniitem{0.35 \linewidth}{$\norm{P} \geq 1$ oder $\norm{P} = 0$}
  \end{itemize}
\end{lem}

\begin{satz}[vom abgeschlossenen Komplement]
  Sei $X$ ein Banachraum, $Y$ abgeschlossener Unterraum sowie $Z$ ein Unterraum mit $X = Y \oplus Z$.
  Dann ist $Z$ genau dann abgeschlossen, wenn es eine stetige Projektion $P \in \mathcal{P}(X)$ auf $Y$ mit $\ker(P) = Z$ gibt.
\end{satz}

\begin{umformulierung}
  Ist $Y$ abgeschlossener Unterraum eines Banachraumes $X$, so besitzt $Y$ ein abgeschlossenes Komplement genau dann, wenn es eine stetige Projektion auf $Y$ gibt.
\end{umformulierung}

% Vorschau: Zwei wichtige Klassen von Unterräumen, die ein abgeschlossenes Komplement besitzen, sind endlich-dimensionale Unterräume beliebiger Banachräume sowie abgeschlossene Unterräume von Hilberträumen.


% Vorlesung vom 23.1.2014

\begin{satz}
  Sei $X$ ein normierter Vektorraum, $E$ ein $n$-dimensionaler Unterraum mit Basis $e_1, ..., e_n$ und $Y$ ein abgeschlossener Unterraum mit $Y \cap E = \{ 0 \}$. Dann gilt:
  \begin{itemize}
    \item $\ex{e_1', ..., e_n' \in X'} e_j'|_Y = 0$ und $\langle e_j', e_i \rangle = \delta_{ij}$.
    \item Es gibt eine stetige Projektion $P$ auf $E$ mit $Y = \ker(P)$, nämlich
    \[ P(x) \coloneqq \sum_{j=1}^n \langle e_j', x \rangle e_j. \]
  \end{itemize}
\end{satz}

\begin{lem}
  Sei $Y$ abgeschlossener Unterraum eines Hilbertraums $X$ und $P$ die orthogonale Projektion aus Kapitel 2, so gilt
  \begin{itemize}
    \begin{multicols}{4}
      \item $P \in \mathcal{P}(X)$
      \item $\im(P) = Y$
      \item $\ker(P) = Y^\perp$
      \item $X = Y \perp Y^\perp$
    \end{multicols}
    \item Ist $Z \subset X$ Unterraum mit $X = Z \perp Y$, so gilt $Z = Y^\perp$.
  \end{itemize}
\end{lem}

% Als Alternative zum Zugang in Abschnitt 2.1 lässt sich festhalten:

\begin{lem}
  Sei $X$ Hilbertraum und $P : X {\to} X$ linear. Dann äquivalent:
  \begin{itemize}
    \item $P$ ist die orthogonale Projektion auf $\im(P)$, d.\,h.
    \[ \fa{x,y \in X} \norm{x-Px} \leq \norm{x - Py}. \]
    \item $P^2 = P$ und $\fa{x,y \in X} \scp{Px}{y} = \scp{x}{Py}$
    \miniitem{0.52 \linewidth}{$\fa{x,y \in X} \scp{x-Px}{Py} = 0$}
    \miniitem{0.45 \linewidth}{$P \in \mathcal{P}(X)$ mit $\norm{P} \leq 1$}
  \end{itemize}
\end{lem}

\begin{samepage}

\begin{bem}
  Sei $X$ ein Banachraum, $\Set{X_n \subset X}{ n \in \N, \dim X_n < \infty }$ endlich-dimensionale Unterräume mit
  \[
    \fa{n \in \N} X_n \subset X_{n{+}1}
    \quad \text{und} \quad
    \bigcup_{\mathclap{n \in \N}} X_n \text{ liegt dicht in $X$.}
  \]
  Sei $P_n$ die Projektion auf $X_n$ für $n \in \N$. Dann gilt
  \[ \fa{x \in X} P_n x \convWith{n} x \text{ in } X. \]
  Nach dem Satz von Banach-Steinhaus folgt $C \coloneqq \sup_{n \in \N} \norm{P_n} < \infty$.
\end{bem}

% Obige Bemerkung in Rohfassung
\iffalse
Sei $X$ Banachraum und $X_n$ endlich-dimensionale Unterräume wie in (2) des ersten Lemmas des Kapitels. Dann gibt es nach Aussage (2) des obigen Satzes also $P_n \in \mathcal{P}(X)$ mit $X_n = \im(P_n)$. Eine stärkere Eigenschaft als (2) des ersten Lemmas ist:

(P1) $\fa{x \in X} P_n x \xrightarrow{n \to \infty} x$

(P1) impliziert nach dem Satz von Banach-Steinhaus $C = \sup_{n \in \N} \norm{P_n} < \infty$.

Wir forden noch:

(P2) $\fa{m, n} P_n \circ P_m = P_{\min(n, m)}$

Man rechnet leicht nach, dass zu einer Folge $(P_n)_{n \in \N}$ mit (P1), (P2) mittels $Q_n \coloneqq P_n - P_{n-1}$ (wobei $P_1 = 0$) bzw. $P_n = \sum_{i=0}^n Q_i$ eine Folge $(Q_n)_{n \in \N}$ in $\mathcal{P}(X)$ mit

(Q1) $\fa{x \in X} \sum_{i=0}^n Q_i x \xrightarrow{n \to \infty} x$
(Q2) $\fa{m,n} Q_n \circ Q_m = \delta_{mn} Q_n$

Die Unterräume $E_n = \im(Q_n)$ erfüllen dann (3) aus dem ersten Lemma und (2) mit $X_n = E_0 \oplus ... \oplus E_n$.
\fi

% Ausgelassene Beispiele
\iffalse
\begin{bsp}
  \begin{itemize}
    \item Ist $X$ Hilbertraum und $X = \overline{\bigcup_{n \in \N} X_n}$ mit $\mathrm{dim} X_n < \infty$, $X_n \subset X_{n+1}$, so sei $P_n$ die orthogonale Projektion auf $X_n$ und mit $X_{n+1} = X_n \perp E_n$ sei $Q_n$ die orthogonale Projektion auf $E_n$. Ist speziell $X_n = \mathrm{span} \Set{ e_i }{ 0 \leq i \leq n }$ mit einer ONB $(e_i)_{i \in \N}$, so ist
    \[ Q_n x = (x | e_n) e_n \quad \text{und} \quad P_n x = \sum_{i=0}^n (x|e_i) e_i \]
    % 2. Beispiel: Definition der Dualbasis eines Banachraums mit Schauderbasis
    \item Zerlege $[0, 1]$ in Punkte $M_n = \Set{ x_{n,i} }{ i = 0, ...., m_n }$ mit $0 = x_{n,0} < ... < x_{n,m} = 1$ und $h_n = \max_{i} \abs{x_{n_i,i} - x_{n_i,i-1}} \xrightarrow{n \to \infty} 0$ sowie $\fa{n \in \N} M_n \subset M_{n+1}$. Sei $A_{n_i,i} = (x_{n_i,i}, x_{n_i,i})$, $h_{n_i,i} = x_{n_i,i} - x_{n_i,i-1}$. Dann ist der Raum der stückweise konstanten Funktionen bzgl. dieser Zerlegung auf Level $n$:
    \[ X_n = \Set{ \sum_{i=1}^m \alpha_i \chi_{A_{n_i,i}} }{ \alpha_i \in \K }, \mathrm{dim}(X_n) = m_n \]
    Für $f \in L^1(\left] 0, 1 \right[)$ definiere $P_n f = \sum_{i=1}^{m_n} (\frac{1}{n_{n_i,i}} \Int{A_{n_i,i}}{}{f(s)}{s}) \chi_{A_{n_i,i}}$.
    Es ist $\mathrm{im}(P_1) = X_n$ und für die Standardzerlegung $x_{n_i,i} = i 2^{-n}$ ist $E_n = \mathrm{span} \Set{ e_{n_i} }{  1 \leq i \leq 2^{n-1} }$ mit $e_0 = \chi_{\left] 0, 1 \right[}, e_{n,i} = \chi_{A_{n,2i-1}} - \chi_{A_{n,2i}}$.
  \end{itemize}
\end{bsp}
\fi


% Kapitel 8.
\section{Kompakte Operatoren}

\end{samepage}

% Im Skript schon in Kapitel 3 definiert
\begin{defn}
  Seien $X$ und $Y$ normierte $\K$-Vektorräume. Dann heißt
  \[ \mathcal{K}(X, Y) \coloneqq \Set{ T \in \LSO(X, Y) }{ \overline{T(B_1(0))} \text{ ist kompakt} } \]
  Menge der \emph{kompakten linearen Operatoren} von $X$ nach $Y$.
\end{defn}

% Wohin hiermit?
\begin{defn}
  Seien $X$, $Y$ Banachräume und $T \in \LSO(X, Y)$. Dann äquivalent:
  \begin{itemize}
    \miniitem{0.24 \linewidth}{$T \in \mathcal{K}(X, Y)$}
    \miniitem{0.35 \linewidth}{$\overline{T(B_1(0))}$ kompakt}
    \miniitem{0.38 \linewidth}{$T(B_1(0))$ präkompakt}
    \item Für alle beschränkten $M \subset X$ ist $T(M) \subset Y$ präkompakt.
    \item Für jede beschränkte Folge $(x_n)_{n \in \N}$ in $X$ besitzt $(T x_n)_{n \in \N}$ eine in $Y$ konvergente Teilfolge.
  \end{itemize}
\end{defn}

% Vorlesung vom 28.1.2014

\begin{defn}
  Seien $X$, $Y$ normierte Räume. Eine lineare Abb. $T : X \to Y$ heißt \emph{vollstetig}, falls für alle Folgen $(x_n)_{n \in \N}$ in $X$ und $x \in X$ gilt:
  \[
    x_n \convWeaklyWith{n} x \enspace \text{schwach in $X$}
    \quad \implies \quad
    T x_n \convWith{n} T x \enspace \text{stark in $Y$.}
  \]
\end{defn}

\begin{lem}
  Seien $X, Y$ Banachräume. Dann gilt:
  \begin{itemize}
    \item Für jede lineare Abbildung $T : X \to Y$ gilt: Wenn $T$ kompakt ist, dann ist $T$ vollstetig. Ist $X$ reflexiv, gilt auch die Rückrichtung. % Nach Eberleyn-Schmuljan
    \item $\mathcal{K}(X, Y)$ ist ein abgeschlossener Unterraum von $\LSO(X, Y)$.
    \item Ist $T \in \LSO(X, Y)$ mit $\dim \im(T) < \infty$, so ist $T \in \mathcal{K}(X, Y)$.
    \item Sei $Y$ ein Hilbertraum und $T \in \LSO(X, Y)$. Es gilt $T \in \mathcal{K}(X, Y)$ genau dann, wenn es eine Folge $(T_n)_{n \in \N}$ in $\LSO(X, Y)$ gibt mit $\fa{n \in \N} \dim \im(T_n) < \infty$, sodass $\norm{T-T_n} \convWith{n} 0$.
    \item Für $P \in \mathcal{P}(X)$ gilt: $P \in \mathcal{K}(X) \iff \dim \mathrm{im}(P) < \infty$.
  \end{itemize}
\end{lem}

% In Anwendungen, in denen Fixpunktsätze für kompakte Operatoren benutzt werden sollen, ist noch folgende Aussage sehr nützlich:

\begin{samepage}

\begin{lem}
  Für $T_1 \in \LSO(X, Y)$ und $T_2 \in \LSO(Y, Z)$ gilt:
  \[
    \text{$T_1$ oder $T_2$ kompakt}
    \enspace \implies \enspace
    \text{$T_2 T_1$ kompakt}
  \]
\end{lem}

\iffalse
\begin{bspe}
  \begin{itemize}
    \item Sei $\Omega \opn \R^n$, beschränkt mit $\mathcal{C}^{0,1}$-Rand. Seien $m_1 > m_2 \in \N$ und $1 \leq p_1, p_2 < \infty$ sowie $m_1 - \tfrac{n}{p_1} > m_2 - \tfrac{n}{p_2}$. Dann ist die Einbettung $\Id : W^{m_1,p_1}(\Omega) \to W^{m_2,p_2}(\Omega)$ stetig und kompakt.
    \item Viele Integraloperatoren, vgl. z.\,B. ÜA24
  \end{itemize}
\end{bspe}
\fi

% Kapitel 9.
\section{Spektraltheorie}

\end{samepage}

% Seien im Folgenden, sofern nicht anders spezifiziert, $X$ ein Banachraum über $\C$ und $T \in \mathcal{L}(X)$

\begin{defn}
  Sei $T \in \LSO(X)$. Dann ist
  \begin{itemize}
    \item die \emph{Resolventenmenge} von $T$ definiert als
    \[ \rho(T) \coloneqq \Set{ \lambda \in \C }{ \ker (\lambda \Id - T) = \{ 0 \} \text{ und } \im (\lambda \Id - T) = X }, \]
    \item das \emph{Spektrum} von $T$ gleich $\sigma(T) \coloneqq \C \setminus \rho(T)$
  \end{itemize}
  Das Spektrum wird noch weiter zerlegt in das
  \begin{itemize}
    \item \emph{Punktspektrum} $\sigma_p(T) \coloneqq \Set{ \lambda \in \sigma(T) }{ \ker(\lambda \Id - T) \not= \{ 0 \} }$,
    \item \emph{kontinuierliche Spektrum}
    \begin{align*}
      \sigma_c(T) \coloneqq \{ \lambda \in \sigma(T) \mid\, &\ker(\lambda \Id - T) = \{ 0 \} \text{ und } \im (\lambda \Id - T) \not= X,\\[-2pt]
      &\text{ aber } \overline{\im (\lambda \Id - T) = X } \},
    \end{align*}
    \item \emph{Restspektrum} (Residualspektrum)
    \[ \sigma_r(T) \coloneqq \Set{ \lambda \in \sigma(T) }{ \ker(\lambda \Id - T) = \{ 0 \} \text{ und } \overline{\im(\lambda \Id - T)} \not= X }. \]
  \end{itemize}
\end{defn}

\begin{bem}
  Sei $T \in \LSO(X)$. Offenbar ist $\lambda \in \rho(T)$ genau dann, $\lambda \Id - T : X \to X$ bijektiv ist. Nach dem Satz von der inversen Abbildung ist dies äquivalent zur Existenz von
  \[ R(\lambda, T) \coloneqq (\lambda \Id - T)^{-1} \in \mathcal{L}(X), \]
  der sogenannten \emph{Resolvente} von $T$ in $\lambda$. Als Funktion von $\lambda$ heißt $R$ auch \emph{Resolventenfunktion}. Weiterhin ist $\lambda \in \sigma_p(T)$ offenbar äquivalent zu $\ex{x \not= 0} Tx = \lambda x$. Dann heißt $\lambda$ \emph{Eigenwert} und $x$ \emph{Eigenvektor} (oder Eigenfunktion). Der Unterraum $\ker(\Id \lambda - T)$ ist der \emph{Eigenraum} von $T$ zum Eigenwert $\lambda$. Er ist $T$-invariant.
\end{bem}

\begin{satz}
  Die Resolventenmenge $\rho(T)$ ist offen und $\lambda \mapsto R(\lambda, T)$ ist eine komplex-analytische Abbildung von $\rho(T)$ nach $\mathcal{L}(X)$. Es gilt
  \[ \fa{\lambda \in \rho(T)} \norm{R(\lambda, T)}^{-1} \leq \dist(\lambda, \rho(T)). \]
\end{satz}

\begin{satz}
  Das Spektrum $\sigma(T)$ ist kompakt und nichtleer (für $X {\not=} 0$) mit
  \[ \sup_{\lambda \in \sigma(T)} = \lim_{m \to \infty} \norm{T^m}^{\tfrac{1}{m}} \leq \norm{T}. \tag*{(\emph{Spektralradius})} \]
\end{satz}

% Vorlesung vom 30.1.2014

\begin{lem}
  \begin{itemize}
    \item Ist $\dim X < \infty$, so ist $\sigma(T) = \sigma_p(T)$.
    \item Ist $\dim X = \infty$ und $T \in K(X)$, so ist $0 \in \sigma(T)$.
  \end{itemize}
\end{lem}

\begin{bem}
  Im Punkt 2 ist i.\,A. $0$ kein Eigenwert, also $0 \not\in \sigma_p(T)$.
\end{bem}

% Im Folgenden wollen wir das Punktspektrum genauer untersuchen.
% Betrachte dazu für $T \in \mathcal{L}(X)$ und $y \in X$ das Problem:
% Finde $\lambda \in \C$ und $x \in X$ mit $Tx - \lambda x = y$.
% Ist $\lambda \in \rho(T)$, so existiert eine eindeutig bestimmte Lösung $x$ dieser Gleichung.
% Ist $\lambda \in \sigma_p(T)$, so ist die Lösung, falls sie existiert, nicht eindeutig bestimmt,
% damit $A_{\lambda} = \lambda \Id - T$ die Addition eines Elements aus $\ker(A_{\lambda})$ eine weitere Lösung ergibt. Auf der anderen Seite muss $\lambda \in \mathrm{im}(A_\lambda)$ sein, damit Lösung existieren kann.

% Eine wichtige Klasse von Operatoren sind in diesem Zusammenhang:

\begin{defn}
  Eine Abb. $A \in \LSO(X, Y)$ heißt \emph{Fredholm-Operator}, falls
  \begin{itemize}
    \miniitem{0.29 \linewidth}{$\dim \ker(A) < \infty$}
    \miniitem{0.41 \linewidth}{$\im(A)$ ist abgeschlossen}
    \miniitem{0.27 \linewidth}{$\codim \im(A) < \infty$}
  \end{itemize}
  Der \emph{Index} eines Fredholm-Operators ist definiert als
  \[ \mathrm{ind}(A) \coloneqq \dim \ker (A) - \codim \im(A). \]
\end{defn}

\iffalse

\begin{bsp}
  Sei $H$ ein Hilbertraum mit ONB $(e_n)_{n \in \N}$. Setze $S e_n \coloneqq e_{n+1}$ für $n \in \N$. Dann ist $S \in \LSO(H, H)$ ein Fredholm-Operator mit $\ind(S) = -1$.
\end{bsp}

% ********
% * TODO *
% ********

\begin{bspe}
  \begin{itemize}
    \item Sei $X = W^{1,2}(\Omega)$, $Y = (W^{1,2}(\Omega))'$. Dann ist $A : W^{1,2}(\Omega) \to (W^{1,2}(\Omega))'$ definiert durch $\langle Au, v \rangle \coloneqq \Int{\Omega}{}{\sum_{i,j} \partial_i v \cdot a_{ij} \partial_j u}{x}$ für $u , v \in W^{1,2}(\Omega)$, der der schwache elliptische Differentialoperatoren mit Neumann-Randbedingungen. Aus Kapitel 4.1 und 6 wissen wir: Der Kern $\ker(A)$ besteht aus den konstanten Funktionen, also ist $\dim \ker(A) = 1$. Das Bild von $A$ ist $\im(A) = \Set{ F \in Y }{ \langle F, 1 \rangle_{W^{1,2}(\Omega)} = 0 }$, also abgeschlossen mit $\codim \im(A) = 1$. Es ist $Y = \im(A) \oplus \mathrm{span} \{ F_0 \}$, wenn $\langle F_0 , v \rangle = \Int{\Omega}{}{v}{x}$. Also ist $A$ ein Fredholm-Operator mit Index $0$.
    \item Für das homogene Dirichlet-Problem ist der Operator $A : W_0^{1,2}(\Omega) \to (W_0^{1,2}(\Omega))'$ ein Isomorphismus.
  \end{itemize}
\end{bspe}

\fi

% Eine wichtige Klasse von Fredhom-Operatoren sind kompakte Störungen von $\Id$. Es gilt (ohne Beweis):

\begin{satz}
  Sei $T \in \mathcal{K}(X)$. Dann gilt für $A = \Id - T$:
  \begin{itemize}
    \begin{multicols}{2}
      \item $\dim \ker T < \infty$
      \item $\im(A)$ ist abgeschlossen
      \item $\ker A = \{ 0 \} \implies \im(A) = X$
      \item $\codim \im(A) = \dim \ker(A)$
    \end{multicols}
  \end{itemize}
  Insbesondere ist $A$ also ein Fredholm-Operator mit Index $0$.
\end{satz}

% Ausgelassen: Namen: Riesz-Schauder
\begin{satz}[Spektralsatz für kompakte Operatoren]
  Für $T \in K(X)$ gilt:
  \begin{itemize}
    \item Die Menge $\sigma(T) \setminus \{ 0 \}$ besteht aus höchstens abzählbar vielen Elementen mit $0$ als einzig möglichem Häufungspunkt. Es gilt:
    \[
      \abs{\sigma(T)} = \infty
      \enspace \implies \enspace
      \overline{\sigma(T)} = \sigma_p(T) \cup \{ 0 \}
    \]
    \item Für $\lambda \in \sigma(T) \setminus \{ 0 \}$ gilt für die \emph{Ordnung} $n_\lambda$ von $\lambda$
    \[ 1 \leq n_{\lambda} \coloneqq \max \Set{n \in \N_{+}}{ \ker(\lambda \Id - T)^{n-1} \not= \ker(\lambda \Id - T)^n } < \infty. \]
    Die Zahl $\dim (\ker (\lambda \Id - T))$ heißt \emph{Vielfachheit} von $\lambda$.
    \item Für $\lambda \in \sigma(T) \setminus \{ 0 \}$ gilt $X = \ker (\lambda \Id - T)^{n_\lambda} \oplus \im (\lambda \Id - T)^{n_\lambda}$.
  \end{itemize}
  Beide Unterräume sind abgeschlossen und $T$-invariant und der \emph{charakteristische Unterraum} $\ker (\lambda \Id - T)^{n_\lambda}$ ist endlich-dim.
  \begin{itemize}
    \item  Für $\lambda \in \sigma(T) \setminus \{ 0 \}$ ist $\sigma(T|_{\im (\lambda \Id - T)^{n_\lambda}}) = \sigma(T) \setminus \{ \lambda \}$
    \item Ist $E_\lambda$ für $\lambda \in \sigma(T) \setminus \{ 0 \}$ die Projektion auf $\ker (\lambda \Id - T)^{n_\lambda}$ gemäß der Zerlegung in Punkt 3, so gilt $E_\lambda E_\mu = \delta_{\lambda \mu} E_\lambda$ für $\lambda, \mu \in \sigma(T) \setminus \{ 0 \}$
  \end{itemize}
\end{satz}

% Vorlesung vom 4.2.2014

% Erste Stunde der Vorlesung: Wiederholung, Beweis des Spektralsatzes

\begin{kor}
  Ist $T \in \mathcal{K}$ und $\lambda \in \sigma(T) \setminus \{ 0 \}$, so hat die Resolventenfunktion $\mu \mapsto R(\mu, T)$ in $\lambda$ einen isolierten \emph{Pol} der Ordnung $n_{\lambda}$, d.\,h.
  \[ \mu \mapsto (\mu - \lambda)^{n_\lambda} R_{\lambda} R_{\lambda}(\mu, T) \]
  kann im Punkt $\lambda$ komplex analytisch fortgesetzt werden. Der Wert in $\lambda$ ist von Null verschieden.
\end{kor}

% Beweisidee: Nutze die Zerlegung (3) des Spektralsatzes $X = \ker((\lambda \Id - T)^{n_\lambda}) \oplus \im((\lambda \Id - T)^{n_\lambda})$.
% Schreibe $R(\lambda+\mu, T) = R(\lambda + \mu, T|_{\im(E_\lambda)}) E_\lambda + R(\lambda + \mu, T|_{\ker(E_\lambda)}) (\Id - E_{\lambda})$ (komplex analytisch nach dem ersten Satz)

\begin{kor}[Fredholmsche Alternative]
  Ist $T \in \mathcal{K}(X)$ und $\lambda \not= 0$, so gilt: Entweder ist die Gleichung $Tx - \lambda x = y$ eindeutig lösbar für jedes $y \in X$ oder die Gleichung $Tx - \lambda x = 0$ hat nichttriviale Lsgen.
\end{kor}

\begin{kor}
  Sei $X$ ein endlich-dimensionaler VR über $\C$ und $T : X {\to} X$ linear. Dann gibt es paarwise verschiedene $\lambda_1, ..., \lambda_m \in \C$, wobei $1 \leq m \leq \dim X$, sodass $\sigma(T) = \sigma_p(T) = \{ \lambda_1, ..., \lambda_m \}$ und Ordnungen $n_{\lambda_j}$ mit $X = \ker \left( (\lambda \Id {-} T)^{n_{\lambda_1}} \right) \oplus ... \oplus \ker \left( (\lambda \Id {-} T)^{n_{\lambda_1}} \right)$.
\end{kor}

% Vorlesung vom 6.2.2014

Sei $X$ ein Hilbertraum über $\C$ und $T \in \mathcal{L}(X)$ \emph{normal}, d.\,h. $T^* T - T T^* = 0$, wobei $T^* \coloneqq R_{X}^{-1} T' R_X$ mit $R_X$ die Isometrie aus dem Rieszschen Darstellungssatz bzw. $\fa{x, y \in X} \scp{x}{T^* y} = \scp{Tx}{y}$, lassen sich die Aussagen noch konkretisieren. Es gilt:

\begin{satz}[Spektralsatz für kompakte normale Operatoren]
  Sei $X$ ein Hilbertraum über $\C$ und $0 \not= T \in \mathcal{K}(X)$ normal. Dann gilt:
  \begin{itemize}
    \item Es existiert eine ONS $(e_k)_{k \in N}$ in $X$ und Folge $(\lambda_k)_{k \in N}$ in $\C$, wobei $N \subset \N$, sodass $\lambda_k \not= 0$ und $T_{e_k} = \lambda_k e_k$ für $k \in N$ sowie $\sigma(T) \setminus \{ 0 \} = \Set{ \lambda_k }{ k \in N }$. Falls $\abs{N} = \infty$, gilt $\lambda_k \convWith{k} 0$.
    \item Für die Ordnungen gilt $n_{\lambda_k} = 1$ für alle $k$.
    \item $X = \ker(T) \perp \mathrm{span} \Set{ e_k }{ k \in N }$
    \item $\fa{x \in X} Tx = \sum_{k \in N} \lambda_k \scp{x}{e_k} e_k$
  \end{itemize}
\end{satz}

\begin{bem}
  Die Werte $\lambda_k$ können für verschiedene $k$ gleich sein.
\end{bem}

\begin{lem}
  Seien $X$ ein Hilbertraum über $\C$ und $T \in \LSO(x)$ normal. Dann gilt:
  \begin{itemize}
    \item  Ist $X \not= 0$, gilt $\sup \sigma(T) \lambda = \norm{T}$
    \item Ist $T$ \emph{selbstadjungiert}, d.\,h. $T^* = T$, so ist $\sigma_p(T) \subset \left[ -\norm{T}, \norm{T} \right]$ und, falls $T$ auch kompakt ist $\norm{T}$ oder $-\norm{T}$ ein Eigenwert.
    \item Ist $T$ selbstadjungiert und \emph{positiv semidefinit}, d.\,h. $\fa{x \in X} \scp{Tx}{x} \geq 0$, so gilt $\sigma_p(T) \subset \left[ 0, \norm{T} \right]$ und falls $T$ auch kompakt ist, sit $\norm{T}$ Eigenwert.
  \end{itemize}
\end{lem}

\iffalse
\begin{bsp}[Eigenwertproblem für den Laplace-Operator]
  Sei $\Omega \subset \R^n$ offen und beschränkt. Betrachte Funktion $v : \overline{\Omega} \times \left] 0, \infty \right[ \to \R$, die die lineare Wellengleichung $\partial_t^2 v(x, t) = \triangle v = 0$ in $\Omega \times \left] 0, \infty \right[$ mit der Randbedingung $v = 0$ auf $\partial \Omega \times \left] 0, \infty \right[$. Im Fall $n=2$ beschreibt $v(x, t)$ bspw. Auslenkung einer dünnen, eingespannten Membran.

  Ansatz: $v(x,t) = u(x) w(t)$. Es folgt: $w^{\|} u = w \triangle u$. Dies ist (für $v \not= 0$) nur möglich, falls es ein $\lambda \in \R$ gibt mit (1) $\triangle u + \lambda u = 0$ in $\Omega$ und (2) $w'' + \lambda w = 0$ in $\left] 0, \infty \right[$. Multiplizieren wir (1) mit $u$ erhalten wir (formal)
  \[ \lambda \norm{u}_{L^2}^2 = - \Int{\Omega}{}{\triangle u \cdot u}{x} = - \Int{\partial \Omega}{}{u \cdot \triangle u}{x} + \Int{\Omega}{}{\abs{\nabla u}^2}{x} \geq 0, \]
  also $\lambda > 0$.

  Gleichung (2) hat für jedes $\lambda > 0$ die allgemeine Lösung
  \[ w(t) = a_1 \cos(\mu t) + a_2 \sin(\mu t) \quad \text{mit $a_i \in \R$, $\mu = \sqrt{\lambda}$}. \]
  Gleichung (1) ist ein Eigenwertproblem. Dessen schwache Form lautet:
  Finde $\lambda \in \R$, $n \in W_0^{1,2}(\Omega, \R)$ mit $n \not= 0$:
  \[ \fa{\zeta \in W_0^{1,2}(\Omega, \R)} \Int{\Omega}{}{\nabla \zeta \cdot \nabla u - \lambda u v}{x} = 0. \]
  Der Spektralsatz, angewendet auf $T = J A^{-1}$, wobei $A^{-1}$ der Lösungsoperator zu:
  Für $f \in \LSO(\Omega)$ finde $u_f \in W_0^{1,2}(\Omega) \,:\, \fa{\zeta \in W_0^{1,2}(\Omega)} \Int{\Omega}{}{\nabla \zeta \cdot \nabla u - \zeta f}{x} = 0$.

  Aus Kapitel 4 wissen wirr: $A^{-1} : L^2(\Omega) \to W_0^{1,2}(\Omega)$ und $J : W_0^{1,2}(\Omega) \to L^2(\Omega)$ die kompakte Einbettung ($x \mapsto x$). Es folgt leicht, dass $T$ kompakt, injektiv, selbstadjungiert und positiv semidefinit. Mit Hilfre des Spektralsatzes folgt, dass es paarweise verschiedene Eigenwerte $\lambda_k > 0$ für $\k \in \N$ und endlich-dimensionale Unterräume $E_k \subset W_0^{1,2}(\Omega, \R)$ (die Eigenräume) mit $\lambda_k \convWith{k} \infty$ und $L^2(\Omega, \R) = \overline{\perp_{k \in \N} E_k}$ und $(\lambda, u) \in \R \times W_0^{1,2}(\Omega, \R)$ mit $u \not= 0$ ist schwache Lösung von (*) genau dann, wenn $\lambda = \lambda_k$, $u \in E_k$ für ein $k \in \N$.
\end{bsp}
\fi

% In Klausur zugelassen: Einseitig beschriebene Din-A4-Seite

\end{document}