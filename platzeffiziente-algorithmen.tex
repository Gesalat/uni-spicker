\documentclass{cheat-sheet}

\pdfinfo{
  /Title (Zusammenfassung Platzeffiziente Algorithmen)
  /Author (Tim Baumann)
}

\usepackage{algorithm}
\usepackage[noend]{algpseudocode}

% Kleinere Klammern
\delimiterfactor=701

\newcommand{\ceil}[1]{\lceil #1 \rceil} % Aufrunden
\newcommand{\floor}[1]{\lfloor #1 \rfloor} % Abrunden
\renewcommand{\O}{\mathcal{O}} % Landau-O
\newcommand{\Push}{\Leftarrow} % etwas auf den Stack pushen
\newcommand{\Pop}{\Leftarrow} % etwas vom Stack nehmen

\begin{document}

\maketitle{Zusammenfassung Platzeffiziente Alg.}

% Vorlesung vom 14.10.2015

% 1. Introduction

\begin{ziel}
  Algorithmen entwerfen, die wenig Speicherplatz und Speicherzugriffe benötigen, aber trotzdem schnell sind.
\end{ziel}

% 2. Reachability

\begin{prob}[Erreichbarkeit]
  Gegeben sei ein gerichteter oder ungerichteter Graph, ein Startknoten und ein Zielknoten darin. \\
  Frage: Ist der Zielknoten vom Startknoten erreichbar?
\end{prob}

\begin{alg}
  Algorithmen, mit denen man Problem lösen kann, sind Breiten- und Tiefensuche.
\end{alg}

% 2.1. Depth-First-Search and Recursion

\begin{lem}
  Es sei ein Graph mit $n$ Knoten und $m$ Kanten gegeben.
  Tiefensuche benötigt $\Theta(n+m)$ Zeit und $\Theta(n \log n)$ Speicherplatz.
\end{lem}

% 2.2. Savitch's Algorithm

\begin{alg}[\emph{Savitch}]\mbox{}\\[4pt]
%\begin{algorithm}
  %\caption{Savitch's Algorithmus}
  \begin{algorithmic}[1]
    \Function{sReachable}{$u, v, k$}
      \If{$u = v$} \Return true \EndIf
      \If{$k = 0$} \Return false \EndIf
      \If{$(u,v) \in E$} \Return true \EndIf
      \If{$k=1$} \Return false \EndIf
      \For{$x \in V$}
        \If{\Call{sReachable}{$u, x, \floor{\tfrac{k}{2}}$} $\wedge$ \Call{sReachable}{$x, v, \ceil{\tfrac{k}{2}}$}}
          \State \Return true
        \EndIf
      \EndFor
      \State \Return false
    \EndFunction
    \Statex
    \State \Return sReachable(s,t,n-1)
  \end{algorithmic}
%\end{algorithm}
\end{alg}

% Thm 2.1
\begin{lem}
  Savitch's Algorithmus löst das Erreichbarkeits-Problem in $\O((\log n)^2)$ Speicherplatz.
\end{lem}

\begin{bem}
  Die Laufzeit von Savitch's Alg. ist allerdings sehr schlecht, im schlechtesten Fall $O(n \cdot \log n)$.
\end{bem}

% Vorlesung vom 21.10.2015

% §3. The Representation of Graphs

% 3.1. Adjacency Matrices and Adjacency Lists
% und 3.2. Graph Representation with Little Memory

\begin{bem}
  Eine Darstellung eines Graphen als Adjazenzmatrix benötigt $O(n^2)$, eine Darstellung als Adjazenzliste/-array $O(m \cdot \log n)$ Bits.
  Manchmal ist es nützlich, zusätzlich Rückwärtskanten oder Aus- und Ingrad von Knoten zu speichern, um diese Informationen nicht mehrmals berechnen zu müssen. Bei bestimmten Algorithmen werden sie auch als gegeben angenommen.
  % Ausgelassen: 'mates', Multigraphen
\end{bem}

% 3.3 Common Input Requirements

\begin{konv}
  Wir werden folgende Graphfunktionen benutzen:
  
  \begin{tabular}{r l}
    Funktion & Ergebnis \\[3pt] \hline
    $\text{adjfirst} : V \to P$ & ersten Eintrag in der Adjazenzliste \\
    $\text{adjhead} : P \to V$ & Knoten zum Eintrag in der Adjazenzliste \\
    $\text{adjnext} : P \to P$ & nächsten Eintrag in der Adjazenzliste \\
    $\text{deg} : V \to \N$ & Ausgrad eines Knoten \\
    $\text{head} : A \to V$ & den $k$-ten Nachbar eines Knoten \\
    $\text{tail} : B \to V$ & den $k$-ten In-Nachbar eines Knoten \\
    $\text{mate} : A \to A$ & den "`Mate"' einer Kante (bei unger. Graphen)
  \end{tabular}
  \begin{align*}
    \text{wobei} \quad
    A &\coloneqq \Set{(v, k) \in V \times \N}{1 \leq k \leq \text{deg}(v)} \\
    B &\coloneqq \Set{(v, k) \in V \times \N}{1 \leq k \leq \text{indeg}(v)}
  \end{align*}
\end{konv}

% §4. Depth-First search


\begin{alg}
  Bei einer Tiefensuche in einem Graphen wird am meisten Platz für den Laufzeitstack verbraucht. Um diesen Platz zu optimieren, ist es geschickt, zunächst den Algorithmus mit explizitem Keller aufzuschreiben:
  \begin{algorithmic}[1]
    \Function{process}{$u$}
      \State $S \Push (u, \Call{adjfirst}{u})$
      \While{$S \neq \emptyset$}
        \State $(u, p) \Pop S$
        \If{$color[u] = white$}
          \State $color[u] \coloneqq gray$
          \State $\Call{preprocess}{u}$
        \EndIf
        \If{$p \neq null$}
          \State $S \Push (u, \Call{adjnext}{p})$
          \State $v \coloneqq \Call{adjhead}{u, p}$
          \State $\Call{preexplore}{u, v, color[v]}$
          \If{$color[v] = white$}
            \State $S \Push (v, \Call{adjfirst}{v})$
          \Else
            \State $\Call{postexplore}{u, v}$
          \EndIf
        \Else
          \State $\Call{postprocess}{u}$
          \If{$S \neq \emptyset$}
            \State $(w, \blank) \coloneqq \Call{peek}{S}$
            \State $\Call{postexplore}{w, u}$
          \EndIf
          \State $color[u] \coloneqq black$
        \EndIf
      \EndWhile
    \EndFunction
  \end{algorithmic}
\end{alg}

\end{document}