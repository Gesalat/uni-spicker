\documentclass{cheat-sheet}

\pdfinfo{
  /Title (Zusammenfassung Algebraische Topologie)
  /Author (Tim Baumann)
}

\usepackage{tikz}
\usetikzlibrary{matrix,shapes,arrows,positioning}

\newcommand{\Tau}{\mathcal{T}} % Großes Tau
%\newcommand{\inte}{\mathop{\mathrm{int}}} % Inneres (interior)
%\newcommand{\conv}{\mathop{\mathrm{conv}}} % Konvexe Hülle
%\newcommand{\rel}{\text{ rel }} % = relativ (Homotopie)
%\newcommand{\Ob}{\mathrm{Ob}} % Objects (of a category)
%\newcommand{\Hom}{\mathrm{Hom}} % Homomorphisms
%\newcommand{\Aut}{\mathrm{Aut}} % Automorphisms
%\newcommand{\Mor}{\mathrm{Mor}} % Morphisms
%\newcommand{\dom}{\mathrm{dom}} % Domain
%\newcommand{\codom}{\mathrm{codom}} % Codomain
%\newcommand{\op}{\text{op}} % Duale Kategorie
%\newcommand{\Deck}{\mathrm{Deck}} % Gruppe der Decktransformationen

\newcommand{\angles}[1]{\langle #1 \rangle}
\newcommand{\Simpl}{\mathcal{S}} % Simplizialkomplex
\newcommand{\Real}[1]{\abs{#1}} % geometrische Realisierung
\newcommand{\CC}[1]{{#1}_{\bullet}} % Kettenkomplex (chain complex)
\newcommand{\Top}{\mathbf{Top}} % Kategorie der topologischen Räume
\newcommand{\AbGrp}{\mathbf{AbGrp}} % Kategorie der abelschen Gruppen
\newcommand{\RH}{\tilde{H}} % Reduzierte Homologie
\newcommand{\inte}{\mathop{\mathrm{int}}} % Inneres (interior)
\newcommand{\clos}[1]{\overline{#1}} % Abschluss

% Kleinere Klammern
\delimiterfactor=701


\begin{document}

\maketitle{Zusammenfassung Algebr. Topologie}

% Vorlesung vom 8.10.2014

% TODO: Definition $C_n$

%\[ \partial \angles{p_0, ..., p_n} \coloneqq \sum_{i=0}^n (-1)^{i} \angles{p_0, ..., \hat{p_i}, ..., p_n} \]
%\[ \partial_n : C_n \to C_{n-1} \]
%\[ Z_n(S) \coloneqq \ker \partial_n \subset C_n(S) \]
%\[ B_n(S) \coloneqq \im \partial_{n+1} \subset C_n(S) \]


% §2. Simpliziale und singuläre Homologie

\begin{defn}
  Ein \emph{affines $n$-Simplex} ist die konvexe Hülle von $n+1$ affin unabhängigen Punkten $p_0, ..., p_{n+1} \in \R^N$. Die konvexe Hülle von einer Teilmenge dieser Eckpunkte wird \emph{Seite} genannt.\\
  Das \emph{Standard-$n$-Simplex} $\Delta_n$ ist das von den $n+1$ Standard-Basisvektoren im $\R^{n+1}$ aufgespannte Simplex.
\end{defn}

\begin{defn}
  Ein (endlicher) \emph{geometrischer Simplizialkomplex} ist eine (endliche) Menge $\mathcal{S}$ endlich vieler affiner Simplizes im $\R^N$, sodass:
  \begin{itemize}
    \item Ist $K \in \Simpl$ und $T \subset K$ eine Seite von $K$, dann ist auch $T \in \Simpl$.
    \item Für alle $K_1, K_2 \in \Simpl$ ist $K_1 \cap K_2$ entweder eine Seite von $K_1$ und $K_2$ oder leer.
  \end{itemize}
\end{defn}

\begin{defn}
  Jeder Simplizialkomplex $\Simpl$ ist eine \emph{Triangulierung} seines \emph{Polyeders}, dem topologischen Raum $\Real{S} \coloneqq \bigcup_{K \in \Simpl} K$.
\end{defn}

\begin{defn}
  Ein geometrischer Simplizialkomplex mit einer Totalordnung auf der Menge der Eckpunkte heißt \emph{geordnet}.
\end{defn}

\begin{nota}
  Ein $n$-Simplex mit Eckpunkten $v_0, ..., v_n$ in einem geordneten geom. Simplizialkomplex wird mit $\angles{v_0, ..., v_n}$ bezeichnet, falls $v_0 < v_1 < ... < v_n$.
\end{nota}

\begin{nota}
  $\Simpl_n \coloneqq \Set{\sigma \in \Simpl}{\text{$\sigma$ ist geordneter $n$-Simplex}}$
\end{nota}

\begin{defn}
  Eine \emph{simpliziale $n$-Kette} in einem geordneten geom. Simplizialkomplex ist eine endliche formale Linearkombination
  \[ \sum_{\sigma \in \Simpl_n} \lambda_\sigma \cdot \sigma, \]
  wobei $\lambda_\sigma \in \Z$. Die Menge solcher Linearkombinationen ist $C_n(\Simpl)$.\\
  Sie ist die freie abelsche Gruppen über der Menge der Simplizes.
\end{defn}

\begin{bem}
  $C_n(\Simpl)$ ist eine Gruppe.
\end{bem}

\begin{defn}
  Der Rand eines orientierten $n$-Simplex $\angles{v_0, ..., v_n} \in \Simpl$ ist
  \[ \delta \angles{v_0, ..., v_n} \coloneqq \sum_{i=0}^n (-1)^i \angles{v_0, ..., \hat{v_i}, ..., v_n}. \]
  Durch lineare Fortsetzung erhalten wir einen Gruppenhomo $\partial_n : C_n(\Simpl) \to C_{n-1}(\Simpl)$.
\end{defn}

% Vorlesung vom 8.10.2014

\begin{defn}
  Ein \emph{Kettenkomplex} $\CC{C}$ ist eine Folge $(C_n)_{n \in \N}$ und Gruppenhomomorphismen $\partial_n : C_n \to C_{n-1}$ mit der Eigenschaft $\partial_{n-1} \circ \partial_n = 0$.
\end{defn}

\begin{defn}
  Sei $\CC{C}$ ein Kettenkomplex.
  \begin{itemize}
    \item $Z_n(\CC{C}) \coloneqq \ker \partial_n \subset C_n(\CC{C})$ heißt Gruppe der \emph{$n$-Zykel},
    \item $B_n(\CC{C}) \coloneqq \im \partial_{n+1} \subset Z_n(\CC{C})$ heißt Gruppe der \emph{$n$-Ränder},
    \item $H_n(\CC{C}) \coloneqq Z_n(\CC{C}) / B_n(\CC{C})$ heißt \emph{$n$-te Homologiegruppe}.
  \end{itemize}
\end{defn}

% 2.1
\begin{prop}
  Für $n \geq 1$ gilt $\partial_{n-1} \circ \partial_n = 0$. Die simplizialen $n$-Ketten bilden also einen Kettenkomplex.
\end{prop}

\begin{defn}
  Ein \emph{singuläres $n$-Simplex} in einem topologischen Raum $X$ ist eine stetige Abbildung $\sigma : \Delta^n \to X$. Wir bezeichnen mit $\Delta_n(X)$ die Menge der singulären $n$-Simplizes in $X$ und mit $C_n(X)$ die freie abelsche Gruppe über $\Delta_n(X)$. Wir definieren
  \[
    \partial_n : C_n(X) \to C_{n-1}(X), \quad
    \sigma \mapsto \sum_{i=0}^n (-1)^i \sigma_{\angles{e_o,...,\hat{e_i},...,e_n}}.
  \]
  Analog zu oben gilt $\partial_{n-1} \circ \partial_n = 0$, man erhält also einen Komplex $\CC{C}(X)$ der singulären Ketten in $X$. Die Homologie dieses Komplexes bezeichnet man mit $H_n(X)$.
\end{defn}

\begin{defn}
  Eine \emph{Kettenabbildung} zwischen Kettenkomplexen $\CC{C}$ und $\CC{D}$ ist eine Familie $(f_n : C_n \to D_n)_{n \in \N}$ von Gruppenhomomor- phismen, welche mit dem Differential verträglich sind, d.\,h.
  \[ \fa{n \in \N} \partial_n^{(D)} \circ f_n = f_{n-1} \circ \partial_n^{(C)}. \]
  Aus einer solchen Abbildung erhält man wiederum eine Abbildung $H_n(f) : H_n(\CC{C}) \to H_n(\CC{C})$ für alle $n \in \N$. Somit definiert $H_n$ einen Funktor von der Kategorie der Kettenkomplexe in die Kategorie der abelschen Gruppen.
\end{defn}

% Ausgelassen: Kettenisomorphismus

\begin{defn}
  Für eine Abbildung $f : X \to Y$ von topologischen Räumen erhalten wir eine Abbildung $f_* : \CC{C}(X) \to \CC{C}(Y)$ definiert durch $f_*(\sigma) \coloneqq f \circ \sigma$ für ein $n$-Simplex $\sigma : \Delta_n \to X$. Die Zuordnung $f \mapsto f_*$ erfüllt die Funktiorialitätsaxiome. Somit definiert $H_n$ für alle $n \in \N$ einen Funktor $\Top \to \AbGrp$.
\end{defn}

\begin{kor}
  Homöomorphe Räume haben isomorphe singuläre Homologiegruppen.
\end{kor}

% Vorlesung vom 23.10.2014

\begin{prop}
  Sei $\pi_0(X)$ die Menge der Wegekomponenten von $X$. Die Inklusionen $A \hookrightarrow X$ (für $A \in \pi_0(X)$) induzieren einen Isomorphismus
  \[ \bigoplus_{A \in \pi_0(X)} H_*(A) \cong H_*(X). \]
\end{prop}

\begin{prop}
  Sei $X \not= \emptyset$ wegzusammenhängend. Dann ist $H_0(X) \cong \Z$.
\end{prop}

\begin{defn}
  Eine \emph{Kettenhomotopie} zw. Kettenabb. $\phi_*, \psi_* : \CC{C} \to \CC{D}$ ist eine Folge von Homomorphismen $P_n : C_n \to D_{n+1}$ mit
  \[ \fa{n \in \N} \partial_{n+1} \circ P_n + P_{n-1} \circ \partial_n = \phi_n - \psi_n. \]
\end{defn}

\begin{prop}
  Seien $\phi_*, \psi_* : \CC{C} \to \CC{D}$ kettenhomotop. Dann gilt
  \[ H_*(\phi_*) = H_*(\psi_*) : H(\CC{C}) \to H(\CC{D}). \]
\end{prop}

\begin{satz}
  Seien $f, g : X \to Y$ homotope Abbildungen. Dann sind $f_*, g_* : \CC{X} \to \CC{Y}$ kettenhomotop.
\end{satz}

\begin{kor}
  \begin{itemize}
    \item Seien $f, g : X \to Y$ homotope Abbildungen. Dann gilt
    \[ f_* = g_* : H_*(X) \to H_*(Y). \]
    \item Homotopieäquivalente Räume haben isomorphe Homologiegruppen.
  \end{itemize}
\end{kor}

% Vorlesung vom 15.10.2014

% §3. Relative Homologie und Ausschneidung

\begin{defn}
  Ein \emph{Unterkomplex} $\CC{D}$ von $\CC{C}$ ist eine Folge von Untergruppen $D_n \subset C_n$, sodass gilt: $\partial D_n \subset D_{n-1}$ für alle $n \geq 1$.
\end{defn}

\begin{defn}
  Ist $\CC{D} \subset \CC{C}$ ein Unterkomplex, so ist der \emph{Quotientenkomplex} $\CC{C}/\CC{D}$ definiert durch
  \[
    (\CC{C}/\CC{D})_n \coloneqq C_n/D_n, \quad
    \partial^{C/D}_n[c] \coloneqq [\partial^C_n(c)].
  \]
\end{defn}

\begin{defn}
  Sei $(X, A)$ ein Raumpaar. Der \emph{relative singuläre Ketten- komplex} $\CC{C}(X, A)$ ist definiert als Quotientenkomplex $\CC{X}/\CC{A}$.\\
  Dessen Homologiegruppen heißen \emph{relative singuläre Homologiegruppen} $H_n(X, A)$.
\end{defn}

\begin{bem}
  $H_n$ ist ein Funktor $\Top(2) \to \AbGrp$.
\end{bem}

\begin{defn}
  Eine Homotopie zwischen Abbildungen von Raumpaaren $f, g : (X, A) \to (Y, B)$ ist eine Homotopie $H : \cinterval{0}{1} \times X \to Y$ zwischen $f$ und $g$ mit $H([0,1] \times A) \subset Y$.
\end{defn}

\begin{prop}
  Homotope Abbildungen von Raumpaaren induzieren dieselbe Abbildung in relativer Homologie.
\end{prop}

\begin{defn}
  Ein Kettenkomplex heißt \emph{exakt} (oder \emph{azyklisch}), falls seine Homologiegruppen alle Null sind, d.\,h. $\fa{n \in \N} \im \partial_{n+1} = \ker \partial_n$.
\end{defn}

\begin{defn}
  Eine \emph{kurze exakte Sequenz} (keS) ist ein exakter Kettenkomplex der Gestalt
  \[ 0 \to A \to B \to C \to 0. \]
\end{defn}

\begin{defn}
  Eine \emph{kurze exakte Sequenz von Kettenkomplexen} ist ein Diagramm der Form
  \[ 0 \to \CC{A} \to \CC{B} \to \CC{C} \to 0, \]
  in jedem Grad eine kurze exakte Sequenz
  \[ 0 \to A_n \to B_n \to C_n \to 0. \]
\end{defn}

\begin{bem}
  Ist $(X, A)$ ein Raumpaar, so erhält man eine kurze exakte Sequenz
  \[ 0 \to \CC{C}(A) \to \CC{C}(X) \to \CC{C}(X, A) \to 0. \]
\end{bem}

% 3.2
\begin{prop}[Schlangenlemma]
  Die ex. Sequenz $0 \to \CC{A} \to \CC{B} \to \CC{C} \to 0$ induziert in jedem Grad einen sogenannten \emph{verbindenden Homomorphismus} $\partial_n : H_n(C) \to H_{n-1}(A)$, sodass die Sequenz
  \[ ... \to H_n(A) \to H_n(B) \to H_n(C) \xrightarrow{\partial_n} H_{n-1}(A) \to H_{n-1}(B) \to ... \]
  exakt ist. Diese Sequenz wird \emph{lange exakte Sequenz} genannt.
\end{prop}

\begin{kor}
  Sei $(X, A)$ ein Raumpaar. Dann gibt es Homomorphis- men $\partial_n : H_n(X, A) \to H_{n-1}(A)$, sodass die Sequenz
  \[
    ... \to H_n(A) \to H_n(B) \to H_n(X, A) \to H_{n-1}(A) \to ...
    \quad \text{exakt ist.}
  \]
\end{kor}

\begin{defn}
  Die \emph{reduzierte Homologie} $\RH_*(X)$ eines topologischen Raumes $X$ ist die Homologie des Kettenkomplexes
  \[ ... \to C_2(X) \xrightarrow{\partial_n} C_1(X) \xrightarrow{\partial_1} C_0(X) \xrightarrow{\epsilon} \to \Z \to 0, \]
  wobei $\epsilon$ der sogenannte \emph{Augmentierungshomomorphismus} ist:
  \[ \epsilon : \sum_{\sigma \in \Delta_0(X)} \lambda_\sigma \cdot \sigma \mapsto \sum_{\sigma} \lambda_\sigma \in \Z. \]
\end{defn}

\begin{prop}
  \begin{itemize}
    \item $\RH_n(X) = H_n(X)$ für $n \geq 1$
    \item Ist $X = \emptyset$, so ist $\RH_n(X) = 0$ für $n \geq 0$ und $\RH_{-1}(X) = \Z$.
    \item Ist $X \not= \emptyset$, so ist $H_n(X) \cong \RH_n(X) \oplus \Z$, jedoch nicht kanonisch.
    \item Ist $X$ kontrahierbar, so gilt $\RH_n(X) = 0$ für alle $n \in \N$.
    \item Ist $(X, A)$ ein Raumpaar, so gibt es eine lange exakte Sequenz
    \[ ... \to \RH_0(A) \to \RH_0(X) \to \RH_0(X,A) \to \RH_{-1}(A) \to \RH_{-1}(X) \to 0 \]
  \end{itemize}
\end{prop}

% Vorlesung vom 20.10.2014

% 3.5
\begin{satz}
  Sei $(X, R)$ ein Raumpaar, $U \subset R$ mit $\clos{U} \subset \inte R$. Dann induziert die Inklusion $(X - U, R - U) \hookrightarrow (X, R)$ Isomorphismen
  \[ H_n(X - U, R - U) \to H_n(X, R) \quad \text{für alle $n \geq 0$.} \]
\end{satz}

\begin{bem}
  Eine äquivalente Aussage zum Ausschneidungssatz ist:\\
  Seien $A, B \subset X$ mit $X = \inte A \cup \inte B$. Dann induziert die Inklusion $(B, A \cap B) \to (X, A)$ Isomorphismen in Homologie.
\end{bem}

% Vorlesung vom 22.10.2014

% Ausgelassen: Hilfspropositionen zum Beweis des Ausschneidungssatzes, U-feine Ketten etc.

% Vorlesung vom 27.10.2014

\begin{defn}[\emph{Eilenberg-Steenrod-Axiome}]
  Eine \emph{Homologietheorie} ist eine Folge von Funktoren
  \[ H_n : \Top(2) \to \AbGrp, \]
  und natürlichen Transformationen
  \[ \delta_n : H_n(X, A) \to H_{n-1}(A, \emptyset) \]
  mit den folgenden Eigenschaften:
  \begin{itemize}
    \item \emph{Homotopieinvarianz}: Seien $f, g : (X, A) \to (B, Y)$ homotop als Abbildungen von Raumpaaren. Dann gilt $f_* = g_* : H_n(X, A) \to H_n(Y, B)$.
    \item \emph{Lange exakte Sequenz}: Die Inklusionen $A \hookrightarrow X$ und $(X, \emptyset) \hookrightarrow (X, A)$ induzieren eine l.e.S.
    \[ ... \to H_n(A) \to H_n(X) \to H_n(X, A) \xrightarrow{\partial_n} H_{n-1}(A) \to ... \]
    \item \emph{Ausschneidung}: Ist $U \subset A$ mit $\clos{A} \subset \inte(A)$, dann induziert die Inklusion $(X - U, A - U) \hookrightarrow (X, A)$ Isomorphismen
    \[ H_n(X-U, A-U) \to H_n(X,A). \]
  \end{itemize}
\end{defn}

\begin{defn}
  Die \emph{Koeffizienten} einer Homologietheorie sind die Homologiegruppen $H_n(pt)$ des einpunktigen Raums.
  Eine Homologietheorie heißt \emph{gewöhnlich}, falls $H_n(pt) = 0$ für $n > 0$.
\end{defn}

\begin{bem}
  Manchmal fordert man auch das Summenaxiom: Für eine Familie $(X_i)_{i \in I}$ von topologischen Räumen induzieren die Inklusionen $X_i \hookrightarrow X$ in die disjunkte Summe $X$ aller $X_i$ einen Iso
  \[ \bigoplus_{i \in I} H_n(X_i) \xrightarrow{\cong} H_n(X). \]
\end{bem}

\begin{bem}
  Die singuläre Homologie ist eine gewöhnliche Homologietheorie, die das Summenaxiom erfüllt.
\end{bem}

% §4. Erste Berechnungen und Anwendungen


\begin{bsp}[Homologie von wichtigen Räumen]
  \[
    H_i(S^n) = H_i(\R^n, \R^n - \{0\}; R) = \begin{cases}
      R, & \text{wenn $i=0,n$} \\
      0, & \text{sonst}
    \end{cases}
  \]
  \[
    H_i(\R P^n) = \begin{cases}
      \Z, & \text{wenn $i=0$} \\
      \Z/2, & \text{wenn $i < n$ ungerade} \\
      \Z, & \text{wenn $i=n$ ungerade} \\
      0, & \text{sonst}
    \end{cases}
  \]
  \[
    H_i(\R P^n; \Z/2) = \begin{cases}
      \Z/2, & \text{wenn $i \leq n$} \\
      0, & \text{sonst}
    \end{cases}
  \]
  \[
    H_i(\C P^n) = \begin{cases}
      \Z, & \text{wenn $i \leq 2n$ und $i$ gerade} \\
      0, & \text{sonst}
    \end{cases}
  \]
\end{bsp}

\end{document}