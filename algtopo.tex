\documentclass{cheat-sheet}

\pdfinfo{
  /Title (Zusammenfassung Algebraische Topologie)
  /Author (Tim Baumann)
}

\usepackage{tikz}
\usetikzlibrary{matrix,shapes,arrows,positioning}

\newcommand{\Tau}{\mathcal{T}} % Großes Tau
%\newcommand{\inte}{\mathop{\mathrm{int}}} % Inneres (interior)
%\newcommand{\conv}{\mathop{\mathrm{conv}}} % Konvexe Hülle
%\newcommand{\rel}{\text{ rel }} % = relativ (Homotopie)
%\newcommand{\Ob}{\mathrm{Ob}} % Objects (of a category)
%\newcommand{\Hom}{\mathrm{Hom}} % Homomorphisms
%\newcommand{\Aut}{\mathrm{Aut}} % Automorphisms
%\newcommand{\Mor}{\mathrm{Mor}} % Morphisms
%\newcommand{\dom}{\mathrm{dom}} % Domain
%\newcommand{\codom}{\mathrm{codom}} % Codomain
%\newcommand{\op}{\text{op}} % Duale Kategorie
%\newcommand{\Deck}{\mathrm{Deck}} % Gruppe der Decktransformationen

\newcommand{\angles}[1]{\langle #1 \rangle}
\newcommand{\Simpl}{\mathcal{S}} % Simplizialkomplex
\newcommand{\Real}[1]{\abs{#1}} % geometrische Realisierung

% Kleinere Klammern
\delimiterfactor=701


\begin{document}

\maketitle{Zusammenfassung Algebr. Topologie}

% Vorlesung vom 8.10.2014

% TODO: Definition $C_n$

%\[ \partial \angles{p_0, ..., p_n} \coloneqq \sum_{i=0}^n (-1)^{i} \angles{p_0, ..., \hat{p_i}, ..., p_n} \]
%\[ \partial_n : C_n \to C_{n-1} \]
%\[ Z_n(S) \coloneqq \ker \partial_n \subset C_n(S) \]
%\[ B_n(S) \coloneqq \im \partial_{n+1} \subset C_n(S) \]


% §2. Simpliziale und singuläre Homologie

\begin{defn}
  Ein \emph{affines $n$-Simplex} ist die konvexe Hülle von $n+1$ affin unabhängigen Punkten $p_0, ..., p_{n+1} \in \R^N$. Die konvexe Hülle von einer Teilmenge dieser Eckpunkte wird \emph{Seite} genannt.
\end{defn}

\begin{defn}
  Ein (endlicher) \emph{geometrischer Simplizialkomplex} ist eine (endliche) Menge $\mathcal{S}$ endlich vieler affiner Simplizes im $\R^N$, sodass:
  \begin{itemize}
    \item Ist $K \in \Simpl$ und $T \subset K$ eine Seite von $K$, dann ist auch $T \in \Simpl$.
    \item Für alle $K_1, K_2 \in \Simpl$ ist $K_1 \cap K_2$ entweder eine Seite von $K_1$ und $K_2$ oder leer.
  \end{itemize}
\end{defn}

\begin{defn}|
  Die $\Real{S} \coloneqq \bigcup_{K \in \Simpl} K$ heißt zu $\Simpl$ gehörender \emph{Polyeder} und $\Simpl$ Triangulierung von $\Real{\Simpl}$.
\end{defn}

\begin{defn}
  Ein geometrischer Simplizialkomplex mit einer Totalordnung auf der Menge der Eckpunkte heißt \emph{geordnet}.
\end{defn}

\begin{nota}
  Ein $n$-Simplex mit Eckpunkten $v_0, ..., v_n$ in einem geordneten geom. Simplizialkomplex wird mit $\angles{v_0, ..., v_n}$ bezeichnet, falls $v_0 < v_1 < ... < v_n$.
\end{nota}

\begin{nota}
  $\Simpl_n \coloneqq \Set{\sigma \in \Simpl}{\text{$\sigma$ ist geordneter $n$-Simplex}}$
\end{nota}

\begin{defn}
  Eine \emph{simpliziale $n$-Kette} in einem geordneten geom. Simplizialkomplex ist eine endliche formale Linearkombination
  \[ \sum_{\sigma \in \Simpl_n} \lambda_\sigma \cdot \sigma, \]
  wobei $\lambda_\sigma \in \Z$. Die Menge solcher Linearkombinationen ist $C_n(\Simpl)$.
\end{defn}

\begin{bem}
  $C_n(\Simpl)$ ist eine Gruppe.
\end{bem}

\begin{defn}
  Der Rand eines orientierten $n$-Simplex $\angles{v_0, ..., v_n} \in \Simpl$ ist
  \[ \delta \angles{v_0, ..., v_n} \coloneqq \sum_{i=0}^n (-1)^i \angles{v_0, ..., \hat{v_i}, ..., v_n}. \]
  Durch lineare Fortsetzung erhalten wir einen Gruppenhomo $\delta_n : C_n(\Simpl) \to C_{n-1}(\Simpl)$.
\end{defn}

% Vorlesung vom 8.10.2014

\end{document}