\documentclass{cheat-sheet}

\pdfinfo{
  /Title (Zusammenfassung Topologie)
  /Author (Tim Baumann)
}

\newcommand{\Tau}{\mathcal{T}} % Großes Tau
\newcommand{\inte}{\mathop{\mathrm{int}}} % Inneres (interior)

% Kleinere Klammern
\delimiterfactor=701


\begin{document}

\maketitle{Zusammenfassung Topologie}

% Vorlesung vom 6.4.2014

\begin{defn}
  Ein \emph{metrischer Raum} $(X, d)$ besteht aus einer Menge $X$ und einer Abbildung $d : X \times X \to \R_{\geq 0}$, sodass f.a. $x,y,z \in X$ gilt:
  \begin{itemize}
    \miniitem{0.44 \linewidth}{$d(x, y) = 0 \iff x = y$}
    \miniitem{0.54 \linewidth}{$d(x, y) = d(y, x)$ \quad (Symmetrie)}
    \item $d(x, z) \leq d(x, y) + d(y, z)$ \pright{$\triangle$-Ungleichung}
  \end{itemize}
\end{defn}

% Ausgelassen: Beispiel $\R^n$
% Ausgelassen: Beispiel Funktionenraum $\mathcal{C}(\cinterval{0}{1}, \R)$ mit Maximumsnorm

\begin{defn}
  Für einen metrischen Raum $(X, d)$ und eine Teilmenge $A \subset X$ ist $(A, d|_A)$ ein metrischer Raum und $d|_A$ heißt \emph{induzierte Metrik}.
\end{defn}

\begin{defn}
  Seien $(X, d_X)$ und $(Y, d_Y)$ metrische Räume. Eine Abbildung $f : X \to Y$ heißt \emph{stetig}, falls für alle $x \in X$ gilt:
  \[ \fa{\epsilon {>} 0} \ex{\delta {>} 0} \fa{x' {\in} X} d_X(x, x') < \delta \implies d_X(f(x), f(x')) < \epsilon. \]
\end{defn}

\begin{defn}
  Die \emph{offene Kugel} von Radius $\epsilon$ um $x \in X$ ist
  \[ B_\epsilon(x) \coloneqq \Set{ p \in X }{ d(p, x) < \epsilon }. \]
\end{defn}

\begin{defn}
  Eine Teilmenge $U \subset X$ eines metrischen Raumes heißt \emph{offen}, falls für alle $u \in U$ ein $\epsilon > 0$ existiert mit $B_{\epsilon}(u) \subset U$.
\end{defn}

\begin{prop}
  Eine Abbildung $f : X \to Y$ zwischen metrischen Räumen ist genau dann offen, wenn für alle offenen Teilmengen $U \subset Y$ das Urbild $f^{-1}(U) \subset X$ offen ist.
\end{prop}

\begin{defn}
  Ein \emph{topologischer Raum} $(X, \Tau)$ besteht aus einer Menge $X$ und einer Menge $\tau \subset \mathcal{P}(X)$ mit den Eigenschaften
  \begin{itemize}
    \miniitem{0.16 \linewidth}{$\emptyset \in \Tau$}
    \miniitem{0.45 \linewidth}{$\fa{U, V \in \Tau} U \cap V \in \Tau$}
    \miniitem{0.36 \linewidth}{$\fa{S \subset \Tau} \bigcap_{\mathclap{U \in S}} U \in \Tau$}
  \end{itemize}
  Die Elemente von $\Tau$ werden \emph{offene Teilmengen} von $X$ genannt. Eine Teilmenge $A \subset X$ heißt \emph{abgeschlossen}, falls $X \setminus A$ offen ist.
\end{defn}

\begin{nota}
  Seien im Folgenden $X$ und $Y$ topologische Räume.
\end{nota}

\begin{bsp}
  Die \emph{diskrete Topologie} auf einer Menge $X$ ist $\Tau = \mathcal{P}(X)$.
\end{bsp}

\begin{bsp}
  Die \emph{Klumpentopologie} auf einer Menge $X$ ist $\Tau = \{ \emptyset, X \}$.
\end{bsp}

\begin{defn}
  Die Menge der offenen Teilmengen eines metrischen Raumes heißt von der Metrik \emph{induzierte Topologie}.
\end{defn}

\begin{defn}
  Sei $(X, \Tau)$ ein topologischer Raum und $A \subset X$. Dann heißt
  \[ \Tau|_A \coloneqq \Set{U \cap A}{U \in \Tau} \]
  \emph{Unterraumtopologie} oder von $\Tau$ \emph{induzierte Topologie}.
\end{defn}

% Vorlesung vom 9.4.2014

\begin{defn}
  Ein topologischer Raum $(X, \Tau)$ heißt \emph{metrisierbar}, falls eine Metrik auf $X$ existiert, sodass die von der Metrik induzierte Topologie mit $\Tau$ übereinstimmt.
\end{defn}

\begin{defn}
  Ein topologischer Raum $(X, \Tau)$ heißt \emph{Hausdorffsch}, falls gilt:
  \[ \fa{x,y \in X} x \not= y \implies \ex{U,V \in \Tau} x \in U \wedge y \in V \wedge U \cap V = \emptyset. \]
\end{defn}

\begin{prop}
  Metrisierbare topologische Räume sind Hausdorffsch.
\end{prop}

\begin{defn}
  Eine Abbildung $f : X \to Y$ zwischen topologischen Räumen $(X, \Tau_X)$ und $(Y, \Tau_Y)$ heißt \emph{stetig}, falls gilt
  \[ \fa{U \in \Tau_Y} f^{-1}(U) \in \Tau_X. \]
\end{defn}

\begin{bem}
  Ist $f : X \to Y$ stetig und $A \subset X$, so ist auch $f|_A : A \to Y$ stetig.
\end{bem}

\begin{defn}
  Falls $f : X \to Y$ bijektiv ist und sowohl $f$ als auch $f^{-1}$ stetig sind, so heißt $f$ ein \emph{Homöomorphismus}.
\end{defn}

\begin{defn}
  Zwei topologische Räume $X$ und $Y$ heißen \emph{homöomorph} (notiert $X \cong Y$), wenn ein Homöomorphismus zwischen $X$ und $Y$ existiert.
\end{defn}

\begin{satz}
  Für $n \not= m$ sind $\R^n$ und $\R^m$ nicht homöomorph.
\end{satz}

\begin{defn}
  Sei $X$ eine Menge und $\Tau, \Tau'$ Topologien auf $X$. Dann sagen wir
  \[ \Tau \text{ ist \emph{gröber} als } \Tau' \coloniff \Tau' \text{ ist \emph{feiner} als } \Tau \coloniff \Tau \subset \Tau'. \]
\end{defn}

% Bemerkung: Die Klumpentopologie ist die gröbste und die diskrete Topologie die feinste Topologie auf $X$.

\begin{defn}
  Eine Menge $\mathcal{B} \subset \Tau$ offener Teilmengen eines topologischen Raumes heißt
  \begin{itemize}
    \item \emph{Basis} der Topologie, falls jede offene Menge $U \in \Tau$ Vereinigung von Mengen aus $\mathcal{B}$ ist.
    \item \emph{Subbasis} der Topologie, falls jede offene Menge $U \in \Tau$ Vereinigung von Mengen ist, von denen jede Schnitt endlich vieler Mengen aus $\mathcal{B}$ ist.
  \end{itemize}
\end{defn}

\begin{bspe}
  \begin{itemize}
    \item Sei $(X, d)$ ein metrischer Raum. Dann ist $\mathcal{B} \coloneqq \Set{B_\epsilon(x)}{x \in X, \epsilon > 0}$ eine Basis der induz. Topologie auf $X$.
    \item $\mathcal{B} \coloneqq \Set{B_\epsilon(x)}{x \in \Q^n, \epsilon \in \Q_{+}}$ ist eine abz. Basis von $(\R^n, \d_{\text{eukl}})$.
  \end{itemize}
\end{bspe}

\begin{prop}
  Jede Teilmenge $\mathcal{B} \subset \mathcal{P}(X)$ ist Subbasis von genau einer Topologie $\Tau$ von $X$.
\end{prop}

\begin{defn}
  Die Topologie heißt die von $\mathcal{B}$ \emph{erzeugte Topologie}.
\end{defn}

\begin{defn}
  Sind $(X, \Tau_X)$ und $(Y, \Tau_Y)$ topologische Räume, so ist auch $(X \times Y, \Tau_X \otimes \Tau_Y)$ ein topologischer Raum mit der \emph{Produkttopologie} $(\Tau_X \otimes \Tau_Y)$, die von
  \[
    \mathcal{B} \coloneqq \Set{U \times Y}{U \in \Tau_X} \cup \Set{X \times V}{V \in \Tau_Y}
    \quad \text{erzeugt wird.}
  \]
  % Ausgelassen: "`Streifen"' und "`Rechtecke"'
\end{defn}

\begin{prop}
  \begin{itemize}
    \item Die Projektionen $\pi_X : X \times Y \to X$ und $\pi_Y : X \times Y \to Y$ sind stetig bzgl. der Produkttopologie.
    \item Ist $\Tau$ eine echt gröbere Topologie auf $X \times Y$ als die Produkttopologie, so sind die Projektionen $\pi_X$ und $\pi_Y$ nicht beide stetig.
  \end{itemize}
\end{prop}

% Bemerkung: Die Produkttopologie ist somit die gröbste Topologie auf $X \times Y$, sodass beide Projektionen stetig sind.

\begin{defn}
  Seien $(X, \Tau_X)$ und $(Y, \Tau_Y)$ topologische Räume. Dann erzeugt $\Tau_X \cup \Tau_Y$ die \emph{Summentopologie} auf $X \cup Y$.
\end{defn}

\begin{bem}
  Sie ist die feinste Topologie auf $X \cup Y$, sodass die beiden Inklusionen $i_X : X \hookrightarrow X \cup Y$ und $i_Y : Y \hookrightarrow X \cup Y$ stetig sind.
\end{bem}

\begin{prop}
  Seien $X, Y, Z$ topologische Räume.
  \begin{itemize}
    \item Falls $X \cap Y = \emptyset$, so ist eine Abbildung $f : X \cup Y \to Z$ genau dann stetig, falls die beiden Kompositionen $f \circ i_X : X \to Z$ und $f \circ i_Y : Y \to Z$ stetig sind.
    \item Eine Abb. $g : Z \to X \cup Y$ ist genau dann stetig, wenn die beiden Kompositionen $\pi_X \circ g : Z \to X$ und $\pi_Y \circ g : Z \to Y$ stetig sind.
  \end{itemize}
\end{prop}

\begin{defn}
  Sei $X$ ein topologischer Raum und $A \subset X$. Dann ist das \emph{Innere} von $A$ (notiert $\inte(A)$) die Vereinigung aller in $A$ enthaltenen offenen Mengen.
\end{defn}

\begin{bem}
  Als Vereinigung offener Mengen ist das Innere offen.
\end{bem}

% $\inte(A)$ ist die größte in $A$ enthaltene in $X$ offene Teilmenge.

\begin{defn}
  Der \emph{Abschluss} $\overline{A}$ einer Menge $A \subset X$ ist der Durchschnitt aller abgeschlossenen Mengen von $X$, die $A$ enthalten.
\end{defn}

\begin{bem}
  Es gilt $\overline{A} = X \setminus (\inte(X \setminus A))$.
\end{bem}

\begin{defn}
  Es sei $X$ ein topologischer Raum, $x \in X$ und $V \subset X$. Wir nennen $V$ eine \emph{Umgebung} von $x$, falls es eine offene Teilmenge $U \subset X$ gibt mit $x \in U$ und $U \subset V$.
\end{defn}

\begin{prop}
  Ein Punkt $x \in X$ liegt genau dann in $\overline{A}$, falls jede Umgebung von $x$ einen Punkt aus $A$ enthält.
\end{prop}

\begin{defn}
  Der \emph{Rand} einer Menge $A \subset X$ ist $\partial A \coloneqq \overline{A} \setminus \inte(A)$.
\end{defn}

\begin{prop}
  Ein Punkt $x \in X$ liegt genau dann in $\partial X$, wenn jede Umgebung von $x$ sowohl einen Punkt aus $A$ wie einen Punkt aus $X \setminus A$ enthält.
\end{prop}

% Vorlesung vom 14.4.2014

% Thema: Zusammenhang und Wegzusammenhang

\begin{defn}
  Ein topologischer Raum $X$ heißt \emph{wegweise zusammen- hängend}, falls es für je zwei Punkte $x, y \in X$ eine stetige Abbildung $\gamma : \cinterval{0}{1} \to X$ mit $\gamma(0) = x$ und $\gamma(1) = y$ gibt.
\end{defn}

\begin{bspe}
  \begin{itemize}
    \item $\R^n$ ist wegzusammenhängend
    \item $(\{ p, q \}, \{ \emptyset, \{ p \}, \{ p, q \} \})$ ist wegzusammenhängend!
    \item $\ointerval{-\infty}{0} \cup \ointerval{0}{\infty} \subset \R$ ist nicht wegzusammenhängend.
  \end{itemize}
\end{bspe}

\begin{defn}
  Die Äquivalenzklassen von
  \[ x \sim y \coloniff x, y \text{ lassen sich durch einen Weg verbinden}. \]
  heißen \emph{Wegzusammenhangskomponenten}.
\end{defn}

\begin{prop}
  Sei $f : X \to Y$ stetig und $X$ wegzusammenhängend. Dann ist auch $f(X)$ bzgl. der Unterraumtopologie wegzusammenhängend.
\end{prop}

\begin{defn}
  Ein topologischer Raum $X$ heißt \emph{zusammenhängend}, falls $X$ nicht disjunkte Vereinigung zweier nichtleerer offener Teilmengen ist.
\end{defn}

\begin{bspe}
  $\Q \subset \R$ und $\R \setminus \{ 0 \}$ sind nicht zusammenhängend.
\end{bspe}

\begin{prop}
  Sei $X$ ein topologischer Raum. Es sind äquivalent:
  \begin{itemize}
    \item $X$ ist zusammenhängend.
    \item Für jede offene und abgeschlossene Menge $A \subset X$ gilt: $A \in \{ X, \emptyset \}$.
    \item Jede stetige Abbildung $f : X \to \{ 0, 1 \}$ in den diskreten Raum mit zwei Elementen ist konstant.
  \end{itemize}
\end{prop}

\begin{prop}
  \begin{itemize}
    \item Sei $f : X \to Y$ stetig und $X$ zusammenhängend, dann ist auch $f(X)$ zusammenhängend.
    \item Sind $A, B$ zusammenhängende Teilmengen eines topologischen Raumes $X$ und gilt $A \cap B \not= \emptyset$, dann ist auch $A \cup B$ zusammenhängend.
  \end{itemize}
\end{prop}

\begin{kor}
  Folgende Relation ist eine Äquivalenzrelation auf $X$:
  \begin{align*}
    x \sim y \coloniff \, &\text{$x$ und $y$ liegen beide in einem zusammenhängenden}\\[-2pt]
    &\text{Unterraum von $X$.}
  \end{align*}
\end{kor}

\begin{defn}
  Die Äquivalenzklassen dieser Relation heißen \emph{Komponenten}. % von $X$
\end{defn}

\begin{bsp}
  Die Komponenten von $\Q \subset \R$ sind genau die Ein-Punkt-Mengen. Trotzdem ist $\Q$ nicht diskret!
\end{bsp}

\begin{prop}
  Die Menge $\cinterval{0}{1}$ ist zusammenhängend.
\end{prop}

\begin{kor}
  Wegzusammenhängende Räume sind zusammenhängend.
\end{kor}

\begin{prop}[ZWS]
  Sei $f : \cinterval{0}{1} \to \R$ stetig. Gilt $f(0) < 0$ und $f(1) > 0$, so existiert ein $t \in \ointerval{0}{1}$ mit $f(t) = 0$.
\end{prop}

% Vorlesung vom 16.4.2014

\begin{defn}
  Sei $(x_n)_{n \in \N}$ eine Folge in $X$. Die Folge $(x_n)$ \emph{konvergiert gegen} $x \in X$, falls für jede Umgebung $U \subset X$ von $x$ ein $N \in \N$ existiert mit $\fa{n \geq N} x_n \in U$.
\end{defn}

\begin{nota}
  $x = \lim_{n \to \infty} x_n$
\end{nota}

\begin{acht}
  Das "`="' ist nicht wörtlich zu verstehen!
\end{acht}

\begin{defn}
  Sei $f : X {\to} Y$ eine Abb. zw. topol. Räumen $X, Y$. Dann heißt $f$
  \begin{itemize}
    \item \emph{stetig in} $x \in X$, falls für jede Umgebung $V \subset Y$ von $f(x)$ das Urbild $f^{-1}(V) \subset X$ eine Umgebung von $x$ ist.
    \item \emph{folgenstetig in} $x \in X$, falls für jede Folge $(x_n)_{n \in \N}$ in $X$ mit $\lim_{n\to\infty} x_n = x$ die Bildfolge $(f(x_n))$ in $Y$ gegen $f(x)$ konvergiert.
  \end{itemize}
\end{defn}

\begin{prop}
  Ist $f$ stetig in $x$, so ist $f$ auch folgenstetig in $x$.
\end{prop}

\begin{defn}
  Eine \emph{Umgebungsbasis} von $x \in X$ ist eine Menge $\mathcal{B} \subset \mathcal{P}(X)$ bestehend aus Umgebungen von $x$, sodass jede Umgebung von $x$ eine der Umgebungen in $\mathcal{B}$ enthält.
\end{defn}

\begin{defn}
  Der Raum $X$ erfüllt das \emph{erste Abzählbarkeitsaxiom}, falls jeder Punkt $x \in X$ eine abzählbare Umgebungsbasis besitzt.
\end{defn}

\begin{bem}
  Jeder metrische Raum $X$ erfüllt das erste Abzählbarkeitsaxiom, da für jeden Punkt $x \in X$ die Menge $\mathcal{B}_x \coloneqq \Set{B_{1/n}(x)}{n\in\N}$ eine abzählbare Umgebungsbasis ist.
\end{bem}

\begin{prop}
  Sei $x \in X$ ein Punkt mit abzählbarer Umgebungsbasis. Dann ist jede in $x$ folgenstetige Abbildung $f : X \to Y$ auch stetig in $x$.
\end{prop}

\begin{defn}
  Eine \emph{gerichtete Menge} ist eine Menge $D$ mit einer partiellen Ordnung $(\le) \subset D \times D$, sodass es für $\alpha, \beta \in D$ immer ein $\gamma \in D$ mit $\gamma \geq \alpha$ und $\gamma \geq \beta$ gibt.
\end{defn}

\begin{defn}
  Ein \emph{Netz} in $X$ ist eine Abbildung $\phi : D \to X$, wobei $D$ eine gerichtete Menge ist.
\end{defn}

\iffalse
\begin{bspe}
  \begin{itemize}
    \item $D = (\N, \leq)$
    \item $X$ beliebige Menge, $D = (\mathcal{P}(X), \subseteq)$ oder $D = (\mathcal{P}(X), \supseteq)$
    \item Sei $(X, \tau)$ ein top. Raum, $x \in X$, $D \coloneqq (\Set{U \subset X}{\text{$U$ Umgebung von $x$}}, \leq)$ mit $U \leq V \coloneqq V \subset U$.
  \end{itemize}
\end{bspe}
\fi

\begin{defn}
  Sei $x \in X$ und $(x_\alpha)_{\alpha \in D}$ ein Netz in $X$. Das Netz $(x_\alpha)$ \emph{konvergiert} gegen $x$, falls es für jede Umgebung $U \subset X$ von $x$ ein $\beta \in D$ gibt mit $x_\alpha \in U$ für alle $\alpha \geq \beta$.
\end{defn}

\begin{nota}
  $\lim_{\alpha \in D} x_\alpha = x$
\end{nota}

\begin{defn}
  Eine Abb. $f : X \to Y$ heißt \emph{netzstetig} in $x \in X$, falls für jedes Netz $(x_\alpha)_{\alpha \in D}$ in $X$ mit $\lim_{\alpha \in D} x_\alpha = x$ das Bildnetz $(f(x_\alpha))_{\alpha \in D}$ gegen $f(x)$ konvergiert.
\end{defn}

\begin{prop}
  Eine Abbildung $f : X \to Y$ ist genau dann stetig in $x \in X$, wenn sie netzstetig in $x$ ist.
\end{prop}

\begin{prop}
  Ist $A \subset X$ eine Teilmenge eines topologischen Raumes, so besteht $\overline{A}$ genau aus den Limiten von Netzen in $A$, die in $X$ konvergieren.
\end{prop}

\begin{defn}
  Ein \emph{Häufungspunkt} eines Netzes $(x_\alpha)_{\alpha \in D}$ in $X$ ist ein Punkt $x \in X$, sodass für jede Umgebung $U \subset X$ von $x$ das Netz \emph{häufig} in $U$ ist, d.\,h. für alle $\alpha \in D$ existiert ein $\beta \geq \alpha$ mit $x_\beta \in U$.
\end{defn}

\begin{defn}
  Sind $D$ und $E$ gerichtete Mengen, so nennen wir eine Abbildung $h : E \to D$ \emph{final}, falls für alle $\delta \in D$ ein $\eta \in E$ existiert mit $h(\gamma) \geq \delta$ für alle $\delta \geq \eta$.
\end{defn}

\begin{defn}
  Ein \emph{Unternetz} eines Netzes $\phi : D \to X$ ist eine Komposition $\phi \circ h : E \to X$ wobei $h : E \to D$ eine finale Funktion ist. Wir schreiben auch $(x_{h(\gamma)})_{\gamma \in E}$
\end{defn}

% Vorlesung vom 23.4.2014

\begin{prop}
  Sei $(x_\alpha)_{\alpha \in D}$ ein Netz in $X$. Ein Punkt $x \in X$ ist genau dann Häufungspunkt von $(x_\alpha)$, falls ein Unternetz von $(x_\alpha)$ gegen $x$ konvergiert.
\end{prop}

\begin{defn}
  Eine Folge $(x_n)_{n \in \N}$ in einem metrischen Raum $(X, d)$ heißt \emph{Cauchy-Folge}, falls es für jedes $\epsilon > 0$ ein $N \in \N$ gibt mit $d(x_n, x_m) < \epsilon$ für alle $n, m \geq N$.
\end{defn}

\begin{defn}
  Der metrische Raum $(X, d)$ heißt \emph{vollständig}, wenn jede Cauchy-Folge in $X$ konvergiert.
\end{defn}

% Beispiele:
% * Wenn $(X_1, d_1)$ und $(X_2, d_2)$ vollständig, dann auch $(X_1 \times X_2, d)$ mit $d((x_1, x_2), (y_1, y_2)) \coloneqq \sqrt{d_1(x_1, y_1)^2 + d_2(x_2, y_2)^2}$
% * $\R^n$ ist vollständig
% * Ist $X$ vollständig und $A \subset X$ abgeschlossen, dann ist auch $A$ mit der induzierten Metrik vollständig
% * Ist allgemeiner $A \subset X$ ein beliebiger Unterraum, dann ist $\overline{A} \subset X$ der kleinste vollständige Unterraum, der $A$ enthält

\begin{acht}
  Vollständigkeit ist keine Homöomorphieinvariante!
  % Gegenbeispiel: $\ointerval{0}{1} \cong \R$, aber $\R$ ist vollständig, $\ointerval{0}{1}$ ist es nicht
\end{acht}

\begin{defn}
  Sei $X$ eine Menge. Dann ist die Menge
  \[ \mathcal{B}(X) \coloneqq \Set{ f : X \to \R }{ \sup_{x \in X} \abs{f(x)} < \infty } \]
  der \emph{beschränkten Fktn.} $X \to \R$ ein metrischer Raum mit
  \[ d(f, g) \coloneqq \sup_{x \in X} \abs{f(x) - g(x)}. \]
\end{defn}

\begin{prop}
  Dieser Raum $(\mathcal{B}(X), d)$ ist vollständig.
\end{prop}

\begin{defn}
  Sie $(X, d)$ und $(X', d')$ metrische Räume, so heißt $f : X \to X'$
  \begin{itemize}
    \item eine \emph{isometrische Einbettung}, falls für alle $x , y \in X$ gilt:
    \[ d'(f(x), f(y)) = d(x, y) \]
    \item eine \emph{Isometrie}, falls $f$ zusätzlich bijektiv ist. In diesem Fall ist auch $f^{-1}$ eine Isometrie und $f$ ein Homöomorphismus.
  \end{itemize}
\end{defn}

\begin{prop}
  Sei $X$ ein metrischer Raum. Dann gibt es eine isome- trische Einbettung von $X$ in einen vollständigen metrischen Raum.
\end{prop}

\begin{satz}
  Ist $X$ ein metrischer Raum, so existiert eine Vervollständigung $X \hookrightarrow Y$.
\end{satz}

\begin{prop}
  Es sei $X$ ein metrischer Raum und es seien
  \[ f_1 : X \to Y_1, \quad f_2 : X \to Y_2 \]
  Vervollständigungen von $X$. Dann existiert genau eine Isometrie $\phi_{21} : Y_1 \to Y_2$ mit $\phi_{21}|_{f_1(X)} = f_2 \circ f_1^{-1}$.
\end{prop}

\end{document}