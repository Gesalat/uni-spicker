\documentclass{cheat-sheet}

\usepackage{syntax}

\pdfinfo{
  /Title (Zusammenfassung Informatik 3)
  /Author (Tim Baumann)
}

\begin{document}

\maketitle{Zusammenfassung Informatik \rom{3}}

\begin{abk}
  \begin{tabular}{r l}
    \textbf{WC} & Worst Case \\
    \textbf{AC} & Worst Case \\
    \textbf{BC} & Best Case
  \end{tabular}
\end{abk}

\begin{alg}[Insertion Sort]
  BC: $O(n)$; AC, WC: $O(n^2)$
\end{alg}

\begin{nota}
  Sei $\mathcal{F}$ die Menge der Funktionen von $\N$ nach $\R_{\geq 0}$. Ist $g \in \mathcal{F}$, dann definieren wir
  \begin{align*}
    O(f) \coloneqq& \, \Set{ g \in \mathcal{F} }{ \exists\,c > 0\,\exists\,n_0 \in \N\,\forall\,n \geq n_0 : g(n) \leq c \cdot f(n) } \\
    \Omega(f) \coloneqq& \, \Set{ g \in \mathcal{F} }{ \exists\,c > 0\,\exists\,n_0 \in \N\,\forall\,n \geq n_0 : g(n) \leq c \cdot f(n) } \\
    o(f) \coloneqq& \, \Set{ g \in \mathcal{F} }{ \forall\,c > 0\,\exists\,n_0 \in \N\,\forall\,n \geq n_0 : g(n) \leq c \cdot f(n) } \\
    \omega(f) \coloneqq& \, \Set{ g \in \mathcal{F} }{ \forall\,c > 0\,\exists\,n_0 \in \N\,\forall\,n \geq n_0 : g(n) \leq c \cdot f(n) } \\
    \Theta(f) \coloneqq& \, \{ g \in \mathcal{F} \,|\, \exists\,c_1, c_2 > 0\,\exists\,n_0 \in \N\,\forall\,n \geq n_0 : \\
    & \qquad \qquad \qquad c_1 \cdot f(n) \leq g(n) \leq c_2 \cdot f(n) \}
              = O(f) \cap \Omega(f) \\
  \end{align*}
\end{nota}

\begin{satz}
  Seien $0 < \alpha < \beta$, $0 < a < b$ und $1 < A < B$. Betrachte
  \begin{multicols}{3}
    \begin{itemize}
      \item $f_1(n) \coloneqq \log \log n$
      \item $f_2(n) \coloneqq (\log n)^\alpha$
      \item $f_3(n) \coloneqq (\log n)^\beta$
      \item $f_4(n) \coloneqq n^a$
      \item $f_5(n) \coloneqq n^a (\log n)^\alpha$
      \item $f_6(n) \coloneqq n^b (\log n)^\alpha$
      \item $f_7(n) \coloneqq n^b$
      \item $f_8(n) \coloneqq A^n$
      \item $f_9(n) \coloneqq A^n \cdot n^a$
      \item $f_{10}(n) \coloneqq A^n \cdot n^b$
      \item $f_{11}(n) \coloneqq B^n$
    \end{itemize}
  \end{multicols}
  Es gilt: $f_i \in o(f_{i+1})$ für $i = 1, ..., 10$.
\end{satz}

% TODO: Definition RAM

\begin{defn}[RAM]
  Die Random Access Access Machine besitzt eine unendlich lange Liste von aufsteigend nummerierten Speicherzellen R[0], R[1], ..., die jeweils eine ganze Zahl beinhalten und einen Programmzähler. Sie kann mittels der folgenden Sprache programmiert werden:
  \begin{grammar}
    <Zieladresse> ::= <Adresse> | R[<Adresse>]

    <Operand> ::= <Literal> | R[<Adresse>]

    <Befehl> ::= <Zieladresse> `:=' <Operand> $\odot$ <Operand>
    \alt `if' <Operand> $\bowtie$ <Operand> `goto' <Label>

    <Programm> ::= <Befehl> `;' <Programm> | `End'
  \end{grammar}
  wobei $\odot \in \{ +, -, *, \div \}$ und $\bowtie \, \in \{ <, \leq, =, \geq, >, \not= \}$. Diese einfache Grammatik lässt sich auch für unbedingte Sprünge nutzen (mittels Bedingung $0 = 0$). Ein Sprung über das Ende des Programms hinaus lässt das Programm anhalten. Per Konvention steht die Größe der Eingabe in der Speicherzelle R[1], während die tatsächliche Eingabe in R[2], ..., R[R[1] + 1] abgelegt wird.
\end{defn},

% TODO: Einführung Bäume / gerichtete Graphen
% TODO: Einführung Stochastik

\begin{alg}
  Zwei sortierte Folgen der Gesamtlänge $n$ können in $O(n)$ Zeit gemischt werden.
\end{alg}

\begin{alg}[Sortieren durch Mischen / Mergesort]
  $n$ Elemente mit Schlüsseln aus einem total geordneten Universum können in $O(n \log n)$ Zeit nach ihren Schlüsseln sortiert werden.
\end{alg}

\begin{satz}[Master-Theorem]
  Seien $a, b, c, k, N$ reelle Zahlen mit $a, c > 0$, $k \geq 0$, $b, N \in \N$ und $b \geq 2$ und sei $T : \N \to \R_{\geq 0}$ eine Funktion, die folgende Rekursionsungleichung erfüllt:
  \[ T(n) \leq \begin{cases}
    c, & \text{ für } n \leq N \\
    c n^k + a T(\lceil n / b \rceil), & \text{ für } n > N
  \end{cases} \]
  Sei ferner $\lambda := \log_b a$. Dann gilt
  \[ T(n) = \begin{cases}
    O(n^k), & \text{ falls } \lambda < k \\
    O(n^k \log n), & \text{ falls } \lambda = k \\
    O(n^\lambda), & \text{ falls } \lambda > k.
  \end{cases} \]
\end{satz}

\begin{satz}
  Seien $a, b, c, k, N$ reelle Zahlen mit $a, c > 0$, $k \geq 0$, $b, N \in \N$ und $b \geq 2$ und sei $T : \N \to \R_{\geq 0}$ eine Funktion, die folgende Rekursionsungleichung erfüllt:
  \[ T(n) \geq \begin{cases}
    c, & \text{ für } n \leq N \\
    c n^k + a T(\lceil n / b \rceil), & \text{ für } n > N
  \end{cases} \]
  Sei ferner $\lambda := \log_b a$. Dann gilt
  \[ T(n) = \begin{cases}
    \Omega(n^k), & \text{ falls } \lambda < k \\
    \Omega(n^k \log n), & \text{ falls } \lambda = k \\
    \Omega(n^\lambda), & \text{ falls } \lambda > k.
  \end{cases} \]
\end{satz}

% Wichtigkeit?
\begin{satz}
  Seien $\beta, c, k, n$ reelle Zahlen mit $c, k > 0$, $n \in \N_0$ und $0 < \beta < 1$ und sei $T : \N_0 \to \R_{\geq 0}$ eine Funktion, die folgende Rekursionsungleichung erfüllt:
  \[ T(n) \leq \begin{cases}
    c, & \text{ für } n \leq N \\
    c n^k + T(\lfloor \beta n \rfloor), & \text{ für } n > N.
  \end{cases} \]
  Dann ist $T(n) = O(n^k)$
\end{satz}

% TODO: Algorithmus beschreiben?

\begin{satz}[Karatsuba und Ofman]
  Zwei $n$-stellige Zahlen können in $O(n^{\log_2 3})$ Zeit multipliziert werden.
\end{satz}

% TODO: Algorithmus beschreiben?

\begin{satz}[Selektion]
  Gegeben seien eine Menge $X$ von $n$ Elementen aus einem total geordneten Universum und eine ganze Zahl $k$ mit $1 \leq k \leq n$. Dann können wir (deterministisch) in $O(n)$ Zeit das $k$-kleinste Element aus $X$ bestimmen.
\end{satz}

\end{document}
