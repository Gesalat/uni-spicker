\documentclass{cheat-sheet}

\pdfinfo{
  /Title (Zusammenfassung Gewöhnliche Differentialgleichungen)
  /Author (Tim Baumann)
}

% Kleinere Klammern
\delimiterfactor=701

\newcommand{\D}{\mathcal{D}}

\begin{document}

\maketitle{Zusammenfassung Gewöhnliche DGLn}

% Kapitel 1. Einführung

% Kapitel 1.1. Beispiele

\iffalse
\begin{bsp}
  Gesucht: Funktion $y : \R \to \R$ mit $\fa{t \in \R} \dot{y}(t) = y(t)$\\
\end{bsp}

\begin{lsg}
  $y(t) = c \cdot e^t$ für $c \in \R$ beliebig. Wenn man als Anfangsbe- dingung $y(0) = 1$ fordert, erhält man eine eindeutige Lösung ($c = 1$).
\end{lsg}

\begin{bsp}
  Gesucht: Lösung von $\left(\dot{y}(t)\right)^2 + \left(y(t)\right)^2 = a$ für $a \in \R$\\
\end{bsp}

\begin{lsg}
  Anzahl der Lösungen hängt von $a$ ab:
  \begin{itemize}
    \miniitem{0.47\linewidth}{Falls $a < 0$: keine reelle Lsg}
    \miniitem{0.51\linewidth}{Falls $a = 0$: Einzige Lsg $y(t) = 0$}
    \item Falls $a > 0$: Lsgn: $y(t) = \sqrt{a} \cos(t + \phi)$ für $\phi \in \R$ bel., $y(t) = \pm \sqrt{a}$
  \end{itemize}
\end{lsg}

% Populationsmodell aus der Biologie/Chemie
\begin{bsp}
  Sei $p(t)$ ist Populationsgröße zur Zeit $t$.
  Angenommen, $\tfrac{\dot{p}(t)}{p(t)} = a$ ist konstant, also $\dot{p}(t) = p(t)$. Sei $p(t_0) = p_0$.
\end{bsp}

\begin{lsg}
  $p(t) = p_0 e^{(t-t_0) a}$
\end{lsg}

% 1837
\begin{bsp}[Verhulst-Modell]
  Gesucht: Lösung zu
  \[ \dot{p}(t) = a_0 p(t) - a_1 \left(p(t)\right)^2 \]
\end{bsp}

\begin{lsg}
  $p(t) = \frac{a_0}{a_1 (1 - c e^{-a_0 t})}$
\end{lsg}

% Mechanik: Mathematisches Pendel
% Skizze: Pendel der
% * Masse $m$
% * Länge $l$
% * Auslenkungswinkel $\phi(t)$ zum Zeitpunkt $t$
% * Position $l \phi(t)$ zum Zeitpunkt $t$
% * Geschwindigkeit $v(t) = \dot{p}(t) = l \dot{\phi}(t)$
% * $a(t) = \ddot{p}(t) = \dot{v}(t) = l \ddot{\phi}(t)$ Beschleunigung
% $F_E = - mg$, $F = ma$
% $F_T$ -- tangentiale Komponente der Gewichtskraft
% $F_T = - mg \sin(\phi)$
% => Gleichung $-mg \sin(\phi) = m a(t) = m l \ddot{t}$, also $\ddot{\phi}(t) = -\tfrac{g}{l} \sin(\phi(t))$
\fi

% Vorlesung vom 9.4.2014

% Literatur
% Alle Bücher haben den Titel "`Gewöhnliche Differentialgleichungen"'
% * B. Aulbach, 2004
% * H. Henser, 2009
% * L. Grüne, O. Junge, 2009
% * W. Walter, 2000

% Kapitel 1.2. Klassifikation von Differentialgleichungen (DGLn)

\begin{defn}[Klassifikation von DGLn]\mbox{}\\
  \begin{enumerate}[label=(\Roman*),leftmargin=2em]
    \item \emph{Gewöhnliche} DGL: Gesucht ist Funktion in einer Variable\\
    \emph{Partielle} DGL: Gesucht ist Funktion in mehreren Variablen
    \item \emph{Ordnung} einer DGL: Höchste Ableitung der gesuchten Funktion, die in Gleichung vorkommt
    \item \emph{Explizite} DGL: Gleichung der Form
    $y^{(k)} {=} f(t, y, \dot{y}, ..., y^{(k{-}1)})$
    \emph{Implizite} DGL: Allgemeinere Form $F(t, y, \dot{y}, ..., y^{(k)}) = 0$
    \item \emph{Skalare} DGL: Gesucht ist Funktion mit Wert in $\R$\\
    \emph{$n$-dimensionale} DGL: Gesuchte Funktion hat Wert in $\R^n$
    \item \emph{Lineare} DGL: Gleichung hat die Form
    \[ a_k(t) y^{(k)}(t) + a_{(k-1)}(t) y^{k-1}(t) + ... + a_1(t) \dot{y}(t) + a_0(t)y(t) = 0 \]
    \item \emph{Autonome} DGL: Gleichung der Form $F(y, \dot{y}, ..., y^{(k)}) = 0$\\
    (keine Abhängigkeit von $t$, Zeitinvarianz)
  \end{enumerate}
\end{defn}

\iffalse
% I)

Unterscheidung zwischen gewöhnliche DGL und partielle DGLn

Beispiele für gewöhnliche DGL
$\dot{y}(t) = h y(t)$
$(\dot{y}(t))^2 + (y(t))^2 = a$

Beispiele für partielle DGLn:

$y_t = \alpha y_{xx} + y$, wobei $y_t(t,x) = \tfrac{\partial}{\partial t} y(t, x)$, $y_{xx}(t, x) = \tfrac{\partial^2}{\partial x^2} y(t, x)$

% II)

Unterscheidung zwischen DGLn 1. Ordnung, DGLn 2. Ordnung und DGLn $k$-ter Ordnung

Beispiel für DGL 1. Ordnung:
$\dot{y} = \alpha y(t)$

Beispiel für DGL 2. Ordnung:
$\ddot{\phi}(t) = - \tfrac{\delta}{e} \sin(\phi(t))$

Beispiel für DGL $k$-ter Ordnung:
$F(t, y(t), \dot{y}(t), ..., y^{(k)}(t)) = 0$

% III)

Unterscheidung zischen expliziten und impliziten DGLn

Beispiel für explizite DGLn:
$\dot{y}(t) = \alpha y(t)$
$\ddot{\phi}(t) = - \tfrac{g}{e} \sin(\phi(t))$
$y^{(k)}(t) = f(t, y, \dot{y}, ..., y^{(k-1)})$

Beispiele für implizite DGLn:
$(\dot{y}(t))^2 + (y(t))^2 = a$
$F(t, y, \dot{y}, ..., y^{(k)}(t)) = 0$

Oder (Gleichungen gehören zusammen)
$\dot{y}_2(t) + y_1(t) = f_1(t)$
$y_2(t) = f_2(t)$
(differentiell-algebraische Gleichung)

% IV)
Unterscheidung zwischen Skalaren DGLn und $n$-dimensionalen DGLn (Systeme von DGLn)

Beispiel für Skalare DGL:
$\dot{y}(t) = f(t, y(t))$, wobei $f : \R \times \R \to \R$ gegeben ist.

Beispiel für ein System von DGLn:
$\dot{y}(t) = f(t, y(t))$, wobei $f : \R \times \R^n \to \R^n$ gegeben und $y : \R \to \R^n$ gesucht

% V)
Unterscheidung zwischen linearen und nicht linearen DGLn

Beispiele für lineare DGLn:
$\dot{y}(t) = \alpha y(t)$
$\dot{y}(t) = A y(t) + g(t)$, $A \in \R^{n \times n}$
$a_k(t) y^{(k)}(t) + a_{(k-1)}(t) y^{k-1}(t) + ... + a_1(t) \dot{y}(t) + a_0(t)y(t) = 0$

Beispiele für nicht lineare DGLn:
$\ddot{\phi}(t) = - \tfrac{g}{e} \sin(\phi(t))$
$(\dot{y}(t))^2 + (y(t))^2 = a$

% VI)
Unterscheidung zwischen autonomen und nicht autonomen DGLn

Beispiele für autonome DGLn:
\begin{itemize}
  \item $\dot{y} = \alpha y(t)$
  \item $(\dot{y}(t))^2 + (y(t))^2 = a$
  \item $\dot{y}(t) = f(y(t))$
  \item $F(y(t), \dot{y}(t), ..., y^{(k)}(t)) = 0$
\end{itemize}

Beispiele für nicht autonome DGLn:
\begin{itemize}
  \item $\dot{y} = \alpha y(t) + e^{t}$
  \item $(\dot{y}(t))^2 + (y(t))^2 0= a + t^2$
  \item $\dot{y}(t) = f(t, y(t))$
  \item $F(t, y(t), \dot{y})(t), ..., y^{(k)}(t)) = 0$
\end{itemize}

Unterschied: Autonome DGLn hängen nicht explizit von der Zeit $t$ ab
\fi

\begin{defn}
  Sei $\D \subset \R \times \R^n$ offen, $f : \D \to \R^n$ und $(t_0, y_0) \in \D$. Dann ist ein \emph{Anfangswertproblem} (AWP) gegeben durch die Gleichungen
  \begin{align}
    \dot{y}(t) = f(t, y(t)), \qquad
    y(t_0) = y_0. \tag{1.1}
  \end{align}
\end{defn}

\begin{nota}
  Seien im Folgenden $I$ und $J$ stets Intervalle in $\R$.
\end{nota}

\begin{defn}
  \begin{itemize}
    \item Sei $\D \subset \R \times \R^n$, $f : \D \to \R^n$. Eine differenzierbare Funktion $y : I \to \R^n$ heißt \emph{Lösung} von $\dot{y} = f(t, y)$, falls für alle $t \in I$ gilt: $\dot{y}(t) = f(t, y(t))$.
    \item Sei $\D \subset \R \times (\R^n)^k = \R \times \R^n \times ... \times \R^n$, $f : \D \to \R^n$. Eine $k$-mal differenzierbare Funktion $y : I \to \R^n$ heißt \emph{Lösung} von
    \[ y^{(k)} = f(t, y, \dot{y}, ..., y^{(k-1)}), \tag{1.2} \]
    falls für alle $t \in I$ gilt:
    $y^{(k)}(t) = f(t, y(t), \dot{y}(t), ..., y^{(k-1)}(t))$
  \end{itemize}
\end{defn}

% 1.1.
\begin{satz}
  \begin{itemize}
    \item Ist $y : I \to \R^n$ eine Lösung von (1.2), dann ist
    \[
      (y_1, ..., y_k) : I \to \R^{kn},\qquad
      t \mapsto (y(t), \dot{y}(t), ..., y^{(k-1)}(t))
    \]
    eine Lösung des Systems von Gleichungen
    \[
      (1.3) \left\{ \begin{array}{ll}
      \dot{y}_1 = y_2\\
      \dot{y}_2 = y_3\\
      \quad\enspace\vdots\\
      \dot{y}_{k-1} = y_k\\
      \dot{y}_k = f(t, y_1, y_2, ..., y_{k-1}, y_k)
      \end{array} \right.
    \]
    \item Ist umgekehrt $(y_1, ..., y_k) : I \to \R^n$ eine Lösung von (1.3), dann ist $y = y_1 : I \to \R^n$ eine Lösung von (1.2).
  \end{itemize}
\end{satz}

% 1.2.
\begin{satz}
  \begin{itemize}
    \item Ist $y : I \to \R^n$ eine Lösung von AWP (1.1), dann ist
    \[
      (y_1, y_2) : I \to \R^{n+1},\qquad
      t \mapsto (y_1(t), y_2(t)) = (t, y(t))
    \]
    eine Lösung des Anfangswertproblems
    \begin{align*}
      (1.4) \left\{ \begin{array}{ll}
        \dot{y}_1(t) = 1, & y_1(t_0) = t_0\\
        \dot{y}_2(t) = f(y_1(t), y_2(t)), & y_2(t_0) = y_0
      \end{array} \right.
    \end{align*}
    \item Ist $(y_1, y_2) : I \to \R^{n+1}$ eine Lösung von (1.4), dann ist $y = y_2 : I \to \R^n$ eine Lösung von (1.1).
  \end{itemize}
\end{satz}

% Vorlesung vom 14.4.2014

% Kapitel 1.3. Einige elementare Lösungstechniken

\begin{prob}
  Gesucht ist eine Lösung $y : I \to \R$ der linearen, skalaren, expliziten DGL 1. Ordnung (mit $a, b : I \to \R$ stetig)
  \begin{align*}
    \dot{y}(t) = a(t) \cdot y(t) + b(t) \tag{1.5}
  \end{align*}
\end{prob}

% 1.3
\begin{satz}
  Die allgemeine Lösung der Gleichung $\dot{y}(t) = a(t) \cdot y(t)$ ist gegeben durch $y_h(t) = c \cdot \exp\left(\Int{t_0}{t}{a(s)}{s}\right)$ mit $c \in \R$.
\end{satz}

% 1.4
\begin{satz}
  Sei $y_p : I \to \R$ eine partikuläre Lösung von (1.5). Dann ist die Menge aller Lösungen von (1.5) gegeben durch
  \[ \Set{ y_p + y_h }{ \text{$y_h : I \to \R$ ist Lösung von $\dot{y_h}(t) = a(t) \cdot y_h(t)$} } \]
\end{satz}

\begin{bem}
  Der Ansatz mit \emph{Variation der Konstanten} $y_p(t) = c(t) \cdot y_h(t)$ für (1.5) führt zu
  \begin{align*}
    c(t) &= \frac{1}{c_0} \Int{t_0}{t}{b(\tau) \cdot \exp\left( - \Int{t_0}{\tau}{a(s)}{s} \right)}{\tau}\\
    \Rightarrow y_p(t) &= \left( \Int{t_0}{t}{b(\tau) \cdot \exp\left( - \Int{t_0}{\tau}{a(s)}{s} \right)}{\tau} \right) \cdot \exp\left( \Int{t_0}{t}{a(s)}{s} \right)
  \end{align*}
\end{bem}

\begin{kor}
  Die Lösung des Anfangswertproblems
  \[
    (1.6) \left\{ \begin{array}{l}
      \dot{y}(t) = a(t) \cdot y(t) + b(t)\\
      y(t_0) = y_0
    \end{array} \right.
  \]
  mit $a, b : I \to \R$ stetig, $t_0 \in I$ und $y_0 \in \R$ ist gegeben durch
  \[ y(t) = \left( y_0 + \Int{t_0}{t}{b(\tau) \cdot \exp\left( - \Int{t_0}{\tau}{a(s)}{s} \right)}{\tau} \right) \cdot \exp\left( \Int{t_0}{t}{a(s)}{s} \right) \]
\end{kor}

\begin{prob}
  Ges. ist Lösung der DGL mit \emph{getrennten Variablen}
  \begin{align}
    \dot{y}(t) = g(t) \cdot h(y) \tag{1.7}
  \end{align}
  mit $g : I \to \R$ und $h : J \to \R$ stetig.
\end{prob}

\begin{lsg}
  \begin{enumerate}
    \item Fall: $h(y_0) = 0$ für ein $y_0 \in J$. Dann ist $y(t) = y_0$ eine Lsg.
    \item Fall: Es gibt kein $y_0 \in J$ mit $h(y_0) = 0$. Sei $H$ eine Stamm- funktion von $\tfrac{1}{h}$ und $G$ eine Stammfunktion von $g$. Da $h$ stetig und nirgends null ist, ist $h$ entweder strikt positiv oder strikt negativ. Somit ist $H$ streng monoton steigend/fallend und somit umkehrbar. Eine Lösung von (1.7) ist nun gegeben durch
    \[ y(t) = H^{-1}(G(t) + c_0) \quad \text{mit $c_0 \in \R$.} \]
  \end{enumerate}
\end{lsg}

\begin{prob}
  Gesucht ist Lösung des AWP mit getrennten Variablen
  \[
    (1.8) \left\{ \begin{array}{l}
      \dot{y}(t) = g(t) \cdot h(y(t))\\
      y(t_0) = y_0
    \end{array} \right.
  \]
\end{prob}

\begin{lsg}
  \begin{enumerate}
    \item Fall: $h(y_0) = 0$. Dann ist $y(t) = y_0$ eine Lösung.
    \item Fall: $h(y_0) \not= 0$. Dann ist $h$ in einer Umgebung von $y_0$ strikt positiv/negativ. Setze
    \[
      H_1(y) \coloneqq \Int{y(t_0)}{y}{\tfrac{1}{h(s)}}{s}, \quad
      G_1(t) \coloneqq \Int{t_0}{t}{g(s)}{s}.
    \]
    Dann ist $H_1$ in einer Umgebung von $y_0$ invertierbar und eine Lösung von (1.8) ist gegeben durch
    \[ y(t) = H_1^{-1}(G_1(t)). \]
  \end{enumerate}
\end{lsg}

% Vorlesung vom 16.4.2014

\begin{technik}[Transformation]
  Manchmal lässt sich eine DGL durch \emph{Substitution} eines Termes in eine einfachere DGL überführen, deren Lösung mit bekannten Methoden gefunden werden kann. Die Lösung der ursprünglichen DGL ergibt sich durch Rücksubstitution.
\end{technik}

\begin{bsp}
  Gegeben sei die DGL $\dot{y} = f(\alpha t + \beta y + \gamma)$ mit $\alpha, \beta, \gamma \in \R$, $\beta \not= 0$ und $f : \R \to \R$ stetig. Substituiere $z(t) = \alpha t + \beta y(t) + \gamma$. Es ergibt sich die neue DGL $\dot{z}(t) = \beta f(z(t)) + \alpha$, die sich durch Trennung der Variablen lösen lässt.
\end{bsp}

\begin{bsp}[Bernoulli-DGL]
  Gegeben sei die DGL $\dot{y}(t) = \alpha(t) \cdot y(t) + \beta(t) \cdot (y(t))^{\delta}$ mit $\alpha, \beta : I \to \R$ stetig und $\delta \in \R \setminus \{0,1\}$. Multiplikation mit $(1-\delta) y^{-\delta}$ und Substitution mit $z(t) = (y(t))^{1-\delta}$ führt zur skalaren linearen DGL 1. Ordnung
  \[ \dot{z}(t) = (1-\delta) \alpha(t) z(t) + (1-\delta) \beta(t). \]
\end{bsp}

% Kapitel 1.4. Richtungsfelder

% Ausgelassen

% Kapitel 2. Existenz- und Eindeutigkeitssätze

% Ausgelassen: Betragsfunktionsbeispiel

% Vorlesung vom 23.4.2014

\begin{defn}
  Sei $\mathcal{D} \subset \R^n$. Eine Abb. $f : \mathcal{D} {\to} \R^n$ heißt \emph{stetig in $x_0 \in \mathcal{D}$}, falls
  \[ \fa{\epsilon {>} 0} \ex{\delta {>} 0} \fa{x \in \mathcal{D}} \norm{x - x_0} < \delta \implies \norm{f(x) - f(x_0)} < \epsilon. \]
  Die Abb. heißt \emph{stetig} in $\mathcal{D}$, falls sie in jedem Punkt in $\mathcal{D}$ stetig ist.
\end{defn}

\begin{nota}
  $\mathcal{C}(I, \R^n) \coloneqq \Set{ f : I \to \R^n }{ \text{$f$ stetig} }$, $\norm{f}_\infty \coloneqq \sup_{t \in I} \norm{f(t)}$
\end{nota}

% Ausgelassen: Banachraum = vollständiger normierter Vektorraum

\begin{bem}
  $(\mathcal{C}(I, \R^n), \norm{\blank}_\infty)$ ist ein Banachraum.
\end{bem}

\begin{defn}
  Eine Teilmenge $A \subset X$ eines topologischen Raumes $X$ heißt \emph{relativ kompakt}, wenn ihr Abschluss $\overline{A}$ kompakt in $X$ ist.
\end{defn}

\begin{defn}
  Seien $(X, \norm{\blank}_X)$ und $(Y, \norm{\blank}_Y)$ Banachräume. Sei $\mathcal{D} \subset X$. Eine Abbildung $T : \mathcal{D} \to Y$ heißt
  \begin{itemize}
    \item \emph{stetig in $x \in \mathcal{D}$}, falls $T x_n \xrightarrow{n \to \infty} T x$ in $Y$ für jede Folge $(x_n)_{n \in \N}$ mit $x_n \xrightarrow{n \to \infty}$ in $\mathcal{D}$ gilt.
    \item \emph{Lipschitz-stetig} in $\mathcal{D}$, falls eine Konstante $\alpha > 0$ existiert mit
    \[ \fa{x_1, x_2 \in \mathcal{D}} \norm{Tx_1 - Tx_2}_Y \leq \alpha \cdot \norm{x_1 - x_2}_X. \]
    \item \emph{kontraktiv}, falls $T$ Lipschitz-stetig mit $\alpha < 1$ ist.
    \item \emph{kompakt}, falls $T$ stetig ist und beschränkte Mengen in $X$ auf relativ kompakte Mengen in $Y$ abgebildet werden, d.\,h. für jede beschränkte Folge $(x_n)_{n \in \N}$ in $\mathcal{D}$ besitzt die Folge $(Tx_n)_{n \in \N}$ eine konvergente Teilfolge.
  \end{itemize}
\end{defn}

\begin{bem}
  Lipschitz-stetige Funktionen sind stetig, die Umkehrung gilt aber nicht.
\end{bem}
% Gegenbeispiel dafür: $f(t) = \sqrt{t}$

% Ausgelassen: Beispiel $T : \mathcal{C}(I, \R) \to \mathcal{C}(I, \R)$

\begin{satz}[Arzelà-Ascoli]
  Sei $I \subset \R$ kompakt. Eine Teilmenge $\mathcal{F} \subset \mathcal{C}(I, \R^n)$ ist genau dann relativ kompakt, wenn
  \begin{itemize}
    \item $\mathcal{F}$ ist \emph{punktweise beschränkt}, d.\,h.
    \[ \fa{t \in I} \ex{M} \fa{f \in \mathcal{F}} \norm{f(t)} \leq M \]
    \item $\mathcal{F}$ ist \emph{gleichgradig stetig}, d.\,h.
    \[ \fa{\epsilon {>} 0} \ex{\delta {>} 0} \fa{t_1, t_2 {\in} I, f {\in} \mathcal{F}} \norm{t_1 - t_2} < \delta \Rightarrow \norm{f(t_1) - f(t_2)} < \epsilon \]
  \end{itemize}
\end{satz}

\begin{satz}[Fixpunktsatz von Banach]
  Sei $(X, \norm{\blank}_X)$ ein Banachraum, $\mathcal{D} \subset X$ nichtleer, abgeschlossen. Sei $T : \mathcal{D} \to \mathcal{D}$ eine Kontraktion. Dann besitzt die Fixpunktgleichung $y = Ty$ genau eine Lösung in $D$.
\end{satz}

\begin{satz}[Fixpunktsatz von Schauder]
  Sei $(X, \norm{\blank}_X)$ ein Banachraum, sei $\mathcal{D} \subset X$ nichtleer, abgeschlossen, beschränkt, konvex. Sei $T : \mathcal{D} \to \mathcal{D}$ eine kompakte Abbildung. Dann besitzt die Fixpunktgleichung $y = Ty$ mindestens eine Lösung in $\mathcal{D}$.
\end{satz}

% Kapitel 2.1. Existenzsatz von Peano

\begin{satz}
  Sei $\mathcal{D} \subset \R \times \R^n$ offen, $f : \mathcal{D} \to \R^n$ stetig, $(t_0, y_0) \in D$. Dann ist das AWP (1.1) lokal lösbar, d.\,h. es existiert ein Intervall $I \subset \R$ mit $t_0 \in I$ und eine stetig diff'bare Funktion $y : I \to \R^n$, die das AWP (1.1) erfüllt.
\end{satz}

\end{document}