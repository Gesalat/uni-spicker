\documentclass{cheat-sheet}

\pdfinfo{
  /Title (Zusammenfassung Gewöhnliche Differentialgleichungen)
  /Author (Tim Baumann)
}

% Kleinere Klammern
\delimiterfactor=701

\newcommand{\D}{\mathcal{D}}
\newcommand{\dist}{\mathrm{dist}} % Entfernung (distance)

\begin{document}

\maketitle{Zusammenfassung Gewöhnliche DGLn}

% Kapitel 1. Einführung

% Kapitel 1.1. Beispiele

\iffalse
\begin{bsp}
  Gesucht: Funktion $y : \R \to \R$ mit $\fa{t \in \R} \dot{y}(t) = y(t)$\\
\end{bsp}

\begin{lsg}
  $y(t) = c \cdot e^t$ für $c \in \R$ beliebig. Wenn man als Anfangsbe- dingung $y(0) = 1$ fordert, erhält man eine eindeutige Lösung ($c = 1$).
\end{lsg}

\begin{bsp}
  Gesucht: Lösung von $\left(\dot{y}(t)\right)^2 + \left(y(t)\right)^2 = a$ für $a \in \R$\\
\end{bsp}

\begin{lsg}
  Anzahl der Lösungen hängt von $a$ ab:
  \begin{itemize}
    \miniitem{0.47\linewidth}{Falls $a < 0$: keine reelle Lsg}
    \miniitem{0.51\linewidth}{Falls $a = 0$: Einzige Lsg $y(t) = 0$}
    \item Falls $a > 0$: Lsgn: $y(t) = \sqrt{a} \cos(t + \phi)$ für $\phi \in \R$ bel., $y(t) = \pm \sqrt{a}$
  \end{itemize}
\end{lsg}

% Populationsmodell aus der Biologie/Chemie
\begin{bsp}
  Sei $p(t)$ ist Populationsgröße zur Zeit $t$.
  Angenommen, $\tfrac{\dot{p}(t)}{p(t)} = a$ ist konstant, also $\dot{p}(t) = p(t)$. Sei $p(t_0) = p_0$.
\end{bsp}

\begin{lsg}
  $p(t) = p_0 e^{(t-t_0) a}$
\end{lsg}

% 1837
\begin{bsp}[Verhulst-Modell]
  Gesucht: Lösung zu
  \[ \dot{p}(t) = a_0 p(t) - a_1 \left(p(t)\right)^2 \]
\end{bsp}

\begin{lsg}
  $p(t) = \frac{a_0}{a_1 (1 - c e^{-a_0 t})}$
\end{lsg}

% Mechanik: Mathematisches Pendel
% Skizze: Pendel der
% * Masse $m$
% * Länge $l$
% * Auslenkungswinkel $\phi(t)$ zum Zeitpunkt $t$
% * Position $l \phi(t)$ zum Zeitpunkt $t$
% * Geschwindigkeit $v(t) = \dot{p}(t) = l \dot{\phi}(t)$
% * $a(t) = \ddot{p}(t) = \dot{v}(t) = l \ddot{\phi}(t)$ Beschleunigung
% $F_E = - mg$, $F = ma$
% $F_T$ -- tangentiale Komponente der Gewichtskraft
% $F_T = - mg \sin(\phi)$
% => Gleichung $-mg \sin(\phi) = m a(t) = m l \ddot{t}$, also $\ddot{\phi}(t) = -\tfrac{g}{l} \sin(\phi(t))$
\fi

% Vorlesung vom 9.4.2014

% Literatur
% Alle Bücher haben den Titel "`Gewöhnliche Differentialgleichungen"'
% * B. Aulbach, 2004
% * H. Henser, 2009
% * L. Grüne, O. Junge, 2009
% * W. Walter, 2000

% Kapitel 1.2. Klassifikation von Differentialgleichungen (DGLn)

\begin{defn}[Klassifikation von DGLn]\mbox{}\\
  \begin{enumerate}[label=(\Roman*),leftmargin=2em]
    \item \emph{Gewöhnliche} DGL: Gesucht ist Funktion in einer Variable\\
    \emph{Partielle} DGL: Gesucht ist Funktion in mehreren Variablen
    \item \emph{Ordnung} einer DGL: Höchste Ableitung der gesuchten Funktion, die in Gleichung vorkommt
    \item \emph{Explizite} DGL: Gleichung der Form
    $y^{(k)} {=} f(t, y, \dot{y}, ..., y^{(k{-}1)})$
    \emph{Implizite} DGL: Allgemeinere Form $F(t, y, \dot{y}, ..., y^{(k)}) = 0$
    \item \emph{Skalare} DGL: Gesucht ist Funktion mit Wert in $\R$\\
    \emph{$n$-dimensionale} DGL: Gesuchte Funktion hat Wert in $\R^n$
    \item \emph{Lineare} DGL: Gleichung hat die Form
    \[ a_k(t) y^{(k)}(t) + a_{(k-1)}(t) y^{k-1}(t) + ... + a_1(t) \dot{y}(t) + a_0(t)y(t) = 0 \]
    \item \emph{Autonome} DGL: Gleichung der Form $F(y, \dot{y}, ..., y^{(k)}) = 0$\\
    (keine Abhängigkeit von $t$, Zeitinvarianz)
  \end{enumerate}
\end{defn}

\iffalse
% I)

Unterscheidung zwischen gewöhnliche DGL und partielle DGLn

Beispiele für gewöhnliche DGL
$\dot{y}(t) = h y(t)$
$(\dot{y}(t))^2 + (y(t))^2 = a$

Beispiele für partielle DGLn:

$y_t = \alpha y_{xx} + y$, wobei $y_t(t,x) = \tfrac{\partial}{\partial t} y(t, x)$, $y_{xx}(t, x) = \tfrac{\partial^2}{\partial x^2} y(t, x)$

% II)

Unterscheidung zwischen DGLn 1. Ordnung, DGLn 2. Ordnung und DGLn $k$-ter Ordnung

Beispiel für DGL 1. Ordnung:
$\dot{y} = \alpha y(t)$

Beispiel für DGL 2. Ordnung:
$\ddot{\phi}(t) = - \tfrac{\delta}{e} \sin(\phi(t))$

Beispiel für DGL $k$-ter Ordnung:
$F(t, y(t), \dot{y}(t), ..., y^{(k)}(t)) = 0$

% III)

Unterscheidung zischen expliziten und impliziten DGLn

Beispiel für explizite DGLn:
$\dot{y}(t) = \alpha y(t)$
$\ddot{\phi}(t) = - \tfrac{g}{e} \sin(\phi(t))$
$y^{(k)}(t) = f(t, y, \dot{y}, ..., y^{(k-1)})$

Beispiele für implizite DGLn:
$(\dot{y}(t))^2 + (y(t))^2 = a$
$F(t, y, \dot{y}, ..., y^{(k)}(t)) = 0$

Oder (Gleichungen gehören zusammen)
$\dot{y}_2(t) + y_1(t) = f_1(t)$
$y_2(t) = f_2(t)$
(differentiell-algebraische Gleichung)

% IV)
Unterscheidung zwischen Skalaren DGLn und $n$-dimensionalen DGLn (Systeme von DGLn)

Beispiel für Skalare DGL:
$\dot{y}(t) = f(t, y(t))$, wobei $f : \R \times \R \to \R$ gegeben ist.

Beispiel für ein System von DGLn:
$\dot{y}(t) = f(t, y(t))$, wobei $f : \R \times \R^n \to \R^n$ gegeben und $y : \R \to \R^n$ gesucht

% V)
Unterscheidung zwischen linearen und nicht linearen DGLn

Beispiele für lineare DGLn:
$\dot{y}(t) = \alpha y(t)$
$\dot{y}(t) = A y(t) + g(t)$, $A \in \R^{n \times n}$
$a_k(t) y^{(k)}(t) + a_{(k-1)}(t) y^{k-1}(t) + ... + a_1(t) \dot{y}(t) + a_0(t)y(t) = 0$

Beispiele für nicht lineare DGLn:
$\ddot{\phi}(t) = - \tfrac{g}{e} \sin(\phi(t))$
$(\dot{y}(t))^2 + (y(t))^2 = a$

% VI)
Unterscheidung zwischen autonomen und nicht autonomen DGLn

Beispiele für autonome DGLn:
\begin{itemize}
  \item $\dot{y} = \alpha y(t)$
  \item $(\dot{y}(t))^2 + (y(t))^2 = a$
  \item $\dot{y}(t) = f(y(t))$
  \item $F(y(t), \dot{y}(t), ..., y^{(k)}(t)) = 0$
\end{itemize}

Beispiele für nicht autonome DGLn:
\begin{itemize}
  \item $\dot{y} = \alpha y(t) + e^{t}$
  \item $(\dot{y}(t))^2 + (y(t))^2 0= a + t^2$
  \item $\dot{y}(t) = f(t, y(t))$
  \item $F(t, y(t), \dot{y})(t), ..., y^{(k)}(t)) = 0$
\end{itemize}

Unterschied: Autonome DGLn hängen nicht explizit von der Zeit $t$ ab
\fi

\begin{defn}
  Sei $\D \subset \R \times \R^n$ offen, $f : \D \to \R^n$ und $(t_0, y_0) \in \D$. Dann ist ein \emph{Anfangswertproblem} (AWP) gegeben durch die Gleichungen
  \[
    (1.1) \left\{ \begin{array}{ll}
      \dot{y}(t) = f(t, y(t)), \\
      y(t_0) = y_0.
    \end{array} \right.
  \]
\end{defn}

\begin{nota}
  Seien im Folgenden $I$ und $J$ stets Intervalle in $\R$.
\end{nota}

\begin{defn}
  \begin{itemize}
    \item Sei $\D \subset \R \times \R^n$, $f : \D \to \R^n$. Eine differenzierbare Funktion $y : I \to \R^n$ heißt \emph{Lösung} von $\dot{y} = f(t, y)$, falls für alle $t \in I$ gilt: $\dot{y}(t) = f(t, y(t))$.
    \item Sei $\D \subset \R \times (\R^n)^k = \R \times \R^n \times ... \times \R^n$, $f : \D \to \R^n$. Eine $k$-mal differenzierbare Funktion $y : I \to \R^n$ heißt \emph{Lösung} von
    \[ y^{(k)} = f(t, y, \dot{y}, ..., y^{(k-1)}), \tag{1.2} \]
    falls für alle $t \in I$ gilt:
    $y^{(k)}(t) = f(t, y(t), \dot{y}(t), ..., y^{(k-1)}(t))$
  \end{itemize}
\end{defn}

% 1.1.
\begin{satz}
  \begin{itemize}
    \item Ist $y : I \to \R^n$ eine Lösung von (1.2), dann ist
    \[
      (y_1, ..., y_k) : I \to \R^{kn},\qquad
      t \mapsto (y(t), \dot{y}(t), ..., y^{(k-1)}(t))
    \]
    eine Lösung des Systems von Gleichungen
    \[
      (1.3) \left\{ \begin{array}{ll}
      \dot{y}_1 = y_2\\
      \dot{y}_2 = y_3\\
      \quad\enspace\vdots\\
      \dot{y}_{k-1} = y_k\\
      \dot{y}_k = f(t, y_1, y_2, ..., y_{k-1}, y_k)
      \end{array} \right.
    \]
    \item Ist umgekehrt $(y_1, ..., y_k) : I \to \R^n$ eine Lösung von (1.3), dann ist $y = y_1 : I \to \R^n$ eine Lösung von (1.2).
  \end{itemize}
\end{satz}

% 1.2.
\begin{satz}
  \begin{itemize}
    \item Ist $y : I \to \R^n$ eine Lösung von AWP (1.1), dann ist
    \[
      (y_1, y_2) : I \to \R^{n+1},\qquad
      t \mapsto (y_1(t), y_2(t)) = (t, y(t))
    \]
    eine Lösung des Anfangswertproblems
    \begin{align*}
      (1.4) \left\{ \begin{array}{ll}
        \dot{y}_1(t) = 1, & y_1(t_0) = t_0\\
        \dot{y}_2(t) = f(y_1(t), y_2(t)), & y_2(t_0) = y_0
      \end{array} \right.
    \end{align*}
    \item Ist $(y_1, y_2) : I \to \R^{n+1}$ eine Lösung von (1.4), dann ist $y = y_2 : I \to \R^n$ eine Lösung von (1.1).
  \end{itemize}
\end{satz}

% Vorlesung vom 14.4.2014

% Kapitel 1.3. Einige elementare Lösungstechniken

\begin{prob}
  Gesucht ist eine Lösung $y : I \to \R$ der linearen, skalaren, expliziten DGL 1. Ordnung (mit $a, b : I \to \R$ stetig)
  \begin{align*}
    \dot{y}(t) = a(t) \cdot y(t) + b(t) \tag{1.5}
  \end{align*}
\end{prob}

% 1.3
\begin{satz}
  Die allgemeine Lösung der Gleichung $\dot{y}(t) = a(t) \cdot y(t)$ ist gegeben durch $y_h(t) = c \cdot \exp\left(\Int{t_0}{t}{a(s)}{s}\right)$ mit $c \in \R$.
\end{satz}

% 1.4
\begin{satz}
  Sei $y_p : I \to \R$ eine partikuläre Lösung von (1.5). Dann ist die Menge aller Lösungen von (1.5) gegeben durch
  \[ \Set{ y_p + y_h }{ \text{$y_h : I \to \R$ ist Lösung von $\dot{y_h}(t) = a(t) \cdot y_h(t)$} } \]
\end{satz}

\begin{bem}
  Der Ansatz mit \emph{Variation der Konstanten} $y_p(t) = c(t) \cdot y_h(t)$ für (1.5) führt zu
  \begin{align*}
    c(t) &= \frac{1}{c_0} \Int{t_0}{t}{b(\tau) \cdot \exp\left( - \Int{t_0}{\tau}{a(s)}{s} \right)}{\tau}\\
    \Rightarrow y_p(t) &= \left( \Int{t_0}{t}{b(\tau) \cdot \exp\left( - \Int{t_0}{\tau}{a(s)}{s} \right)}{\tau} \right) \cdot \exp\left( \Int{t_0}{t}{a(s)}{s} \right)
  \end{align*}
\end{bem}

\begin{kor}
  Die Lösung des Anfangswertproblems
  \[
    (1.6) \left\{ \begin{array}{l}
      \dot{y}(t) = a(t) \cdot y(t) + b(t)\\
      y(t_0) = y_0
    \end{array} \right.
  \]
  mit $a, b : I \to \R$ stetig, $t_0 \in I$ und $y_0 \in \R$ ist gegeben durch
  \[ y(t) = \left( y_0 + \Int{t_0}{t}{b(\tau) \cdot \exp\left( - \Int{t_0}{\tau}{a(s)}{s} \right)}{\tau} \right) \cdot \exp\left( \Int{t_0}{t}{a(s)}{s} \right) \]
\end{kor}

\begin{prob}
  Ges. ist Lösung der DGL mit \emph{getrennten Variablen}
  \begin{align}
    \dot{y}(t) = g(t) \cdot h(y) \tag{1.7}
  \end{align}
  mit $g : I \to \R$ und $h : J \to \R$ stetig.
\end{prob}

\begin{lsg}
  \begin{enumerate}
    \item Fall: $h(y_0) = 0$ für ein $y_0 \in J$. Dann ist $y(t) = y_0$ eine Lsg.
    \item Fall: Es gibt kein $y_0 \in J$ mit $h(y_0) = 0$. Sei $H$ eine Stamm- funktion von $\tfrac{1}{h}$ und $G$ eine Stammfunktion von $g$. Da $h$ stetig und nirgends null ist, ist $h$ entweder strikt positiv oder strikt negativ. Somit ist $H$ streng monoton steigend/fallend und somit umkehrbar. Eine Lösung von (1.7) ist nun gegeben durch
    \[ y(t) = H^{-1}(G(t) + c_0) \quad \text{mit $c_0 \in \R$.} \]
  \end{enumerate}
\end{lsg}

\begin{prob}
  Gesucht ist Lösung des AWP mit getrennten Variablen
  \[
    (1.8) \left\{ \begin{array}{l}
      \dot{y}(t) = g(t) \cdot h(y(t))\\
      y(t_0) = y_0
    \end{array} \right.
  \]
\end{prob}

\begin{lsg}
  \begin{enumerate}
    \item Fall: $h(y_0) = 0$. Dann ist $y(t) = y_0$ eine Lösung.
    \item Fall: $h(y_0) \not= 0$. Dann ist $h$ in einer Umgebung von $y_0$ strikt positiv/negativ. Setze
    \[
      H_1(y) \coloneqq \Int{y(t_0)}{y}{\tfrac{1}{h(s)}}{s}, \quad
      G_1(t) \coloneqq \Int{t_0}{t}{g(s)}{s}.
    \]
    Dann ist $H_1$ in einer Umgebung von $y_0$ invertierbar und eine Lösung von (1.8) ist gegeben durch
    \[ y(t) = H_1^{-1}(G_1(t)). \]
  \end{enumerate}
\end{lsg}

% Vorlesung vom 16.4.2014

\begin{technik}[Transformation]
  Manchmal lässt sich eine DGL durch \emph{Substitution} eines Termes in eine einfachere DGL überführen, deren Lösung mit bekannten Methoden gefunden werden kann. Die Lösung der ursprünglichen DGL ergibt sich durch Rücksubstitution.
\end{technik}

\begin{bsp}
  Gegeben sei die DGL $\dot{y} = f(\alpha t + \beta y + \gamma)$ mit $\alpha, \beta, \gamma \in \R$, $\beta \not= 0$ und $f : \R \to \R$ stetig. Substituiere $z(t) = \alpha t + \beta y(t) + \gamma$. Es ergibt sich die neue DGL $\dot{z}(t) = \beta f(z(t)) + \alpha$, die sich durch Trennung der Variablen lösen lässt.
\end{bsp}

\begin{bsp}[Bernoulli-DGL]
  Gegeben sei die DGL $\dot{y}(t) = \alpha(t) \cdot y(t) + \beta(t) \cdot (y(t))^{\delta}$ mit $\alpha, \beta : I \to \R$ stetig und $\delta \in \R \setminus \{0,1\}$. Multiplikation mit $(1-\delta) y^{-\delta}$ und Substitution mit $z(t) = (y(t))^{1-\delta}$ führt zur skalaren linearen DGL 1. Ordnung
  \[ \dot{z}(t) = (1-\delta) \alpha(t) z(t) + (1-\delta) \beta(t). \]
\end{bsp}

% Kapitel 1.4. Richtungsfelder

% Ausgelassen

% Kapitel 2. Existenz- und Eindeutigkeitssätze

% Ausgelassen: Betragsfunktionsbeispiel

% Vorlesung vom 23.4.2014

\begin{defn}
  Sei $\D \subset \R^n$. Eine Abb. $f : \D {\to} \R^n$ heißt \emph{stetig in $x_0 \in \D$}, falls
  \[ \fa{\epsilon {>} 0} \ex{\delta {>} 0} \fa{x \in \D} \norm{x - x_0} < \delta \implies \norm{f(x) - f(x_0)} < \epsilon. \]
  Die Abb. heißt \emph{stetig} in $\D$, falls sie in jedem Punkt in $\D$ stetig ist.
\end{defn}

\begin{nota}
  $\mathcal{C}(I, \R^n) \coloneqq \Set{ f : I \to \R^n }{ \text{$f$ stetig} }$, $\norm{f}_\infty \coloneqq \sup_{t \in I} \norm{f(t)}$
\end{nota}

% Ausgelassen: Banachraum = vollständiger normierter Vektorraum

\begin{bem}
  $(\mathcal{C}(I, \R^n), \norm{\blank}_\infty)$ ist ein Banachraum.
\end{bem}

\begin{defn}
  Eine Teilmenge $A \subset X$ eines topologischen Raumes $X$ heißt \emph{relativ kompakt}, wenn ihr Abschluss $\overline{A}$ kompakt in $X$ ist.
\end{defn}

\begin{defn}
  Seien $(X, \norm{\blank}_X)$ und $(Y, \norm{\blank}_Y)$ Banachräume. Sei $\D \subset X$. Eine Abbildung $T : \D \to Y$ heißt
  \begin{itemize}
    \item \emph{stetig in $x \in \D$}, falls $T x_n \xrightarrow{n \to \infty} T x$ in $Y$ für jede Folge $(x_n)_{n \in \N}$ mit $x_n \xrightarrow{n \to \infty} x$ in $\D$ gilt.
    \item \emph{Lipschitz-stetig} in $\D$, falls eine Konstante $\alpha > 0$ existiert mit
    \[ \fa{x_1, x_2 \in \D} \norm{Tx_1 - Tx_2}_Y \leq \alpha \cdot \norm{x_1 - x_2}_X. \]
    \item \emph{kontraktiv}, falls $T$ Lipschitz-stetig mit $\alpha < 1$ ist.
    \item \emph{kompakt}, falls $T$ stetig ist und beschränkte Mengen in $X$ auf relativ kompakte Mengen in $Y$ abgebildet werden, d.\,h. für jede beschränkte Folge $(x_n)_{n \in \N}$ in $\D$ besitzt die Folge $(Tx_n)_{n \in \N}$ eine konvergente Teilfolge.
  \end{itemize}
\end{defn}

\begin{bem}
  Lipschitz-stetige Funktionen sind stetig, die Umkehrung gilt aber nicht.
\end{bem}
% Gegenbeispiel dafür: $f(t) = \sqrt{t}$

% Ausgelassen: Beispiel $T : \mathcal{C}(I, \R) \to \mathcal{C}(I, \R)$

\begin{satz}[Arzelà-Ascoli]
  Sei $I \subset \R$ kompakt. Eine Teilmenge $\mathcal{F} \subset \mathcal{C}(I, \R^n)$ ist genau dann relativ kompakt, wenn
  \begin{itemize}
    \item $\mathcal{F}$ ist \emph{punktweise beschränkt}, d.\,h.
    \[ \fa{t \in I} \ex{M} \fa{f \in \mathcal{F}} \norm{f(t)} \leq M \]
    \item $\mathcal{F}$ ist \emph{gleichgradig stetig}, d.\,h.
    \[ \fa{\epsilon {>} 0} \ex{\delta {>} 0} \fa{t_1, t_2 {\in} I, f {\in} \mathcal{F}} \norm{t_1 - t_2} < \delta \Rightarrow \norm{f(t_1) - f(t_2)} < \epsilon \]
  \end{itemize}
\end{satz}

\begin{satz}[Fixpunktsatz von Banach]
  Sei $(X, \norm{\blank}_X)$ ein Banachraum, $\D \subset X$ nichtleer, abgeschlossen. Sei $T : \D \to \D$ eine Kontraktion. Dann besitzt die Fixpunktgleichung $y = Ty$ genau eine Lösung in $D$.
\end{satz}

\begin{satz}[Fixpunktsatz von Schauder]
  Sei $(X, \norm{\blank}_X)$ ein Banachraum, sei $\D \subset X$ nichtleer, abgeschlossen, beschränkt, konvex. Sei $T : \D \to \D$ eine kompakte Abbildung. Dann besitzt die Fixpunktgleichung $y = Ty$ mindestens eine Lösung in $\D$.
\end{satz}

% Kapitel 2.1. Existenzsatz von Peano

\begin{satz}
  Sei $\D \subset \R \times \R^n$ offen, $f : \D \to \R^n$ stetig, $(t_0, y_0) \in D$. Dann ist das AWP (1.1) lokal lösbar, d.\,h. es existiert ein Intervall $I \subset \R$ mit $t_0 \in I$ und eine stetig diff'bare Funktion $y : I \to \R^n$, die das AWP (1.1) erfüllt.
\end{satz}

% Vorlesung vom 28.4.2014

% Ausgelassen: Beispiel Differentialgleichung mit nicht eindeutiger Lösung

\begin{defn}
  Sei $\D \subset \R \times \R^N$ offen, $f : \D \to \R^N$, $(t_0, y_0) \in \D$. Sei $y : I \to \R^N$ eine Lösung des AWP (1.1).
  \begin{itemize}
    \item Eine Lösung $u : J \to \R^N$ heißt \emph{Fortsetzung} (bzw. \emph{echte Fortsetzung}) der Lösung $y$, falls $I \subset J$ (bzw. $I \subsetneq J$) und $u|_I = y$.
    \item Die Lösung $y$ heißt \emph{maximale Lösung} des AWP (1.1), falls keine echte Fortsetzung von $y$ existiert. Das Intervall $I$ heißt dann \emph{maximales Existenzintervall}.
  \end{itemize}
\end{defn}

% Ausgelassen: Formulierung des Zornschen Lemmas

% 2.5.
\begin{satz}
  Sei $\D \subset \R \times \R^N$ offen, $f : \D \to \R^N$ stetig und $(t_0, y_0) \in \D$.
  \begin{itemize}
    \item Jede lokale Lösung des AWP (1.1) kann zu einer maximalen Lösung fortgesetzt werden.
    \item Sei $y : I \to \R^N$ eine max. Lsg. des AWP (1.1). Dann ist $I$ offen.
  \end{itemize}
\end{satz}

% Kapitel 2.2. Eindeutigkeitssätze

\begin{defn}
  Sei $\D \subset \R \times \R^N$ und $f : \D \to \R^n$.
  \begin{itemize}
    \item Die Funktion $f $ heißt \emph{Lipschitz-stetig bzgl. $y$} auf $\D$, falls eine Konstante $\mathcal{L} > 0$ existiert, sodass
    \[ \fa{(t, y_1), (t, y_2) \in \D} \norm{f(t, y_1) - f(t, y_2)} \leq \mathcal{L} \cdot \norm{y_1 - y_2}. \]
    \item Wenn für alle $(t, y) \in \D$ eine Umgebung $U_{(t, y)} \subset \D$ existiert, sodass $f|_{U_{(t,y)}}$ Lipschitz-stetig bzgl. $y$ ist, so heißt $f$ \emph{lokal Lipschitz-stetig bzgl. $y$} auf $\D$.
  \end{itemize}
\end{defn}

\begin{lem}
  Sei $\D \subset \R \times \R^N$, $f : \D \to \R^N$ stetig und stetig diff'bar nach $y$ in $\D$. Dann ist $f$ lokal Lipschitz-stetig bzgl. $y$.
\end{lem}

% Vorlesung vom 30.4.2014

\begin{satz}[Picard-Lindelöf, lokal quantitativ]
  Seien $\D \subset \R \times \R^n$ offen, $f : \D \to \R^n$ stetig, $(t_0, y_0) {\in} \D$ und $R_{a,b} {\coloneqq} \cinterval{ t_0 {-} a }{ t_0 {+} a } \times \overline{B_b}(y_0) \subset \D$. Sei $f$ Lipschitz-stetig bzgl. $y$ auf $R_{a,b}$. Dann besitzt das AWP (1.1) im Rechteck $R_{a,b}$ genau eine Lösung $y : I_y \to \R^n$ auf $I_{\gamma} \coloneqq \cinterval{t_0 - y}{t_0 + y}$ mit $\gamma = \min(a, \tfrac{b}{M})$ und $M = \sup_{\mathclap{(t,y) \in R_{a,b}}} \norm{f(t, y)}$.
\end{satz}

\begin{bem}
  Im Beweis des Satzes definiert man die Picard-Iterierten $u_j : I_\gamma \to \R^n$ für $j \in \N$ durch
  \[
    u_0(t) \coloneqq y_0, \qquad
    u_{j+1}(t) \coloneqq y_0 + \Int{t_0}{t}{f(\tau, y_j(\tau))}{\tau}.
  \]
  Man zeigt: Die Funktionenfolge $(u_j)_{j \in \N}$ konvergiert gleichmäßig gegen eine Lösung $u_\infty : I_\gamma \to \R^n$ des AWP.
\end{bem}

% Vorlesung vom 5.5.2014

\begin{satz}[Picard-Lindelöf, lokal qualitativ]
  Seien $\D \subset \R \times \R^n$ offen, $f : \D \to \R^n$ stetig, lokal Lipschitz-stetig bzgl. $y$ auf $\D$, $(t_0, y_0) \in \D$. Dann besitzt das AWP (1.1) eine eindeutige lokale Lösung, d.\,h. es existiert $\gamma = \gamma(t_0, y_0) > 0$, sodass das AWP (1.1) auf $I_\gamma \coloneqq \cinterval{t_0 - \gamma}{t_0 + \gamma}$ genau eine Lösung besitzt.
\end{satz}

\begin{satz}[Picard-Lindelöf, global]
  Seien $\D \subset \R \times \R^n$ offen, $f : \D \to \R^n$ stetig, lokal Lipschitz-stetig bzgl. $y$ auf $\D$, $(t_0, y_0) \in \D$. Dann gibt es ein eindeutig bestimmtes offenes Intervall $I = \ointerval{a_-}{a_+}$ mit $t_0 {\in} I$ und
  \begin{itemize}
    \item Das AWP (1.1) besitzt genau eine Lösung $\gamma$ auf $I$.
    \item Ist $\tilde{z} : \tilde{I} \to \R^n$ eine beliebige Lösung von (1.1) mit $t_0 \in \tilde{I}$, so gilt $\tilde{I} \subseteq I$ und $z = y|_{\tilde{I}}$.
  \end{itemize}
\end{satz}

% Vorlesung vom 7.5.2014

% 2.9.
\begin{satz}
  Sei $I \subset \R$ ein offenes Intervall, $f : I \times \R^n \to \R^n$ stetig, $(t_0, y_0) \in I \times \R^n$. Falls $f$ für jedes kompaktes Intervall $\tilde{I} \subset I$ global Lipschitz-stetig bzgl. $y$ auf $\tilde{I} \times \R^n$ ist, dann hat das AWP (1.1) genau eine globale Lösung $y$ auf $I$.
\end{satz}

% 2.10.
\begin{satz}
  Sei $I \subset \R$ offen, $f : I \times \R^n \to \R^n$ stetig, lokal Lipschitz-stetig bzgl. $y$ auf $I \times \R^n$, $(t_0, y_0) \in I \times \R^n$. Ist das Wachstum von $f$ linear beschränkt in $y$ auf $I \times \R^n$, d.\,h. gibt es stetige Funktionen $a, b : I \to \cointerval{0}{\infty}$ mit $\norm{f(t, y)} \leq a(t) \norm{y} + b(t)$ für alle $(t, y) \in I \times \R^n$, dann besitzt das AWP (1.1) eine eindeutige Lösung auf $I$.
\end{satz}

% Vorlesung vom 12.5.2014

% Ausgelassen: Beispiele zum Randverhalten

\begin{satz}[Randverhalten maximaler Lösungen]
  Sei $\D \subset \R \times \R^n$ offen, $(t_0, y_0) \in \D$ und $f : \D \to \R^n$ stetig und lokal Lipschitz-stetig bzgl. $y$. Sei $y : \ointerval{a_-}{a_+} \to \R^n$ eine maximale Lösung des AWP (1.1).
  \begin{itemize}
    \item Ist $a_+ < \infty$, so ist $y$ auf $\cointerval{t_0}{a_+}$ unbeschränkt
    oder der Rand $\partial \D$ ist nicht
    leer und $\lim_{t \uparrow a_+} \dist((t, y(t)), \partial \D) = 0$.
    \item Ist $a_- > -\infty$, so ist $y$ auf $\ocinterval{a_-}{t_0}$ unbeschränkt oder der Rand $\partial \D$ ist nicht leer und $\lim_{t \downarrow a_-} \dist((t, y(t)), \partial \D) = 0$.
  \end{itemize}
\end{satz}

% Beispiel: Graph nähert sich an Rand an, allerdings muss er nicht gegen einen Randpunkt konvergieren

% Kapitel 3. Lineare Differentialgleichungen

\begin{prob}
  Sei $I \subset \R$ und $A : I \to \R^{n \times n}$, $f : I \to \R^n$ stetig. Für $(t_0, y_0) \in I \times \R^n$ betrachten wir das AWP
  \[
    (3.1) \left\{ \begin{array}{l}
    \dot{y}(t) = A(t) \cdot y + f(t)\\
    y(t_0) = y_0
    \end{array} \right.
  \]
\end{prob}

\begin{satz}
  Sei $I \subset \R$, $A : I \to \R^{n \times n}$, $f : I \to \R^n$ stetig, $(t_0, y_0) \in I \times \R^n$. Dann besitzt das AWP (3.1) eine eindeutige (globale) Lösung auf $I$.
\end{satz}

\begin{nota}
  $\mathcal{C}(I, \R^n) \coloneqq \Set{ u : I \to \R^n }{ u \text{ stetig} }$
\end{nota}

% Ausgelassen: Definition Untervektorraum

\begin{defn}
  Eine Teilmenge $M \subset V$ eines Vektorraums $V$ heißt \emph{affiner Teilraum}, wenn es ein $y \in V$ und einen Unterraum $U \subset V$ mit $M = y + U \coloneqq \Set{ y + U }{ u \in U }$ gibt.
\end{defn}

% 3.2
\begin{satz}
  Seien $I \subset \R$ offen, $A : I \to \R^{n \times n}$ stetig. Setze
  \[ U_h \coloneqq \Set{ y \in \mathcal{C}^1(I, \R^n) }{ \dot{y}(t) = A(t) \cdot y \text{ auf $I$} }. \]
  Dann ist $U_h$ ein $n$-dimensionaler UVR von $\mathcal{C}^1(I, \R^n)$ und für Funktionen $y_1, ..., y_m \in U_h$ sind äquivalent:
  \begin{itemize}
    \item $y_1, ..., y_m$ sind linear unabhängig in $\mathcal{C}^1(I, \R^n)$.
    \item Es gibt $t^* \in I$, sodass $y_1(t^*), ..., y_m(t^*)$ linear unabh. in $\R^n$ sind.
    \item Für alle $t \in I$ sind $y_1(t), ..., y_m(t)$ linear unabhängig in $\R^n$.
  \end{itemize}
\end{satz}

% Vorlesung vom 19.5.2014

\begin{defn}
  Sei $I \subseteq \R$ offen, $A : I \to \R^{n \times n}$ stetig. Eine Menge $y_1, ..., y_n$ von linear unabhängigen Lösungen von $\dot{y} = A(t) \cdot y$ heißt ein \emph{Fundamentalsystem} und $(y_1, ..., y_n)$ \emph{Fundamentalmatrix} der DGL.
\end{defn}

% 3.3.
\begin{satz}
  Seien $I \subseteq \R$ offen, $A : I \to \R^{n \times n}$ stetig.
  \begin{itemize}
    \item Jede Fundamentalmatrix von $\dot{y} = A(t) y$ ist invertierbar f.\,a. $t \in I$.
    \item Jede Fundamentalmatrix $Y : I \to \R^{n \times n}$ ist stetig differenzierbar.
    \item Die globale eindeutige Lösung von
    \[
      \left\{ \begin{array}{l}
        \dot{y}(t) = A(t) \cdot y\\
        y(t_0) = y_0
        \end{array} 
      \right.
    \]
    ist gegeben durch $y(t) = Y(t) \left( Y(t_0) \right)^{-1} y_0$.
  \end{itemize}
\end{satz}

% 3.4.
\begin{satz}%[Struktur des Lösungsraums]
  Seien $I \subseteq \R$, $A : \R^{n \times n}$, $f : I \to \R^n$ stetig, $U_h$ wie oben und $U \coloneqq \Set{ y \in \mathcal{C}^1(I, \R^n) }{ \dot{y} = A(t) y + f }$. Dann gilt:
  \begin{itemize}
    \item $U$ ist nicht leer.
    \item Ist $y_p \in U$ eine partikuläre Lösung, dann ist $U = y_p + U_h$, d.\,h. $U$ ist affiner Unterraum von $\mathcal{C}^1(I, \R^n)$.
    \item Sei $y_p, \tilde{y}_p \in U$, dann ist $y_p - \tilde{y}_p \in U_h$.
  \end{itemize}
\end{satz}

% 3.5.
\begin{satz}[Variation der Konstanten]
  Sei $I \subseteq \R$ offen, $A : I \to \R^{n \times n}$, $f : I \to \R^n$ stetig. Sei $Y(t)$ die Fundamentalmatrix. Dann gilt:
  \begin{itemize}
    \item Eine partikuläre Lösung von $\dot{y} = A(t) y + f(t)$ ist gegeben durch
    \[ y_p(t) \coloneqq \Int{t_0}{t}{Y(t) (Y(s))^{-1} f(s)}{s}, \quad t_0 \in I. \]
    \item Es gilt $U = \Set{ \Int{t_0}{t}{Y(t) (Y(s))^{-1} f(s)}{s} + Y(t) c }{ c \in \R^n }$
    \item Die globale eindeutige Lsg vom AWP (3.1) ist gegeben durch
    \[ y(t) \coloneqq Y(t) (Y(t_0))^{-1} y_0 + \Int{t_0}{t}{Y(t) (Y(s))^{-1} f(s)}{s}, \quad t \in I. \]
  \end{itemize}
\end{satz}

% Vorlesungvom 21.5.2014

\begin{defn}
  Die Matrix-Expontentialfunktion ist definiert als
  \[ \exp : \R^{n \times n} \to \R^{n \times n}, \qquad A \mapsto e^A \coloneqq \sum_{k=0}^\infty \frac{A^k}{k!}. \]
\end{defn}

% 3.6.
\begin{satz}
  Seien $A, B \in \R^{n \times n}$. Dann gilt:
  \begin{itemize}
    \item Falls $A$ und $B$ kommutieren, d.\,h. $AB = BA$, dann gilt
    \[ e^{A+B} = e^A \cdot e^B = e^B \cdot e^A. \]
    \item Aus $e^{t(A + B)} = e^{tA} \cdot e^{tB}$ für alle $t \in \R$ folgt $AB = BA$.
    \item $e^{A}$ ist invertierbar mit $(e^A)^{-1} = e^{-A}$.
    \item Wenn $B$ invertierbar ist, dann gilt $e^{B^{-1}AB} = B^{-1} e^A B$.
    \item Ist $A$ eine Diagonalmatrix mit Einträgen $\lambda_1, ..., \lambda_n$, so gilt
    \[ e^A = \begin{pmatrix} e^{\lambda_1} & \cdots & 0 \\ \vdots & \ddots & \vdots \\ 0 & \cdots & e^{\lambda_n} \end{pmatrix} \]
    \miniitem{0.38 \linewidth}{$e^{(t+s)A} = e^{tA} \cdot e^{sA}$}
    \miniitem{0.48 \linewidth}{$e^{t(A + \lambda I)} = e^{\lambda t} \cdot e^{t A}$}
  \end{itemize}
\end{satz}

% 3.7.
\begin{satz}
  Sei $A \in \R^{n \times n}$ diagonalisierbar, d.\,h. es existiere eine Basis aus Eigenvektoren $s_1, ..., s_n \in \C^n$ zu den Eigenwerten $\lambda_1, ..., \lambda_n \in \C$ sodass $S^{-1} A S =: D$ mit $S \coloneqq (s_1, ..., s_n)$ diagonal mit Einträgen $\lambda_1, ..., \lambda_n$ ist. Dann gilt
  \[ e^{tA} = S e^{tD} S^{-1} = S \begin{pmatrix} e^{\lambda_1} & \cdots & 0 \\ \vdots & \ddots & \vdots \\ 0 & \cdots & e^{\lambda_n} \end{pmatrix} S^{-1} \]
\end{satz}

\begin{bem}
  Wenn $e^{tA}$ eine Fundamentalmatrix ist, dann ist auch $e^{tA} S = S e^{tD}$ eine Fundamentalmatrix.
\end{bem}

% Ausgelassen: Zwei Beispiele

% Vorlesung vom 26.5.2014

\begin{defn}
  Der \emph{Jordanblock} der Größe $n$ zum EW $\lambda$ ist die Matrix
  \[ J(\lambda, n) \coloneqq \begin{pmatrix}
    \lambda & 1 & & 0 \\
    & \ddots & \ddots & \\
    & & \ddots & 1 \\
    0 & & & \lambda
  \end{pmatrix}. \]
\end{defn}

\begin{bem}[Jordan-Normalform]
  Aus der Linearen Algebra ist bekannt: Sei $A \in \R^{n \times n}$ eine Matrix, für die gilt:
  \[ P_A(\lambda) = (-1)^N (\lambda - \lambda_1)^{a_1} \cdot ... \cdot (\lambda - \lambda_k)^{a_k}, \]
  wobei $\lambda_1, ..., \lambda_k \in \C$ paarweise verschiedene komplexe Eigenwerte mit algebraischen Vielfachheiten $a_1, ..., a_k$ sind. Dann gibt es eine Matrix $S \in GL_n(\C)$ mit
  \begin{align*}
    S^{-1} A S = J_A :=& \begin{pmatrix}
      A_1 & \cdots & 0\\
      \vdots & \ddots & \vdots\\
      0 & \cdots & A_k
    \end{pmatrix},
    \quad \text{wobei}\\
    A_j =& \begin{pmatrix}
      J(\lambda_j, p_{j1}) & \cdots & 0\\
      \vdots & \ddots & \vdots\\
      0 & \cdots & J(\lambda_j, p_{j g_j})
    \end{pmatrix}
  \end{align*}
\end{bem}

\begin{prop}
  Für $\lambda \in \C$, $p \in \N$ gilt:
  \[
    e^{t \cdot J(\lambda, p)} = e^{t \lambda} \cdot \begin{pmatrix}
      1 & t & \tfrac{t^2}{2!} & \cdots & \tfrac{t^{p-1}}{(p-1)!} \\
      0 & 1 & t & \cdots & \tfrac{t^{p-2}}{(p-2)!} \\
      \vdots & \ddots & \ddots & \ddots & \vdots \\
      0 & \cdots & 0 & 1 & t \\
      0 & \cdots & \cdots & 0 & 1
    \end{pmatrix}
  \]
\end{prop}

% 3.8.
\begin{satz}
  Sei $S^{-1} A S = J_A$ in Jordan-NF. Dann gilt:
  \begin{align*}
    e^{At} &= S e^{J_A t} S^{-1} \text{ mit } e^{J_A} = \begin{pmatrix}
      e^{A_1 t} & \cdots & 0\\
      \vdots & \ddots & \vdots\\
      0 & \cdots & e^{A_k t}
    \end{pmatrix}
    \quad \text{wobei}\\
    A_j &= \begin{pmatrix}
      e^{t J(\lambda_j, p_{j1})} & \cdots & 0\\
      \vdots & \ddots & \vdots\\
      0 & \cdots & e^{t J(\lambda_j, p_{j g_j})}
    \end{pmatrix}.
  \end{align*}
\end{satz}

% Ausgelassen: Berechnung der Jordan-NF

% 3.9.
\begin{satz}
  Kommutieren die Matrizen $(A(t))_{t \in I}$ miteinander, d.\,h.
  \[ A(t) \cdot A(s) = A(s) \cdot A(t) \quad \text{für alle $t, s \in I$}, \]
  so ist eine Fundamentalmatrix von $\dot{y} = A(t) y + f(t)$ gegeben durch
  \[
    Y(t) \coloneqq \exp(\Int{t_0}{t}{A(s)}{s})
    \qquad \text{für alle $t \in I$.}
  \]
\end{satz}

% Vorlesung vom 28.5.2014

% Kapitel 3.3. Lineare skalare DGLn höherer Ordnung

\begin{prob}
  Gegeben seien $f, a_0, ..., a_{n-1} \in \mathcal{I}$, $t_0 \in I$ und $z_0, ..., z_{n-1} \in \R$. Betrachte die lineare skalare DGL höherer Ordnung
  \[
    (3.1) \left\{ \begin{array}{l}
      y^{(n)}(t) + a_{n-1}(t) y^{(n-1)}(t) + ... + a_1(t) \dot{y}(t) + a_0(t) y(t) = f(t)\\
      y(t_0) = z_0, \enspace \dot{y}(t_0) = z_1, ..., y^{(n-1)}(t_0) = z_{n-1}.
    \end{array} \right.
  \]
\end{prob}

\begin{bem}
  Nach Satz 1.1 ist dieses Problem äquivalent zum AWP
  \[
    \frac{\partial}{\partial y} \begin{pmatrix}
      y_1 \\ \vdots \\ y_n
    \end{pmatrix} = \begin{pmatrix}
      0 & 1 && \\
      & \ddots & \ddots & \\
      && 0 & 1\\
      - a_0(t) & \cdots & \cdots & -a_{n-1}(t)
    \end{pmatrix} \cdot \begin{pmatrix}
      y_1 \\ \vdots \\ y_n
    \end{pmatrix} + \begin{pmatrix}
      0 \\ \vdots \\ 0 \\ f(t)
    \end{pmatrix}
  \]
  \[
    \text{mit } \begin{pmatrix}
      y_1(t_0) \\ \vdots \\ y_n(t_0)
    \end{pmatrix} = \begin{pmatrix}
      z_0 \\ \vdots \\ z_{n-1}
    \end{pmatrix}.
    \quad \text{Korresp. der Lsgn: }
    \begin{pmatrix}
      y \\ \dot{y} \\ \vdots \\ y^{(n-1)}
    \end{pmatrix} = \begin{pmatrix}
      y_1 \\ \vdots \\ y_n
    \end{pmatrix}.
  \]
\end{bem}

% 3.10.
\begin{satz}
  Sei $I \subseteq \R$ offen, $a_0, ..., a_{n-1} : I \to \R$, $f : I \to \R$ stetig, $t_0 \in I$, $z = (z_0, ..., z_{n-1})^T \in \R^n$. Dann besitzt das AWP (3.1) eine eindeutige globale Lösung $y : I \to \R$.
\end{satz}

% 3.11.
\begin{satz}
  Sei $I \subseteq \R$ offen, $a_0, ..., a_{n-1} : I \to \R$, $f : I \to \R$ stetig. Setze
  \begin{align*}
    U_{h,n} &\coloneqq \Set{ y \in \mathcal{C}^n(I, \R) }{ y^{(n)} + a_{n-1}(t) y^{(n-1)} + ... + a_0(t) y = 0 },\\
    U_{n} &\coloneqq \Set{ y \in \mathcal{C}^n(I, \R) }{ y^{(n)} + a_{n-1}(t) y^{(n-1)} + ... + a_0(t) y = f(t) }.
  \end{align*}
  Dann ist $U_{h,n}$ ein $n$-dimensionaler UVR von $\mathcal{C}^n(I, \R)$ und für Funktionen $x_1, ..., x_m \in U_{h_m}$ sind äquivalent:
  \begin{itemize}
    \item $x_1, ..., x_n$ sind linear unabhängig in $\mathcal{C}^n(I, \R)$.
    \item $(x_1, \dot{x_1}, ..., x_1^{(n-1)}), ..., (x_m, \dot{x_m}, ..., x_m^{(n-1)})$ sind linear unabhängig in $\mathcal{C}^1(I, \R^n)$.
    \item $(x_1(t^*), \dot{x_1}(t^*), ..., x_1^{(n-1)}(t^*)), ..., (x_m(t^*), \dot{x_m}(t^*), ..., x_m^{(n-1)}(t^*))$ sind linear unabhängig in $\R^n$ für ein $t^* \in I$.
    \item $(x_1(t), \dot{x_1}(t), ..., x_1^{(n-1)}(t)), ..., (x_m(t), \dot{x_m}(t), ..., x_m^{(n-1)}(t))$ sind linear unabhängig in $\R^n$ für alle $t \in I$.
  \end{itemize}
  Bezeichnen wir als Fundamentalsystem der DGL in (3.1) eine Menge von $n$ global linear unabhängigen Lösungen der DGL in (3.1) mit $f(t) \equiv 0$, dann gilt:
  \begin{itemize}
    \item Ist $\{ x_1, ..., x_n \}$ ein Fundamentalsystem, so ist
    \[ U_{h,n} = \Set{c_1 x_1 + ... + c_n x_n}{c_1, ..., c_n \in \R}. \]
    \item $U_n \not= \emptyset$ und für eine partikuläre Lösung $y_p \in U_n$ der DGL (3.1) ist $U_n = y_p + U_{h,n}$ ein affiner Unterraum von $\mathcal{C}^n(I, \R)$.
  \end{itemize}
\end{satz}

\begin{bem}
  Funktionen $x_1, ..., x_n \in U_{h_n}$ bilden ein Fundamentalsystem genau dann, wenn
  \[
    W(t) \coloneqq \det \begin{pmatrix}
      x_1(t) & \cdots & x_n(t) \\
      \dot{x_1}(t) & \cdots & \dot{x_n}(t) \\
      \vdots & \ddots & \vdots \\
      x^{(n-1)}(t) & \cdots & x^{(n-1)}(t)
    \end{pmatrix} \not= 0.
  \]
\end{bem}

\begin{bem}
  Seien die Funktionen $a_j(t) \equiv a_j \in \R$, $j = 0, ..., n-1$ konstant. Dann heißt die Matrix
  \[
    A = \begin{pmatrix}
      0 & 1 && \\
      & \ddots & \ddots & \\
      && 0 & 1\\
      - a_0 & \cdots & \cdots & -a_{n-1}
    \end{pmatrix}
  \]
  \emph{Begleitmatrix} und es gilt
  \[ P_A(\lambda) = (-1)^n (\lambda^n + a_{n-1} \lambda^{n-1} + ... + a_1 \lambda + a_0). \]
  Angenommen, es zerfällt $P_A(\lambda) = (\lambda - \lambda_1)^{\alpha_1} \cdot ... \cdot (\lambda - \lambda_k)^{\alpha_k}$ wobei $\lambda_1, ..., \lambda_k$ verschiedene Nullstellen mit algebraischen Vielfachheiten $\alpha_1, ..., \alpha_k$.
  Falls $k = n$ (bzw. $\alpha_1 = ... = \alpha_k = 1$), so ist $\{ e^{\lambda_1 t}, ..., e^{\lambda_k t} \}$ ein Fundamentalsystem von (3.1). Ansonsten ist
  \[
    \tilde{F} \coloneqq \tilde{F}_1 \cup ... \cup \tilde{F}_k
    \quad \text{mit} \quad
    \tilde{F}_j \coloneqq \{ e^{\lambda_j t}, e^{\lambda_j t} t, ..., e^{\lambda_j t} t^{\alpha_j - 1} \}
  \]
  ein Fundamentalsystem von (3.1).
\end{bem}

% Vorlesung vom 2.6.2014

\begin{bem}
  Um den Lösungsraum von (3.1) zu bestimmen, benötigen wir noch eine partikuläre Lösung $y_p$. Diese kann zum einen durch den Ansatz
  \[
    \begin{pmatrix}
      y_{1,p}(t)\\
      \vdots\\
      y_{n,p}(t)
    \end{pmatrix} \coloneqq \Int{t}{t_0}{Y(t) \cdot Y(s)^{-1} \begin{pmatrix}
      0\\ \vdots \\ 0 \\ f(s)
    \end{pmatrix}}{s}, \qquad
    y_p \coloneqq y_{1,p}.
  \]
  bestimmt werden. Falls $f$ elementaransatzfähig ist, d.\,h.
  \[ f(t) = g(t) \cdot e^{\mu t} \cos(\omega t) + q(t) \cdot e^{\mu t} \sin(\omega t) \]
  für Polynome $g(t), q(t) \in \R[t]$ und $\omega, \mu \in \R$ gilt, dann gibt es eine Lösung der Form
  \begin{align*}
    y_p(t) &= t^{\nu} (\gamma_0 + \gamma_1 t + ... + \gamma_m t^m) e^{\mu t} \cos(\omega t)\\
    & + t^{\nu} (\beta_0 + \beta_1 t + ... + \beta_m t^m) e^{\mu t} \sin(\omega t).
  \end{align*}
  Dabei ist $m \coloneqq \max(\deg g, \deg q)$ und
  \[
    \nu \coloneqq \begin{cases}
      0, & \text{falls $\mu + i \omega$ keine Nullstelle von $P_A(\lambda)$ ist,}\\
      k, & \text{falls $\mu + i \omega$ eine $k$-fache Nullstelle von $P_A(\lambda)$ ist}.
    \end{cases}
  \]
  Die Zahlen $\gamma_0, ..., \gamma_m, \beta_0, ..., \beta_m \in \R$ sind noch zu bestimmen.
\end{bem}

\begin{bsp}
  Das AWP
  \[
    \left\{ \begin{array}{ll}
      \ddot{y} + \Theta^2 y = \cos(\omega t)\\
      y(0) = 0, \dot{y}(0) = 0
    \end{array} \right.
  \]
  besitzt die Lösung
  \[
    y(t) = \begin{cases}
      \tfrac{1}{\omega^2 - \Theta^2} \cos(\Theta t) + \tfrac{1}{\Theta^2 - \omega^2} \cos(\omega t), &\text{falls $\omega^2 \not= \Theta^2$,}\\
      \tfrac{t}{2 \omega} \sin(\omega t), &\text{falls $\omega^2 = \Theta^2$.}
    \end{cases}
  \]
\end{bsp}

% Kapitel 4. Asymptotik und Stabilität

% Ausgelassen: Drei Beispiele am Anfang des Kapitels

% Vorlesung vom 4.6.2014

% Kapitel 4.1. Stabilitätstheorie

\begin{defn}
  Sei $\D \subset \R^{n+1}$ offen, $f : \D \to \R^n$ stetig, lokal Lipschitz-stetig bzgl. $y$ auf $\D$. Eine Lösung $y : I \to \R^n$ von $\dot{y} = f(t, y)$ und $y(t_0) = y_0$ auf einem Intervall $\cointerval{t_0}{\infty}$ für ein $t_0 \in \R$ heißt
  \begin{itemize}
    \item \emph{(Lyapunov-) stabil} auf $\cointerval{t_0}{\infty}$, wenn es für alle $\epsilon > 0$ ein $\delta > 0$ gibt, sodass für alle $(t_0, z_0) \in \D$ und $\norm{y_0 - z_0} < \delta$ das AWP
    \[
      (\dagger) \left\{ \begin{array}{ll}
        \dot{z} = f(t, z),\\
        z(t_0) = z_0
      \end{array} \right.
    \]
    eine Lösung $z$ auf $\cointerval{t_0}{\infty}$ besitzt, welche die Ungleichung $\norm{y(t) - z(t)} < \epsilon$ für alle $t \geq t_0$ erfüllt.
    \item \emph{attraktiv}, wenn es ein $\delta > 0$ gibt, sodass für alle $(t_0, z_0) \in \D$ mit $\norm{y_0 - z_0} < \delta$ das AWP ($\dagger$) eine Lösung $z$ auf $\cointerval{t_0}{\infty}$ besitzt mit
    \[ \lim_{t \to \infty} \norm{y(t) - z(t)} = 0. \]
    \item \emph{asymptotisch stabil}, falls $y$ stabil und attraktiv ist.
    \item \emph{exponentiell stabil}, wenn $\delta, \alpha, \omega > 0$ existieren, sodass für alle $(t_0, z_0) \in \D$ mit $\norm{y_0 - z_0} < \delta$ das AWP ($\dagger$) eine Lsg $z$ besitzt mit
    \[ \norm{y(t) - z(t)} \leq \alpha \norm{y_0 - z_0} e^{-\omega (t-t_0)}. \]
  \end{itemize}
\end{defn}

\begin{bem}
  exponentiell stabil $\Rightarrow$ asymptotisch stabil
\end{bem}

\begin{bsp}
  Eine Lösung $y$ des AWP $\dot{y} = \alpha y$, $y(t_0) = y_0$ ist
  \begin{itemize}
    \miniitem{0.53 \linewidth}{exponentiell stabil, falls $\alpha < 0$,}
    \miniitem{0.44 \linewidth}{stabil, falls $\alpha = 0$ und}
    \item instabil (d.\,h. weder stabil noch attraktiv), falls $\alpha > 0$.
  \end{itemize}
\end{bsp}

% Kapitel 4.2. Autonome DGLn

% 4.1.
\begin{satz}
  Seien $\D \subset \R^{n+1}$ offen, $f : \D \to \R^n$ stetig, lokal Lipschitz stetig.
  \begin{itemize}
    \item Die DGL $\dot{y} = f(y)$ ist translationsinvariant, d.\,h. ist $y$ eine Lösung von $\dot{y} = f(y)$ auf $\cointerval{t_0}{\infty}$, dann ist $z(t) \coloneqq y(t_0 + t)$ eine Lösung von $\dot{z} = f(z)$ auf $\cointerval{0}{\infty}$.
    \item Die Lösung $y$ ist genau dann (exponentiell/asymptotisch) stabil, wenn $z$ (exponentiell/asymptotisch) stabil ist.
    \item Eine Lösung von $\dot{y} = f(y)$ ist genau dann konstant, also $y(t) \equiv c \in \R^n$, wenn $f(c) = 0$.
  \end{itemize}
\end{satz}

\begin{sprech}
  Konstante Lösungen werden auch stationäre Lösungen oder zeitinvariante Lösungen genannt.
\end{sprech}

\begin{defn}
  Seien $\D \subset \R^n$, $f : \D \to \R^n$ stetig, lokal Lipschitz-stetig auf $\D$. Jede Nullstelle von $f$ heißt \emph{Ruhelage} (Gleichgewichtspunkt, kritischer Punkt) von $f$.
\end{defn}

% Kapitel 4.2.1. Lineare autonome DGLn

\begin{bem}
  Eine Lösung $y : \cointerval{0}{\infty} \to \R$ der Gleichung
  \[ \dot{y} = A y, \quad A \in \R^{n \times n} \]
  ist genau dann (exponentiell/asymptotisch) stabil, wenn es die konstante Lösung $y(t) \equiv 0$ ist.
\end{bem}

% Vorlesung vom 11.6.2014

% 4.2.
\begin{satz}
  Sei $A \in \R^{n \times n}$ und $S \in \C^{n \times n}$ invertierbar. Das GGW $y_* \equiv 0$ der DGL $\dot{y} = Ay$ ist genau dann (exp/asympt) stabil, wenn das GGW $z_* \equiv 0$ der DGL $\dot{z} = S^{-1} A S$ (exp/asympt) stabil ist.
\end{satz}

% 4.3.
\begin{satz}[Stabilität von linearen, autonomen DGLn]
  Sei $A \in \R^{n \times n}$ und $\lambda_j = \alpha_j + i \beta_j \in \C$ für die Eigenwerte von $A$. Das GGW $y_* \equiv 0$ von $\dot{y} = Ay$ ist genau dann
  \begin{itemize}
    \item stabil, wenn für alle EWe $\Re(\lambda_j) = \alpha_j \leq 0$ gilt und alle EWe mit $\Re(\lambda_j) = 0$ halbeinfach sind, d,\,h. die geometrische Vielfachheit mit der algebraischen übereinstimmt,
    \item asymptotisch stabil, wenn für alle EWe $\Re(\lambda_j) = \alpha_j < 0$ gilt,
    \item ansonsten instabil.
  \end{itemize}
\end{satz}

\end{document}