\documentclass{cheat-sheet}

\pdfinfo{
  /Title (Zusammenfassung Gewöhnliche Differentialgleichungen)
  /Author (Tim Baumann)
}

% Kleinere Klammern
\delimiterfactor=701


\begin{document}

\maketitle{Zusammenfassung Gew. Diff'gleichungen}

% Kapitel 1. Einführung

% Kapitel 1.1. Beispiele

\begin{bsp}
  Gesucht: Funktion $y : \R \to \R$ mit $\fa{t \in \R} \dot{y}(t) = y(t)$\\
\end{bsp}

\begin{lsg}
  $y(t) = c \cdot e^t$ für $c \in \R$ beliebig. Wenn man als Anfangsbe- dingung $y(0) = 1$ fordert, erhält man eine eindeutige Lösung ($c = 1$).
\end{lsg}

\begin{bsp}
  Gesucht: Lösung von $\left(\dot{y}(t)\right)^2 + \left(y(t)\right)^2 = a$ für $a \in \R$\\
\end{bsp}

\begin{lsg}
  Anzahl der Lösungen hängt von $a$ ab:
  \begin{itemize}
    \miniitem{0.47\linewidth}{Falls $a < 0$: keine reelle Lsg}
    \miniitem{0.51\linewidth}{Falls $a = 0$: Einzige Lsg $y(t) = 0$}
    \item Falls $a > 0$: Lsgn: $y(t) = \sqrt{a} \cos(t + \phi)$ für $\phi \in \R$ bel., $y(t) = \pm \sqrt{a}$
  \end{itemize}
\end{lsg}

% Populationsmodell aus der Biologie/Chemie
\begin{bsp}
  Sei $p(t)$ ist Populationsgröße zur Zeit $t$.
  Angenommen, $\tfrac{\dot{p}(t)}{p(t)} = a$ ist konstant, also $\dot{p}(t) = p(t)$. Sei $p(t_0) = p_0$.
\end{bsp}

\begin{lsg}
  $p(t) = p_0 e^{(t-t_0) a}$
\end{lsg}

% 1837
\begin{bsp}[Verhulst-Modell]
  Gesucht: Lösung zu
  \[ \dot{p}(t) = a_0 p(t) - a_1 \left(p(t)\right)^2 \]
\end{bsp}

\begin{lsg}
  $p(t) = \frac{a_0}{a_1 (1 - c e^{-a_0 t})}$
\end{lsg}

% Mechanik: Mathematisches Pendel
% Skizze: Pendel der
% * Masse $m$
% * Länge $l$
% * Auslenkungswinkel $\phi(t)$ zum Zeitpunkt $t$
% * Position $l \phi(t)$ zum Zeitpunkt $t$
% * Geschwindigkeit $v(t) = \dot{p}(t) = l \dot{\phi}(t)$
% * $a(t) = \dotdot{p}(t) = \dot{v}(t) = l \dotdot{\phi}(t)$ Beschleunigung
% $F_E = - mg$, $F = ma$
% $F_T$ -- tangentiale Komponente der Gewichtskraft
% $F_T = - mg \sin(\phi)$
% => Gleichung $-mg \sin(\phi) = m a(t) = m l \dotdot{t}$, also $\dotdot{\phi}(t) = -\tfrac{g}{l} \sin(\phi(t))$

\begin{bsp}
  
\end{bsp}

\end{document}