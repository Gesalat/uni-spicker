\documentclass{cheat-sheet}

\pdfinfo{
  /Title (Zusammenfassung Modellkategorien)
  /Author (Tim Baumann)
}

% Kategorientheorie-Makros

% Konzepte
\DeclareMathOperator{\Ob}{Ob} % Objekte (einer Kategorie)
\DeclareMathOperator{\Mor}{Mor} % Morphismenmenge / -klasse
\DeclareMathOperator{\Hom}{Hom} % Homomorphisms
\DeclareMathOperator{\Nat}{Nat} % Natürliche Transformationen
\DeclareMathOperator{\dom}{dom} % Domain
\DeclareMathOperator{\codom}{codom} % Codomain
\newcommand{\op}{\mathrm{op}} % opposite category
\DeclareMathOperator{\Aut}{Aut} % Automorphismengruppe
\newcommand{\ladj}{\dashv} % Links-adjungiert (left-adjoint)
\newcommand{\Lim}{\lim} % Limes
\DeclareMathOperator{\colim}{colim} % Kolimes
\newcommand{\Colim}{\colim} % Kolimes
%\newcommand{\myint}[2]{{\textstyle \int\limits_{#1}^{#2}}}
\newcommand{\EndC}[2]{\myint{#1}{} #2} % Ende
\newcommand{\CoEndC}[2]{\myint{}{#1} #2} % Koende
\DeclareMathOperator{\Ran}{Ran} % Rechts-Kan-Erweiterung
\DeclareMathOperator{\Lan}{Lan} % Links-Kan-Erweiterung

% Platzhalter
\newcommand{\DiaTodo}{\fcolorbox{red}{white}{TODO: Diagramm einfügen!}}

% Konkrete Kategorien
\newcommand{\SetC}{\mathbf{Set}} % Kategorie der Mengen
\newcommand{\sSet}{\mathbf{sSet}} % Kategorie der simplizialen Mengen
\newcommand{\Top}{\mathbf{Top}} % Kategorie der topologischen Räume
\newcommand{\AbGrp}{\mathbf{Ab}} % Kategorie der abelschen Gruppen
\newcommand{\Grp}{\mathbf{Grp}} % Kategorie der Gruppen
\newcommand{\RMod}{\mathbf{R\text{-}Mod}} % Kategorie der R-Moduln
\newcommand{\Ouv}{\mathbf{Ouv}} % Kategorie der offenene Mengen eines topol. Raumes
\newcommand{\KHaus}{\mathbf{KHaus}} % Kategorie der kompakten Hausdorffräume
\newcommand{\CatC}{\mathbf{Cat}} % Kategorie der kleinen Kategorien
\newcommand{\Vect}{\mathbf{Vect}} % Kategorie der Vektorräume über einem Körper
\newcommand{\Alg}{\mathbf{Alg}} % Kategorie der Algebren über einem Körper/Ring
\newcommand{\VectFin}{\mathbf{Vect}_{\mathrm{fin}}} % Kategorie der endlichen Vektorräume über einem Körper
\newcommand{\kVect}{\text{$k$-$\Vect$}} % Kategorie der k-Vektorräume über einem Körper k
\newcommand{\kVectFin}{\text{$k$-$\VectFin$}} % Kategorie der endlichen k-Vektorräume über einem Körper k
\newcommand{\Mod}{\mathbf{Mod}} % Kategorie der Moduln über einem Ring
\newcommand{\Kom}{\mathbf{Kom}} % Kategorie der Komplexe in einer abelschen Kategorie
\newcommand{\Der}{\mathcal{D}} % abgeleitete Kategorie einer abelschen Kategorie
\newcommand{\kAlg}{k\text{-}\Alg} % Kategorie der k-Algebren

% Bezeichnungen für Variablen, die für Kategorien stehen
\newcommand{\Aat}{\mathcal{A}} % Category-A
\newcommand{\Bat}{\mathcal{B}} % Category-B
\newcommand{\Cat}{\mathcal{C}} % Category-C
\newcommand{\Dat}{\mathcal{D}} % Category-D
\newcommand{\Eat}{\mathcal{E}} % Category-E
\newcommand{\Fat}{\mathcal{F}} % Category-F
\newcommand{\Gat}{\mathcal{G}} % Category-G
\newcommand{\Iat}{\mathcal{I}} % Category-I (Indexkategorie)
\newcommand{\Jat}{\mathcal{J}} % Category-J (Indexkategorie)
\newcommand{\MatC}{\mathcal{M}} % Category-M
\newcommand{\Sit}{\mathcal{S}} % Situs-S
 % Kategorientheorie-Makros

\newcommand{\Ord}{\mathcal{O}_n} % Menge der Ordinalzahlen
\newcommand{\Prop}{\mathbf{Prop}} % Typ der Propositionen

% Kleinere Klammern
\delimiterfactor=701

\begin{document}

\maketitle{Zusammenfassung Modellkategorien}

\begin{bem}
  Die \href{http://timbaumann.info/uni-spicker/topo.pdf}{Topologie-Zusammenfassung} bietet eine Übersicht über Grundbegriffe der Kategorientheorie. Weiterführende Begriffe werden in der \href{http://timbaumann.info/uni-spicker/homoalg.pdf}{Homologische-Algebra-Zusammenfassung} behandelt.
\end{bem}

% Übung vom 15.4.2015

\section{Die Ordinalzahlen}

\begin{defn}
  Eine \emph{Wohlordnung} auf einer Menge $S$ ist eine Totalordnung auf $S$ bezüglich der jede nichtleere Teilmenge $A \subseteq S$ ein kleinstes Element besitzt. Eine wohlgeordnete Menge ist ein Tupel $(S, \leq)$ bestehend aus einer Menge $S$ und einer Wohlordnung $\leq$ auf $S$.
\end{defn}

\begin{bem}
  Eine äquivalente Bedingung lautet: Es gibt in $S$ keine nach rechts unendlichen absteigenden Folgen
  $\ldots > a_i > a_{i+1} > a_{i+2} > \ldots$
\end{bem}

\begin{bem}
  Äquivalent zum Auswahlaxiom ist:
\end{bem}

\begin{axiom}[Wohlordnungssatz]
  Auf jeder Menge ex. eine Wohlord.
\end{axiom}

\begin{defn}
  Zwei wohlgeordnete Mengen heißen isomorph, wenn es eine monotone Bijektion zwischen ihnen gibt.
\end{defn}

% Ausgelassen: Intuition

\begin{defn}
  Eine \emph{Ordinalzahl} ist eine Isomorphieklasse von wohlgeordneten Mengen.
\end{defn}

\begin{bem}
  Die Klasse aller Ordinalzahlen wird mit $\Ord$ bezeichnet und ist eine echte Klasse, keine Menge.
  Sie ist selbst wohlgeordnet mittels
  \[ [(S, \leq_S)] \leq [(T, \leq_T)] \coloniff \exists \, \text{inj. monotone Abb. $(S, \leq_S) \to (T, \leq_T)$}. \]
\end{bem}

\begin{nota}
  \inlineitem{$0 \coloneqq [\emptyset]$,} \enspace
  \inlineitem{$n \coloneqq [\{ 1, \ldots, n \}]$ für $n \in \N$,} \enspace
  \inlineitem{$\omega \coloneqq [\N]$} \\
  mit der jeweils kanonischen Ordnungsrelation.
\end{nota}

\begin{bem}
  Die ersten Ordinalzahlen sind
  \[
    0, 1, 2, \ldots, \omega, \omega + 1, \omega + 2, \ldots, \omega \cdot 2, \omega \cdot 2 + 1, \ldots, \omega \cdot 3, \ldots, \omega^\omega, \ldots
  \]
\end{bem}

\begin{prinzip}[\emph{Transfinite Induktion}]\mbox{}\\
  Sei $P : \Ord \to \Prop$ eine Aussage über Ordinalzahlen. Dann gilt:
  \[ \left( \fa{\beta \in \Ord} (\fa{\gamma < \beta} P(\gamma)) \implies P(\beta) \right) \implies \fa{\alpha \in \Ord} P(\alpha) \]
\end{prinzip}

\begin{defn}
  Arithmetik von Ordinalzahlen ist folgendermaßen definiert: \\
  Für $\alpha = [(S, \leq_S)]$ und $\beta = [(T, \leq_T)] \in \Ord$ ist
  \begin{itemize}
    \item $\alpha + \beta \coloneqq [(S \amalg T, \leq_{S \amalg T})]$, wobei gilt:
    \[
      \leq_{S \amalg T}|_{S \times S} \,\coloneqq\, \leq_S, \quad
      \leq_{S \amalg T}|_{T \times T} \,\coloneqq\, \leq_T, \quad
      %s_1 \leq_{S \amalg T} s_2 \coloniff s_1 \leq_S s_2, \quad
      %t_1 \leq_{S \amalg T} t_2 \coloniff t_1 \leq_T t_2, \quad
      S <_{S \amalg T} T.
    \]
    \item $\alpha \cdot \beta \coloneqq [(S \times T, \leq_{S \rtimes T})]$ mit der lexikogr. Ordnung
    \[ (s_1, t_1) \leq_{S \rtimes T} (s_2, t_2) \coloneqq t_1 < t_2 \,\vee\, (t_1 = t_2 \wedge s_1 \leq_S s_2) \]
    \item $\alpha^\beta \coloneqq [(\{ \text{Abb. $f : S \to T$ mit $f(s) = 0$ für fast alle $s \in S$} \}, \leq)]$ mit
    \[ f < g \coloniff \ex{t \in T} f(t) < g(t) \wedge \left( \fa{t_2 >_T t} f(t_2) = g(t_2) \right) \]
  \end{itemize}
  %\begin{align*}
  %\end{align*}
\end{defn}

\begin{bem}
  Es gibt drei Typen von Ordinalzahlen:
  \begin{enumerate}[label=\alph*),leftmargin=1.6em]
    \item Die Null $0 \coloneqq [(\emptyset, \leq)] \in \Ord$.
    \item Die Nachfolgerzahl $\alpha + 1$ einer Zahl $\alpha \in \Ord$.
    \item Die Limeszahl $\lim A \coloneqq \sup A$ einer Teil{\em menge} $A \subset \Ord$.
  \end{enumerate}
\end{bem}

% http://en.wikipedia.org/wiki/Ordinal_arithmetic
\begin{bem}
  Die Rechenop. können auch rekursiv definiert werden durch
  \begin{tabular}{ l l l }
    a) & b) & c) \\[2pt]
    $\alpha + 0 \coloneqq \alpha$
    & $\alpha \!+\! (\beta \!+\! 1) \coloneqq (\alpha \!+\! \beta) \!+\! 1$
    & $\alpha \!+\! \lim A \coloneqq \lim \, \Set{\alpha \!+\! \gamma}{\gamma \!\in\! A}$ \\

    $\alpha \cdot 0 \coloneqq 0$
    & $\alpha \cdot (\beta + 1) \coloneqq (\alpha \cdot \beta) + \alpha$
    & $\alpha \cdot \lim A \coloneqq \lim \, \Set{\alpha \cdot \gamma}{\gamma \in A}$ \\

    $\alpha^0 \coloneqq 1$
    & $\alpha^{\beta + 1} \coloneqq \alpha^\beta \cdot \alpha$
    & $\alpha^{\lim A} \coloneqq \lim \, \Set{\alpha^\gamma}{\gamma \in A}$
  \end{tabular}
  \iffalse
  \begin{align*}
    \text{a) } \alpha + 0 \coloneqq \alpha, \qquad
    & \text{b) } \alpha + (\beta + 1) \coloneqq (\alpha + \beta) + 1, \\
    \text{c) } \alpha + \lim A &\coloneqq \lim \, \Set{\alpha + \gamma}{\gamma \in A}, \\[2pt]
    \text{a) } \alpha \cdot 0 \coloneqq 0, \qquad
    & \text{b) } \alpha \cdot (\beta + 1) \coloneqq (\alpha \cdot \beta) + \alpha, \\
    \text{c) } \alpha \cdot \lim A &\coloneqq \lim \, \Set{\alpha \cdot \gamma}{\gamma \in A}, \\[2pt]
    \text{a) } \alpha^0 \coloneqq 1, \qquad
    & \text{b) } \alpha^{\beta + 1} \coloneqq \alpha^\beta + 1, \\
    \text{c) } \alpha^{\lim A} &\coloneqq \lim \, \Set{\alpha^\gamma}{\gamma \in A}.
  \end{align*}
  \fi
\end{bem}

\begin{defn}
  Ein \emph{Fast-Halbring} ist ein Tupel $(S, +, \cdot, 0)$, sodass $(S, +, 0)$ ein Monoid und $(S, \cdot)$ eine Halbgruppe ist mit \\
  \inlineitem{$a \cdot (b+c) = a \cdot b + a \cdot c$,} \quad
  \inlineitem{$a \cdot 0 = 0$.}
\end{defn}

\begin{lem}[Rechenregeln in $\Ord$]
  \inlineitem{$\alpha \cdot 0 = 0 = 0 \cdot \alpha$} \enspace
  \inlineitem{$\alpha \cdot 1 = \alpha = 1 \cdot \alpha$} \\
  \inlineitem{$\alpha^0 = 1$} \quad
  \inlineitem{$0^\alpha = 0$ für $\alpha > 0$} \quad
  \inlineitem{$1^\alpha = 1$} \quad
  \inlineitem{$\alpha^1 = \alpha$} \\
  \inlineitem{$\alpha^\beta \cdot \alpha^\gamma = \alpha^{\beta+\gamma}$} \enspace
  \inlineitem{$(\alpha^\beta)^\gamma = \alpha^{\beta \cdot \gamma}$} \enspace
  %\inlineitem{$$} \enspace
  \begin{itemize}
    \item $\Ord$ ist ein Fast-Halbring (mit einer Klasse statt Menge)
    \item Das andere Distributivgesetz stimmt {\em nicht}!
    \item Weder Addition noch Multiplikation sind kommutativ.
    \item Addition und Mult. erlauben das Kürzen von Elementen nur links.
    \item Addition, Multiplikation und Potenzieren sind in beiden Argu- menten monoton, allerdings nur im zweiten strikt monoton:
    \[
      \forall \, \beta < \gamma: \quad
      \alpha + \beta < \alpha + \gamma, \quad
      \alpha \cdot \beta < \alpha \cdot \gamma \enspace (\alpha \!>\! 0), \quad
      \alpha^\beta < \alpha^\gamma \enspace (\alpha \!>\! 1).
    \]
  \end{itemize}
\end{lem}

% http://en.wikipedia.org/wiki/Ordinal_arithmetic
\begin{lem}
  Jedes $\alpha \in \Ord$ kann geschrieben werden in \emph{Cantor-NF}:
  \[ \alpha = \omega^{\beta_1} c_1 + \omega^{\beta_2} c_2 + \ldots + \omega^{\beta_k} c_k \]
  mit $k \in \N$, $c_1, \ldots, c_k \in \N_{> 0}$ und $\beta_1 > \ldots > \beta_k \in \Ord$.
\end{lem}

% TODO: Verbindungen zu Kardinalzahlen

\end{document}
