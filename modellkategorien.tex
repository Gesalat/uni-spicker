\documentclass{cheat-sheet}

\pdfinfo{
  /Title (Zusammenfassung Modellkategorien)
  /Author (Tim Baumann)
}

\usepackage{tikz}
\usetikzlibrary{arrows,cd}
\usepackage{cancel} % für durchgestrichenen Text

\usepackage{mathabx} % für \boxslash
\usepackage{stmaryrd} % für \leftarrowtriangle

% http://tex.stackexchange.com/questions/117732/tikz-and-babel-error
% Es ist schierer Wahnsinn, welche Hacks LaTeX benötigt!
\tikzset{
  every picture/.prefix style={
    execute at begin picture=\shorthandoff{"}
  }
}

% Kategorientheorie-Makros

% Konzepte
\DeclareMathOperator{\Ob}{Ob} % Objekte (einer Kategorie)
\DeclareMathOperator{\Mor}{Mor} % Morphismenmenge / -klasse
\DeclareMathOperator{\Hom}{Hom} % Homomorphisms
\DeclareMathOperator{\Nat}{Nat} % Natürliche Transformationen
\DeclareMathOperator{\dom}{dom} % Domain
\DeclareMathOperator{\codom}{codom} % Codomain
\newcommand{\op}{\mathrm{op}} % opposite category
\DeclareMathOperator{\Aut}{Aut} % Automorphismengruppe
\newcommand{\ladj}{\dashv} % Links-adjungiert (left-adjoint)
\newcommand{\Lim}{\lim} % Limes
\DeclareMathOperator{\colim}{colim} % Kolimes
\newcommand{\Colim}{\colim} % Kolimes
%\newcommand{\myint}[2]{{\textstyle \int\limits_{#1}^{#2}}}
\newcommand{\EndC}[2]{\myint{#1}{} #2} % Ende
\newcommand{\CoEndC}[2]{\myint{}{#1} #2} % Koende
\DeclareMathOperator{\Ran}{Ran} % Rechts-Kan-Erweiterung
\DeclareMathOperator{\Lan}{Lan} % Links-Kan-Erweiterung

% Platzhalter
\newcommand{\DiaTodo}{\fcolorbox{red}{white}{TODO: Diagramm einfügen!}}

% Konkrete Kategorien
\newcommand{\SetC}{\mathbf{Set}} % Kategorie der Mengen
\newcommand{\sSet}{\mathbf{sSet}} % Kategorie der simplizialen Mengen
\newcommand{\Top}{\mathbf{Top}} % Kategorie der topologischen Räume
\newcommand{\AbGrp}{\mathbf{Ab}} % Kategorie der abelschen Gruppen
\newcommand{\Grp}{\mathbf{Grp}} % Kategorie der Gruppen
\newcommand{\RMod}{\mathbf{R\text{-}Mod}} % Kategorie der R-Moduln
\newcommand{\Ouv}{\mathbf{Ouv}} % Kategorie der offenene Mengen eines topol. Raumes
\newcommand{\KHaus}{\mathbf{KHaus}} % Kategorie der kompakten Hausdorffräume
\newcommand{\CatC}{\mathbf{Cat}} % Kategorie der kleinen Kategorien
\newcommand{\Vect}{\mathbf{Vect}} % Kategorie der Vektorräume über einem Körper
\newcommand{\Alg}{\mathbf{Alg}} % Kategorie der Algebren über einem Körper/Ring
\newcommand{\VectFin}{\mathbf{Vect}_{\mathrm{fin}}} % Kategorie der endlichen Vektorräume über einem Körper
\newcommand{\kVect}{\text{$k$-$\Vect$}} % Kategorie der k-Vektorräume über einem Körper k
\newcommand{\kVectFin}{\text{$k$-$\VectFin$}} % Kategorie der endlichen k-Vektorräume über einem Körper k
\newcommand{\Mod}{\mathbf{Mod}} % Kategorie der Moduln über einem Ring
\newcommand{\Kom}{\mathbf{Kom}} % Kategorie der Komplexe in einer abelschen Kategorie
\newcommand{\Der}{\mathcal{D}} % abgeleitete Kategorie einer abelschen Kategorie
\newcommand{\kAlg}{k\text{-}\Alg} % Kategorie der k-Algebren

% Bezeichnungen für Variablen, die für Kategorien stehen
\newcommand{\Aat}{\mathcal{A}} % Category-A
\newcommand{\Bat}{\mathcal{B}} % Category-B
\newcommand{\Cat}{\mathcal{C}} % Category-C
\newcommand{\Dat}{\mathcal{D}} % Category-D
\newcommand{\Eat}{\mathcal{E}} % Category-E
\newcommand{\Fat}{\mathcal{F}} % Category-F
\newcommand{\Gat}{\mathcal{G}} % Category-G
\newcommand{\Iat}{\mathcal{I}} % Category-I (Indexkategorie)
\newcommand{\Jat}{\mathcal{J}} % Category-J (Indexkategorie)
\newcommand{\MatC}{\mathcal{M}} % Category-M
\newcommand{\Sit}{\mathcal{S}} % Situs-S
 % Kategorientheorie-Makros
% Garbentheorie-Makros

% Konzepte

% Konkrete Garben
\renewcommand{\O}{\mathcal{O}} % Strukturgarbe der stetigen Funktionen
\newcommand{\constSh}[1]{\underline{#1}} % konstante Garbe

% Kategorien von Garben
% Notation übernommen von http://stacks.math.columbia.edu/download/sheaves.pdf
\newcommand{\PShSet}{\mathbf{PSh}} % Prägarben von Mengen
\newcommand{\ShSet}{\mathbf{Sh}} % Garben von Mengen
\newcommand{\PShAb}{\mathbf{PAb}} % Prägarben von abelschen Gruppen
\newcommand{\ShAb}{\mathbf{Ab}} % Garben von abelschen Gruppen

% Bezeichnungen für Variablen, die für Garben stehen
\newcommand{\Fais}{\mathcal{F}} % Faisceau-F (Garbe auf französisch)
\newcommand{\Garb}{\mathcal{G}} % Garben-G
\newcommand{\Harb}{\mathcal{H}} % Garben-H
\newcommand{\Karb}{\mathcal{H}} % Garben-K (Kern)
\newcommand{\Carb}{\mathcal{H}} % Garben-C (Kokern)
\newcommand{\Iarb}{\mathcal{H}} % Garben-I (Image)
 % Garbentheorie-Makros (für Beispiele)

\newcommand{\nspace}[1]{\foreach \i in {1,...,#1}{ \! }} % Negativer Abstand
\newcommand{\Ord}{\mathcal{O}_n} % Menge der Ordinalzahlen
\newcommand{\Prop}{\mathbf{Prop}} % Typ der Propositionen
\newcommand{\xtwoheadrightarrow}[1]{\xrightarrow{#1}\nspace{8}\to\,\,} % Pfeil mit zwei Spitzen (HACK!!!)
\DeclareMathOperator{\Quot}{Quot} % Quotientenkörper
\newcommand{\Cyl}[1]{#1 \!\times\! I} % Zylinderobjekt
\newcommand{\PO}[1]{{#1}^I} % Pfadobjekt
\newcommand{\lhhe}{\boxslash} % Linkshochhebungseigenschaftsdiagramm-im-Miniaturformat-Symbol
\newcommand{\Weak}{\mathcal{W}} % weak equivalences
\newcommand{\Cof}{\mathcal{C}} % cofibrations
\newcommand{\Fib}{\mathcal{F}} % fibrations
\newcommand{\ModC}{\mathcal{M}} % model category
\newcommand{\gut}{\text{gut}}
\newcommand{\sg}{\text{sehr gut}}
\DeclareMathOperator{\Ho}{Ho} % Homotopiekategorie
\DeclareMathOperator{\Cell}{Cell} % Zellkomplexe
\DeclareMathOperator{\Coff}{Cof} % Kofaserungen
\newcommand{\ModStr}{$(\Weak, \Cof, \Fib)$} % Daten einer Modellstruktur

\newenvironment{centertikzcd}
  {\begin{center}\begin{tikzcd}}
  {\end{tikzcd}\end{center}}

% Kleinere Klammern
\delimiterfactor=701

\begin{document}

\maketitle{Zusammenfassung Modellkategorien}

\section{Kategorientheorie}

\begin{bem}
  Die \href{http://timbaumann.info/uni-spicker/topo.pdf}{Topologie-Zusammenfassung} bietet eine Übersicht über Grundbegriffe der Kategorientheorie. Weiterführende Begriffe werden in der \href{http://timbaumann.info/uni-spicker/homoalg.pdf}{Homologische-Algebra-Zusammenfassung} behandelt.
\end{bem}

% Vorlesung vom 22.4.2015

\begin{defn}
  Eine \emph{(schwache) 2-Kategorie} $\C$ besteht aus
  \begin{itemize}
    \item einer Ansammlung $\Ob(\C)$ von Objekten,
    \item für jedes Paar $(\Cat, \Dat)$ von Objekten einer Kategorie
    \[
      \Hom_{\C}(\Cat, \Dat) = \left\{\,
      \begin{tikzcd}
        A \arrow[r, bend left=50, "F"{above}, ""{name=U, below}]
        \arrow[r, bend right=50, "G"{below}, ""{name=D, above}]
        & B
        \arrow[Rightarrow, from=U, to=D]
      \end{tikzcd}
      \,\right\},
    \]
    \item für jedes Tripel $(\Cat, \Dat, \Eat)$ von Objekten einem Funktor
    \[
      \Hom_\C(\Cat, \Dat) \times \Hom_\C(\Dat, \Eat) \to \Hom_\C(\Cat, \Eat), \enspace
      (F, G) \mapsto G \circ F,
    \]
    \item für jedes Objekt $\Cat \in \Ob(\C)$ einem Objekt $\Id_\Cat \in \Hom_\C(\Cat, \Cat)$,
    \item für alle $\Cat, \Dat, \Eat, \Fat \in \Ob(\C)$ einem natürlichen Isomorphismus
    \[ \alpha_{\Cat, \Dat, \Eat, \Fat} : \enspace \blank \circ (\blank \circ \blank) \Longrightarrow (\blank \circ \blank) \circ \blank, \]
    wobei beide Seiten Funktoren sind vom Typ
    \[ \Hom(\Eat, \Fat) \times \Hom(\Dat, \Eat) \times \Hom(\Cat, \Dat) \to \Hom(\Cat, \Fat), \]
    \iffalse
    \begin{centertikzcd}[column sep=1cm, row sep=0.5cm]
      & \Hom(\Cat, \Eat) \times \Hom(\Eat, \Fat) \arrow[rd] \\
      \Hom(\Cat, \Dat) \times \Hom(\Dat, \Eat) \times \Hom(\Eat, \Fat) \arrow[ru] \arrow[rd] & \Hom(\Cat, \Fat) \\
      & \Hom(\Cat, \Dat) \times \Hom(\Dat, \Fat)
    \end{centertikzcd}
    \fi
    \item und für alle $\Cat, \Dat \in \Ob(\C)$ natürlichen Isomorphismen
    \[
      \lambda_{\Cat, \Dat} : (\Id_\Dat \circ \, \blank) \Rightarrow \Id_{\Hom(\Cat, \Dat)}, \enspace
      \rho_{\Cat, \Dat} : (\blank \circ \Id_\Cat) \Rightarrow \Id_{\Hom(\Cat, \Dat)},
    \]
  \end{itemize}
  sodass folgende \emph{Kohärenzbedingungen} erfüllt sind:
  \begin{itemize}
    \item Für alle $(\Cat \xrightarrow{F} \Dat \xrightarrow{G} \Eat \xrightarrow{H} \Fat \xrightarrow{K} \Gat) \in \Cat$ kommutiert
    \begin{centertikzcd}[column sep=1cm, row sep=0.5cm]
      K (H (G F)) \arrow[r, "\alpha_{\Cat, \Eat, \Fat, \Gat}"] \arrow[d, "K \alpha_{\Cat, \Dat, \Eat, \Fat}"] &
      (K H) (G F) \arrow[r, "\alpha_{\Cat, \Dat, \Eat, \Gat}"] &
      ((K H) G) H \\
      K ((H G) F) \arrow[rr, "\alpha_{\Cat, \Dat, \Fat, \Gat}"] &&
      (K (H G)) F \arrow[u, "\alpha_{\Dat,\Eat,\Fat,\Gat} F"]
    \end{centertikzcd}
    \item Für alle $(\Cat \xrightarrow{F} \Dat \xrightarrow{G} \Eat) \in \C$ kommutiert
    \begin{centertikzcd}[column sep=1cm, row sep=0.5cm]
      G \circ (\Id_\Dat \circ F) \arrow[rr, "\alpha_{\Cat,\Dat,\Dat,\Eat}"] \arrow[rd, "G \lambda_{\Cat, \Dat}"] &&
      (G \circ \Id_\Dat) \circ F \arrow[ld, "\rho_{\Dat, \Eat} F"] \\
      & G \circ F
    \end{centertikzcd}
  \end{itemize}
\end{defn}

% TODO: Bemerkung: Es gibt drei assoziierte "duale" Kategorien

\begin{bspe}
  \begin{itemize}
    \item Die Kategorie $\CatC$ der Kategorien ist eine 2-Kategorie.
    \item Jede Kategorie $\Cat$ ist natürlich eine 2-Kategorie.
    \item Die Kategorie der Ringe $\R$ mit $\Ob(\R) \coloneqq \{ \, \text{Ringe mit Eins} \, \}$ und $\Hom_\R(A, B) \coloneqq \text{Kat. der $B$-$A$-Bimoduln}$ mit $N \circ M \coloneqq N \otimes_B M$ für $M \in \Hom(A, B)$ und $N \in \Hom(B, C)$. Dabei ist $\Id_A \coloneqq A$. %als $A$-$A$-Bimodul.
  \end{itemize}
\end{bspe}

\begin{defn}
  Eine \emph{monoidale Kategorie} ist eine 2-Kategorie mit genau einem Objekt.
  In der Regel wird dann $\otimes$ anstelle von $\circ$ geschrieben.
\end{defn}

% §2. Universelle Eigenschaften

% Motto: Interessante Objekte einer mathematischen Theorie werden durch universelle Eigenschaften definiert

% Ausgelassen: Definition "darstellen"

% Ausgelassen: Limiten, terminales/initiales Objekt

% Ausgelassen: Produkt, Koprodukt, Differenzkern, Kodifferenzkern

\begin{defn}
  Sei $S : \Cat^\op \times \Cat \to \Aat$ ein Funktor. Ein \emph{Ende} $E \in \Ob(\Aat)$ von $S$ ist eine Familie $\alpha_c : E \to S(c, c)$, $c \in \Ob(\Cat)$ von Morphismen in $\Aat$, sodass für alle $(f : c \to c') \in \Cat$ das Diagramm
  \begin{centertikzcd}[column sep=1.4cm, row sep=0.05cm]
    & S(c, c) \arrow[rd, "{S(\id_c, f)}"] \\
    E \arrow[ru, "\alpha_c"] \arrow[rd, "\alpha_{c'}"{below}] && S(c, c') \\
    & S(c', c') \arrow[ru, "{S(f, \id_{c'})}"{below}]
  \end{centertikzcd}
  kommutiert, und $E$ universell (terminal) mit dieser Eigenschaft ist. \\
  Sprechweise: Ein Ende ist ein terminaler $S$-\emph{Keil}.
\end{defn}

\begin{nota}
  $E = \EndC{c}{S(c,c)}$.
\end{nota}

\begin{bem}
  Enden sind spezielle Limiten, und umgekehrt sind Limiten spezielle Enden: $\lim F = \EndC{c}{F(c)}$; der Integrand ist $\Cat^\op \times \Cat \to \Cat \stackrel{F}{\to} \Aat$.
\end{bem}

\begin{bem}
  Das duale Konzept ist das eines \cancel{Anfangs} Koendes $\CoEndC{c} S(c, c)$.
\end{bem}

\begin{bsp}
  Seien $F, G : \Cat \to \Aat$ zwei Funktoren. Dann ist
  %$\Hom_\Aat(F(\blank), G(\blank)) : \Cat^\op \times \Cat \to \SetC$ ein Funktor mit Ende
  \[ \EndC{c}{\Hom_\Aat(F(c), G(c))} \enspace\cong\enspace \Nat(F, G). \]
\end{bsp}

\begin{satz}[Fubini]
  Sei $S : \Dat^\op \!\times\! \Dat \!\times\! \Cat^\op \!\times\! \Cat \to \Aat$ ein Funktor. Dann gilt
  \[ \EndC{(d,c)}{S(d,d,c,c)} \enspace\cong\enspace \EndC{d}{\EndC{c}{S(d,d,c,c)}}, \]
  falls die rechte Seite und $\EndC{c}{S(d,d',c,c)}$ für alle $d, d' \in \Dat$ existieren.
\end{satz}

\begin{bsp}
  Sei $R$ ein Ring, aufgefasst als präadditive Kategorie mit einem Objekt $*$.
  Ein additiver Funktor $R^{(\op)} \to \AbGrp$ ist nichts anderes als ein $R$-Linksmodul (bzw. $R$-Rechtsmodul). Dann ist
  \[ A \otimes_R B \cong \CoEndC{* \in R}{A \otimes_{\Z} B}. \]
\end{bsp}

% Übungsblatt 2, Aufgabe 4
\begin{lem}[Ninja-Yoneda-Lemma]
  Für jede Prägarbe $F : \Cat^\op \to \SetC$ gilt
  \[ F \cong \CoEndC{c}{F(c) \times \Hom_\Cat(\blank, c)}. \]
\end{lem}

\begin{defn}
  Sei $\C$ eine 2-Kategorie. Seien $\Cat, \Dat \in \C$. Eine \emph{Adjunktion} von $F \in \Hom_\C(\Cat, \Dat)$ und $G \in \Hom_\C(\Dat, \Cat)$ ist geg. durch Morphismen $\eta : \Id_\Cat \Rightarrow G \circ F$ (genannt \emph{Eins}) und $\epsilon : F \circ G \Rightarrow \Id_\Dat$ (\emph{Koeins}) mit $G \epsilon \circ \eta G = \Id_G$ und $\epsilon F \circ F \eta = \Id_F$.
  Man notiert $F \ladj G$.
\end{defn}

\begin{lem}
  R/L-Adjungierte sind eindeutig bis auf eindeutige Isomorphie.
\end{lem}

\begin{bem}
  Seien $F : \Cat \to \Dat$ und $G : \Dat \to \Cat$ Funktoren. Dann gilt $F \ladj G$ genau dann, wenn es einen nat. Iso zwischen den Hom-Mengen gibt:
  \[ \Hom(F \circ \blank, \blank) \cong \Hom(\blank, G \circ \blank) \]
\end{bem}

\begin{bsp}
  $\exists \,_f \ladj f^* \ladj \forall \,_f$
\end{bsp}

\begin{bsp}
  Betrachte die 2-Kat. der Ringe. Dann gilt: Ein $B$-$A$-Modul $M$ ist genau dann ein Linksadjungierter, wenn $M$ als Rechts-$A$-Modul endlich erzeugt und projektiv ist.
\end{bsp}

\begin{bem}
  Sind $\eta$ und $\epsilon$ in $F \ladj G$ sogar Isomorphismen, so heißt $F \ladj G$ auch \emph{adjungierte Äquivalenz}. Jede beliebige Äquivalenz lässt sich stets (unter Beibehaltung von $F$ und $G$ sowie einem der Morphismen $\epsilon$, $\eta$) zu einer adj. Äquivalenz verfeinern.
\end{bem}

\vfill
\columnbreak

\subsection{Kan-Erweiterungen}

\begin{defn}
  Sei $\Aat \xleftarrow{T} \ModC \xrightarrow{K} \Cat$ ein Ausschnitt einer 2-Kategorie. Eine \emph{Rechts-Kan-Erw.} (RKE) $(R, \epsilon)$ von $T$ längs $K$ besteht aus
  \begin{itemize}
    \miniitem{0.45 \linewidth}{einem Morph. $R : \Cat \to \Aat$}
    \miniitem{0.48 \linewidth}{einem 2-Morph. $\epsilon : R \circ K \Rightarrow T$,}
  \end{itemize}
  sodass gilt:
  % $(R, \epsilon)$ unter diesen Daten terminal ist:
  Für alle Möchtegern-RKE $(S : \Cat \to \Aat, \eta : S \circ K \Rightarrow T)$ gibt es genau ein $\sigma : S \Rightarrow R$ mit $\epsilon \circ \sigma K = \eta$.
  Notation: $R = \Ran_K(T)$
\end{defn}

\begin{bem}
  Es sind äquivalent: \enspace
  \inlineitem{$(R, \epsilon)$ ist RKE von $T$ längs $K$} \\
  \inlineitem{$\eta \mapsto \epsilon \circ \eta K : \Nat(S, R) \to \Nat(S \circ K, T)$ ist bij. $\forall \, S : \Cat \to \Aat$}
\end{bem}

% Bemerkung vom Guide zum dritten Übungsblatt
\begin{bem}
  Es gilt $R = \Ran_K(T)$ genau dann, wenn es in $S \in [\Cat, \Aat]$ natürliche Isomorphismen
  $\Nat(S, R) \cong \Nat(S \circ K, T)$ gibt.
\end{bem}

\begin{prop}
  RKEs sind eindeutig bis auf eindeutige Isomorphie.
\end{prop}

\begin{bspe}
  \begin{itemize}
    \item Die RKE eines bel. Morphismus $T : \ModC \to \Aat$ längs $\Id_\ModC$ existiert stets und ist gegeben durch $(T, T \circ \Id_M \Rightarrow T)$.
    \item In der 2-Kategorie der Ringe existieren alle RKE:
    \[ \Ran_K(T) = (\Hom_M(K, T), \enspace ev : \Hom_\ModC(K, T) \otimes_C K \Rightarrow T). \]
  \end{itemize}
\end{bspe}

\begin{bsp}
  Sei $K : \ModC \to \mathbf{1}, \enspace * \mapsto 1$ und $T : \ModC \to \Aat$ irgendein Funktor. Dann ist eine RKE von $T$ längs $K$ dasselbe wie ein Limes von $T$.
\end{bsp}

\begin{thm}
  Seien $K : \ModC \to \Cat$ und $T : \ModC \to \Aat$ Funktoren.
  Existiere für alle $c \in \Ob(\Cat)$ der Limes
  $R(c) \coloneqq \Lim ((f : c \to Km) \mapsto Tm)$. \\
  Dabei ist die Indexkategorie des Limes die Kommakat. $\Delta(c) \downarrow K$. \\
  Dann lässt sich diese Setzung zu einem Funktor $\Cat \to \Aat$ ausdehnen und zwar zu einer RKE von $T$ längs $K$.
\end{thm}

\begin{bem}
  Ist $\ModC$ klein und $\Cat$ lokal klein und ist $\Aat$ vollständig, so sind die Voraussetzungen des Theorems für jeden Funktor $K : \ModC \to \Cat$, $T : \ModC \to \Aat$ erfüllt. Insbesondere ist dann jede solche RKE von der Form im Theorem. Solche RKE heißen auch \emph{punktweise RKE}.
\end{bem}

\begin{lem}
  Eine RKE ist genau dann punktweise, wenn sie für alle $a \in \Ob(\Aat)$ unter dem Funktor $\Hom_\Aat(a, \blank)$ erhalten bleibt.
\end{lem}

\begin{thm}
  Sei $K \!:\! M \to C$ ein Funktor. Betrachte $K^* \!:\! [\Cat, \Aat] \to [\ModC, \Aat]$.
  \begin{itemize}
    \item Wenn ein Funktor $\Ran_K : [\ModC,\Aat] \to [\Cat, \Aat]$ mit $K^* \ladj \Ran_K$ ex., so ist $\Ran_K(T)$ für alle $T : \ModC \!\to\! \Aat$ eine RKE von $T$ längs $K$.
    \item Existiere für alle $T : \ModC \to \Aat$ eine RKE $\Ran_K(T)$. Dann kann man die Zuordnung $T \mapsto \Ran_K(T)$ zu einem Rechtsadjungierten von $K^*$ ausdehnen.
  \end{itemize}
\end{thm}

\begin{thm}
  Sei $G : \Aat \to \Xat$ in einer 2-Kategorie. Dann sind äquivalent:
  \begin{itemize}
    \item $G$ besitzt einen Linksadjungierten.
    \item $\Ran_G(\Id_\Aat)$ existiert und $G \circ \Ran_G(\Id_\Aat) = \Ran_G(G \circ \Id_\Aat)$.
  \end{itemize}
  In diesem Fall gilt $\Ran_G(\Id_\Aat) \ladj G$ und $\Ran_G(\Id_\Aat)$ wird sogar von allen Morphismen $H : \Aat \to \Yat$ bewahrt.
\end{thm}

\begin{thm}
  Rechtsadjungierte bewahren RKE.
\end{thm}

\begin{kor}
  Rechtsadjungierte bewahren Limiten (RAPL)
\end{kor}

\iffalse
\begin{bsp}
  Sei $G : \Field \to \Ring$ der Vergissfunktor. Dann ist
  \[ (\Ran_G(G))(R) = \prod_{\mathfrak{p} \subset R} \Quot(R/{\mathfrak{p}}). \]
\end{bsp}
\fi

% §. Algebraische Strukturen in Kategorien
\subsection{Algebraische Strukturen in Kategorien}

\begin{defn}
  Eine \emph{Retrakt} ist ein Morphismus $r : Y \to X$, sodass ein Morphismus $i : X \to Y$ mit $r \circ i = \id_X$ existiert.
  Sprechweise: $X$ ist ein Retrakt von $Y$ (vermöge $i$).
\end{defn}

% Ausgelassen: $S^1 \hookrightarrow \R^2$ besitzt kein Linksinverses, $S^1 \hookrightarrow \R^2 \setminus \{ 0 \}$ schon.

\begin{bsp}
  Ein Modul $U$ ist genau dann Retrakt von einem Modul $M$, wenn $U$ ein direkter Summand von $M$ ist.
\end{bsp}

\begin{prop}
  "`$\blank$ ist Retrakt von $\blank$\,"' ist eine reflexive und trans. Relation.
\end{prop}

\begin{defn}
  Ein \emph{Retrakt eines Morphismus} $(A \xrightarrow{f} B) \!\in\! \Cat$ ist ein Morph. $g : X \to Y$, sodass es ein komm. Diagramm folgender Form gibt:
  \vspace{-8pt}
  \begin{centertikzcd}
    A \arrow[r, "i"] \arrow[d, "f"] &
    X \arrow[r, "r"] \arrow[d, "g"] &
    A \arrow[d, "f"] \\
    B \arrow[r, "j"] &
    Y \arrow[r, "s"] &
    B
  \end{centertikzcd}
  \vspace{-8pt}
\end{defn}

\begin{bem}
  Ein Retrakt von $f \in \Mor(\Cat)$ ist ein Retrakt von $f \in \Ob(\Cat^{\to})$.
\end{bem}

\begin{prop}
  \begin{itemize}
    \item Retrakte von Isomorphismen sind Isomorphismen.
    \item Sei $f \circ g = \id$. Dann ist $f$ ein Retrakt von $g \circ f$.
  \end{itemize}
\end{prop}

% Ausgelassen: Retrakte sind abgeschlossen unter Komposition mit Isos

\begin{prop}
  Sei $F : \Cat \to \Dat$ ein Funktor. Dann ist die Klasse $\Set{f \in \Mor(\Cat)}{F(f) \text{ ist ein Iso}}$ abgeschlossen unter Retrakten.
\end{prop}

\begin{defn}
  Sei $i : A \to X$ und $p : E \to B$. Dann werden als äq. definiert:
  \begin{itemize}
    \miniitem{0.3 \linewidth}{$p$ ist \emph{$i$-injektiv}}
    \miniitem{0.35 \linewidth}{$i$ ist \emph{$p$-projektiv}}
    \miniitem{0.3 \linewidth}{$i \lhhe p$}
    \item $i$ hat die Linkshochhebungseigenschaft (LHHE) bzgl. $p$
    \item $p$ hat die Rechtshochhebungseigenschaft (RHHE) bzgl. $i$
    \item Für alle $f$, $g$ wie unten, sodass das Quadrat kommutiert, gibt es ein diagonales $\lambda$, sodass die Dreiecke kommutieren:
    \begin{centertikzcd}
      A \arrow[r, "g"] \arrow[d, "i"] &
      E \arrow[d, "p"] \\
      X \arrow[r, "f"] \arrow[ur, "\exists\, \lambda", dashed] &
      B
    \end{centertikzcd}
  \end{itemize}
\end{defn}

\begin{bsp}
  Wegeliftung aus der Topologie: $i : \{ 0 \} \to \cinterval{0}{1}$ erfüllt die LHHE bezüglich allen Überlagerungen $\pi : E \to B$.
\end{bsp}

\begin{samepage}
\begin{bsp}
  Ein Objekt $P$ einer ab. Kat. $\Aat$ ist genau dann \emph{projektiv}, wenn $(0 \to P)$ die LHHE bzgl. aller Epis in $\Aat$ hat.
  Dual ist $I \!\in\! \Ob(\Aat)$ injektiv g.d.w. alle Monos in $\Aat$ die LHHE bzgl. $(I \to 0)$ besitzen.
\end{bsp}

\begin{bsp}
  In $\SetC$ gilt: Alle Inj. haben die LHHE bzgl. aller Surjektionen.
\end{bsp}

\begin{lem}[\emph{Retrakt-Argument}]\mbox{}
  Sei $f \!=\! q \circ j$.
  \begin{itemize}
    \item Ist $f$ $q$-projektiv ($f \lhhe q$), so ist $f$ ein Retrakt von $j$.
    \item Ist $f$ $j$-injektiv ($j \lhhe f$), so ist $f$ ein Retrakt von $q$.
  \end{itemize}
\end{lem}

\subsection{Zellenkomplexe}

\end{samepage}

\begin{defn}
  Sei $\lambda$ eine Ordinalzahl. Eine \emph{$\lambda$-Sequenz} in einer Kategorie $\Cat$ ist ein kolimesbewahrender Funktor $X : \lambda \to \Cat$ (wobei man $\lambda$ als Präordnungskategorie aller $\beta < \lambda$ auffasst).
  Ihre \emph{transfinite Komposition} ist der induzierte Morphismus $X_0 \to \colim_{\beta < \lambda} X_\beta$.
\end{defn}

\begin{bem}
  Kolimesbewahrung bedeutet: $\colim_{\alpha < \beta} X_\alpha \!=\! X_\beta$ für alle $\beta \!<\! \lambda$.
\end{bem}

% Ausgelassen: Bsp $n$-Sequenz ($n \in \N$)

% Ausgelassen: Verklebung von zwei Halbsphären entlang $S^1$ zu einer $S^2$
% Verklebung von $[0, 1]$ an den Endpunkten zur $S^1$

\begin{defn}
  Sei $\Cat$ eine kovollständige Kategorie, $I \subset \Mor(\Cat)$ eine Menge.
  %Sei $I$ eine Menge von Morphismen in einer kovollständigen Kategorie $\Cat$.
  \begin{itemize}
    \item Ein \emph{relativer $I$-Zellenkomplex} ist eine transf. Komp. einer $\lambda$-Sequenz $Z$, sodass $\forall \, \alpha \!\in\! \Ord$ mit $\alpha \!+\! 1 \!<\! \lambda$ ein Pushoutdiagramm
    \begin{centertikzcd}[row sep=0.3cm]
      C \arrow[r] \arrow[d, "f"] \arrow[rd, phantom, "\ulcorner", very near end] &
      Z_\alpha \arrow[d] &
      \text{$\leftarrowtriangle$ \emph{Anklebeabbildung}} \\
      B \arrow[r] &
      Z_{\alpha + 1} &
      \mathrlap{\text{$\leftarrowtriangle$ \emph{Zelle}}}
      \phantom{\text{$\leftarrowtriangle$ \emph{Anklebeabbildung}}}
    \end{centertikzcd}
    mit $f \in I$ existiert. Sprechweise: \\
    "`$Z_{\alpha+1}$ entsteht aus $Z_\alpha$, indem wir $B$ längs $C$ ankleben"'
    \item Ein Objekt $A \in \Ob(\Cat)$ heißt \emph{$I$-Zellenkomplex}, wenn der Morph. $0 \to A$ aus dem initialen Obj. ein relativer $I$-Zellenkomplex ist.
  \end{itemize}
\end{defn}

\begin{bsp}
  CW-Komplexe aus der algebraischen Topologie sind $I$-Zellenkomplexe mit
  $I \coloneqq \Set{S^{n-1} \hookrightarrow B^n}{n \geq 0}$
  (und $\Cat = \Top$).
\end{bsp}

\begin{bspe}
  \begin{itemize}
    \item Identitäten $A \to A$ sind relative $I$-Zellenkomplexe.
    \item Das initiale Objekt ist ein absoluter $I$-Zellenkomplex.
  \end{itemize}
\end{bspe}

\begin{samepage}
\begin{lem}
  Sei $Z : \lambda \to \Cat$ eine $\lambda$-Sequenz.
  Sei jeder Morphismus $Z_\beta \to Z_{\beta + 1}$ ($\beta + 1 < \lambda$) ein Pushout eines Morphismus aus $I$. \\
  Dann ist die transfinite Komposition von $Z$ ein $I$-Zellenkomplex.
\end{lem}

\begin{thm}
  Die Klasse der relativen $I$-Zellenkomplex ist abgeschl. unter: \\
  \inlineitem{transfiniten Kompositionen} \enspace
  \inlineitem{Isomorphismen} \enspace
  \inlineitem{Koprodukten}
\end{thm}

% Vorlesung vom 6.5.2015
\subsection{Faktorisierungssysteme}
\end{samepage}

\begin{defn}
  Eine Unterkat. $\Lat \subseteq \Cat$ heißt \emph{links-saturiert}, falls $\Lat$ abgeschl. ist unter Pushouts, transfiniten Kompositionen und Retrakten.
\end{defn}

\begin{lem}
  Sei $\Lat \subseteq \Cat$ links-saturiert. Dann ist $\Lat$ unter Koprodukten abgeschlossen und enthält alle Isomorphismen.
\end{lem}

\begin{bsp}
  Sei $R \subset \Mor(\Cat)$. Dann ist die Unterkategorie $\Lat \subseteq \Cat$ mit
  $\Mor(\Lat) \coloneqq \prescript{\lhhe}{} R \coloneqq \Set{i \!\in\! \Mor(\Cat)}{\fa{r \!\in\! R\!}\! i \lhhe r}$
  links-saturiert.
\end{bsp}

% Schwache Faktorisierungssysteme

% Ausgelassen, da trivial
\iffalse
\begin{prop}
  Für $L_1 \subseteq L_2 \subseteq \Mor(\Cat)$ gilt $L_1^\lhhe \leq L_2^\lhhe$.
\end{prop}
\fi

\begin{defn}
  \begin{itemize}
    \item $L \subseteq \Mor(\Cat)$ heißt \emph{proj. abgeschlossen}, falls $L \supseteq \prescript{\lhhe}{} (L^\lhhe)$.
    \item $R \subseteq \Mor(\Cat)$ heißt \emph{injektiv abgeschlossen}, falls $R \supseteq (\prescript{\lhhe}{} L)^\lhhe$.
  \end{itemize}
\end{defn}

\begin{prop}
  \begin{itemize}
    \item $\prescript{\lhhe}{}(L^\lhhe)$ ist die projektive Hülle von $L$, \dh{} die kleinste Klasse von Morphismen, die projektiv abgeschl. ist und $L$ umfasst.
    \item Die projektive Hülle von $L$ ist links-saturiert.
    Ist $L$ schon projektiv abgeschlossen, so ist $L$ insbesondere links-saturiert.
  \end{itemize}
\end{prop}

\begin{defn}
  \begin{itemize}
    \item Ein Paar $(L, R)$ von Klassen von Morphismen von $\Cat$ \emph{faktorisiert} $\Cat$, falls
    $\fa{f \in \Mor(\Cat)} \ex{i \in L, p \in R} f = p \circ i$.
    \item Ein faktorisierendes Paar $(L, R)$ heißt \emph{schwaches Faktorisierungssystem} (SFS), falls $L = \prescript{\lhhe}{} R$ und $R = L^\lhhe$.
    \item Ein SFS $(L, R)$ heißt \emph{orth. Faktorisierungssystem}, falls jedes $i \!\in\! L$ die eindeutige LHHE bzgl. allen $p \in R$ erfüllt.
  \end{itemize}
\end{defn}

\begin{prop}
  Sei $(L, R)$ faktorisierend. Dann ist $(L, R)$ genau dann ein SFS, wenn $L \lhhe R$ und $L$ und $R$ unter Retrakten abgeschlossen sind.
  % L \lhhe R \coloniff \fa{i \in L} \fa{p \in R} i \lhhe p
\end{prop}

\begin{bsp}
  $(\{ \, \text{Surjektionen} \, \}, \{ \, \text{Injektionen} \, \})$ ist ein (S)FS in $\SetC$
\end{bsp}

\vfill
\columnbreak

% §4. Modellkategorien
\section{Modellkategorien}

\begin{motto}
  Modellkat. sind ein Werkzeug, math. Theorien zu studieren.
\end{motto}

\begin{defn}
  Eine Klasse $W \subseteq \Mor(\Cat)$ von Morphismen erfüllt die \emph{2-aus-3-Eigenschaft}, falls für jede Komposition $h = g \circ f$ in $\Cat$ gilt: Liegen zwei der drei Morphismen $f$, $g$, $h$ in $W$, so auch der dritte.
\end{defn}

\begin{defn}
  $\Weak \subseteq \Cat$ wie eben heißt \emph{Unterkat. schwacher Äquivalenzen}, falls $\Weak$ die 2-aus-3-Eig. erfüllt und abgeschl. unter Retrakten ist.
\end{defn}

\begin{bsp}
  Sei $F : \Cat \to \Dat$ ein Funktor. Dann ist $\Weak \coloneqq F^{-1}(\{ \, \text{Isos in $\Dat$} \, \})$ eine Unterkategorie schwacher Äquivalenzen.
\end{bsp}

\begin{defn}
  Ein Tripel $(\Weak, \Cof, \Fib)$ von Unterkategorien einer Kategorie $\ModC$ heißt \emph{Modellstruktur} auf $\ModC$, falls sowohl $(\Cof, \Fib \cap \Weak)$ als auch $(\Cof \cap \Weak, \Fib)$ schwache Faktorisierungssysteme sind und $\Weak$ die 2-aus-3-Eigenschaft erfüllt.
\end{defn}

\begin{defn}
  Eine bivollständige Kategorie $\ModC$ zusammen mit einer Modellstruktur \ModStr{} heißt eine \emph{Modellkategorie}.
\end{defn}

\begin{sprech}
  Man verwendet folgende Bezeichnungen und Pfeile:
  \begin{center}
    \begin{tabular}{r l l}
      $\Weak$ & $\xrightarrow{{\sim}}$ & \emph{schwache Äquivalenz} \\
      $\Cof$ & $\xhookrightarrow{\phantom{{\sim}}}$ & \emph{Kofaserung} \\
      $\Cof \cap \Weak$ & $\xhookrightarrow{{\sim}}$ & \emph{azyklische Kofaserung} \\
      $\Fib$ & $\xtwoheadrightarrow{\phantom{{\sim}}}$ & \emph{Faserung} \\
      $\Fib \cap \Weak$ & $\xtwoheadrightarrow{{\sim}}$ & \emph{azyklische Faserung}
    \end{tabular}
  \end{center}
\end{sprech}

\begin{bem}
  Ist $(\Weak, \Cof, \Fib)$ eine Modellstruktur auf $\ModC$, so ist $(\Weak^\op, \Fib^\op, \Cof^\op)$ eine Modellstruktur auf $\ModC^\op$.
\end{bem}

\begin{bem}
  Wegen $\Cof = \prescript{\lhhe}{} (\Fib \cap \Weak)$ bzw. $\Fib = (\Cof \cap \Weak)^\lhhe$ ist das Datum $(\Weak, \Cof, \Fib)$ überbestimmt.
\end{bem}

\begin{bsp}
  Sei $\ModC$ bivollständig. Sei $\Weak \coloneqq \Cof \coloneqq \{ \, \text{Isos in $\ModC$} \, \}$. \\
  Dann wird $\ModC$ mit $\Fib \coloneqq \ModC$ eine Modellkategorie.
\end{bsp}

\begin{prop}
  In einer Modellkategorie sind $\Cof$ und $\Cof \cap \Weak$ links-saturiert.
\end{prop}

\begin{lem}
  $\Weak$ enthält alle Isomorphismen und ist unter Retrakten abgeschlossen, bildet also eine Unterkat. schwacher Äquivalenzen.
\end{lem}

\begin{nota}
  Das initiale Objekt von $\ModC$ wird mit $\emptyset$, das terminale Objekt mit $*$ bezeichnet.
  %Mit $\emptyset$ wird das initiale Objekt und mit $*$ das terminale Objekt von $\ModC$ bezeichnet.
\end{nota}

\begin{defn}
  \begin{itemize}
    \item Ein Objekt $X \in \Ob(\ModC)$ heißt \emph{kofasernd}, falls $\emptyset \to X$ eine Kofaserung ist.
    Eine azyklische Faserung $q : QX \xtwoheadrightarrow{{\sim}} X$ mit $QX$ kofasernd heißt \emph{kofasernder Ersatz} (oder Approx.) von $X$. \\
    \item Dual heißt $X \in \Ob(\ModC)$ \emph{fasernd}, falls $X$ in $\ModC^\op$ kofasernd ist und $X \xhookrightarrow{{\sim}} RX$ mit $RX$ fasernd heißt \emph{fasernder Ersatz} von $X$.
  \end{itemize}
\end{defn}

\begin{bsp}
  Sei $X \in \Ob(\ModC)$ beliebig. Dann faktorisiere $\emptyset \to X$ wie folgt:
  \begin{centertikzcd}[row sep=0.2cm]
    \emptyset \arrow[rr] \arrow[rd, hook] && X \\
    & QX \arrow[ru, twoheadrightarrow, swap, "\sim"]
  \end{centertikzcd}
  Man erhält also immer einen kofasernden Ersatz $QX$ für $X$. \\
  Dual gibt es immer einen fasernden Ersatz $RX$ für $X$.
\end{bsp}

\begin{prop}
  Seien $q : QX \xtwoheadrightarrow{{\sim}} X$ und $q' : Q' X \xtwoheadrightarrow{{\sim}} X$ zwei kofasernde Approximationen von $X$. Dann existiert eine schwache Äquivalenz $\xi : QX \xrightarrow{{\sim}} Q' X$ mit $q' \circ \xi = q$.
\end{prop}

\begin{defn}
  Ein Obj. $X$ heißt \emph{bifasernd}, falls es fasernd und kofasernd ist.
\end{defn}

\begin{prop}
  Für alle $X \in \Ob(\ModC)$ sind $RQX$ und $QRX$ schwach äquivalent und beide bifasernd.
\end{prop}

\begin{lem}[Ken Brown]
  Sei $F : \ModC \to \NatC$ ein Funktor, $\ModC$ eine Modell- kategorie, $\NatC$ besitze eine Unterkat. $\Weak'$ schwacher Äquivalenzen. \\
  Wenn $F$ azyklische Kofaserungen zwischen kofasernden Objekten nach $\Weak'$ abbildet, so bildet $F$ alle schwachen Äquivalenzen zwischen kofasernden Objekten nach $\Weak'$ ab.
\end{lem}

% Homotopien in Modellkategorien

\begin{defn}
  Sei $\ModC$ eine Modellkategorie.
  Ein \emph{Zylinderobjekt} $\Cyl{X}$ zu einem $X \!\in\! \Ob(\ModC)$ ist ein Obj. zusammen mit Morphismen wie folgt:
  \begin{centertikzcd}[row sep=0.2cm]
    X \arrow{rd}{i_0}[swap]{{\sim}} \arrow[rrd, bend left, "\id"] \\
    & \Cyl{X} \arrow{r}{p}[swap]{{\sim}} & X \\
    X \arrow{ru}{{\sim}}[swap]{i_1} \arrow[rru, bend right, swap, "\id"]
  \end{centertikzcd}
  Der Zylinder $\Cyl{X}$ heißt \emph{gut}, falls $X \amalg X \to \Cyl{X}$ eine Kofaserung ist.
  Ein guter Zylinder heißt \emph{sehr gut}, falls $p : \Cyl{X} \to X$ eine azyklische Faserung ist.
\end{defn}

\begin{bem}
  Sei die Kodiagonale $\nabla : X \amalg X \to X$ wie folgt faktorisiert:
  \begin{centertikzcd}[row sep=0.2cm]
    X \arrow[rd, hook] \\
    & X \amalg X \arrow[r, hook] \arrow[rr, bend left, "\nabla"] & \Cyl{X} \arrow[r, twoheadrightarrow, "\sim"] & X \\
    X \arrow[ru, hook]
  \end{centertikzcd}
  Dann erhalten wir ein Zylinderobjekt $\Cyl{X}$ für $X$. \\
\end{bem}

\begin{defn}
  Zwei Morphismen $f, g : X \to Y$ in $\ModC$ heißen \emph{links-homotop} (notiert $f \sim^l g$), falls ein Zylinder $\Cyl{X}$ und ein Diagramm der Form
  \begin{centertikzcd}[row sep=0.2cm]
    X \arrow{rd}[swap]{\sim} \arrow[rrd, bend left, "f"] \\
    & \Cyl{X} \arrow[r, dashed, "h"] & Y \\
    X \arrow{ru}{\sim} \arrow[rru, bend right, swap, "g"]
  \end{centertikzcd}
  existiert. Wir definieren $\pi^l(X, Y) \coloneqq \Hom_\ModC(X, Y) / \langle {\sim^l} \rangle$, wobei $\langle {\sim^l} \rangle$ die von der symmetrischen, refl. Relation ${\sim^l}$ erzeugte Äq'relation ist. \\
  Die Homotopie heißt (sehr) gut, wenn der Zylinder $\Cyl{X}$ es ist.
\end{defn}

\begin{beob}
  Sei $X \amalg X \xrightarrow{i} C \xrightarrow{p \, \sim} X$ irgendein Zylinderobjekt.
  Faktorisiere $i = q \circ i'$ in Kofaserung und azyklische Faserung. Dann ist auch
  \[ X \amalg X \xhookrightarrow{i'} X' \xrightarrow{pq \, \sim} X \]
  ein Zylinderobjekt, sogar ein gutes.
  Ebenso kann man $p$ faktorisieren und ein ein anderes Zylinderobjekt erhalten.
\end{beob}

\begin{lem}
  Sei $X$ kofasernd, $X \amalg X \to \Cyl{X} \to X$ ein gutes Zylinderobj. \\
  Dann sind $i_{0,1} : X \to X \amalg X \to \Cyl{X}$ azyklische Kofaserungen.
\end{lem}

\begin{lem}
  Sei $h : f \simeq^l g$. Dann: $f \in \Weak \iff g \in \Weak$.
\end{lem}

\begin{defn}
  Ein \emph{Pfadobjekt} $\PO{X}$ ist eine Faktorisierung
  \[ X \xrightarrow[\sim]{i} \PO{X} \xrightarrow{p} X \times X \]
  des Diagonalmorph. $\Delta : X \to X \times X$.
  Das Pfadobjekt $\PO{X}$ heißt gut, wenn $p$ eine Faserung und sehr gut, wenn zus. $i$ eine Kofaserung ist.
\end{defn}

\begin{defn}
  Eine Rechtshomotopie $h : f \simeq^r g$ ist ein Diagramm der Form
  \begin{centertikzcd}[row sep=0.2cm]
    && Y \\
    X \arrow[r, "h"] \arrow[rru, bend left, "f"] \arrow[rrd, bend right, "g"] & Y^I \arrow[ru, "p_0"] \arrow[rd, "p_1"] \\
    && Y
  \end{centertikzcd}
\end{defn}

\begin{bem}
  Ein Pfadobj. in $\ModC$ ist dasselbe wie ein Zylinderobj. in $\ModC^\op$.
\end{bem}

% Ausgelassen: Dualisierte Versionen der techn. Lemmate über Zylinderobjekte

\begin{lem}
  Seien $f, g : X \to Y$ und $e : W \to X$, $d : Y \to Z$.
  \begin{itemize}
    \item $\exists \, h : f \simeq^l g \iff \exists \, h' : f \simeq^{l,\gut} g$.
    \item Sei $Y$ fasernd. Dann: $\exists \, h : f \simeq^{l,\gut} g \iff \exists \, h' : f \simeq^{l,\sg} g$
    \item $\exists \, h : f \simeq^l g \implies \exists \, h' : d \circ f \simeq^l d \circ g$
    \item $\exists \, h : f \simeq^{l,\sg} g \implies \exists \, h' : f \circ e \simeq^{l,\sg} g \circ e$
    \item Sei $X$ kofasernd. Dann ist $\simeq^l$ eine Äq'relation auf $\Hom_\ModC(X, Y)$.
  \end{itemize}
\end{lem}

\begin{kor}
  Sei $Y$ fasernd. Dann induziert Komposition eine Abbildung
  \[
    \pi^l(X, Y) \times \pi^l(W, X) \to \pi^l(W, Y), \quad
    ([g], [f]) \mapsto [g \circ f].
  \]
\end{kor}

\begin{prop}
  Seien $f, g : X \to Y$.
  \begin{itemize}
    \item Sei $X$ kofasernd. Dann: $f \simeq^l g \implies f \simeq^r g$
    \item Sei $Y$ fasernd.\phantom{ko} Dann: $f \simeq^l g \impliedby f \simeq^r g$
  \end{itemize}
\end{prop}

\begin{nota}
  Wenn $X$ kofasernd und $Y$ fasernd ist, schreibt man
  \[ \pi(X, Y) \coloneqq \pi^l(X, Y) = \pi^r(X, Y). \]
\end{nota}

\begin{thm}
  Sei $X$ kofasernd. Sei $p : Z \xtwoheadrightarrow{\sim} Y$ eine azyklische Faserung. \\
  Dann ist $p_* : \pi^l(X, Z) \to \pi^l(X, Y), \, [f] \mapsto [p \circ f]$ eine Bijektion.
\end{thm}

\begin{thm}[\emph{Whitehead}]\mbox{}\\
  Für einen Morphismus $f : X \to Y$ zw. bifasernden Objekten gilt
  \begin{align*}
    f \in \Weak \iff & \text{$f$ ist eine Homotopieäquivalenz} \\
    \coloniff & \ex{g : Y \to X} g \circ f \simeq \id_X \wedge f \circ g \simeq \id_Y.
  \end{align*}
\end{thm}

\begin{lem}
  Sei $f : X \to Y$. Seien $RX$ und $RY$ fixierte fasernde Approx. an $X$ bzw. $Y$.
  Dann hängt $Rf : RX \to RY$ bis auf Rechts- und auch Linkshomotopie nur von der Rechtshomotopieklasse von $r \circ f$ ab.
\end{lem}

\begin{acht}
  I.\,A. ist $f \mapsto R(f)$ nicht funktoriell.
\end{acht}

\subsection{Die Homotopiekategorie einer Modellkategorie}

% Ausgelassen: Erinnerung an die Lokalisierung von Ringen

% Aufgabenstellung: Sei $\ModC$ eine Modellkategorie. Gibt es eine Kategorie $\Ho \ModC$ zusammen mit einem Funktor $\gamma : \ModC \to \Ho \ModC$, sodass gilt:
% 1. $\gamma$ schickt schwache Äquivalenzen auf Isos
% 2. $\gamma$ ist initial unter dieser Voraussetzung

\begin{defn}
  Sei $\Cat$ ein Kategorie, $S \subset \Mor(\Cat)$ eine Klasse von Morphismen. Die \emph{Lokalisierung} $\Cat[S^{-1}]$ von $\Cat$ ist eine Kategorie, die folgende 2-universelle Eigenschaft erfüllt:
  \begin{itemize}
    \item $\gamma : \Cat \to \Cat[S^{-1}]$ schickt Morphismen aus $S$ aus Isos.
    \item Für jede Kategorie $\Dat$ ist $\gamma^* : [\Cat[S^{-1}], \Dat] \to [\Cat, \Dat]_{S \mapsto \text{Isos}}$ eine Kategorienäquivalenz.
  \end{itemize}
\end{defn}

\begin{bem}
  Die \href{http://timbaumann.info/uni-spicker/homoalg.pdf}{Homologische-Algebra-Zusammenfassung} behandelt Lokalisierung von Kategorien.
\end{bem}

\begin{defn}
  Die \emph{Homotopiekategorie} $\Ho \ModC$ einer Modellkategorie $\ModC$ ist die Lokalisierung von $\ModC$ an der Klasse der schwachen Äquivalenzen.
\end{defn}

\begin{konstr}
  Ganz explizit:
  \begin{align*}
    \Ob(\Ho \ModC) & \coloneqq \Ob(\ModC) \\
    \Hom_{\Ho \ModC}(X, Y) & \coloneqq \pi(RQX, RQY)
  \end{align*}
  Nach einem früheren Lemma ist die Komposition $([f], [g]) \mapsto [f \circ g]$ wohldefiniert.
  Der Funktor $\gamma : \ModC \to \Ho \ModC$ ist gegeben durch
  \[
    X \mapsto X, \quad
    f \mapsto [RQf].
  \]
\end{konstr}

\begin{lem}
  Sei $f : X \to Y$ in $\ModC$. Dann gilt $f \!\in\! \Weak \Leftrightarrow Qf \!\in\! \Weak \Leftrightarrow RQf \!\in\! \Weak$.
\end{lem}

\begin{lem}
  $\gamma$ wie definiert ist ein Funktor.
\end{lem}

% Probleme der Lokalisierung:
% 1. Das ist keine Kategorie im Marc'schen Sinne (da Hom-Mengen keine Mengen)
% 2. Man hat keine explizite Beschreibung für die Morphismen in der lokalisierten Kategorie

\begin{lem}
  $f \in \Weak \iff \gamma(f)$ ist ein Iso.
\end{lem}

\begin{lem}
  Sei $X$ kofasernd und $Y$ fasernd. Dann ist die Abbildung
  \[
    \pi(X, Y) \to \Hom_{\Ho \ModC}(X, Y), \quad
    [f] \mapsto [RQf]
  \]
  eine Bijektion.
\end{lem}

\begin{lem}
  Ist $F : \ModC \to \Cat$ ein Funktor, der schwache Äq. auf Isos schickt, dann identifiziert $F$ links- bzw. rechtshomotope Morphismen.
\end{lem}

\begin{lem}
  Jeder Morphismus in $\Ho \ModC$ ist Komposition von Morphismen der Form $\gamma(f)$, $f \in \Mor(\ModC)$ und der Form $\gamma(f)^{-1}$, $f \in \Weak$.
\end{lem}

\begin{lem}
  Sei $\ModC_c \subset \ModC$ die volle Unterkategorie der kofasernden Objekte und $F : \ModC_c \to \Cat$ ein Funktor, der azyklische Kofaserungen auf Isos schickt. Dann identifiziert $F$ rechtshomotope Morphismen.
\end{lem}

\begin{thm}
  Ein Morphismus $p : Z \to Y$ zw. fasernden Objekten ist genau dann eine schwache Äquivalenz, wenn $p_* : \pi(X, Z) \to \pi(X, Y)$ bijektiv ist für alle kofasernden Objekte $X \in \ModC$.
\end{thm}

% §5. Kombinatorische und eigentliche Modellkategorien
\section{Klassen von Modellkategorien}

% §5.1
\subsection{Lokal präsentierbare Kategorien}

% soll andeuten: Die Objekte einer lokal präsentierbareen Kategorien sind präsentierbar, nicht die Kategorie als solche

\begin{motto}
  Eine lokal präsentierare Kategorie ist eine große Kategorie, welche erzeugt wird von kleinen Objekten unter kleinen Kolimiten.
\end{motto}

\begin{defn}
  Eine $\infty$-große Kardinalzahl $\kappa$ heißt \emph{regulär}, wenn die Vereinigung von weniger als $\kappa$ vielen Mengen, die alle weniger als $\kappa$-viele Elem. enthalten, selbst weniger als $\kappa$-viele Elemente enthält.
\end{defn}

\begin{bem}
  Zu jeder Kardinalzahl $\lambda$ existiert ein reguläres $\kappa$ mit $\lambda \leq \kappa$.
\end{bem}

\begin{defn}
  Sei $\kappa$ eine Kardinalzahl. Eine Kategorie heißt \emph{$\kappa$-klein}, falls sie nur $\kappa$-viele Morphismen besitzt.
\end{defn}

\begin{bem}
  Sei $\kappa$ regulär. Dann ist eine Kat. bereits dann $\kappa$-klein, falls sie nur $\kappa$-viele Objekte besitzt und alle Hom-Mengen $\kappa$-klein sind.
\end{bem}

\begin{defn}
  Eine Kategorie heißt \emph{$\kappa$-filtriert}, wobei $\kappa$ eine reguläre Kardinalzahl ist, wenn jedes $\alpha$-kleine Diagramm in der Kategorie einen Kokegel besitzt, wobei $\alpha < \kappa$.
\end{defn}

\begin{defn}
  Eine teilweise geordnete Menge $(I, \leq)$ heißt \emph{$\alpha$-gerichtet}, falls die zugehörige Kategorie $\alpha$-filtriert ist, \dh{} jeweils weniger als $\alpha$-viele Elemente haben eine obere Schranke.
\end{defn}

\begin{bem}
  Sei $\lambda \geq \kappa$. Dann ist jede $\lambda$-filtrierte Kategorie auch $\kappa$-filtriert.
\end{bem}

\begin{defn}
  Ein Objekt $X$ einer Kat. $\Cat$ heißt \emph{$\kappa$-kompakt} oder \emph{$\alpha$-klein}, wenn $\Hom(X, \blank) : \Cat \to \SetC$ mit $\kappa$-filtrierten Kolimiten vertauscht:
  \[
    \Colim_i \Hom_\Cat(X, T_i) \xrightarrow{\cong} \Hom_\Cat(X, \Colim_i T_i)
  \]
  für alle $\kappa$-filtrierte Diagramme $(T_i)_{i \in \Iat}$.
\end{defn}

\begin{defn}
  Ein Objekt heißt genau dann \emph{klein}, wenn es $\kappa$-kompakt ist für irgendeine reguläre Kardinalzahl $\kappa$.
\end{defn}

\iffalse
\begin{idee}
  Sei $X$ eine endliche Menge. Sei $X \subset \cup_{i \in \N} T_i$, $T_i \subseteq T_{i+1}$. Dann liegt $X$ schon vollständig in einem der $T_i$.
\end{idee}
\fi

% TODO: Was sind die interessantesten Beispiele?
\begin{bspe}
  \begin{itemize}
    \item Jede endliche Menge ist $\aleph_0$-kompakt in $\SetC$.
    \item Jeder endlich-dim. VR ist $\aleph_0$-kompakt in $\Vect(\R)$.
    \item Jeder endlich-präsentierte Modul ist $\aleph_0$-kompakt in $\Mod(R)$.
    \item Unendliche Mengen sind nicht $\aleph_0$-kompakt in $\SetC$.
    \item Jeder nicht diskrete topologische Raum ist nicht $\aleph_0$-kompakt.
    \item $\SetC$ ist lokal $\aleph_0$-präsentierbar mit $S = \{ \heartsuit \}$.
    \item $\Mod(R)$ ist lokal $\aleph_0$-präsentierbar mit $S = \Set{R^n/\im(A)}{n \geq 0, A \in R^{n \times m}, m \geq 0}$
  \end{itemize}
\end{bspe}

\begin{defn}
  Eine \emph{lokal $\kappa$-präsentierbare Kategorie} ist eine lokal kleine und kovollständige Kategorie, sodass eine Menge $S \subseteq \Ob(\Cat)$ von $\kappa$-kompakten Objekten existiert, sodass jedes Objekt aus $\Cat$ kleiner Kolimes von Objekten aus $S$ ist.
\end{defn}

% Ausgelassen: Intuition

\begin{defn}
  Eine Kategorie heißt genau dann \emph{lokal präsentierbar}, wenn sie lokal $\kappa$-präsentierbar für eine reguläre Kardinalzahl $\kappa$ ist.
\end{defn}

\begin{bspe}
  \begin{itemize}
    \item $\sSet$ ist lokal präsentierbar.
    \item Sei $\Cat$ klein. Dann ist $\PShSet(\Cat) \!=\! [\Cat^\op, \SetC]$ lokal präsentierbar.
    \item $\FinSetC$ ist nicht lokal präsentierbar (weil nicht kovollständig)
  \end{itemize}
\end{bspe}

\begin{fun}
  Sei $\Cat$ lokal präsentierbar. Wenn auch $\Cat^\op$ lokal präsen- tierbar ist, dann ist $\Cat$ die zu einer Quasiordnung gehörige Kategorie!
\end{fun}

\begin{lem}
  Sei $X : \Iat \times \Jat \to \SetC$ ein Funktor, wobei $\Iat$ $\alpha$-filtriert und $\Jat$ $\alpha$-klein.
  Dann ist der kanonische Isomorphismus $\colim_i \lim_j X(i, j) \to \lim_j \colim_i X(i, j)$ eine Bijektion.
\end{lem}

\begin{bsp}
  $\alpha$-kleine Kolimiten $\alpha$-kleiner Objekte sind wieder $\alpha$-kompakt.
\end{bsp}

\subsection{Kombinatorische Modellkategorien}

\begin{lem}[\emph{Kleines-Objekt-Argument}]\mbox{}\\
  Sei $\Cat$ lokal präsentierbar, $\Iat \subset \Mor(\Cat)$ eine Menge, $\Cell(\Iat)$ die Unterkat. der relativen $\Iat$-Zellenkomplexe und $\Coff(\Iat)$ die Unterkat. der Retrakte von $\Cell(\Iat)$.
  Dann ist $(\Coff(\Iat), \Iat^\lhhe)$ ein SFS.
\end{lem}

\begin{defn}
  Eine \emph{kombinatorische Modellkategorie} ist eine lokal präsentierte Modellkat. $\ModC$ mit Modellstr. \ModStr{}, sodass Mengen $\Iat, \Jat \subset \Mor(\ModC)$ existieren, sodass $\Cof = \Cof(\Iat)$ und $\Cof \cap \Weak = \Cof(\Jat)$.
\end{defn}

\begin{sprech}
  Die Kofaserungen in $\Iat$ heißen \emph{erzeugende Kofaserungen}, die in $\Jat$ \emph{azyklische erzeugende Kofaserungen}.
\end{sprech}

\begin{satz}
  Sei $\ModC$ eine lokal präsentierbare Kategorie. % => $\ModC$ bivollständig
  Sei $\Weak \subseteq \Mor(\ModC)$ eine Unterkat. schw. Äquivalenzen.
  Seien $\Iat, \Jat \subseteq \Mor(\ModC)$ Mengen.
  Dann sind $\Iat$ und $\Jat$ genau dann erzeugende (azyklische) Kofaserungen einer Modellstruktur auf $\ModC$, falls \\[2pt]
  \inlineitem{$\Cell(\Jat) \subseteq \Iat$ \enspace (Azyklizität)} \quad
  \inlineitem{$\Iat^\lhhe = \Jat^\lhhe \cap \Weak$ \enspace (Kompatibilität)}
\end{satz}

\subsection{Eigentliche Modellkategorien}

\begin{defn}
  Eine Modellkategorie $\ModC$ heißt \emph{linkseigentlich}, falls für alle Pushouts der Form
  \begin{centertikzcd}
    A \arrow[r, "\sim"] \arrow[d, hook] \arrow[rd, phantom, "\ulcorner", very near end] & B \arrow[d, hook] \\
    X \arrow[r, "g"] & Y
  \end{centertikzcd}
  auch der Morphismus $g : X \to Y$ eine schwache Äquivalenz ist. \\
  $\ModC$ heißt rechtseigentlich, falls $\ModC^\op$ linkseigentlich ist, \dh{} Pullbacks schwacher Äquivalenzen längs Faserungen wieder schwache Äquivalenzen sind.
\end{defn}

\begin{bsp}
  Eine Modellkategorie, in der jedes Objekt kofasernd ist, ist linkseigentlich.
\end{bsp}

\begin{defn}
  $\ModC$ heißt \emph{eigentlich}, falls $\ModC$ links- und rechtseigentlich ist.
\end{defn}

\begin{prop}
  In jeder Modellkategorie ist der Pushout einer schwachen Äquivalenz zwischen kofasernden Objekten längs Kofaserungen wieder eine schache Äquivalenz.
\end{prop}

% Einschub: Homotopie-Erweiterungs-Eigenschaft
\begin{bem}
  Gute Homotopien kann man längs Kofaserungen erweitern:
  \begin{centertikzcd}
    A \arrow[r, "H"] \arrow[d, hook, "i"] & Y^I \arrow{d}{p_0}[swap, twoheadrightarrow]{{\sim}} \\
    X \arrow[r, "f"] \arrow[ru, dashed, "\overline{h}"] & Y
  \end{centertikzcd}
\end{bem}

% TODO: Scheibenkategorien als Modellkategorien
% TODO: 2-aus-6-Eigenschaft von Isos, schwachen Äquivalenzen

\pagebreak

% Übung vom 15.4.2015

\section{Anhang: Die Ordinalzahlen}

\begin{defn}
  Eine \emph{Wohlordnung} auf einer Menge $S$ ist eine Totalordnung auf $S$ bezüglich der jede nichtleere Teilmenge $A \subseteq S$ ein kleinstes Element besitzt. Eine wohlgeordnete Menge ist ein Tupel $(S, \leq)$ bestehend aus einer Menge $S$ und einer Wohlordnung $\leq$ auf $S$.
\end{defn}

\begin{bem}
  Eine äquivalente Bedingung lautet: Es gibt in $S$ keine nach rechts unendlichen absteigenden Folgen
  $\ldots > a_i > a_{i+1} > a_{i+2} > \ldots$
\end{bem}

\begin{bem}
  Äquivalent zum Auswahlaxiom ist:
\end{bem}

\begin{axiom}[Wohlordnungssatz]
  Auf jeder Menge ex. eine Wohlord.
\end{axiom}

\begin{defn}
  Zwei wohlgeordnete Mengen heißen isomorph, wenn es eine monotone Bijektion zwischen ihnen gibt.
\end{defn}

% Ausgelassen: Intuition

\begin{defn}
  Eine \emph{Ordinalzahl} ist eine Isomorphieklasse von wohlgeordneten Mengen.
\end{defn}

\begin{bem}
  Die Klasse aller Ordinalzahlen wird mit $\Ord$ bezeichnet und ist eine echte Klasse, keine Menge.
  Sie ist selbst wohlgeordnet mittels
  \[ [(S, \leq_S)] \leq [(T, \leq_T)] \coloniff \exists \, \text{inj. monotone Abb. $(S, \leq_S) \to (T, \leq_T)$}. \]
\end{bem}

\begin{nota}
  \inlineitem{$0 \coloneqq [\emptyset]$,} \enspace
  \inlineitem{$n \coloneqq [\{ 1, \ldots, n \}]$ für $n \in \N$,} \enspace
  \inlineitem{$\omega \coloneqq [\N]$} \\
  mit der jeweils kanonischen Ordnungsrelation.
\end{nota}

\begin{bem}
  Die ersten Ordinalzahlen sind
  \[
    0, 1, 2, \ldots, \omega, \omega + 1, \omega + 2, \ldots, \omega \cdot 2, \omega \cdot 2 + 1, \ldots, \omega \cdot 3, \ldots, \omega^\omega, \ldots
  \]
\end{bem}

\begin{prinzip}[\emph{Transfinite Induktion}]\mbox{}\\
  Sei $P : \Ord \to \Prop$ eine Aussage über Ordinalzahlen. Dann gilt:
  \[ \left( \fa{\beta \in \Ord} (\fa{\gamma < \beta} P(\gamma)) \implies P(\beta) \right) \implies \fa{\alpha \in \Ord} P(\alpha) \]
\end{prinzip}

\begin{defn}
  Arithmetik von Ordinalzahlen ist folgendermaßen definiert: \\
  Für $\alpha = [(S, \leq_S)]$ und $\beta = [(T, \leq_T)] \in \Ord$ ist
  \begin{itemize}
    \item $\alpha + \beta \coloneqq [(S \amalg T, \leq_{S \amalg T})]$, wobei gilt:
    \[
      \leq_{S \amalg T}|_{S \times S} \,\coloneqq\, \leq_S, \quad
      \leq_{S \amalg T}|_{T \times T} \,\coloneqq\, \leq_T, \quad
      %s_1 \leq_{S \amalg T} s_2 \coloniff s_1 \leq_S s_2, \quad
      %t_1 \leq_{S \amalg T} t_2 \coloniff t_1 \leq_T t_2, \quad
      S <_{S \amalg T} T.
    \]
    \item $\alpha \cdot \beta \coloneqq [(S \times T, \leq_{S \rtimes T})]$ mit der lexikogr. Ordnung
    \[ (s_1, t_1) \leq_{S \rtimes T} (s_2, t_2) \coloneqq t_1 < t_2 \,\vee\, (t_1 = t_2 \wedge s_1 \leq_S s_2) \]
    \item $\alpha^\beta \coloneqq [(\{ \text{Abb. $f : S \to T$ mit $f(s) = 0$ für fast alle $s \in S$} \}, \leq)]$ mit
    \[ f < g \coloniff \ex{t \in T} f(t) < g(t) \wedge \left( \fa{t_2 >_T t} f(t_2) = g(t_2) \right) \]
  \end{itemize}
  %\begin{align*}
  %\end{align*}
\end{defn}

\begin{bem}
  Es gibt drei Typen von Ordinalzahlen:
  \begin{enumerate}[label=\alph*),leftmargin=1.6em]
    \item Die Null $0 \coloneqq [(\emptyset, \leq)] \in \Ord$.
    \item Die Nachfolgerzahl $\alpha + 1$ einer Zahl $\alpha \in \Ord$.
    \item Die Limeszahl $\lim A \coloneqq \sup A$ einer Teil{\em menge} $A \subset \Ord$.
  \end{enumerate}
\end{bem}

% http://en.wikipedia.org/wiki/Ordinal_arithmetic
\begin{bem}
  Die Rechenop. können auch rekursiv definiert werden durch
  \tabcolsep=0.11cm
  \begin{tabular}{ l l l }
    a) & b) & c) \\[2pt]
    $\alpha + 0 \coloneqq \alpha$
    & $\alpha \!+\! (\beta \!+\! 1) \coloneqq (\alpha \!+\! \beta) \!+\! 1$
    & $\alpha \!+\! \lim A \coloneqq \lim \, \Set{\alpha \!+\! \gamma}{\gamma \!\in\! A}$ \\

    $\alpha \cdot 0 \coloneqq 0$
    & $\alpha \cdot (\beta \!+\! 1) \coloneqq (\alpha \cdot \beta) + \alpha$
    & $\alpha \cdot \lim A \coloneqq \lim \, \Set{\alpha \cdot \gamma}{\gamma \in A}$ \\

    $\alpha^0 \coloneqq 1$
    & $\alpha^{\beta + 1} \coloneqq \alpha^\beta \cdot \alpha$
    & $\alpha^{\lim A} \coloneqq \lim \, \Set{\alpha^\gamma}{\gamma \in A}$
  \end{tabular}
  \iffalse
  \begin{align*}
    \text{a) } \alpha + 0 \coloneqq \alpha, \qquad
    & \text{b) } \alpha + (\beta + 1) \coloneqq (\alpha + \beta) + 1, \\
    \text{c) } \alpha + \lim A &\coloneqq \lim \, \Set{\alpha + \gamma}{\gamma \in A}, \\[2pt]
    \text{a) } \alpha \cdot 0 \coloneqq 0, \qquad
    & \text{b) } \alpha \cdot (\beta + 1) \coloneqq (\alpha \cdot \beta) + \alpha, \\
    \text{c) } \alpha \cdot \lim A &\coloneqq \lim \, \Set{\alpha \cdot \gamma}{\gamma \in A}, \\[2pt]
    \text{a) } \alpha^0 \coloneqq 1, \qquad
    & \text{b) } \alpha^{\beta + 1} \coloneqq \alpha^\beta + 1, \\
    \text{c) } \alpha^{\lim A} &\coloneqq \lim \, \Set{\alpha^\gamma}{\gamma \in A}.
  \end{align*}
  \fi
\end{bem}

\begin{defn}
  Ein \emph{Fast-Halbring} ist ein Tupel $(S, +, \cdot, 0)$, sodass $(S, +, 0)$ ein Monoid und $(S, \cdot)$ eine Halbgruppe ist mit \\
  \inlineitem{$a \cdot (b+c) = a \cdot b + a \cdot c$,} \quad
  \inlineitem{$a \cdot 0 = 0$.}
\end{defn}

\begin{lem}[Rechenregeln in $\Ord$]
  \inlineitem{$\alpha \cdot 0 = 0 = 0 \cdot \alpha$} \enspace
  \inlineitem{$\alpha \cdot 1 = \alpha = 1 \cdot \alpha$} \\
  \inlineitem{$\alpha^0 = 1$} \quad
  \inlineitem{$0^\alpha = 0$ für $\alpha > 0$} \quad
  \inlineitem{$1^\alpha = 1$} \quad
  \inlineitem{$\alpha^1 = \alpha$} \\
  \inlineitem{$\alpha^\beta \cdot \alpha^\gamma = \alpha^{\beta+\gamma}$} \enspace
  \inlineitem{$(\alpha^\beta)^\gamma = \alpha^{\beta \cdot \gamma}$} \enspace
  %\inlineitem{$$} \enspace
  \begin{itemize}
    \item $\Ord$ ist ein Fast-Halbring (mit einer Klasse statt Menge)
    \item Das andere Distributivgesetz stimmt {\em nicht}!
    \item Weder Addition noch Multiplikation sind kommutativ.
    \item Addition und Mult. erlauben das Kürzen von Elementen nur links.
    \item Addition, Multiplikation und Potenzieren sind in beiden Argu- menten monoton, allerdings nur im zweiten strikt monoton:
    \[
      \forall \, \beta < \gamma: \quad
      \alpha + \beta < \alpha + \gamma, \quad
      \alpha \cdot \beta < \alpha \cdot \gamma \enspace (\alpha \!>\! 0), \quad
      \alpha^\beta < \alpha^\gamma \enspace (\alpha \!>\! 1).
    \]
  \end{itemize}
\end{lem}

% http://en.wikipedia.org/wiki/Ordinal_arithmetic
\begin{lem}
  Jedes $\alpha \in \Ord$ kann geschrieben werden in \emph{Cantor-NF}:
  \[ \alpha = \omega^{\beta_1} c_1 + \omega^{\beta_2} c_2 + \ldots + \omega^{\beta_k} c_k \]
  mit $k \in \N$, $c_1, \ldots, c_k \in \N_{> 0}$ und $\beta_1 > \ldots > \beta_k \in \Ord$.
\end{lem}

% TODO: Verbindungen zu Kardinalzahlen
 % Ordinalzahlen-Anhang

\end{document}
