\documentclass{cheat-sheet}

\pdfinfo{
  /Title (Zusammenfassung Riemannsche Geometrie)
  /Author (Tim Baumann)
}

\usepackage{tikz}
\usetikzlibrary{matrix,shapes,arrows,positioning}

\newcommand{\Tau}{\mathcal{T}} % Großes Tau
\newcommand{\angles}[1]{\langle #1 \rangle}
\renewcommand{\O}{\mathcal{O}} % Offene Mengen im $\R^n$
\newcommand{\A}{\mathcal{A}} % Atlanten
\newcommand{\K}{\mathbb{K}} % Körper
\DeclareMathOperator{\GL}{GL} % General Linear Group
\newcommand{\coord}[1]{\tfrac{\partial^\phi}{\partial x^{#1}}} % Standard-Koordinaten
\newcommand{\Coord}[1]{\frac{\partial^\phi}{\partial x^{#1}}} % Standard-Koordinaten (größerer Bruch)
\newcommand{\lie}[2]{\left[ {#1}, {#2} \right]} % Lie-Klammer
\DeclareMathOperator{\Exp}{Exp} % Exponentialabbildung
\newcommand{\vinterval}{\ointerval{-\epsilon}{\epsilon}} % Variationsintervall
\DeclareMathOperator{\Span}{span} % Spann
\DeclareMathOperator{\diam}{diam} % Diameter
\DeclareMathOperator{\Ric}{Ric} % Ricci-Tensor
\newcommand{\abinterval}{\cinterval{a}{b}} % [a,b]
\DeclareMathOperator{\Iso}{Iso} % Isometriegruppe
\DeclareMathOperator{\Trans}{Trans} % Transvektionsgruppe
\DeclareMathOperator{\Hol}{Hol} % Holonomiegruppe
\DeclareMathOperator{\Diff}{Diff}
\newcommand{\Gie}{\mathfrak{G}} % Lie-G
\newcommand{\KF}{K\!F} % Killing-Fields
\DeclareMathOperator{\Lie}{Lie} % Transvektionsgruppe
\newcommand{\Cont}{\mathcal{C}} % Menge der stetigen/diff'baren Funktionen
\newcommand{\LC}{\nabla^{\mathrm{LC}}} % Levi-Civita-Zusammenhang
\newcommand{\VF}{\mathcal{X}} % Menge der Vektorfelder

% Kleinere Klammern
\delimiterfactor=701


\begin{document}

\maketitle{Kurzfassung Riemannsche Geometrie}

% §1. Mannigfaltigkeiten

\begin{defn}
  Eine \emph{topologische Mannigfaltigkeit} (Mft) der Dim. $m$ ist ein topologischer Raum $M^m$ mit folgenden Eigenschaften:
  \begin{itemize}
    \item $M^m$ ist \emph{hausdorffsch}, \dh{}
    \begin{align*}
      \fa{x, y \in M^m} &x \not= y \implies \ex{U_x \opn M^m} \ex{U_y \opn M^m}\\
      &x \in U_x \wedge y \in U_y \wedge U_x \cap U_y = \emptyset.
    \end{align*}
    \item $M^m$ erfüllt das \emph{zweite Abzählbarkeitsaxiom}, \dh{} es gibt eine abzählbare Menge $\Set{ U_i }{ i \in \N } \subset \Tau$, sodass
    \[ \fa{A \opn M^m} \ex{K \subset \N} A = \bigcup_{k \in K} U_k. \]
    \item $M^m$ ist \emph{lokal euklidisch}, \dh{} für alle $x \in M^m$ gibt es eine offene Umgebung $U_x$ von $x$ und einen Homöomorphismus $\phi : U_x \to \O$ mit $\O \subset \R^m$ offen.
  \end{itemize}
\end{defn}

\begin{bem}
  lokal euklidisch $\not\Rightarrow$ hausdorffsch
\end{bem}

\begin{prop}
  Sei $M$ eine topologische Mannigfaltigkeit. Dann gilt
  \[ \text{$M$ zusammenhängend} \iff \text{$M$ wegzusammenhängend.} \]
\end{prop}

\begin{defn}
  \begin{itemize}
    \item Sei $M$ eine $m$-dim. topol. Mft. Ein \emph{Atlas} ist eine Menge $\A = \Set{(U_j, \phi_j : U_j \to \O_j)}{j \in J}$ mit $U_j \opn M$ und $\O_j \subset \R^n$ offen und Homöomorphismen $\phi_j$, für die gilt $\bigcup_{j \in J} U_j = M$.
    \item Die Paare $(U_j, \phi_j)$ werden \emph{Karten} genannt.
    \item Für je zwei Karten $(U_j, \phi_j)$ und $(U_k, \phi_k)$ gibt es eine \emph{Kartenwechselabbildung}
    \[ \phi_{kj} \coloneqq \phi_k \circ \phi_j^{-1} |_{\phi_j(U_j \cap U_k)} : \phi_j(U_j \cap U_k) \to \phi_k(U_j \cap U_k). \]
    \item Ein Atlas heißt \emph{differenzierbar}, wenn alle Kartenwechselabbildungen $\Cont^\infty$-Abbildungen sind.
    \item Ein Atlas $\A$ heißt \emph{differenzierbare Struktur} von $M$, wenn gilt: Ist $(\tilde{U}, \tilde{\phi_j})$ eine Karte von $M$ und $\tilde{\A} \coloneqq \A \cup \{ (\tilde{U}, \tilde{\phi_j}) \}$ ein differenzierbarer Atlas, dann gilt $\A = \tilde{\A}$.
    \item Eine topol. Mft versehen mit einer differenzierbaren Struktur heißt \emph{differenzierbare Mannigfaltigkeit}.
  \end{itemize}
\end{defn}

% §2. Differenzierbare Abbildungen

\begin{nota}
  Seien ab jetzt $M^m$ und $N^n$ differenzierbare Mften der Dimensionen $m$ und $n$.
\end{nota}

\begin{defn}
  \begin{itemize}
    \item Eine Abb. $f : M \to N$ heißt in $x \in M$ \emph{differenzierbar}, wenn es eine Karte $(U_x, \phi : U_x \to \O) \in \A_M$ und eine Karte $(\tilde{U}_{f(x)}, \tilde{\phi} : \tilde{U}_{f(x)} \to \tilde\O) \in \A_N$ gibt, sodass
    \[
      \tilde{\phi} \circ f|_{U_x} \circ \phi^{-1} : \O \to \tilde\O
      \quad \text{differenzierbar ($\Cont^\infty$) ist.}
    \]
    \item Die Abb. $f$ heißt \emph{diff'bar}, wenn sie in allen $x \in M$ diff'bar ist.
  \end{itemize}
\end{defn}

\begin{nota}
  $\Cont^\infty(M, N) \coloneqq \Set{f : M \to N}{\text{$f$ ist differenzierbar}}$
\end{nota}

\begin{bem}
  Die Definition ist unabh. von Wahl der Karten um $x$ und $f(x)$.
\end{bem}

\begin{defn}
  Eine Abbildung $f : M \to N$ heißt \emph{Diffeomorphismus}, wenn $f$ ein Homöo ist und $f$ und $f^{-1}$ differenzierbar sind.
\end{defn}

% §3. Tangentialvektoren

\begin{defn}
  Sei $p \in M$. Zwei Funktionen $f : U_p \to \R$ und $g : V_p \to \R$ mit $U_p, V_p \opn M$ heißen äquivalent, falls es eine offene Umgebung $W \subset U_p \cap V_p$ mit $f|_W = g|_W$ gibt. Die Äquivalenzklasse $[f]$ bezüglich der so definierten Äq'relation heißt \emph{Funktionskeim} in $p$.
\end{defn}

\begin{nota}
  $\Cont^\infty(M, p) \coloneqq \Set{ [f] }{ \text{$[f]$ Funktionskeim in $p$} }$
\end{nota}

\begin{bem}
  Die Menge der Funktionskeime ist eine $\R$-Algebra.
\end{bem}

\begin{defn}
  Eine lineare Abb. $\delta : \Cont^\infty(M, p) \to \R$ heißt \emph{Derivation}, falls
  \[ \fa{[f], [g] \in \Cont^\infty(M, p)} \delta[f \cdot g] = \delta[f] \cdot g(p) + f(p) \cdot \delta[g]. \]
\end{defn}

\begin{defn}
  Der gewöhnliche Tangentialraum des $\R^n$ im Punkt $p$ ist
  \[ \tilde{T}_p \R^n \coloneqq \Set{(p, v)}{v \in \R^n} \]
  mit $(p, v) + (p, w) \coloneqq (p, v + w)$ und $\lambda \cdot (p, v) \coloneqq (p, \lambda \cdot v)$.
\end{defn}

\begin{defn}
  Der \emph{Tangentialraum} von $M$ im Punkt $p \in M$ ist
  \[ T_p M \coloneqq \Set{ \partial : \Cont^\infty(\R^n, p) \to \R }{ \text{$\partial$ linear, derivativ} } \]
  Ein Element $v \in T_p M$ heißt \emph{Tangentialvektor} an $M$ in $p$.
\end{defn}

\begin{bem}
  Wir erhalten eine Abbildung
  \[
    T_p M \times \Cont^\infty(M, p) \to \R, \quad
    (v, [f]) \mapsto v.f \coloneqq v[f].
  \]
\end{bem}

\begin{bem}
  $T_p M$ ist ein $\R$-Vektorraum.
\end{bem}

\begin{satz}
  Die Vektorräume $T_p \R^n$ und $\tilde{T}_p \R^n$ sind isomorph. Insbesondere gilt $\dim(T_p \R^n) = n$.
\end{satz}

\begin{kor}
  Für eine $m$-dim. diff'bare Mft $M$ gilt: $\dim(T_p M) = m$.
\end{kor}

\begin{bem}
  Sei $c : \vinterval \to M$ eine differenzierbare Kurve. Dann kann man $\dot{c}(0)$ auffassen als Tangentialvektor an $M$ in $c(0)$ mittels
  \[ \dot{c}(0)[f] \coloneqq \tfrac{\d}{\d t}|_{t=0} (f \circ c). \]
\end{bem}

\begin{bem}
  Sei $(U, \phi)$ eine Karte von $M$. Wir setzen
  \begin{align*}
    \tfrac{\partial^{\phi}}{\partial x_i}|_p [f] & \coloneqq (\phi^{-1} \circ \alpha_i)^{{\cdot}} (0) [f] = \tfrac{\d}{\d t}|_{t=0} (f \circ \phi^{-1} \circ \alpha_i)\\
    & \text{mit } \alpha_i : \ointerval{-\epsilon}{\epsilon} \to U, \enspace t \mapsto \phi(p) + t e_i.
  \end{align*}
  Wir erhalten $\tfrac{\partial^{\phi}}{\partial x_i}|_p \in T_p M$.
\end{bem}

\begin{defn}
  Sei $f : M \to N$ diff'bar. Die \emph{Ableitung} von $f$ in $p \in M$ ist %die Abbildung
  \[
    T_p f \!=\! f_{*p} : T_p M \to T_{f(p)} N, \enspace v \mapsto f_{*p}, \quad
     \text{wobei } f_{*p}(v).[g] \coloneqq v.[g \circ f].
  \]
\end{defn}

\begin{lem}
  Sei $M$ eine diff'bare Mft, $p \in M$. Dann gilt
  \begin{itemize}
    \miniitem{0.3 \linewidth}{$f_{*p}$ ist linear}
    \miniitem{0.3 \linewidth}{$(\id_M)_{*p} = \id_{T_p M}$}
    \item Kettenregel: Seien $N$, $P$ diff'bare Mften. Dann gilt
    \[ \fa{p \in M} (f \circ g)_{*p} = f_{* g(p)} \circ g_{*p}. \]
  \end{itemize}
\end{lem}

\begin{kor}
  Wenn $f : M \to N$ ein Diffeomorphismus ist, dann ist $f_{*p} : T_p M \to T_{f(p)} N$ ein VR-Isomorphismus für alle $p \in M$.
\end{kor}

\begin{satz}
  Sei $M$ eine $m$-dimensionale Mft, $p \in M$ und $(U, \phi)$ eine Karte.
  \begin{itemize}
    \item Es gilt $T_p M = \Set{ \dot{c}(0) }{ c : \ointerval{-\epsilon}{\epsilon} \to M \text{ diff'bar}, \, c(0) = p }$
    \item $\Set{\coord{i}|_p}{i=1, \ldots, n}$ ist eine Basis von $T_p M$.
  \end{itemize}
\end{satz}

\begin{defn}
  $TM \coloneqq \!\! \bigsqcup_{p \in M} \!\! T_p M$ heißt \emph{Tangentialbündel} von $M$.
\end{defn}

% Ausgelassen: Bemerkung, dass es sich dabei um eine disjunkte Vereinigung handelt (weil $M$ hausdorffsch)

\begin{defn}
  Die \emph{Fußpunktabb.} ist die Proj.
  $\pi : TM \to M, \enspace v \in T_p M \mapsto p$.
\end{defn}

\begin{defn}
  Ein \emph{Vektorfeld} auf $M$ ist eine Abbildung $X : M \to TM$, sodass $\pi \circ X = \id_M$. Dies ist äquivalent zu $\fa{p \in M} X(p) \in T_p(M)$.
\end{defn}

\begin{bem}
  Sei $X : M \to TM$ ein Vektorfeld, $(U, \phi)$ eine Karte. \\
  Dann gibt es Funktionen $\xi^j : U \to \R$, $j = 1, \ldots, n$ mit
  \[ \fa{p \in U} X(p) = \sum_{j=1}^n \xi^j(p) \coord{j}|_p. \]
\end{bem}

\begin{defn}
  \begin{itemize}
    \item Ein VF $X$ auf $M$ heißt in $p \in M$ \emph{diff'bar} (bzw. $\Cont^\infty$), wenn es eine Karte $(U, \phi)$ um $p$ gibt, sodass die Funktionen $\xi^1, \ldots, \xi^n$ diff'bar (bzw. $\Cont^\infty$) sind.
    \item $X$ heißt \emph{differenzierbar}, wenn $X$ in allen $p \in M$ diff'bar ist.
  \end{itemize}
\end{defn}

\begin{lem}
  Wenn die Koordinatenfunktionen $\xi^1, \ldots, \xi^n$ für eine bestimmte Karte $(U, \phi : U \to \O)$ differenzierbar sind, dann sind sie es für jede andere Karte $(\tilde{U}, \psi : \tilde{U} \to \tilde{\O})$ mit $\tilde{U} \subseteq U$.
\end{lem}

\begin{defn}
  Sei $M$ eine $m$-dimensionale diff'bare Mft mit diff'barer Struktur $\A = \Set{(U_j, \phi_j)}{j \in J}$. Dann ist $TM$ eine $2m$-dimensionale Mft mit Atlas $\tilde{\A} \coloneqq \Set{(\tilde{U}_j \coloneqq \pi^{-1}(U_j), \tilde{\Phi}_j)}{j \in J}$, wobei
  \[
    \tilde{\Phi}_j : \pi^{-1}(U_j) \to \phi_j(U_j) \times \R^m, \quad
    \sum_{k=1}^m \xi^k(p) \tfrac{\partial^{\phi_j}}{\partial x^k}|_p \mapsto (\phi_j(p), \xi^1(p), \nldots, \xi^n(p)).
  \]
  Eine Menge $V \subseteq TM$ heißt offen, wenn $\tilde{\Phi}_j(V \cap \pi^{-1}(U_j)) \opn \R^{2n}$ offen ist für alle $j \in J$.
\end{defn}

\begin{nota}
  $\VF(M) \coloneqq \{\text{ diff'bare Vektorfelder auf $M$ }\}$
\end{nota}

\begin{bem}
  $\VF(M)$ ist ein $\R$-VR und ein $\Cont^\infty(M)$-Modul.
\end{bem}

\begin{lem}
  Jedes $X \in \VF(M)$ induziert eine lineare, derivative Abb.
  \[
    X : \Cont^\infty(M) \to \Cont^\infty(M), \quad
    \phi \mapsto X(\phi) \coloneqq p \mapsto X(p) . [\phi].
  \]
\end{lem}

\begin{lem}
  %Seien $X, Y \in \VF(M)$ mit $\fa{f \in \Cont^\infty(M)} X(f) = Y(f)$. Dann gilt $X \equiv Y$.
  $\fa{X, Y \!\in\! \VF(M)\!}\! (\fa{f \!\in\! \Cont^\infty(M)\!}\! X(f) \!=\! Y(f)) \!\iff\! X \equiv Y$
\end{lem}

\begin{defn}
  Der \emph{Kommutator} (o. \emph{Lie-Klammer}) von $X, Y \in \VF(M)$ ist das Vektorfeld $[X, Y] \in \VF(M)$ definiert durch
  \[
    [X, Y] : \Cont^\infty(M) \to \Cont^\infty(M), \enspace
    f \mapsto X(Y(f)) - Y(X(f)).
  \]
\end{defn}

\begin{satz}
  Für $X, Y_1, Y_2 \in \VF(M)$ und $f \in \Cont^\infty(M)$ gilt
  \[ [X, Y_1 + f Y_2] = [X, Y_1] + X(f) \cdot Y_2 + f \cdot [X, Y_2]. \]
\end{satz}

% TODO: Dingsda-Identität nachrechnen?

\begin{defn}
  Eine diff'bare Kurve $c : \ointerval{a}{b} \to M$ heißt \emph{Integralkurve} von einem VF $X \in \VF(M)$, falls
  $\fa{t \in \ointerval{a}{b}} \dot{c}(t) = X_{c(t)}$.
\end{defn}

\begin{lem}
  Sei $X \in \VF(M)$, $p \in M$ und $v \in T_p M$. Dann hat das AWP
  \[ \dot{c}(t) = X_{c(t)}, \enspace c(0) = p \]
  eine eindeutige lokale Lösung $c = c_p^X : \vinterval \to M$.
\end{lem}

\begin{defn}
  $\Phi_X : U \times \vinterval \to M, \enspace (p, t) \mapsto c_p^X(t)$ heißt \emph{Fluss} von $X$.
\end{defn}

% §5. Lie-Algebren und Lie-Gruppen

\begin{defn}
  Ein $\K$-Vektorraum $V$ mit einer $\K$-bilinearen Abbildung $[\blank,\blank] : V \times V \to V, \enspace (v, w) \mapsto [v,w]$ heißt \emph{Lie-Algebra}, falls
  \begin{itemize}
    \item die Abb. antisymmetrisch ist, \dh{} $\fa{v, w \in V} [v, w] = - [w, v]$
    \item die \emph{Jacobi-Identität} erfüllt ist, \dh{}
    \[ \fa{v,w,z \in V} [v, [w, z]] + [z, [v, w]] + [w, [z, v]] = 0. \]
  \end{itemize}
\end{defn}

\begin{bspe}
  \begin{itemize}
    \item $(\VF(M), [\blank,\blank])$ ist eine Lie-Algebra.
    \item $\K^{n \times n}$ ist eine Lie-Algebra mit $[A, B] \coloneqq AB - BA$.
  \end{itemize}
\end{bspe}

\begin{defn}
  Eine Gruppe $G$, welche ebenfalls eine diff'bare Mft ist, heißt \emph{Lie-Gruppe}, wenn gilt:
  \begin{itemize}
    \item $\mu : G \times G \to G, \enspace (g_1, g_2) \mapsto g_1 \cdot g_2$ ist diff'bar.
    \item $\iota : G \to G, \enspace g \mapsto g^{-1}$ ist diff'bar.
  \end{itemize}
\end{defn}

\begin{bsp}
  Die allg. lin. Gruppe $\GL(n, \R) \subset \R^{n \times n} \approx \R^{(n^2)}$ ist eine Lie-
  Gruppe. Die Diff'barkeit der Inv. folgt aus der Cramerschen Regel.
\end{bsp}

\begin{defn}
  Sei $G$ eine Lie-Gruppe und $g \in G$. Dann sind
  \begin{align*}
    lg : G \to G, &\quad x \mapsto g \cdot x = \mu(g, x)\\
    rg : G \to G, &\quad x \mapsto x \cdot g = \mu(x, g)
  \end{align*}
  Diffeomorphismen mit Umkehrabbildung $l(g^{-1})$ bzw. $r(g^{-1})$.
\end{defn}

% Vorlesung vom 24.10.2014

\begin{bsp}
  Abgeschl. Untergruppen von $\GL(n, \R)$ sind Lie-Gruppen, \zB
  \begin{itemize}
    \miniitem{0.40 \linewidth}{$\GL(n, \C) \subset \GL(2n, \R)$}
    \miniitem{0.29 \linewidth}{$O_n \subset \GL(n, \R)$}
    \miniitem{0.27 \linewidth}{$U_n \subset \GL(2n, \R)$}
  \end{itemize}
\end{bsp}

\begin{defn}
  Sei $f : M \to N$ ein Diffeomorphismus und $X \in \VF(M)$. Dann ist
  \[
    f_* X : N \to TN, \enspace
    x \mapsto f_{*f^{-1}(x)} X(f^{-1}(x))
  \]
\end{defn}

\begin{defn}
  Ein Vektorfeld $X \in \VF(G)$ heißt \emph{linksinvariant}, wenn gilt:
  \[
    \fa{g, h \in G} X(g \cdot h) = lg_{*h} X(h) \quad
    \text{(kürzer: $\fa{g \in G} lg_* X = X$).}
  \]
\end{defn}

\begin{nota}
  $\mathcal{L}(G) \coloneqq \Set{ X \in \VF(G) }{\text{$X$ ist linksinvariant}} \subset \VF(G)$
\end{nota}

\begin{bem}
  Ein linksinv. VF $X \!\in\! \VF(G)$ ist eindeutig bestimmt durch $X(e)$.
  Andererseits: Ist $x \in T_e G$, dann gibt es ein linksinv. VF $X \in \VF(G)$ mit $X(e) = x$.
  Somit gibt es einen VR-Isomorphismus
  \[
    i : \mathcal{L}(G) \to T_e G, \enspace
    X \mapsto X(e).
  \]
\end{bem}

\begin{lem}
  Seien $X, Y \in \mathcal{L}(G)$. Dann ist $[X, Y] \in \mathcal{L}(G)$.
\end{lem}

\begin{kor}
  $(\mathcal{L}(G), [\blank,\blank])$ ist eine $\dim(G)$-dimensionale Unter-Lie-Algebra von $(\VF(G), [\blank,\blank])$.
\end{kor}

\begin{nota}
  $\Gie \coloneqq \Lie(G) \coloneqq \mathcal{L}(G) \cong T_e G$
\end{nota}

% §6. Riemannsche Mannigfaltigkeiten

\begin{defn}
  Ein Skalarprodukt auf einem Vektorraum $V$ ist eine symmetrische, positiv definite Bilinearform $\langle \blank, \blank \rangle$. \\
  Die davon induzierte Norm ist $\norm{v} \coloneqq \sqrt{\abs{\langle v, v \rangle}}$.
\end{defn}

\begin{defn}
  Eine \emph{Riemannsche Metrik} auf einer diff. Mft $M$ ist eine Familie $g = (g_p)_{p \in M}$ von Skalarprodukten $g_p : T_p M \times T_p M \to \R$, die differenzierbar von $p$ abhängt, \dh{} für alle $X, Y \in \VF(M)$ ist $g(X, Y) : M \to \R, p \mapsto g_p(X(p), Y(p))$ differenzierbar ($\Cont^\infty$).\\
  Das Tupel $(M, g)$ heißt \emph{Riemannsche Mannigfaltigkeit}.
\end{defn}

\begin{bem}
  Sei $(U, \phi)$ eine Karte von $M$. Setze
  \[
    g_{ij}^{\phi} : U \to \R, \enspace
    p \mapsto g(\coord{i}|_p, \coord{j}|_p).
  \]
  Seien $X = \sum_{i=1}^n v^i \coord{i}$ und $Y = \sum_{j=1}^n w^j \coord{j}$ zwei VF in $U$. Dann gilt
  \[
    g(X, Y)(p) = g_p(X(p), Y(p)) = \sum_{i,j=1}^n v^i(p) w^j(p) g_{ij}(p).
  \]
\end{bem}

\begin{defn}
  Seien $(M, g_M)$, $(N, g_N)$ Riemannsche Mannigfaltigkeiten. \\
  Eine Abbildung $f : M \to N$ heißt \emph{Isometrie}, wenn gilt:
  \begin{itemize}
    \item $f$ ist ein Diffeomorphismus
    \item $f$ erhält Riemannsche Metriken, \dh{} für alle $X, Y \in \VF(M)$ gilt:
    \[ g_M(X, Y) = g_N(f_* X, f_* Y) \circ f, \]
    also $\fa{p {\in} M} \fa{v, w {\in} T_p M} g_{M,p} (v, w) = g_{N,f(p)}(f_{*p}(v), f_{*p}(w))$.
  \end{itemize}
\end{defn}

\begin{defn}
  $\Iso(M) \!\coloneqq\! \Set{ \tau : M \!\to\! M }{ \tau \text{ Isometrie} }$ heißt \emph{Isometriegruppe}.
\end{defn}

\begin{bem}
  $\Iso(M)$ ist in kan. Weise eine Lie-Gruppe (Myers-Steenrod).
\end{bem}

% 6.1.
\begin{satz}
  Jede diff'bare Mannigfaltigkeit hat eine Riemannsche Metrik.
\end{satz}

\begin{bsp}
  Das Oberer-Halbraum-Modell des hyperbolischen Raum ist
  \[ H^n \coloneqq \Set{x \in \R^n}{\langle x, e_n \rangle_{\text{eukl}} > 0} \opn \R^n \]
  mit dem offensichtlichen Atlas und der Riemannschen Metrik
  \[ g_p^{\text{Hyp}} ((p, \tilde{v}), (p, \tilde{w})) \coloneqq \frac{\langle \tilde{v}, \tilde{w} \rangle_{\text{eukl}}}{\langle p, e_n \rangle^2}. \]
\end{bsp}

\begin{defn}
  Eine diff'bare Abb. $f : M \to N$ zwischen diff'baren Mften heißt \emph{Immersion}, falls $f_{*p} : T_p M \to T_{f(p)} N$ f.\,a. $p \in M$ injektiv ist.
\end{defn}

\begin{defn}
  Angenommen, $N$ ist sogar eine Riem. Mft mit Metrik $g_N$. Dann erhalten wir eine Riem. Metrik auf $M$, die mit $f$ \emph{zurückgeholte Metrik}, durch
  \[ (f^* g_N)_p (v, w) \coloneqq g_{N,f(p)}(f_{*p}(v), f_{*p}(w)). \]
\end{defn}

\begin{defn}
  Eine Immersion $f : (M, g^M) \to (N, g^N)$ heißt \emph{isometrisch}, falls $g^M = f^* g^N$.
\end{defn}

\begin{prop}
  Sei $M$ eine zshgde Mft. Dann gibt es für alle $p, q \in M$ einen stückweise diff'baren Weg $\gamma : \cinterval{0}{1} \to M$ mit $\gamma(0) = p$ und $\gamma(1) = q$.
\end{prop}

\begin{defn}
  Für $\gamma : \abinterval \to M$ stückweise $\Cont^1$ heißt
  \[
    L(\gamma) \coloneqq \Int{a}{b}{\norm{\dot{\gamma}(\tau)}}{\tau}
    \quad \text{\emph{Länge} von $\gamma$.}
  \]
\end{defn}

\begin{defn}
  Der \emph{Riem. Abstand} auf $(M, g)$ ist geg. durch die Metrik
  \begin{align*}
    d_g : M \times M \to \R, \quad
    (p, q) \mapsto \inf \{\, L(\gamma) \,\,|\,\, & \gamma : \abinterval \to M \text{ stückweise } \Cont^{1}\\
    & \left. \text{mit $\gamma(a) \!=\! p$ und $\gamma(b) \!=\! q$} \, \right\}.
  \end{align*}
\end{defn}

\begin{bem}
  Nach dem Satz von Hopf-Rinow stimmt die von $d_g$ induzierte Topologie mit der von $M$ überein.
\end{bem}

% §7. Kovariante Ableitungen

\begin{defn}
  Ein \emph{Zusammenhang} (kov. Ableitung) ist eine Abbildung
  \[
    \nabla : \VF(M) \times \VF(M) \to \VF(M), \quad
    (X, Y) \mapsto \nabla_X Y
  \]
  sodass für $X, X_1, X_2, Y, Y_1, Y_2 \in \VF(M)$ und $f \in \Cont^\infty(M)$ gilt:
  \begin{itemize}
    \item $\nabla_{X_1 + f X_2} Y = \nabla_{X_1} Y + f \nabla_{X_2} Y$
    \item $\nabla_X (Y_1 + Y_2) = \nabla_X Y_1 + \nabla_X Y_2$
    \item $\nabla_X (f Y) = f \left( \nabla_X Y \right) + (X(f)) \cdot Y$ \enspace (Leibniz-Regel)
  \end{itemize}
\end{defn}

\begin{defn}
  Sei $\nabla$ ein Zusammenhang auf $M$. Dann heißt
  \[
    T^\nabla(X, Y) \coloneqq \nabla_X Y - \nabla_Y X - [X, Y]
    \qquad \text{\emph{Torsion} von $\nabla$.}
  \]
  Wenn $T^\nabla \equiv 0$, dann heißt $\nabla$ \emph{torsionsfrei}.
\end{defn}

\begin{defn}
  Ein Zshg $\nabla$ auf einer Riem. Mft. heißt \emph{metrisch}, wenn
  \[ \fa{X, Y, Z \in \VF(M)} g(\nabla_X Y, Z) + g(Y, \nabla_X Z) = X g(Y, Z). \]
\end{defn}

\begin{satz}
  Auf jeder Riem. Mft. $(M, g)$ gibt es genau einen torsionsfreien, metrischen Zusammenhang. Für diesen gilt:
  \begin{align*}
    2 g(\nabla_X Y, Z) = \, & X g(Y, Z) + Y g(X, Z) - Z g(X, Y)\\
    & + g([X, Y], Z) + g([Z, X], Y) + g([Z, Y], X)
  \end{align*}
\end{satz}

\begin{defn}
  Der eindeutige torsionsfreie und metrische Zusammenhang auf $(M, g)$ heißt \emph{Levi-Civita-Zusammenhang} auf $(M, g)$.
\end{defn}

\begin{bem}
  Sei $(M, g)$ eine Riemannsche Mft., $(U, \phi)$ eine Karte von $M$. Dann gibt es diff'bare Ftk. $\Gamma_{ij}^k : U \to \R$ für $i,j,k \in \{ 1, \ldots, n \}$, sodass
  \[
    \nabla_{\left( \coord{i} \right)} \left( \coord{j} \right) = \sum_{k=1}^n \Gamma_{ij}^k \coord{k}.
  \]
  Die Funktionen $\Gamma_{ij}^k$ heißen \emph{Christoffel-Symbole} von $\nabla$.
\end{bem}

% Vorlesung vom 5.11.2014

% 7.2
\begin{lem}
  $\lie{\coord{j}}{\coord{k}} = 0$
\end{lem}

% 7.3
\begin{satz}
  Für die Christoffel-Symbole gilt
  \[ \Gamma_{ij}^k = \tfrac{1}{2} \sum_{l=1}^n g^{kl} \left( \Coord{j} g_{il} + \Coord{i} g_{jl} - \Coord{l} g_{ij} \right), \]
  wobei
  \begin{align*}
    g_{ij} : U \to \R, & \enspace p \mapsto g_p \left( \coord{i} (p), \coord{j} (p) \right) \\
    g^{kl} : U \to \R & \enspace \text{definiert ist durch} \enspace \sum_{r=1}^n g^{jr} g_{rk} = \delta_k^j.
  \end{align*}
\end{satz}

\begin{defn}
  Sei $\nabla$ ein Zusammenhang auf $M$. Dann heißt $X \in \VF(M)$ \emph{parallel}, falls
  $\nabla X : \VF(M) \to \VF(M), \enspace Y \mapsto \nabla_Y X$
  verschwindet.
\end{defn}

\begin{defn}
  Ein \emph{Tensorfeld} vom Typ $(j, k)$ mit $k \in \N$ und $j \in \{ 0, 1 \}$ ist eine multilineare Abbildung
  \[
    T : \VF(M) \times \ldots \times \VF(M) \to
    \begin{cases}
      \Cont^\infty(M), & \text{falls } j = 0,\\
      \VF(M), & \text{falls } j = 1.
    \end{cases}
  \]
\end{defn}

\begin{bspe}
  \begin{itemize}
    \item $T^\nabla : \VF(M) \times \VF(M) \to \VF(M)$ ist Tensor vom Typ $(1, 2)$.
    \item $\nabla Y : \VF(M) \to \VF(M), \enspace X \mapsto \nabla_X Y$ ist Tensor vom Typ $(1, 1)$.
    \item Alternierende $k$-Formen auf $\R^n$ sind Tensoren vom Typ $(0, k)$.
    \item Riemannsche Metriken sind Tensorfelder vom Typ $(0, 2)$.
  \end{itemize}
\end{bspe}

% 8.1
\begin{satz}
  Sei $T$ ein Tensorfeld auf $M$ vom Typ $(j, k)$. Sei $p \in M$. Seien $X_1, \ldots, X_k \in \VF(M)$. Dann hängt $T(X_1, \ldots, X_k)(p)$ nur von $X_1(p), \ldots, X_k(p)$ ab.
\end{satz}

% Vorlesung vom 7.11.2014

\begin{bem}
  Sei $(U, \phi)$ eine Karte von $M$ und $T$ ein Tensorfeld vom Typ $(1, k)$ auf $M$. Dann gibt es Funktionen $T_{i_1, \ldots, i_k}^l$, sodass
  \[ T(\coord{i_1}, \ldots, \coord{i_k}) = \sum_{l=1}^n T_{i_1, \ldots, i_k}^l \coord{l}. \]
\end{bem}

\begin{nota}
  $\nabla_v Y \coloneqq (\nabla_X Y)(p)$ für $v \in T_p M$ und $X$ ein VF mit $X_p = v$ (wohldefiniert).
\end{nota}

\begin{satz}
  Sei $\nabla$ ein Zusammenhang auf $M$. Sei $p \in M$, $v \in T_p M$ und $Y, \tilde{Y} \in \VF(M)$. Falls für eine diff'bare Kurve $c : \vinterval \to M$ gilt
  \[
    c(0) = p, \enspace
    \dot{c}(0) = v \enspace \text{und} \enspace
    \fa{t \in \vinterval} Y(c(t)) = \tilde{Y}(c(t)),
  \]
  dann gilt $\nabla_v Y = \nabla_v \tilde{Y}$.
\end{satz}

% §9. Kovariante Ableitung längs Kurven

\begin{defn}
  Ein \emph{VF längs einer Kurve} $c : I \to M$ ist eine Abbildung
  \[
    X : I \to TM
    \quad \text{mit} \quad
    X(t) = X_t \in T_{c(t)} M,
  \]
  welche diff'bar ist, \dh{} für alle $t_0 \in I$ existiert eine Karte $(U, \phi)$ um $c(t_0)$, sodass man schreiben kann
  \[
    X(t) = \sum_{i=1}^n \xi^i(t) \coord{j}|_{c(t)} \quad \text{für alle } t \in c^{-1}(U)
  \]
  mit diff'baren Funktionen $\xi^i : c^{-1}(U) \to \R$.
\end{defn}

\begin{bem}
  $X_t$ muss nicht Einschränkung eines VF auf $M$ sein.
\end{bem}

\begin{nota}
  $\VF_c \coloneqq \{ \text{ Vektorfelder längs $c$ } \}$
\end{nota}

\begin{bem}
  $\VF_c$ ist ein Modul über $\Cont^\infty(I, \R)$.
\end{bem}

\begin{satz}
  Sei $\nabla$ ein Zusammenhang auf $M$, sei $c : I \to M$ eine diff'bare Kurve. Dann gibt es eine eindeutige Abbildung
  \[ \frac{\nabla}{\d t} = \frac{D}{\d t} = \frac{D^\nabla}{\d t} : \VF_c \to \VF_c, \]
  sodass für $X, \tilde{X} \in \VF_c$, $Y \in \VF(M)$ und $f \in \Cont^\infty(I, \R)$ gilt:
  \begin{itemize}
    \miniitem{0.48 \linewidth}{$\tfrac{D}{\d t} (X + \tilde X) = \tfrac{D}{\d t} X + \tfrac{D}{\d t} \tilde{X}$,}
    \miniitem{0.49 \linewidth}{$\tfrac{D}{\d t} (f \cdot X) = f \cdot \tfrac{D}{\d t} X + f' X$,}
    \item $\tfrac{D (Y \circ c)}{\d t} = \nabla_{\dot{c}} Y$.
  \end{itemize}
\end{satz}

\begin{defn}
  $\tfrac{D}{\d t}$ heißt von $\nabla$ \emph{induzierte kovariante Ableitung längs $c$}.
\end{defn}

% Vorlesung vom 11.11.2014

% 9.2.
\begin{satz}
  Sei $(M, g)$ eine Riem. Mft, $\nabla$ der Levi-Civita-Zusammenhang und $c : I \to M$ diff'bar. Dann gilt
  \[ \fa{X, Y \in \VF_c} g(X, Y)' = g(\tfrac{D X}{\d t}, Y) + g(X, \tfrac{D Y}{\d t}). \]
\end{satz}

% §10. Parallelverschiebung

\begin{defn}
  $X \in \VF_c$ heißt \emph{parallel} längs $c$ (bzgl. $\nabla$), wenn $\tfrac{D X}{\d t} = 0$.
\end{defn}

\begin{bem}
  Sei $(U, \phi)$ eine Karte, $\tilde{I} \subset I$ mit $c(\tilde{I}) \subset U$.
  %und $X = \sum_{k=1}^n \xi^k (\coord{k} \circ c)$
  In lokalen Koordinaten lässt sich Parallelität ausdrücken durch
  \[
    (\xi^{k})' + \sum_{i,j=1}^n \dot{c}^i(t) \xi^j (t) \Gamma_{ij}^k(c(t)) = 0
    \quad \text{für $k = 1, \ldots, n$ und alle $t \in \tilde{I}$.}
  \]
  Für die Funktionen $\xi^k$ ist das ein System linearer DGL mit nichtkonstanten Koeffizienten
  \[
    \begin{pmatrix}
      \xi^1 \\
      \vdots \\
      \xi^n
    \end{pmatrix}' = A(t) \cdot
    \begin{pmatrix}
      \xi^1 \\
      \vdots \\
      \xi_n
    \end{pmatrix}
  \]
  Dieses System ist linear beschränkt, es folgt daher die Existenz von parallelen Vektorfeldern in Kartenumgebungen.
\end{bem}

% 10.1
\begin{satz}
  Sei $t_0 \in I = \ointerval{a}{b}$ und $v \in T_{c(t_0)} M$ vorgegeben. \\
  Dann gibt es genau ein VF $X \in \VF_c$ mit
  \[
    \tfrac{D X}{\d t} \equiv 0
    \quad \text{und} \quad
    X(t_0) = v.
  \]
\end{satz}

\begin{defn}
  Die \emph{Parallelverschiebung} längs einer diff'baren Kurve $c : \abinterval \to M$ bzgl. eines Zshg $\nabla$ ist
  \[
    P_c : T_{c(a)} M \to T_{c(b)} M, \enspace v \mapsto X^{v}(b),
    \enspace \text{wobei} \enspace \tfrac{D X^v}{\d t} \!\equiv\! 0 \text{ und } X^{v}(a) \!=\! v.
  \]
\end{defn}

% 10.1 bis
\begin{satz}
  $P_c$ ist linear.
\end{satz}

% 10.2
\begin{satz}
  Ist $(M, g)$ Riem. Mft und $\nabla$ der LC-Zshg, dann gilt
  \[ g_{c(b)}(P_c(v), P_c(w)) = g_{c(a)}(v, w). \]
  Mit anderen Worten: $P_c$ ist eine lineare Isometrie.
\end{satz}

\begin{bem}
  Wir können nun die Definition der Ableitung als Limes des Differenzenquotienten auf Mften übertragen: Sei $v \in T_x M$, $X \in \VF(M)$ und $c : \vinterval \to M$ mit $c(0) = x$ und $\dot{c}(0) = v$. Dann ist
  \[ \nabla_v X = \lim_{t \to 0} \frac{P_{c(t)}^{-1}(X(c(t))) - X(c(0))}{t} \]
\end{bem}

\begin{bem}
  Parallelverschiebung ist auch möglich und sinnvoll entlang stückweise glatter Kurven.
\end{bem}

\begin{defn}
  Die \emph{Holonomiegruppe} von $M$ in $x \in M$ bzgl. $\nabla$ ist
  \[ \Hol_x^\nabla \coloneqq \Set{ P_c : T_x M \to T_x M }{ \text{$c$ stückweise glatt mit $c(0) = c(1) = x$.} } \]
  Dabei ist $P_{c} \circ P_{\tilde{c}} = P_{c \circ \tilde{c}}$ und $(P_c)^{-1} = P_{c^{-1}}$.
\end{defn}

\begin{bem}
  $\mathrm{Hol}_x^\nabla$ ist sogar eine Lie-Gruppe und Untergr. von $O(T_x M, g_x)$.
\end{bem}

% §11. Geodäten

\begin{defn}
  Eine glatte Kurve $c : I \to M$ heißt \emph{Geodäte} bzgl. $\nabla$, falls
  \[
    \frac{D^\nabla \dot{c}}{\d t} \equiv 0, \quad
    \text{\dh{} das Tangential-VF $\dot{c}$ ist parallel längs $c$.}
  \]
\end{defn}

\begin{bem}
  Sei $(U, \phi)$ eine Karte, $\tilde{I} \subset I$ mit $c(\tilde{I}) \subset U$.
  In lokalen Koord. lässt sich diese Bed. ausdrücken durch die \emph{Geodätengleichung}
  \[
    (\ddot{c}^{k})'(t) + \sum_{i,j=1}^n \dot{c}^i(t) \dot{c}^j(t) \Gamma_{ij}^k(c(t)) = 0
    \quad \text{für $k = 1, \nldots, n$ und alle $t \in \tilde{I}$.}
  \]
\end{bem}

% 11.1
\begin{satz}
  Zu jedem $p \in M$ und $v \in T_p M$ gibt es ein $\epsilon > 0$ und genau eine Geodäte $c : \vinterval \to M$ mit $c(0) = p$ und $\dot{c}(0) = v$.
\end{satz}

% 11.2
\begin{satz}
  Seien $c_{1,2} : I_{1,2} \to M$ zwei Geodäten bzgl $\nabla$ mit $0 \in I_1 \cap I_2$. Falls $c_1(0) = c_2(0)$ und $\dot{c}_1(0) = \dot{c}_2(0)$, dann gilt $c_1|_{I_1 \cap I_2} \equiv c_2|_{I_1 \cap I_2}$.
\end{satz}

% 11.3
\begin{satz}
  Gegeben $p \in M$ und $v \in T_p M$, dann gibt es genau ein Intervall $I_v \opn \R$ mit $0 \in I_v$ und eine Geodäte
  \[
    c_v : I_v \to M
    \quad \text{mit} \quad
    c_v(0) = p, \enspace \dot{c}_v(0) = v,
  \]
  die maximal im folgenden Sinn ist: Für jede Geodäte $c : I \to M$ mit $\dot{c}(0) = v$ gilt: $I \subseteq I_v$ und $c = c_v|_I$.
\end{satz}

\begin{nota}
  Für $v \in T_p M$ sei $c_v : I_v \to M$ die zugeh. max. Geodäte.
\end{nota}

\begin{defn}
  Ein Zshg $\nabla$ auf $M$ heißt \emph{vollständig}, wenn $\fa{v \in TM} I_v {=} \R$.
\end{defn}

% §12. Die Exponentialabbildung

% 12.1
\begin{lem}[\emph{Spray-Eigenschaft}]
  Ist $v \in T_p M$, $c_v : I_v \to M$ die maximale Geodäte mit $\dot{c_v}(0) = v$.
  Sei $\lambda \not= 0$, dann ist
  \[
    c_{\lambda v} : I_{\lambda v} \to M, \enspace
    t \mapsto c_v(\lambda t) \quad \text{wobei} \quad
    I_{\lambda v} \coloneqq \tfrac{1}{\lambda} I_v
  \]
  die maximale Geodäte mit $\dot{c_{\lambda v}}(0) = \lambda v$.
\end{lem}

% Vorlesung vom 18.11.2014

\begin{defn}
  Sei $M$ eine Mft mit Zshg $\nabla$ und $p \in M$. Dann heißt
  \[
    \Exp_p : \widetilde{T_p M} \to M, \enspace
    v \mapsto c_v(1), \quad
    \widetilde{T_p M} \coloneqq \Set{ v \in T_p M }{ 1 \in I_v }
  \]
  \emph{Exponentialabbildung} von $\nabla$ in $p$.
\end{defn}

% 12.2
\begin{lem}
  \begin{itemize}
    \item $\widetilde{T_p M}$ ist sternförmig bzgl. $0$
    \item $\fa{v \in \widetilde{T_p M}} \fa{t \in \cinterval{0}{1}} \Exp_p(tv) = c_v(t)$
  \end{itemize}
\end{lem}

% 12.3 und 12.4.
\begin{satz}
  \begin{itemize}
    \item Es gibt eine offene Umgebung $\hat{U} \opn T_p M$ mit $0 \in \hat{U} \subseteq \widetilde{T_p M}$, sodass $\Exp_p|_{\hat{U}} : \hat{U} \to M$ eine $\Cont^\infty$-Abbildung ist.
    \item Wir können $\hat{U}$ so wählen, dass $\Exp_p|_{\hat{U}} : \hat{U} \to \Exp_p(\hat{U})$ ein Diffeomorphismus ist.
  \end{itemize}
\end{satz}

\begin{bem}
  Man kann zeigen: \enspace
  \inlineitem{$\widetilde{T_p M} \opn T_p M$}
  \begin{itemize}
    \item $\Exp_p : \widetilde{T_p M} \to M$ ist überall $\Cont^\infty$, aber nicht überall ein lokaler Diffeomorphismus (Schnittpunkt-Phänomen)
    \item Ist $(M, \nabla)$ geodätisch vollständig, dann gilt $\widetilde{T_p M} = T_p M$.
  \end{itemize}
\end{bem}

% Vorlesung vom 21.11.2014

% §13. Erste Variationsformel

\begin{defn}
  Eine Kurve $c : I \to M$ heißt \emph{nach} / \emph{proportional zur BL parametrisiert}, wenn gilt:
  \[
    \norm{\dot{c}(t)} \equiv 1
    \quad / \quad
    \norm{\dot{c}(t)} \equiv \text{konst}.
  \]
\end{defn}

\begin{bem}
  \begin{itemize}
    \item Jede Geodäte ist proportional zur BL parametrisiert.
    \item Eine Kurve ist genau dann prop. zur BL parametrisiert, wenn es $\alpha \geq 0$ gibt mit $L(c|_{\abinterval}) = \alpha \cdot (b - a)$.
  \end{itemize}
\end{bem}

\begin{defn}
  Eine \emph{Variation} von $c : \abinterval \to M$ ist eine $\Cont^\infty$-Abbildung
  \[
    \vinterval \times \abinterval \to M, \quad
    (s, t) \mapsto \alpha(s, t)
    \quad \text{mit} \enspace
    \fa{t \in \abinterval} \alpha(0, t) = c(t).
  \]
  Sie heißt \emph{Variation mit festen Endpunkten}, wenn zudem gilt:
  \[
    \fa{s \in \vinterval}
    \alpha(s, a) = c(a) \wedge \alpha(s, b) = c(b)
  \]
\end{defn}

\begin{sprech}
  $s$ heißt \emph{Variationsparameter}
\end{sprech}

\begin{defn}
  Eine Variation einer stückweise glatten Kurve $c : \abinterval \to M$ (mit $c$ glatt auf den Teilintervallen $\cinterval{t_{i-1}}{t_i}$) ist eine stetige Abb.
  \[
    \alpha : \vinterval \times \abinterval \to M,
    \enspace (s, t) \mapsto \alpha_s(t)
    \quad \text{mit }
    \alpha|_{\vinterval \times \cinterval{t_{i-1}}{t_i}} \text{ ist } \Cont^\infty.
  \]
\end{defn}

\begin{nota}
  \begin{itemize}
    \item $\tfrac{\partial \alpha}{\partial s} (s_0, t_0)$ ist der Tang.-Vektor an $s {\mapsto} \alpha(s, t_0)$ in $s_0$.
    \item $\tfrac{\partial \alpha}{\partial t} (s_0, t_0)$ ist der Tangentialvektor an $s \mapsto \alpha(s_0, t)$ in $t_0$.
  \end{itemize}
\end{nota}

\begin{defn}
  Eine Abbildung $X : \vinterval \times \abinterval \to TM$ mit $X(s, t) \in T_{\alpha(s, t)} M$ heißt \emph{Vektorfeld längs $\alpha$}, wenn $X$ differenzierbar (bzw. stückweise diff'bar) ist.
\end{defn}

\begin{nota}
  Sei $X$ ein VF längs $\alpha(s, t)$. Dann
  \begin{align*}
    \tfrac{D X}{\partial s} (s_0, t_0) & \coloneqq \tfrac{D}{\d s} |_{s=s_0} (s \mapsto X(s, t_0)) \\
    \tfrac{D X}{\partial t} (s_0, t_0) & \coloneqq \tfrac{D}{\d t} |_{t=t_0} (s \mapsto X(s_0, t))
  \end{align*}
\end{nota}

% 13.1
\begin{lem}
  $\frac{D}{\partial s} \frac{\partial \alpha}{\partial t} = \frac{D}{\partial t} \frac{\partial \alpha}{\partial s}$
\end{lem}

% 13.2
\begin{satz}[\emph{1. Variationsformel}]
  Sei $\alpha : \vinterval \times \abinterval \to M$ eine $\Cont^\infty$-Variation von einer $\Cont^\infty$-Kurve $c = \alpha_0 : \abinterval \to M$. Sei $\norm{\dot{c}(t)} = \text{konst} \not= 0$. Dann gilt mit $X(t) \coloneqq \tfrac{\partial \alpha}{\partial s} (0, t)$
  \[ \frac{\d}{\d s} |_{s=0} L(\alpha_s) = \frac{1}{\norm{\dot{c}}} \left( g(X, \dot{c})|_a^b - \Int{a}{b}{g(X(\tau), \tfrac{D \dot{c}}{\d t})}{\tau} \right) \]
\end{satz}

\begin{sprech}
  $X(t) \!\coloneqq\! \tfrac{\partial \alpha}{\partial s} (0, t)$ heißt \emph{Variationsvektorfeld} (VVF).
\end{sprech}

% Satz 13.2bis

\begin{satz}[\emph{1. Variationsformel} für stückweise glattes $c$]\mbox{}\\
  Sei $\alpha : \vinterval \times \abinterval \to M$ eine stückweise glatte Variation, glatt auf $\vinterval \times \cinterval{t_{i-1}}{t_i}$ mit $a = t_0 < \ldots < t_k = b$. Dann ist
  \begin{align*}
    \frac{\d \alpha_s}{\d s} |_{s=0} = & \frac{1}{\norm{\dot{c}}} \left( g(X, \dot{c})|_a^b + \sum_{i=1}^{k-1} g(X(t_i), \nabla_i \dot{c}) - \Int{a}{b}{g(X, \frac{D \dot{c}}{\d t})}{t} \right) \\
    & \text{mit } \nabla_i \dot{c} = \dot{c}(t_i^-) - \dot{c}(t_i^+)
  \end{align*}
\end{satz}

\begin{nota}
  $\dot{c}(t_i^+) = \lim_{t \downarrow t_i} \dot{c}(t)$, \enspace
  $\dot{c}(t_i^-) = \lim_{t \uparrow t_i} \dot{c}(t)$
\end{nota}

% 13.3
\begin{satz}
  Zu jedem (stückw.) glatten $X \in \VF_c$ gibt es eine (stückw.) glatte Variation $\alpha$ von $c$ mit $X = \tfrac{\partial \alpha}{\partial s} (0, t)$. Wenn $X(a) = X(b) = 0$, so kann man $\alpha$ als Variation mit festen Endpunkten wählen.
\end{satz}

% 13.4
\begin{satz}
  %Für eine stückweise glatte Kurve $c : \abinterval \to M$ mit $\norm{\dot{c}} = \text{konst}$ sind äquivalent:
  Für $c : \abinterval \to M$ stückw. glatt mit $\norm{\dot{c}} = \text{konst}$ sind äquiv.:
  \begin{itemize}
    \item $c$ ist eine Geodäte
    \item $\tfrac{\d}{\d s}|_{s=0} L(\alpha_s) = 0$ für jede stückweise glatte Variation $\alpha$ von $c$ mit festen Endpunkten.
  \end{itemize}
\end{satz}

% Vorlesung vom 28.11.2014

\begin{kor}
  Sei $c : \abinterval \to M$ stückweise glatt und kürzeste stückweise Verbindung ihrer Endpunkte (\dh{} für alle $\tilde{c} : \abinterval \to M$ stückweise glatt mit $c(a) = \tilde{c}(a)$ und $c(b) = \tilde{c}(b)$ ist $L(c) \leq L(\tilde{c})$). Dann ist $c$ eine glatte Geodäte.
\end{kor}

\begin{acht}
  Geodäten sind i.\,A. nicht global kürzeste Verbindungen, die Umkehrung gilt also nicht!
\end{acht}

\begin{nota}
  $\Omega_{p,q} \coloneqq \Set{ c : \cinterval{0}{1} \to M }{ c(0) {=} p, c(1) {=} q, \text{$c$ stückw. glatt } }$
\end{nota}

\begin{bem}
  Geodäten sind "`kritische Punkte"' von $L : \Omega_{p,q} \to \R$ unter der Nebenbedingung $\norm{\dot{c}} = \text{konst}$. Ersetzt man das Längenfunktional durch die Energie, so ist diese NB unnötig.
\end{bem}

% §14. Geodäten sind lokal kürzeste

\begin{nota}
  $S_\rho(0) = \Set{ x \in T_p M }{ \norm{x} = \rho }$
\end{nota}

% 14.1
\begin{satz}[\emph{Gaußlemma}] % 1827
  Sei $(M, g)$ eine zshgde Riem. Mft, $\nabla = \LC$. Sei $p \in M$ und $\epsilon > 0$, sodass
  \[ \Exp_p|_{B_{\epsilon}(0)} : B_{\epsilon}(0) \to \Exp_p(B_{\epsilon}(0)) \]
  ein Diffeo ist. Dann schneiden die radialen Geodäten
  \[ t \mapsto \Exp_p(tv) = c_v(t), \quad v \in T_p M \setminus \{ 0 \},  \]
  die Hyperflächen $\Exp_p(S_\rho(0))$, $\rho \in \ointerval{0}{\epsilon}$ orthogonal.
\end{satz}

% 14.2
\begin{satz}
  Seien $p \in M$, $\epsilon > 0$, $\rho \in \cointerval{0}{\epsilon}$ wie eben. Dann ist
  \[
    c_v|_{\cinterval{0}{\rho}} : \cinterval{0}{\rho} \to M, \quad
    t \mapsto c_v(t) = \Exp_p(t v) \qquad
    (v \in T_p M, \norm{v} = 1)
  \]
  die kürzeste Verbindung ihrer Endpunkte, genauer: \\
  Es gilt $\rho = L(c_v|_{\cinterval{0}{\rho}}) \leq L(\gamma)$ für jedes $\gamma : \abinterval \to M$ stückweise glatt mit $\gamma(a) = p$, $\gamma(b) = c_v(\rho)$.
  Gleichheit gilt genau dann, wenn $\gamma(t) = c_v(r(t))$ mit $r : \abinterval \to \cinterval{0}{\rho}$ monoton wachsend.
\end{satz}

% Vorlesung vom 2.12.2014

\begin{defn}
  $i(p) \coloneqq \sup \Set{ \epsilon > 0 }{ \Exp_p|B_{\epsilon}(0) \text{ ist Diffeo aufs Bild } }$
  heißt \emph{Injektivitätsradius} von $M$ in $p$.
\end{defn}

% 14.3
\begin{satz}
  Sei $M$ eine zshgde Riemannsche Mannigfaltigkeit.
  \begin{itemize}
    \item Ist $p \in M$, $\epsilon \in \ointerval{0}{i(p)}$, dann ist
    \begin{align*}
      \Exp_p(B_\epsilon(0)) & = B_\epsilon(p) \coloneqq \Set{q \in M}{d(p, q) < \epsilon},\\
      \Exp_p(S_\epsilon(0)) & = S_\epsilon(p) \coloneqq \Set{q \in M}{d(p, q) = \epsilon}.
    \end{align*}
    \item $d : M \times M \to \R_{\geq 0}$ ist eine Metrik.
    \item Die durch $d$ ind. Topologie stimmt mit der gegebenen überein.
  \end{itemize}
\end{satz}

% Vorlesung vom 5.12.2014

% 15.1
\begin{satz}[\emph{Hopf-Rinow} 1]
  Sei $M$ eine zshgde Riem. Mft, $p \in M$. Angenommen, alle Geodäten $\gamma$ auf $M$ mit $\gamma(0) = p$ sind auf ganz $\R$ definiert (m.a.W: $\Exp_p$ ist auf ganz $T_p M$ definiert). Dann gibt es für alle $q \in M$ eine kürzeste Geodäte von $p$ nach $q$.
\end{satz}

% 15.2
\begin{satz}[\emph{Hopf-Rinow} 2]
  Für eine zusammenhängende Riemannsche Mannigfaltigkeit $M$ sind äquivalent:
  \begin{itemize}
    \item $M$ ist geodätisch vollständig.
    \item $\fa{p \in M} \Exp_p$ ist auf ganz $T_p M$ definiert.
    \item Beschränkte und abgeschlossene Teilmengen von $M$ sind kompakt.
    \item $M$ ist ein vollständiger metrischer Raum.
  \end{itemize}
\end{satz}

\begin{kor}
  Jede kompakte Riemannsche Mft ist geodätisch vollständig und zwei ihrer Punkte können durch eine kürzeste Geodäte verbunden werden.
\end{kor}

\begin{kor}
  Unter-Mften des $\R^n$ sind geodätisch vollständig.
\end{kor}

% §16. Krümmung

\begin{defn}
  Der \emph{Krümmungstensor} von einem Zshg $\nabla$ auf $M$ ist
  \begin{align*}
    R^{\nabla} & = R : \VF(M) \times \VF(M) \times \VF(M) \to \VF(M) \\
    (X, Y, Z) & \mapsto R(X, Y)Z \coloneqq \nabla_X (\nabla_Y Z) - \nabla_Y (\nabla_X Z) - \nabla_{[X, Y]} Z.
  \end{align*}
\end{defn}

\begin{bem}
  $R^\nabla$ ist ein (1,3)-Tensor.
\end{bem}

\begin{nota}
  $R_p(u, v)w \coloneqq (R(X, Y)Z)(p)$ für $u, v, w \in T_p M$, wobei $X, Y, Z \in \VF(M)$ mit $X(p) \!=\! u$, $Y(p) \!=\! v$, $Z(p) \!=\! w$.
\end{nota}

% 16.1
\begin{satz}
  Es gilt für $X,Y,Z,W \in \VF(M)$:
  \begin{itemize}
    \item $- R(X, Y)Z = R(Y, X)Z$
    \item Falls $\nabla$ torsionsfrei: \emph{1. Bianchi-Identität} / \emph{Jacobi-Identität}:
    \[ R(X,Y)Z + R(Z,X)Y + R(Y,Z)X = 0. \]
    \item Ist $(M, g)$ Riemannsch und $\nabla$ metrisch, dann gilt
    \[ g(R(X,Y)Z, W) = -g(R(X,Y)W, Z). \]
    \item Ist $\nabla$ der LC-Zshg von $(M, g)$ Riemannsch, dann ist
    \[ g(R(X, Y)Z, W) = g(R(Z,W)X,Y). \]
  \end{itemize}
\end{satz}

% Übung: Durch Polarisation genügt es, $\fa{X,Y \in \VF(M)} g(R(X,Y)X, Y)$ zu kennen

\begin{defn}
  Sei $p \in M$, $\sigma = \Span(v,w) \in T_p M$ ein 2-dim UVR. Dann heißt
  \[ \sec(\sigma) = \kappa(\sigma) \coloneqq \frac{g(R(v,w)w,v)}{\norm{v}^2 \cdot \norm{w}^2 - g(v,w)^2} \]
  \emph{Riemannsche Schnittkrümmung} von $\sigma$.
\end{defn}

% 16.2
\begin{lem}
  $\sec(\sigma)$ ist unabhängig von der Basiswahl.
\end{lem}

% §17. 2. Variation der Länge

% 17.1
\begin{satz}
  Sei $\alpha : \vinterval \times \abinterval \to M$ eine glatte Variation einer Kurve $\alpha_0 : \abinterval \to M, t \mapsto \alpha(0, t)$.
  Sei $X : \vinterval \times \abinterval \to TM$ ein VF längs $\alpha$. Dann gilt:
  \[
    \frac{D}{\partial s} \frac{DX}{\partial t} - \frac{D}{\partial t} \frac{DX}{\partial s} = R\left(\frac{\partial \alpha}{\partial s}, \frac{\partial \alpha}{\partial t}\right) X
  \]
\end{satz}

% 17.2
\begin{satz}[\emph{2. Variationsformel} für die Länge]
  Sei $c: \abinterval \to M$ eine Geodäte, $\alpha : \vinterval \times \abinterval \to M$ eine glatte Variation von $c$ mit festen Endpunkten, $X(t) \coloneqq \tfrac{\partial \alpha}{\partial s}(0, t) \in \VF_c$ das VVF mit $X^\perp \coloneqq X - g(X, \tfrac{\dot{c}}{\norm{\dot{c}}}) \tfrac{\dot{c}}{\norm{\dot{c}}}$ senkrechtem Anteil zu $\dot{c}$. Dann gilt
  \[
    \frac{\d^2}{\d s^2} |_{s=0} L(\alpha_s) = \tfrac{1}{\norm{\dot{c}}} \Int{a}{b}{\norm{\tfrac{D X^\perp}{\d t}}^2 - g(R(X,\dot{c})\dot{c},X)}{t}.
  \]
\end{satz}

% §18. Satz von Myers

\begin{defn}
  Der \emph{Durchmesser} einer Riemannschen Mft $(M, g)$ ist
  \[ \diam(M) \coloneqq \sup \Set{d(p,q)}{ p,q \in M }. \]
\end{defn}

% 18.1
\begin{satz}[\emph{Myers} 1935]
  Jede vollständige zshgde Riem. Mft. mit $\sec \geq \delta > 0$ ist kompakt mit Durchmesser $\diam(M) \leq \tfrac{\pi}{\sqrt{\delta}}$.
\end{satz}

\begin{bem}
  Das Bsp der Sphären zeigt: Die Schranke ist optimal.
\end{bem}

% 18.2
\begin{kor}
  Sei $M$ eine vollständige zshge Mft, $\dim(M) \geq 2$ mit $\sec \geq \delta > 0$. Dann ist $\pi_1(M)$ endlich.
\end{kor}

\begin{defn}
  Sei $p \in M$, $v \in T_p M$ mit $\norm{v} = 1$, $v = e_1, e_2, \ldots, e_n$ eine ONB von $T_p M$. Die \emph{Ricci-Krümmung} von $M$ in Richtung $v$ ist dann
  \[ \Ric(v) \coloneqq \sum_{j=2}^n \sec(\Span(v, e_j)). \]
\end{defn}

\begin{bem}
  $\Ric(v)$ ist unabhängig von der Wahl der ONB:
  \begin{align*}
    \Ric(v) & = \sum_{j=2}^n \sec(v,e_j) = \sum_{j=2}^n g(R(e_j,v)v, e_j) =\\
    & = \sum_{j=1}^n g(R(e_j,v)v, e_j) = \spur(x \mapsto R(x,v)v)
  \end{align*}
\end{bem}

\begin{defn}
  $\Ric_p : T_p M \times T_p M \to \R, \enspace (v,w) \mapsto \spur(x \mapsto R(x,v)w)$\\
  heißt \emph{Ricci-Tensor}.
\end{defn}

\begin{bem}
  Der Ricci-Tensor ist ein (2,0)-Tensor und es gilt: \\
  \inlineitem{$\Ric_p(v,w) = \Ric_p(w,v)$,} \quad
  \inlineitem{$\Ric(v) = \Ric(v,v)$.}
\end{bem}

\begin{defn}
  $(M, g)$ heißt \emph{Einstein-Mft}, wenn die Ricci-Krümmung konstant ist, \dh{}
  $\fa{p \in M} \fa{x,y \in T_p M} \Ric(x,y) = c \cdot g(x,y)$.
\end{defn}

\begin{beob}
  \begin{itemize}
    \item $\sec \geq \delta \implies \Ric(v) \geq (n{-}1) \delta$
    \item Mften mit konstanter Schnittkrümmung sind Einstein.
  \end{itemize}
\end{beob}

% 18.3
\begin{satz}[\emph{Myers}]
  Jede vollständige zshgde Riem. Mft. mit $\Ric \geq (n{-}1)\delta$ ist kompakt mit Durchmesser $\diam(M) \leq \tfrac{\pi}{\sqrt{\delta}}$.
\end{satz}

% §19. Jacobi-Felder

\begin{defn}
  Sei $(M, g)$ eine Riem. Mft, $c : I \to M$ glatt, $Y \in \VF_c$ heißt \emph{Jacobi-Feld}, wenn die \emph{Jacobi-Gleichung} gilt:
  \[
    Y'' + R(Y, \dot{c}) \dot{c} = 0 \quad
    \left( Y'' \coloneqq \tfrac{D}{\d t} \left( \tfrac{D Y}{\d t} \right) \right).
  \]
\end{defn}

\begin{bem}
  Die Jacobi-Gleichung ist linear in $Y$, somit ist $\Set{ X \in \VF_c }{ X \text{ ist ein Jacobi-Feld} }$ ein UVR von $\VF_c$.
\end{bem}

% 19.1
\begin{satz}
  Sei $c : \abinterval \to M$ eine Geodäte, $\alpha : \vinterval \times \abinterval \to M$ eine glatte Variation von $c = \alpha_0$ durch Geodäten (\dh{} $\alpha_s$ ist Geodäte für alle $s \in \vinterval$). Dann ist das VVF $X = \tfrac{\partial \alpha}{\partial s}(0,t)$ ein Jacobi-Feld.
\end{satz}

% 19.2
\begin{satz}
  Sei $c : I \to M$ eine Kurve, $t_0 \in I$. Dann gibt es für alle $v, w \in T_{c(t_0)} M$ genau ein Jacobi-Feld $Y \in \VF_c$ mit
  \[
    Y(t_0) = v
    \quad \text{und} \quad
    Y'(t_0) = w.
  \]
\end{satz}

% 19.3
\begin{satz}
  Sei $v \in T_p M$, $w \in T_p M \cong T_v \left( T_p M \right)$. Dann gilt $(\Exp_p)_{*v}(w) = Y(1)$, wobei $Y \in \VF_c$ ein Jacobi-Feld längs $c_v(t) = \Exp_p(tv)$ mit $Y(0) = 0$ und $Y'(0) = w$.
\end{satz}

% Vorlesung vom 9.1.2015

% §20. Satz von Hadamard-Cartan

% 20.1
\begin{satz}
  Sei $Y$ ein Jacobifeld längs einer Geodäten $c$ in $(M, g)$. \\
  Wenn $\sec \leq 0$, dann gilt
  \begin{itemize}
    \item $(t \mapsto \norm{Y(t)}^2)$ ist konvex.
    \item Wenn $Y$ zwei verschiedene Nullstellen hat, dann $Y \equiv 0$.
    \item Es gibt keine konjugierten Punkte längs $c$.
  \end{itemize}
\end{satz}

% 20.2
\begin{kor}
  Falls $(M, g)$ vollständig mit $\sec \leq 0$, dann ist $\Exp_p$ für alle $p$ ein lokaler Diffeomorphismus, \dh{}
  \[
    \fa{v \in T_p M} \ex{U_v \opn T_p M} \Exp_p |_{U_v} : U_v \to \Exp_p(U_v)
    \enspace \text{ist Diffeo.}
  \]
\end{kor}

% 20.3
\begin{satz}
  Sei $X$ wegzshgd, $Y$ einfach zshgd, $\pi : X \to Y$ eine Überlagerung. Dann ist $\pi$ ein Homöomorphismus.
\end{satz}

\begin{defn}
  Eine Abbildung $\pi : (M_1, g_1) \to (M_2, g_2)$ zwischen Riemannsche Mften heißt \emph{Riemannsche Überlagerung}, wenn gilt:
  \begin{itemize}
    \miniitem{0.66 \linewidth}{$\pi$ ist eine topologische Überlagerung}
    \miniitem{0.27 \linewidth}{$\pi$ ist diffbar}
    \item $\pi_{*p} : T_p M_1 \to T_{\pi(p)} M_2$ ist eine orthogonale Abb f.\,a. $p \in M_1$.
  \end{itemize}
\end{defn}

% 20.4
\begin{satz}
  Sei $\pi : (M_1, g_1) \to (M_2, g_2)$ eine surjektive lokale Isometrie zwischen Riem. Mften. Wenn $M_1$ vollständig ist, dann ist $\pi$ eine Riemannsche Überlagerung.
\end{satz}

% 20.5
\begin{satz}[\emph{Cartan-Hadamard}]
  Sei $(M, g)$ eine vollständige, zshgde Riemannsche Mft. mit Schnittkrümmung $\sec \leq 0$, $p \in M$. \\
  Dann ist $\Exp_p : T_p M \to M$ eine Überlagerung.
\end{satz}

% Vorlesung vom 13.1.2015

% 20.6
\begin{kor}
  Falls $(M^n, g)$ zusätzlich einfach zshgd ist, dann gilt $M \cong \R^n$.
  Je zwei Punkte in $M$ lassen sich durch genau eine nach BL param. Geodäte verbinden (bis auf Umkehrung, Parametershift).
\end{kor}

% §21. Satz von Synge

% 21.1
\begin{satz}[\emph{Synge} 1936]
  Jede zshgde kompakte orientierte Riem. Mft gerader Dimension mit $\sec > 0$ ist einfach zshgd.
\end{satz}

% 21.2
\begin{satz}[\emph{Weinstein} 1968, \emph{Synge} 1936]
  Sei $M^n$ kompakte, zshgde, orientierte Riem. Mft, $\sec > 0$, $n$ gerade. Sei $f : M^n \to M^n$ eine orientierungstreue Isometrie. Dann hat $f$ einen Fixpunkt.
\end{satz}

% Vorlesung vom 16.1.2015

% §22. Symmetrische Räume

% 22.1.
\begin{prop}
  Sei $(M, g)$ eine vollständige Riem. Mft, $p \in M$. Seien $f, g \in \Iso(M)$. Wenn $f(p) = g(p)$ und $f_{*p} = g_{*p}$, dann gilt $f \equiv g$.
\end{prop}

\begin{defn}
  Eine zshgde Riem. Mft $P$ heißt \emph{Symmetrischer Raum}, wenn
  \[
    \fa{p \in P} \ex{s_p \in \Iso(P)}
    s_p(p) = p \enspace \wedge \enspace (s_p)_{*p} = - \id_{T_p P}.
  \]
\end{defn}

\begin{sprech}
  $s_p$ heißt \emph{(geodätische) Spiegelung} in $p$.
\end{sprech}

% 22.2
\begin{lem}
  Sei $P$ ein symmetrischer Raum, $\gamma : \vinterval \to P$ eine Geodäte, $p = \gamma(0)$. Dann gilt
  $\fa{t \in \vinterval} (s_p \circ \gamma)(t) = \gamma(-t)$.
\end{lem}

% 22.3
\begin{lem}
  Sei $P$ ein sym. Raum, $\gamma : \vinterval \to P$ eine Geodäte, $\gamma(0) = p$, $\tau \in \vinterval$, $q \coloneqq \gamma(\tau)$. Dann gilt $(s_q \circ s_p)(\gamma(t)) = \gamma(t + 2 \tau)$, wenn $t + 2 \tau \in \vinterval$.
\end{lem}

\begin{kor}
  Symmetrische Räume sind geodätisch vollständig.
\end{kor}

\begin{defn}
  Eine Riem. Mft $M$ heißt \emph{homogen} (homogener Raum), wenn
  \[ \fa{p, q \in M} \ex{f \in \Iso(M, g)} f(p) = q. \]
\end{defn}

% 22.5
\begin{lem}
  Symmetrische Räume sind homogen.
\end{lem}

% 22.6
\begin{lem}
  Sei $P$ ein symm. Raum, $p, q \in P$, $f \in \Iso(P)$ mit $f(p) = q$. Dann gilt $s_q = f \circ s_p \circ f^{-1}$.
\end{lem}

% 22.7
\begin{kor}
  Ist $(M, g)$ eine homogene zshgde Riem. Mft, sodass
  \[
    \ex{m \!\in\! M} \ex{s_m \!\in\! \Iso(M)} s_m(m) = m
    \enspace \text{und} \enspace
    (s_m)_{*m} = - \id_{T_m M}.
  \]
  Dann ist $M$ ein symmetrischer Raum.
\end{kor}

\begin{defn}
  Sei $M$ eine Mft mit Zshg $\nabla$. Sei $T$ ein Tensorfeld auf $M$ vom Typ $(1,k)$. Dann ist $\nabla T$ das durch
  \[
    (\nabla T)(X_1, \nldots, X_k, Y) \coloneqq \nabla_Y(T(X_1, \nldots, X_k)) - \sum_{i=1}^k T(X_1, \nldots, \nabla_Y X_i, \nldots, X_k)
  \]
  definierte Tensorfeld vom Typ $(1, k+1)$.
\end{defn}

\begin{bsp}
  Sei $(M, g)$ Riem, $\nabla = \LC$. Dann gilt $\nabla g = 0$ ($\nabla$ metrisch).
\end{bsp}

\begin{defn}
  $T$ heißt \emph{parallel}, wenn $\nabla T = 0$.
\end{defn}

\begin{satz}
  $P$ symmetrisch $\implies$ $\LC R = 0$
\end{satz}

\begin{bem}
  Die Umkehrung gilt nur lokal.
\end{bem}

% Vorlesung vom 20.1.2015

% §23. Transvektionen und Holonomie

\begin{nota}
  Sei $P$ im Folgenden ein symmetrischer Raum.
\end{nota}

\begin{defn}
  Eine \emph{Transvektion} von $P$ ist eine Isometrie der Form
  \[ t_{pq} = s_p \circ s_q \quad \text{mit } p, q \in P, \]
  \dh{} ein Produkt geodätischer Spiegelungen.
\end{defn}

\begin{bsp}
  Im $\R^n$ sind die Transvektionen genau die Translationen.
\end{bsp}

\begin{defn}
  Die von den Transvektionen erzeugte abgeschl. Untergruppe
  \[ \Trans(P) \coloneqq \langle t_{pq} \,|\, p, q \in P \rangle_{c} \]
  heißt \emph{Transvektionsgruppe} von $P$.
\end{defn}

% 23.1
\begin{lem}
  Sei $\gamma : \R \to P$ eine Geodäte, $p = \gamma(0)$, $X \in \VF_\gamma$ parallel. Sei
  \[
    Y \coloneqq (s_p)_* X : \R \to TP, \quad
    t \mapsto (s_p)_{*\gamma(t)} X(t)
  \]
  Dann gilt $Y(t) = -X(-t)$.
\end{lem}

% 23.2
\begin{lem}
  Sei $\gamma : \R \to P$ eine Geodäte. Dann gilt f.\,a. $\tau \in \R$:
  \begin{itemize}
    \item $(t_{\gamma(\tau)\gamma(0)} \circ \gamma)(t) = \gamma(t + 2\tau)$
    \item $(t_{\gamma(\tau)\gamma(0)*} X)(t) = X(t + 2 \tau)$ für $X \in \VF_\gamma$ parallel.
  \end{itemize}
\end{lem}

% 23.3
\begin{lem}
  Sei $\gamma : \R \to P$ eine Geodäte. Dann ist die Abbildung
  \[
    t^\gamma : \R \to \Iso(P), \quad
    \tau \mapsto t_{\gamma(\tau)\gamma(0)}
  \]
  eine Ein-Parameter-Untergruppe.
\end{lem}

\begin{defn}
  $t^\gamma : \R \to \Iso(P)$ heißt \emph{Transvektion} längs $\gamma$.
\end{defn}

% 23.4
\begin{satz}
  Jede maximale Geodäte in $P$ ist Bahn einer 1-Parameter-UG von Isometrien, nämlich von $\gamma(\tau) \coloneqq (t^\gamma(\tau))(c(0))$.
\end{satz}

\begin{defn}
  $\lambda \in \R_{\geq 0}$ heißt \emph{Periode} einer Geodäten $\gamma$, wenn f.\,a. $t \in \R$ gilt: $\gamma(t) = \gamma(t + \lambda)$. Die Menge aller Perioden wird mit $P_\gamma$ bezeichnet.
\end{defn}

% Ausgelassen: Hilfslemma

% 23.5
\begin{lem}
  Sei $b > a$ und $c(a) = c(b)$. Dann ist $\lambda \coloneqq b - a \in P_\gamma$.
\end{lem}

\begin{kor}
  Hat eine Geodäte $\gamma$ in $P$ einen Selbstschnitt, so ist $\gamma$ periodisch. Sei $\lambda_0$ die minimale nichttriviale Periode einer nichttrivialen Geodäten $\gamma$ in $P$. Dann ist $\gamma|_{\cointerval{t}{t+\lambda_0}}$ injektiv für alle $t$.
\end{kor}

% Vorlesung vom 23.1.2015

\begin{satz}[\emph{Sphärensatz}]
  Sei $M^n$ eine kompakte, einfach zsghde Riem. Mft. mit $\tfrac{1}{4} < \sec \leq 1$. Dann ist $M$ diffeomorph zur $n$-Sphäre.
\end{satz}

\begin{defn}
  Sei $M$ eine Riem. Mft, $p \in M$.
  \[
    \Iso_p(M) \coloneqq \Set{ f \in \Iso(M) }{ f(p) = p } \quad
    \text{heißt \emph{Isotropiegruppe} von $p$.}
  \]
\end{defn}

% 23.7
\begin{lem}
  Seien $p, q \in M$, $f \in \Iso(M)$ mit $f(p) = q$. Dann ist
  \[ \Iso_q(M) = \Set{ f \circ g \circ f^{-1} }{ g \in \Iso_p(M) }. \]
\end{lem}

\begin{kor}
  Ist $M$ homogen, so sind alle Isotropiegruppen isomorph.
\end{kor}

\begin{bem}
  Sei $M$ zshgd, vollständig. Dann ist
  \[
    \phi : \Iso_p(M) \to O(T_p M), \quad
    f \mapsto f_{*p}
  \]
  ein injektiver Gruppenhomomorphismus.
\end{bem}

% 23.8
\begin{satz}
  $\phi(\Iso_p(M))$ ist abgeschlossen in $O(T_p M)$, also kompakt.
\end{satz}

% Ist vorher schon definiert
\iffalse
\begin{defn}
  Sei $M$ eine Riem. Mft, $p \in M$. Dann heißt
  \[
    \Hol_p(M) \coloneqq \Set{ P_c }{ c : \I \to M \text{ stückw. reg. Kurve, } c(0) \!=\! c(1) \!=\! p }
  \]
  \emph{Holonomiegruppe} von $p$ ($\Hol_p(M) \subseteq O(T_p M)$).
\end{defn}
\fi

% 23.9
\begin{satz}
  Sei $P$ ein sym. Raum. Dann ist $\Hol_p(P) \subseteq \phi(\Iso_p(P))$.
\end{satz}

\begin{bem}
  \begin{itemize}
    \item Umkehrung: Sei $M$ einfach zshgde, Riem. Mft. mit $\Hol_p(M) \subseteq \phi(\Iso_p(M))$. Dann ist $P$ ein symmetrischer Raum. % Berger/Simons/Olmos
    \item Für $P = \R^n$, $p = 0$ gilt $\Hol_p(P) \subsetneq \phi(\Iso_p(P))$.
    \item Für eine zshgde, vollst. Riem. Mft. $M$ gilt:
    \begin{align*}
      \phi(\Iso_p(M)) & = \text{Normalisator von $\Hol_p(M)$ in $O(T_p M)$} \\
      & = \Set{ g \in O(T_p M) }{ g \Hol_p(M) g^{-1} = \Hol_p(M) }.
    \end{align*}
  \end{itemize}
\end{bem}

% 23.10
\begin{satz}
  Sei $P$ ein kompakter sym. Raum. Dann ist $\pi_1(P)$ abelsch.
\end{satz}

% Ausgelassen: Hilfslemma: Ist $G \to G, g \mapsto g^{-1}$ ein Gruppenhomo, dann ist $G$ abelsch.

% Vorlesung vom 27.1.2015

\begin{defn}
  Eine \emph{Darstellung} einer Gruppe $G$ ist ein Gruppenhomo- morphismus $\rho : G \to \GL(V)$ mit $V$ ein Vektorraum.
\end{defn}

\begin{defn}
  Eine Darstellung $\rho : G \to \GL(V)$ heißt \emph{irreduzibel}, wenn
  \[ \fa{U \subset V \text{ UVR}} (\fa{g \in G} \rho(g)(U) = U) \implies U \in \{ \{ 0 \}, V \}. \]
\end{defn}

\begin{satz}[de Rham]
  Sei $(M, g)$ eine einfach zshgde vollständige Riem. Mft. Dann ist $M$ isometrisch zu einem Riemannschen Produkt
  \[ M \cong M_0 \times M_1 \times \ldots \times M_k \quad \text{mit} \]
  \begin{itemize}
    \item $M_0$ ist ein euklidischer VR (evtl. $\{ 0 \}$)
    \item $M_1, \ldots, M_k$ sind vollständige, einfach zshgde, unzerlegbare (im Sinne dieses Satzes) Riem. Mft, für die gilt:
    $\Hol_{p_j}(M_j)$ wirkt irreduzibel auf $T_{p_j} M_j$.
  \end{itemize}
\end{satz}

\begin{defn}
  Eine Riem. Mft $(M, g)$ heißt \emph{Isotropie-irreduzibel}, wenn gilt:
  Für alle $p \in M$ wirkt $\Iso_p(M)$ irreduzibel auf $T_p M$.
\end{defn}

% 23.11
\begin{satz}
  Sei $P$ ein zshgder, de-Rham-unzerlegbarer sym. Raum. \\
  Dann ist $P$ Isotropie-irreduzibel.
\end{satz}

% 23.12
\begin{lem}
  Sei $(M, g)$ ein Isotropie-irreduzibler homogener Raum und $B : TM \times TM \to \R$ ein symmetrischer $(0,2)$-Tensor. \\
  Angenommen, $B$ ist Isometrie-invariant, \dh{}
  \[ \fa{f \!\in\! \Iso(P)} \fa{p \!\in\! M} \fa{x, y \!\in\! T_p M} B_{f(p)}(f_{*p}x, f_{*p}y) = B_p(x, y). \]
  Dann gilt $\ex{\lambda \in \R} B = \lambda \cdot g$.
\end{lem}

% 23.13
\begin{satz}
  Zshgde Isotropie-irred. homogene Räume sind Einsteinsch.
\end{satz}

% Bsp: Einfach zshgde de-Rham-unzerlegbare sym. Räume

% §24. Killing-Felder

\begin{defn}
  Eine \emph{Wirkung} einer eine Lie-Gruppe $G$ auf einer diff'baren Mft $M$ ist ein Gruppenhomomorphismus $\phi : G \to \Diff(M)$, sodass $G \times M \to M, \enspace (g, m) \mapsto \phi(g)(m)$ glatt ist.
\end{defn}

\begin{defn}
  Das \emph{Wirkungsvektorfeld} von $\phi$ zu $x \in \Gie \cong T_e G$ ist
  \[
    X^\phi : M \to TM, \quad
    p \mapsto \tfrac{\d}{\d t}|_{t=0} \phi_{g_x(t)}(p).
  \]
  Dabei ist $g_x : \vinterval \to G$ glatt mit $g_x(0) = e$, $\dot{g_x}(0) = x$.
\end{defn}

% Ausgelassen: Beispiele von Wirkungen

% 24.1
\begin{lem}
  Sei $G$ eine Lie-Gruppe, $X \in \VF(G)$ ein linksinvariantes VF. \\
  Dann ist die Integralkurve $c_e$ eine 1-Parameter-Untergruppe von $G$.
\end{lem}

% 24.2
\begin{lem}
  Sei $X \in \VF(G)$ linksinvariant. Dann ist $c_g = L_g \circ c_e$.
\end{lem}

% 24.3
\begin{lem}
  Jede 1-Param-UG $\phi : \R \to G$ definiert ein linksinv. VF $X \in \VF(G)$, dessen Integralkurve durch $e$ gerade $\phi$ ist: $c_e = \phi$.
\end{lem}

\begin{fazit}
  $\fa{x \in T_e G} \exu{\text{1-Param-UG } \phi_x : \R \to G} \dot{\phi_x}(0) = x$
\end{fazit}

\begin{defn}
  Die \emph{Exponentialabbildung} der Lie-Gruppe $G$ ist
  \[
    \exp : T_e G \cong \Gie \to G, \quad
    x \mapsto \phi_x(1).
  \]
\end{defn}

\begin{bem}
  Wenn $G$ eine bi-inv. Metrik hat, dann ist $\exp = \Exp_e$.
\end{bem}

% Vorlesung vom 28.1.2015

\begin{defn}
  Ein VF $X \in \VF(M)$ heißt \emph{Killing-Feld}, wenn die lokalen Flüsse $\Phi_t$ von $X$ aus lokalen Isometrien bestehen, \dh{}
  \[
    \fa{x \in U} \fa{v, w \in T_x M}
    g_{\Phi_t(x)}(\Phi_{t*} v, \Phi_{t*} w) = g_x(v, w).
  \]
\end{defn}

\begin{nota}
  $\KF(M) \coloneqq \Set{X \in \VF(M)}{X \text{ Killing}}$
\end{nota}

% 24.4
\begin{lem}
  Ein Vektorfeld $X \in \VF(M)$ ist genau dann ein Killing-VF, wenn $\nabla X$ schiefsymmetrisch ist, \dh{}
  \[ \fa{X, Y, Z \in \VF(M)} g(\nabla_Y X, Z) = - g(\nabla_Z X, Y) \]
\end{lem}

\begin{facts}
  Sei $X \in \KF(M)$.
  \begin{itemize}
    \item $\KF(M)$ ist eine Unter-Lie-Algebra von $\VF(M)$.
    \item Für jede Geodäte $\gamma$ ist $X \circ \gamma \in \VF_\gamma$ ein Jacobi-Feld längs $\gamma$.
    \item $\fa{A, B \!\in\! \VF(M)} L_X(A, B) \!\coloneqq\! \nabla_A \nabla_B X - \nabla_{\nabla_A B} X + R(X, A)B = 0$
    \item Ist $(M, g)$ vollständig, dann ist $\Phi_t : M \!\to\! M$ für alle $t \!\in\! \R$ definiert.
  \end{itemize}
\end{facts}

% Killing-Felder auf symmetrischen Räumen

% 24.5
\begin{satz}
  Sei $P$ ein sym. Raum, $G \coloneqq \Iso(P)$, $\Gie \coloneqq \mathcal{L}(G) \cong T_e G$ die Lie-Algebra von $G$. Dann ist die Abbildung
  \[
    \iota : \Gie \to \KF(P), \quad
    x \mapsto (X : p \mapsto \tfrac{\d}{\d t}|_0 \exp(tx).p)
  \]
  ist ein $\R$-VR-Isomorphismus.
\end{satz}

\begin{acht}
  Es gilt $\iota([x, y]_\Gie) = - [\iota(x), \iota(y)]_{\VF(P)}$, es ist $\iota$ also fast (bis auf Vorzeichen) ein Lie-Algebra-Isomorphismus.
\end{acht}

\begin{defn}
  Sei $P$ ein symmetrischer Raum, $p \in P$. Setze
  \begin{align*}
    k_p & \coloneqq \iota^{-1}(\Set{X \in \KF(P)}{X(p) = 0}) \subset \Gie, \\
    p_p & \coloneqq \iota^{-1}(\Set{X \in \KF(P)}{\nabla X(p) = 0}) \subset \Gie.
  \end{align*}
\end{defn}

% 24.6
\begin{lem}
  Sei $P$ ein symmetrischer Raum, $p \in P$. Dann gilt
  \[ \fa{v \in T_p P} \exu{\tilde{v} \in p_p} \fa{s \in \R} \exp(s \tilde{v}) = t^{\gamma_v}(s). \]
\end{lem}

% Vorlesung vom 30.1.2015

% 24.7
\begin{prop}
  $k_p = \mathfrak{G}_p \coloneqq \mathcal{L}(\Iso_p(P)) \cong T_e \Iso_p(P)$
\end{prop}

% 24.8 (ausgelassen, da Teil der nächsten Proposition)
\iffalse
\begin{lem}
  $p_p \cap k_p = \{ 0 \}$
\end{lem}
\fi

% 24.9
\begin{prop}
  $\Gie = p_p \oplus k_p$ (direkte Summe von UVR)
\end{prop}

% 24.10
\begin{prop}[\emph{Cartan-Relationen}]\mbox{}\\
  \inlineitem{$[k_p, k_p]_\Gie \subseteq k_p$} \quad
  \inlineitem{$[k_p, p_p]_\Gie \subseteq p_p$} \quad
  \inlineitem{$[p_p, p_p]_\Gie \subseteq k_p$}
\end{prop}

% 24.11
\begin{prop}
  Sei $\Gie = k \oplus p$ eine Zerlegung einer reellen Lie-Algebra. \\
  Es gelten die Cartan-Relationen genau dann, wenn es eine Involution $\nabla : \Gie \to \Gie$ (\dh{} ein Lie-Algebra-Autom. mit $\nabla^2 = \id$) gibt, sodass $k$ der ER zum EW $+1$ und $p$ der ER zum EW $-1$ von $\nabla$ ist.
\end{prop}

% 24.12
\begin{prop}
  Sei $P$ ein symmetrischer Raum, $p \in P$. Dann ist
  \[
    R_p : p_p \to T_p P, \quad
    x \mapsto \iota(x)(p)
  \]
  ein VR-Isomorphismus und es gilt
  \[ (R(\iota(v), \iota(w)) \iota(u))(p) = \iota([u, [v, w]_\Gie]_\Gie)(p). \]
\end{prop}

\begin{kor}
  Sei $P$ ein symmetrischer Raum. Dann ist
  \[
    R_p(a, b) : T_p P \to T_p P, \quad
    x \mapsto R_p(a, b) x
  \]
  eine Derivation von $R_p$, \dh{} für alle $A, B, X, Y, Z \in \VF(P)$ gilt
  \begin{align*}
    R(A, B)(R(X, Y)Z) = \, & R(R(A, B)X, Y)Z + R(X, R(A, B)Y)Z \\
    & + R(X, Y)(R(A, B)Z).
  \end{align*}
\end{kor}

% §25. Konstruktion symmetrischer Räume



\end{document}
