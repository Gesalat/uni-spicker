\documentclass{cheat-sheet}

\pdfinfo{
  /Title (Zusammenfassung Numerik von partiellen Differentialgleichungen)
  /Author (Tim Baumann)
}

\usepackage{nicefrac}
\usepackage{tikz}
\usepackage{bbm} % Für 1 mit Doppelstrich (Indikatorfunktion)
\usepackage{tikzsymbols} % Smileys!

% Kleinere Klammern
\delimiterfactor=701

\newcommand{\Cont}{\mathcal{C}} % Menge der stetigen/diff'baren Funktionen
\newcommand{\clos}[1]{\overline{#1}} % topologischer Abschluss
\newcommand{\cOmega}{\clos{\Omega}} % weil's so häufig gebraucht wird
\newcommand{\bOmega}{\partial \Omega} % weil's so häufig gebraucht wird
\newcommand{\LL}{\mathcal{L}} % Lösungsoperator
\newcommand{\RR}{\mathcal{R}} % Randoperator
\newcommand{\Laplace}{\Delta}
\DeclareMathOperator{\dive}{div} % Divergenz
\newcommand{\tss}[1]{\textsubscript{#1}} % Subskript, kürzer
\newcommand{\rn}{\text{rn}} % randnah
\newcommand{\rf}{\text{rf}} % randfern
\renewcommand{\O}{\mathcal{O}} % Landau-Symbol
\newcommand{\DO}{\mathcal{D}} % Differentialoperator
\newcommand{\scp}[2]{\left\langle #1 , #2 \right\rangle} % Skalarprodukt
\DeclareMathOperator*{\esssup}{ess\,sup} % Essentielles Supremum
\newcommand{\loc}{\text{loc}} % lokal
\newcommand{\ind}{\mathbbm{1}} % Indikatorfunktion
\newcommand{\Testfun}{\mathcal{D}} % Raum der Testfunktionen

\begin{document}

\maketitle{Zusammenfassung Numerik von PDEs}

% Vorlesung vom 13. Oktober 2015

% 1. Einführung

% Ausgelassen: Notationen

% 1.1. Klassifikation von partiellen DGLn

\begin{defn}
  Sei $\Omega \subseteq \R^n$ offen.
  Eine DGL der Form
  \[ F(x, u, Du, \ldots, D^k u) = 0 \]
  heißt \emph{partielle DGL/PDE} der Ordnung $k \geq 1$, wobei
  \[ F : \Omega \times \R \times \R^n \times \ldots \times \R^{n^k} \to \R \]
  eine gegebene Funktion und $u : \Omega \to \R$ gesucht ist.
\end{defn}

\begin{defn}[\emph{Klassifikation von PDEs}]\mbox{}\\
  \begin{itemize}
    \item Die PDE heißt \emph{linear}, wenn sie die Form
    \[ \sum_{\abs{\alpha} \leq k} a_\alpha(x) D^\alpha u = f(x) \]
    mit Funktionen $a_\alpha, f : \Omega \to \R$ besitzt.
    \item Die PDE heißt \emph{semilinear}, wenn sie die Form
    \[ \sum_{\abs{\alpha} = k} a_\alpha(x) D^\alpha u + a_0(x, u, D_u, \ldots, D^{k-1} u) = 0 \]
    besitzt, wobei $a_\alpha : \Omega \to \R$ und $a_0 : \Omega \times \R \times \R^n \times \ldots \times \R^{n^k} \to \R$ gegeben sind.
    \item Die PDE heißt \emph{quasilinear}, wenn sie die Form
    \[ \sum_{\abs{\alpha} = k} a_\alpha(x, u, Du, \ldots, D^{k-1} u) D^\alpha u + a_0(x, u, D_u, \ldots, D^{k-1} u) = 0 \]
    hat, wobei $a_\alpha, a_0 : \Omega \times \R \times \R^n \times \ldots \times \R^{n^k}$ gegeben sind.
    \item Die PDE heißt \emph{nichtlinear}, falls die Ableitungen der höchsten Ordnung nicht linear vorkommen.
  \end{itemize}
\end{defn}

% Ausgelassen: Beispiele: Poisson-, Laplace-, Wärmeleitungs-, Wellengleichung sowie Navier-Stokes-Gleichung

\begin{defn}
  Sei $\Omega \subseteq \R^n$ offen und $F : \Omega \times \R \times \R^n \times \R^{n \times n} \to \R$ eine gegebene Funktion.
  Eine PDE der Form
  \[ F(x, u, \partial_{x_1} u, \ldots, \partial_{x_n} u, \partial_{x_1} \partial_{x_1} u, \ldots, \partial{x_1} \partial_{x_n} u, \ldots, \partial_{x_n} \partial_{x_n} u) = 0 \]
  heißt \emph{PDE zweiter Ordnung}.
\end{defn}

\begin{nota}
  $p_i \coloneqq \partial_{x_i} u$, $p_{ij} \coloneqq \partial^2_{x_i x_j} u$
  \[
    M(x) \coloneqq \begin{pmatrix}
      \tfrac{\partial F}{\partial p_{11}} & \hdots & \tfrac{\partial F}{\partial p_{1n}} \\
      \vdots && \vdots \\
      \tfrac{\partial F}{\partial p_{n1}} & \hdots & \tfrac{\partial F}{\partial p_{nn}}
    \end{pmatrix} = M(x)^{T}.
  \]
\end{nota}

\begin{defn}[\emph{Typeneinteilung für PDEs der 2. Ordnung}]\mbox{}\\
  Obige PDE zweiter Ordnung heißt
  \begin{itemize}
    \item \emph{elliptisch} in $x$, falls die Matrix $M(x)$ positiv o. definit ist.
    \item \emph{parabolisch} in $x$, falls genau ein EW von $M(x)$ gleich null ist und alle anderen dasselbe Vorzeichen haben.
    \item \emph{hyperbolisch} in $x$, falls genau ein EW ein anderes Vorzeichen als die anderen EWe hat.
  \end{itemize}
\end{defn}

% Ausgelassen: Beispiele

% Vorlesung vom 19. Oktober 2015

% 2. Klassische Lösungstheorie für elliptische PDEs
\section{Lösungstheorie elliptischer PDEs}

\begin{defn}
  Sei $\Omega \subset \R^n$ offen, zusammenhängend und beschränkt.
  \begin{itemize}
    \item $\Cont(\cOmega, \R^m) \coloneqq \Set{u : \cOmega \to \R^m}{u \text{ stetig}}$, $\Cont(\cOmega) \coloneqq \Cont(\cOmega, \R)$, mit Norm
    \[
      \norm{u}_{\Cont(\cOmega, \R^m)} = \sup_{x \in \cOmega} \norm{u(x)}.
      \tag{Supremumsnorm}
    \]
    \item $\Cont^k(\cOmega, \R^m)$, $k \in \N$ ist der Raum aller auf $\Omega$ $k$-mal stetig diff'baren Funktionen $u : \Omega \to \R^m$, die zusammen mit ihren Ableitungen bis zur Ordnung $k$ stetig auf $\cOmega$ fortgesetzt werden können.
    \[ \norm{u}_{\Cont^k(\overline{\Omega}, \R^m)} = \sum_{\abs{\alpha} \leq k} \norm{D^\alpha u}_{\Cont(\overline{\Omega}, \R^m)} \]
    \item Sei $\alpha \in \cointerval{0}{1}$. $\Cont^{0,\alpha}(\cOmega, \R^m) = \Set{u \in \Cont(\cOmega, \R^m)}{H_\alpha(u, \cOmega) < \infty}$ mit
    \[
      H_\alpha(u, \cOmega) \coloneqq \sup_{x, y \in \cOmega, x \neq y} \tfrac{\norm{u(x)-u(y)}}{\norm{x-y}^\alpha}
      \tag{Hölder-Koeffizient}
    \]
    heißt \emph{Raum der glm. Hölder-stetigen Fktn} zum Exponent $\alpha$.
    Der Hölder-Koeffizient ist dabei eine Seminorm auf $\Cont^{0,\alpha}(\cOmega, \R^m)$.
    \item $\Cont^{k,\alpha}(\cOmega, \R^m) \coloneqq \Set{u \in \Cont^k(\cOmega, \R^m)}{\fa{\abs{\gamma} = k} D^\gamma u \in \Cont^{0, \alpha}(\cOmega, \R^m)}$ heißt \emph{Hölder-Raum}.
    Eine Norm ist gegeben durch
    \[ \norm{u}_{\Cont^{k,\alpha}(\cOmega, \R^m)} \coloneqq \norm{u}_{\Cont^k(\cOmega, \R^m)} + \sum_{\abs{\gamma} = k} H_\alpha(D^\gamma u, \cOmega). \]
  \end{itemize}
\end{defn}

\begin{bem}
  \begin{itemize}
    \item Jede Hölder-stetige Funktion ist gleichmäßig stetig.
    \item $\Cont^{0,1}(\cOmega, \R^m)$ heißt \emph{Raum der Lipschitz-stetigen Funktionen}.
    \item $\Cont$, $\Cont^k$ und $\Cont^{k,\alpha}$ sind Banach-Räume mit den jeweiligen Normen.
  \end{itemize}
\end{bem}

\iffalse
\begin{bsp}
  Betrachte $u(x) = \abs{x}^\beta$ auf $\ointerval{-1}{1} = \Omega$. Dann ist
  \[ \tfrac{\abs{u(x) - u(0)}}{\abs{x - 0}^\alpha} = \abs{X}^{\beta - \alpha} \]
  genau dann beschränkt, falls $\beta \geq \alpha$.
  In diesem Fall ist $u$ Hölder-stetig zum Exponent $\alpha$.
\end{bsp}
\fi

\begin{defn}
  Sei $\Omega \subset \R^n$ offen, zusammenhängend und beschränkt. \\
  Das Gebiet $\Omega$ gehört zur \emph{Klasse $\Cont^{k,\alpha}$}, wenn in jedem Punkt $x \in \bOmega$ eine Umgebung in $\bOmega$ existiert, die sich in einem geeigneten Koordinatensystem als ein Graph einer Funktion aus $\Cont^{k,\alpha}$ darstellen lässt und $\Omega$ lokal immer auf einer Seite von $\bOmega$ liegt.
\end{defn}

% Ausgelassen: Beispiele und Gegenbeispiele von Gebieten aus $\Cont^{k,\alpha}$

\begin{satz}[\emph{Gauß'scher Integralsatz}]
  Sei $\Omega \subset \R^n$ ein Lipschitz-Gebiet und $u \in \Cont(\cOmega, \R^n) \cap \Cont^1(\Omega, \R^n)$.
  Dann gilt
  \[ \Int{\Omega}{}{\dive u}{x} = \Int{\Omega}{}{\sum_{i=1}^n \tfrac{\partial u_i}{\partial x_i}}{x} = \Int{\bOmega}{}{\sum_{i=1}^n u_i \nu_i}{\rho(x)} = \Int{\bOmega}{}{u \cdot \nu}{\rho(x)}, \]
  wobei $\nu$ der äußere Normalenvektor an an den Rand von $\Omega$ ist.
\end{satz}

\begin{prob}
  Wir betrachten das Randwertproblem
  \[
    \text{(RWP)} \enspace \left\{ \begin{array}{rllll}
      \LL u &=& f &\text{ in $\Omega$} & \text{(PDE)} \\
      \RR u &=& g &\text{ auf $\bOmega$} & \text{(Randbedingung)}
    \end{array} \right.
  \]
  wobei $\LL$ der lineare Differentialoperator
  \[
    \LL u = - \sum_{i,j=1}^n a_{ij}(x) \tfrac{\partial^2 u}{\partial x_i \partial x_j} + \sum_{i=1}^n b_i(x) \tfrac{\partial u}{\partial x_i} + c(x) u
  \]
  mit Fktn $a_{ij}, b_i, c, f : \Omega \to \R$, $g : \bOmega \to \R$ ist, sodass $A(x) \coloneqq (a_{ij}(x))$ symmetrisch ist.
  Als Randbedingung (RB) verlangen wir:
  \[ \begin{array}{rllll}
    \text{\emph{Dirichlet-RB}:} & u &=& g & \text{auf $\bOmega$,} \\
    \text{\emph{Neumann-RB}:} & (A(x) \nabla u) \cdot \nu &=& g & \text{auf $\bOmega$ oder} \\
    \text{\emph{Robin-RB}:} & (A(x) \nabla u) \cdot \nu + \delta u &=& g & \text{auf $\bOmega$.}
  \end{array} \]
\end{prob}

\begin{bem}
  Man kann auch auf verschiedenen Teilstücken des Randes verschiedene Bedingungen stellen.
\end{bem}

\begin{bem}
  Falls die Funktionen $a_{ij}$ differenzierbar sind, so kann $\LL$ in \emph{Divergenzform} geschrieben werden:
  \begin{align*}
    %\LL u & = - \sum_{i,j=1}^n a_{ij}(x) \tfrac{\partial^2 u}{\partial x_i \partial x_j} + \sum_{i=1}^n b_i(x) \tfrac{\partial u}{\partial x_i} + c(x) u  \\
    %& = - \sum_{i,j=1}^n \left( \tfrac{\partial}{\partial x_j} \left( a_{ij}(x) \tfrac{\partial u}{\partial x_i} \right) - \left( \tfrac{\partial}{\partial x_j} a_{ij}(x) \right) \tfrac{\partial u}{\partial x_i} \right) + \sum_{i=1}^n b_i(x) \tfrac{\partial u}{\partial x_i} + c(x) u \\
    \LL u & = - \!\!\! \sum_{i,j=1}^n \tfrac{\partial}{\partial x_j} \left( a_{ij}(x) \tfrac{\partial u}{\partial x_i} \right) \!+\! \sum_{i=1}^n \underbrace{\left((\sum_{j=1}^n \tfrac{\partial}{\partial x_j} a_{ij}(x) ) \!+\! b_i(x)\right)}_{\tilde{b}(x) \coloneqq} \tfrac{\partial u}{\partial x_i} \!+\! c(x) u \\
    & = - \dive(A(x) \nabla u) + \tilde{b}(x) \cdot \nabla u + c(x) u
  \end{align*}
\end{bem}
% TODO: \tilde{b}_i(x) ist der Term in Klammern im mittleren Term

\begin{voraussetzung}
  Wir nehmen im Folgenden an:
  \begin{itemize}
    \item $\LL$ ist \emph{gleichmäßig elliptisch}, \dh{}
    \[ \ex{\lambda_0 > 0} \fa{\xi \in \R^n} \fa{x \in \Omega} \xi^T A(x) \xi \geq \lambda_0 \norm{\xi}^2 \]
    Dabei heißt $\lambda_0$ \emph{Elliptizitätskonstante}.
    \item $a_{ij}, b_i, c, f \in \Cont(\cOmega)$, $g \in \Cont(\bOmega)$
  \end{itemize}
\end{voraussetzung}

\begin{bem}
  $\LL = f$ ist elliptisch auf $\Omega$ $\iff$ $A(x) > 0$ (spd) für alle $x \in \Omega$
\end{bem}

\begin{defn}
  Eine Fkt $u \in \Cont^2(\Omega) \cap \Cont(\cOmega)$ heißt \emph{klassische Lsg} vom (RWP) mit $\RR u \coloneqq u$, wenn die beiden Gleichungen in (RWP) in jedem Punkt von $\Omega$ bzw. des Randes $\bOmega$ erfüllt sind.
\end{defn}

% 2.1
\begin{satz}[\emph{Maximumsprinzip}]
  Sei $\Omega \subset \R^n$ offen, zshgd u. beschränkt.
  Sei $u \in \Cont^2(\omega) \cap \Cont(\cOmega)$ eine Lösung vom (RWP), $f \leq 0$ in $\Omega$ und $c \equiv 0$. \\
  Dann nimmt $u$ sein Maximum auf dem Rand $\bOmega$ an, \dh{}
  \[ \sup_{x \in \cOmega} u(x) = \sup_{x \in \bOmega} u(x) = \sup_{x \in \bOmega} g(x) \]
\end{satz}

% Vorlesung vom 20.10.2015

% 2.2
\begin{kor}
  Sei $c \geq 0$ und $f \leq 0$.
  Dann gilt
  $\sup_{x \in \cOmega} u(x) \leq \max \{ \sup_{x \in \bOmega} u(x), 0 \}$.
\end{kor}

% 2.3
\begin{kor}[\emph{Vergleichsprinzip}]
  Für $u_1, u_2 \in \Cont^2(\Omega) \cap \Cont(\cOmega)$ und $c \geq 0$ gelte $\LL u_1 \leq \LL u_2$ in $\Omega$ und $u_1 \leq u_2$ auf $\bOmega$.
  Dann gilt $u_1 \leq u_2$ auf $\cOmega$.
\end{kor}

% 2.4
\begin{kor}[Eindeutigkeit]
  Sei $c \geq 0$. Dann hat (RWP) höchstens eine Lösung $u \in \Cont^2(\Omega) \cap \Cont(\cOmega)$.
\end{kor}

\iffalse
\begin{bsp}
  Betrachte $-u'' - \lambda u = 0$ in $\Omega = \ointerval{0}{1}$ mit $\lambda > 0$, $u(0) = u(1) = 0$.
  \begin{itemize}
    \item $u \equiv 0$ ist eine Lösung
    \item Für $\lambda = k^2 \pi^2$ ist $u(x) = a \sin(k \pi x)$ auch eine Lösung
  \end{itemize}
\end{bsp}
\fi

% 2.5
\begin{satz}
  Sei $\Omega$ ein beschr. Lipschitz-Gebiet, $a_{ij}, b_i, c, f \in \Cont(\cOmega)$, $c \geq 0$, $g \in \Cont(\bOmega)$.
  Dann besitzt (RWP) genau eine Lsg $u \in \Cont^2(\Omega) \cap \Cont(\cOmega)$.
\end{satz}

% Beweis siehe J.H.Michael, "A general theory for linear elliptic partial differential equations", 1977

\begin{acht}
  Es muss aber nicht $u \in \Cont^2(\cOmega)$ gelten!
\end{acht}

\iffalse
\begin{bspe}
  \begin{itemize}
    \item $- \Laplace u = 0$ in $\ointerval{0}{1} \times \ointerval{0}{1}$, $u(0, x_2) = 0$, $u(1, x_2) = x_2$, $u(x_1, 0) = 0$, $u(x_1, 1) = x_1$ für $x_1, x_2 \in \cinterval{0}{1}$.
    Lösung: $u(x_1, x_2) = x_1 x_2$
    \item $- \Laplace u = 0$ in $\Omega = \ointerval{0}{1} \times \ointerval{0}{1}$,
    $u(x_1, x_2) = x_1^2$.
    Nach Satz 2.5 existiert eine Lösung $u \in \Cont^2(\Omega) \cap \Cont(\cOmega)$ aber $u \not\in \Cont^2(\cOmega)$, denn
    \[ \tfrac{\partial^2 u}{\partial x_1^2} + \tfrac{\partial^2 u}{\partial x_2^2} = 2 \neq 0 \]
    bei $x = (1, 1)$, auf $\bOmega$.
  \end{itemize}
\end{bspe}
\fi

% 3. Differenzenverfahren
\section{Differenzenverfahren}

% 3.1. Differenzenverfahren für die Poisson-Gleichung in $\Omega = \ointerval{0}{1}$

\begin{verf}[\emph{DV}]
  Am Beispiel des Poisson-Problems
  \[
    \text{(RWP\tss{1})}\enspace \left\{ \begin{array}{rl}
      - \Laplace u = f &\text{ in $\Omega = \ointerval{0}{1}$} \\
      u(0) = g_0, u(1) = g_1 &\text{ auf $\bOmega$}
    \end{array} \right.
  \]
  Wir führen folgende Schritte durch:
  \begin{enumerate}
    \item Diskretisierung: Wähle $n \in \N$, setze $h \coloneqq \tfrac{1}{n}$ und
    \begin{align*}
      \Omega_h &\coloneqq \Set{x_i \coloneqq ih}{i = 1, \ldots, n-1}
      \tag{innere Gitterpunkte} \\
      \partial \Omega_h &\coloneqq \{ x_0 = 0, x_n = 1 \}
      \tag{Randpunkte}
    \end{align*}
    \item Approx. der Ableitungen durch Differenzenquotienten (DQ)
    \begin{align*}
      u'(x_i) &\approx \tfrac{1}{h} \left(u(x_i + h) - u(x_i)\right)
      \tag{\emph{Vorwärts-DQ}} \\
      u'(x_i) &\approx \tfrac{1}{h} \left(u(x_i) - u(x_i - h)\right)
      \tag{\emph{Rückwärts-DQ}} \\
      u'(x_i) &\approx \tfrac{1}{2h} \left(u(x_i + h) - u(x_i - h)\right)
      \tag{\emph{zentraler DQ}}
    \end{align*}
    Für die zweite Ableitung ergibt sich
    \begin{align*}
      u''(x_i) &= (u'(x_i))' \approx \tfrac{1}{h} \left(u'(x_i + h) - u'(x_i)\right) \approx \\
      &\approx \tfrac{1}{h} \cdot \left(\tfrac{1}{h} \left( u(x_i + h) - u(x_i) \right) - \tfrac{1}{h} \left( u(x_i) - u(x_i - h) \right)\right) \\
      &= \tfrac{1}{h^2} \left( u(x_i + h) - 2 \cdot u(x_i) + u(x_i - h) \right) =: \Laplace_h u
    \end{align*}
    Dabei heißt $\Laplace_h$ der diskrete eindim. Laplace-Operator. \\
    Das diskretisierte Randwertproblem ist nun
    % TODO: Bezeichnung davon?
    \[
      \text{(RWP\tss{1})\tss{h}} \enspace
      \left\{ \begin{array}{rl}
        - \Laplace_h u_h = f &\text{ in $\Omega_h$,} \\
        u_h(0) = g_0, u_h(1) = g_1 &\text{ auf $\partial \Omega_h$.}
      \end{array} \right.
    \]
    \item Aufstellen des linearen Gleichungssystems
    \begin{align*}
      \tfrac{1}{h^2} \left( 2 u_h(x_1) - u_h(x_2) \right) &= f(x_1) + \tfrac{g_0}{h^2}
      \tag{$i\!=\!1$} \\
      \tfrac{1}{h^2} \left( -u_h(x_{i-1}) + u_h(x_i) - u_h(x_{i+1}) \right) &= f(x_i)
      \tag{$i=2, \nldots, n-2$} \\
      \tfrac{1}{h^2} \left( -u_h(x_{n-2}) + 2 u_h(x_{n-1}) \right) &= f(x_{n-1}) + \tfrac{g_1}{h^2}
      \tag{$i\!=\!n{-}1$}
    \end{align*}
    
    Als lineares Gleichungssystem: $- \tilde{\Laplace}_h \tilde{u}_h = \tilde{f}_h$ mit
    \[
      - \tilde{\Laplace}_h = \frac{1}{h^2} \begin{pmatrix}
        2 & -1 & &&& 0 \\
        -1 & 2 & -1 &&& \\
        & -1 & 2 & -1 && \\
        && \ddots & \ddots & \ddots \\
        &&& -1 & 2 & -1 \\
        0 &&&& -1 & 2
      \end{pmatrix} \in \R^{(n-1) \times (n-1)},
    \]
    \[
      \tilde{u}_h = \begin{pmatrix}
        u_h(x_1) \\ \vdots \\ u_h(x_{n-1})
      \end{pmatrix}, \quad
      \tilde{f}_h = \begin{pmatrix}
        f(x_1) + \tfrac{g_0}{h^2} \\
        f(x_2) \\
        \vdots \\
        f(x_{n-2}) \\
        f(x_{n-1}) + \tfrac{g_1}{h^2}
      \end{pmatrix}
    \]
  \end{enumerate}
\end{verf}

% Vorlesung vom 26.10.2015

% Konvergenz, Konsistenz, Stabilität

\begin{ziel}
  Herausfinden, was die Lösung~$u_h$ von (RWP)\tss{h} (die man durch Lösen von (LGS) erhält) mit der Lösung~$u$ zum ursprünglichen Problem (RWP) zu tun hat.
  Ist etwa~$u_h$ die Einschränkung von~$u$, oder zumindest annäherungsweise?
  Wenn ja, wie klein muss man~$h$ wählen, damit die Approximation gut wird?
  \begin{align*}
    \text{(RWP)} \enspace & \left\{ \begin{array}{rl}
      - \LL u = f &\text{ in $\Omega$,} \\
      u = g &\text{ auf $\partial \Omega$}
    \end{array} \right. \\
    \text{(RWP)\tss{h}} \enspace & \left\{ \begin{array}{rl}
      - \LL_h u = f_h &\text{ in $\Omega_h$,} \\
      u_h = g_h &\text{ auf $\partial \Omega_h$}
    \end{array} \right. \\
    \text{(LGS)} \enspace & \tilde{\LL}_h \tilde{u}_h = \tilde{f}_h
  \end{align*}
\end{ziel}

\begin{nota}
  %Sei $U_h$ der Raum aller Funktionen mit Werten in $\R$, die auf dem Gitter $\Omega_h$ definiert sind und sei  die Einschränkung stetiger Funktionen auf $\Omega_h$. \\
  $U_h \coloneqq \{ \Omega_h \to \R \}$, \quad
  $R_h : \Cont(\cOmega) \to U_h, \enspace u \mapsto u|_{\Omega_h}$
\end{nota}

\begin{defn}
  Das Differenzenverfahren (RWP)\tss{h} heißt
  \begin{itemize}
    \item \emph{konvergent} von der Ordnung~$p$, falls $C > 0$, $h_0 > 0$ existieren, sodass für die Lösung~$u$ von (RWP) und die Lösung~$u_h$ von (RWP)\tss{h} gilt:
    \[
      \norm{u_h - R_h u}_h \leq C h^p \quad
      \text{für alle $0 < h \leq h_0$,}
    \]
    wobei~$\norm{\blank}_h$ eine Norm zu~$U_h$ ist, wie \zB{} $\norm{u_h}_h \coloneqq \max_{x \in \Omega_h} \abs{u_h(x)}$.
    \item \emph{konsistent} von der Ordnung~$p$, falls
    \[
      \norm{\LL_h R_h u - R_h \LL u}_h \leq c h^p \norm{u}_{\Cont^{p+2}(\cOmega)} \quad
      \forall u \in \Cont^{p+2}(\cOmega).
    \]
    % Falls u eine Lösung von (RWP) ist, so gilt
    %\[ \LL_h R_h u - R_h \LL u = \LL_h R_h u - R_h f = \LL_h u - f_h \]
    \item \emph{stabil}, falls $\tilde{L}_h$ invertierbar ist und ein $h_0 > 0$ existiert mit
    \[
      \sup_{0 < h \leq h_0} \norm{\tilde{\LL}_h^{-1}}_h < \infty, \quad
      \text{wobei} \enspace
      \norm{\tilde{\LL}_h^{-1}}_h \coloneqq \sup_{f \neq 0} \tfrac{\norm{\tilde{\LL}^{-1}_h f}_h}{\norm{f}_h}.
    \]
  \end{itemize}
\end{defn}

\begin{bem}
  Die ind. Matrixnorm ist $\norm{\tilde{\LL}_h^{-1}}_h = \norm{\tilde{\LL}_h^{-1}}_\infty = \max_{1 \leq i \leq n} \sum_{j=1}^n \abs{l_{ij}}$.
\end{bem}

% Was ist das Residuum??? Was heißt Konditionierung eines Problems?

% 3.1
\begin{satz}
  Ist das DV (RWP)\tss{h} konsistent und stabil, so auch konvergent.
  Genauer gilt: Ist (RWP)\tss{h} stabil und konsistent von der Ordnung~$p$ und $u \in \Cont^{p+2}(\cOmega)$, dann ist (RWP)\tss{h} konvergent von der Ordnung~$p$.
\end{satz}

% Was ist Spalten-, Spektrale und Zeilensummennorm?
% Spa-Spe-Zei

\begin{proof}
  Setze $w_h \coloneqq u_h - R_h u$.
  Für $x \in \partial \Omega_h$ gilt dann $w_h(x) = 0$ und für $x \in \Omega_h$ gilt
  \begin{align*}
    \tilde{\LL}_h w_h(x) &= \LL_h w_h(x) = \LL_h u_h(x) - \LL_h R_h u(x) \\
    &= f_h(x) - \LL_h R_h u(x) = R_h f(x) - \LL_h R_h u(x) \\
    &= R_h \LL u(x) - \LL_h R_h u(x)
  \end{align*}
  Somit gilt $w_h = \tilde{\LL}_h^{-1} \left( R_h \LL u - \LL_h R_h u \right)$ in $\Omega_h$, also
  \begin{align*}
    \norm{w_h}_h &= \norm{\tilde{\LL}_h^{-1} \left( R_h \LL u - \LL_h R_h u \right)}
    \leq \norm{\tilde{\LL}_h^{-1}}_h \cdot \norm{R_h \LL u - \LL_h R_h u}_h \\
    & \leq c_1 \cdot c_2 \cdot h^p \cdot \norm{u}_{\Cont^{p+2}(\cOmega)}
    \leq C h^p \qquad
    \text{für $0 < h \leq h_0$.} \qedhere
  \end{align*}
\end{proof}

% 3.2
\begin{lem}
  Das DV (RWP\tss{1})\tss{h} ist konsistent von der Ordnung 2.
  Es gilt
  \[
    \norm{\Laplace_h R_h u - R_h \Laplace u}_h \leq \tfrac{1}{12} \norm{u}_{\Cont^4(\cOmega)} h^2 \quad
    \forall u \in \Cont^4(\cOmega).
  \]
\end{lem}

% Beweis ähnlich wie auf ÜB2 mit Taylorn

\begin{bem}
  Um zu zeigen, dass (RWP\tss{1})\tss{h} konvergent ist, müssen wir noch zeigen, dass $\tilde{L}_h = - \tilde{\Laplace}_h$ invertierbar ist und $\sup_{0 < h \leq h_0} \norm{\tilde{\Laplace}_h} < \infty$.
\end{bem}

\begin{defn}
  Eine Matrix $A = (a_{ij}) \in \R^{n \times n}$ heißt \emph{M-Matrix}, falls
  \begin{enumerate}[label=\alph*),leftmargin=1.6em]
    \miniitem{0.43 \linewidth}{$a_{ii} > 0$ für $i = 1, \ldots, n$,}
    \miniitem{0.53 \linewidth}{$a_{ij} \leq 0$ für $i \neq j$, $i, j = 1, \ldots, n$,}
    \miniitem{0.43 \linewidth}{$A$ invertierbar ist und}
    \miniitem{0.53 \linewidth}{für $A^{-1} =: B = (b_{ij})$ gilt $b_{ij} \geq 0$.}
  \end{enumerate}
\end{defn}

% 4.3.9 im Hackbusch
\begin{lem}
  Erfülle $A \in \R^{n \times n}$ die Bedingungen a) und b).
  Zerlege $A = D + L + R$ in eine Diagonalmatrix und strikte untere/obere Dreiecksmatrizen.
  Dann ist $A$ genau dann eine M-Matrix wenn
  \[ \rho(D^{-1}(L + R)) < 1. \]
\end{lem}

\begin{bem}
  Es gilt folgende Monotonie-Eigenschaft für M-Matrizen:
  \[
    x \leq y \implies
    A^{-1} x \leq A^{-1} y.
  \]
  % je komponentenweise
\end{bem}

\begin{defn}
  Eine Matrix $A \in \R^{n \times n}$ heißt \emph{reduzibel} (oder zerlegbar), wenn es eine Permutationsmatrix $P \in \R^{n \times n}$ gibt, sodass
  \[
    PAP^T = \begin{psmallmatrix}
      A_{11} & A_{12} \\
      0 & A_{22}
    \end{psmallmatrix} \quad
    \text{mit $A_{11} \in R^{k \times k}$, $0 < k < n$.}
  \]
  % Ist $A$ nicht reduzibel, so heißt $A$ irreduzibel
\end{defn}

% Lem 4.3.4 im Hackbusch
\begin{lem}[Gerschgorin]
  Alle EWe einer Matrix $A = (a_{ij}) \in \C^{n \times n}$ liegen in der Menge
  \[
    \bigcup_{i=1}^n \clos{B_{r_i}(a_{ii})}
    \quad \text{mit} \quad
    r_i \coloneqq \sum_{j=1}^n \abs{a_{ij}}.
  \]
  Falls $A$ irreduzibel ist, so liegen sie sogar in
  \[
    \left( \bigcup_{i=1}^n B_{r_i}(a_{ii}) \right) \cup
    \left( \bigcap_{i=1}^n \partial B_{r_i}(a_{ii}) \right)
  \]
\end{lem}

\begin{defn}
  Sei $A = (a_{ij}) \in \R^{n \times n}$ eine Matrix.
  \begin{itemize}
    \item $A$ heißt \emph{schwach diagonaldominant}, falls
    \[
      \sum_{\substack{j=1\\j \neq i}}^n \abs{a_{ij}} \leq \abs{a_{ii}} \quad
      \text{für $i = 1, \ldots, n$}
    \]
    und ein $i_0$ existiert, sodass die Ungleichung strikt ist.
    \item $A$ heißt \emph{diagonaldominant}, falls
    \[
      \sum_{\substack{j=1\\j \neq i}}^n \abs{a_{ij}} < \abs{a_{ii}} \quad
      \text{für $i = 1, \ldots, n$}
    \]
    % Ausgelassen: Beispiel einer reduziblen Matrix
    % Vorlesung vom 27.10.2015
    \item $A$ heißt \emph{irreduzibel diagonaldominant}, falls $A$ irreduzibel und schwach diagonaldominant ist.
  \end{itemize}
\end{defn}

% 3.3
\begin{lem}
  Sei $A = (a_{ij}) \in \R^{n \times n}$ eine Matrix mit $a_{ii} > 0$, $i = 1, \ldots, n$ und $a_{ij} \leq 0$, $i, j = 1, \ldots, n$, $i \neq j$, die diagonaldominant oder irreduzibel diagonaldominant ist.
  Dann ist $A$ eine M-Matrix.
\end{lem}

\begin{bem}
  $- \tilde{\Laplace}_h$ ist irreduzibel diagonaldominant, also eine M-Matrix.
\end{bem}

% 4.3.11 im Hackbusch
\begin{lem}
  Sei $A$ eine irreduzible M-Matrix. Dann gilt $A^{-1} > 0$.
\end{lem}

% Noch zu zeigen: $\sup_{0 < h \leq h_0} \norm{\tilde{\Laplace}_h^{-1}} < \infty$

% 3.4
\begin{lem}
  Sei $A \in \R^{n \times n}$ eine M-Matrix und es existiere ein Vektor $v$, sodass $(Av)_j \geq 1$, $j = 1, \ldots, n$.
  Dann gilt $\norm{A^{-1}}_\infty \leq \norm{v}_\infty$.
\end{lem}

% 3.5
\begin{lem}
  $\norm{\tilde{\Laplace}_h^{-1}}_\infty \leq \tfrac{1}{8}$
\end{lem}

% Beweis folgt aus dem letzten Lemma mit $v = (v_1, \ldots, v_n)^T$ mit $v_i \coloneqq h^2/2 i (n-i)$

% 3.6
\begin{satz}
  Das DV (RWP\tss{1})\tss{h} ist konvergent von der Ordnung 2, falls die Lösung von (RWP\tss{1}) zu $\Cont^4(\cinterval{0}{1})$ gehört.
  Es gilt die Abschätzung
  \[ \norm{u_h - R_h u}_\infty \leq \tfrac{h^2}{96} \norm{u}_{\Cont^4(\cinterval{0}{1})}. \]
\end{satz}

% 3.2. Differenzenverfahren für die Poisson-Gleichung in $\Omega = \ointerval{0}{1} \times \ointerval{0}{1}$

\begin{problem}
  Wir betrachten nun
  \[
    \text{(RWP\tss{2})} \left\{ \begin{array}{rlll}
      - \Laplace u &=& f &\text{ in $\Omega = \ointerval{0}{1} \times \ointerval{0}{1}$} \\
      u &=& g &\text{ auf $\bOmega$}
    \end{array} \right.
  \]
\end{problem}

\begin{enumerate}
  \item Diskretisierung: Setze $h \coloneqq \tfrac{1}{n}$, $n \in \N$ und
  \begin{align*}
    \Omega_h & \coloneqq \Set{(x, y) \in \Omega}{x = ih, y = jh, i,j = 1, \ldots, {n-1}} \\
    \partial \Omega_h & \coloneqq \Set{(x, y) \in \bOmega}{x = ih, y = jh, i,j = 1, \ldots, {n-1}}
  \end{align*}
  \item Approximation der Ableitungen
  \begin{align*}
  - & \Laplace u (x, y)
  = - \tfrac{\partial^2 u}{\partial x^2} (x,y) - \tfrac{\partial^2 u}{\partial y^2} (x, y) \\
  &\approx - \tfrac{u(x + h, y) - 2 u(x, y) + u(x-h, y)}{h^2} - \tfrac{u(x, y+h) - 2 u(x,y) + u(x, y-h)}{h^2} \\
  &= - \tfrac{u(x+h,y) + u(x-h,y) - 4 u(x, y) + u(x, y+h) + u(x, y-h)}{h^2} =: - \Laplace_h u
  \end{align*}
  Dabei hat der diskrete Laplace-Operator $\Laplace_h$ die Form eines Differenzensterns.
  Gesucht ist die Lsg $u_h : \Omega_h \cup \partial \Omega_h \to \R$ von
  \[
    \text{(RWP\tss{2})\tss{h}} \left\{ \begin{array}{rlll}
      - \Laplace_h u_h &=& f_h &\text{ in $\Omega_h$} \\
      u_h &=& g &\text{ auf $\bOmega_h$.}
    \end{array} \right.
  \]
  \item Aufstellen des linearen Gleichungssystems $- \tilde{\Laplace}_h \tilde{u}_h = f_h$:
  % wir verwenden hier die zeilenweise Nummerierung
  \[
    \tilde{u}_h = \begin{pmatrix}
      u_{11} \\
      u_{12} \\
      \vdots \\
      u_{n-1,n-2} \\
      u_{n-1,n-1}
    \end{pmatrix} \in \R^{(n-1)^2},
  \]
  \begin{align*}
    - \tilde{\Laplace}_h = \frac{1}{h^2} & \begin{pmatrix}
      A & -I & && 0 \\
      -I & A & -I && \\
      & \ddots & \ddots & \ddots \\
      && -I & A & -I \\
      0 &&& -I & A
    \end{pmatrix} \in \R^{(n-1)^2 \times (n-1)^2}, \\
    A = & \begin{pmatrix}
      4 & -1 & && 0 \\
      -1 & 4 & -1 && \\
      & \ddots & \ddots & \ddots \\
      && -1 & 4 & -1 \\
      0 &&& -1 & 4
    \end{pmatrix} \in \R^{n-1 \times n-1}
  \end{align*}
\end{enumerate}

% Vorlesung vom 2.11.2015

% 3.7
\begin{lem}
  Das DV (RWP\tss{2})\tss{h} ist konsistent von der Ordnung 2.
  Es gilt
  \[ \norm{\Laplace_h R_h u - R_h \Laplace u}_h \leq \tfrac{1}{6} \norm{u}_{\Cont^r(\cOmega)} h^2. \]
\end{lem}

% 3.8
\begin{lem}
  Das DV (RWP\tss{2})\tss{h} ist stabil.
  Es gilt $\norm{\tilde{D}_h^{-1}}_\infty \leq \nicefrac{1}{8}$.
\end{lem}

% Begründung: $-\tilde{\Laplace}_h$ ist eine M-Matrix, da schwach diagonaldominant und irreduzibel

% 3.9
\begin{satz}
  Das DV (RWP\tss{2})\tss{h} ist konvergent von der Ordnung 2, falls die Lösung von (RWP\tss{2}) zu $\Cont^4(\cOmega)$ gehört.
  Es gilt
  \[ \norm{u_h - R_h u}_h \leq \nicefrac{1}{48} \norm{u} \]
\end{satz}

\begin{bem}
  Durch die Einbeziehung weiterer Gitterpunkte zur Approximation des Differentialoperators lässt sich die Konvergenzordnung erhöhen:
  % Dies geht mit dem 9-Punkte-Stern
  %             1
  %             |
  %            -16
  %             |
  % 1 -- -16 -- 60 -- -16 -- 1
  %             |
  %            -16
  %             |
  %             1
  \begin{align*}
    - \Laplace_h^{(9)} & u(x, y) = \tfrac{1}{12 h^2} \left( u(x{-}2h, y) - 16 u(x{-}h, y) + 30 u(x, y) \right. \\
    & \left. - 16 u(x{+}h, y) + u(x{+}2h, y) + u(x, y{-}2h) - 16 u(x, y{-}h) \right. \\
    & \left. + 30 u(x, y) - 16 u(x, y{+}h) + u(x, y{+}2h) \right) \approx - \Laplace u(x, y)
  \end{align*}
  Damit erreicht man die Konsistenzordnung 4.
  % oder:
  % -1/3  -1/3  -1/3
  % -1/3   8/3  -1/3
  % -1/3  -1/3  -1/3
  % Die Erweiterung auf nichtrechteckige Gebiete ist trivial
\end{bem}

% 3.3. Diskretisierung in einem beschränkten Gebiet in R^2

\begin{situation}
  Sei $\Omega \subset \R^2$ beschränkt.
\end{situation}

\begin{defn}
  \begin{itemize}
    \item $\Omega_h \coloneqq \Set{x, y \in \Omega}{\nicefrac{x}{h}, \nicefrac{y}{h} \in \Z}$ heißen \emph{innere Gitterpkte}.
    \item Ein Punkt $z_R \in \bOmega$ heißt \emph{Randgitterpunkt} (notiert $z_R \in \partial \Omega_h$), falls es einen inneren Gitterpunkt $z \in \Omega_h$ gibt, sodass $z_R = r + \alpha h e_1$ oder $z_R = z + \alpha h e_2$ mit $\abs{\alpha} \leq 1$.
    Die Nachbarn $N(x, y)$ eines Punktes $(x, y)$ sind $(x + s_r h, y)$, $(x - s_l h, y)$, $(x, y + y_o h)$, $(x, y - s_u h) $, falls $s_r, s_l, s_o, s_u \in \ocinterval{0}{1}$ und die Verbindungsstrecken zu $(x, y)$ in $\Omega$ liegen.
    \item Ein Punkt $(x, y) \in \Omega_h$ heißt \emph{randnah}, falls $(x, y)$ die Nachbarn $(x - s_l h, y)$, $(x + s_r h, y)$, $(x, y - s_u h)$, $(x, y+ s_o h)$ hat mit mindestens einem $s_i < 1$.
    Ansonsten heißt $(x, y)$ \emph{randfern}.
  \end{itemize}
\end{defn}

\begin{nota}
  Wir haben eine Einteilung $\Omega_h = \Omega_h^\rn \sqcup \Omega_h^\rf$ der Gitterpunkte in randnahe und randferne Punkte.
\end{nota}

\begin{center}
  \begin{tikzpicture}[scale=0.6]
    \draw[->] (-3,1.5) -- node [left] {$s_o$} (-3,2.5);
    \draw[->] (-3,1.5) -- node [left] {$s_u$} (-3,0);
    \draw[->] (-3,1.5) -- node [below] {$s_r$} (-1.5,1.5);
    \draw[->] (-3,1.5) -- node [below] {$s_l$} (-4.5,1.5);
    \draw [magenta,fill=magenta] (-3,1.5) circle (.06) node [above right] {$x$};
    \draw [blue,fill=blue] (-3,2.6) circle (.06);
    \draw [blue,fill=blue] (-1.4,1.5) circle (.06);
    \draw [green,fill=green] (-3,-0.1) circle (.06);
    \draw [green,fill=green] (-4.6,1.5) circle (.06);
    
    
    \draw (-0.5,0) -- (3.5,0);
    \draw (-0.5,1) -- (3.5,1);
    \draw (-0.5,2) -- (3.5,2);
    \draw (-0.5,3) -- (3.5,3);
    \draw (0,-0.5) -- (0,3.5);
    \draw (1,-0.5) -- (1,3.5);
    \draw (2,-0.5) -- (2,3.5);
    \draw (3,-0.5) -- (3,3.5);
    \draw (-0.5,2.5) .. controls (0.5,2.75) and (0,3.3) .. (2.75,3.3) .. controls (3.5,3.3) and (3.5,2) .. (3, 1.5) .. controls (2.5,1) and (2.5,1.75) .. (2,1) .. controls (1.5,0) and (1.5,0) .. (1.5,-0.5);
    \foreach \point in {(1,1), (2,2), (1,2), (0,0), (0,1)} {
      \draw [green,fill=green] \point circle (.06);
    }
    \foreach \point in {(0,2), (1,3), (2,3), (3,3), (3,2), (1,0)} {
      \draw [magenta,fill=magenta] \point circle (.06);
    }
    \foreach \point in {(0,2.7), (1,3.16), (2,3.28), (3,3.22), (3.22,3), (3.28,2), (3,1.5), (2,1), (1.55,0)} {
      \draw [blue,fill=blue] \point circle (.06);
    }
    
    \draw [blue,fill=blue] (4.5,3) circle (.1);
    \node at (5,3) [right] { Randgitterpunkt };
    \draw [magenta,fill=magenta] (4.5,2) circle (.1);
    \node at (5,2) [right] { randnaher Punkt };
    \draw [green,fill=green] (4.5,1) circle (.1);
    \node at (5,1) [right] { randferner Punkt };
  \end{tikzpicture}
\end{center}

% 3.3.1 Shortley-Weller-Diskretisierung

\begin{lem}[\emph{Dividierte Differenzen von Newton}]\mbox{}\\
  Für $u \in \Cont^3(\cinterval{x_l}{x_r})$, $x \in \ointerval{x_l}{x_r}$ gilt
  \[
    u''(x) = \underbrace{\frac{2}{x_r - x_l} \left( \frac{u(x_r) - u(x)}{x_r - x} - \frac{u(x) - u(x_l)}{x - x_l} \right)}_{\mathclap{= \frac{2}{x_r - x_l} \left( \tfrac{1}{x_r - x} u(x_r) + \tfrac{1}{x - x_l} u(x_l) \right) - \frac{2}{(x_r - x)(x - x_l)} u(x)}} + \O(x_r - x_l)
  \]
  % = & \frac{2}{x_r - x_l} \left( \tfrac{1}{x_r - x} u(x_r) + \tfrac{1}{x - x_l} u(x_l) \right) \\
  % & - \frac{2}{(x_r - x)(x - x_l)} u(x) + \O(x_r - x_l).
\end{lem}

\begin{verf}[\emph{Shortley-Weller-Diskretisierung}] \mbox{}\\
  Dadurch inspiriert approximieren wir den Laplace-Operator durch
  \begin{align*}
    \DO_h & u(x, y) = \frac{1}{h^2} \left( \frac{2 u(x - s_l h, y)}{s_l (s_r {+} s_l)} + \frac{2 u(x+s_r h, y)}{s_r (s_r {+} s_l)} \right. \\
    & \left. + \frac{2 u(x, y {-} s_u h)}{s_u (s_o {+} s_u)}  + \frac{2 u(x, y {+} s_o h)}{s_o (s_o {+} s_u)} - \left( \frac{2}{s_l s_r} + \frac{2}{s_o s_u} \right) u(x, y) \right)
  \end{align*}
  wobei
  $x_r - x = s_r h$, \enspace
  $x - x_l = s_l h$, \enspace
  $y_o - y = s_o h$, \enspace
  $y - y_u = s_u h$.
  % Vorlesung vom 3.11.2015
  Wir betrachten nun
  \begin{align*}
    \text{(RWP\tss{2})'\tss{h}}\enspace & \left\{ \begin{array}{rrcl}
      - \DO_h u_h &=& f_h &\text{ in $\Omega_h$} \\
      u_h &=& g &\text{ auf $\bOmega_h$}
    \end{array} \right. \\
    \text{(LGS\tss{2})'}\enspace & \left\{ \begin{array}{rcl}
      - \tilde{\DO}_h \tilde{u}_h &=& \tilde{f}_h \\
      \tilde{f}_h &=& f_h + g_h
    \end{array} \right.
  \end{align*}
  \[
    \text{mit} \quad
    g_h(x, y) = \nicefrac{1}{h^2} \qquad \sum_{\mathclap{(x_N, y_N) \in N(x, y) \cap \partial \Omega_h}} \qquad S_{x_N,y_N} g(x_N, y_N)
  \]
  wobei
  \begin{align*}
    S_{x_N,y_N} \coloneqq & \begin{cases}
      \nicefrac{2}{s_l (s_l + s_r)} & \text{falls $(x_N, y_N) = (x - s_l h, y)$,} \\
      \nicefrac{2}{s_r (s_l + s_r)} & \text{falls $(x_N, y_N) = (x + s_r h, y)$,} \\
      \nicefrac{2}{s_o (s_o + s_u)} & \text{falls $(x_N, y_N) = (x, y + s_o h)$,} \\
      \nicefrac{2}{s_u (s_o + s_u)} & \text{falls $(x_N, y_N) = (x, y - s_u h)$,} \\
    \end{cases} \\
    - \tilde{\DO}_h = (d_{ij}) & \quad \text{mit} \quad
    d_{ii} = \nicefrac{1}{h^2} \left( \tfrac{2}{s_{il} s_{ir}} + \tfrac{2}{s_{iu} s_{io}} \right) \quad \text{und} \\
    d_{ij} = \nicefrac{1}{h^2} & \begin{cases}
      - \nicefrac{2}{s_{il} (s_{il} + s_{ir})} & \text{falls $j$ der linke Nachbar von $i$ ist,} \\
      - \nicefrac{2}{s_{ir} (s_{il} + s_{ir})} & \text{falls $j$ der rechte Nachbar von $i$ ist,} \\
      - \nicefrac{2}{s_{iu} (s_{iu} + s_{io})} & \text{falls $j$ der untere Nachbar von $i$ ist,} \\
      - \nicefrac{2}{s_{io} (s_{iu} + s_{io})} & \text{falls $j$ der obere Nachbar von $i$ ist.} \\
    \end{cases}
  \end{align*}
\end{verf}

% 3.10
\begin{lem}
  \begin{itemize}
    \item Die Matrix $- \tilde{\DO}_h$ ist eine M-Matrix.
    \item Sei $\Omega \subset \R^2$ beschränkt und gehöre zu dem Streifen $(x_0, x_0 + d) \times \R$ oder $\R \times (y_0, y_0 + d)$.
    Dann gilt $\norm{\tilde{D}_h^{-1}} \leq \nicefrac{d^2}{8}$.
  \end{itemize}
\end{lem}

% Beweis: W. Hackbusch, Thm. 4.8.4

\begin{bem}
  Das DV (RWP\tss{2})'\tss{h} hat in den randnahen Punkten nur die Konsistenzordnung 1. Dennoch gilt:
  \iffalse
  Sei $u \in \Cont^4(\cOmega)$.
  Dann ist
  \begin{align*}
    u(x + s_r h, y) & = u(x, y) + s_r h u_x(x, y) + \tfrac{s_r^2 h^2}{2} u_{xx}(x,y) + \tfrac{s_r^3 h^3}{6} u_{xxx}(x, y) \\
    & + \tfrac{s_r^4 h^4}{24} u_{xxxx}(\xi_1, y)
    \qquad \text{für ein $\xi_1 \in \ointerval{x}{x+s_r h}$,} \\
    u(x - s_l h, y) & = u(x, y) - s_l h u_x(x, y) + \tfrac{s_l^2 h^2}{2} u_{xx}(x, y) - \tfrac{s_l^3 h^3}{6} u_{xxx}(x, y) \\
    & + \tfrac{s_r^4 h^4}{24} u_{xxxx}(\xi_2, y)
    \qquad \text{für ein $\xi_2 \in \ointerval{x - s_l h}{x}$.}
  \end{align*}
  
  Es folgt
  \begin{align*}
    & \nicefrac{1}{h^2} \left( \tfrac{2}{s_l (s_l + s_r)} u(x - x_l h, y) + \tfrac{2}{s_r (s_r + s_l)} u(x+s_r h, y) - \tfrac{2}{s_r s_l} u(x, y) \right) = \\
    & u_{xx}(x, y) + \tfrac{1}{3} (s_r - s_l) h u_{xxx}(x, y) + \tfrac{1}{12} h^2 \left( \tfrac{s_r^3 u_{xxxx}(\xi_1, y)}{s_r + s_l} + \tfrac{s_l^3 u_{xxxx}(\xi_2, y)}{s_r + s_l} \right)
  \end{align*}
  % 0)
  \[ \nicefrac{1}{h^2} \left( \tfrac{2}{s_l (s_r + s_l)} + \tfrac{2}{s_r (s_l + s_r)} + \tfrac{2}{s_r s_l} \right) u(x, y) = 0 \]
  % 1)
  \[ \nicefrac{1}{h^2} \left( \tfrac{2 s_r h}{s_r (s_r + s_l)} + \tfrac{- 2 s_l h}{s_l (s_r + s_l)} \right) u_x(x,y) = 0 \]
  % 2)
  \[ \nicefrac{1}{h^2} \left( \tfrac{2 s_r^2 h^2}{2 s_r (s_r + s_l)} + \tfrac{2 s_l^2 s_r^2}{2 s_l (s_r + s_l)} \right) u_{xx}(x, y) = u_{xx}(x, y) \]
  % 3)
  \[ \nicefrac{1}{h^2} \left( \tfrac{2 s_r^3 h^3}{6 s_r (s_r + s_l)} + \tfrac{- 2 s_l^3 h^3}{6 s_l (s_r + s_l)} \right) u_{xxx}(x, y) = \nicefrac{1}{3} (s_r - s_l) h u_{xxx}(x, y) \]
  % 4) ...
  \[
    \norm{\DO_h R_h u - R_h \Laplace u}_h \leq
    \begin{cases}
      \nicefrac{1}{6} h^2 \norm{u}_{\Cont^4(\cOmega)} & \text{falls $s_r = s_l$, $s_o = s_u$}, \\
      \left( \tfrac{s_r^2 + s_l^2}{s (s_r + s_l)} + \tfrac{s_u^2 + s_o^2}{3 (s_u + s_o)} \right) h \norm{u}_{\Cont^3(\cOmega)} & \text{sonst}
    \end{cases}
  \]
  Somit: Konsistenzordnung = 1
  \fi
\end{bem}

% 3.11, 4.8.6 im Hackbusch
\begin{satz}
  Sei $\Omega \subset \R^3$ beschränkt und Teilmenge des Streifens $\ointerval{x_0}{x_0 + d} \times \R$ oder $\R \times (y_0, y_0 + d)$.
  Dann ist das Verfahren (RWP\tss{2})'\tss{h} konvergent von der Ordnung 2.
  Es gilt
  \[ \norm{u_h - R_h u}_h \leq \left( \nicefrac{1}{3} h^3 + \nicefrac{d^2}{48} h^2 \right) \norm{u}_{\Cont^4(\cOmega)}. \]
\end{satz}

% Vorlesung vom 9.11.2015

% §3.3.2 Interpolation

\begin{idee}
  Bestimme den Wert von $u$ bei randnahen Punkten $(x, y)$ durch lineare Interpolation:
  \begin{itemize}
    \item $u(x, y) \approx \frac{s_r}{s_r + s_l} u(x - s_l h, y) + \frac{s_l}{s_r + s_l} u(x+s_r h, y)$
    \item $u(x, y) \approx \frac{s_o}{s_u + s_o} u(x, y - s_u h) + \frac{s_u}{s_u + s_o} u(x, y + s_o h)$
  \end{itemize}
\end{idee}

\[
  \text{(RWP\tss{2})''\tss{h}} \enspace \left\{ \begin{array}{rlll}
    - \DO_h u &=& f_h &\text{ in $\Omega_h$} \\
    u_h &=& g &\text{ auf $\bOmega_h$}
  \end{array} \right.
\]
\[
  \text{(LGS\tss{2})''} \enspace
  - \tilde{\DO}_h \tilde{u}_h = \tilde{f}_h
\]

\begin{lem}
  Dieses Verfahren besitzt Konsistenzordnung (und somit Konvergenzordnung) 2.
\end{lem}

\iffalse
Konsistenz: Mit Taylorn ...

\begin{itemize}
  \item \[ -s_r u_h(x - s_l h, y) + (s_r + s_l) u_h(x, y) - s_l u(x + s_r h, y) = 0 \]
  \item \[ -s_o u_h(x, y - s_u h) + (s_o + s_u) u_h(x, y) - s_u u(x, y + s_o h) = 0 \]
\end{itemize}
\fi

% §3.4 Allgemeine Differentialoperatoren

\begin{prob}
  Wir betrachten nun
  \[ \text{(RWP\tss{3})} \enspace \left\{ \begin{array}{rlll}
    - \LL u &=& f &\text{ in $\Omega = \ointerval{0}{1} \times \ointerval{0}{1}$} \\
    u &=& g &\text{ auf $\bOmega$}
  \end{array} \right. \]
  mit
  \begin{align*}
    - \LL u&  = - (a_{11}(x, y) u_{xx} + 2 a_{12}(x, y) u_{xy} + a_{22}(x, y) u_{yy}) \\
    & + b_1(x, y) u_x + b_2(x, y) u_y + c(x, y) u
  \end{align*}
  wobei $c(x, y) \leq 0$, $\xi^T A(x, y) \xi \geq \lambda_0 \norm{\xi}^2$, $\lambda_0 > 0$ und
  \[
    A(x, y) = \begin{pmatrix}
      a_{11}(x,y) & a_{12}(x,y) \\
      a_{21}(x,y) & a_{22}(x,y)
    \end{pmatrix}
  \]
\end{prob}

\begin{verf}
  \begin{enumerate}
    \item Diskretisierung: $h = \nicefrac{1}{n}$, $\Omega_h$, $\partial \Omega_h$ wie früher.
    \item Approximation:
    \begin{alignat*}{4}
      u_x(x, y) & \approx \tfrac{u(x + h, y) - u(x - h, y)}{2h}, \quad
      & u_y(x, y) & \approx \ldots \\
      u_{xx}(x, y) & \approx \tfrac{u(x + h, y) - 2 u(x, y) + u(x - h, y)}{h^2}, \quad
      & u_{yy}(x, y) & \approx \ldots
    \end{alignat*}
    Für die Approx. von $u_{xy}$ haben wir mehrere Möglichkeiten: \\ Wir könnten etwa den zentralen DQ in $x$- und $y$-Richrung verwenden und erhalten
    \begin{align*}
      u_{xy}(x, y) \approx \tfrac{1}{4 h^2} & \left( u(x{+}h, y{+}h) - u(x{+}h, y{-}h) \right. \\
      & \left. - u(x{-}h, y{+}h) + u(x{-}h, y{-}h) \right)
    \end{align*}
    Diese Annäherung hat allerdings den Nachteil, dass sie zu keiner M-Matrix führt.
    Stattdessen nehmen wir
    \iffalse
    \begin{align*}
      u_{xy}(x, y) & \approx \tfrac{1}{2 h^2} (- u(x-h, y) + 2 u(x,y) - u(x+h,y) \\
      & + u(x-h,y-h) - u(x, y+h) - u(x,y-h) \\
      & + u(x+h, y+h) ) \qquad \text{falls $a_{12}(x, y) \geq 0$,} \\
      u_{xy}(x, y) & \approx \tfrac{1}{2 h^2} ( u(x-h, y) - 2 u(x, y) + u(x+h, y) \\
      & + u(x, y+h) - u(x-h, y+h) - u(x+h, y-h) \\
      & + u(x,y-h) ) \qquad \text{falls $a_{12}(x, y) < 0$}
    \end{align*}
    \fi
    \begin{alignat*}{4}
      u_{xy}(x, y) \approx \qquad
      & \tfrac{1}{2 h^2} \begin{pmatrix}
        0 & -1 & 1 \\
        -1 & 2 & -1 \\
        1 & -1 & 0
      \end{pmatrix} \qquad
      && \tfrac{1}{2 h^2} \begin{pmatrix}
        -1 & 1 & 0 \\
        1 & -2 & 1 \\
        0 & 1 & -1
      \end{pmatrix} \\
      & \text{für } a_{12} \geq 0
      && \text{für } a_{12} < 0.
    \end{alignat*}

    \iffalse
    Insgesamt haben wir als Approximation
    \begin{align*}
      - \LL u(x, y) & \approx \tfrac{1}{h^2} \left( (-a_{11} + \abs{a_{12}} + \nicefrac{h}{2} b_1) u(x-h, y) \right. \\
      & + (-a_{11} + \abs{a_{12}} - \nicefrac{h}{2} b_1) u(x + h, y) \\
      & + (2 a_{11} + 2 a_{22} - 2 \abs{a_{12}} + h^2 c) u(x, y) \\
      & + (-a_{22} + \abs{a_{12}} + \nicefrac{h}{2} b_2) u(x, y-h) \\
      & + (-a_{22} + \abs{a_{12}} - \nicefrac{h}{2} b_2) u(x, y+h) \\
      & + (- \max(a_{12}, 0) u(x-h, y-h) \\
      & + (- \max(a_{12}, 0)) u(x+h, y+h) \\
      & + \min(a_{12}, 0) u(x-h, y+h) \\
      & \left. + \min(a_{12}, 0) u(x + h, x - h)) \right)
    \end{align*}
    \fi
    
    Wir fassen diese Approx. in folgendem 7-Stern zusammen:
    \begin{align*}
      - \LL_h u \coloneqq \enspace
      & \tfrac{1}{h^2} \begin{pmatrix}
        a_{12}^{-} & \abs{a_{12}} - a_{22} & a_{12}^{+} \\
        \abs{a_{12}} - a_{11} & 2 (a_{11} + a_{22} - \abs{a_{12}}) & \abs{a_{12}} - a_{11} \\
        - a_{12}^{+} & \abs{a_{12}} - a_{22} & a_{12}
      \end{pmatrix} \\
      & + \tfrac{1}{h} \begin{pmatrix}
        & - b_2 \\
        - b_1 & 0 & b_1 \\
        & b_2
      \end{pmatrix}
      + \begin{pmatrix}
        \\ & c & \\ &
      \end{pmatrix}
    \end{align*}
  \end{enumerate}
  Dabei ist $a_{ij}^{+} \coloneqq \max(a_{ij}, 0)$ und $a_{ij}^{-} \coloneqq \min(a_{ij}, 0)$.
  \begin{align*}
    \text{(RWP\tss{3})\tss{h}} \enspace & \left\{ \begin{array}{rlll}
      - \LL_h u_h &=& f_h &\text{ in $\Omega_h$} \\
      u_h &=& g &\text{ auf $\bOmega_h$}
    \end{array} \right. \\
    \text{(LGS\tss{3})} \enspace
    & - \tilde{\LL}_h \tilde{u}_h = \tilde{f}_h
  \end{align*}
\end{verf}

% Vorlesung vom 10.11.2015

\begin{satz}
  Sei $\abs{a_{12}} \leq \min(a_{11}, a_{22})$, $c \geq 0$ in $\Omega$, $\LL$ gleichmäßig elliptisch.
  Falls $a_{ii} > \abs{a_{12}} + \tfrac{h}{2} \abs{b_i}$ für $i = 1, 2$ in $\Omega$ und $u \in \Cont^4(\cOmega)$, so ist das DV (RWP\tss{3})\tss{h} konvergent von der Ordnung 2.
\end{satz}

% 3.5 Differenzenverfahren für parabolische DGLn

\begin{prob}\mbox{}
  \vspace{-1.5em}
  \[
    \arraycolsep=2.5pt
    \text{(RWP\tss{4})} \enspace \left\{
    \begin{array}{rcll}
      u_t(x,t) - \Laplace_x u(x, t) &=& f(x,t) &\text{in $\Omega = \ointerval{0}{1} \times \ointerval{0}{T}$} \\
      u(x,0) &=& g(x) & \text{für } x \in \ointerval{0}{1} \\
      u(0, t) &=& g_0(t) & \text{für } t \in \cinterval{0}{T} \\
      u(1, t) &=& g_1(t) & \text{für } t \in \cinterval{0}{T}
    \end{array} \right.
  \]
\end{prob}

\begin{verf}
  \begin{enumerate}
    \item Diskretisierung mit $n$ Raum- und $m$ Zeitschritten:
    \[
      x_i = ih, \enspace
      h = \nicefrac{1}{n}, \enspace
      t_k = k \tau, \enspace
      \tau = \nicefrac{T}{m}, \enspace
      u(x_i, t_k) \approx u_i^k
    \]
    \item Approximation der Ableitungen:
    \[ u_{xx}(x,t) \approx \tfrac{1}{h^2} \left( u(x-h, t) - 2u(x,t) + u(x+h, t) \right) =: \Laplace_h u(x,t) \]
    Wir wollen nun eine Lösung von
    \[ \left\{ \begin{array}{rll}
      \dot{u}_h(t) - \tilde{\Laplace}_h u_h(t) &=& f_h(t) \\
      u_h(0) &=& g_h
    \end{array} \right. \]
    für alle Zeiten $t$ mit
    \[
      u_h(t) = \begin{pmatrix}
        u_h(h, t) \\
        u_h(2h, t) \\
        \vdots \\
        u_h(1-h,t)
      \end{pmatrix}, \quad
      f_h(t) = \begin{pmatrix}
        f(h, t) + \tfrac{1}{h^2} g_0(t) \\
        f(2h, t) \\
        \vdots \\
        f(1-h,t) + \tfrac{1}{h^2} g_1(t)
      \end{pmatrix}
    \]
    berechnen.
    Dazu verwenden wir ein Einschrittverfahren, wie das expl./impl. Gauß-Verfahren oder das Crank-Nicolson-Verfahren:
    \begin{align*}
      \text{(EEV)} \enspace
      & \left\{ \begin{array}{rcl}
        \tfrac{1}{\tau} (u_i^{k+1} - u_i^k) - \tilde{\Laplace}_h u_i^k &=& f_i^k \\
        u_i^0 &=& g_h
      \end{array} \right. \\
      %\tag{Expl. Euler-Verf}
      \text{(IEV)} \enspace
      & \left\{ \begin{array}{rcl}
        \tfrac{1}{\tau} (u_i^{k+1} - u_i^k) - \tilde{\Laplace}_h u_i^{k+1} &=& f_i^{k+1} \\
        u_i^0 &=& g_h
      \end{array} \right. \\
      %\tag{Impl. Euler-Verf}
      \text{(CNV)} \enspace
      & \left\{ \begin{array}{rcl}
        \tfrac{1}{\tau} (u_i^{k+1} - u_i^k) - \tfrac{1}{2} \tilde{\Laplace}_h (u_i^k + u_i^{k+1}) &=& f(x_i, t_k + \tfrac{\tau}{2}) \\
        u_i^0 &=& g_h
      \end{array} \right.
      %\tag{Crank-Nicolson-Verf}
    \end{align*}
  \end{enumerate}
\end{verf}

% 3.13
\begin{lem}
  Sei $f(x, \blank) \in \Cont^1(\cinterval{0}{T})$ für alle $x \in \cinterval{0}{1}$. \\
  Dann gilt für die Approximation von (RWP\tss{4}):
  \begin{itemize}
    \item Die Verfahren (EEV) und (IEV) besitzen einen Konsistenzfehler von $\O(h^2 + \tau)$, falls $u \in \Cont^4(\cinterval{0}{1} \times \cinterval{0}{T})$
    \item Das Verfahren (CNV) besitzt einen Konsistenzfehler von $\O(h^2 + \tau^2)$, falls $u \in \Cont^4(\cinterval{0}{1} \times \cinterval{0}{T})$.
  \end{itemize}
\end{lem}

% Vorlesung vom 16.11.2015

% Stabilität: $u_i^k \xra[k \to \infty]{} 0$

% 3.14
\begin{lem}
  Es gelte $2 \tau \leq h^2$ für (EEV).
  Die Verfahren (EEV), (IEV) und (CNV) sind stabil.
\end{lem}

\iffalse
\[
  (I - \sigma \tau \tilde{\Laplace}_h ) u_h(t_{k+1}) = (I + (1 - \sigma) \tau \tilde{\Laplace}_h ) u_h(t_k) + \tau f_h(t_k + \sigma \tau)
\]

Man erhält für $\sigma=0$ das EEV, für $\sigma=1$ das IEV und für $\sigma = \nicefrac{1}{2}$ das CNV.
Es folgt

\begin{align*}
  u_h(t_{k+1}) & = (I - \sigma \tau \tilde{\Laplace}_h)^{-1} (I + (1-\sigma) \tau \tilde{\Laplace}_h) u_h(t_k) + (I - \sigma \tau \tilde{\Laplace}_h)^{-1} f_h(t_k + \sigma \tau) = \ldots \\
  & \left( (I - \sigma \tau \tilde{\Laplace}_h)^{-1} (I + (1-\sigma) \tau \tilde{\Laplace}_h) \right)^{k} u_h(t_0) + (I - \sigma \tau \tilde{\Laplace}_h)^{-1} \sum_{j=0}^k \mu_j f_h(t_j + \sigma \tau)
\end{align*}

Ziel: $u_h(t_{k+1}) \xra[k \to \infty]{} 0$

Für Stabilität sollte $\abs{\lambda_j(M)} < 1$ gelten, wobei $M = (I - \sigma \tau \tilde{\Laplace}_h)^{-1} (I + (1-\sigma) \tau \tilde{\Laplace}_h)$.

Es gilt: $\lambda_j(M) = \frac{1 + (1 - \sigma) \tau \lambda_j(\tilde{\Laplace})}{1 - \sigma \tau \lambda_j(\tilde{\Laplace})}$

\begin{bem}
  Konsistenz + Stabilität $\implies$ Konvergenz
\end{bem}
\fi

% §3.6 Differenzenverfahren für hyperbolische Gleichungen

\begin{prob}
  Wellengleichung
  \[
    \left\{ \begin{array}{rl}
      \partial_{tt} u - c^2 \partial_{xx} u = f(x,t) & \text{in $\Omega = \ointerval{0}{1} \times \cinterval{0}{T}$} \\
      u(0, t) = g_0(t), u(1, t) = g_1(t) & \text{für $t \in \cinterval{0}{T}$} \\
      u(x, 0) = q_0(x), u_t(x, 0) = q_1(x) & \text{für $x \in \ointerval{0}{1}$}
    \end{array} \right.
  \]
\end{prob}

\begin{verf}
  \begin{enumerate}
    \item Diskretisierung: $x_i = ih$, $h = \tfrac{1}{n}$, $t_k = k \tau$, $\tau = \tfrac{T}{m}$
    \item Approximation:
    \begin{align*}
      \partial_{xx} u(x_i, t_k) & \approx \tfrac{1}{h^2} \left( u(x_{i-1}, t_k) - 2 u(x_i, t_k) + u(x_{i+1}, t_k) \right) \\
      \partial_{tt} u(x_i, t_k) & \approx \tfrac{1}{\tau^2} \left( u(x_i, t_{k-1}) - 2 u(x_i, t_k) + u(x_i, t_{k+1}) \right) \\
      \partial_t u(x_i, 0) & \approx \tfrac{1}{2 \tau} ( u(x_i, t_1) - u(x_i, t_{-1}) )
    \end{align*}
    Wir erhalten das lineare Gleichungssystem
    \[
      \left\{ \begin{array}{r}
        \tfrac{1}{\tau^2} (u_i^{k-1} - 2 u_i^k + u_i^{k+1}) - \tfrac{c^2}{h^2} ( U_{i-1}^k - 2 u_i^k + u_{i+1}^k) = f_i^k \\
        \text{für $i = 1, \ldots, n-1$ und $k = 0, \ldots, m$.} \\
        u_0^k = g_0^k = g_0(t_k), \quad
        u_n^k = g_1^k = g_1(t_k), \\
        u_i^0 = q_{0,i} = q_0(x_i), \quad
        \tfrac{1}{2 \tau} ( u_i^1 - u_i^{-1} ) = q_{1,i} = q_1(x_i)
      \end{array} \right.
    \]
    % (ausgelassen: wie man jetzt ein LGS in Matrixschreibweise bekommt)
  \end{enumerate}
\end{verf}

\begin{bem}
  Das Differenzenverfahren \ldots
  \begin{description}
    \item[{\Smiley[1.2]}] \ldots{} ist einfach in der Herleitung und Implementierung. 
    \item[{\Neutrey[1.2]}] \ldots{} besitzt eine gute Konvergenz (\zB{} Ordnung 2) bei genügend glatter Lösung. 
    \item[{\Sadey[1.2]}] \ldots{} ermöglicht Adaptivität bzw. unregelm. Gitter nur schwer. 
  \end{description}
\end{bem}

% Vorlesung vom 17.11.2015

\section{Schwache Lsgstheorie für elliptische DGLn}

\iffalse
\begin{bspe}
  \begin{itemize}
    \item $u''(x) = f$ in $\Omega = \ointerval{0}{1}$, $u(0) = g_0$, $u'(0) = g_1$
    \item $u_t(x, t) + c u_x(x, t) = 0$, $u(x, 0) = g(x)$, $g \in \Cont^1$. Lösung: $u(x, t) = g(x - c t)$
  \end{itemize}
\end{bspe}
\fi

% §4.1 Grundlagen aus der Funktionalanalysis

\begin{defn}
  Der \emph{$L^p$-Raum} ist für $1 \leq p < \infty$ definiert durch
  \[
    L^p(\Omega) \coloneqq \Set{v : \Omega \to \R}{\norm{v}_p < \infty}
    \quad \text{mit} \enspace
    \norm{v}_p \coloneqq \left( \Int{\Omega}{}{\abs{v(x)}^p}{x} \right)^{\nicefrac{1}{p}},
  \]
  für $p = \infty$ durch
  \[
    L^\infty(\Omega) \coloneqq \Set{v : \Omega \to \R}{\norm{v}_\infty < \infty}
    \quad \text{mit} \enspace
    \norm{v}_\infty \coloneqq \esssup_{x \in \Omega} \abs{v(x)}.
  \]
\end{defn}

\begin{bem}
  $(L^p(\Omega), \norm{\blank}_p)$ ist ein Banachraum, für $p = 2$ sogar ein Hilbertraum mit Skalarprodukt $\scp{u}{v}_{L^2(\Omega)} \coloneqq \Int{\Omega}{}{u(x) v(x)}{x}$.
\end{bem}

\begin{satz}[\emph{Höldersche Ungleichung}]
  Sei $u \in L^p(\Omega)$ und $v \in L^q(\Omega)$ mit $1 \leq p, q, r < \infty$ und $\nicefrac{1}{p} + \nicefrac{1}{q} = \tfrac{1}{r}$.
  Dann ist $uv \in L^r(\Omega)$ mit
  \[ \norm{uv}_r \leq \norm{u}_p \cdot \norm{v}_q. \]
\end{satz}

\begin{defn}
  Die Menge aller $k$-mal stetig differenzierbaren Funktionen auf $\Omega$ mit \emph{kompaktem Träger} ist
  \[ \Cont_0^k(\Omega) \coloneqq \Set{\varphi \in \Cont^k(\Omega)}{\supp(\varphi) \coloneqq \clos{\Set{x \in \Omega}{\varphi(x) \neq 0}} \text{ ist kompakt}}. \]
\end{defn}

\begin{defn}
  $\Testfun(\Omega) \coloneqq \Cont^\infty_0 (\Omega)$ heißt Raum der \emph{Testfunktionen} in $\Omega$.
\end{defn}

\begin{lem}[Partielle Intgration]
  Für $u, v \in \Cont^1(\cOmega)$ gilt
  \[ \Int{\Omega}{}{v(x) \DO_i u(x)}{x} = \Int{\partial \Omega}{}{v(x) u(x) \eta_i(x)}{x} - \Int{\Omega}{}{\DO_i v(x) u(x)}{x}. \]
  % Falls $v(x) = 0$ auf $\partial \Omega$: $\Int{\Omega}{}{v(x) \DO_i u(x)}{x} = - \Int{\Omega}{}{\DO_i v(x) u(x)}{x}$
  Für $u \in \Cont^k(\cOmega)$, $\varphi \in \Cont_0^k(\Omega)$ und $\alpha = (\alpha_1, \ldots, \alpha_n)$, $\abs{\alpha} \leq k$ gilt
  \[ \Int{\Omega}{}{\varphi(x) \DO^\alpha u(x)}{x} = (-1)^{\abs{\alpha}} \Int{\Omega}{}{\DO^\alpha \varphi(x) u(x)}{x}. \]
\end{lem}

\begin{defn}
  $L_\loc^1(\Omega) \coloneqq \Set{v : \Omega \to \R}{v|_K \in L^1(K) \text{ für jedes kpkte $K \subset \Omega$}}$
  heißt Raum der \emph{lokal integrierbaren Funktionen}.
\end{defn}

\begin{defn}
  Sei $u \in L^1_\loc(\Omega)$ und $\alpha \in \N^n$.
  Eine Funktion $v \in L^1_\loc(\Omega)$ heißt \emph{schwache (partielle) Ableitung} von $u$ der Ordung $\alpha$ (oder die Ableitung von $u$ im distributionellen Sinn), wenn
  \[
    \Int{\Omega}{}{\DO^\alpha \varphi(x)}{x} = (-1)^{\abs{\alpha}} \Int{\Omega}{}{v(x)}{\varphi(x)}{x}
    \quad \text{für alle $\varphi \in \Testfun(\Omega)$}.
  \]
\end{defn}

\begin{bem}
  Ist eine Funktion im klassischen Sinne diff'bar, so auch im schwachen mit derselben Ableitung.
\end{bem}

\begin{lem}[Fundamentallemma der Variationsrechung] \mbox{}\\
  Sei $u \in L^1_\loc$ und $\Int{\Omega}{}{u(x) \varphi(x)}{x} = 0$ für alle $\varphi \in \Testfun(\Omega)$. \\
  Dann gilt $u(x) = 0$ für fast alle $x \in \Omega$.
\end{lem}

\begin{kor}
  Die schwache Ableitung ist eindeutig bestimmt, \dh{} sind $v, w \in L^1_\loc(\Omega)$ schwache Ableitungen von $u$, so gilt $v = w$ f.\,ü. in $\Omega$.
\end{kor}

\begin{bsp}
  Die schw. Abl. von $u(x) = \abs{x}$ ist $v(x) = \ind_{\ointerval{0}{\infty}} - \ind_{\ointerval{- \infty}{0}}$.
\end{bsp}

\begin{lem}
  \inlineitem{$\DO^\alpha (u + \lambda v) = \DO^\alpha u + \lambda \DO^\alpha v$} \quad
  \inlineitem{$\DO^{\alpha + \beta} u = \DO^\alpha (\DO^\beta u)$}
\end{lem}

% 4.1.2
\begin{defn}
  Der \emph{Sobolev-Raum} für $1 \leq p < \infty$ ist
  \begin{align*}
    W^{k,p}(\Omega) & = \left\{ u \in L^p(\Omega) \;\middle|\; \begin{array}{l}
      \fa{\alpha \in \N^n \text{ mit } \abs{\alpha} \leq k} \\
      \exists \, \text{schwache Ableitung $\DO^\alpha u \in L^p(\Omega)$}
    \end{array} \right\} \\
    \norm{u}_{k,p} & \coloneqq \left( \sum_{\abs{\alpha} \leq k} \norm{\DO^\alpha u}_p^p \right)^{\nicefrac{1}{p}}.
  \end{align*}
\end{defn}

\begin{nota}
  $H^k(\Omega) \coloneqq W^{k,2}(\Omega)$
\end{nota}

% Bem: $W^{0,p}(\Omega) = L^p(\Omega)$

\begin{satz}
  $(W^{k,p}(\Omega), \norm{\blank}_{k,p})$ ist ein Banachraum.
\end{satz}

% Vorlesung vom 23.11.2015

% 4.2
\begin{bem}
  \begin{itemize}
    \item Auf $H^k(\Omega)$ wird durch
    \[
      \scp{u}{v}_{H^k(\Omega)} \coloneqq \Int{\Omega}{}{\sum_{\alpha \leq k} D^\alpha u D^\alpha v}{x}
    \]
    ein Skalarprodukt definiert, das die Norm $\norm{\blank}_{k,2}$ induziert.
    \item $(H^k(\Omega), \scp{\blank}{\blank}_{H^k(\Omega)})$ ist ein Hilbertraum.
  \end{itemize}
\end{bem}

% Beispiel: $H_1(\Omega)$, $\scp{u}{v}_{H^1(\Omega)} = \Int{\Omega}{}{uv}{x} + \Int{\Omega}{}{\nabla u \cdot \nabla v}{x}$.

\begin{satz}["`$H = W$"'] % Meyers/Serrin
  $W^{k,p}(\Omega) \cap \Cont^\infty(\Omega)$ liegt dicht in $W^{k,p}(\Omega)$, \dh{}
  \[
    %H^{k,p} \coloneqq
    \clos{W^{k,p}(\Omega) \cap \Cont^\infty(\Omega)}^{\norm{\blank}_{k,p}} = W^{k,p}(\Omega).
  \]
\end{satz}

\begin{defn}
  $W_0^{k,p}(\Omega) \coloneqq \clos{\Testfun(\Omega)}^{\norm{\blank}_{k,p}}$, \quad
  $H_0^1(\Omega) \coloneqq W^{1,2}_0(\Omega)$
\end{defn}

\begin{satz}
  Sei $\Omega$ ein beschränktes $\Cont^1$-Gebiet und $1 \leq p < \infty$. \\
  Dann existiert eine lineare stetige Abbildung $\tau : W^{1,p}(\Omega) \to L^p(\partial \Omega)$, sodass für alle $u \in W^{1,p}(\Omega) \cap \Cont(\clos{\Omega})$ gilt: $\tau(u) = u|_{\partial \Omega}$.
\end{satz}

% Beweis: Buch von Evans, Satz 1 in Abschnitt 5.5

\begin{defn}
  Die Abbildung $\tau$ heißt \emph{Spuroperator}, $\tau(u)$ heißt die \emph{Spur} von $u \in W^{1,p}(\Omega)$ auf $\partial \Omega$.
\end{defn}

\begin{satz}
  Sei $\Omega$ ein beschränktes $\Cont^1$-Gebiet.
  Dann gilt
  \[
    W_0^{1,p}(\Omega) = \Set{v \in W^{1,p}(\Omega)}{\tau(v) = 0}.
  \]
\end{satz}

% Beweis: Evans, Satz 2 in Abschnitt 5.5

% 4.1.3 Dualräume

\begin{defn}
  Sei $(U, \norm{\blank}_U)$ ein Banachraum. \\
  Der \emph{Dualraum} von $U$ ist $(U', \norm{\blank}_{U'})$ mit
  \begin{align*}
    U' & \coloneqq \{ \text{lineare, stetige Abbildungen } \psi : U \to \R \} \\
    \norm{\psi}_{U'} & \coloneqq \sup_{u \in U \setminus \{ 0 \}} \frac{\abs{\psi(u)}}{\norm{u}_U}.
  \end{align*}
\end{defn}

\begin{bsp}
  %$(L^p(\Omega))' \cong L^q(\Omega)$
  Gelte $\nicefrac{1}{p} + \nicefrac{1}{q} = 1$ mit $p, q \in \ointerval{0}{\infty}$ Die Abbildung
  \[
    j : L^q(\Omega) \to (L^p(\Omega))', \quad
    f \mapsto (g \mapsto \Int{\Omega}{}{f(x) g(x)}{x})
  \]
  ist ein isometrischer Isomorphismus.
\end{bsp}

\begin{nota}
  $\scp{\psi}{u}_{U', U} \coloneqq \psi(u)$ für $\psi \in U'$, $u \in U$.
\end{nota}

% 4.6
\begin{satz}[\emph{Riesz'scher Darstellungssatz}] \mbox{} \\
  Sei $(H, \scp{\blank}{\blank}_H)$ ein Hilbertraum.
  Dann ist 
  \[
    j : H \to H', \quad
    \psi \mapsto (\phi \mapsto \scp{\psi}{\phi}_H)
  \]
  ein isometrischer Isomorphismus.
\end{satz}

\begin{defn}
  $W^{-1,q} \coloneqq (W_0^{1,p}(\Omega))'$, \quad
  $H^{-1}(\Omega) \coloneqq (H_0^1(\Omega))'$.
\end{defn}

% 4.2. Variationsgleichungen

\begin{situation}
  Sei $\Omega \subset \R^d$ beschränkt.
  Wir betrachten nun wieder
  \[
    \text{(RWP\tss{1})} \enspace \left\{ \begin{array}{rlll}
      \LL u &=& f &\text{ in $\Omega$} \\
      u &=& 0 &\text{ auf $\bOmega$}
    \end{array} \right.
  \]
  \begin{align*}
    \text{mit} \quad
    \LL u & = - \sum_{i=1}^d \DO_i (\sum_{j=1}^d a_{ij}(x) \DO_j u) + \sum_{i=1}^d b_i(x) \DO_i u + c(x) u \\
    & = - \dive(A(x) \DO u) + b(x) \cdot \DO u + c(x) u.
  \end{align*}
  Sei $u$ eine Lsg von (RWP\tss{1}) und $\phi \in \Testfun(\Omega)$.
  Dann gilt
  \begin{align*}
    & \enspace \Int{\Omega}{}{f(x) \phi(x)}{x}
    = \enspace \Int{\Omega}{}{\LL u \phi}{x} \\
    = & \enspace - \Int{\Omega}{}{\dive(A(x) \DO u(x)) \phi(x)}{x} + \Int{\Omega}{}{\left( b(x) \cdot \DO u(x) + c(x) u(x) \right) \cdot \phi(x)}{x} \\
    = & \enspace \Int{\Omega}{}{A(x) \DO u(x) \cdot \DO \phi(x)}{x} + \Int{\Omega}{}{\left( b(x) \cdot \DO u(x) + c(x) u(x) \right) \cdot \phi(x)}{x}
  \end{align*}

  \iffalse
  Nebenrechnung:
  \begin{align*}
    & - \Int{\Omega}{}{\sum_i \DO_i (\sum_j a_{ij}(x) \DO_j u(x) \phi(x))}{x} \\
    = & - \Int{\partial \Omega}{}{\sum_i (\sum_j a_{ij}(x) \DO_j u(x)) \eta_i(x) \phi(x)}{s(x)} \\
    & + \Int{\Omega}{}{\sum_i (\sum_j a_{ij}(x) \DO_j u(x)) \DO_i \phi(x)}{x} \\
    = & \Int{\Omega}{}{A(x) \DO u(x) \cdot \DO \phi(x)}{x}
    % weil die Testfunktion auf dem Rand verschwindet
  \end{align*}
  \fi
\end{situation}

% Vorlesung vom 24.11.2015

\begin{defn}
  Die Funktion $u \in H_0^1(\Omega)$ heißt \emph{schwache Lösung} von (RWP\tss{1}), wenn $u$ folgende Variationsgleichung erfüllt:
  \begin{align*}
    & \Int{\Omega}{}{A(x) \DO u(x) \cdot \DO \phi(x) + b(x) \cdot \DO u(x) \phi(x) + c(x) u(x) \phi(x)}{x} \\
    = \enspace & \Int{\Omega}{}{f(x) \phi(x)}{x}
    \quad \text{für alle } \phi \in H_0^1(\Omega).
    \tag{VGL\tss{1}}
  \end{align*}
\end{defn}

% Fragen:
% 1) Ist jede klassische Lösung von (RWP\tss{1}) auch Lösung von (VGL\tss{1})?
% 2) Ist (VGL\tss{1}) lösbar? Ist die Lösung von (VGL\tss{1}) eindeutig? Hängt die Lösung von (VGL\tss{1}) stetig von den Daten ab?
% 3) Wie löst man (VGL\tss{1}) numberisch?
% 4) Wie lautet die Variationsgleichungbei inhomogenen RWBn?

\begin{prob}[Allgemeines Variationsproblem] \mbox{} \\
  Seien Abb. $B : H_0^1(\Omega) \times H_0^1(\Omega) \to \R$ und $l : H_0^1(\Omega) \to \R$ gegeben. \\
  Gesucht: $u \in H_0^1(\Omega)$, sodass $B(u, \phi) = l(\phi)$ für alle $\phi \in H_0^1(\Omega)$.
\end{prob}

\begin{bem}
  Im obigen Setting ist \enspace $l(\phi) \coloneqq \Int{\Omega}{}{f(x) \phi(x)}{x}$,
  \begin{align*}
    & B(u, \phi) \coloneqq \Int{\Omega}{}{A(x) \DO u(x) \cdot \DO \phi(x) + b(x) \cdot \DO u(x) \phi(x) + c(x) u(x) \phi(x)}{x} \\
    & 
  \end{align*}
\end{bem}

\begin{defn}
  Sei $X$ ein Banachraum.
  Eine Abb. $B : X \times X \to X$ heißt
  \begin{itemize}
    %\item \emph{bilinear}, falls $B(u, \blank)$ und $B(\blank, \phi)$ linear für alle $u, \phi \in X$ sind,
    %\item \emph{symmetrisch}, falls $B(u, \phi) = B(\phi, u)$ für alle $u, \phi \in X$,
    \item \emph{positiv}, falls $B(u, u) > 0$ für alle $u \in X \setminus \{ 0 \}$,
    \item \emph{stark positiv} (oder \emph{koerziv}), falls $\lambda > 0$ existiert, sodass
    \[ \fa{u \in X} B(u, u) \geq \lambda \norm{u}_X^2, \]
    \item \emph{beschränkt}, falls ein $\mu > 0$ existiert, sodass
    \[ \fa{u, \phi \in X} \abs{B(u, \phi)} \leq \mu \norm{u}_X \norm{\phi}_X. \]
  \end{itemize}
\end{defn}

\begin{lem}
  \begin{itemize}
    \item Die Abbildung $B$ in (VGL\tss{1})' ist bilinear und beschränkt.
    \item Die Abbildung $l$ in (VGL\tss{1})' linear und stetig.
  \end{itemize}
\end{lem}

\begin{satz}
  Sei $\Omega$ ein beschr. Lipschitz-Gebiet.
  Dann ist jede klassische Lsg $u \in \Cont^2 \cap \Cont^1(\bOmega)$ von (RWP\tss{1}) eine schwache Lsg von (VGL\tss{1})'.
\end{satz}

% 4.3 Eindeutige Lösung elliptischer PDEs

% 4.9
\begin{satz}[\emph{Lax-Milgram}]
  Sei H ein Hilbertraum, $B : H \times H \to H$ eine beschränkte, koerzitive Bilinearform.
  Dann gibt es für jedes $l \in H'$ eine eindeutige Lösung $u \in H$ von  $\fa{\phi \!\in\! H\!\!}\!\! B(u, \phi) \!=\! l(\phi)$.
  Es gilt $\norm{u}_H \leq \nicefrac{1}{\lambda} \norm{l}_{H'}$
\end{satz}

% Vorlesung vom 30.11.2015

% 4.10
\begin{lem}[\emph{Poincaré-Ungleichung}]
  Sei $\Omega \subset \R^d$ beschränkt.
  Dann existiert eine Konstante $C > 0$, sodass
  \[
    \norm{u}_{L^2(\Omega)} \leq C \left( \Int{\Omega}{}{\sum \abs{\DO_i u}^2}{x} \right)^{\nicefrac{1}{2}} \quad
    \forall \, u \in H_0^1(\Omega)
  \]
\end{lem}

\begin{kor}
  Mit $C_1 \coloneqq (1 + C^2)^{\nicefrac{-1}{2}}$ und $C_2 \coloneqq 1$ gilt
  \[ C_1 \norm{u}_{L^2(\Omega)} \leq \left( \Int{\Omega}{}{\sum \abs{\DO_i u}^2}{x} \right)^{\nicefrac{1}{2}} \leq C_2 \norm{u}_{H'(\Omega)}. \]
\end{kor}

% 4.11
\begin{lem}
  Falls $b(x) = 0$ und $c(x) \geq 0$ in $\Omega$, so ist $B$ in (VGL\tss{1})' koerziv.
\end{lem}

% 4.12
\begin{satz}
  Sei $\Omega \subset \R^d$ beschränkt und sei $\mathcal{L} u = - \dive (A(x) \DO u) + c(x) u$ glm. elliptisch, $c(x) \geq 0$ in $\Omega$, $a_{ij}, c_j \in L^\infty(\Omega)$, $f \in L^2(\Omega)$.
  Dann besitzt (VGL\tss{1})' eine eindeutige Lösung $u \in H_0^1(\Omega)$.
  Außerdem existiert ein $\hat{C} > 0$, sodass $\norm{u}_{H'(\Omega)} \leq \hat{C} \norm{f}_{L^2(\Omega)}$.
\end{satz}

\begin{bemn}
  \begin{itemize}
    \item Sei $f \in H^{-1}(\Omega)$.
    Eine Fktn $u \in H_0^1(\Omega)$ heißt schw. Lsg von (RWP\tss{1}), falls $B(u, \phi) = \scp{f}{\phi}_{H^{-1}(\Omega), H_0'(\Omega)}$ $\forall \, \phi \in H_0'(\Omega)$.
    Gelte $b = 0$, $c \geq 0$, glm. Elliptizität, $c, a_{ij} \in L^\infty(\Omega)$. \\
    Dann ex. nach Lax-Milgram genau ein $u \in H_0^1(\Omega)$ mit \[
      B(u, \phi) = \scp{f}{\phi}_{H^{-1}(\Omega), H_0^1(\Omega)} \quad
      \text{für alle $\phi$.}
    \]
    \item Es existiert ein $\mu_0 > 0$, sodass für alle $\mu > \mu_0$ das RWP
    \[
      \text{(RWP)} \enspace \left\{ \begin{array}{rlll}
        \LL u + \mu u &=& f &\text{ in $\Omega$} \\
        u &=& 0 &\text{ auf $\bOmega$}
      \end{array} \right.
    \]
    für alle $f \in H^{-1}(\Omega)$ eine eindeutige Lsg $u \in H_0^1(\Omega)$ besitzt.
  \end{itemize}
\end{bemn}

% Vorlesung vom 1.12.2015

% §4.4 RWPe mit inhomogenen Dirichlet- und/oder Robin-Randbedingungen

% a)

\begin{prob}
  Wir untersuchen nun das inhomogene Randwertproblem
  \[
    \text{(RWP\tss{2})} \enspace
    \left\{ \begin{array}{r c l l}
      \LL u & = & f & \text{in } \Omega, \\
      u & = & g & \text{auf } \bOmega
    \end{array} \right.
  \]
  Angenommen, $g : \bOmega \to \R$ besitzt eine Fortsetzung $\tilde{g} : \cOmega \to \R$ mit $\tilde{g}|_{\bOmega} = g$.
  Dann ist $u \in \Cont^2(\Omega) \cap \Cont(\cOmega)$ genau dann eine Lösung von (RWP\tss{2}), wenn $v \coloneqq u - \tilde{g} \in \Cont^2(\Omega) \cap \Cont(\cOmega)$ eine Lösung von
  \[
    \text{(RWP\tss{2})'} \enspace
    \left\{ \begin{array}{r c l l}
      \LL v & = & f - \LL \tilde{g} & \text{in } \Omega, \\
      v & = & 0 & \text{auf } \bOmega
    \end{array} \right.
  \]
  ist.
  Schwache Formulierung von (RWP\tss{2})':
  Ges. ist $v \in H_0'(\Omega)$ mit
  \begin{align*}
    & \Int{\Omega}{}{(A(x) \DO v(x) \cdot \DO \phi(x) + b(x) \cdot \DO v(x) \phi(x) + c(x) v(x) \phi(x))}{x} \\
    = \enspace & \Int{\Omega}{}{f \cdot \phi}{x} + \Int{\Omega}{}{(A(x) \DO \tilde{g} \cdot \DO \phi + b \cdot \DO \tilde{g} \phi + c \tilde{g} \phi)}{x} \quad
    \forall \, \phi \in H_0^1(\Omega).
  \end{align*}
  Voraussetzungen: $a_{ij}, b_i, c \in L^\infty(\Omega)$, $f \in L^2(\Omega)$, $\tilde{g} \in \Cont(\cOmega) \cap H^1(\Omega)$. \\
  Gesucht ist also ein $u \in U \coloneqq \Set{w \in H^1(\Omega)}{\tau(w) = g}$ mit
  % Dabei ist $\tau : H^1(\Omega) \to L^2(\bOmega)$ der Spuroperator
  \[
    \underbrace{\Int{\Omega}{}{A(x) \DO u \cdot \DO \phi + b \DO u \phi + c u \phi}{x}}_{B(u, \phi) \coloneqq} =
    \underbrace{\Int{\Omega}{}{f \phi}{x}}_{l(\phi) \coloneqq} \quad
    \forall \, \phi \in H_0^1(\Omega).
    \tag{VGL\tss{2}}
  \]
\end{prob}

\begin{satz}
  Sei $B : H_0^1(\Omega) \times H_0^1(\Omega) \to \R$ beschränkt und koerziv. \\
  Dann besitzt (VGL\tss{2}) genau dann eine eindeutige Lösung $u \in U$, wenn ein $u_0 \in H^1(\Omega)$ existiert, sodass $\tau(u_0) = g$.
\end{satz}

% b)

\begin{prob}
  Wir betrachten nun die Randbedingung
  \[
    \text{(RWP\tss{3})} \enspace
    \left\{ \begin{array}{r c l l}
      \LL u & = & f & \text{in } \Omega, \\
      A(x) \DO u \cdot \nu + \mu u & = & g & \text{auf } \bOmega
    \end{array} \right.
  \]
  Falls $u \in \Cont^2(\Omega) \cap \Cont(\cOmega)$ ein Lösung ist und $\phi \in \Cont^\infty(\cOmega)$, so gilt
  \begin{align*}
    \Int{\Omega}{}{f \phi}{x}
    & = - \Int{\Omega}{}{\dive(A(x) \DO u) \phi + b \DO u \phi + c u \phi}{x} \\
    & = - \Int{\bOmega}{}{A(x) \DO u \cdot \nu \phi}{s} + \Int{\Omega}{}{A(x) \DO u \DO \phi}{x} + \Int{\Omega}{}{b(x) \DO u \phi}{x} + \Int{\Omega}{}{c u \phi}{x}.
  \end{align*}
  Aus der Randbedingung bekommen wir
  \[
    \Int{\partial \Omega}{}{A(x) \DO u \cdot \nu \phi}{s} + \mu \Int{\partial \Omega}{}{u \phi}{s} = \Int{\partial \Omega}{}{g \phi}{s}.
  \]
  Zusammengesetzt erhalten wir die Variationsgleichung
  \[
    \mu \Int{\partial \Omega}{}{u \phi}{s} + \Int{\Omega}{}{A(x) \DO u \DO \phi + b(x) \cdot \DO u \phi + c(x) u \phi}{x} = \Int{\Omega}{}{f \phi}{x} + \Int{\partial \Omega}{}{g \phi}{s}.
  \]
  Wegen Dichtheit gilt diese Gleichung nicht nur für $\phi \in C^\infty(\cOmega)$ sondern allgemeiner für $\phi \in H^1(\Omega)$.
\end{prob}

\begin{defn}
  Sei $\mu \in \R$, $f \in L^2(\Omega)$, $g \in L^2(\partial \Omega)$.
  Eine Fktn $u \in H^1(\Omega)$ heißt \emph{schwache Lösung} von (RWP\tss{3}), falls sie folgende Gleichung erfüllt:
  \[
    \mu \Int{\bOmega}{}{u \phi}{s} + \Int{\Omega}{}{A \DO u \O \phi + b \DO u \phi + c u \phi}{x} =
    \Int{\Omega}{}{f \phi}{x} + \Int{\bOmega}{}{g \phi}{s} \quad
    \forall \, \phi \in H^1(\Omega)
  \]
\end{defn}

\end{document}