\documentclass{cheat-sheet}

\pdfinfo{
  /Title (Zusammenfassung Numerik von partiellen Differentialgleichungen)
  /Author (Tim Baumann)
}

% Kleinere Klammern
\delimiterfactor=701


\begin{document}

\maketitle{Zusammenfassung Numerik von PDEs}

% 1. Einführung

% Ausgelassen: Notationen

% 1.1. Klassifikation von partiellen DGLn

\begin{defn}
  Sei $\Omega \subseteq \R^n$ offen.
  Eine DGL der Form
  \[ F(x, u, Du, \ldots, D^k u) = 0 \]
  heißt \emph{partielle DGL/PDE} der Ordnung $k \geq 1$, wobei
  \[ F : \Omega \times \R \times \R^n \times \ldots \times \R^{n^k} \to \R \]
  eine gegebene Funktion und $u : \Omega \to \R$ gesucht ist.
\end{defn}

\begin{defn}[\emph{Klassifikation von PDEs}]\mbox{}\\
  \begin{itemize}
    \item Die PDE heißt \emph{linear}, wenn sie die Form
    \[ \sum_{\abs{\alpha} \leq k} a_\alpha(x) D^\alpha u = f(x) \]
    mit Funktionen $a_\alpha, f : \Omega \to \R$ besitzt.
    \item Die PDE heißt \emph{semilinear}, wenn sie die Form
    \[ \sum_{\abs{\alpha} = k} a_\alpha(x) D^\alpha u + a_0(x, u, D_u, \ldots, D^{k-1} u) = 0 \]
    besitzt, wobei $a_\alpha : \Omega \to \R$ und $a_0 : \Omega \times \R \times \R^n \times \ldots \times \R^{n^k} \to \R$ gegeben sind.
    \item Die PDE heißt \emph{quasilinear}, wenn sie die Form
    \[ \sum_{\abs{\alpha} = k} a_\alpha(x, u, Du, \ldots, D^{k-1} u) D^\alpha u + a_0(x, u, D_u, \ldots, D^{k-1} u) = 0 \]
    hat, wobei $a_\alpha, a_0 : \Omega \times \R \times \R^n \times \ldots \times \R^{n^k}$ gegeben sind.
    \item Die PDE heißt \emph{nichtlinear}, falls die Ableitungen der höchsten Ordnung nicht linear vorkommen.
  \end{itemize}
\end{defn}

% Ausgelassen: Beispiele: Poisson-, Laplace-, Wärmeleitungs-, Wellengleichung sowie Navier-Stokes-Gleichung

\begin{defn}
  Sei $\Omega \subseteq \R^n$ offen und $F : \Omega \times \R \times \R^n \times \R^{n \times n} \to \R$ eine gegebene Funktion.
  Eine PDE der Form
  \[ F(x, u, \partial_{x_1} u, \ldots, \partial_{x_n} u, \partial_{x_1} \partial_{x_1} u, \ldots, \partial{x_1} \partial_{x_n} u, \ldots, \partial_{x_n} \partial_{x_n} u) = 0 \]
  heißt \emph{PDE zweiter Ordnung}.
\end{defn}

\begin{nota}
  $p_i \coloneqq \partial_{x_i} u$, $p_{ij} \coloneqq \partial^2_{x_i x_j} u$
  \[
    M(x) \coloneqq \begin{pmatrix}
      \tfrac{\partial F}{\partial p_{11}} & \hdots & \tfrac{\partial F}{\partial p_{1n}} \\
      \vdots && \vdots \\
      \tfrac{\partial F}{\partial p_{n1}} & \hdots & \tfrac{\partial F}{\partial p_{nn}}
    \end{pmatrix} = M(x)^{T}.
  \]
\end{nota}

\begin{defn}[\emph{Typeneinteilung für PDEs der 2. Ordnung}]\mbox{}\\
  Obige PDE zweiter Ordnung heißt
  \begin{itemize}
    \item \emph{elliptisch} in $x$, falls die Matrix $M(x)$ positiv o. definit ist.
    \item \emph{parabolisch} in $x$, falls genau ein EW von $M(x)$ gleich null ist und alle anderen dasselbe Vorzeichen haben.
    \item \emph{hyperbolisch} in $x$, falls genau ein EW ein anderes Vorzeichen als die anderen EWe hat.
  \end{itemize}
\end{defn}

% Ausgelassen: Beispiele

\end{document}