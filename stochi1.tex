\documentclass{cheat-sheet}

\pdfinfo{
  /Title (Zusammenfassung Stochastik 1)
  /Author (Tim Baumann)
}

\usepackage{bbm} % Für 1 mit Doppelstrich (Indikatorfunktion)

\newcommand{\Alg}{\mathfrak{A}} % (Mengen-)Algebra
\newcommand{\Ring}{\mathfrak{R}} % (Mengen-)Ring
\newcommand{\LebAlg}{\mathfrak{L}} % Lebesgue-Borel-Mengen
\renewcommand{\P}{\mathbb{P}} % Wahrscheinlichkeitsmaß
\newcommand{\E}{\mathbb{E}} % Elementare Funktion, Erwartungswert
\newcommand{\Bor}{\mathfrak{B}} % Borel
\newcommand{\Leb}{\mathcal{L}} % Lebesgue
\newcommand{\ind}{\mathbbm{1}} % Indikatorfunktion

% TODO: Groß- oder Kleinschreibung?
\newcommand{\Var}{\mathrm{Var}} % Varianz
\newcommand{\var}{\mathrm{Var}} % Varianz
\newcommand{\Cov}{\mathrm{Cov}} % Kovarianz
\newcommand{\cov}{\mathrm{Cov}} % Kovarianz
\newcommand{\Cor}{\mathrm{Cor}} % Korrelation
\newcommand{\cor}{\mathrm{Cor}} % Korrelation

\DeclareMathOperator*{\esssup}{ess\,sup} % Essentielles Supremum
\newcommand{\fue}{\overset{\text{f.ü.}}} % fast überall

\newcommand{\IntO}[2]{\Int{\Omega}{}{#1}{#2}} % Integral über \Omega
\newcommand{\IntOmu}[1]{\Int{\Omega}{}{#1}{\mu}} % Integral über \Omega bzgl. \mu

\usepackage{relsize}
\let\myBinom\binom
\renewcommand{\binom}[2]{\mathsmaller{\myBinom{#1}{#2}}}

\begin{document}

\maketitle{Zusammenfassung Stochastik \rom{1}}

\section{Der abstrakte Maßbegriff}

% TODO: Ereignisalgebra?

\begin{defn}
  Eine \emph{Ereignisalgebra} oder \emph{Boolesche Algebra} ist eine Menge $\mathfrak{A}$ mit zweistelligen Verknüpfungen $\wedge$ (\glqq und\grqq) und $\vee$ (\glqq oder\grqq), einer einstelligen Verknüpfung $\overline{\,\cdot\,}$ (Komplement) und ausgezeichneten Elementen $U \in \mathfrak{A}$ (unmögliches Ereignis) und $S \in \mathfrak{A}$ (sicheres Ereignis), sodass für $A, B, C \in \mathfrak{A}$ gilt:

  \begin{multicols}{2}
    \scriptsize
    \begin{enumerate}[label=\roman*.,leftmargin=2em]
      \item $A \wedge A = A$
      \item $A \wedge B = B \wedge A$
      \item $A \wedge S = A$
      \item $A \wedge U = U$
      \item $A \wedge \overline{A} = U$
      \item $A \wedge (B \wedge C) = (A \wedge B) \wedge C$
      \item $A \vee A = A$
      \item $A \vee S = S$
      \item $A \vee U = A$
      \item $A \vee \overline{A} = S$
      \item $A \vee (B \vee C) = (A \vee B) \vee C$
      \item $A \wedge (B \vee C) = (A \wedge B) \vee (A \wedge C)$
    \end{enumerate}
  \end{multicols}
\end{defn}

\begin{defn}
  Sei $\mathfrak{A}$ eine Boolesche Algebra. Dann definiert
  \[ A \leq B \colon\iff A \wedge B = B \]
  eine Partialordnung auf $\mathfrak{A}$, gesprochen $A$ impliziert $B$.
\end{defn}

\begin{defn}
  Eine \emph{Algebra} (auch Mengenalgebra) $\mathfrak{A} \subset \mathcal{P}(\Omega)$ ist ein System von Teilmengen einer Menge $\Omega$ mit $\emptyset \in \mathfrak{A}$, das unter folgenden Operationen stabil ist:
  \begin{itemize}
    \item Vereinigung: $A, B \in \mathfrak{A} \implies A \cup B \in \mathfrak{A}$
    \item Durchschnitt: $A, B \in \mathfrak{A} \implies A \cap B \in \mathfrak{A}$
    \item Komplementbildung: $A \in \mathfrak{A} \implies A^c \coloneqq \Omega \backslash A \in \mathfrak{A}$
  \end{itemize}
\end{defn}

\begin{satz}[Isomorphiesatz von Stone]
Zu jeder Booleschen Algebra $\mathfrak{A}$ gibt es eine Menge $\Omega$ derart, dass $\mathfrak{A}$ isomorph zu einer Mengenalgebra $\mathfrak{A}$ in $\mathcal{P}(\Omega)$ ist.
\end{satz}

\begin{defn}
  Eine \emph{$\sigma$-Algebra} ist eine Algebra $\mathfrak{A} \subset \mathcal{P}(\Omega)$, die nicht nur unter endlichen, sondern sogar unter abzählbaren Vereinigungen stabil ist, d.\,h.

  \[ (A_n)_{n \in \N} \text{ Folge in } \mathfrak{A} \implies \bigcup_{n = 0}^{\infty} A_n \in \mathfrak{A}. \]
\end{defn}

\begin{bem}
  Es gilt damit:

  \begin{itemize}
    \item $\Omega = \emptyset^c \in \mathfrak{A}$
    \item Abgeschlossenheit unter abzählbaren Schnitten:
  \[ (A_n)_{n \in \N} \text{ Folge in } \mathfrak{A} \implies \bigcap_{n = 0}^{\infty} A_n = \left( \bigcup_{n = 0}^{\infty} (A_n)^c \right)^c \in \mathfrak{A}. \]
  \end{itemize}
\end{bem}

\begin{defn}
  Sei $(A_n)_{n \in \N}$ eine Folge in einer $\sigma$-Algebra $\mathfrak{A}$. Setze
  \[
    \limsup_{n \to \infty} A_n \coloneqq \bigcap_{n = 1}^{\infty} \bigcup_{m = n}^{\infty} A_n \in \mathfrak{A}, \quad
    \liminf_{n \to \infty} A_n \coloneqq \bigcup_{n = 1}^{\infty} \bigcap_{m = n}^{\infty} A_n \in \mathfrak{A}.
  \]
\end{defn}

\begin{bem}
  In einer $\sigma$-Algebra, in der die Mengen mögliche Ereignisse beschreiben, ist der Limes Superior das Ereignis, das eintritt, wenn unendlich viele Ereignisse der Folge $A_n$ eintreten. Der Limes Infinum tritt genau dann ein, wenn alle bis auf endlich viele Ereignisse der Folge $A_n$ eintreten.
\end{bem}

\begin{defn}
  Ein \emph{Ring} $\mathfrak{A} \subset \mathcal{P}(\Omega)$ ist ein System von Teilmengen einer Menge $\Omega$ mit $\emptyset \in \mathfrak{A}$, das unter folgenden Operation stabil ist:

  \begin{itemize}
    \item Vereinigung: $A, B \in \mathfrak{A} \implies A \cup B \in \mathfrak{A}$
    \item Differenz: $A, B \in \mathfrak{A} \implies B \backslash A = B \cap A^c \in \mathfrak{A}$
  \end{itemize}

  Ein Ring, der nicht nur unter endlicher, sondern sogar unter abzählbarer Vereinigung stabil ist, heißt \emph{$\sigma$-Ring}.
\end{defn}

\begin{bem}
  $\mathfrak{A}$ ($\sigma$-)\,Algebra $\iff$ $\mathfrak{A}$ ($\sigma$-)\,Ring und $\Omega \in \mathfrak{A}$.
\end{bem}

\begin{satz}
  Sei $(\mathfrak{A}_i)_{(i \in I)}$ eine Familie von ($\sigma$-)\,Ringen / ($\sigma$-)\,Algebren über einer Menge $\Omega$. Dann ist auch $\cup_{i \in I} \mathfrak{A}_i$ ein ($\sigma$-)\,Ring / eine ($\sigma$-)\,Algebra über $\Omega$.
\end{satz}

% TODO: Definition Inhalt

\begin{satz}
  Sei $\Ring$ ein Ring und $\mu$ ein Inhalt. Es gelten für $n \in \N$ und $A_1, ..., A_n \in \Ring$ die Ein- und Ausschlussformeln
  \begin{align*}
    \mu(A_1 \cup ... \cup A_n) &= \sum_{k=1}^n (-1)^{k-1} \quad \sum_{\mathclap{1 \leq i_1 < ... < i_k \leq n}} \quad \mu(A_{i_1} \cap ... \cap A_{i_k}), \\
    \mu(A_1 \cap ... \cap A_n) &= \sum_{k=1}^n (-1)^{k-1} \quad \sum_{\mathclap{1 \leq i_1 < ... < i_k \leq n}} \quad \mu(A_{i_1} \cup ... \cup A_{i_k}).
  \end{align*}
\end{satz}





% Anfang strukturierte sortierte Einträge



% Vorlesung 9

\begin{bem}
  Sei $\mu$ ein W-Maß auf $\LebAlg(\R^1)$. Dann definiert $x \mapsto F_{\mu}(x) \coloneqq \mu(\left] -\infty, x \right])$ eine VF. Für eine VF $F : \R \to [0,1]$ definiert umgekehrt $\mu_F(\left] a, b \right]) \coloneqq F(b) - F(a)$ ein W-Maß auf $\LebAlg(\R^1)$. Analog funktioniert dies auf dem $\R^d$.
\end{bem}

% Wichtige Verteilungsfunktionen:

\begin{defn}[Wichtige Verteilungsfunktionen]\mbox{}\\
  \begin{itemize}
    \item \emph{Normalverteilung} (Gaußverteilung) mit EW $\mu$ und Varianz $\sigma^2$:
    \[ F_{\mu \sigma^2}(x) = \tfrac{1}{\sqrt{2 \pi \sigma^2}} \Int{-\infty}{x}{\exp\left(\tfrac{-(t-\mu)^2}{2 \sigma^2}\right)}{t} \]
    erfüllt $F_{\mu \sigma^2}'(x) = \exp\left(\tfrac{-(x-\mu)^2}{2 \sigma^2}\right)$, $F_{\mu \sigma^2}(\mu-x) = 1 - F_{\mu \sigma^2}(\mu+x)$
    \item \emph{Exponentialverteilung} mit Parameter $\lambda > 0$:
    \[ F_{\lambda}(x) = \begin{cases} 0, & \text{ für } x \leq 0 \\
                                      1 - \exp(- \lambda x), & \text{ für } x > 0
    \end{cases} \]
    \item \emph{Poisson-Verteilung} mit Parameter $\lambda > 0$:
    \[ F_{\lambda}(x) = \sum_{0 \leq n \leq x} \tfrac{\lambda^n}{n!} \exp(-\lambda) \]
  \end{itemize}
\end{defn}

% Ausgelassen: Mehrdimensionale Normalverteilung, zweidimensionale Exponentialverteilung
% Ausgelassen: Gauß-Levy-Kuzmin-Theorem, und zugehörige Verteilung
% Ausgelassen: Wiener Maß für Modellierung von Brown'scher Bewegung

% Kapitel 2.

\section{Elementare Wahrscheinlichkeitsrechnung}

% Kapitel 2.1. Definition der relativen und bedingten relativen Häufigkeit

\begin{defn}
  Ein Ereignis $A \in \Alg$ trete bei $n$ Versuchen genau $h_n(A) \in \N$ mal auf. Dann heißt
  \begin{itemize}
    \item $h_n(A)$ \emph{absolute Häufigkeit} von $A$,
    \item $H_n(A) \coloneqq \tfrac{h_n(A)}{n}$ \emph{relative Häufigkeit} von $A$.
  \end{itemize}
\end{defn}

\begin{bem}
  Unmittelbar klar:
  \begin{itemize}
    \miniitem{0.35\linewidth}{$H_n(A) \in [0,1]$}
    \miniitem{0.60\linewidth}{$H_n(A) \leq H_n(B)$ für $A \subset B$}
    \item $H_n(A \sqcup B) = H_n(A) + H_n(B)$ für $A \cap B = \emptyset$
  \end{itemize}
\end{bem}

\begin{bem}
  Bei wachsendem $n$ stabilisiert sich normalerweise der Wert $H_n(A)$. Dieser Grenzwert ist die Wahrscheinlichkeit von $A$.
\end{bem}


% Vorlesung 10

\begin{defn}
  Seien $A, B \in \Alg$ Ereignisse, $n \in \N$ die Anzahl der Versuche. Dann heißt
  \[ H_n(A \mid B) \coloneqq \tfrac{H_n(A \cap B)}{H_n(B)} = \tfrac{h_n(A \cap B)}{h_n(B)} \]
  die \emph{relative Wahrscheinlichkeit} von $A$ unter der Bedingung $B$.
\end{defn}

\begin{bem}
  Offenbar gilt:
  \begin{itemize}
    \miniitem{0.35\linewidth}{$H_n(A \mid B) \in [0,1]$}
    \miniitem{0.60\linewidth}{$H_n(A_1 \mid B) \leq H_n(A_2 \mid B)$ für $A_1 \subset A_2$}
    \item $H_n(A_1 \sqcup A_2 \mid B) = H_n(A_1 \mid B) + H_n(A_2 \mid B)$ für $A_1 \cap A_2 = \emptyset$
  \end{itemize}
\end{bem}

% Kapitel 2.2. Geometrische Wahrscheinlichkeiten

\begin{defn}
  Sei $\Omega \in \LebAlg(\R^d)$ mit $\lambda_d(\Omega) > 0$. Dann heißt das W-Maß
  \[ \P : \LebAlg(\Omega) \to [0, 1], \quad A \mapsto \tfrac{\lambda_d(A)}{\lambda_d(\Omega)} \quad \text{\emph{Gleichverteilung}.} \]
\end{defn}

\begin{defn}
  Sei $\Omega$ eine endliche Menge. Dann definiert
  \[ \P : \mathcal{P} \to [0, 1], \quad A \mapsto \tfrac{|A|}{|\Omega|} = \tfrac{\text{\# günstige Fälle}}{\text{\# mögliche Fälle}} \]
  ein W-Maß auf $(\Omega, \mathcal{P}(\Omega))$, genannt \emph{Laplace'sche Wkt}.
\end{defn}

\begin{bem}
  Damit sind Berechnungen von Wkten mit kombinatorischen Überlegungen möglich.
\end{bem}

% Kapitel 2.4. Berechnung von Wahrscheinlichkeiten durch kombinatorische Überlegungen

\begin{lem}[Fundamentalprinzip des Zählens]
  Seien $A_1, ..., A_n$ endliche Mengen. Dann gilt $| A_1 \times ... \times A_n | = |A_1| \cdots |A_n|$.
\end{lem}

\begin{lem}
  Sei $A$ eine endliche Menge, $r \leq n \coloneqq |A| < \infty$. Dann ist die Anzahl der $r$-Tupel mit Elementen aus $A$ gleich

  \begin{center}
    \begin{tabular}{ r | l l }
      & Mit Wdh. & Ohne Wdh. \\ \hline
      Mit Ordnung & $n^r$ & $\tfrac{n!}{(n-r)!}$ \\
      Ohne Ordnung & $\tfrac{(n+r-1)!}{r!}$ & $\binom{n}{r} \coloneqq \tfrac{n!}{r!(n-r)!}$
    \end{tabular}
  \end{center}
\end{lem}


% Vorlesung 11

\begin{lem}
  Sei $A$ eine endliche Menge, $n \coloneqq |A| < \infty$. Dann ist die Anzahl der möglichen Zerlegungen von $A$ in disjunkte Mengen $B_1, ..., B_k$ mit $|B_i| = n_i$ und $n_1 + ... + n_k = n$ gleich
  \[
    \binom{n}{n_1, ..., n_k} \coloneqq \tfrac{n!}{n_1! \cdots n_k!}. \quad \text{\emph{(Multinomialkoeffizient)}}
  \]
\end{lem}

% Ausgelassen: Beispiel Kartenverteilungen beim Skatspiel

% Kapitel 2.5. Berechnung von geometrischen Wahrscheinlichkeiten

% Ausgelassen: Beispiel Bertrand'sches Paradoxon (http://en.wikipedia.org/wiki/Bertrand_paradox_(probability))

% Ausgelassen: Beispiel Buffonsches Nadelproblem (http://de.wikipedia.org/wiki/Buffonsches_Nadelproblem)
% und dessen Verallgemeinerung auf Polygone


% Vorlesung 12

% Kapitel 2.6. Hypergeometrische Verteilung

\begin{modell}
  Eine Urne enthalte $N$ Kugeln, darunter $M \leq N$ schwarze. Dann ist ist die Wkt für das Ereignis $A^n_m$, dass sich unter $n$ gezogenen Kugeln genau $m \leq \min(n, M)$ schwarze Kugeln befinden,
  \[ \P(A^n_m) = \frac{\binom{M}{m} \binom{N - M}{n - m}}{\binom{N}{n}}. \quad \text{\emph{(hypergeometrische Verteilung)}} \]
\end{modell}

% Ausgelassen: Beispiel: $m$-er beim Lotto
% Ausgelassen: Beispiel: Capture-Recapture-Problem, Maximum-Likelihood-Schätzung

\begin{bem}
  Für Maximum-Likelihood-Schätzungen:
  \begin{itemize}
    \item Der Ausdruck $\binom{N-M}{n-m} / \binom{N}{n}$ wird maximal bei $N \coloneqq \lfloor \tfrac{n-M}{m} \rfloor$.
    \item Der Ausdruck $\binom{M}{m} \cdot \binom{N-M}{n-m}$ wird maximal bei $M \coloneqq \lfloor \tfrac{m (N-1)}{n} \rfloor$.
  \end{itemize}
\end{bem}

% Ausgelassen: Beispiel statistische Qualitätskontrolle

\begin{modell}
  Eine Urne enthalte $N$ Kugeln in $k \leq N$ verschiedenen Farben, darunter $N_1$ in der ersten Farbe, ..., $N_k$ in der $k$-ten Farbe, $N_1 + ... + N_k = N$. Dann ist ist die Wkt für das Ereignis $A^n_{n_1,...,n_k}$, dass sich unter $n$ gezogenen Kugeln genau $n_1 \leq N_1$ Kugeln der ersten Farbe, ..., und $n_k \leq N_k$ Kugeln der $k$-ten Farbe befinden, $n_1 + ... + n_k = n$, gleich
  \[ \P(A^n_{n_1, ..., n_k}) = \frac{\binom{N_1}{n_1} \cdots \binom{N_k}{n_k}}{\binom{N}{n}}. \]
  Diese W-Verteilung heißt \emph{polyhypergeometrische Verteilung}.
\end{modell}

% Ausgelassen: Zusatz-5er beim Lotto

% Kapitel 2.7. Bedingte Wahrscheinlichkeiten

\begin{defn}
  Sei $(\Omega, \Alg, \P)$ ein W-Raum und $A, B \in \Alg$. Dann heißt
  \[ \P(A \mid B) \coloneqq \begin{cases} \tfrac{\P(A \cap B)}{\P(B)}, & \text{ falls } \P(B) > 0 \\
  0, & \text{ falls } \P(B) = 0 \end{cases} \]
  Wahrscheinlichkeit von $A$ unter der Bedingung $B$.
\end{defn}

\begin{bem}
  Falls $\P(B) > 0$ gilt, so ist $\P(- \mid B)$ ein W-Maß über $B$ auf der Spur-$\sigma$-Algebra $\Alg|_B$.
\end{bem}

\begin{lem}
  Seien $A_1, ..., A_k \in \Alg$, dann gilt die Pfadregel:
  \[ \P(A_1 \cap ... \cap A_k) = \P(A_1) \cdot \prod_{i=2}^k \P(A_i \mid A_1 \cap ... \cap A_{i-1}). \]
\end{lem}

% Ausgelassen: Beispiel: Lose ziehen

% Kapitel 2.8. Formel der totalen Wahrscheinlichkeit und Bayes'sche Formel


% Vorlesung 13

\begin{satz}
  Sei $(\Omega, \Alg, \P)$ ein W-Raum und $A_1, ... \in \Alg$ ein vollständiges Ereignissystem, d.\,h. paarweise disjunkt mit
  \[ \Omega = \bigsqcup_{i = 1}^\infty A_i. \]
  Dann gilt für jedes $B \in \Alg$ mit $\P(B) > 0$
  \begin{align*}
    \P(B) &= \sum_{i=1}^{\infty} \P(B \mid A_i) \cdot \P(A_i) & \text{(Formel der totalen Wkt)} \\
    \P(A_n \mid B) &= \frac{\P(B \mid A_n) \cdot \P(A_n)}{\sum_{i=1}^\infty \P(B \mid A_i) \cdot \P(A_i)} & \text{(\emph{Bayessche Formel})}
  \end{align*}
\end{satz}

% Ausgelassen: Interpretation als "Mischformel"

\begin{sprech}
  In der Bayesischen Statistik heißt
  \begin{itemize}
    \item $\P(A_i)$ \quad \,\, \emph{A-priori-Wahrscheinlichkeit},
    \item $\P(A_i \mid B)$ \emph{A-posteriori-Wahrscheinlichkeit}.
  \end{itemize}
\end{sprech}

% Ausgelassen: Beispiel der Produktion auf zwei Maschinen mit untersch. Ausschussraten

% Kapitel 2.9. Unabhängigkeit von Ereignissen

\begin{defn}
  Zwei Ereignisse $A, B \in \Alg$ heißen \emph{($\P$-)unabhängig}, falls
  \[ \P(A \cap B) = \P(A) \cdot \P(B). \]
\end{defn}

\begin{bem}
  \begin{itemize}
    \item $A \in \Alg$ mit $\P(A) = 0$ ist unabhängig zu jedem $B \in \Alg$.
    \item Wenn $A, B \in \Alg$ unabhängig, dann sind auch unabhängig:
    \[ (A^c, B), \quad (A, B^c), \quad (A^c, B^c) \]
  \end{itemize}
\end{bem}

\begin{satz}
  $A, B \in \Alg$ unabhängig $\iff$ $\P{B \mid A} = \P(B)$.
\end{satz}

% Ausgelassen: Beispiel: Ziehen von Karo-, Ass- und Karo-Ass-Karten

\begin{defn}
  Sei $(A_i)_{i \in I}$ ($I$ bel.) eine Familie von Ereignissen in $\Alg$.
  \begin{itemize}
    \item \emph{vollständig unabhhängig}, falls
    \[ \P(A_{i_1} \cap A_{i_2} \cap ... \cap A_{i_m}) = \P(A_{i_1}) \cdot \P(A_{i_2}) \cdots \P(A_{i_n}) \]
    für alle $i_1, ..., i_n \in I$ mit $2 \leq n < \infty$ und
    \item \emph{paarweise unabhängig}, falls
    \[ \P(A_i \cap A_j) = \P(A_i) \cdot \P(A_j) \quad \text{ für alle } i, j \in I, i \not= j. \]
  \end{itemize}
\end{defn}

\begin{acht}
  Aus paarweiser Unabhängigkeit folgt nicht vollständige Unabhängigkeit (Gegenbeispiel: Bernsteins Tetraeder).
\end{acht}

\begin{defn}
  Sei $(\Omega, \Alg, \P)$ ein W-Raum und $\Alg_1, \Alg_2 \subset \Alg$ Ereignissysteme. Dann heißen $\Alg_1$ und $\Alg_2$ \emph{unabhängig}, falls
  \[ \P(A_1 \cap A_2) = \P(A_1) \cdot \P(A_2) \quad \text{für alle } A_1 \in \Alg_1, A_2 \in \Alg_2. \]
\end{defn}

\begin{satz}
  Seien $\Alg_1, \Alg_2 \subset \Alg$ unabhängige Ereignissysteme, die Algebren sind. Dann sind auch die $\sigma$-Algebren $\sigma(\Alg_1)$ und $\sigma(\Alg_2)$ unabhängig.
\end{satz}

% Ausgelassen: Anwendung in der Zuverlässigkeitsanalyse (Reihenschaltung von Parallelschaltungen)


% Vorlesung 14

% Kapitel 2.10. Bernoulli-Schema

\begin{satz}
  Sei $(\Omega, \Alg, \P)$ ein W-Raum, $(A_i)_{i \in \N}$ Folge von unabhängigen Ereignissen mit gleicher Erfolgswkt $\P(A_i) = p$ für alle $i \in \N$. Für $k \leq n$, $k, n \in \N$ ist dann die Wahrscheinlichkeit, dass genau $k$ Stück der Ereignisse $A_1, ..., A_n$ eintreten, genau
  \[ B(k, n, p) \coloneqq \binom{n}{k} \, p^k \, (1-p)^{n-k} \]
  Die zugehörige VF $x \mapsto \sum_{\mathclap{0 \leq k \leq x}} B(k,n,p)$ heißt \emph{Binomialverteilung}.
\end{satz}

% Ausgelassen: Beispiel: Rosinen werden in einen Teig gemengt

\begin{lem}
  Voraussetzung wir im vorherigen Satz. Sei $r, k \in \N$, $1 \leq r$, dann ist die Wkt für das Ereignis $A_k^{(r)}$, dass beim Versuch $A_{k+r}$ der $r$-te Erfolg eintritt, gleich
  \[ \P(A_k^{(r)}) = \binom{k+r-1}{r-1} \, p^r \, (1-p)^k. \]
  Im Spezialfall $r = 1$ ist $\P(A_k^{(1)}) = p \, (1-p)^k$.
\end{lem}

\begin{satz}
  Sei $(\Omega, \Alg, \P)$ ein W-Raum, $A_1, ..., A_r \in \Alg$ mit $p_i \coloneqq \P(A_i)$ für $i = 1, ..., k$ und $p_1 + ... + p_r = 1$. Dann ist die Wahrscheinlichkeit, dass bei $n \in \N$ Versuchen $A_1$ genau $n_1$-mal, $A_2$ genau $n_2$-mal, ..., $A_r$ genau $n_r$-mal auftritt ($n_1 + ... + n_r = n$), genau
  \[ B(n_1, ..., n_r, n, p_1, ..., p_r) \coloneqq \binom{n}{n_1, ..., n_r} p_1^{n_1} \cdots p_r^{n_r}. \]
  Diese W-Verteilung heißt \emph{Multinomialverteilung}.
\end{satz}

% Kapitel 2.11. Grenzwertsatz von Poisson -- Gesetz der kleinen Zahlen

\begin{satz}
  Für $0 \leq m \leq n$, $p \in [0, 1]$ gilt
  \[ \frac{\binom{M}{m} \binom{N-M}{n-m}}{\binom{N}{n}} \enspace \xrightarrow[M/N \to p]{M, N \to \infty} \enspace \binom{n}{m} \, p^m \, (1-p)^{n-m}. \]
\end{satz}

% Intuitive Interpretation: Ein Urnenmodell mit vielen Kugeln können wir auch als Bernoulli-Experiment auffassen, ohne uns einen zu großen Fehler einzuhandeln

\begin{satz}[GWS von Poisson]
  Für $k \in \N$, $\lambda \in \R_{>0}$ gilt
  \[ \binom{n}{m} \, p_n^m \, (1-p_n)^{n-m} \enspace \xrightarrow[n p_n \to \lambda]{n \to \infty} \enspace \frac{\lambda^k}{k!} \exp(-\lambda). \]
\end{satz}


% Vorlesung 15

\begin{defn}
  Sei $(\Omega, \Alg)$ ein messbarer Raum mit zwei W-Maßen $\P_1$ und $\P_2$. Dann heißt
  \[ \mathrm{d}_\infty(\P_1, \P_2) \coloneqq \sup_{A \in \Alg} \abs{\P_1(A) - \P_2(A)} \]
  \emph{Totalvariation} des signierten Maßes $\P_1 - \P_2$.
\end{defn}

\begin{satz}
  Seien $\P_1$ und $\P_2$ zwei W-Maße auf $(\N, \mathcal{P}(\N))$, $\P_1(\{ i \}) = p_i$, $\P_2(\{ i \}) = q_i$ für alle $i \in \N$. Dann gilt
  \[ \mathrm{d}_\infty(\P_1, \P_2) = \tfrac{1}{2} \sum_{i=0}^\infty \abs{p_i - q_i}. \]
\end{satz}

\begin{lem}
  Für $n, k \in \N$, $p \in \left[0,1\right]$ und $\P_1$ und $\P_2$ wie eben definiert durch $p_i \coloneqq \binom{n}{k} p^k (1-p)^{n-k}$, $q_i \coloneqq \tfrac{(np)^k}{k!} \exp(-np)$ gilt
  \[ \mathrm{d}_\infty(\P_1, \P_2) \leq 2 n p^2. \]
\end{lem}

% Kapitel 2.10. 0-1-Gesetze von Borel-Cantelli und Kolmogorow

\begin{lem}[Borel-Cantelli]
  Sei $(A_n)_{n \in \N}$ eine Folge von Ereignissen über $(\Omega, \Alg, \P)$. Dann gilt für $A = \limsup_{n \to \infty} A_n$
  \[ \sum_{n=1}^\infty \P(A_n) < \infty \enspace \implies \enspace \P(A) = 0. \]
  Falls die Ereignisse $(A_n)_{n \in \N}$ unabhängig sind, so gilt
  \[ \sum_{n=1}^\infty \P(A_n) = \infty \enspace \implies \enspace \P(A) = 1, \]
  also zusammengefasst $\P(A) \in \{ 0, 1 \}$.
\end{lem}

\begin{defn}
  Sei $(\Alg_n)_{n \in \N}$ Folge von $\sigma$-Algebren über $\Omega$. Dann ist
  \[ \mathcal{T}_\infty = \bigcap_{n=1}^\infty \mathcal{T}_n \quad \text{mit} \quad \mathcal{T}_n \coloneqq \sigma \left( \bigcup_{k=n}^\infty \Alg_k \right) \]
  die \emph{terminale $\sigma$-Algebra} von $(\Alg_n)_{n \in \N}$.
\end{defn}

% Was bedeutet Unabhängigkeit von $\sigma$-Algebren?

\begin{satz}[Null-Eins-Gesetz von Kolmogorow]
  Sei $(\Alg_n)_{n \in \N}$ eine Folge von unabhängigen Unter-$\sigma$-Algebren in einem W-Raum $(\Omega, \Alg, \P)$. Dann gilt $\P(A) \in \{ 0, 1 \}$ für alle Ereignisse $A \in \mathcal{T}_\infty$ der terminalen $\sigma$-Algebra.
\end{satz}

% Kapitel 3. Integrationstheorie

\section{Integrationstheorie}

% Kapitel 3.1. Messbare Abbildungen und Zufallsgrößen

\begin{defn}
  Seien $(\Omega_1, \Alg_1)$ und $(\Omega_2, \Alg_2)$ messbare Räume. Dann heißt $f : \Omega_1 \to \Omega_2$ \emph{$(\Alg_1, \Alg_2)$-messbar}, falls
  \[ f^{-1}(A_2) \in \Alg_1 \quad \text{für alle } A_2 \in \Alg_2. \]
\end{defn}

\begin{nota}
  Für solches $f$ schreiben wir $f : (\Omega_1, \Alg_1) \to (\Omega_2, \Alg_2)$.
\end{nota}

\begin{nota}
  Sei $\Omega$ eine Menge, $A \subset \Omega$. Dann heißt
  \[ \chi_1 = \ind_A : \Omega \to \R, \quad \omega \mapsto \abs{\Set{ \star }{ \omega \in A }} =
  \begin{cases}
    1, & \text{falls } \omega \in A \\
    0, & \text{falls } \omega \not\in A
  \end{cases} \]
  \emph{Indikatorfunktion} von $A$.
\end{nota}

\begin{beobachtung}
  Sei $(\Omega, \Alg)$ messbarer Raum, $A \subset \Omega$, dann gilt
  \[ \ind_A \text{ $(\Alg, \LebAlg(\R^1))$-messbar} \enspace \iff \enspace A \in \Alg. \]
\end{beobachtung}


% Vorlesung 16

\begin{lem}
  Die Verkettung messbarer Abbildungen ist messbar, d.\,h. für $f : (\Omega_1, \Alg_1) \to (\Omega_2, \Alg_2)$ und $g : (\Omega_2, \Alg_2) \to (\Omega_3, \Alg_3)$ gilt $g \circ f : (\Omega_1, \Alg_1) \to (\Omega_3, \Alg_3)$.
\end{lem}

\begin{lem}
  Sei $f : \Omega \to \Omega'$ eine Abb. und $\mathcal{E}' \subset \mathcal{P}(\Omega)$, dann ist
  \[ \Alg(f^{-1}(\mathcal{E}')) = f^{-1}(\Alg(\mathcal{E}')). \]
\end{lem}

\begin{lem}
  Sei $(\Omega, \Alg)$ ein messbarer Raum und $f : \Omega \to \Omega'$ eine Abbildung, sowie $\mathcal{E} \subset \P(\Omega')$. Dann gilt
  \[ f \text{ ist $(\Alg, \sigma(\mathcal{E}))$-messbar} \enspace \iff \enspace f^{-1}(E) \in \Alg \text{ für alle } E \in \mathcal{E}. \]
\end{lem}

\begin{nota}
  Seien $f, g : \Omega \to \ER$ zwei numerische Funktionen. Setze
    \[ \{ f \leq g \} \coloneqq \Set{ \omega \in \Omega }{ f(\omega) \leq g(\omega) } \subset \Omega \]
  und definiere analog $\{ f < g \}$, $\{ f \geq g \}$, $\{ f > g \}$, $\{ f = g \}$, $\{ f \not= g \}$.
\end{nota}

\begin{satz}
  Für eine numerische Fkt. $f : (\Omega, \Alg) \to (\ER, \overline{\Bor})$ sind äquivalent:
  \begin{itemize}
    \miniitem{0.27 \linewidth}{$f$ ist messbar}
    \miniitem{0.7 \linewidth}{$\forall \, a \in \R \,:\, \{ f \geq a \} = f^{-1}([a, \infty]) \in \Alg$}
    \begin{multicols}{2}
      \item $\fa{a \in \R} \{ f > a \} \in \Alg$
      \item $\fa{a \in \R} \{ f \leq a \} \in \Alg$
    \end{multicols}
    \item $\fa{a \in \R} \{ f < a \} \in \Alg$
  \end{itemize}
\end{satz}

\begin{defn}
  \begin{itemize}
    \item Sei $(\Omega, \Alg)$ ein messbarer Raum, $f : \Omega' \to \Omega$ eine Abbildung, dann heißt
    \[ \sigma(f) \coloneqq f^{-1}(\Alg) \coloneqq \Set{ f^{-1}(A) }{ A \in \Alg } \]
    die \emph{von $f$ erzeugte $\sigma$-Algebra}.
    \item Sei $(\Omega_i, \Alg_i)_{i \in I}$ eine Familie von messbaren Räumen, $f_i : \Omega' \to \Omega_i$ für alle $i \in I$ eine Abbildung. Dann heißt
    \[ \sigma((f_i)_{i \in I}) \coloneqq \sigma(\bigcup_{i \in I} \sigma(f_i)) = \sigma(\bigcup_{i \in I} f_i^{-1}(\Alg_i)) \]
    die \emph{von der Familie $(f_i)_{i \in I}$ erzeugte $\sigma$-Algebra}.
  \end{itemize}
\end{defn}

\begin{defn}
  Sei $(\Omega', \Alg', \mu')$ ein Maßraum, $(\Omega, \Alg)$ ein messbarer Raum, $f : (\Omega', \Alg) \to (\Omega, \Alg)$. Dann ist durch
  \[ \mu'_f \coloneqq \mu' \circ f^{-1} : \Alg \to [0, \infty], \quad A \mapsto \mu'(f^{-1}(A)) \]
  ein Maß auf $(\Omega, \Alg)$, das sog. \emph{Bildmaß} von $\mu'$ unter $f$, definiert.
\end{defn}

\begin{satz}
  Für zwei numerische Funktionen $f, g : (\Omega, \Alg) \to (\ER, \overline{\Bor})$ gilt:
  \begin{multicols}{3}
    \begin{itemize}
      \item $\{ f < g \} \in \Alg$
      \item $\{ f \leq g \} \in \Alg$
      \item $\{ f > g \} \in \Alg$
      \item $\{ f \geq g \} \in \Alg$
      \item $\{ f = g \} \in \Alg$
      \item $\{ f \not= g \} \in \Alg$
    \end{itemize}
  \end{multicols}
\end{satz}

\begin{satz}
  Seien $f, g : (\Omega, \Alg) \to (\ER, \overline{\Bor})$ messbare numerische Funktionen und $\lambda, \mu \in \R$. Dann auch messbar (\ddag: falls $0 \not\in \mathrm{Bild}(f)$):
  \begin{multicols}{5}
    \begin{itemize}
      \item $\lambda \cdot f$
      \item $f + \mu \cdot g$
      \item $f \cdot g$
      \item $\tfrac{1}{f}$ (\ddag)
      \item $\tfrac{g}{f}$ (\ddag)
    \end{itemize}
  \end{multicols}
\end{satz}

\begin{satz}
  Seien $f_n : (\Omega, \Alg) \to (\ER, \overline{\Bor}), n \in \N$ messbare numerische Funktionen, dann auch messbar:
  \begin{multicols}{4}
    \begin{itemize}
      \item $\sup_{n \in \N} f_n$
      \item $\inf_{n \in \N} f_n$
      \item $\liminf_{n \in \N} f_n$
      \item $\limsup_{n \in \N} f_n$
    \end{itemize}
  \end{multicols}
  \vspace{4pt}
  Dabei werden Infimum, Supremum, usw. punktweise gebildet.
\end{satz}

\begin{defn}
  Für $f : \Omega \to \ER$ heißen die Funktionen
  \begin{itemize}
    \item $\left|f\right| \coloneqq \max(f, -f) : \Omega \to [0, \infty]$ \emph{Betrag} von $f$
    \item $f^+ \coloneqq \,\,\,\, \max(f, 0) : \Omega \to [0, \infty]$ \emph{Positivteil} von $f$
    \item $f^- \coloneqq -\min(f, 0) : \Omega \to [0, \infty]$ \emph{Negativteil} von $f$
  \end{itemize}
\end{defn}

\begin{satz}
  Falls $f : (\Omega, \Alg) \to (\ER, \overline\Bor)$ messbar, dann auch $\left|f\right|$, $f^+$ und $f^-$.
\end{satz}


% Vorlesung 17

\begin{satz}
  \begin{itemize}
    \item Sei $(\Omega, \mathcal{O})$ ein topologischer Raum und $f : \Omega \to \R^n$ stetig. Dann ist $f$ $(\sigma(\mathcal{O}), \Bor(\R^n))$-messbar. % und $(\LebAlg(\mathcal{O}), \LebAlg(\R^n))$-messbar.
    \item $\sigma(\mathcal{O})$ ist die kleinste $\sigma$-Algebra, bezüglich der alle stetigen Funktionen $f : \Omega \to \R^n$ Borel-messbar sind.
  \end{itemize}
\end{satz}

\begin{satz}[von Lusin]
  Sei $M \in \LebAlg(\R^n)$ mit $\lambda_n(M) < \infty$ und $f : M \to \R$ beschränkt. Dann ist $f$ genau dann Borel-messbar, wenn gilt:
  \[ \fa{\epsilon > 0} \ex{K_\epsilon \subset M \text{ kompakt}} \lambda_n(M \setminus K_\epsilon) < \epsilon \text{ und } f|_{K_\epsilon} \text{ stetig}. \]
\end{satz}

\begin{defn}
  Eine Funktion $f : \R \to \R$ heißt \emph{Càdlàg-Funktion} (continue à droite, limite à gauche), falls für alle $x \in \R$ gilt:
  \[ \lim_{y \uparrow x} f(y) \text{ existiert} \quad \text{und} \quad \lim_{y \downarrow x} f(y) = f(x). \]
\end{defn}

\begin{beobachtung}
  Jede kumulierte VF ist eine Càdlàg-Funktion.
\end{beobachtung}

\begin{defn}
Die \emph{Variation} von $g : [a, b] \to \R$ bzgl. einer Zerlegung $Z = \{ a = x_0 < ... < x_n = b \}$ von $[a, b]$ ist die nicht-negative Zahl
\[ V(g, Z) \coloneqq \sum_{j=1}^{n} |g(x_j) - g(x_{j-1})|. \]
Die \emph{Totalvariation} von $g : [a, b] \to \R$ ist
\[ V_a^b(g) \coloneqq \sup\left\{ V(g, Z) : Z \text{ Zerlegung von } [a, b] \right\} \in \R_{\ge 0} \cup \{ \infty \}. \]
Falls $V_a^b(g) < \infty$, so heißt $g$ \emph{von beschränkter Variation}.
\end{defn}

\begin{satz}
  Es sind messbar:
  \begin{itemize}
    \begin{multicols}{2}
      \item Monotone Funktionen
      \item Càdlàg-Funktionen
    \end{multicols}
    \item Funktionen von beschränkter Variation
  \end{itemize}
\end{satz}

\begin{defn}
  Eine $\Alg$-messbare numerische Funktion $X$ über einem W-Raum $(\Omega, \Alg, \P)$ heißt \emph{Zufallsgröße} (ZG) oder \emph{Zufallsvariable}.
\end{defn}

% Bezeichnung: X, Y, Z, ...

\begin{bem}
  Häufig fordert man zusätzlich $\P(\{ X = \pm \infty \}) = 0$.
\end{bem}

\begin{defn}
  Das durch die ZG $X$ induzierte Bildmaß
  \[ P_X : \LebAlg(\R^1) \to [0,1], \quad B \mapsto \P(\{ X \in B \}) = \P(X^{-1}(B)) \]
  heißt \emph{Verteilungsgesetz} der ZG $X$ und
  \[ F_X : \R \to \R, \quad x \mapsto P_X(\left]-\infty, x\right]) = \P(\{ X \leq x \}) \]
  heißt \emph{Verteilungsfunktion} (VF) der ZG $X$.
\end{defn}

% Kapitel 3.2. Existenz einer ZG bei gegebener VF

\begin{satz}
  Sei $F$ eine VF auf $\R^1$. Dann existiert ein W-Raum $(\Omega, \Alg, \P)$ und eine ZG $X$ auf $\Omega$ derart, dass $F_X = F$.
\end{satz}

\begin{beweis}
  \begin{enumerate}
    \item Möglichkeit: Wähle $\Omega \coloneqq \R^1$, $\Alg \coloneqq \LebAlg(\R^1)$ und $\P \coloneqq \mu_F$ als das von von $F$ erzeugte Maß und setze $X \coloneqq \id$.
    \item Möglichkeit: Wähle $\Omega \coloneqq [0,1]$, $\Alg \coloneqq \Leb([0,1])$, $\P \coloneqq \lambda_1$. Setze
    \[
      X(w) \coloneqq F^{-}(w) \coloneqq \inf \{ F \geq w \} \enspace \text{ für } w > 0, \quad
      X(0) \coloneqq \lim_{w \downarrow 0} F^{-}(w)
    \]
  \end{enumerate}
\end{beweis}

% Ausgelassen: Wichtiges Prinzip bei der erzeugung von Pseudo-Zufallszahlen


% Vorlesung 18

\begin{defn}
  Sei $X_1, ..., X_n$ eine endliche Familie von ZGen über $(\Omega, \Alg, \P)$. Diese Familie heißt \emph{stochastisch unabhängig}, falls
  \[ \P(\bigcap_{i=1}^n \{ X_i \in B_i \}) = \prod_{i=1}^n \P(\{ X_i \in B_i \}) \quad \text{für alle $B_1, ..., B_n \in \LebAlg(\ER^1)$.} \]
\end{defn}

\begin{satz}
  Seien $X_1, ..., X_n$ unabhängige ZGen über $(\Omega, \Alg, \P)$ und $g_1, ..., g_n : \R \to \R$ Borel-messbar. Setze $Y_i \coloneqq g_i(X_i) \coloneqq g_i \circ X_i$ für $i = 1, ..., n$, dann sind auch $Y_1, ..., Y_n$ unabhängige ZGen.
\end{satz}

% Kapitel 3.3. Integrale nichtnegativer messbarer Funktionen

\begin{defn}
  Eine Funktion $f : (\Omega, \Alg) \to (\R, \Bor)$ heißt \emph{einfache Funktion} oder \emph{Elementarfunktion} auf $(\Omega, \Alg)$, wenn gilt:
  \begin{multicols}{3}
    \begin{itemize}
      \item $f$ ist messbar
      \item $f(\Omega) \subset \left[0, \infty\right[$
      \item $f(\Omega)$ ist endlich
    \end{itemize}
  \end{multicols}
  Die Menge aller elementaren Funktionen auf $(\Omega, \Alg)$ ist $\E(\Omega, \Alg)$.
\end{defn}

\begin{nota}
  $a \wedge b \coloneqq \min \{a, b\} \enspace \text{und} \enspace a \vee b \coloneqq \max \{a, b\}$
\end{nota}

\begin{satz}
  Seien $f, g \in \E(\Omega, \Alg)$ und $a \geq 0$. Dann auch in $\E(\Omega, \Alg)$:
  \begin{multicols}{5}
    \begin{itemize}
      \item $f + g$
      \item $f \cdot g$
      \item $f \vee g$
      \item $f \wedge g$
      \item $a \cdot f$
    \end{itemize}
  \end{multicols}
\end{satz}

\begin{defn}
  Sei $f \in \E(\Omega, \Alg)$ und $\Omega = A_1 \sqcup ... \sqcup A_k$ eine disjunkte Vereinigung von Mengen mit $A_j \in \Alg$ für alle $j = 1, ..., k$, sodass $f(A_j) = \{ y_j \}$, dann heißt die Darstellung
  \[ f = \sum_{j=1}^k y_j \cdot \ind_{A_j} \quad \text{\emph{kanonische Darstellung}.} \]
\end{defn}

\begin{defn}
  Sei $(\Omega, \Alg, \mu)$ ein Maßraum, $f : \Omega \to \R$ elementar. Dann heißt die (von der Darstellung $f = \sum_{j=1}^k y_j \cdot \ind_{A_j}$ unabh.) Zahl
  \[ \IntOmu{f} \coloneqq \sum_{j=1}^k y_j \mu(A_j) \quad \text{\emph{$\mu$-Integral} von $f$.} \]
\end{defn}

\begin{satz}
  Es gilt für $f, g \in \E(\Omega, \Alg)$, $a, b \geq 0$:
  \begin{itemize}
    \miniitem{0.3\linewidth}{$\IntOmu{\ind_A} = \mu(A)$}
    \miniitem{0.5\linewidth}{$f \leq g \implies \IntOmu{f} \leq \IntOmu{g}$}
    \item $\IntOmu{a \cdot f + b \cdot g} = a \cdot \IntOmu{f} + b \cdot \IntOmu{g}$
  \end{itemize}
\end{satz}

% Ausgelassen: Beispiel Dirichlet-Funktion $\ind_{\Q}$

\begin{satz}
  Sei $(f_n)_{n \in \N}$ eine isotone (= monotone) Folge elementarer Funktionen über $(\Omega, \Alg)$. Dann gilt für jede elementare Funktion $f$ mit $f \leq \sup_{n \in N} f_n$ die Ungleichung $\IntOmu{f} \leq \sup_{n \in \N} \IntOmu{f_n}$.
\end{satz}

\begin{korollar}
  Seien $(f_n)_{n \in \N}$ und $(g_n)_{n \in \N}$ isotone Folgen elementarer Funktionen mit $\sup_{n \in \N} f_n = \sup_{n \in \N} g_n$. Dann ist $\sup_{n \in \N} \IntOmu{f_n} = \sup_{n \in \N} \IntOmu{g_n}$.
\end{korollar}

\begin{satz}
  Sei $f : (\Omega, \Alg, \mu) \to (\ER, \LebAlg(\ER))$ nichtnegativ. Dann gibt es eine isotone Folge $(f_n)_{n \in \N}$ elementarer Funktionen mit $\sup_{n \in \N} f_n = f$.
\end{satz}

\begin{defn}
  Sei $f : (\Omega, \Alg, \mu) \to (\ER, \LebAlg(\ER))$ nichtnegativ und $(f_n)_{n \in \N}$ eine Folge elementarer Funktionen mit $\sup_{n \in \N} f_n = f$. Dann heißt
  \[ \IntOmu{f} \coloneqq \sup_{n \in \N} \IntOmu{f_n} \quad \text{\emph{$\mu$-Integral} von $f$.} \]
\end{defn}

% Kapitel 3.4. Integral einer beliebigen, messbaren Funktion

\begin{defn}
  Eine $\Alg$-messbare, numerische Fkt. $f : (\Omega, \Alg, \mu) \to (\ER, \LebAlg(\ER))$ heißt \emph{$\mu$-integrierbar}, falls
  \[ \IntOmu{f^+} < \infty \quad \text{und} \quad \IntOmu{f^-} < \infty. \]
  In diesem Fall definieren wir das \emph{Lebesgue-Integral} von $f$ als
  \[ \IntOmu{f} \coloneqq \IntOmu{f^+} - \IntOmu{f^-}. \]
\end{defn}


% Vorlesung 19

\begin{satz}
  Sei $f : (\Omega, \Alg, \mu) \to (\ER^1, \LebAlg(\R^1))$ messbar. Dann sind äquivalent:
  \begin{itemize}
    \miniitem{0.38 \linewidth}{$f$ ist $\mu$-integrierbar}
    \miniitem{0.6 \linewidth}{$f^+$ und $f^-$ sind $\mu$-integrierbar}
    \miniitem{0.38 \linewidth}{$\abs{f}$ ist $\mu$-integrierbar}
    \miniitem{0.6 \linewidth}{$\exists$ $\mu$-integrierbare Funktion $g$ mit $\abs{f} \leq g$}
  \end{itemize}
\end{satz}

% Ausgelassen: Bemerkung, dass i.A. uneigentliche Riemann-Integrale nicht Lebesgue-integrierbar sind

\begin{satz}
  Seien $f, g : (\Omega, \Alg, \mu) \to (\ER^1, \LebAlg(\ER^1))$ $\mu$-integrierbar und $\alpha, \beta \in \R$. Dann sind auch $\mu$-integrierbar:
  \begin{itemize}
    \begin{multicols}{4}
      \item $f \pm g$
      \item $f \vee g$
      \item $f \wedge g$
      \item $\alpha \cdot f$
    \end{multicols}
  \end{itemize}
  Es gilt: \enspace
  \begin{minipage}{0.86\linewidth}
    \begin{itemize}
      \item $\IntOmu{(\alpha \cdot f + \beta \cdot g)} = \alpha \IntOmu{f} + \beta \IntOmu{g}$ \pright{Linearität}
    \end{itemize}
  \end{minipage}
  \vspace{-4pt}
  \begin{itemize}
    \miniitem{0.35 \linewidth}{$\abs{\IntOmu{f}} \leq \IntOmu{\abs{f}}$} %\enspace ($\triangle$-Ungl.)}
    \miniitem{0.63 \linewidth}{$f \leq g \implies \IntOmu{f} \leq \IntOmu{g}$ \pright{Monotonie}}
  \end{itemize} 
\end{satz}

\begin{acht}
  Das Produkt $(f \cdot g)$ ist i.\,A. nicht $\mu$-integrierbar!
\end{acht}

\begin{defn}
  Sei $(\Omega, \Alg, \mu)$ ein Maßraum und $f : (\Omega, \Alg) \to (\ER, \mathcal{L}(\ER))$. Für $p \in \left[1, \infty\right[$ heißt $f$ \emph{$p$-integrierbar}, falls $\abs{f}^p$ $\mu$-integrierbar ist.
  \begin{align*}
    L^p(\Omega, \Alg, \mu) &\coloneqq \Set{ f : \Omega \to \ER }{\text{$f$ $p$-integrierbar, also } \IntOmu{\abs{f}^p} < \infty}, \\
    L^\infty(\Omega, \Alg, \mu) &\coloneqq \Set{ f : \Omega \to \ER }{\exists \, C > 0 \,:\, \abs{f} \leq C \text{ fast überall}}
  \end{align*}
  ist dann ein VR, genannt \emph{Lebesgue-Raum} ($L^p$-Raum), mit Norm
  \begin{align*}
    \norm{f}_p &\coloneqq ( \IntOmu{\abs{f}^p} )^{\tfrac{1}{p}} \\
    \norm{f}_\infty &\coloneqq \esssup_{\omega \in \Omega} \, \abs{f(\omega)} \coloneqq \inf \Set{ C \in \R }{ \abs{f} \leq C \text{ fast-überall} }
  \end{align*}
  Wir betrachten in $L^p$ zwei Funktionen als gleich, wenn sie bis auf einer Nullmenge übereinstimmen. Die $\triangle$-Ungleichung in $L^p$ wird auch \emph{Minkowski-Ungleichung} genannt.
\end{defn}

\begin{satz}
  Der $L^p(\mu)$ ist ein vollständiger normierter Raum, d.\,h. jede Cauchy-Folge bzgl. der Norm $\| \cdot \|_p$ ist auch konvergent.
\end{satz}

\begin{satz}
  Sei $f \in L^p(\Omega, \Alg, \mu)$, $g \in \L^q(\Omega, \Alg, \mu)$ mit $\tfrac{1}{p} + \tfrac{1}{q} = 1$. Dann ist $fg \in \L^1(\Omega, \Alg, \mu)$ und es gilt
  \[ \norm{fg}_1 \leq \norm{f}_p \cdot \norm{g}_q \quad \text{\emph{(Hölder-Ungleichung).}} \]
\end{satz}

\begin{bem}
  Für $p = 2$ ist $L^p(\Omega, \Alg, \mu)$ der Hilbertraum der quadratisch integrierbaren Funktionen mit dem Skalarprodukt
  \[ \langle f , g \rangle \coloneqq \IntOmu{(f \cdot g)}. \]
  Mit $q = 2$ folgt aus der Hölder-Ungleichung
  \[ \abs{\langle f , g \rangle} = \norm{fg}_1 \leq \norm{f}_2 \cdot \norm{g}_2 \quad \text{\emph{(Cauchy-Schwarz-Ungl.)}} \]
\end{bem}

\begin{satz}
  Sei $f : (\Omega, \Alg, \mu) \to (\ER^1, \mathcal{L}(\ER^1))$ nichtnegativ. Dann gilt
  \[ \IntOmu{f} = 0 \quad \iff \quad f \fue{=} 0. \]
\end{satz}

% Kapitel 3.5. Konvergenzsätze für Integrale messbarer Funktionen

% Kapitel 3.5.1. Satz von der monotonen Konvergenz

\begin{satz}[von der monotonen Konvergenz]
  Sei für alle $n \in \N$ die Funktion $f_n : (\Omega, \Alg, \mu) \to (\R^1, \LebAlg(\R^1))$ nicht negativ und $\Alg$-messbar, sodass $(f_n)_{n \in \N}$ eine isotone Folge ist. Dann gilt
  \[ \IntOmu{\lim_{n \to \infty} f_n} = \IntOmu{\sup_{n \in \N} f_n} = \sup_{n \in \N} \IntOmu{f_n} = \lim_{n \to \infty} \IntOmu{f_n}. \]
\end{satz}

\begin{kor}[Beppo Levi]
  Sei $(f_n)_{n \in \N}$ eine Folge nicht negativer, $\Alg$-messbarer, numerischer Funktionen auf $(\Omega, \Alg, \mu)$. Dann gilt
  \[ \IntOmu{\sum_{n=1}^{\infty} f_n} = \sum_{n=1}^\infty \IntOmu{f_n}. \]
\end{kor}

\begin{defn}
  Sei $(\Omega, \Alg)$ ein messbarer Raum mit Maßen $\mu$ und $\eta$. Dann heißt $\mu$ \emph{absolut stetig} bezüglich $\eta$ (notiert $\mu \ll \eta$), falls
  \[ \eta(A) = 0 \implies \mu(A) = 0 \quad \text{ für alle } A \in \Alg. \]
\end{defn}

% TODO: Namen dieses Maßes?
\begin{satz}
  Sei $(\Omega, \Alg, \mu)$ ein Maßraum und $f : (\Omega, \Alg) \to (\ER^1, \Leb(\ER^1))$ nichtnegativ und $\mu$-integrierbar. Dann definiert
  \[ \nu : \Alg \to \left[0, \infty \right], \quad A \mapsto \Int{A}{}{f}{\mu} = \IntOmu{f \cdot \chi_A} \]
  ein zu $\mu$ absolut stetiges, endliches Maß auf $(\Omega, \Alg)$.
\end{satz}

\begin{lem}[Fatou]
  Sei für $n \in \N$ die Fkt. $f_n : (\Omega, \Alg, \mu) \to (\R^1, \LebAlg(\R^1))$ nicht negativ und $\Alg$-messbar. Dann gilt
  \[ \IntOmu{\liminf_{n \to \infty} f_n} \leq \liminf_{n \to \infty} \IntOmu{f_n}. \]
  Falls $\IntOmu{\sup_{n \in \N} f_n} < \infty$, gilt zusätzlich
  \[ \IntOmu{\limsup_{n \to \infty} f_n} \geq \limsup_{n \to \infty} \IntOmu{f_n}. \]
\end{lem}


% Vorlesung 20

% Kapitel 3.5.3. Satz von F. Riesz

\begin{defn}
  Eine Folge $(f_n)_{n \in \N}$ $\Alg$-messbarer, numerischer Fktn. über $(\Omega, \Alg, \mu)$ \emph{konvergiert $\mu$-fast-überall} gegen $f : \Omega \to \Alg$, falls
  \[ \lim_{n \to \infty} f_n(\omega) = f(\omega) \quad \text{für $\mu$-fast-alle $\omega \in \Omega$ gilt.} \]
\end{defn}

% Ausgelassen: Notation

\begin{satz}[Riesz]
  Sei $(f_n)_{n \in \N}$ Folge in $L^p(\Omega, \Alg, \mu)$, $f_n \xrightarrow[n \to \infty]{\text{f.ü.}} f$ mit $f \in L^p(\mu)$. Dann gilt
  \[ f_n \xrightarrow[n \to \infty]{L^p(\mu)} f \quad \iff \quad \Int{\Omega}{}{\abs{f_n}^p}{\mu} \xrightarrow{n \to \infty} \Int{\Omega}{}{\abs{f}^p}{\mu}. \]
\end{satz}

% Kapitel 3.5.4. Satz von der majorisierten Konvergenz

\begin{satz}[von der majorisierten Konvergenz]
  Sei $(f_n)_{n \in \N}$ Folge $\Alg$-messbarer numerischer Funktionen auf $(\Omega, \Alg, \mu)$ und $g \in L^1(\mu)$ nicht negativ, sodass $\abs{f_n} \leq g$ für alle $n \in \N$. Sei desweiteren $f : \Omega \to \ER$ messbar mit $f_n \xrightarrow[n \to \infty]{\text{$\mu$-f.ü.}} f$. Dann ist
  \[ f \in L^1(\mu) \quad \text{mit} \quad \Int{\Omega}{}{f}{\mu} = \lim_{n \to \infty} \Int{\Omega}{}{f_n}{\mu}. \]
\end{satz}

% Kapitel 3.6. Integration bzgl. des Bildmaßes

\begin{satz}
  Sei $f : (\Omega, \Alg, \mu) \to (\Omega', \Alg')$ und $\mu' \coloneqq \mu \circ f^{-1}$ das Bildmaß von $\mu$ unter $f$. Sei $g : (\Omega', \Alg') \to (\R^1, \Bor(\R^1))$ nicht negativ. Dann gilt
  \[ \Int{\Omega'}{}{g}{\mu'} = \Int{\Omega}{}{(g \circ f)}{\mu}. \]
\end{satz}

% Vorlesung 21

% TODO: Klare Abtrennung Lebesgue-Borel-Maß und Lebesgue-Maß
% TODO: Klare Abtrennung Borel-Mengen und Lebesgue-Mengen
% TODO: Einführung des Symbols $\lambda_d$ für das Lebesgue-Maß auf $\R^d$
\begin{satz}[Transformationssatz]
  Sei $U, \widetilde{U} \opn \R^d$ und sei $\phi : U \to \widetilde{U}$ ein $\mathcal{C}^1$-Diffeomorphismus. Dann ist eine Funktion $f : \widetilde{U} \to \ER$ genau dann auf $\widetilde{U}$ Lebesgue-Borel-integrierbar, wenn $(f \circ \phi) \cdot \left|\det(D\phi)\right| : U \to \ER$ auf $U$ Lebesgue-Borel-interierbar ist. In diesem Fall gilt
  \[ \Int{U}{}{(f \circ \phi) \cdot \left|\det(D\phi)\right|}{\lambda_d} = \Int{\phi(U)}{}{f}{\lambda_d} = \Int{\widetilde{U}}{}{f}{\lambda_d}. \]
  Obige Gleichung ist auch erfüllt, wenn lediglich $f \geq 0$ gilt.
\end{satz}

\begin{defn}
  Für eine ZG $X : (\Omega, \Alg, \P) \to (\ER^1, \Bor(\ER^1))$ heißt die Zahl
  \[ \E X := \Int{\Omega}{}{X}{\P} \qquad \text{\emph{Erwartungswert} von $X$.} \]
\end{defn}

\begin{satz}
  $\E X = \Int{\R^1}{}{\id}{P_X}$, wobei $P_X \coloneqq \P \circ X^{-1}$.
\end{satz}
% TODO: Eigene Definition für $P_X$, genannt "`Verteilungsgesetz"'

\begin{kor}
  Sei $g : \R^1 \to \R^1$ messbar und $P_X$-integrierbar. Dann gilt
  \[ \E g(X) = \Int{\R^1}{}{g}{P_X} = \Int{-\infty}{\infty}{g(x)}{P_X(x)}, \]
  wobei rechts ein uneigentliches Riemann-Stieltjes-Integral steht.
\end{kor}

% TODO: Definition "`Zufallsvektor"'?
\begin{defn}
  Für Zufallsvektoren $X = (X_1, ..., X_k)$ mit Werten in $\R^k$ definieren wir $\E X \coloneqq (\E X_1, ..., \E X_k)$.
\end{defn}

% TODO: Was ist eine numerische Funktion?

\begin{bem}
  Sei $X = (X_1, ..., X_k)$ ein Zufallsvektor und $g : \R^k \to \R$ Borel-messbar und $P_X$-integrierbar. Dann ist
  \[ \E g(X_1, ..., X_k) = \Int{\R^k}{}{g}{P_{X}}. \]
\end{bem}

% Kapitel 3.7. Berechnung von Erwartungswerten

% Kapitel 3.7.1. Typen von Verteilungsfunktionen

\begin{satz}
  Sei $F_X$ VF einer ZG $X : (\Omega, \Alg, \P) \to (\R^1, \LebAlg(\R^1)$. Dann existiert für Lebesgue-fast-alle $x \in \R^1$ die Ableitung $F'(x)$.
\end{satz}

% TODO: Definition Verteilungsfunktion
\begin{defn}
  Sei $F_X$ VF einer ZG $X : (\Omega, \Alg, \P) \to (\R^1, \LebAlg(\R^1)$.
  \begin{itemize}
    \item $F_X$ heißt \emph{diskret}, falls $F_X$ höchstens abzählbar viele Sprungstellen $x_1, x_2, ... \in \R$ besitzt mit
      \[ \fa{k \in J \subset \N} p_k \coloneqq F_X(x_k) - \lim_{\mathclap{x \uparrow x_k}} F_X(x) > 0, \quad \sum_{k=1}^\infty p_k = 1. \]
    Dann ist $F_X$ zwischen den Sprüngen konstant.
    \item $F_X$ heißt \emph{stetig} (diffus, atomlos), wenn $F_X$ in jedem Punkt stetig ist. Dann gilt $P_X(\{ X = x \}) = 0$ für alle $x \in \R$.
  \end{itemize}
  Für stetige Verteilungen ergibt sich eine weitere Unterteilung:
  \begin{itemize}
    \item $F_X$ heißt \emph{absolutstetig} (totalstetig), wenn es für alle $\epsilon > 0$ ein $\delta > 0$ gibt, sodass für höchstens abzählbar viele, disjunkte Intervalle $I_k = \left]a_k, b_k\right]$ mit $k \in J \subset \N$ gilt:
    \[ \sum_{k \in J} (b_k - a_k) < \delta \enspace \implies \enspace \sum_{k \in J} (F_X(b_k) - F_X(a_k)) < \epsilon. \]
    % Ausgelassen: Beispiel: Lipschitz-stetige Funktionen
    \item $F_X$ heißt \emph{singulärstetig} (stetig, aber nicht absolutstetig), wenn die Wachstumspunkte von $F_X$ eine Nullmenge bilden, also
    \[ \lambda_1(\Set{ x \in \R^1 }{ \fa{\epsilon > 0} F(x+\epsilon)- F(x-\epsilon) > 0 }) = 0 \]
    oder äquivalent dazu die Ableitung fast-überall verschwindet, also
    \[ \lambda_1(\Set{ x \in \R^1 }{ F_X'(x) = 0 }) = 1. \]
    % Ausgelassen: Beispiel: Cantor-Treppe (Teufelstreppe)
  \end{itemize}
\end{defn}

\begin{satz}
  Jede VF $F$ auf $\R^1$ besitzt eine eindeutige Zerlegung (Lebesgue-Zerlegung) als konvexe Linearkombination einer diskreten, einer singulär-stetigen und einer absolut-stetigen VF
  \[ F = \alpha_d F_d + \alpha_s F_s + \alpha_a F_a \quad \text{mit} \quad \alpha_d, \alpha_s, \alpha_a \geq 0, \, \alpha_d {+} \alpha_s {+} \alpha_a = 1. \]
\end{satz}

\begin{defn}
  Falls $F_X$ absolut-stetig, dann heißt die nicht negative, Lebesgue-messbare Funktion
  \[ f = f_X = F'_X : \R \to \R_{\geq 0}, \quad x \mapsto \begin{cases} F'_X(x), & \text{falls Ableitung ex.} \\ 0, & \text{sonst} \end{cases} \]
  \emph{(Wahrscheinlichkeits-)Dichte} (WD) von $F_X$  bzw. von $X$.
\end{defn}

\begin{bem}
  Dann gilt für alle $y \in \R$:
  \[ \Int{\mathclap{-\infty}}{y}{f_X(x)}{x} = F_X(y), \quad \text{also insbesondere} \quad \Int{\mathclap{-\infty}}{\infty}{f_X(x)}{x} = 1. \]
\end{bem}

\begin{bem}
  $F_X$ ist als VF genau dann absolut stetig, wenn das Maß $P_X$ bezüglich $\lambda_1$ absolut stetig ist (also $P_Y \ll \lambda_1$ gilt).
\end{bem}

% Vorlesung vom 10.1.2014

% Ausgelassen: Beispiele:
% * Dichte bei Standard-Normalverteilung und Exponentialverteilung
% * Sprunghöhen bei Poissonverteilung

\begin{satz}\mbox{}\vspace{-16pt}
  \[ \E(X) = \begin{cases}
    \Int{\R^1}{}{x \cdot f_X(x)}{x}, & \text{falls $F_X$ absolutstetig mit WD $f_X$} \\
    \sum_{k \in J} x_k \cdot p_k, & \parbox[t]{.6\textwidth}{falls $F_X$ diskret mit Sprüngen $p_k$ \\ bei $x_k$, $k \in J \subset \N$}
  \end{cases} \]
\end{satz}

% Ausgelassen: Deutung der $\E X$ als Massenschwerpunkte
% Ausgelassen: Beispiel Würfelwurf

% TODO





% Ende bereinigte/verschönerte Einträge






% Approximationssatz für Maße?

% TODO: Verteilungen:
% * Bernoulli
% * geometrische Verteilung
% * Pascal-Verteilung

% \[ \E X = \Int \]

...

% Analoge Betrachtungen für $k$-dimensionale Z-Verteilungen $(X_1, ..., X_k)$

$F_(X_1, ..., X_k)$ heißt \emph{absolut stetig}, falls für alle $\epsilon > 0$ ein $\delta > 0$ existiert, sodass für $I_{\alpha} = \left] a_j, b_j \right]$, $j = 1, 2, ...$ mit $\sum_{j \geq 1} \lambda_k(I_j) \leq \delta$ gilt:
\[ \sum_{j \geq 1} \P_{(X_1, ..., X_k)}(I) = \sum_{j \geq 1} (triangle F_{(X_1, ..., X_k)}) I_j \leq \epsilon \]

Genau dann existiert eine (Lebesgue-) Borel-messbare Funktion $f_{(X_1, ..., Xk)}(x_1, ..., x_k) \geq 0$ mit $\Int{\R^1}{}{f_{(X_1, ..., X_k)}}{\lambda_k} = 1$

Sei $g : \R^k \to \R^1$ Borel-messbar

$\E g(X_1, ..., X_k) = \Int{\R^1}{}{g \cdot f_{(X_1, ..., X_k)}}{\lambda_k}$
 
Falls $F_{(X_1, ..., X_k)}$ "hinreichend glatt", so ergibt sich
$f_{X_1, ..., X_k}(x_1, ..., x_k) = \frac{\partial^k}{\partial_{x_1} \cdot \partial_{x_k}} F_{(X_1, ..., X_k)}(x_1, ..., x_k)$

$F_{X_1, ..., X_k}$ heißt singulär-stetig, falls $P_{(X_1, ..., X_k)}(\{ x \}) = 0 \forall x \in \R^k$
und es existiert eine Lebesgue-messbare Menge $S$ mit $\lambda_k(S) = 0$ und $P((X_1, ..., X_k))(S) = 1$.

$F_{(X_1, ..., X_k)}$ heißt diskret, falls eine höchstens abzählbare Punktmenge $S = \{ x_1, ... \} \subset \R^k$ und $p_i = P_{(X_1, ..., X_k)}(\{ x_i \}) > 0$ mit $\sum_{i \geq 1} p_i = 1$

Sei $x_i = (x_i^{(1)}, ..., x_i^{(k)}) \in \R^k$

$\E g(X_1, ..., X_k) = \sum_{i \geq 1} g(x_i^{(1)}, ..., x_i^{(k)}) p_i$

% Riemann-Stieltjes-Integral

$g : \R^1 \to \R^1$ sei zunächst beliebig (stetig oder hinreichend glatt)

Sei $a = \xi_0^{(n)} < \xi_1^{(n)} < ... < x_{k_n}^{(n)}$ und $x_k^{(n)} \in \left] \xi_{k-1}^{(n)}, \xi_{k}^{(n)} \right[$.

\begin{defn}
  $(\xi_n)$ sei eine Zerlegungsfolge mit $\max_{1 \leq k \leq k_n} (x_k^{(n)} - x_{k-1}^{(n)}) \xrightarrow{n \to \infty} 0$

  $(x_k^{(n)})$ sei eine Zwischenwertfolge

  \[ \lim_{n \to \infty} \sum_{k=1}^{k_n} g(x_k^{(n)}) (F(x_k^{(n)}) - F(x_{k-1}^{(n)})) = \Int{a}{b}{g(x)}{F(x)} = \Int{[a, b]}{}{g}{F\lambda_1} \]

  wobei $F : \R^1 \to \R^1$ zunächst beliebig (monoton oder von beschränkter Variation)
\end{defn}

% partielle Integration für R-S-Integrale

Sei $g$ bzgl. $F$ R-S-integrierbar, d.\,h. der Grenzwert oben existiert
Dann ist auch $F$ bzgl. $g$ R-S-integrierbar und es gilt

\[ \Int{a}{b}{g(x)}{F(x)} = [g(x) \cdot F(x)]_a^b - \Int{a}{b}{F(x)}{g(x)} \]

Ausnutzen der partiellen Integration zur Berechnung von Erwartungswerten

\[ \E X = \Int{- \infty}{\infty}{x}{F(x)} \]
\[ \Int{a}{b}{x}{F_X(x)} = \lim_{a \to -\infty, b \to \infty} [x \cdot F_X(-x)]_0^{-a} - \Int{0}{-a}{F_X(-x)}{x} + [x (F_X(x) - 1)]_0^b - \Int{0}{b}{(F_X(x) - 1)}{x} \]

Falls $\lim{x \to \infty} x F_X(-x) = \lim{x \to \infty} x (1 - F_X(x)) = 0$, so gilt

$\E X = \Int{0}{\infty}{1 - F_X(x) - F_X(-x)}{x}$, falls $\E |X| < \infty$

$\E |X| = \Int{0}{\infty}{1 - F_X(x) - F_X(-x)}{x}$

Genauso werden Erwartungswerte von Funktionen von $X$ berechnet, z.B. mit $x^2 F_X(-x) \xrightarrow{x \to \infty} 0$ und $x^2 (1 - F_X(x)) \xrightarrow{x \to \infty} 0$

$\E X^2 = 2 \Int{0}{\infty}{x (1 - F_X(x) + F_X(-x))}{x} = \P(|X| > x)$ (falls $F_X$ stetig)

$\E |X|^k = k \Int{0}{\infty}{x^{k-1} (1 - F_X(x) + F_X(-x))}{x}$

\begin{defn}
  $\E X^k$ ($\E |X|^k$) heißt $k$-tes (absolutes) Moment der ZG $X$. $\E (X - \E X)^k$ heißt $k$-tes zentriertes Moment der ZG X. $\mathrm{Var}(X) \coloneqq D^2 X = \E (X - \E X)^2 = \E X^2$ heißt \emph{Streuung} (Dispersion, Varianz) der ZG $X$.
\end{defn}



% Vorlesung (15.1.2014)

% Kapitel 2.8.

$X$ sei ZG über $(\Omega, \Alg, \P)$ mit $\E \abs{X}^n < \infty$ für alle $n \in \N$.

\[ \E X^n = \Int{\R^n}{}{x^n \P_X}{x} = \Int{-\infty}{\infty}{x^n}{F_X(x)} \]

heißt \emph{$n$-tes Moment}

$\E X$ Erwartungswert (Schwerpunkt der W-Massenverteilung) "`Lageparameter"'

$\mathbb{D}^2 X = \Var(X) = \E(X-\E X)^2 = \E(X^2) - (\E X)^2 \geq 0$

Eigenschaft: $\Var(aX+b) = a^2 \Var(X)$ für $a, b \in \R^1$

$\Var(X) \leq \E (X-c)^2$ für alle $c \in \R^1$

$\Var(X) = 0$ $\iff$ $X - \E X = 0$ $\iff$ $X = \mathrm{const}$ $\P$-fast-sicher

\begin{satz}
  $X$ sei eine ZG und $g : \left[ 0, \infty \right[ \to [0,\infty]$ nichtfallend. Dann gilt $\P(\abs{X} \geq \epsilon) \leq \frac{\E g(\abs{X})}{g(\epsilon)}$ für beliebiges $\epsilon > 0$.
\end{satz}

Spezialfälle:

\begin{itemize}
  \item $g(x) = x$, dann $\P(\abs{X} \geq \epsilon) \leq \frac{\E X}{\epsilon}$ \emph{Markow-Ungleichung}
  \item $g(x) = x^2$, dann $\P(\abs{X - \E X} \geq \epsilon) \leq \frac{\E (X - \E X)^2}{\epsilon^2} = \frac{\Var(X)}{\epsilon^2}$ \emph{Tschebyschew-Ungleichung}
  \item $g(x) = \exp(ax)$ für $a > 0$, dann $\P(\abs{X} \geq \epsilon) \leq \exp(-a\epsilon) \E \exp(a \abs{X})$
\end{itemize}

$0 > B = \E \exp(a \abs{X}) \geq \frac{a^n \E \abs{X}^n}{n!}$

$\implies$ $\E \abs{X}^n \leq \tfrac{B}{a^n} n!$ für alle $n \in \N$

$\implies$ $\abs{\E X^n} \leq \tfrac{B}{a^n} n!$

$\implies$ $\E \exp(z X)$ ist analytisch für $\abs{z} < a$

\begin{defn}
  $\E \exp(z X)$ heißt momenterzeugende Funktion der ZG $X$ oder VF $F_X$.
\end{defn}

% Beispiel
$X = N(\mu, \sigma^2)$ $\implies$ $\E \exp(z X) = \exp\left( z\mu + \tfrac{\sigma^2}{2} z^2 \right)$ für $z \in \C$

\emph{Höldersche Ungleichung}:

$\abs{\E X Y} \leq \E \abs{XY} \leq (\E \abs{X}^p)^{\tfrac{1}{p}} \cdot (\E \abs{X}^q)^{\tfrac{1}{q}}$ für $p, q \geq 1$

$\implies$ Cauchy-Schwarz-Bunjakowski-Ungleichung für $p = q = 2$:

$\abs{\E XY} \leq \sqrt{ \E (X^2) \E(Y^2) }$

Verallgemeinerung:

$\E(X_1^{n_1} \cdots X_k^{n_k}) \leq (\E \abs{X_1}^n)^{\tfrac{n_1}{n}} \cdots (\E \abs{X_k}^n)^{\tfrac{n_k}{n}}$, $n = n_1 + ... + n_k$

\emph{Jensensche Ungleichung}

Sei $g : \R^1 \to \R^1$ konvex auf einem Intervall $J$, d.\,h.

$g(\alpha x + (1-\alpha)y) \leq \alpha g(x) + (1-\alpha) g(y)$ für alle $x, y \in I$ und $\alpha \in [0,1]$

Per Induktion folgt: $g(\sum_{i=1}^n \alpha_i x_i) \leq \sum_{i=1}^n \alpha_i g(x_i)$ für $x_1, ..., x_n \in J$, $\alpha_i \geq 0$, $\alpha_1 + ... + \alpha_n = 1$

\begin{satz}
  $g(\E X) \leq \E g(X)$, falls $\P(X \in J) = 1$ und $\E \abs{X} < \infty$
\end{satz}

% Bespiel

$g(x) = \abs{x}^{\tfrac{n}{m}}$ für $0 < m \leq n$, $\implies$ \emph{Ljapunow-Ungleichung}

\begin{problem}[Momentenproblem]
  Unter welchen Bedingungen ist eine Zahlenfolge $c_0 = 1, c_1, c_2, ...$ eine Momentenfolge einer ZG $X$, d.\,h. $c_n = \E X^n$.
\end{problem}

\begin{antwort}
  $0 \leq \E(z_0 + z_1 X + ... + z_n X^n)^2 = \E(\sum_{i,j=1}^n z_i z_j X^{i+j}) = \sum_{i,j=1}^n z_i z_j c_{i+j}$ genau dann, wenn

  $\det \begin{pmatrix} c_0 & c_1 & \cdots & c_n \\ c_1 & c_2 & \cdots & c_{n+1} \\ \vdots & \vdots & & \vdots \\ c_n & c_{n+1} & \cdots & c_{2n} \end{pmatrix} \geq 0$.
\end{antwort}

\begin{problem}
  Wann ist die zugehörige VF $F_X$ eindeutig festgelegt? $c_n = \Int{0}{\infty}{x^n}{F_X(x)}$ (Stieltjes-MP), $c_n = \Int{-\infty}{\infty}{x^n}{F_X(x)}$ (Hamburger MP)

  Hinreichende Bedingung für Bestimmtheit:

  Stieltjes-MP: $\sum_{n=1}^\infty \frac{1}{c_{2n}^{\tfrac{1}{n}}} = \infty$

  Hamburger MP: $\sum_{n=1}^\infty \frac{1}{c_{2n}^{\tfrac{1}{2n}}} = \infty$

  (Carleman-Kriterien)
\end{problem}


% Vorlesung vom 17.1.2014

% Kapitel 3.9. Randverteilungen von ZVen und Unabhängigkeit

\begin{defn}
  Sei $X = (X_1, ..., X_k)$ eine $k$-dimensionale ZV über $(\Omega, \Alg, \P)$. $X_1, ..., X_k$ heißen \emph{stochastisch unabhängig}, falls
  \[ \P(\bigcap_{i=1}^k \{ X_i \in B_i \}) = \prod_{i=1}^k \P(\{ X_i \in B_i \}) \]
  für alle $B_1, ..., B_k \in \mathcal{L}(\R^1)$. Dies ist genau dann der Fall, wenn
  \[ \P(X_1 \leq x_1, ..., X_k \leq x_k) = \P(X_1 \leq x_1) \cdots \P(X_k \leq X_k) \]
  für alle $x_1, ..., x_k \in \R$. Falls die W-Dichte $f_X(x_1, ..., x_k) = \tfrac{\partial}{\partial x_1 \cdots \partial x_k} F(x_1, ..., x_k)$ existiert (also $F_X$ absolut stetig), ist dies äquivalent zu
  \[ f_X(x_1, ..., x_k) = f_{X_1}(x_1) \cdots f_{X_k}(x_k). \]
  Für diskrete Verteilungen ist dies äquivalent zu
  \[ \P(X_1 = x_1, ..., X_k = x_k) = \P(X_1 = x_1) \cdots \P(X_k = x_k) \]
  für alle $x_1, ..., x_k \in \R$.
\end{defn}

\begin{defn}
  Für eine $k$-dimensionale ZV $X = (X_1, ..., X_k)$ heißt
  \[ F_{(X_{i_1}, ..., X_{i_l})}(x_{i_1}, ..., x_{i_l}) = \lim_{\substack{x_j \to \infty}_{j \in \{ 1, ..., k\} \setminus \{ i_1, ..., i_l \}}} F_(X_1, ..., X_k)(x_1, ..., x_k) \]
  für $1 \leq i_1 < ... < i_l \leq k$, $l = 1, ..., k-1$ \emph{$l$-dimensionale Rand-(Marginal-)verteilungsfunktion}.
\end{defn}

Falls $f_X(x_1, ..., x_k)$ existiert, so existieren sämtliche Randdichten

\[ f_{(X_{i_1}, ..., X_{i_k})}(x_{i_1}, ..., x_{i_k}) = \Int{\R^{k-l}}{}{f_{(X_1, ..., X_k)}(x_1, ..., x_k)}{(x_1, \widehat{x_{i_1}}, ..., \widehat{x_{i_k}}, ..., x_k)} \]

Analog folgt für eine diskrete ZV die Diskretheit der Randverteilungen $k=2$:

\[ \P(X_1 = x_m^{(1)}) = \sum_{x_m^{(2)}} \P(X_1 = x_m^{(1)}, X_2 = x_m^{(2)}) \]
\[ \P(X_2 = x_m^{(2)}) = \sum_{x_m^{(1)}} \P(X_1 = x_m^{(1)}, X_2 = x_m^{(2)}) \]

wobei $x_m = (x_m^{(1)}, ..., x_m^{(k)})$ die Massenschwerpunkte sind.

Wichtig: Im allgemeinen bestimmen die Randverteilungen nicht die gemeinsame Verteilung des Vektors.

% Copula: Vorgabe des Zusammenhangs der Komponenten und eindimensionale Randverteilungen

% Kapitel 3.10. Kovarianz und Korrelation

\begin{defn}
  $(X, Y)$ sei eine zweidimensionale ZVüber $(\Omega, \Alg, \P)$ mit $\E X^2 < \infty$, $\E Y^2 < \infty$. Dann heißt
  \[ \cov(X, Y) \coloneqq \E(XY) - \E X \E Y = \E ((X - \E X) \cdot (Y - \E Y)) \]
  \emph{Kovarianz} von $X$ und $Y$ und
  \[ \cor(X, Y) \coloneqq \frac{\cov(X, Y)}{\sqrt{\Var(X) \cdot \Var(Y)}} \]
  \emph{Korrelation} von $X$ und $Y$.
\end{defn}

\begin{satz}
  \begin{itemize}
    \item Falls $X, Y$ unabhängig, so gilt $\cov(X, Y) = \cor(X, Y) = 0$
    \item $\abs{\cor(X, Y)} \leq 1$
    \item $\cor(X, Y) = 1 \iff \exists \, a, b \in \R^1 \,:\, \P(Y = aX + b) = 1$.
  \end{itemize}
\end{satz}

\begin{defn}
  Falls $\cor(X, Y) = 0$, so heißen $X, Y$ \emph{unkorreliert}.
\end{defn}

\begin{acht}
  Aus Unkorreliertheit folgt i.\,A. nicht Unabhängigkeit!
\end{acht}

\begin{bsp}
  Sei $X$ eine ZG mit der symmetrischen Dichte $f_X(x) = f_X(-x)$ und $\Int{-\infty}{\infty}{\abs{x}^3 f_X(x)}{x} < \infty$, dann ist $\cov(X, X^2) = 0$, aber $X$ und $X^2$ nicht unabhängig.
\end{bsp}

\begin{bem}
  \begin{itemize}
    \item $\cor(X, Y) = 1$: positive Korrelation
    \item $\cor(X, Y) = -1$: negative Korrelation
    \item $\cor(X, Y) = 0$: Unkorreliertheit
  \end{itemize}
\end{bem}

Wichtig: Falls $(X, Y)$ eine zweidimensionale Normalverteilung besitzt, so folgt aus $\cor(X, Y) = 0$ die Unabhängigkeit von $X$ und $Y$.

\begin{satz}
  $X_1, ..., X_n$ seien paarweise unkorrelierte ZGen mit $\E X_i^2 < \infty$ für $i = 1, ..., n$. Dann gilt
  \[ \Var(X_1 + ... + X_n) = \Var(X_1) + ... + \Var(X_n) \]
\end{satz}


% Vorlesung vom 22.1.2014

Wir wollen die Kovarianz der ZG $X$ und $Y$ berechnen.

1. Fall: Es existiert eine gemeinsame WD von $(X, Y)$, $f_{(X, Y)}(x, y)$ (Lebesgue-messbar, $\geq 0$, $\Int{\R^2}{}{f_{(X,Y)}(x,y)}{(x, y)} = 1$)

Beispiel: $2$-dimensionale Normalverteilungsdichte

\[ f_{(X,Y)}(x,y) = \frac{1}{2 \pi \sigma_1 \sigma_2 \sqrt{1 - \rho^2}} \exp \left( - \frac{1}{2 (1 - \rho^2)} \left( \frac{(x-\mu_1)^2}{\sigma_1^2} + \frac{(y-\mu_2)^2}{\sigma_2^2} - \frac{2 \rho (x-\mu_1)(y-\mu_2)}{\sigma_1 \sigma_2} \right) \right), \]

wobei $\mu_1 = \E X$, $\mu_2 = \E Y$, $\sigma_1^2 = \Var(X)$, $\sigma_2^2 = \Var(Y)$

$\E (X \cdot Y) = \Int{\R^2}{}{x \cdot y \cdot f_{(X,Y)}(x,y)}{(x, y)} = \mu_1 \cdot \mu_2 + \sigma_1 \cdot \sigma_2 \cdot \rho$

Also: $\Cov(X, Y) = \sigma_1 \cdot \sigma_2 \cdot \rho$ und $\cor(X, Y) = \rho \in [-1, 1]$.

Randverteilungen:

$f_X(x) = \Int{-\infty}{\infty}{f_{(X,Y)}(x,y)}{y}$
$f_Y(y) = \Int{-\infty}{\infty}{f_{(X,Y)}(x,y)}{x}$

$\E X = \Int{-\infty}{\infty}{x f_X(x)}{x} = \Int{-\infty}{\infty}{\Int{-\infty}{\infty}{x f_{(X,Y)}(x,y)}{x}}{y}$

2. Fall: $(X, Y)$ mit Werten in einer höchstens abzählbaren Menge an $(x_i, y_i)$ wird mit Wahrscheinlichkeit $p_{ij} > 0$ angenommen, $i, j = 1, 2, ...$.

\[ \P((X, Y) = (x_i, y_i)) = \P(X=x_1, Y=y_j) \]

\[ \E(X \cdot Y) = \sum_{i=1} \sum_{j=1} x_i \cdot y_j p_{ij} \]

\[ \E X = \sum_{i=1} x_i \P(X = x_i) = \sum_{i=1} \sum_{j=1} x_i p_{ij} \]

\[ \E Y = \sum_{j=1} y_j \P(Y = y_j) = \sum_{j=1} \sum_{i=1} y_j p_{ij} \]

\[ \Cov(X, Y) = \E (X \cdot Y) - (\E X) \cdot (\E Y) = \sum_{i=1} \sum_{j=1} x_i y_j (p_{ij} - p_i \cdot q_j) \]

mit $p_i \coloneqq \sum_{j=1} p_{ij}$, $q_j \coloneqq \sum_{i=1} p_{ij}$

3. Fall: $(X, Y)$ ist singulär-stetig verteilt oder besitzt eine singulär-stetige Komponente

Beispiel: Zweidimensionale Exponentialverteilung \ldots

\[ \overline{F}(x,y) = \P(X > x, Y > y) = \exp \left( -(\lambda x + \mu y + \nu \max(x,y)) \right) \]

\[ F(x, y) = \P(A^c \cap B^c) = \P((A \cup B)^c) = 1 - \P(A \cup B) = 1 - \P(A) - \P(B) + \P(A \cap B) = 1 - \overline{F}(x,0) - \overline{F}(0,y) + \overline{F}(x,y) \]

\[ \E XY = \Int{\R^2}{}{x \cdot y}{F(x,y)} = ... = - \Int{0}{\infty}{[y \cdot \P(X > x, Y > y)]_{y=0}^{y=\infty}}{x} + \Int{\R_{\geq 0}^2}{}{\P(X > x, Y > y)}{(x,y)} \]

Für $(X, Y) \in \R^2$ gilt

\[ \E (XY) = \Int{0}{\infty}{\Int{0}{\infty}{\underbrace{\P(X > x, Y > y)}_{= \overline{F}(x,y)}}{x}}{y} \]

Falls $(X, Y) \in \R^2$, so gilt:

\[ \E (X \cdot Y) = \Int{0}{\infty}{\Int{0}{\infty}{(\P(X > x, Y > y) + \P(X \leq -x, Y \leq -y) - \P(X \geq x, Y \leq -y) - \P(X \leq -x, Y > y))}{y}}{x} \]

Im Beispiel:

\[ \E (X \cdot Y) = \Int{0}{\infty}{\Int{0}{\infty}{\exp(-\lambda x - \mu y - \nu \max(x,y))}{x}}{y} = ... = \frac{1}{\lambda (\mu + \nu)} + \frac{1}{\mu (\lambda + \nu)} - (\frac{1}{\lambda} + \frac{1}{\mu}) \frac{1}{\lambda + \mu + \nu} \]

\[ \E X = \Int{0}{\infty}{x}{F(x)} = \Int{0}{\infty}{\P(X > x)}{x} = \frac{1}{\lambda + \nu} \]

\[ \cor(X,Y) = ... = \frac{\nu}{(\lambda + \nu) (\mu + \nu) (\lambda + \mu + \nu)} \]

Also: $\nu = 0$ genau dann, wenn $X$ und $Y$ unabhängig

% Kapitel 4. Funktionen von Zufalssgrößen und -vektoren

% Kapitel 4.1. Funktion einer Zufallsgröße

$Y = g(X)$, $g : \R^1 \to \R^1$ Borel-messbar

\[ F_Y(y) = \P(Y \leq y) = \P(Y in \left] -\infty, y \right[) = \P(X \in g^{-1}(\left] -\infty, y \right])) \]

Beispiel: $Y = X^2$, $y \geq 0$

$F_Y(y) = \P(X^2 \leq y) = \P(\abs{X} \leq \sqrt{y}) = \P(-\sqrt{y} \leq X \leq \sqrt{y}) = \P(X = -\sqrt{y}) + F_X(\sqrt{y}) - F_X(-\sqrt{y}) = F_X(\sqrt{y}) - \lim_{z \uparrow -\sqrt{y}} F_X(z)$

\begin{satz}
  $X$ sei absolut stetig mit Dichte $f_X$ und $\P(X \in D) = 1$ für $D \subset \R^1$ offen und $g : D \to \R^1$ eine $\mathcal{C}^1$-Funktion mit $g'(x) > 0$ für alle $x \in D$. Dann ist $Y = g(X)$ absolut-stetig mit der Dichte
  \[ f_Y(y) = \begin{cases} 0, & \text{für } y \in \R^1 \setminus g(D) \\ \frac{f_X(g^{-1}(y))}{\abs{g'(g^{-1}(y))}} & \text{für } y \in g(D) \end{cases} \]
\end{satz}

\begin{bsp}
  $Y = e^{N (\mu, \sigma^2)} + c$

  Dann heißt $Y$ logarithmisch normalverteilt

  $g(x) = e^x + c$

  Ausgelassen: Rechnungen
\end{bsp}

\end{document}
